\documentclass[11pt,a4paper]{article}
\usepackage[T1]{fontenc}
\usepackage{isabelle,isabellesym}
\usepackage{amsmath}

% this should be the last package used
\usepackage{pdfsetup}

% urls in roman style, theory text in math-similar italics
\urlstyle{rm}
\isabellestyle{it}


\begin{document}

\title{Low Degree Hypergraphs}
\author{Ata Keskin}
\maketitle

\begin{abstract}
The goal of this entry is to prove Theorem 4.6 in \cite{micciancio_complexity_2002}, which serves as a fundamental component in establishing the NP-hardness of approximating the shortest vector problem under RUR-reductions. However, the theorem's original formulation relies on integer lattices and matrices, which overlooks its combinatorial essence. To address this, the theorem is reformulated in terms of hypergraphs. The formalization follows the proof given by Goldwasser and Micciancio.
\end{abstract}

\tableofcontents

\section{Introduction}

A hypergraph is defined as a pair $(V, Z)$ consisting of a finite set of vertices, V, and a collection of subsets of V known as hyperedges. A hypergraph is called regular if all hyperedges have the same size, referred to as the degree of the hypergraph. An $h$-regular hypergraph $(V, Z)$ is considered, and a vector $T = (T_1, ..., T_n)$ of subsets of $V$ is chosen randomly. For any subset of vertices $U \subseteq V$, $T(U)$ represents a transformed vector obtained by taking the intersection sizes of the sets $T_i$ with $U$. $T(Z)$ denotes the collection of transformed sequences for all subsets $U \in Z$. This reformulation allows for a direct correspondence between the hypergraph representation and the matrix representation, where the hyperedges correspond to characteristic vectors in $\{0, 1\}^{\lvert V\rvert}$ and the vector $T = (T_1, ..., T_n)$ corresponds to a matrix $T \in \{0, 1\}^{n \times \lvert V\rvert}$ with rows being the characteristic vectors of the sets $T_i$. Consequently, $T(U) = Tu$, where $u$ represents the characteristic vector of set $U$. Notably, the scalar product of two vectors $x, y \in \{0, 1\}^{\lvert V\rvert}$ corresponds to the size of the intersection of the corresponding sets.

The proof of Theorem 4.6 is divided into two stages. Initially, a weaker result is established, demonstrating that every vector $\{x \in \{0, 1\}^n$ belongs to $T(Z)$ with a high probability. Subsequently, a stronger property, as stated in Theorem 4.6, is proven. The distinction between the weak and strong versions lies in the order of quantification. While the strong version guarantees that $T$ is effective for all target vectors x with high probability, the weak version asserts that for any fixed target vector $x$, there exists an effective matrix $T$ with high probability.

The weak version of the theorem is demonstrated in Theorem 6.8, employing a straightforward argument based on Chebyshev's inequality. In the last section, it is shown that the strong version of the theorem can be derived easily from the weak version using ideas similar to those employed in the proof of Sauer's Lemma, which is a combinatorial result established by Sauer, Perles, and Shelah. Sauer's Lemma, and its slightly weaker form established by Vapnik and Chervonenkis, provides insight into the existence of a solution comprising singleton sets $\lvert T_i \rvert = 1$ for any hypergraph $(V, Z)$ with $\lvert Z \rvert \ge \lvert V \rvert \cdot n$. However, the proof of Sauer's Lemma is non-constructive, only asserting the existence of $T$ without offering an effective (even probabilistic) method to find it. Theorem 4.6, therefore, can be regarded as a probabilistic variant of Sauer's Lemma.

% include generated text of all theories
%
\begin{isabellebody}%
\setisabellecontext{Measure{\isacharunderscore}{\kern0pt}Space{\isacharunderscore}{\kern0pt}Addendum}%
%
\isadelimtheory
%
\endisadelimtheory
%
\isatagtheory
\isacommand{theory}\isamarkupfalse%
\ Measure{\isacharunderscore}{\kern0pt}Space{\isacharunderscore}{\kern0pt}Addendum\isanewline
\ \ \isakeyword{imports}\ {\isachardoublequoteopen}HOL{\isacharminus}{\kern0pt}Analysis{\isachardot}{\kern0pt}Measure{\isacharunderscore}{\kern0pt}Space{\isachardoublequoteclose}\isanewline
\isakeyword{begin}%
\endisatagtheory
{\isafoldtheory}%
%
\isadelimtheory
%
\endisadelimtheory
%
\isadelimdocument
%
\endisadelimdocument
%
\isatagdocument
%
\isamarkupsubsection{Sigma Algebra Generated by a Family of Functions%
}
\isamarkuptrue%
%
\endisatagdocument
{\isafolddocument}%
%
\isadelimdocument
%
\endisadelimdocument
\isacommand{definition}\isamarkupfalse%
\ sigma{\isacharunderscore}{\kern0pt}gen\ {\isacharcolon}{\kern0pt}{\isacharcolon}{\kern0pt}\ {\isachardoublequoteopen}{\isacharprime}{\kern0pt}a\ set\ {\isasymRightarrow}\ {\isacharprime}{\kern0pt}b\ measure\ {\isasymRightarrow}\ {\isacharparenleft}{\kern0pt}{\isacharprime}{\kern0pt}a\ {\isasymRightarrow}\ {\isacharprime}{\kern0pt}b{\isacharparenright}{\kern0pt}\ set\ {\isasymRightarrow}\ {\isacharprime}{\kern0pt}a\ measure{\isachardoublequoteclose}\ \isakeyword{where}\isanewline
\ \ {\isachardoublequoteopen}sigma{\isacharunderscore}{\kern0pt}gen\ {\isasymOmega}\ N\ S\ {\isasymequiv}\ sigma\ {\isasymOmega}\ {\isacharparenleft}{\kern0pt}{\isasymUnion}f\ {\isasymin}\ S{\isachardot}{\kern0pt}\ {\isacharbraceleft}{\kern0pt}f\ {\isacharminus}{\kern0pt}{\isacharbackquote}{\kern0pt}\ A\ {\isasyminter}\ {\isasymOmega}\ {\isacharbar}{\kern0pt}\ A{\isachardot}{\kern0pt}\ A\ {\isasymin}\ N{\isacharbraceright}{\kern0pt}{\isacharparenright}{\kern0pt}{\isachardoublequoteclose}\isanewline
\isanewline
\isacommand{lemma}\isamarkupfalse%
\isanewline
\ \ \isakeyword{shows}\ sets{\isacharunderscore}{\kern0pt}sigma{\isacharunderscore}{\kern0pt}gen{\isacharcolon}{\kern0pt}\ {\isachardoublequoteopen}sets\ {\isacharparenleft}{\kern0pt}sigma{\isacharunderscore}{\kern0pt}gen\ {\isasymOmega}\ N\ S{\isacharparenright}{\kern0pt}\ {\isacharequal}{\kern0pt}\ sigma{\isacharunderscore}{\kern0pt}sets\ {\isasymOmega}\ {\isacharparenleft}{\kern0pt}{\isasymUnion}f\ {\isasymin}\ S{\isachardot}{\kern0pt}\ {\isacharbraceleft}{\kern0pt}f\ {\isacharminus}{\kern0pt}{\isacharbackquote}{\kern0pt}\ A\ {\isasyminter}\ {\isasymOmega}\ {\isacharbar}{\kern0pt}\ A{\isachardot}{\kern0pt}\ A\ {\isasymin}\ N{\isacharbraceright}{\kern0pt}{\isacharparenright}{\kern0pt}{\isachardoublequoteclose}\ \isanewline
\ \ \ \ \isakeyword{and}\ space{\isacharunderscore}{\kern0pt}sigma{\isacharunderscore}{\kern0pt}gen{\isacharbrackleft}{\kern0pt}simp{\isacharbrackright}{\kern0pt}{\isacharcolon}{\kern0pt}\ {\isachardoublequoteopen}space\ {\isacharparenleft}{\kern0pt}sigma{\isacharunderscore}{\kern0pt}gen\ {\isasymOmega}\ N\ S{\isacharparenright}{\kern0pt}\ {\isacharequal}{\kern0pt}\ {\isasymOmega}{\isachardoublequoteclose}\isanewline
%
\isadelimproof
\ \ %
\endisadelimproof
%
\isatagproof
\isacommand{by}\isamarkupfalse%
\ {\isacharparenleft}{\kern0pt}auto\ simp\ add{\isacharcolon}{\kern0pt}\ sigma{\isacharunderscore}{\kern0pt}gen{\isacharunderscore}{\kern0pt}def\ sets{\isacharunderscore}{\kern0pt}measure{\isacharunderscore}{\kern0pt}of{\isacharunderscore}{\kern0pt}conv\ space{\isacharunderscore}{\kern0pt}measure{\isacharunderscore}{\kern0pt}of{\isacharunderscore}{\kern0pt}conv{\isacharparenright}{\kern0pt}%
\endisatagproof
{\isafoldproof}%
%
\isadelimproof
\isanewline
%
\endisadelimproof
\isanewline
\isacommand{lemma}\isamarkupfalse%
\ measurable{\isacharunderscore}{\kern0pt}sigma{\isacharunderscore}{\kern0pt}gen{\isacharcolon}{\kern0pt}\isanewline
\ \ \isakeyword{assumes}\ {\isachardoublequoteopen}f\ {\isasymin}\ S{\isachardoublequoteclose}\ {\isachardoublequoteopen}f\ {\isasymin}\ {\isasymOmega}\ {\isasymrightarrow}\ space\ N{\isachardoublequoteclose}\isanewline
\ \ \isakeyword{shows}\ {\isachardoublequoteopen}f\ {\isasymin}\ sigma{\isacharunderscore}{\kern0pt}gen\ {\isasymOmega}\ N\ S\ {\isasymrightarrow}\isactrlsub M\ N{\isachardoublequoteclose}\isanewline
%
\isadelimproof
\ \ %
\endisadelimproof
%
\isatagproof
\isacommand{using}\isamarkupfalse%
\ assms\ \isacommand{by}\isamarkupfalse%
\ {\isacharparenleft}{\kern0pt}intro\ measurableI{\isacharcomma}{\kern0pt}\ auto\ simp\ add{\isacharcolon}{\kern0pt}\ sets{\isacharunderscore}{\kern0pt}sigma{\isacharunderscore}{\kern0pt}gen{\isacharparenright}{\kern0pt}%
\endisatagproof
{\isafoldproof}%
%
\isadelimproof
\isanewline
%
\endisadelimproof
\isanewline
\isacommand{lemma}\isamarkupfalse%
\ measurable{\isacharunderscore}{\kern0pt}sigma{\isacharunderscore}{\kern0pt}gen{\isacharunderscore}{\kern0pt}singleton{\isacharcolon}{\kern0pt}\isanewline
\ \ \isakeyword{assumes}\ {\isachardoublequoteopen}f\ {\isasymin}\ {\isasymOmega}\ {\isasymrightarrow}\ space\ N{\isachardoublequoteclose}\isanewline
\ \ \isakeyword{shows}\ {\isachardoublequoteopen}f\ {\isasymin}\ sigma{\isacharunderscore}{\kern0pt}gen\ {\isasymOmega}\ N\ {\isacharbraceleft}{\kern0pt}f{\isacharbraceright}{\kern0pt}{\isasymrightarrow}\isactrlsub M\ N{\isachardoublequoteclose}\isanewline
%
\isadelimproof
\ \ %
\endisadelimproof
%
\isatagproof
\isacommand{using}\isamarkupfalse%
\ assms\ measurable{\isacharunderscore}{\kern0pt}sigma{\isacharunderscore}{\kern0pt}gen\ \isacommand{by}\isamarkupfalse%
\ blast%
\endisatagproof
{\isafoldproof}%
%
\isadelimproof
\isanewline
%
\endisadelimproof
\isanewline
\isacommand{lemma}\isamarkupfalse%
\ measurable{\isacharunderscore}{\kern0pt}iff{\isacharunderscore}{\kern0pt}contains{\isacharunderscore}{\kern0pt}sigma{\isacharunderscore}{\kern0pt}gen{\isacharcolon}{\kern0pt}\isanewline
\ \ \isakeyword{shows}\ {\isachardoublequoteopen}{\isacharparenleft}{\kern0pt}f\ {\isasymin}\ M\ {\isasymrightarrow}\isactrlsub M\ N{\isacharparenright}{\kern0pt}\ {\isasymlongleftrightarrow}\ f\ {\isasymin}\ space\ M\ {\isasymrightarrow}\ space\ N\ {\isasymand}\ sigma{\isacharunderscore}{\kern0pt}gen\ {\isacharparenleft}{\kern0pt}space\ M{\isacharparenright}{\kern0pt}\ N\ {\isacharbraceleft}{\kern0pt}f{\isacharbraceright}{\kern0pt}\ {\isasymsubseteq}\ M{\isachardoublequoteclose}\isanewline
%
\isadelimproof
%
\endisadelimproof
%
\isatagproof
\isacommand{proof}\isamarkupfalse%
\ {\isacharparenleft}{\kern0pt}standard{\isacharcomma}{\kern0pt}\ goal{\isacharunderscore}{\kern0pt}cases{\isacharparenright}{\kern0pt}\isanewline
\ \ \isacommand{case}\isamarkupfalse%
\ {\isadigit{1}}\isanewline
\ \ \isacommand{hence}\isamarkupfalse%
\ {\isachardoublequoteopen}f\ {\isasymin}\ space\ M\ {\isasymrightarrow}\ space\ N{\isachardoublequoteclose}\ \isacommand{using}\isamarkupfalse%
\ measurable{\isacharunderscore}{\kern0pt}space\ \isacommand{by}\isamarkupfalse%
\ fast\isanewline
\ \ \isacommand{thus}\isamarkupfalse%
\ {\isacharquery}{\kern0pt}case\ \isacommand{unfolding}\isamarkupfalse%
\ sets{\isacharunderscore}{\kern0pt}sigma{\isacharunderscore}{\kern0pt}gen\ \isacommand{by}\isamarkupfalse%
\ {\isacharparenleft}{\kern0pt}simp{\isacharcomma}{\kern0pt}\ intro\ sigma{\isacharunderscore}{\kern0pt}algebra{\isachardot}{\kern0pt}sigma{\isacharunderscore}{\kern0pt}sets{\isacharunderscore}{\kern0pt}subset{\isacharcomma}{\kern0pt}\ {\isacharparenleft}{\kern0pt}blast\ intro{\isacharcolon}{\kern0pt}\ sets{\isachardot}{\kern0pt}sigma{\isacharunderscore}{\kern0pt}algebra{\isacharunderscore}{\kern0pt}axioms\ measurable{\isacharunderscore}{\kern0pt}sets{\isacharbrackleft}{\kern0pt}OF\ {\isadigit{1}}{\isacharbrackright}{\kern0pt}{\isacharparenright}{\kern0pt}{\isacharplus}{\kern0pt}{\isacharparenright}{\kern0pt}\ \isanewline
\isacommand{next}\isamarkupfalse%
\isanewline
\ \ \isacommand{case}\isamarkupfalse%
\ {\isadigit{2}}\isanewline
\ \ \isacommand{thus}\isamarkupfalse%
\ {\isacharquery}{\kern0pt}case\ \isacommand{using}\isamarkupfalse%
\ measurable{\isacharunderscore}{\kern0pt}mono{\isacharbrackleft}{\kern0pt}OF\ {\isacharunderscore}{\kern0pt}\ refl\ {\isacharunderscore}{\kern0pt}\ space{\isacharunderscore}{\kern0pt}sigma{\isacharunderscore}{\kern0pt}gen{\isacharcomma}{\kern0pt}\ of\ N\ M{\isacharbrackright}{\kern0pt}\ measurable{\isacharunderscore}{\kern0pt}sigma{\isacharunderscore}{\kern0pt}gen{\isacharunderscore}{\kern0pt}singleton\ \isacommand{by}\isamarkupfalse%
\ fast\isanewline
\isacommand{qed}\isamarkupfalse%
%
\endisatagproof
{\isafoldproof}%
%
\isadelimproof
\isanewline
%
\endisadelimproof
\isanewline
\isacommand{lemma}\isamarkupfalse%
\ measurable{\isacharunderscore}{\kern0pt}family{\isacharunderscore}{\kern0pt}iff{\isacharunderscore}{\kern0pt}contains{\isacharunderscore}{\kern0pt}sigma{\isacharunderscore}{\kern0pt}gen{\isacharcolon}{\kern0pt}\isanewline
\ \ \isakeyword{shows}\ {\isachardoublequoteopen}{\isacharparenleft}{\kern0pt}S\ {\isasymsubseteq}\ M\ {\isasymrightarrow}\isactrlsub M\ N{\isacharparenright}{\kern0pt}\ {\isasymlongleftrightarrow}\ S\ {\isasymsubseteq}\ space\ M\ {\isasymrightarrow}\ space\ N\ {\isasymand}\ sigma{\isacharunderscore}{\kern0pt}gen\ {\isacharparenleft}{\kern0pt}space\ M{\isacharparenright}{\kern0pt}\ N\ S\ {\isasymsubseteq}\ M{\isachardoublequoteclose}\isanewline
%
\isadelimproof
%
\endisadelimproof
%
\isatagproof
\isacommand{proof}\isamarkupfalse%
\ {\isacharparenleft}{\kern0pt}standard{\isacharcomma}{\kern0pt}\ goal{\isacharunderscore}{\kern0pt}cases{\isacharparenright}{\kern0pt}\isanewline
\ \ \isacommand{case}\isamarkupfalse%
\ {\isadigit{1}}\isanewline
\ \ \isacommand{hence}\isamarkupfalse%
\ subset{\isacharcolon}{\kern0pt}\ {\isachardoublequoteopen}S\ {\isasymsubseteq}\ space\ M\ {\isasymrightarrow}\ space\ N{\isachardoublequoteclose}\ \isacommand{using}\isamarkupfalse%
\ measurable{\isacharunderscore}{\kern0pt}space\ \isacommand{by}\isamarkupfalse%
\ fast\isanewline
\ \ \isacommand{have}\isamarkupfalse%
\ {\isachardoublequoteopen}{\isacharbraceleft}{\kern0pt}f\ {\isacharminus}{\kern0pt}{\isacharbackquote}{\kern0pt}\ A\ {\isasyminter}\ space\ M\ {\isacharbar}{\kern0pt}A{\isachardot}{\kern0pt}\ A\ {\isasymin}\ N{\isacharbraceright}{\kern0pt}\ {\isasymsubseteq}\ M{\isachardoublequoteclose}\ \isakeyword{if}\ {\isachardoublequoteopen}f\ {\isasymin}\ S{\isachardoublequoteclose}\ \isakeyword{for}\ f\ \isacommand{using}\isamarkupfalse%
\ measurable{\isacharunderscore}{\kern0pt}iff{\isacharunderscore}{\kern0pt}contains{\isacharunderscore}{\kern0pt}sigma{\isacharunderscore}{\kern0pt}gen{\isacharbrackleft}{\kern0pt}unfolded\ sets{\isacharunderscore}{\kern0pt}sigma{\isacharunderscore}{\kern0pt}gen{\isacharcomma}{\kern0pt}\ of\ f{\isacharbrackright}{\kern0pt}\ {\isadigit{1}}\ subset\ that\ \isacommand{by}\isamarkupfalse%
\ blast\isanewline
\ \ \isacommand{then}\isamarkupfalse%
\ \isacommand{show}\isamarkupfalse%
\ {\isacharquery}{\kern0pt}case\ \isacommand{unfolding}\isamarkupfalse%
\ sets{\isacharunderscore}{\kern0pt}sigma{\isacharunderscore}{\kern0pt}gen\ \isacommand{using}\isamarkupfalse%
\ sets{\isachardot}{\kern0pt}sigma{\isacharunderscore}{\kern0pt}algebra{\isacharunderscore}{\kern0pt}axioms\ \isacommand{by}\isamarkupfalse%
\ {\isacharparenleft}{\kern0pt}simp\ add{\isacharcolon}{\kern0pt}\ subset{\isacharcomma}{\kern0pt}\ intro\ sigma{\isacharunderscore}{\kern0pt}algebra{\isachardot}{\kern0pt}sigma{\isacharunderscore}{\kern0pt}sets{\isacharunderscore}{\kern0pt}subset{\isacharcomma}{\kern0pt}\ blast{\isacharplus}{\kern0pt}{\isacharparenright}{\kern0pt}\isanewline
\isacommand{next}\isamarkupfalse%
\isanewline
\ \ \isacommand{case}\isamarkupfalse%
\ {\isadigit{2}}\isanewline
\ \ \isacommand{hence}\isamarkupfalse%
\ subset{\isacharcolon}{\kern0pt}\ {\isachardoublequoteopen}S\ {\isasymsubseteq}\ space\ M\ {\isasymrightarrow}\ space\ N{\isachardoublequoteclose}\ \isacommand{by}\isamarkupfalse%
\ simp\isanewline
\ \ \isacommand{show}\isamarkupfalse%
\ {\isacharquery}{\kern0pt}case\isanewline
\ \ \isacommand{proof}\isamarkupfalse%
\ {\isacharparenleft}{\kern0pt}standard{\isacharcomma}{\kern0pt}\ goal{\isacharunderscore}{\kern0pt}cases{\isacharparenright}{\kern0pt}\isanewline
\ \ \ \ \isacommand{case}\isamarkupfalse%
\ {\isacharparenleft}{\kern0pt}{\isadigit{1}}\ x{\isacharparenright}{\kern0pt}\isanewline
\ \ \ \ \isacommand{have}\isamarkupfalse%
\ {\isachardoublequoteopen}sigma{\isacharunderscore}{\kern0pt}gen\ {\isacharparenleft}{\kern0pt}space\ M{\isacharparenright}{\kern0pt}\ N\ {\isacharbraceleft}{\kern0pt}x{\isacharbraceright}{\kern0pt}\ {\isasymsubseteq}\ M{\isachardoublequoteclose}\ \isacommand{by}\isamarkupfalse%
\ {\isacharparenleft}{\kern0pt}metis\ {\isacharparenleft}{\kern0pt}no{\isacharunderscore}{\kern0pt}types{\isacharcomma}{\kern0pt}\ lifting{\isacharparenright}{\kern0pt}\ {\isadigit{1}}\ {\isadigit{2}}\ sets{\isacharunderscore}{\kern0pt}sigma{\isacharunderscore}{\kern0pt}gen\ SUP{\isacharunderscore}{\kern0pt}le{\isacharunderscore}{\kern0pt}iff\ sigma{\isacharunderscore}{\kern0pt}sets{\isacharunderscore}{\kern0pt}le{\isacharunderscore}{\kern0pt}sets{\isacharunderscore}{\kern0pt}iff\ singletonD{\isacharparenright}{\kern0pt}\isanewline
\ \ \ \ \isacommand{thus}\isamarkupfalse%
\ {\isacharquery}{\kern0pt}case\ \isacommand{using}\isamarkupfalse%
\ measurable{\isacharunderscore}{\kern0pt}iff{\isacharunderscore}{\kern0pt}contains{\isacharunderscore}{\kern0pt}sigma{\isacharunderscore}{\kern0pt}gen\ subset{\isacharbrackleft}{\kern0pt}THEN\ subsetD{\isacharcomma}{\kern0pt}\ OF\ {\isadigit{1}}{\isacharbrackright}{\kern0pt}\ \isacommand{by}\isamarkupfalse%
\ fast\ \isanewline
\ \ \isacommand{qed}\isamarkupfalse%
\isanewline
\isacommand{qed}\isamarkupfalse%
%
\endisatagproof
{\isafoldproof}%
%
\isadelimproof
\isanewline
%
\endisadelimproof
%
\isadelimtheory
\isanewline
%
\endisadelimtheory
%
\isatagtheory
\isacommand{end}\isamarkupfalse%
%
\endisatagtheory
{\isafoldtheory}%
%
\isadelimtheory
%
\endisadelimtheory
%
\end{isabellebody}%
\endinput
%:%file=Measure_Space_Addendum.tex%:%
%:%10=1%:%
%:%11=1%:%
%:%12=2%:%
%:%13=3%:%
%:%27=5%:%
%:%37=7%:%
%:%38=7%:%
%:%39=8%:%
%:%40=9%:%
%:%41=10%:%
%:%42=10%:%
%:%43=11%:%
%:%44=12%:%
%:%47=13%:%
%:%51=13%:%
%:%52=13%:%
%:%57=13%:%
%:%60=14%:%
%:%61=15%:%
%:%62=15%:%
%:%63=16%:%
%:%64=17%:%
%:%67=18%:%
%:%71=18%:%
%:%72=18%:%
%:%73=18%:%
%:%78=18%:%
%:%81=19%:%
%:%82=20%:%
%:%83=20%:%
%:%84=21%:%
%:%85=22%:%
%:%88=23%:%
%:%92=23%:%
%:%93=23%:%
%:%94=23%:%
%:%99=23%:%
%:%102=24%:%
%:%103=25%:%
%:%104=25%:%
%:%105=26%:%
%:%112=27%:%
%:%113=27%:%
%:%114=28%:%
%:%115=28%:%
%:%116=29%:%
%:%117=29%:%
%:%118=29%:%
%:%119=29%:%
%:%120=30%:%
%:%121=30%:%
%:%122=30%:%
%:%123=30%:%
%:%124=31%:%
%:%125=31%:%
%:%126=32%:%
%:%127=32%:%
%:%128=33%:%
%:%129=33%:%
%:%130=33%:%
%:%131=33%:%
%:%132=34%:%
%:%138=34%:%
%:%141=35%:%
%:%142=36%:%
%:%143=36%:%
%:%144=37%:%
%:%151=38%:%
%:%152=38%:%
%:%153=39%:%
%:%154=39%:%
%:%155=40%:%
%:%156=40%:%
%:%157=40%:%
%:%158=40%:%
%:%159=41%:%
%:%160=41%:%
%:%161=41%:%
%:%162=41%:%
%:%163=42%:%
%:%164=42%:%
%:%165=42%:%
%:%166=42%:%
%:%167=42%:%
%:%168=42%:%
%:%169=43%:%
%:%170=43%:%
%:%171=44%:%
%:%172=44%:%
%:%173=45%:%
%:%174=45%:%
%:%175=45%:%
%:%176=46%:%
%:%177=46%:%
%:%178=47%:%
%:%179=47%:%
%:%180=48%:%
%:%181=48%:%
%:%182=49%:%
%:%183=49%:%
%:%184=49%:%
%:%185=50%:%
%:%186=50%:%
%:%187=50%:%
%:%188=50%:%
%:%189=51%:%
%:%190=51%:%
%:%191=52%:%
%:%197=52%:%
%:%202=53%:%
%:%207=54%:%

%
\begin{isabellebody}%
\setisabellecontext{Elementary{\isacharunderscore}{\kern0pt}Metric{\isacharunderscore}{\kern0pt}Spaces{\isacharunderscore}{\kern0pt}Addendum}%
%
\isadelimtheory
%
\endisadelimtheory
%
\isatagtheory
\isacommand{theory}\isamarkupfalse%
\ Elementary{\isacharunderscore}{\kern0pt}Metric{\isacharunderscore}{\kern0pt}Spaces{\isacharunderscore}{\kern0pt}Addendum\isanewline
\ \ \isakeyword{imports}\ {\isachardoublequoteopen}HOL{\isacharminus}{\kern0pt}Analysis{\isachardot}{\kern0pt}Elementary{\isacharunderscore}{\kern0pt}Metric{\isacharunderscore}{\kern0pt}Spaces{\isachardoublequoteclose}\ {\isachardoublequoteopen}HOL{\isacharminus}{\kern0pt}Analysis{\isachardot}{\kern0pt}Bochner{\isacharunderscore}{\kern0pt}Integration{\isachardoublequoteclose}\isanewline
\isakeyword{begin}%
\endisatagtheory
{\isafoldtheory}%
%
\isadelimtheory
\isanewline
%
\endisadelimtheory
\isanewline
\isacommand{lemma}\isamarkupfalse%
\ diameter{\isacharunderscore}{\kern0pt}comp{\isacharunderscore}{\kern0pt}strict{\isacharunderscore}{\kern0pt}mono{\isacharcolon}{\kern0pt}\isanewline
\ \ \isakeyword{fixes}\ s\ {\isacharcolon}{\kern0pt}{\isacharcolon}{\kern0pt}\ {\isachardoublequoteopen}nat\ {\isasymRightarrow}\ {\isacharprime}{\kern0pt}a\ {\isacharcolon}{\kern0pt}{\isacharcolon}{\kern0pt}\ real{\isacharunderscore}{\kern0pt}normed{\isacharunderscore}{\kern0pt}vector{\isachardoublequoteclose}\isanewline
\ \ \isakeyword{assumes}\ {\isachardoublequoteopen}strict{\isacharunderscore}{\kern0pt}mono\ r{\isachardoublequoteclose}\ {\isachardoublequoteopen}bounded\ {\isacharbraceleft}{\kern0pt}s\ i\ {\isacharbar}{\kern0pt}i{\isachardot}{\kern0pt}\ r\ n\ {\isasymle}\ i{\isacharbraceright}{\kern0pt}{\isachardoublequoteclose}\isanewline
\ \ \isakeyword{shows}\ {\isachardoublequoteopen}diameter\ {\isacharbraceleft}{\kern0pt}s\ {\isacharparenleft}{\kern0pt}r\ i{\isacharparenright}{\kern0pt}\ {\isacharbar}{\kern0pt}\ i{\isachardot}{\kern0pt}\ n\ {\isasymle}\ i{\isacharbraceright}{\kern0pt}\ {\isasymle}\ diameter\ {\isacharbraceleft}{\kern0pt}s\ i\ {\isacharbar}{\kern0pt}\ i{\isachardot}{\kern0pt}\ r\ n\ {\isasymle}\ i{\isacharbraceright}{\kern0pt}{\isachardoublequoteclose}\isanewline
%
\isadelimproof
%
\endisadelimproof
%
\isatagproof
\isacommand{proof}\isamarkupfalse%
\ {\isacharparenleft}{\kern0pt}rule\ diameter{\isacharunderscore}{\kern0pt}subset{\isacharparenright}{\kern0pt}\isanewline
\ \ \isacommand{show}\isamarkupfalse%
\ {\isachardoublequoteopen}{\isacharbraceleft}{\kern0pt}s\ {\isacharparenleft}{\kern0pt}r\ i{\isacharparenright}{\kern0pt}\ {\isacharbar}{\kern0pt}\ i{\isachardot}{\kern0pt}\ n\ {\isasymle}\ i{\isacharbraceright}{\kern0pt}\ {\isasymsubseteq}\ {\isacharbraceleft}{\kern0pt}s\ i\ {\isacharbar}{\kern0pt}\ i{\isachardot}{\kern0pt}\ r\ n\ {\isasymle}\ i{\isacharbraceright}{\kern0pt}{\isachardoublequoteclose}\ \isacommand{using}\isamarkupfalse%
\ assms{\isacharparenleft}{\kern0pt}{\isadigit{1}}{\isacharparenright}{\kern0pt}\ monotoneD\ strict{\isacharunderscore}{\kern0pt}mono{\isacharunderscore}{\kern0pt}mono\ \isacommand{by}\isamarkupfalse%
\ fastforce\isanewline
\isacommand{qed}\isamarkupfalse%
\ {\isacharparenleft}{\kern0pt}intro\ assms{\isacharparenleft}{\kern0pt}{\isadigit{2}}{\isacharparenright}{\kern0pt}{\isacharparenright}{\kern0pt}%
\endisatagproof
{\isafoldproof}%
%
\isadelimproof
\isanewline
%
\endisadelimproof
\isanewline
\isacommand{lemma}\isamarkupfalse%
\ diameter{\isacharunderscore}{\kern0pt}bounded{\isacharunderscore}{\kern0pt}bound{\isacharprime}{\kern0pt}{\isacharcolon}{\kern0pt}\isanewline
\ \ \isakeyword{fixes}\ S\ {\isacharcolon}{\kern0pt}{\isacharcolon}{\kern0pt}\ {\isachardoublequoteopen}{\isacharprime}{\kern0pt}a\ {\isacharcolon}{\kern0pt}{\isacharcolon}{\kern0pt}\ metric{\isacharunderscore}{\kern0pt}space\ set{\isachardoublequoteclose}\isanewline
\ \ \isakeyword{assumes}\ S{\isacharcolon}{\kern0pt}\ {\isachardoublequoteopen}bdd{\isacharunderscore}{\kern0pt}above\ {\isacharparenleft}{\kern0pt}case{\isacharunderscore}{\kern0pt}prod\ dist\ {\isacharbackquote}{\kern0pt}\ {\isacharparenleft}{\kern0pt}S{\isasymtimes}S{\isacharparenright}{\kern0pt}{\isacharparenright}{\kern0pt}{\isachardoublequoteclose}\ {\isachardoublequoteopen}x\ {\isasymin}\ S{\isachardoublequoteclose}\ {\isachardoublequoteopen}y\ {\isasymin}\ S{\isachardoublequoteclose}\isanewline
\ \ \isakeyword{shows}\ {\isachardoublequoteopen}dist\ x\ y\ {\isasymle}\ diameter\ S{\isachardoublequoteclose}\isanewline
%
\isadelimproof
%
\endisadelimproof
%
\isatagproof
\isacommand{proof}\isamarkupfalse%
\ {\isacharminus}{\kern0pt}\isanewline
\ \ \isacommand{have}\isamarkupfalse%
\ {\isachardoublequoteopen}{\isacharparenleft}{\kern0pt}x{\isacharcomma}{\kern0pt}y{\isacharparenright}{\kern0pt}\ {\isasymin}\ S{\isasymtimes}S{\isachardoublequoteclose}\ \isacommand{using}\isamarkupfalse%
\ S\ \isacommand{by}\isamarkupfalse%
\ auto\isanewline
\ \ \isacommand{then}\isamarkupfalse%
\ \isacommand{have}\isamarkupfalse%
\ {\isachardoublequoteopen}dist\ x\ y\ {\isasymle}\ {\isacharparenleft}{\kern0pt}SUP\ {\isacharparenleft}{\kern0pt}x{\isacharcomma}{\kern0pt}y{\isacharparenright}{\kern0pt}{\isasymin}S{\isasymtimes}S{\isachardot}{\kern0pt}\ dist\ x\ y{\isacharparenright}{\kern0pt}{\isachardoublequoteclose}\ \isacommand{by}\isamarkupfalse%
\ {\isacharparenleft}{\kern0pt}rule\ cSUP{\isacharunderscore}{\kern0pt}upper{\isadigit{2}}{\isacharbrackleft}{\kern0pt}OF\ assms{\isacharparenleft}{\kern0pt}{\isadigit{1}}{\isacharparenright}{\kern0pt}{\isacharbrackright}{\kern0pt}{\isacharparenright}{\kern0pt}\ simp\isanewline
\ \ \isacommand{with}\isamarkupfalse%
\ {\isacartoucheopen}x\ {\isasymin}\ S{\isacartoucheclose}\ \isacommand{show}\isamarkupfalse%
\ {\isacharquery}{\kern0pt}thesis\ \isacommand{by}\isamarkupfalse%
\ {\isacharparenleft}{\kern0pt}auto\ simp{\isacharcolon}{\kern0pt}\ diameter{\isacharunderscore}{\kern0pt}def{\isacharparenright}{\kern0pt}\isanewline
\isacommand{qed}\isamarkupfalse%
%
\endisatagproof
{\isafoldproof}%
%
\isadelimproof
\isanewline
%
\endisadelimproof
\isanewline
\isacommand{lemma}\isamarkupfalse%
\ bounded{\isacharunderscore}{\kern0pt}imp{\isacharunderscore}{\kern0pt}dist{\isacharunderscore}{\kern0pt}bounded{\isacharcolon}{\kern0pt}\isanewline
\ \ \isakeyword{assumes}\ {\isachardoublequoteopen}bounded\ {\isacharparenleft}{\kern0pt}range\ s{\isacharparenright}{\kern0pt}{\isachardoublequoteclose}\isanewline
\ \ \isakeyword{shows}\ {\isachardoublequoteopen}bounded\ {\isacharparenleft}{\kern0pt}{\isacharparenleft}{\kern0pt}{\isasymlambda}{\isacharparenleft}{\kern0pt}i{\isacharcomma}{\kern0pt}\ j{\isacharparenright}{\kern0pt}{\isachardot}{\kern0pt}\ dist\ {\isacharparenleft}{\kern0pt}s\ i{\isacharparenright}{\kern0pt}\ {\isacharparenleft}{\kern0pt}s\ j{\isacharparenright}{\kern0pt}{\isacharparenright}{\kern0pt}\ {\isacharbackquote}{\kern0pt}\ {\isacharparenleft}{\kern0pt}{\isacharbraceleft}{\kern0pt}n{\isachardot}{\kern0pt}{\isachardot}{\kern0pt}{\isacharbraceright}{\kern0pt}\ {\isasymtimes}\ {\isacharbraceleft}{\kern0pt}n{\isachardot}{\kern0pt}{\isachardot}{\kern0pt}{\isacharbraceright}{\kern0pt}{\isacharparenright}{\kern0pt}{\isacharparenright}{\kern0pt}{\isachardoublequoteclose}\isanewline
%
\isadelimproof
\ \ %
\endisadelimproof
%
\isatagproof
\isacommand{using}\isamarkupfalse%
\ bounded{\isacharunderscore}{\kern0pt}dist{\isacharunderscore}{\kern0pt}comp{\isacharbrackleft}{\kern0pt}OF\ bounded{\isacharunderscore}{\kern0pt}fst\ bounded{\isacharunderscore}{\kern0pt}snd{\isacharcomma}{\kern0pt}\ OF\ bounded{\isacharunderscore}{\kern0pt}Times{\isacharparenleft}{\kern0pt}{\isadigit{1}}{\isacharcomma}{\kern0pt}{\isadigit{1}}{\isacharparenright}{\kern0pt}{\isacharbrackleft}{\kern0pt}OF\ assms{\isacharparenleft}{\kern0pt}{\isadigit{1}}{\isacharcomma}{\kern0pt}{\isadigit{1}}{\isacharparenright}{\kern0pt}{\isacharbrackright}{\kern0pt}{\isacharbrackright}{\kern0pt}\ \isacommand{by}\isamarkupfalse%
\ {\isacharparenleft}{\kern0pt}rule\ bounded{\isacharunderscore}{\kern0pt}subset{\isacharcomma}{\kern0pt}\ force{\isacharparenright}{\kern0pt}%
\endisatagproof
{\isafoldproof}%
%
\isadelimproof
\ \isanewline
%
\endisadelimproof
\isanewline
\isacommand{lemma}\isamarkupfalse%
\ cauchy{\isacharunderscore}{\kern0pt}iff{\isacharunderscore}{\kern0pt}diameter{\isacharunderscore}{\kern0pt}tends{\isacharunderscore}{\kern0pt}to{\isacharunderscore}{\kern0pt}zero{\isacharunderscore}{\kern0pt}and{\isacharunderscore}{\kern0pt}bounded{\isacharcolon}{\kern0pt}\isanewline
\ \ \isakeyword{fixes}\ s\ {\isacharcolon}{\kern0pt}{\isacharcolon}{\kern0pt}\ {\isachardoublequoteopen}nat\ {\isasymRightarrow}\ {\isacharprime}{\kern0pt}a\ {\isacharcolon}{\kern0pt}{\isacharcolon}{\kern0pt}\ real{\isacharunderscore}{\kern0pt}normed{\isacharunderscore}{\kern0pt}vector{\isachardoublequoteclose}\isanewline
\ \ \isakeyword{shows}\ {\isachardoublequoteopen}Cauchy\ s\ {\isasymlongleftrightarrow}\ {\isacharparenleft}{\kern0pt}{\isacharparenleft}{\kern0pt}{\isasymlambda}n{\isachardot}{\kern0pt}\ diameter\ {\isacharbraceleft}{\kern0pt}s\ i\ {\isacharbar}{\kern0pt}\ i{\isachardot}{\kern0pt}\ i\ {\isasymge}\ n{\isacharbraceright}{\kern0pt}{\isacharparenright}{\kern0pt}\ {\isasymlonglonglongrightarrow}\ {\isadigit{0}}\ {\isasymand}\ bounded\ {\isacharparenleft}{\kern0pt}range\ s{\isacharparenright}{\kern0pt}{\isacharparenright}{\kern0pt}{\isachardoublequoteclose}\isanewline
%
\isadelimproof
%
\endisadelimproof
%
\isatagproof
\isacommand{proof}\isamarkupfalse%
\ {\isacharminus}{\kern0pt}\isanewline
\ \ \isacommand{have}\isamarkupfalse%
\ {\isachardoublequoteopen}{\isacharbraceleft}{\kern0pt}s\ i\ {\isacharbar}{\kern0pt}i{\isachardot}{\kern0pt}\ N\ {\isasymle}\ i{\isacharbraceright}{\kern0pt}\ {\isasymnoteq}\ {\isacharbraceleft}{\kern0pt}{\isacharbraceright}{\kern0pt}{\isachardoublequoteclose}\ \isakeyword{for}\ N\ \isacommand{by}\isamarkupfalse%
\ blast\isanewline
\ \ \isacommand{hence}\isamarkupfalse%
\ diameter{\isacharunderscore}{\kern0pt}SUP{\isacharcolon}{\kern0pt}\ {\isachardoublequoteopen}diameter\ {\isacharbraceleft}{\kern0pt}s\ i\ {\isacharbar}{\kern0pt}i{\isachardot}{\kern0pt}\ N\ {\isasymle}\ i{\isacharbraceright}{\kern0pt}\ {\isacharequal}{\kern0pt}\ {\isacharparenleft}{\kern0pt}SUP\ {\isacharparenleft}{\kern0pt}i{\isacharcomma}{\kern0pt}\ j{\isacharparenright}{\kern0pt}\ {\isasymin}\ {\isacharbraceleft}{\kern0pt}N{\isachardot}{\kern0pt}{\isachardot}{\kern0pt}{\isacharbraceright}{\kern0pt}\ {\isasymtimes}\ {\isacharbraceleft}{\kern0pt}N{\isachardot}{\kern0pt}{\isachardot}{\kern0pt}{\isacharbraceright}{\kern0pt}{\isachardot}{\kern0pt}\ dist\ {\isacharparenleft}{\kern0pt}s\ i{\isacharparenright}{\kern0pt}\ {\isacharparenleft}{\kern0pt}s\ j{\isacharparenright}{\kern0pt}{\isacharparenright}{\kern0pt}{\isachardoublequoteclose}\ \isakeyword{for}\ N\ \isacommand{unfolding}\isamarkupfalse%
\ diameter{\isacharunderscore}{\kern0pt}def\ \isacommand{by}\isamarkupfalse%
\ {\isacharparenleft}{\kern0pt}auto\ intro{\isacharbang}{\kern0pt}{\isacharcolon}{\kern0pt}\ arg{\isacharunderscore}{\kern0pt}cong{\isacharbrackleft}{\kern0pt}of\ {\isacharunderscore}{\kern0pt}\ {\isacharunderscore}{\kern0pt}\ Sup{\isacharbrackright}{\kern0pt}{\isacharparenright}{\kern0pt}\isanewline
\ \ \isacommand{show}\isamarkupfalse%
\ {\isacharquery}{\kern0pt}thesis\ \isanewline
\ \ \isacommand{proof}\isamarkupfalse%
\ {\isacharparenleft}{\kern0pt}{\isacharparenleft}{\kern0pt}standard\ {\isacharsemicolon}{\kern0pt}\ clarsimp{\isacharparenright}{\kern0pt}{\isacharcomma}{\kern0pt}\ goal{\isacharunderscore}{\kern0pt}cases{\isacharparenright}{\kern0pt}\isanewline
\ \ \ \ \isacommand{case}\isamarkupfalse%
\ {\isadigit{1}}\isanewline
\ \ \ \ \isacommand{have}\isamarkupfalse%
\ {\isachardoublequoteopen}{\isasymexists}N{\isachardot}{\kern0pt}\ {\isasymforall}n{\isasymge}N{\isachardot}{\kern0pt}\ norm\ {\isacharparenleft}{\kern0pt}diameter\ {\isacharbraceleft}{\kern0pt}s\ i\ {\isacharbar}{\kern0pt}i{\isachardot}{\kern0pt}\ n\ {\isasymle}\ i{\isacharbraceright}{\kern0pt}{\isacharparenright}{\kern0pt}\ {\isacharless}{\kern0pt}\ e{\isachardoublequoteclose}\ \isakeyword{if}\ e{\isacharunderscore}{\kern0pt}pos{\isacharcolon}{\kern0pt}\ {\isachardoublequoteopen}e\ {\isachargreater}{\kern0pt}\ {\isadigit{0}}{\isachardoublequoteclose}\ \isakeyword{for}\ e\isanewline
\ \ \ \ \isacommand{proof}\isamarkupfalse%
\ {\isacharminus}{\kern0pt}\isanewline
\ \ \ \ \ \ \isacommand{obtain}\isamarkupfalse%
\ N\ \isakeyword{where}\ dist{\isacharunderscore}{\kern0pt}less{\isacharcolon}{\kern0pt}\ {\isachardoublequoteopen}dist\ {\isacharparenleft}{\kern0pt}s\ n{\isacharparenright}{\kern0pt}\ {\isacharparenleft}{\kern0pt}s\ m{\isacharparenright}{\kern0pt}\ {\isacharless}{\kern0pt}\ {\isacharparenleft}{\kern0pt}{\isadigit{1}}{\isacharslash}{\kern0pt}{\isadigit{2}}{\isacharparenright}{\kern0pt}\ {\isacharasterisk}{\kern0pt}\ e{\isachardoublequoteclose}\ \isakeyword{if}\ {\isachardoublequoteopen}n\ {\isasymge}\ N{\isachardoublequoteclose}\ {\isachardoublequoteopen}m\ {\isasymge}\ N{\isachardoublequoteclose}\ \isakeyword{for}\ n\ m\ \isacommand{using}\isamarkupfalse%
\ {\isadigit{1}}\ CauchyD\ e{\isacharunderscore}{\kern0pt}pos\ dist{\isacharunderscore}{\kern0pt}norm\ \isacommand{by}\isamarkupfalse%
\ {\isacharparenleft}{\kern0pt}metis\ mult{\isacharunderscore}{\kern0pt}pos{\isacharunderscore}{\kern0pt}pos\ zero{\isacharunderscore}{\kern0pt}less{\isacharunderscore}{\kern0pt}divide{\isacharunderscore}{\kern0pt}iff\ zero{\isacharunderscore}{\kern0pt}less{\isacharunderscore}{\kern0pt}numeral\ zero{\isacharunderscore}{\kern0pt}less{\isacharunderscore}{\kern0pt}one{\isacharparenright}{\kern0pt}\isanewline
\ \ \ \ \ \ \isacommand{{\isacharbraceleft}{\kern0pt}}\isamarkupfalse%
\isanewline
\ \ \ \ \ \ \ \ \isacommand{fix}\isamarkupfalse%
\ r\ \isacommand{assume}\isamarkupfalse%
\ {\isachardoublequoteopen}r\ {\isasymge}\ N{\isachardoublequoteclose}\isanewline
\ \ \ \ \ \ \ \ \isacommand{hence}\isamarkupfalse%
\ {\isachardoublequoteopen}dist\ {\isacharparenleft}{\kern0pt}s\ n{\isacharparenright}{\kern0pt}\ {\isacharparenleft}{\kern0pt}s\ m{\isacharparenright}{\kern0pt}\ {\isacharless}{\kern0pt}\ {\isacharparenleft}{\kern0pt}{\isadigit{1}}{\isacharslash}{\kern0pt}{\isadigit{2}}{\isacharparenright}{\kern0pt}\ {\isacharasterisk}{\kern0pt}\ e{\isachardoublequoteclose}\ \isakeyword{if}\ {\isachardoublequoteopen}n\ {\isasymge}\ r{\isachardoublequoteclose}\ {\isachardoublequoteopen}m\ {\isasymge}\ r{\isachardoublequoteclose}\ \isakeyword{for}\ n\ m\ \isacommand{using}\isamarkupfalse%
\ dist{\isacharunderscore}{\kern0pt}less\ that\ \isacommand{by}\isamarkupfalse%
\ simp\isanewline
\ \ \ \ \ \ \ \ \isacommand{hence}\isamarkupfalse%
\ {\isachardoublequoteopen}{\isacharparenleft}{\kern0pt}SUP\ {\isacharparenleft}{\kern0pt}i{\isacharcomma}{\kern0pt}\ j{\isacharparenright}{\kern0pt}\ {\isasymin}\ {\isacharbraceleft}{\kern0pt}r{\isachardot}{\kern0pt}{\isachardot}{\kern0pt}{\isacharbraceright}{\kern0pt}\ {\isasymtimes}\ {\isacharbraceleft}{\kern0pt}r{\isachardot}{\kern0pt}{\isachardot}{\kern0pt}{\isacharbraceright}{\kern0pt}{\isachardot}{\kern0pt}\ dist\ {\isacharparenleft}{\kern0pt}s\ i{\isacharparenright}{\kern0pt}\ {\isacharparenleft}{\kern0pt}s\ j{\isacharparenright}{\kern0pt}{\isacharparenright}{\kern0pt}\ {\isasymle}\ {\isacharparenleft}{\kern0pt}{\isadigit{1}}{\isacharslash}{\kern0pt}{\isadigit{2}}{\isacharparenright}{\kern0pt}\ {\isacharasterisk}{\kern0pt}\ e{\isachardoublequoteclose}\ \isacommand{by}\isamarkupfalse%
\ {\isacharparenleft}{\kern0pt}intro\ cSup{\isacharunderscore}{\kern0pt}least{\isacharparenright}{\kern0pt}\ fastforce{\isacharplus}{\kern0pt}\isanewline
\ \ \ \ \ \ \ \ \isacommand{also}\isamarkupfalse%
\ \isacommand{have}\isamarkupfalse%
\ {\isachardoublequoteopen}{\isachardot}{\kern0pt}{\isachardot}{\kern0pt}{\isachardot}{\kern0pt}\ {\isacharless}{\kern0pt}\ e{\isachardoublequoteclose}\ \isacommand{using}\isamarkupfalse%
\ e{\isacharunderscore}{\kern0pt}pos\ \isacommand{by}\isamarkupfalse%
\ simp\isanewline
\ \ \ \ \ \ \ \ \isacommand{finally}\isamarkupfalse%
\ \isacommand{have}\isamarkupfalse%
\ {\isachardoublequoteopen}diameter\ {\isacharbraceleft}{\kern0pt}s\ i\ {\isacharbar}{\kern0pt}i{\isachardot}{\kern0pt}\ r\ {\isasymle}\ i{\isacharbraceright}{\kern0pt}\ {\isacharless}{\kern0pt}\ e{\isachardoublequoteclose}\ \isacommand{using}\isamarkupfalse%
\ diameter{\isacharunderscore}{\kern0pt}SUP\ \isacommand{by}\isamarkupfalse%
\ presburger\isanewline
\ \ \ \ \ \ \isacommand{{\isacharbraceright}{\kern0pt}}\isamarkupfalse%
\isanewline
\ \ \ \ \ \ \isacommand{moreover}\isamarkupfalse%
\ \isacommand{have}\isamarkupfalse%
\ {\isachardoublequoteopen}diameter\ {\isacharbraceleft}{\kern0pt}s\ i\ {\isacharbar}{\kern0pt}i{\isachardot}{\kern0pt}\ r\ {\isasymle}\ i{\isacharbraceright}{\kern0pt}\ {\isasymge}\ {\isadigit{0}}{\isachardoublequoteclose}\ \isakeyword{for}\ r\ \isacommand{unfolding}\isamarkupfalse%
\ diameter{\isacharunderscore}{\kern0pt}SUP\ \isacommand{using}\isamarkupfalse%
\ bounded{\isacharunderscore}{\kern0pt}imp{\isacharunderscore}{\kern0pt}dist{\isacharunderscore}{\kern0pt}bounded{\isacharbrackleft}{\kern0pt}OF\ cauchy{\isacharunderscore}{\kern0pt}imp{\isacharunderscore}{\kern0pt}bounded{\isacharcomma}{\kern0pt}\ THEN\ bounded{\isacharunderscore}{\kern0pt}imp{\isacharunderscore}{\kern0pt}bdd{\isacharunderscore}{\kern0pt}above{\isacharcomma}{\kern0pt}\ OF\ {\isadigit{1}}{\isacharbrackright}{\kern0pt}\ \isacommand{by}\isamarkupfalse%
\ {\isacharparenleft}{\kern0pt}intro\ cSup{\isacharunderscore}{\kern0pt}upper{\isadigit{2}}{\isacharcomma}{\kern0pt}\ auto{\isacharparenright}{\kern0pt}\isanewline
\ \ \ \ \ \ \isacommand{ultimately}\isamarkupfalse%
\ \isacommand{show}\isamarkupfalse%
\ {\isacharquery}{\kern0pt}thesis\ \isacommand{by}\isamarkupfalse%
\ auto\isanewline
\ \ \ \ \isacommand{qed}\isamarkupfalse%
\ \ \ \ \ \ \ \ \ \ \ \ \ \ \ \ \ \isanewline
\ \ \ \ \isacommand{thus}\isamarkupfalse%
\ {\isacharquery}{\kern0pt}case\ \isacommand{using}\isamarkupfalse%
\ cauchy{\isacharunderscore}{\kern0pt}imp{\isacharunderscore}{\kern0pt}bounded{\isacharbrackleft}{\kern0pt}OF\ {\isadigit{1}}{\isacharbrackright}{\kern0pt}\ \isacommand{by}\isamarkupfalse%
\ {\isacharparenleft}{\kern0pt}simp\ add{\isacharcolon}{\kern0pt}\ LIMSEQ{\isacharunderscore}{\kern0pt}iff{\isacharparenright}{\kern0pt}\isanewline
\ \ \isacommand{next}\isamarkupfalse%
\isanewline
\ \ \ \ \isacommand{case}\isamarkupfalse%
\ {\isadigit{2}}\isanewline
\ \ \ \ \isacommand{have}\isamarkupfalse%
\ {\isachardoublequoteopen}{\isasymexists}N{\isachardot}{\kern0pt}\ {\isasymforall}n{\isasymge}N{\isachardot}{\kern0pt}\ {\isasymforall}m{\isasymge}N{\isachardot}{\kern0pt}\ dist\ {\isacharparenleft}{\kern0pt}s\ n{\isacharparenright}{\kern0pt}\ {\isacharparenleft}{\kern0pt}s\ m{\isacharparenright}{\kern0pt}\ {\isacharless}{\kern0pt}\ e{\isachardoublequoteclose}\ \isakeyword{if}\ e{\isacharunderscore}{\kern0pt}pos{\isacharcolon}{\kern0pt}\ {\isachardoublequoteopen}e\ {\isachargreater}{\kern0pt}\ {\isadigit{0}}{\isachardoublequoteclose}\ \isakeyword{for}\ e\isanewline
\ \ \ \ \isacommand{proof}\isamarkupfalse%
\ {\isacharminus}{\kern0pt}\isanewline
\ \ \ \ \ \ \isacommand{obtain}\isamarkupfalse%
\ N\ \isakeyword{where}\ diam{\isacharunderscore}{\kern0pt}less{\isacharcolon}{\kern0pt}\ {\isachardoublequoteopen}diameter\ {\isacharbraceleft}{\kern0pt}s\ i\ {\isacharbar}{\kern0pt}i{\isachardot}{\kern0pt}\ r\ {\isasymle}\ i{\isacharbraceright}{\kern0pt}\ {\isacharless}{\kern0pt}\ e{\isachardoublequoteclose}\ \isakeyword{if}\ {\isachardoublequoteopen}r\ {\isasymge}\ N{\isachardoublequoteclose}\ \isakeyword{for}\ r\ \isacommand{using}\isamarkupfalse%
\ LIMSEQ{\isacharunderscore}{\kern0pt}D\ {\isadigit{2}}{\isacharparenleft}{\kern0pt}{\isadigit{1}}{\isacharparenright}{\kern0pt}\ e{\isacharunderscore}{\kern0pt}pos\ \isacommand{by}\isamarkupfalse%
\ fastforce\isanewline
\ \ \ \ \ \ \isacommand{{\isacharbraceleft}{\kern0pt}}\isamarkupfalse%
\isanewline
\ \ \ \ \ \ \ \ \isacommand{fix}\isamarkupfalse%
\ n\ m\ \isacommand{assume}\isamarkupfalse%
\ {\isachardoublequoteopen}n\ {\isasymge}\ N{\isachardoublequoteclose}\ {\isachardoublequoteopen}m\ {\isasymge}\ N{\isachardoublequoteclose}\isanewline
\ \ \ \ \ \ \ \ \isacommand{hence}\isamarkupfalse%
\ {\isachardoublequoteopen}dist\ {\isacharparenleft}{\kern0pt}s\ n{\isacharparenright}{\kern0pt}\ {\isacharparenleft}{\kern0pt}s\ m{\isacharparenright}{\kern0pt}\ {\isacharless}{\kern0pt}\ e{\isachardoublequoteclose}\ \isacommand{using}\isamarkupfalse%
\ cSUP{\isacharunderscore}{\kern0pt}lessD{\isacharbrackleft}{\kern0pt}OF\ bounded{\isacharunderscore}{\kern0pt}imp{\isacharunderscore}{\kern0pt}dist{\isacharunderscore}{\kern0pt}bounded{\isacharbrackleft}{\kern0pt}THEN\ bounded{\isacharunderscore}{\kern0pt}imp{\isacharunderscore}{\kern0pt}bdd{\isacharunderscore}{\kern0pt}above{\isacharbrackright}{\kern0pt}{\isacharcomma}{\kern0pt}\ OF\ {\isadigit{2}}{\isacharparenleft}{\kern0pt}{\isadigit{2}}{\isacharparenright}{\kern0pt}\ diam{\isacharunderscore}{\kern0pt}less{\isacharbrackleft}{\kern0pt}unfolded\ diameter{\isacharunderscore}{\kern0pt}SUP{\isacharbrackright}{\kern0pt}{\isacharbrackright}{\kern0pt}\ \isacommand{by}\isamarkupfalse%
\ auto\isanewline
\ \ \ \ \ \ \isacommand{{\isacharbraceright}{\kern0pt}}\isamarkupfalse%
\isanewline
\ \ \ \ \ \ \isacommand{thus}\isamarkupfalse%
\ {\isacharquery}{\kern0pt}thesis\ \isacommand{by}\isamarkupfalse%
\ blast\isanewline
\ \ \ \ \isacommand{qed}\isamarkupfalse%
\isanewline
\ \ \ \ \isacommand{then}\isamarkupfalse%
\ \isacommand{show}\isamarkupfalse%
\ {\isacharquery}{\kern0pt}case\ \isacommand{by}\isamarkupfalse%
\ {\isacharparenleft}{\kern0pt}intro\ CauchyI{\isacharcomma}{\kern0pt}\ simp\ add{\isacharcolon}{\kern0pt}\ dist{\isacharunderscore}{\kern0pt}norm{\isacharparenright}{\kern0pt}\isanewline
\ \ \isacommand{qed}\isamarkupfalse%
\ \ \ \ \ \ \ \ \ \ \ \ \isanewline
\isacommand{qed}\isamarkupfalse%
%
\endisatagproof
{\isafoldproof}%
%
\isadelimproof
\isanewline
%
\endisadelimproof
\isanewline
\isacommand{context}\isamarkupfalse%
\isanewline
\ \ \isakeyword{fixes}\ s\ r\ {\isacharcolon}{\kern0pt}{\isacharcolon}{\kern0pt}\ {\isachardoublequoteopen}nat\ {\isasymRightarrow}\ {\isacharprime}{\kern0pt}a\ {\isasymRightarrow}\ {\isacharprime}{\kern0pt}b\ {\isacharcolon}{\kern0pt}{\isacharcolon}{\kern0pt}\ {\isacharbraceleft}{\kern0pt}second{\isacharunderscore}{\kern0pt}countable{\isacharunderscore}{\kern0pt}topology{\isacharcomma}{\kern0pt}\ real{\isacharunderscore}{\kern0pt}normed{\isacharunderscore}{\kern0pt}vector{\isacharcomma}{\kern0pt}\ banach{\isacharbraceright}{\kern0pt}{\isachardoublequoteclose}\ \isakeyword{and}\ M\isanewline
\ \ \isakeyword{assumes}\ bounded{\isacharcolon}{\kern0pt}\ {\isachardoublequoteopen}{\isasymAnd}x{\isachardot}{\kern0pt}\ x\ {\isasymin}\ space\ M\ {\isasymLongrightarrow}\ bounded\ {\isacharparenleft}{\kern0pt}range\ {\isacharparenleft}{\kern0pt}{\isasymlambda}i{\isachardot}{\kern0pt}\ s\ i\ x{\isacharparenright}{\kern0pt}{\isacharparenright}{\kern0pt}{\isachardoublequoteclose}\isanewline
\isakeyword{begin}\isanewline
\isanewline
\isacommand{lemma}\isamarkupfalse%
\ borel{\isacharunderscore}{\kern0pt}measurable{\isacharunderscore}{\kern0pt}diameter{\isacharcolon}{\kern0pt}\ \isanewline
\ \ \isakeyword{assumes}\ {\isacharbrackleft}{\kern0pt}measurable{\isacharbrackright}{\kern0pt}{\isacharcolon}{\kern0pt}\ {\isachardoublequoteopen}{\isasymAnd}i{\isachardot}{\kern0pt}\ {\isacharparenleft}{\kern0pt}s\ i{\isacharparenright}{\kern0pt}\ {\isasymin}\ borel{\isacharunderscore}{\kern0pt}measurable\ M{\isachardoublequoteclose}\isanewline
\ \ \isakeyword{shows}\ {\isachardoublequoteopen}{\isacharparenleft}{\kern0pt}{\isasymlambda}x{\isachardot}{\kern0pt}\ diameter\ {\isacharbraceleft}{\kern0pt}s\ i\ x\ {\isacharbar}{\kern0pt}i{\isachardot}{\kern0pt}\ n\ {\isasymle}\ i{\isacharbraceright}{\kern0pt}{\isacharparenright}{\kern0pt}\ {\isasymin}\ borel{\isacharunderscore}{\kern0pt}measurable\ M{\isachardoublequoteclose}\isanewline
%
\isadelimproof
%
\endisadelimproof
%
\isatagproof
\isacommand{proof}\isamarkupfalse%
\ {\isacharminus}{\kern0pt}\isanewline
\ \ \isacommand{have}\isamarkupfalse%
\ {\isachardoublequoteopen}{\isacharbraceleft}{\kern0pt}s\ i\ x\ {\isacharbar}{\kern0pt}i{\isachardot}{\kern0pt}\ N\ {\isasymle}\ i{\isacharbraceright}{\kern0pt}\ {\isasymnoteq}\ {\isacharbraceleft}{\kern0pt}{\isacharbraceright}{\kern0pt}{\isachardoublequoteclose}\ \isakeyword{for}\ x\ N\ \isacommand{by}\isamarkupfalse%
\ blast\isanewline
\ \ \isacommand{hence}\isamarkupfalse%
\ diameter{\isacharunderscore}{\kern0pt}SUP{\isacharcolon}{\kern0pt}\ {\isachardoublequoteopen}diameter\ {\isacharbraceleft}{\kern0pt}s\ i\ x\ {\isacharbar}{\kern0pt}i{\isachardot}{\kern0pt}\ N\ {\isasymle}\ i{\isacharbraceright}{\kern0pt}\ {\isacharequal}{\kern0pt}\ {\isacharparenleft}{\kern0pt}SUP\ {\isacharparenleft}{\kern0pt}i{\isacharcomma}{\kern0pt}\ j{\isacharparenright}{\kern0pt}\ {\isasymin}\ {\isacharbraceleft}{\kern0pt}N{\isachardot}{\kern0pt}{\isachardot}{\kern0pt}{\isacharbraceright}{\kern0pt}\ {\isasymtimes}\ {\isacharbraceleft}{\kern0pt}N{\isachardot}{\kern0pt}{\isachardot}{\kern0pt}{\isacharbraceright}{\kern0pt}{\isachardot}{\kern0pt}\ dist\ {\isacharparenleft}{\kern0pt}s\ i\ x{\isacharparenright}{\kern0pt}\ {\isacharparenleft}{\kern0pt}s\ j\ x{\isacharparenright}{\kern0pt}{\isacharparenright}{\kern0pt}{\isachardoublequoteclose}\ \isakeyword{for}\ x\ N\ \isacommand{unfolding}\isamarkupfalse%
\ diameter{\isacharunderscore}{\kern0pt}def\ \isacommand{by}\isamarkupfalse%
\ {\isacharparenleft}{\kern0pt}auto\ intro{\isacharbang}{\kern0pt}{\isacharcolon}{\kern0pt}\ arg{\isacharunderscore}{\kern0pt}cong{\isacharbrackleft}{\kern0pt}of\ {\isacharunderscore}{\kern0pt}\ {\isacharunderscore}{\kern0pt}\ Sup{\isacharbrackright}{\kern0pt}{\isacharparenright}{\kern0pt}\isanewline
\ \ \isanewline
\ \ \isacommand{have}\isamarkupfalse%
\ {\isachardoublequoteopen}case{\isacharunderscore}{\kern0pt}prod\ dist\ {\isacharbackquote}{\kern0pt}\ {\isacharparenleft}{\kern0pt}{\isacharbraceleft}{\kern0pt}s\ i\ x\ {\isacharbar}{\kern0pt}i{\isachardot}{\kern0pt}\ n\ {\isasymle}\ i{\isacharbraceright}{\kern0pt}\ {\isasymtimes}\ {\isacharbraceleft}{\kern0pt}s\ i\ x\ {\isacharbar}{\kern0pt}i{\isachardot}{\kern0pt}\ n\ {\isasymle}\ i{\isacharbraceright}{\kern0pt}{\isacharparenright}{\kern0pt}\ {\isacharequal}{\kern0pt}\ {\isacharparenleft}{\kern0pt}{\isacharparenleft}{\kern0pt}{\isasymlambda}{\isacharparenleft}{\kern0pt}i{\isacharcomma}{\kern0pt}\ j{\isacharparenright}{\kern0pt}{\isachardot}{\kern0pt}\ dist\ {\isacharparenleft}{\kern0pt}s\ i\ x{\isacharparenright}{\kern0pt}\ {\isacharparenleft}{\kern0pt}s\ j\ x{\isacharparenright}{\kern0pt}{\isacharparenright}{\kern0pt}\ {\isacharbackquote}{\kern0pt}\ {\isacharparenleft}{\kern0pt}{\isacharbraceleft}{\kern0pt}n{\isachardot}{\kern0pt}{\isachardot}{\kern0pt}{\isacharbraceright}{\kern0pt}\ {\isasymtimes}\ {\isacharbraceleft}{\kern0pt}n{\isachardot}{\kern0pt}{\isachardot}{\kern0pt}{\isacharbraceright}{\kern0pt}{\isacharparenright}{\kern0pt}{\isacharparenright}{\kern0pt}{\isachardoublequoteclose}\ \isakeyword{for}\ x\ \isacommand{by}\isamarkupfalse%
\ fast\isanewline
\ \ \isacommand{hence}\isamarkupfalse%
\ {\isacharasterisk}{\kern0pt}{\isacharcolon}{\kern0pt}\ {\isachardoublequoteopen}{\isacharparenleft}{\kern0pt}{\isasymlambda}x{\isachardot}{\kern0pt}\ diameter\ {\isacharbraceleft}{\kern0pt}s\ i\ x\ {\isacharbar}{\kern0pt}i{\isachardot}{\kern0pt}\ n\ {\isasymle}\ i{\isacharbraceright}{\kern0pt}{\isacharparenright}{\kern0pt}\ {\isacharequal}{\kern0pt}\ \ {\isacharparenleft}{\kern0pt}{\isasymlambda}x{\isachardot}{\kern0pt}\ Sup\ {\isacharparenleft}{\kern0pt}{\isacharparenleft}{\kern0pt}{\isasymlambda}{\isacharparenleft}{\kern0pt}i{\isacharcomma}{\kern0pt}\ j{\isacharparenright}{\kern0pt}{\isachardot}{\kern0pt}\ dist\ {\isacharparenleft}{\kern0pt}s\ i\ x{\isacharparenright}{\kern0pt}\ {\isacharparenleft}{\kern0pt}s\ j\ x{\isacharparenright}{\kern0pt}{\isacharparenright}{\kern0pt}\ {\isacharbackquote}{\kern0pt}\ {\isacharparenleft}{\kern0pt}{\isacharbraceleft}{\kern0pt}n{\isachardot}{\kern0pt}{\isachardot}{\kern0pt}{\isacharbraceright}{\kern0pt}\ {\isasymtimes}\ {\isacharbraceleft}{\kern0pt}n{\isachardot}{\kern0pt}{\isachardot}{\kern0pt}{\isacharbraceright}{\kern0pt}{\isacharparenright}{\kern0pt}{\isacharparenright}{\kern0pt}{\isacharparenright}{\kern0pt}{\isachardoublequoteclose}\ \isacommand{using}\isamarkupfalse%
\ diameter{\isacharunderscore}{\kern0pt}SUP\ \isacommand{by}\isamarkupfalse%
\ {\isacharparenleft}{\kern0pt}simp\ add{\isacharcolon}{\kern0pt}\ case{\isacharunderscore}{\kern0pt}prod{\isacharunderscore}{\kern0pt}beta{\isacharprime}{\kern0pt}{\isacharparenright}{\kern0pt}\isanewline
\ \ \isanewline
\ \ \isacommand{have}\isamarkupfalse%
\ {\isachardoublequoteopen}bounded\ {\isacharparenleft}{\kern0pt}{\isacharparenleft}{\kern0pt}{\isasymlambda}{\isacharparenleft}{\kern0pt}i{\isacharcomma}{\kern0pt}\ j{\isacharparenright}{\kern0pt}{\isachardot}{\kern0pt}\ dist\ {\isacharparenleft}{\kern0pt}s\ i\ x{\isacharparenright}{\kern0pt}\ {\isacharparenleft}{\kern0pt}s\ j\ x{\isacharparenright}{\kern0pt}{\isacharparenright}{\kern0pt}\ {\isacharbackquote}{\kern0pt}\ {\isacharparenleft}{\kern0pt}{\isacharbraceleft}{\kern0pt}n{\isachardot}{\kern0pt}{\isachardot}{\kern0pt}{\isacharbraceright}{\kern0pt}\ {\isasymtimes}\ {\isacharbraceleft}{\kern0pt}n{\isachardot}{\kern0pt}{\isachardot}{\kern0pt}{\isacharbraceright}{\kern0pt}{\isacharparenright}{\kern0pt}{\isacharparenright}{\kern0pt}{\isachardoublequoteclose}\ \isakeyword{if}\ {\isachardoublequoteopen}x\ {\isasymin}\ space\ M{\isachardoublequoteclose}\ \isakeyword{for}\ x\ \isacommand{by}\isamarkupfalse%
\ {\isacharparenleft}{\kern0pt}rule\ bounded{\isacharunderscore}{\kern0pt}imp{\isacharunderscore}{\kern0pt}dist{\isacharunderscore}{\kern0pt}bounded{\isacharbrackleft}{\kern0pt}OF\ bounded{\isacharcomma}{\kern0pt}\ OF\ that{\isacharbrackright}{\kern0pt}{\isacharparenright}{\kern0pt}\isanewline
\ \ \isacommand{hence}\isamarkupfalse%
\ bdd{\isacharcolon}{\kern0pt}\ {\isachardoublequoteopen}bdd{\isacharunderscore}{\kern0pt}above\ {\isacharparenleft}{\kern0pt}{\isacharparenleft}{\kern0pt}{\isasymlambda}{\isacharparenleft}{\kern0pt}i{\isacharcomma}{\kern0pt}\ j{\isacharparenright}{\kern0pt}{\isachardot}{\kern0pt}\ dist\ {\isacharparenleft}{\kern0pt}s\ i\ x{\isacharparenright}{\kern0pt}\ {\isacharparenleft}{\kern0pt}s\ j\ x{\isacharparenright}{\kern0pt}{\isacharparenright}{\kern0pt}\ {\isacharbackquote}{\kern0pt}\ {\isacharparenleft}{\kern0pt}{\isacharbraceleft}{\kern0pt}n{\isachardot}{\kern0pt}{\isachardot}{\kern0pt}{\isacharbraceright}{\kern0pt}\ {\isasymtimes}\ {\isacharbraceleft}{\kern0pt}n{\isachardot}{\kern0pt}{\isachardot}{\kern0pt}{\isacharbraceright}{\kern0pt}{\isacharparenright}{\kern0pt}{\isacharparenright}{\kern0pt}{\isachardoublequoteclose}\ \isakeyword{if}\ {\isachardoublequoteopen}x\ {\isasymin}\ space\ M{\isachardoublequoteclose}\ \isakeyword{for}\ x\ \isacommand{using}\isamarkupfalse%
\ that\ bounded{\isacharunderscore}{\kern0pt}imp{\isacharunderscore}{\kern0pt}bdd{\isacharunderscore}{\kern0pt}above\ \isacommand{by}\isamarkupfalse%
\ presburger\isanewline
\ \ \isacommand{have}\isamarkupfalse%
\ {\isachardoublequoteopen}fst\ p\ {\isasymin}\ borel{\isacharunderscore}{\kern0pt}measurable\ M{\isachardoublequoteclose}\ {\isachardoublequoteopen}snd\ p\ {\isasymin}\ borel{\isacharunderscore}{\kern0pt}measurable\ M{\isachardoublequoteclose}\ \isakeyword{if}\ {\isachardoublequoteopen}p\ {\isasymin}\ s\ {\isacharbackquote}{\kern0pt}\ {\isacharbraceleft}{\kern0pt}n{\isachardot}{\kern0pt}{\isachardot}{\kern0pt}{\isacharbraceright}{\kern0pt}\ {\isasymtimes}\ s\ {\isacharbackquote}{\kern0pt}\ {\isacharbraceleft}{\kern0pt}n{\isachardot}{\kern0pt}{\isachardot}{\kern0pt}{\isacharbraceright}{\kern0pt}{\isachardoublequoteclose}\ \isakeyword{for}\ p\ \isacommand{using}\isamarkupfalse%
\ that\ \isacommand{by}\isamarkupfalse%
\ fastforce{\isacharplus}{\kern0pt}\isanewline
\ \ \isacommand{hence}\isamarkupfalse%
\ {\isachardoublequoteopen}{\isacharparenleft}{\kern0pt}{\isasymlambda}x{\isachardot}{\kern0pt}\ fst\ p\ x\ {\isacharminus}{\kern0pt}\ snd\ p\ x{\isacharparenright}{\kern0pt}\ {\isasymin}\ borel{\isacharunderscore}{\kern0pt}measurable\ M{\isachardoublequoteclose}\ \isakeyword{if}\ {\isachardoublequoteopen}p\ {\isasymin}\ s\ {\isacharbackquote}{\kern0pt}\ {\isacharbraceleft}{\kern0pt}n{\isachardot}{\kern0pt}{\isachardot}{\kern0pt}{\isacharbraceright}{\kern0pt}\ {\isasymtimes}\ s\ {\isacharbackquote}{\kern0pt}\ {\isacharbraceleft}{\kern0pt}n{\isachardot}{\kern0pt}{\isachardot}{\kern0pt}{\isacharbraceright}{\kern0pt}{\isachardoublequoteclose}\ \isakeyword{for}\ p\ \isacommand{using}\isamarkupfalse%
\ that\ borel{\isacharunderscore}{\kern0pt}measurable{\isacharunderscore}{\kern0pt}diff\ \isacommand{by}\isamarkupfalse%
\ simp\isanewline
\ \ \isacommand{hence}\isamarkupfalse%
\ {\isachardoublequoteopen}{\isacharparenleft}{\kern0pt}{\isasymlambda}x{\isachardot}{\kern0pt}\ case\ p\ of\ {\isacharparenleft}{\kern0pt}f{\isacharcomma}{\kern0pt}\ g{\isacharparenright}{\kern0pt}\ {\isasymRightarrow}\ dist\ {\isacharparenleft}{\kern0pt}f\ x{\isacharparenright}{\kern0pt}\ {\isacharparenleft}{\kern0pt}g\ x{\isacharparenright}{\kern0pt}{\isacharparenright}{\kern0pt}\ {\isasymin}\ borel{\isacharunderscore}{\kern0pt}measurable\ M{\isachardoublequoteclose}\ \isakeyword{if}\ {\isachardoublequoteopen}p\ {\isasymin}\ s\ {\isacharbackquote}{\kern0pt}\ {\isacharbraceleft}{\kern0pt}n{\isachardot}{\kern0pt}{\isachardot}{\kern0pt}{\isacharbraceright}{\kern0pt}\ {\isasymtimes}\ s\ {\isacharbackquote}{\kern0pt}\ {\isacharbraceleft}{\kern0pt}n{\isachardot}{\kern0pt}{\isachardot}{\kern0pt}{\isacharbraceright}{\kern0pt}{\isachardoublequoteclose}\ \isakeyword{for}\ p\ \isacommand{unfolding}\isamarkupfalse%
\ dist{\isacharunderscore}{\kern0pt}norm\ \isacommand{using}\isamarkupfalse%
\ that\ \isacommand{by}\isamarkupfalse%
\ measurable\isanewline
\ \ \isacommand{moreover}\isamarkupfalse%
\ \isacommand{have}\isamarkupfalse%
\ {\isachardoublequoteopen}countable\ {\isacharparenleft}{\kern0pt}s\ {\isacharbackquote}{\kern0pt}\ {\isacharbraceleft}{\kern0pt}n{\isachardot}{\kern0pt}{\isachardot}{\kern0pt}{\isacharbraceright}{\kern0pt}\ {\isasymtimes}\ s\ {\isacharbackquote}{\kern0pt}\ {\isacharbraceleft}{\kern0pt}n{\isachardot}{\kern0pt}{\isachardot}{\kern0pt}{\isacharbraceright}{\kern0pt}{\isacharparenright}{\kern0pt}{\isachardoublequoteclose}\ \isacommand{by}\isamarkupfalse%
\ {\isacharparenleft}{\kern0pt}intro\ countable{\isacharunderscore}{\kern0pt}SIGMA\ countable{\isacharunderscore}{\kern0pt}image{\isacharcomma}{\kern0pt}\ auto{\isacharparenright}{\kern0pt}\isanewline
\ \ \isacommand{ultimately}\isamarkupfalse%
\ \isacommand{show}\isamarkupfalse%
\ {\isacharquery}{\kern0pt}thesis\ \isacommand{unfolding}\isamarkupfalse%
\ {\isacharasterisk}{\kern0pt}\ \isacommand{by}\isamarkupfalse%
\ {\isacharparenleft}{\kern0pt}auto\ intro{\isacharbang}{\kern0pt}{\isacharcolon}{\kern0pt}\ borel{\isacharunderscore}{\kern0pt}measurable{\isacharunderscore}{\kern0pt}cSUP\ bdd{\isacharparenright}{\kern0pt}\isanewline
\isacommand{qed}\isamarkupfalse%
%
\endisatagproof
{\isafoldproof}%
%
\isadelimproof
\isanewline
%
\endisadelimproof
\isanewline
\isacommand{lemma}\isamarkupfalse%
\ integrable{\isacharunderscore}{\kern0pt}bound{\isacharunderscore}{\kern0pt}diameter{\isacharcolon}{\kern0pt}\isanewline
\ \ \isakeyword{fixes}\ f\ {\isacharcolon}{\kern0pt}{\isacharcolon}{\kern0pt}\ {\isachardoublequoteopen}{\isacharprime}{\kern0pt}a\ {\isasymRightarrow}\ real{\isachardoublequoteclose}\isanewline
\ \ \isakeyword{assumes}\ {\isachardoublequoteopen}integrable\ M\ f{\isachardoublequoteclose}\ \isanewline
\ \ \ \ \ \ \isakeyword{and}\ {\isacharbrackleft}{\kern0pt}measurable{\isacharbrackright}{\kern0pt}{\isacharcolon}{\kern0pt}\ {\isachardoublequoteopen}{\isasymAnd}i{\isachardot}{\kern0pt}\ {\isacharparenleft}{\kern0pt}s\ i{\isacharparenright}{\kern0pt}\ {\isasymin}\ borel{\isacharunderscore}{\kern0pt}measurable\ M{\isachardoublequoteclose}\isanewline
\ \ \ \ \ \ \isakeyword{and}\ {\isachardoublequoteopen}{\isasymAnd}x\ i{\isachardot}{\kern0pt}\ x\ {\isasymin}\ space\ M\ {\isasymLongrightarrow}\ norm\ {\isacharparenleft}{\kern0pt}s\ i\ x{\isacharparenright}{\kern0pt}\ {\isasymle}\ f\ x{\isachardoublequoteclose}\isanewline
\ \ \ \ \isakeyword{shows}\ {\isachardoublequoteopen}integrable\ M\ {\isacharparenleft}{\kern0pt}{\isasymlambda}x{\isachardot}{\kern0pt}\ diameter\ {\isacharbraceleft}{\kern0pt}s\ i\ x\ {\isacharbar}{\kern0pt}i{\isachardot}{\kern0pt}\ n\ {\isasymle}\ i{\isacharbraceright}{\kern0pt}{\isacharparenright}{\kern0pt}{\isachardoublequoteclose}\isanewline
%
\isadelimproof
%
\endisadelimproof
%
\isatagproof
\isacommand{proof}\isamarkupfalse%
\ {\isacharminus}{\kern0pt}\isanewline
\ \ \isacommand{have}\isamarkupfalse%
\ {\isachardoublequoteopen}{\isacharbraceleft}{\kern0pt}s\ i\ x\ {\isacharbar}{\kern0pt}i{\isachardot}{\kern0pt}\ N\ {\isasymle}\ i{\isacharbraceright}{\kern0pt}\ {\isasymnoteq}\ {\isacharbraceleft}{\kern0pt}{\isacharbraceright}{\kern0pt}{\isachardoublequoteclose}\ \isakeyword{for}\ x\ N\ \isacommand{by}\isamarkupfalse%
\ blast\isanewline
\ \ \isacommand{hence}\isamarkupfalse%
\ diameter{\isacharunderscore}{\kern0pt}SUP{\isacharcolon}{\kern0pt}\ {\isachardoublequoteopen}diameter\ {\isacharbraceleft}{\kern0pt}s\ i\ x\ {\isacharbar}{\kern0pt}i{\isachardot}{\kern0pt}\ N\ {\isasymle}\ i{\isacharbraceright}{\kern0pt}\ {\isacharequal}{\kern0pt}\ {\isacharparenleft}{\kern0pt}SUP\ {\isacharparenleft}{\kern0pt}i{\isacharcomma}{\kern0pt}\ j{\isacharparenright}{\kern0pt}\ {\isasymin}\ {\isacharbraceleft}{\kern0pt}N{\isachardot}{\kern0pt}{\isachardot}{\kern0pt}{\isacharbraceright}{\kern0pt}\ {\isasymtimes}\ {\isacharbraceleft}{\kern0pt}N{\isachardot}{\kern0pt}{\isachardot}{\kern0pt}{\isacharbraceright}{\kern0pt}{\isachardot}{\kern0pt}\ dist\ {\isacharparenleft}{\kern0pt}s\ i\ x{\isacharparenright}{\kern0pt}\ {\isacharparenleft}{\kern0pt}s\ j\ x{\isacharparenright}{\kern0pt}{\isacharparenright}{\kern0pt}{\isachardoublequoteclose}\ \isakeyword{for}\ x\ N\ \isacommand{unfolding}\isamarkupfalse%
\ diameter{\isacharunderscore}{\kern0pt}def\ \isacommand{by}\isamarkupfalse%
\ {\isacharparenleft}{\kern0pt}auto\ intro{\isacharbang}{\kern0pt}{\isacharcolon}{\kern0pt}\ arg{\isacharunderscore}{\kern0pt}cong{\isacharbrackleft}{\kern0pt}of\ {\isacharunderscore}{\kern0pt}\ {\isacharunderscore}{\kern0pt}\ Sup{\isacharbrackright}{\kern0pt}{\isacharparenright}{\kern0pt}\isanewline
\ \ \isacommand{{\isacharbraceleft}{\kern0pt}}\isamarkupfalse%
\isanewline
\ \ \ \ \isacommand{fix}\isamarkupfalse%
\ x\ \isacommand{assume}\isamarkupfalse%
\ x{\isacharcolon}{\kern0pt}\ {\isachardoublequoteopen}x\ {\isasymin}\ space\ M{\isachardoublequoteclose}\isanewline
\ \ \ \ \isacommand{let}\isamarkupfalse%
\ {\isacharquery}{\kern0pt}S\ {\isacharequal}{\kern0pt}\ {\isachardoublequoteopen}{\isacharparenleft}{\kern0pt}{\isasymlambda}{\isacharparenleft}{\kern0pt}i{\isacharcomma}{\kern0pt}\ j{\isacharparenright}{\kern0pt}{\isachardot}{\kern0pt}\ dist\ {\isacharparenleft}{\kern0pt}s\ i\ x{\isacharparenright}{\kern0pt}\ {\isacharparenleft}{\kern0pt}s\ j\ x{\isacharparenright}{\kern0pt}{\isacharparenright}{\kern0pt}\ {\isacharbackquote}{\kern0pt}\ {\isacharparenleft}{\kern0pt}{\isacharbraceleft}{\kern0pt}n{\isachardot}{\kern0pt}{\isachardot}{\kern0pt}{\isacharbraceright}{\kern0pt}\ {\isasymtimes}\ {\isacharbraceleft}{\kern0pt}n{\isachardot}{\kern0pt}{\isachardot}{\kern0pt}{\isacharbraceright}{\kern0pt}{\isacharparenright}{\kern0pt}{\isachardoublequoteclose}\isanewline
\ \ \ \ \isacommand{have}\isamarkupfalse%
\ {\isachardoublequoteopen}case{\isacharunderscore}{\kern0pt}prod\ dist\ {\isacharbackquote}{\kern0pt}\ {\isacharparenleft}{\kern0pt}{\isacharbraceleft}{\kern0pt}s\ i\ x\ {\isacharbar}{\kern0pt}i{\isachardot}{\kern0pt}\ n\ {\isasymle}\ i{\isacharbraceright}{\kern0pt}\ {\isasymtimes}\ {\isacharbraceleft}{\kern0pt}s\ i\ x\ {\isacharbar}{\kern0pt}i{\isachardot}{\kern0pt}\ n\ {\isasymle}\ i{\isacharbraceright}{\kern0pt}{\isacharparenright}{\kern0pt}\ {\isacharequal}{\kern0pt}\ {\isacharparenleft}{\kern0pt}{\isasymlambda}{\isacharparenleft}{\kern0pt}i{\isacharcomma}{\kern0pt}\ j{\isacharparenright}{\kern0pt}{\isachardot}{\kern0pt}\ dist\ {\isacharparenleft}{\kern0pt}s\ i\ x{\isacharparenright}{\kern0pt}\ {\isacharparenleft}{\kern0pt}s\ j\ x{\isacharparenright}{\kern0pt}{\isacharparenright}{\kern0pt}\ {\isacharbackquote}{\kern0pt}\ {\isacharparenleft}{\kern0pt}{\isacharbraceleft}{\kern0pt}n{\isachardot}{\kern0pt}{\isachardot}{\kern0pt}{\isacharbraceright}{\kern0pt}\ {\isasymtimes}\ {\isacharbraceleft}{\kern0pt}n{\isachardot}{\kern0pt}{\isachardot}{\kern0pt}{\isacharbraceright}{\kern0pt}{\isacharparenright}{\kern0pt}{\isachardoublequoteclose}\ \isacommand{by}\isamarkupfalse%
\ fast\isanewline
\ \ \ \ \isacommand{hence}\isamarkupfalse%
\ {\isacharasterisk}{\kern0pt}{\isacharcolon}{\kern0pt}\ {\isachardoublequoteopen}diameter\ {\isacharbraceleft}{\kern0pt}s\ i\ x\ {\isacharbar}{\kern0pt}i{\isachardot}{\kern0pt}\ n\ {\isasymle}\ i{\isacharbraceright}{\kern0pt}\ {\isacharequal}{\kern0pt}\ \ Sup\ {\isacharquery}{\kern0pt}S{\isachardoublequoteclose}\ \isacommand{using}\isamarkupfalse%
\ diameter{\isacharunderscore}{\kern0pt}SUP\ \isacommand{by}\isamarkupfalse%
\ {\isacharparenleft}{\kern0pt}simp\ add{\isacharcolon}{\kern0pt}\ case{\isacharunderscore}{\kern0pt}prod{\isacharunderscore}{\kern0pt}beta{\isacharprime}{\kern0pt}{\isacharparenright}{\kern0pt}\isanewline
\ \ \ \ \isanewline
\ \ \ \ \isacommand{have}\isamarkupfalse%
\ {\isachardoublequoteopen}bounded\ {\isacharquery}{\kern0pt}S{\isachardoublequoteclose}\ \isacommand{by}\isamarkupfalse%
\ {\isacharparenleft}{\kern0pt}rule\ bounded{\isacharunderscore}{\kern0pt}imp{\isacharunderscore}{\kern0pt}dist{\isacharunderscore}{\kern0pt}bounded{\isacharbrackleft}{\kern0pt}OF\ bounded{\isacharbrackleft}{\kern0pt}OF\ x{\isacharbrackright}{\kern0pt}{\isacharbrackright}{\kern0pt}{\isacharparenright}{\kern0pt}\isanewline
\ \ \ \ \isacommand{hence}\isamarkupfalse%
\ Sup{\isacharunderscore}{\kern0pt}S{\isacharunderscore}{\kern0pt}nonneg{\isacharcolon}{\kern0pt}{\isachardoublequoteopen}{\isadigit{0}}\ {\isasymle}\ Sup\ {\isacharquery}{\kern0pt}S{\isachardoublequoteclose}\ \isacommand{by}\isamarkupfalse%
\ {\isacharparenleft}{\kern0pt}auto\ intro{\isacharbang}{\kern0pt}{\isacharcolon}{\kern0pt}\ cSup{\isacharunderscore}{\kern0pt}upper{\isadigit{2}}\ x\ bounded{\isacharunderscore}{\kern0pt}imp{\isacharunderscore}{\kern0pt}bdd{\isacharunderscore}{\kern0pt}above{\isacharparenright}{\kern0pt}\isanewline
\isanewline
\ \ \ \ \isacommand{have}\isamarkupfalse%
\ {\isachardoublequoteopen}dist\ {\isacharparenleft}{\kern0pt}s\ i\ x{\isacharparenright}{\kern0pt}\ {\isacharparenleft}{\kern0pt}s\ j\ x{\isacharparenright}{\kern0pt}\ {\isasymle}\ \ {\isadigit{2}}\ {\isacharasterisk}{\kern0pt}\ f\ x{\isachardoublequoteclose}\ \isakeyword{for}\ i\ j\ \isacommand{by}\isamarkupfalse%
\ {\isacharparenleft}{\kern0pt}intro\ dist{\isacharunderscore}{\kern0pt}triangle{\isadigit{2}}{\isacharbrackleft}{\kern0pt}THEN\ order{\isacharunderscore}{\kern0pt}trans{\isacharcomma}{\kern0pt}\ of\ {\isacharunderscore}{\kern0pt}\ {\isadigit{0}}{\isacharbrackright}{\kern0pt}{\isacharparenright}{\kern0pt}\ {\isacharparenleft}{\kern0pt}metis\ norm{\isacharunderscore}{\kern0pt}conv{\isacharunderscore}{\kern0pt}dist\ assms{\isacharparenleft}{\kern0pt}{\isadigit{3}}{\isacharparenright}{\kern0pt}\ x\ add{\isacharunderscore}{\kern0pt}mono\ mult{\isacharunderscore}{\kern0pt}{\isadigit{2}}{\isacharparenright}{\kern0pt}\isanewline
\ \ \ \ \isacommand{hence}\isamarkupfalse%
\ {\isachardoublequoteopen}{\isasymforall}c\ {\isasymin}\ {\isacharquery}{\kern0pt}S{\isachardot}{\kern0pt}\ c\ {\isasymle}\ {\isadigit{2}}\ {\isacharasterisk}{\kern0pt}\ f\ x{\isachardoublequoteclose}\ \isacommand{by}\isamarkupfalse%
\ force\isanewline
\ \ \ \ \isacommand{hence}\isamarkupfalse%
\ {\isachardoublequoteopen}Sup\ {\isacharquery}{\kern0pt}S\ {\isasymle}\ {\isadigit{2}}\ {\isacharasterisk}{\kern0pt}\ f\ x{\isachardoublequoteclose}\ \isacommand{by}\isamarkupfalse%
\ {\isacharparenleft}{\kern0pt}intro\ cSup{\isacharunderscore}{\kern0pt}least{\isacharcomma}{\kern0pt}\ auto{\isacharparenright}{\kern0pt}\isanewline
\ \ \ \ \isacommand{hence}\isamarkupfalse%
\ {\isachardoublequoteopen}norm\ {\isacharparenleft}{\kern0pt}Sup\ {\isacharquery}{\kern0pt}S{\isacharparenright}{\kern0pt}\ {\isasymle}\ {\isadigit{2}}\ {\isacharasterisk}{\kern0pt}\ norm\ {\isacharparenleft}{\kern0pt}f\ x{\isacharparenright}{\kern0pt}{\isachardoublequoteclose}\ \isacommand{using}\isamarkupfalse%
\ Sup{\isacharunderscore}{\kern0pt}S{\isacharunderscore}{\kern0pt}nonneg\ \isacommand{by}\isamarkupfalse%
\ auto\isanewline
\ \ \ \ \isacommand{also}\isamarkupfalse%
\ \isacommand{have}\isamarkupfalse%
\ {\isachardoublequoteopen}{\isachardot}{\kern0pt}{\isachardot}{\kern0pt}{\isachardot}{\kern0pt}\ {\isacharequal}{\kern0pt}\ norm\ {\isacharparenleft}{\kern0pt}{\isadigit{2}}\ {\isacharasterisk}{\kern0pt}\isactrlsub R\ f\ x{\isacharparenright}{\kern0pt}{\isachardoublequoteclose}\ \isacommand{by}\isamarkupfalse%
\ simp\isanewline
\ \ \ \ \isacommand{finally}\isamarkupfalse%
\ \isacommand{have}\isamarkupfalse%
\ {\isachardoublequoteopen}norm\ {\isacharparenleft}{\kern0pt}diameter\ {\isacharbraceleft}{\kern0pt}s\ i\ x\ {\isacharbar}{\kern0pt}i{\isachardot}{\kern0pt}\ n\ {\isasymle}\ i{\isacharbraceright}{\kern0pt}{\isacharparenright}{\kern0pt}\ {\isasymle}\ norm\ {\isacharparenleft}{\kern0pt}{\isadigit{2}}\ {\isacharasterisk}{\kern0pt}\isactrlsub R\ f\ x{\isacharparenright}{\kern0pt}{\isachardoublequoteclose}\ \isacommand{unfolding}\isamarkupfalse%
\ {\isacharasterisk}{\kern0pt}\ \isacommand{{\isachardot}{\kern0pt}}\isamarkupfalse%
\isanewline
\ \ \isacommand{{\isacharbraceright}{\kern0pt}}\isamarkupfalse%
\isanewline
\ \ \isacommand{hence}\isamarkupfalse%
\ {\isachardoublequoteopen}AE\ x\ in\ M{\isachardot}{\kern0pt}\ norm\ {\isacharparenleft}{\kern0pt}diameter\ {\isacharbraceleft}{\kern0pt}s\ i\ x\ {\isacharbar}{\kern0pt}i{\isachardot}{\kern0pt}\ n\ {\isasymle}\ i{\isacharbraceright}{\kern0pt}{\isacharparenright}{\kern0pt}\ {\isasymle}\ norm\ {\isacharparenleft}{\kern0pt}{\isadigit{2}}\ {\isacharasterisk}{\kern0pt}\isactrlsub R\ f\ x{\isacharparenright}{\kern0pt}{\isachardoublequoteclose}\ \isacommand{by}\isamarkupfalse%
\ blast\isanewline
\ \ \isacommand{thus}\isamarkupfalse%
\ \ {\isachardoublequoteopen}integrable\ M\ {\isacharparenleft}{\kern0pt}{\isasymlambda}x{\isachardot}{\kern0pt}\ diameter\ {\isacharbraceleft}{\kern0pt}s\ i\ x\ {\isacharbar}{\kern0pt}i{\isachardot}{\kern0pt}\ n\ {\isasymle}\ i{\isacharbraceright}{\kern0pt}{\isacharparenright}{\kern0pt}{\isachardoublequoteclose}\ \isacommand{using}\isamarkupfalse%
\ borel{\isacharunderscore}{\kern0pt}measurable{\isacharunderscore}{\kern0pt}diameter\ \isacommand{by}\isamarkupfalse%
\ {\isacharparenleft}{\kern0pt}intro\ Bochner{\isacharunderscore}{\kern0pt}Integration{\isachardot}{\kern0pt}integrable{\isacharunderscore}{\kern0pt}bound{\isacharbrackleft}{\kern0pt}OF\ assms{\isacharparenleft}{\kern0pt}{\isadigit{1}}{\isacharparenright}{\kern0pt}{\isacharbrackleft}{\kern0pt}THEN\ integrable{\isacharunderscore}{\kern0pt}scaleR{\isacharunderscore}{\kern0pt}right{\isacharbrackleft}{\kern0pt}of\ {\isadigit{2}}{\isacharbrackright}{\kern0pt}{\isacharbrackright}{\kern0pt}{\isacharbrackright}{\kern0pt}{\isacharcomma}{\kern0pt}\ measurable{\isacharparenright}{\kern0pt}\isanewline
\isacommand{qed}\isamarkupfalse%
%
\endisatagproof
{\isafoldproof}%
%
\isadelimproof
\isanewline
%
\endisadelimproof
\isacommand{end}\isamarkupfalse%
\ \ \ \ \isanewline
%
\isadelimtheory
\ \ \isanewline
%
\endisadelimtheory
%
\isatagtheory
\isacommand{end}\isamarkupfalse%
%
\endisatagtheory
{\isafoldtheory}%
%
\isadelimtheory
%
\endisadelimtheory
%
\end{isabellebody}%
\endinput
%:%file=Elementary_Metric_Spaces_Addendum.tex%:%
%:%10=1%:%
%:%11=1%:%
%:%12=2%:%
%:%13=3%:%
%:%18=3%:%
%:%21=4%:%
%:%22=5%:%
%:%23=5%:%
%:%24=6%:%
%:%25=7%:%
%:%26=8%:%
%:%33=9%:%
%:%34=9%:%
%:%35=10%:%
%:%36=10%:%
%:%37=10%:%
%:%38=10%:%
%:%39=11%:%
%:%40=11%:%
%:%45=11%:%
%:%48=12%:%
%:%49=13%:%
%:%50=13%:%
%:%51=14%:%
%:%52=15%:%
%:%53=16%:%
%:%60=17%:%
%:%61=17%:%
%:%62=18%:%
%:%63=18%:%
%:%64=18%:%
%:%65=18%:%
%:%66=19%:%
%:%67=19%:%
%:%68=19%:%
%:%69=19%:%
%:%70=20%:%
%:%71=20%:%
%:%72=20%:%
%:%73=20%:%
%:%74=21%:%
%:%80=21%:%
%:%83=22%:%
%:%84=23%:%
%:%85=23%:%
%:%86=24%:%
%:%87=25%:%
%:%90=26%:%
%:%94=26%:%
%:%95=26%:%
%:%96=26%:%
%:%101=26%:%
%:%104=27%:%
%:%105=28%:%
%:%106=28%:%
%:%107=29%:%
%:%108=30%:%
%:%115=31%:%
%:%116=31%:%
%:%117=32%:%
%:%118=32%:%
%:%119=32%:%
%:%120=33%:%
%:%121=33%:%
%:%122=33%:%
%:%123=33%:%
%:%124=34%:%
%:%125=34%:%
%:%126=35%:%
%:%127=35%:%
%:%128=36%:%
%:%129=36%:%
%:%130=37%:%
%:%131=37%:%
%:%132=38%:%
%:%133=38%:%
%:%134=39%:%
%:%135=39%:%
%:%136=39%:%
%:%137=39%:%
%:%138=40%:%
%:%139=40%:%
%:%140=41%:%
%:%141=41%:%
%:%142=41%:%
%:%143=42%:%
%:%144=42%:%
%:%145=42%:%
%:%146=42%:%
%:%147=43%:%
%:%148=43%:%
%:%149=43%:%
%:%150=44%:%
%:%151=44%:%
%:%152=44%:%
%:%153=44%:%
%:%154=44%:%
%:%155=45%:%
%:%156=45%:%
%:%157=45%:%
%:%158=45%:%
%:%159=45%:%
%:%160=46%:%
%:%161=46%:%
%:%162=47%:%
%:%163=47%:%
%:%164=47%:%
%:%165=47%:%
%:%166=47%:%
%:%167=47%:%
%:%168=48%:%
%:%169=48%:%
%:%170=48%:%
%:%171=48%:%
%:%172=49%:%
%:%173=49%:%
%:%174=50%:%
%:%175=50%:%
%:%176=50%:%
%:%177=50%:%
%:%178=51%:%
%:%179=51%:%
%:%180=52%:%
%:%181=52%:%
%:%182=53%:%
%:%183=53%:%
%:%184=54%:%
%:%185=54%:%
%:%186=55%:%
%:%187=55%:%
%:%188=55%:%
%:%189=55%:%
%:%190=56%:%
%:%191=56%:%
%:%192=57%:%
%:%193=57%:%
%:%194=57%:%
%:%195=58%:%
%:%196=58%:%
%:%197=58%:%
%:%198=58%:%
%:%199=59%:%
%:%200=59%:%
%:%201=60%:%
%:%202=60%:%
%:%203=60%:%
%:%204=61%:%
%:%205=61%:%
%:%206=62%:%
%:%207=62%:%
%:%208=62%:%
%:%209=62%:%
%:%210=63%:%
%:%211=63%:%
%:%212=64%:%
%:%218=64%:%
%:%221=65%:%
%:%222=66%:%
%:%223=66%:%
%:%224=67%:%
%:%225=68%:%
%:%226=69%:%
%:%227=70%:%
%:%228=71%:%
%:%229=71%:%
%:%230=72%:%
%:%231=73%:%
%:%238=74%:%
%:%239=74%:%
%:%240=75%:%
%:%241=75%:%
%:%242=75%:%
%:%243=76%:%
%:%244=76%:%
%:%245=76%:%
%:%246=76%:%
%:%247=77%:%
%:%248=78%:%
%:%249=78%:%
%:%250=78%:%
%:%251=79%:%
%:%252=79%:%
%:%253=79%:%
%:%254=79%:%
%:%255=80%:%
%:%256=81%:%
%:%257=81%:%
%:%258=81%:%
%:%259=82%:%
%:%260=82%:%
%:%261=82%:%
%:%262=82%:%
%:%263=83%:%
%:%264=83%:%
%:%265=83%:%
%:%266=83%:%
%:%267=84%:%
%:%268=84%:%
%:%269=84%:%
%:%270=84%:%
%:%271=85%:%
%:%272=85%:%
%:%273=85%:%
%:%274=85%:%
%:%275=85%:%
%:%276=86%:%
%:%277=86%:%
%:%278=86%:%
%:%279=86%:%
%:%280=87%:%
%:%281=87%:%
%:%282=87%:%
%:%283=87%:%
%:%284=87%:%
%:%285=88%:%
%:%291=88%:%
%:%294=89%:%
%:%295=90%:%
%:%296=90%:%
%:%297=91%:%
%:%298=92%:%
%:%299=93%:%
%:%300=94%:%
%:%301=95%:%
%:%308=96%:%
%:%309=96%:%
%:%310=97%:%
%:%311=97%:%
%:%312=97%:%
%:%313=98%:%
%:%314=98%:%
%:%315=98%:%
%:%316=98%:%
%:%317=99%:%
%:%318=99%:%
%:%319=100%:%
%:%320=100%:%
%:%321=100%:%
%:%322=101%:%
%:%323=101%:%
%:%324=102%:%
%:%325=102%:%
%:%326=102%:%
%:%327=103%:%
%:%328=103%:%
%:%329=103%:%
%:%330=103%:%
%:%331=104%:%
%:%332=105%:%
%:%333=105%:%
%:%334=105%:%
%:%335=106%:%
%:%336=106%:%
%:%337=106%:%
%:%338=107%:%
%:%339=108%:%
%:%340=108%:%
%:%341=108%:%
%:%342=109%:%
%:%343=109%:%
%:%344=109%:%
%:%345=110%:%
%:%346=110%:%
%:%347=110%:%
%:%348=111%:%
%:%349=111%:%
%:%350=111%:%
%:%351=111%:%
%:%352=112%:%
%:%353=112%:%
%:%354=112%:%
%:%355=112%:%
%:%356=113%:%
%:%357=113%:%
%:%358=113%:%
%:%359=113%:%
%:%360=113%:%
%:%361=114%:%
%:%362=114%:%
%:%363=115%:%
%:%364=115%:%
%:%365=115%:%
%:%366=116%:%
%:%367=116%:%
%:%368=116%:%
%:%369=116%:%
%:%370=117%:%
%:%376=117%:%
%:%379=118%:%
%:%380=118%:%
%:%383=119%:%
%:%388=120%:%

%
\begin{isabellebody}%
\setisabellecontext{Bochner{\isacharunderscore}{\kern0pt}Integration{\isacharunderscore}{\kern0pt}Addendum}%
%
\isadelimtheory
%
\endisadelimtheory
%
\isatagtheory
\isacommand{theory}\isamarkupfalse%
\ Bochner{\isacharunderscore}{\kern0pt}Integration{\isacharunderscore}{\kern0pt}Addendum\isanewline
\ \ \isakeyword{imports}\ {\isachardoublequoteopen}HOL{\isacharminus}{\kern0pt}Analysis{\isachardot}{\kern0pt}Bochner{\isacharunderscore}{\kern0pt}Integration{\isachardoublequoteclose}\isanewline
\isakeyword{begin}%
\endisatagtheory
{\isafoldtheory}%
%
\isadelimtheory
%
\endisadelimtheory
%
\isadelimdocument
%
\endisadelimdocument
%
\isatagdocument
%
\isamarkupsubsection{Simple Functions%
}
\isamarkuptrue%
%
\endisatagdocument
{\isafolddocument}%
%
\isadelimdocument
%
\endisadelimdocument
\isacommand{lemma}\isamarkupfalse%
\ integrable{\isacharunderscore}{\kern0pt}implies{\isacharunderscore}{\kern0pt}simple{\isacharunderscore}{\kern0pt}function{\isacharunderscore}{\kern0pt}sequence{\isacharcolon}{\kern0pt}\isanewline
\ \ \isakeyword{fixes}\ f\ {\isacharcolon}{\kern0pt}{\isacharcolon}{\kern0pt}\ {\isachardoublequoteopen}{\isacharprime}{\kern0pt}a\ {\isasymRightarrow}\ {\isacharprime}{\kern0pt}b{\isacharcolon}{\kern0pt}{\isacharcolon}{\kern0pt}{\isacharbraceleft}{\kern0pt}banach{\isacharcomma}{\kern0pt}\ second{\isacharunderscore}{\kern0pt}countable{\isacharunderscore}{\kern0pt}topology{\isacharbraceright}{\kern0pt}{\isachardoublequoteclose}\isanewline
\ \ \isakeyword{assumes}\ {\isachardoublequoteopen}integrable\ M\ f{\isachardoublequoteclose}\isanewline
\ \ \isakeyword{obtains}\ s\ \isakeyword{where}\ {\isachardoublequoteopen}{\isasymAnd}i{\isachardot}{\kern0pt}\ simple{\isacharunderscore}{\kern0pt}function\ M\ {\isacharparenleft}{\kern0pt}s\ i{\isacharparenright}{\kern0pt}{\isachardoublequoteclose}\isanewline
\ \ \ \ \ \ \isakeyword{and}\ {\isachardoublequoteopen}{\isasymAnd}i{\isachardot}{\kern0pt}\ emeasure\ M\ {\isacharbraceleft}{\kern0pt}y\ {\isasymin}\ space\ M{\isachardot}{\kern0pt}\ s\ i\ y\ {\isasymnoteq}\ {\isadigit{0}}{\isacharbraceright}{\kern0pt}\ {\isasymnoteq}\ {\isasyminfinity}{\isachardoublequoteclose}\isanewline
\ \ \ \ \ \ \isakeyword{and}\ {\isachardoublequoteopen}{\isasymAnd}x{\isachardot}{\kern0pt}\ x\ {\isasymin}\ space\ M\ {\isasymLongrightarrow}\ {\isacharparenleft}{\kern0pt}{\isasymlambda}i{\isachardot}{\kern0pt}\ s\ i\ x{\isacharparenright}{\kern0pt}\ {\isasymlonglonglongrightarrow}\ f\ x{\isachardoublequoteclose}\isanewline
\ \ \ \ \ \ \isakeyword{and}\ {\isachardoublequoteopen}{\isasymAnd}x\ i{\isachardot}{\kern0pt}\ x\ {\isasymin}\ space\ M\ {\isasymLongrightarrow}\ norm\ {\isacharparenleft}{\kern0pt}s\ i\ x{\isacharparenright}{\kern0pt}\ {\isasymle}\ {\isadigit{2}}\ {\isacharasterisk}{\kern0pt}\ norm\ {\isacharparenleft}{\kern0pt}f\ x{\isacharparenright}{\kern0pt}{\isachardoublequoteclose}\isanewline
%
\isadelimproof
%
\endisadelimproof
%
\isatagproof
\isacommand{proof}\isamarkupfalse%
{\isacharminus}{\kern0pt}\isanewline
\ \ \isacommand{have}\isamarkupfalse%
\ f{\isacharcolon}{\kern0pt}\ {\isachardoublequoteopen}f\ {\isasymin}\ borel{\isacharunderscore}{\kern0pt}measurable\ M{\isachardoublequoteclose}\ {\isachardoublequoteopen}{\isacharparenleft}{\kern0pt}{\isasymintegral}\isactrlsup {\isacharplus}{\kern0pt}x{\isachardot}{\kern0pt}\ norm\ {\isacharparenleft}{\kern0pt}f\ x{\isacharparenright}{\kern0pt}\ {\isasympartial}M{\isacharparenright}{\kern0pt}\ {\isacharless}{\kern0pt}\ {\isasyminfinity}{\isachardoublequoteclose}\ \isacommand{using}\isamarkupfalse%
\ assms\ \isacommand{unfolding}\isamarkupfalse%
\ integrable{\isacharunderscore}{\kern0pt}iff{\isacharunderscore}{\kern0pt}bounded\ \isacommand{by}\isamarkupfalse%
\ auto\isanewline
\ \ \isacommand{obtain}\isamarkupfalse%
\ s\ \isakeyword{where}\ s{\isacharcolon}{\kern0pt}\ {\isachardoublequoteopen}{\isasymAnd}i{\isachardot}{\kern0pt}\ simple{\isacharunderscore}{\kern0pt}function\ M\ {\isacharparenleft}{\kern0pt}s\ i{\isacharparenright}{\kern0pt}{\isachardoublequoteclose}\ {\isachardoublequoteopen}{\isasymAnd}x{\isachardot}{\kern0pt}\ x\ {\isasymin}\ space\ M\ {\isasymLongrightarrow}\ {\isacharparenleft}{\kern0pt}{\isasymlambda}i{\isachardot}{\kern0pt}\ s\ i\ x{\isacharparenright}{\kern0pt}\ {\isasymlonglonglongrightarrow}\ f\ x{\isachardoublequoteclose}\ {\isachardoublequoteopen}{\isasymAnd}i\ x{\isachardot}{\kern0pt}\ x\ {\isasymin}\ space\ M\ {\isasymLongrightarrow}\ norm\ {\isacharparenleft}{\kern0pt}s\ i\ x{\isacharparenright}{\kern0pt}\ {\isasymle}\ {\isadigit{2}}\ {\isacharasterisk}{\kern0pt}\ norm\ {\isacharparenleft}{\kern0pt}f\ x{\isacharparenright}{\kern0pt}{\isachardoublequoteclose}\ \isacommand{using}\isamarkupfalse%
\ borel{\isacharunderscore}{\kern0pt}measurable{\isacharunderscore}{\kern0pt}implies{\isacharunderscore}{\kern0pt}sequence{\isacharunderscore}{\kern0pt}metric{\isacharbrackleft}{\kern0pt}OF\ f{\isacharparenleft}{\kern0pt}{\isadigit{1}}{\isacharparenright}{\kern0pt}{\isacharbrackright}{\kern0pt}\ \isacommand{unfolding}\isamarkupfalse%
\ norm{\isacharunderscore}{\kern0pt}conv{\isacharunderscore}{\kern0pt}dist\ \isacommand{by}\isamarkupfalse%
\ metis\isanewline
\ \ \isacommand{{\isacharbraceleft}{\kern0pt}}\isamarkupfalse%
\isanewline
\ \ \ \ \isacommand{fix}\isamarkupfalse%
\ i\isanewline
\ \ \ \ \isacommand{have}\isamarkupfalse%
\ {\isachardoublequoteopen}{\isacharparenleft}{\kern0pt}{\isasymintegral}\isactrlsup {\isacharplus}{\kern0pt}x{\isachardot}{\kern0pt}\ norm\ {\isacharparenleft}{\kern0pt}s\ i\ x{\isacharparenright}{\kern0pt}\ {\isasympartial}M{\isacharparenright}{\kern0pt}\ {\isasymle}\ {\isacharparenleft}{\kern0pt}{\isasymintegral}\isactrlsup {\isacharplus}{\kern0pt}x{\isachardot}{\kern0pt}\ ennreal\ {\isacharparenleft}{\kern0pt}{\isadigit{2}}\ {\isacharasterisk}{\kern0pt}\ norm\ {\isacharparenleft}{\kern0pt}f\ x{\isacharparenright}{\kern0pt}{\isacharparenright}{\kern0pt}\ {\isasympartial}M{\isacharparenright}{\kern0pt}{\isachardoublequoteclose}\ \isacommand{using}\isamarkupfalse%
\ s\ \isacommand{by}\isamarkupfalse%
\ {\isacharparenleft}{\kern0pt}intro\ nn{\isacharunderscore}{\kern0pt}integral{\isacharunderscore}{\kern0pt}mono{\isacharcomma}{\kern0pt}\ auto{\isacharparenright}{\kern0pt}\isanewline
\ \ \ \ \isacommand{also}\isamarkupfalse%
\ \isacommand{have}\isamarkupfalse%
\ {\isachardoublequoteopen}{\isasymdots}\ {\isacharless}{\kern0pt}\ {\isasyminfinity}{\isachardoublequoteclose}\ \isacommand{using}\isamarkupfalse%
\ f\ \isacommand{by}\isamarkupfalse%
\ {\isacharparenleft}{\kern0pt}simp\ add{\isacharcolon}{\kern0pt}\ nn{\isacharunderscore}{\kern0pt}integral{\isacharunderscore}{\kern0pt}cmult\ ennreal{\isacharunderscore}{\kern0pt}mult{\isacharunderscore}{\kern0pt}less{\isacharunderscore}{\kern0pt}top\ ennreal{\isacharunderscore}{\kern0pt}mult{\isacharparenright}{\kern0pt}\isanewline
\ \ \ \ \isacommand{finally}\isamarkupfalse%
\ \isacommand{have}\isamarkupfalse%
\ sbi{\isacharcolon}{\kern0pt}\ {\isachardoublequoteopen}Bochner{\isacharunderscore}{\kern0pt}Integration{\isachardot}{\kern0pt}simple{\isacharunderscore}{\kern0pt}bochner{\isacharunderscore}{\kern0pt}integrable\ M\ {\isacharparenleft}{\kern0pt}s\ i{\isacharparenright}{\kern0pt}{\isachardoublequoteclose}\ \isacommand{using}\isamarkupfalse%
\ s\ \isacommand{by}\isamarkupfalse%
\ {\isacharparenleft}{\kern0pt}intro\ simple{\isacharunderscore}{\kern0pt}bochner{\isacharunderscore}{\kern0pt}integrableI{\isacharunderscore}{\kern0pt}bounded{\isacharparenright}{\kern0pt}\ auto\isanewline
\ \ \ \ \isacommand{hence}\isamarkupfalse%
\ {\isachardoublequoteopen}emeasure\ M\ {\isacharbraceleft}{\kern0pt}y\ {\isasymin}\ space\ M{\isachardot}{\kern0pt}\ s\ i\ y\ {\isasymnoteq}\ {\isadigit{0}}{\isacharbraceright}{\kern0pt}\ {\isasymnoteq}\ {\isasyminfinity}{\isachardoublequoteclose}\ \isacommand{by}\isamarkupfalse%
\ {\isacharparenleft}{\kern0pt}auto\ intro{\isacharcolon}{\kern0pt}\ integrableI{\isacharunderscore}{\kern0pt}simple{\isacharunderscore}{\kern0pt}bochner{\isacharunderscore}{\kern0pt}integrable\ simple{\isacharunderscore}{\kern0pt}bochner{\isacharunderscore}{\kern0pt}integrable{\isachardot}{\kern0pt}cases{\isacharparenright}{\kern0pt}\isanewline
\ \ \isacommand{{\isacharbraceright}{\kern0pt}}\isamarkupfalse%
\isanewline
\ \ \isacommand{thus}\isamarkupfalse%
\ {\isacharquery}{\kern0pt}thesis\ \isacommand{using}\isamarkupfalse%
\ that\ s\ \isacommand{by}\isamarkupfalse%
\ blast\isanewline
\isacommand{qed}\isamarkupfalse%
%
\endisatagproof
{\isafoldproof}%
%
\isadelimproof
\isanewline
%
\endisadelimproof
\isanewline
\isacommand{lemma}\isamarkupfalse%
\ simple{\isacharunderscore}{\kern0pt}function{\isacharunderscore}{\kern0pt}indicator{\isacharunderscore}{\kern0pt}representation{\isacharcolon}{\kern0pt}\isanewline
\ \ \isakeyword{fixes}\ f\ {\isacharcolon}{\kern0pt}{\isacharcolon}{\kern0pt}{\isachardoublequoteopen}{\isacharprime}{\kern0pt}a\ {\isasymRightarrow}\ {\isacharprime}{\kern0pt}b\ {\isacharcolon}{\kern0pt}{\isacharcolon}{\kern0pt}\ {\isacharbraceleft}{\kern0pt}second{\isacharunderscore}{\kern0pt}countable{\isacharunderscore}{\kern0pt}topology{\isacharcomma}{\kern0pt}\ banach{\isacharbraceright}{\kern0pt}{\isachardoublequoteclose}\isanewline
\ \ \isakeyword{assumes}\ f{\isacharcolon}{\kern0pt}\ {\isachardoublequoteopen}simple{\isacharunderscore}{\kern0pt}function\ M\ f{\isachardoublequoteclose}\ \isakeyword{and}\ x{\isacharcolon}{\kern0pt}\ {\isachardoublequoteopen}x\ {\isasymin}\ space\ M{\isachardoublequoteclose}\isanewline
\ \ \isakeyword{shows}\ {\isachardoublequoteopen}f\ x\ {\isacharequal}{\kern0pt}\ {\isacharparenleft}{\kern0pt}{\isasymSum}y\ {\isasymin}\ f\ {\isacharbackquote}{\kern0pt}\ space\ M{\isachardot}{\kern0pt}\ indicator\ {\isacharparenleft}{\kern0pt}f\ {\isacharminus}{\kern0pt}{\isacharbackquote}{\kern0pt}\ {\isacharbraceleft}{\kern0pt}y{\isacharbraceright}{\kern0pt}\ {\isasyminter}\ space\ M{\isacharparenright}{\kern0pt}\ x\ {\isacharasterisk}{\kern0pt}\isactrlsub R\ y{\isacharparenright}{\kern0pt}{\isachardoublequoteclose}\isanewline
\ \ {\isacharparenleft}{\kern0pt}\isakeyword{is}\ {\isachardoublequoteopen}{\isacharquery}{\kern0pt}l\ {\isacharequal}{\kern0pt}\ {\isacharquery}{\kern0pt}r{\isachardoublequoteclose}{\isacharparenright}{\kern0pt}\isanewline
%
\isadelimproof
%
\endisadelimproof
%
\isatagproof
\isacommand{proof}\isamarkupfalse%
\ {\isacharminus}{\kern0pt}\isanewline
\ \ \isacommand{have}\isamarkupfalse%
\ {\isachardoublequoteopen}{\isacharquery}{\kern0pt}r\ {\isacharequal}{\kern0pt}\ {\isacharparenleft}{\kern0pt}{\isasymSum}y\ {\isasymin}\ f\ {\isacharbackquote}{\kern0pt}\ space\ M{\isachardot}{\kern0pt}\isanewline
\ \ \ \ {\isacharparenleft}{\kern0pt}if\ y\ {\isacharequal}{\kern0pt}\ f\ x\ then\ indicator\ {\isacharparenleft}{\kern0pt}f\ {\isacharminus}{\kern0pt}{\isacharbackquote}{\kern0pt}\ {\isacharbraceleft}{\kern0pt}y{\isacharbraceright}{\kern0pt}\ {\isasyminter}\ space\ M{\isacharparenright}{\kern0pt}\ x\ {\isacharasterisk}{\kern0pt}\isactrlsub R\ y\ else\ {\isadigit{0}}{\isacharparenright}{\kern0pt}{\isacharparenright}{\kern0pt}{\isachardoublequoteclose}\ \isacommand{by}\isamarkupfalse%
\ {\isacharparenleft}{\kern0pt}auto\ intro{\isacharbang}{\kern0pt}{\isacharcolon}{\kern0pt}\ sum{\isachardot}{\kern0pt}cong{\isacharparenright}{\kern0pt}\isanewline
\ \ \isacommand{also}\isamarkupfalse%
\ \isacommand{have}\isamarkupfalse%
\ {\isachardoublequoteopen}{\isachardot}{\kern0pt}{\isachardot}{\kern0pt}{\isachardot}{\kern0pt}\ {\isacharequal}{\kern0pt}\ \ indicator\ {\isacharparenleft}{\kern0pt}f\ {\isacharminus}{\kern0pt}{\isacharbackquote}{\kern0pt}\ {\isacharbraceleft}{\kern0pt}f\ x{\isacharbraceright}{\kern0pt}\ {\isasyminter}\ space\ M{\isacharparenright}{\kern0pt}\ x\ {\isacharasterisk}{\kern0pt}\isactrlsub R\ f\ x{\isachardoublequoteclose}\ \isacommand{using}\isamarkupfalse%
\ assms\ \isacommand{by}\isamarkupfalse%
\ {\isacharparenleft}{\kern0pt}auto\ dest{\isacharcolon}{\kern0pt}\ simple{\isacharunderscore}{\kern0pt}functionD{\isacharparenright}{\kern0pt}\isanewline
\ \ \isacommand{also}\isamarkupfalse%
\ \isacommand{have}\isamarkupfalse%
\ {\isachardoublequoteopen}{\isachardot}{\kern0pt}{\isachardot}{\kern0pt}{\isachardot}{\kern0pt}\ {\isacharequal}{\kern0pt}\ f\ x{\isachardoublequoteclose}\ \isacommand{using}\isamarkupfalse%
\ x\ \isacommand{by}\isamarkupfalse%
\ {\isacharparenleft}{\kern0pt}auto\ simp{\isacharcolon}{\kern0pt}\ indicator{\isacharunderscore}{\kern0pt}def{\isacharparenright}{\kern0pt}\isanewline
\ \ \isacommand{finally}\isamarkupfalse%
\ \isacommand{show}\isamarkupfalse%
\ {\isacharquery}{\kern0pt}thesis\ \isacommand{by}\isamarkupfalse%
\ auto\isanewline
\isacommand{qed}\isamarkupfalse%
%
\endisatagproof
{\isafoldproof}%
%
\isadelimproof
\isanewline
%
\endisadelimproof
\isanewline
\isacommand{lemma}\isamarkupfalse%
\ simple{\isacharunderscore}{\kern0pt}function{\isacharunderscore}{\kern0pt}indicator{\isacharunderscore}{\kern0pt}representation{\isacharunderscore}{\kern0pt}AE{\isacharcolon}{\kern0pt}\isanewline
\ \ \isakeyword{fixes}\ f\ {\isacharcolon}{\kern0pt}{\isacharcolon}{\kern0pt}{\isachardoublequoteopen}{\isacharprime}{\kern0pt}a\ {\isasymRightarrow}\ {\isacharprime}{\kern0pt}b\ {\isacharcolon}{\kern0pt}{\isacharcolon}{\kern0pt}\ {\isacharbraceleft}{\kern0pt}second{\isacharunderscore}{\kern0pt}countable{\isacharunderscore}{\kern0pt}topology{\isacharcomma}{\kern0pt}\ banach{\isacharbraceright}{\kern0pt}{\isachardoublequoteclose}\isanewline
\ \ \isakeyword{assumes}\ f{\isacharcolon}{\kern0pt}\ {\isachardoublequoteopen}simple{\isacharunderscore}{\kern0pt}function\ M\ f{\isachardoublequoteclose}\isanewline
\ \ \isakeyword{shows}\ {\isachardoublequoteopen}AE\ x\ in\ M{\isachardot}{\kern0pt}\ f\ x\ {\isacharequal}{\kern0pt}\ {\isacharparenleft}{\kern0pt}{\isasymSum}y\ {\isasymin}\ f\ {\isacharbackquote}{\kern0pt}\ space\ M{\isachardot}{\kern0pt}\ indicator\ {\isacharparenleft}{\kern0pt}f\ {\isacharminus}{\kern0pt}{\isacharbackquote}{\kern0pt}\ {\isacharbraceleft}{\kern0pt}y{\isacharbraceright}{\kern0pt}\ {\isasyminter}\ space\ M{\isacharparenright}{\kern0pt}\ x\ {\isacharasterisk}{\kern0pt}\isactrlsub R\ y{\isacharparenright}{\kern0pt}{\isachardoublequoteclose}\ \ \isanewline
%
\isadelimproof
\ \ %
\endisadelimproof
%
\isatagproof
\isacommand{by}\isamarkupfalse%
\ {\isacharparenleft}{\kern0pt}metis\ {\isacharparenleft}{\kern0pt}mono{\isacharunderscore}{\kern0pt}tags{\isacharcomma}{\kern0pt}\ lifting{\isacharparenright}{\kern0pt}\ AE{\isacharunderscore}{\kern0pt}I{\isadigit{2}}\ simple{\isacharunderscore}{\kern0pt}function{\isacharunderscore}{\kern0pt}indicator{\isacharunderscore}{\kern0pt}representation\ f{\isacharparenright}{\kern0pt}%
\endisatagproof
{\isafoldproof}%
%
\isadelimproof
\isanewline
%
\endisadelimproof
\isanewline
\isacommand{lemmas}\isamarkupfalse%
\ simple{\isacharunderscore}{\kern0pt}function{\isacharunderscore}{\kern0pt}scaleR{\isacharbrackleft}{\kern0pt}intro{\isacharbrackright}{\kern0pt}\ {\isacharequal}{\kern0pt}\ simple{\isacharunderscore}{\kern0pt}function{\isacharunderscore}{\kern0pt}compose{\isadigit{2}}{\isacharbrackleft}{\kern0pt}\isakeyword{where}\ h{\isacharequal}{\kern0pt}{\isachardoublequoteopen}{\isacharparenleft}{\kern0pt}{\isacharasterisk}{\kern0pt}\isactrlsub R{\isacharparenright}{\kern0pt}{\isachardoublequoteclose}{\isacharbrackright}{\kern0pt}\isanewline
\isacommand{lemmas}\isamarkupfalse%
\ integrable{\isacharunderscore}{\kern0pt}simple{\isacharunderscore}{\kern0pt}function\ {\isacharequal}{\kern0pt}\ simple{\isacharunderscore}{\kern0pt}bochner{\isacharunderscore}{\kern0pt}integrable{\isachardot}{\kern0pt}intros{\isacharbrackleft}{\kern0pt}THEN\ has{\isacharunderscore}{\kern0pt}bochner{\isacharunderscore}{\kern0pt}integral{\isacharunderscore}{\kern0pt}simple{\isacharunderscore}{\kern0pt}bochner{\isacharunderscore}{\kern0pt}integrable{\isacharcomma}{\kern0pt}\ THEN\ integrable{\isachardot}{\kern0pt}intros{\isacharbrackright}{\kern0pt}\ \isanewline
\isanewline
\isanewline
%
\isadelimimportant
\isanewline
%
\endisadelimimportant
%
\isatagimportant
\isacommand{lemma}\isamarkupfalse%
\ simple{\isacharunderscore}{\kern0pt}integrable{\isacharunderscore}{\kern0pt}function{\isacharunderscore}{\kern0pt}induct{\isacharbrackleft}{\kern0pt}consumes\ {\isadigit{2}}{\isacharcomma}{\kern0pt}\ case{\isacharunderscore}{\kern0pt}names\ cong\ indicator\ add{\isacharcomma}{\kern0pt}\ induct\ set{\isacharcolon}{\kern0pt}\ simple{\isacharunderscore}{\kern0pt}function{\isacharbrackright}{\kern0pt}{\isacharcolon}{\kern0pt}\isanewline
\ \ \isakeyword{fixes}\ f\ {\isacharcolon}{\kern0pt}{\isacharcolon}{\kern0pt}\ {\isachardoublequoteopen}{\isacharprime}{\kern0pt}a\ {\isasymRightarrow}\ {\isacharprime}{\kern0pt}b\ {\isacharcolon}{\kern0pt}{\isacharcolon}{\kern0pt}\ {\isacharbraceleft}{\kern0pt}second{\isacharunderscore}{\kern0pt}countable{\isacharunderscore}{\kern0pt}topology{\isacharcomma}{\kern0pt}\ banach{\isacharbraceright}{\kern0pt}{\isachardoublequoteclose}\isanewline
\ \ \isakeyword{assumes}\ f{\isacharcolon}{\kern0pt}\ {\isachardoublequoteopen}simple{\isacharunderscore}{\kern0pt}function\ M\ f{\isachardoublequoteclose}\ {\isachardoublequoteopen}emeasure\ M\ {\isacharbraceleft}{\kern0pt}y\ {\isasymin}\ space\ M{\isachardot}{\kern0pt}\ f\ y\ {\isasymnoteq}\ {\isadigit{0}}{\isacharbraceright}{\kern0pt}\ {\isasymnoteq}\ {\isasyminfinity}{\isachardoublequoteclose}\isanewline
\ \ \isakeyword{assumes}\ cong{\isacharcolon}{\kern0pt}\ {\isachardoublequoteopen}{\isasymAnd}f\ g{\isachardot}{\kern0pt}\ simple{\isacharunderscore}{\kern0pt}function\ M\ f\ {\isasymLongrightarrow}\ emeasure\ M\ {\isacharbraceleft}{\kern0pt}y\ {\isasymin}\ space\ M{\isachardot}{\kern0pt}\ f\ y\ {\isasymnoteq}\ {\isadigit{0}}{\isacharbraceright}{\kern0pt}\ {\isasymnoteq}\ {\isasyminfinity}\ \isanewline
\ \ \ \ \ \ \ \ \ \ \ \ \ \ \ \ \ \ \ \ {\isasymLongrightarrow}\ simple{\isacharunderscore}{\kern0pt}function\ M\ g\ {\isasymLongrightarrow}\ emeasure\ M\ {\isacharbraceleft}{\kern0pt}y\ {\isasymin}\ space\ M{\isachardot}{\kern0pt}\ g\ y\ {\isasymnoteq}\ {\isadigit{0}}{\isacharbraceright}{\kern0pt}\ {\isasymnoteq}\ {\isasyminfinity}\ \isanewline
\ \ \ \ \ \ \ \ \ \ \ \ \ \ \ \ \ \ \ \ {\isasymLongrightarrow}\ {\isacharparenleft}{\kern0pt}{\isasymAnd}x{\isachardot}{\kern0pt}\ x\ {\isasymin}\ space\ M\ {\isasymLongrightarrow}\ f\ x\ {\isacharequal}{\kern0pt}\ g\ x{\isacharparenright}{\kern0pt}\ {\isasymLongrightarrow}\ P\ f\ {\isasymLongrightarrow}\ P\ g{\isachardoublequoteclose}\isanewline
\ \ \isakeyword{assumes}\ indicator{\isacharcolon}{\kern0pt}\ {\isachardoublequoteopen}{\isasymAnd}A\ y{\isachardot}{\kern0pt}\ A\ {\isasymin}\ sets\ M\ {\isasymLongrightarrow}\ emeasure\ M\ A\ {\isacharless}{\kern0pt}\ {\isasyminfinity}\ {\isasymLongrightarrow}\ P\ {\isacharparenleft}{\kern0pt}{\isasymlambda}x{\isachardot}{\kern0pt}\ indicator\ A\ x\ {\isacharasterisk}{\kern0pt}\isactrlsub R\ y{\isacharparenright}{\kern0pt}{\isachardoublequoteclose}\isanewline
\ \ \isakeyword{assumes}\ add{\isacharcolon}{\kern0pt}\ {\isachardoublequoteopen}{\isasymAnd}f\ g{\isachardot}{\kern0pt}\ simple{\isacharunderscore}{\kern0pt}function\ M\ f\ {\isasymLongrightarrow}\ emeasure\ M\ {\isacharbraceleft}{\kern0pt}y\ {\isasymin}\ space\ M{\isachardot}{\kern0pt}\ f\ y\ {\isasymnoteq}\ {\isadigit{0}}{\isacharbraceright}{\kern0pt}\ {\isasymnoteq}\ {\isasyminfinity}\ {\isasymLongrightarrow}\ \isanewline
\ \ \ \ \ \ \ \ \ \ \ \ \ \ \ \ \ \ \ \ \ \ simple{\isacharunderscore}{\kern0pt}function\ M\ g\ {\isasymLongrightarrow}\ emeasure\ M\ {\isacharbraceleft}{\kern0pt}y\ {\isasymin}\ space\ M{\isachardot}{\kern0pt}\ g\ y\ {\isasymnoteq}\ {\isadigit{0}}{\isacharbraceright}{\kern0pt}\ {\isasymnoteq}\ {\isasyminfinity}\ {\isasymLongrightarrow}\ \isanewline
\ \ \ \ \ \ \ \ \ \ \ \ \ \ \ \ \ \ \ \ \ \ {\isacharparenleft}{\kern0pt}{\isasymAnd}z{\isachardot}{\kern0pt}\ z\ {\isasymin}\ space\ M\ {\isasymLongrightarrow}\ norm\ {\isacharparenleft}{\kern0pt}f\ z\ {\isacharplus}{\kern0pt}\ g\ z{\isacharparenright}{\kern0pt}\ {\isacharequal}{\kern0pt}\ norm\ {\isacharparenleft}{\kern0pt}f\ z{\isacharparenright}{\kern0pt}\ {\isacharplus}{\kern0pt}\ norm\ {\isacharparenleft}{\kern0pt}g\ z{\isacharparenright}{\kern0pt}{\isacharparenright}{\kern0pt}\ {\isasymLongrightarrow}\isanewline
\ \ \ \ \ \ \ \ \ \ \ \ \ \ \ \ \ \ \ \ \ \ P\ f\ {\isasymLongrightarrow}\ P\ g\ {\isasymLongrightarrow}\ P\ {\isacharparenleft}{\kern0pt}{\isasymlambda}x{\isachardot}{\kern0pt}\ f\ x\ {\isacharplus}{\kern0pt}\ g\ x{\isacharparenright}{\kern0pt}{\isachardoublequoteclose}\isanewline
\ \ \isakeyword{shows}\ {\isachardoublequoteopen}P\ f{\isachardoublequoteclose}%
\endisatagimportant
{\isafoldimportant}%
%
\isadelimimportant
\isanewline
%
\endisadelimimportant
%
\isadelimproof
%
\endisadelimproof
%
\isatagproof
\isacommand{proof}\isamarkupfalse%
{\isacharminus}{\kern0pt}\isanewline
\ \ \isacommand{let}\isamarkupfalse%
\ {\isacharquery}{\kern0pt}f\ {\isacharequal}{\kern0pt}\ {\isachardoublequoteopen}{\isasymlambda}x{\isachardot}{\kern0pt}\ {\isacharparenleft}{\kern0pt}{\isasymSum}y{\isasymin}f\ {\isacharbackquote}{\kern0pt}\ space\ M{\isachardot}{\kern0pt}\ indicat{\isacharunderscore}{\kern0pt}real\ {\isacharparenleft}{\kern0pt}f\ {\isacharminus}{\kern0pt}{\isacharbackquote}{\kern0pt}\ {\isacharbraceleft}{\kern0pt}y{\isacharbraceright}{\kern0pt}\ {\isasyminter}\ space\ M{\isacharparenright}{\kern0pt}\ x\ {\isacharasterisk}{\kern0pt}\isactrlsub R\ y{\isacharparenright}{\kern0pt}{\isachardoublequoteclose}\isanewline
\ \ \isacommand{have}\isamarkupfalse%
\ f{\isacharunderscore}{\kern0pt}ae{\isacharunderscore}{\kern0pt}eq{\isacharcolon}{\kern0pt}\ {\isachardoublequoteopen}f\ x\ {\isacharequal}{\kern0pt}\ {\isacharquery}{\kern0pt}f\ x{\isachardoublequoteclose}\ \isakeyword{if}\ {\isachardoublequoteopen}x\ {\isasymin}\ space\ M{\isachardoublequoteclose}\ \isakeyword{for}\ x\ \isacommand{using}\isamarkupfalse%
\ simple{\isacharunderscore}{\kern0pt}function{\isacharunderscore}{\kern0pt}indicator{\isacharunderscore}{\kern0pt}representation{\isacharbrackleft}{\kern0pt}OF\ f{\isacharparenleft}{\kern0pt}{\isadigit{1}}{\isacharparenright}{\kern0pt}\ that{\isacharbrackright}{\kern0pt}\ \isacommand{{\isachardot}{\kern0pt}}\isamarkupfalse%
\isanewline
\ \ \isacommand{moreover}\isamarkupfalse%
\ \isacommand{have}\isamarkupfalse%
\ {\isachardoublequoteopen}emeasure\ M\ {\isacharbraceleft}{\kern0pt}y\ {\isasymin}\ space\ M{\isachardot}{\kern0pt}\ {\isacharquery}{\kern0pt}f\ y\ {\isasymnoteq}\ {\isadigit{0}}{\isacharbraceright}{\kern0pt}\ {\isasymnoteq}\ {\isasyminfinity}{\isachardoublequoteclose}\ \isacommand{by}\isamarkupfalse%
\ {\isacharparenleft}{\kern0pt}metis\ {\isacharparenleft}{\kern0pt}no{\isacharunderscore}{\kern0pt}types{\isacharcomma}{\kern0pt}\ lifting{\isacharparenright}{\kern0pt}\ Collect{\isacharunderscore}{\kern0pt}cong\ calculation\ f{\isacharparenleft}{\kern0pt}{\isadigit{2}}{\isacharparenright}{\kern0pt}{\isacharparenright}{\kern0pt}\isanewline
\ \ \isacommand{moreover}\isamarkupfalse%
\ \isacommand{have}\isamarkupfalse%
\ {\isachardoublequoteopen}P\ {\isacharparenleft}{\kern0pt}{\isasymlambda}x{\isachardot}{\kern0pt}\ {\isasymSum}y{\isasymin}S{\isachardot}{\kern0pt}\ indicat{\isacharunderscore}{\kern0pt}real\ {\isacharparenleft}{\kern0pt}f\ {\isacharminus}{\kern0pt}{\isacharbackquote}{\kern0pt}\ {\isacharbraceleft}{\kern0pt}y{\isacharbraceright}{\kern0pt}\ {\isasyminter}\ space\ M{\isacharparenright}{\kern0pt}\ x\ {\isacharasterisk}{\kern0pt}\isactrlsub R\ y{\isacharparenright}{\kern0pt}{\isachardoublequoteclose}\isanewline
\ \ \ \ \ \ \ \ \ \ \ \ \ \ \ \ {\isachardoublequoteopen}simple{\isacharunderscore}{\kern0pt}function\ M\ {\isacharparenleft}{\kern0pt}{\isasymlambda}x{\isachardot}{\kern0pt}\ {\isasymSum}y{\isasymin}S{\isachardot}{\kern0pt}\ indicat{\isacharunderscore}{\kern0pt}real\ {\isacharparenleft}{\kern0pt}f\ {\isacharminus}{\kern0pt}{\isacharbackquote}{\kern0pt}\ {\isacharbraceleft}{\kern0pt}y{\isacharbraceright}{\kern0pt}\ {\isasyminter}\ space\ M{\isacharparenright}{\kern0pt}\ x\ {\isacharasterisk}{\kern0pt}\isactrlsub R\ y{\isacharparenright}{\kern0pt}{\isachardoublequoteclose}\isanewline
\ \ \ \ \ \ \ \ \ \ \ \ \ \ \ \ {\isachardoublequoteopen}emeasure\ M\ {\isacharbraceleft}{\kern0pt}y\ {\isasymin}\ space\ M{\isachardot}{\kern0pt}\ {\isacharparenleft}{\kern0pt}{\isasymSum}x{\isasymin}S{\isachardot}{\kern0pt}\ indicat{\isacharunderscore}{\kern0pt}real\ {\isacharparenleft}{\kern0pt}f\ {\isacharminus}{\kern0pt}{\isacharbackquote}{\kern0pt}\ {\isacharbraceleft}{\kern0pt}x{\isacharbraceright}{\kern0pt}\ {\isasyminter}\ space\ M{\isacharparenright}{\kern0pt}\ y\ {\isacharasterisk}{\kern0pt}\isactrlsub R\ x{\isacharparenright}{\kern0pt}\ {\isasymnoteq}\ {\isadigit{0}}{\isacharbraceright}{\kern0pt}\ {\isasymnoteq}\ {\isasyminfinity}{\isachardoublequoteclose}\isanewline
\ \ \ \ \ \ \ \ \ \ \ \ \ \ \ \ \isakeyword{if}\ {\isachardoublequoteopen}S\ {\isasymsubseteq}\ f\ {\isacharbackquote}{\kern0pt}\ space\ M{\isachardoublequoteclose}\ \isakeyword{for}\ S\ \isacommand{using}\isamarkupfalse%
\ simple{\isacharunderscore}{\kern0pt}functionD{\isacharparenleft}{\kern0pt}{\isadigit{1}}{\isacharparenright}{\kern0pt}{\isacharbrackleft}{\kern0pt}OF\ assms{\isacharparenleft}{\kern0pt}{\isadigit{1}}{\isacharparenright}{\kern0pt}{\isacharcomma}{\kern0pt}\ THEN\ rev{\isacharunderscore}{\kern0pt}finite{\isacharunderscore}{\kern0pt}subset{\isacharcomma}{\kern0pt}\ OF\ that{\isacharbrackright}{\kern0pt}\ that\ \isanewline
\ \ \isacommand{proof}\isamarkupfalse%
\ {\isacharparenleft}{\kern0pt}induction\ rule{\isacharcolon}{\kern0pt}\ finite{\isacharunderscore}{\kern0pt}induct{\isacharparenright}{\kern0pt}\isanewline
\ \ \ \ \isacommand{case}\isamarkupfalse%
\ empty\isanewline
\ \ \ \ \isacommand{{\isacharbraceleft}{\kern0pt}}\isamarkupfalse%
\isanewline
\ \ \ \ \ \ \isacommand{case}\isamarkupfalse%
\ {\isadigit{1}}\isanewline
\ \ \ \ \ \ \isacommand{then}\isamarkupfalse%
\ \isacommand{show}\isamarkupfalse%
\ {\isacharquery}{\kern0pt}case\ \isacommand{using}\isamarkupfalse%
\ indicator{\isacharbrackleft}{\kern0pt}of\ {\isachardoublequoteopen}{\isacharbraceleft}{\kern0pt}{\isacharbraceright}{\kern0pt}{\isachardoublequoteclose}{\isacharbrackright}{\kern0pt}\ \isacommand{by}\isamarkupfalse%
\ force\isanewline
\ \ \ \ \isacommand{next}\isamarkupfalse%
\isanewline
\ \ \ \ \ \ \isacommand{case}\isamarkupfalse%
\ {\isadigit{2}}\isanewline
\ \ \ \ \ \ \isacommand{then}\isamarkupfalse%
\ \isacommand{show}\isamarkupfalse%
\ {\isacharquery}{\kern0pt}case\ \isacommand{by}\isamarkupfalse%
\ force\ \isanewline
\ \ \ \ \isacommand{next}\isamarkupfalse%
\isanewline
\ \ \ \ \ \ \isacommand{case}\isamarkupfalse%
\ {\isadigit{3}}\isanewline
\ \ \ \ \ \ \isacommand{then}\isamarkupfalse%
\ \isacommand{show}\isamarkupfalse%
\ {\isacharquery}{\kern0pt}case\ \isacommand{by}\isamarkupfalse%
\ force\ \isanewline
\ \ \ \ \isacommand{{\isacharbraceright}{\kern0pt}}\isamarkupfalse%
\isanewline
\ \ \isacommand{next}\isamarkupfalse%
\isanewline
\ \ \ \ \isacommand{case}\isamarkupfalse%
\ {\isacharparenleft}{\kern0pt}insert\ x\ F{\isacharparenright}{\kern0pt}\isanewline
\ \ \ \ \isacommand{have}\isamarkupfalse%
\ {\isachardoublequoteopen}{\isacharparenleft}{\kern0pt}f\ {\isacharminus}{\kern0pt}{\isacharbackquote}{\kern0pt}\ {\isacharbraceleft}{\kern0pt}x{\isacharbraceright}{\kern0pt}\ {\isasyminter}\ space\ M{\isacharparenright}{\kern0pt}\ {\isasymsubseteq}\ {\isacharbraceleft}{\kern0pt}y\ {\isasymin}\ space\ M{\isachardot}{\kern0pt}\ f\ y\ {\isasymnoteq}\ {\isadigit{0}}{\isacharbraceright}{\kern0pt}{\isachardoublequoteclose}\ \isakeyword{if}\ {\isachardoublequoteopen}x\ {\isasymnoteq}\ {\isadigit{0}}{\isachardoublequoteclose}\ \isacommand{using}\isamarkupfalse%
\ that\ \isacommand{by}\isamarkupfalse%
\ blast\isanewline
\ \ \ \ \isacommand{moreover}\isamarkupfalse%
\ \isacommand{have}\isamarkupfalse%
\ {\isachardoublequoteopen}{\isacharbraceleft}{\kern0pt}y\ {\isasymin}\ space\ M{\isachardot}{\kern0pt}\ f\ y\ {\isasymnoteq}\ {\isadigit{0}}{\isacharbraceright}{\kern0pt}\ {\isacharequal}{\kern0pt}\ space\ M\ {\isacharminus}{\kern0pt}\ {\isacharparenleft}{\kern0pt}f\ {\isacharminus}{\kern0pt}{\isacharbackquote}{\kern0pt}\ {\isacharbraceleft}{\kern0pt}{\isadigit{0}}{\isacharbraceright}{\kern0pt}\ {\isasyminter}\ space\ M{\isacharparenright}{\kern0pt}{\isachardoublequoteclose}\ \isacommand{by}\isamarkupfalse%
\ blast\isanewline
\ \ \ \ \isacommand{moreover}\isamarkupfalse%
\ \isacommand{have}\isamarkupfalse%
\ {\isachardoublequoteopen}space\ M\ {\isacharminus}{\kern0pt}\ {\isacharparenleft}{\kern0pt}f\ {\isacharminus}{\kern0pt}{\isacharbackquote}{\kern0pt}\ {\isacharbraceleft}{\kern0pt}{\isadigit{0}}{\isacharbraceright}{\kern0pt}\ {\isasyminter}\ space\ M{\isacharparenright}{\kern0pt}\ {\isasymin}\ sets\ M{\isachardoublequoteclose}\ \isacommand{using}\isamarkupfalse%
\ simple{\isacharunderscore}{\kern0pt}functionD{\isacharparenleft}{\kern0pt}{\isadigit{2}}{\isacharparenright}{\kern0pt}{\isacharbrackleft}{\kern0pt}OF\ f{\isacharparenleft}{\kern0pt}{\isadigit{1}}{\isacharparenright}{\kern0pt}{\isacharbrackright}{\kern0pt}\ \isacommand{by}\isamarkupfalse%
\ blast\isanewline
\ \ \ \ \isacommand{ultimately}\isamarkupfalse%
\ \isacommand{have}\isamarkupfalse%
\ fin{\isacharunderscore}{\kern0pt}{\isadigit{0}}{\isacharcolon}{\kern0pt}\ {\isachardoublequoteopen}emeasure\ M\ {\isacharparenleft}{\kern0pt}f\ {\isacharminus}{\kern0pt}{\isacharbackquote}{\kern0pt}\ {\isacharbraceleft}{\kern0pt}x{\isacharbraceright}{\kern0pt}\ {\isasyminter}\ space\ M{\isacharparenright}{\kern0pt}\ {\isacharless}{\kern0pt}\ {\isasyminfinity}{\isachardoublequoteclose}\ \isakeyword{if}\ {\isachardoublequoteopen}x\ {\isasymnoteq}\ {\isadigit{0}}{\isachardoublequoteclose}\ \isacommand{using}\isamarkupfalse%
\ that\ \isacommand{by}\isamarkupfalse%
\ {\isacharparenleft}{\kern0pt}metis\ emeasure{\isacharunderscore}{\kern0pt}mono\ f{\isacharparenleft}{\kern0pt}{\isadigit{2}}{\isacharparenright}{\kern0pt}\ infinity{\isacharunderscore}{\kern0pt}ennreal{\isacharunderscore}{\kern0pt}def\ top{\isachardot}{\kern0pt}not{\isacharunderscore}{\kern0pt}eq{\isacharunderscore}{\kern0pt}extremum\ top{\isacharunderscore}{\kern0pt}unique{\isacharparenright}{\kern0pt}\isanewline
\ \ \ \ \isacommand{hence}\isamarkupfalse%
\ fin{\isacharunderscore}{\kern0pt}{\isadigit{1}}{\isacharcolon}{\kern0pt}\ {\isachardoublequoteopen}emeasure\ M\ {\isacharbraceleft}{\kern0pt}y\ {\isasymin}\ space\ M{\isachardot}{\kern0pt}\ indicat{\isacharunderscore}{\kern0pt}real\ {\isacharparenleft}{\kern0pt}f\ {\isacharminus}{\kern0pt}{\isacharbackquote}{\kern0pt}\ {\isacharbraceleft}{\kern0pt}x{\isacharbraceright}{\kern0pt}\ {\isasyminter}\ space\ M{\isacharparenright}{\kern0pt}\ y\ {\isacharasterisk}{\kern0pt}\isactrlsub R\ x\ {\isasymnoteq}\ {\isadigit{0}}{\isacharbraceright}{\kern0pt}\ {\isasymnoteq}\ {\isasyminfinity}{\isachardoublequoteclose}\ \isakeyword{if}\ {\isachardoublequoteopen}x\ {\isasymnoteq}\ {\isadigit{0}}{\isachardoublequoteclose}\ \isacommand{by}\isamarkupfalse%
\ {\isacharparenleft}{\kern0pt}metis\ {\isacharparenleft}{\kern0pt}mono{\isacharunderscore}{\kern0pt}tags{\isacharcomma}{\kern0pt}\ lifting{\isacharparenright}{\kern0pt}\ emeasure{\isacharunderscore}{\kern0pt}mono\ f{\isacharparenleft}{\kern0pt}{\isadigit{1}}{\isacharparenright}{\kern0pt}\ indicator{\isacharunderscore}{\kern0pt}simps{\isacharparenleft}{\kern0pt}{\isadigit{2}}{\isacharparenright}{\kern0pt}\ linorder{\isacharunderscore}{\kern0pt}not{\isacharunderscore}{\kern0pt}less\ mem{\isacharunderscore}{\kern0pt}Collect{\isacharunderscore}{\kern0pt}eq\ scaleR{\isacharunderscore}{\kern0pt}eq{\isacharunderscore}{\kern0pt}{\isadigit{0}}{\isacharunderscore}{\kern0pt}iff\ simple{\isacharunderscore}{\kern0pt}functionD{\isacharparenleft}{\kern0pt}{\isadigit{2}}{\isacharparenright}{\kern0pt}\ subsetI\ that{\isacharparenright}{\kern0pt}\isanewline
\isanewline
\ \ \ \ \isacommand{have}\isamarkupfalse%
\ {\isacharasterisk}{\kern0pt}{\isacharcolon}{\kern0pt}\ {\isachardoublequoteopen}{\isacharparenleft}{\kern0pt}{\isasymSum}y{\isasymin}insert\ x\ F{\isachardot}{\kern0pt}\ indicat{\isacharunderscore}{\kern0pt}real\ {\isacharparenleft}{\kern0pt}f\ {\isacharminus}{\kern0pt}{\isacharbackquote}{\kern0pt}\ {\isacharbraceleft}{\kern0pt}y{\isacharbraceright}{\kern0pt}\ {\isasyminter}\ space\ M{\isacharparenright}{\kern0pt}\ xa\ {\isacharasterisk}{\kern0pt}\isactrlsub R\ y{\isacharparenright}{\kern0pt}\ {\isacharequal}{\kern0pt}\ {\isacharparenleft}{\kern0pt}{\isasymSum}y{\isasymin}F{\isachardot}{\kern0pt}\ indicat{\isacharunderscore}{\kern0pt}real\ {\isacharparenleft}{\kern0pt}f\ {\isacharminus}{\kern0pt}{\isacharbackquote}{\kern0pt}\ {\isacharbraceleft}{\kern0pt}y{\isacharbraceright}{\kern0pt}\ {\isasyminter}\ space\ M{\isacharparenright}{\kern0pt}\ xa\ {\isacharasterisk}{\kern0pt}\isactrlsub R\ y{\isacharparenright}{\kern0pt}\ {\isacharplus}{\kern0pt}\ indicat{\isacharunderscore}{\kern0pt}real\ {\isacharparenleft}{\kern0pt}f\ {\isacharminus}{\kern0pt}{\isacharbackquote}{\kern0pt}\ {\isacharbraceleft}{\kern0pt}x{\isacharbraceright}{\kern0pt}\ {\isasyminter}\ space\ M{\isacharparenright}{\kern0pt}\ xa\ {\isacharasterisk}{\kern0pt}\isactrlsub R\ x{\isachardoublequoteclose}\ \isakeyword{for}\ xa\ \isacommand{by}\isamarkupfalse%
\ {\isacharparenleft}{\kern0pt}metis\ {\isacharparenleft}{\kern0pt}no{\isacharunderscore}{\kern0pt}types{\isacharcomma}{\kern0pt}\ lifting{\isacharparenright}{\kern0pt}\ Diff{\isacharunderscore}{\kern0pt}empty\ Diff{\isacharunderscore}{\kern0pt}insert{\isadigit{0}}\ add{\isachardot}{\kern0pt}commute\ insert{\isachardot}{\kern0pt}hyps{\isacharparenleft}{\kern0pt}{\isadigit{1}}{\isacharparenright}{\kern0pt}\ insert{\isachardot}{\kern0pt}hyps{\isacharparenleft}{\kern0pt}{\isadigit{2}}{\isacharparenright}{\kern0pt}\ sum{\isachardot}{\kern0pt}insert{\isacharunderscore}{\kern0pt}remove{\isacharparenright}{\kern0pt}\isanewline
\ \ \ \ \isacommand{have}\isamarkupfalse%
\ {\isacharasterisk}{\kern0pt}{\isacharasterisk}{\kern0pt}{\isacharcolon}{\kern0pt}\ {\isachardoublequoteopen}{\isacharbraceleft}{\kern0pt}y\ {\isasymin}\ space\ M{\isachardot}{\kern0pt}\ {\isacharparenleft}{\kern0pt}{\isasymSum}x{\isasymin}insert\ x\ F{\isachardot}{\kern0pt}\ indicat{\isacharunderscore}{\kern0pt}real\ {\isacharparenleft}{\kern0pt}f\ {\isacharminus}{\kern0pt}{\isacharbackquote}{\kern0pt}\ {\isacharbraceleft}{\kern0pt}x{\isacharbraceright}{\kern0pt}\ {\isasyminter}\ space\ M{\isacharparenright}{\kern0pt}\ y\ {\isacharasterisk}{\kern0pt}\isactrlsub R\ x{\isacharparenright}{\kern0pt}\ {\isasymnoteq}\ {\isadigit{0}}{\isacharbraceright}{\kern0pt}\ {\isasymsubseteq}\ {\isacharbraceleft}{\kern0pt}y\ {\isasymin}\ space\ M{\isachardot}{\kern0pt}\ {\isacharparenleft}{\kern0pt}{\isasymSum}x{\isasymin}F{\isachardot}{\kern0pt}\ indicat{\isacharunderscore}{\kern0pt}real\ {\isacharparenleft}{\kern0pt}f\ {\isacharminus}{\kern0pt}{\isacharbackquote}{\kern0pt}\ {\isacharbraceleft}{\kern0pt}x{\isacharbraceright}{\kern0pt}\ {\isasyminter}\ space\ M{\isacharparenright}{\kern0pt}\ y\ {\isacharasterisk}{\kern0pt}\isactrlsub R\ x{\isacharparenright}{\kern0pt}\ {\isasymnoteq}\ {\isadigit{0}}{\isacharbraceright}{\kern0pt}\ {\isasymunion}\ {\isacharbraceleft}{\kern0pt}y\ {\isasymin}\ space\ M{\isachardot}{\kern0pt}\ indicat{\isacharunderscore}{\kern0pt}real\ {\isacharparenleft}{\kern0pt}f\ {\isacharminus}{\kern0pt}{\isacharbackquote}{\kern0pt}\ {\isacharbraceleft}{\kern0pt}x{\isacharbraceright}{\kern0pt}\ {\isasyminter}\ space\ M{\isacharparenright}{\kern0pt}\ y\ {\isacharasterisk}{\kern0pt}\isactrlsub R\ x\ {\isasymnoteq}\ {\isadigit{0}}{\isacharbraceright}{\kern0pt}{\isachardoublequoteclose}\ \isacommand{unfolding}\isamarkupfalse%
\ {\isacharasterisk}{\kern0pt}\ \isacommand{by}\isamarkupfalse%
\ fastforce\ \ \ \ \isanewline
\ \ \ \ \isacommand{{\isacharbraceleft}{\kern0pt}}\isamarkupfalse%
\isanewline
\ \ \ \ \ \ \isacommand{case}\isamarkupfalse%
\ {\isadigit{1}}\isanewline
\ \ \ \ \ \ \isacommand{hence}\isamarkupfalse%
\ x{\isacharcolon}{\kern0pt}\ {\isachardoublequoteopen}x\ {\isasymin}\ f\ {\isacharbackquote}{\kern0pt}\ space\ M{\isachardoublequoteclose}\ \isakeyword{and}\ F{\isacharcolon}{\kern0pt}\ {\isachardoublequoteopen}F\ {\isasymsubseteq}\ f\ {\isacharbackquote}{\kern0pt}\ space\ M{\isachardoublequoteclose}\ \isacommand{by}\isamarkupfalse%
\ auto\isanewline
\ \ \ \ \ \ \isacommand{show}\isamarkupfalse%
\ {\isacharquery}{\kern0pt}case\ \isanewline
\ \ \ \ \ \ \isacommand{proof}\isamarkupfalse%
\ {\isacharparenleft}{\kern0pt}cases\ {\isachardoublequoteopen}x\ {\isacharequal}{\kern0pt}\ {\isadigit{0}}{\isachardoublequoteclose}{\isacharparenright}{\kern0pt}\isanewline
\ \ \ \ \ \ \ \ \isacommand{case}\isamarkupfalse%
\ True\isanewline
\ \ \ \ \ \ \ \ \isacommand{then}\isamarkupfalse%
\ \isacommand{show}\isamarkupfalse%
\ {\isacharquery}{\kern0pt}thesis\ \isacommand{unfolding}\isamarkupfalse%
\ {\isacharasterisk}{\kern0pt}\ \isacommand{using}\isamarkupfalse%
\ insert{\isacharparenleft}{\kern0pt}{\isadigit{3}}{\isacharparenright}{\kern0pt}{\isacharbrackleft}{\kern0pt}OF\ F{\isacharbrackright}{\kern0pt}\ \isacommand{by}\isamarkupfalse%
\ simp\isanewline
\ \ \ \ \ \ \isacommand{next}\isamarkupfalse%
\isanewline
\ \ \ \ \ \ \ \ \isacommand{case}\isamarkupfalse%
\ False\isanewline
\ \ \ \ \ \ \ \ \isacommand{have}\isamarkupfalse%
\ norm{\isacharunderscore}{\kern0pt}argument{\isacharcolon}{\kern0pt}\ {\isachardoublequoteopen}norm\ {\isacharparenleft}{\kern0pt}{\isacharparenleft}{\kern0pt}{\isasymSum}y{\isasymin}F{\isachardot}{\kern0pt}\ indicat{\isacharunderscore}{\kern0pt}real\ {\isacharparenleft}{\kern0pt}f\ {\isacharminus}{\kern0pt}{\isacharbackquote}{\kern0pt}\ {\isacharbraceleft}{\kern0pt}y{\isacharbraceright}{\kern0pt}\ {\isasyminter}\ space\ M{\isacharparenright}{\kern0pt}\ z\ {\isacharasterisk}{\kern0pt}\isactrlsub R\ y{\isacharparenright}{\kern0pt}\ {\isacharplus}{\kern0pt}\ indicat{\isacharunderscore}{\kern0pt}real\ {\isacharparenleft}{\kern0pt}f\ {\isacharminus}{\kern0pt}{\isacharbackquote}{\kern0pt}\ {\isacharbraceleft}{\kern0pt}x{\isacharbraceright}{\kern0pt}\ {\isasyminter}\ space\ M{\isacharparenright}{\kern0pt}\ z\ {\isacharasterisk}{\kern0pt}\isactrlsub R\ x{\isacharparenright}{\kern0pt}\ {\isacharequal}{\kern0pt}\ norm\ {\isacharparenleft}{\kern0pt}{\isasymSum}y{\isasymin}F{\isachardot}{\kern0pt}\ indicat{\isacharunderscore}{\kern0pt}real\ {\isacharparenleft}{\kern0pt}f\ {\isacharminus}{\kern0pt}{\isacharbackquote}{\kern0pt}\ {\isacharbraceleft}{\kern0pt}y{\isacharbraceright}{\kern0pt}\ {\isasyminter}\ space\ M{\isacharparenright}{\kern0pt}\ z\ {\isacharasterisk}{\kern0pt}\isactrlsub R\ y{\isacharparenright}{\kern0pt}\ {\isacharplus}{\kern0pt}\ norm\ {\isacharparenleft}{\kern0pt}indicat{\isacharunderscore}{\kern0pt}real\ {\isacharparenleft}{\kern0pt}f\ {\isacharminus}{\kern0pt}{\isacharbackquote}{\kern0pt}\ {\isacharbraceleft}{\kern0pt}x{\isacharbraceright}{\kern0pt}\ {\isasyminter}\ space\ M{\isacharparenright}{\kern0pt}\ z\ {\isacharasterisk}{\kern0pt}\isactrlsub R\ x{\isacharparenright}{\kern0pt}{\isachardoublequoteclose}\ \isakeyword{if}\ z{\isacharcolon}{\kern0pt}\ {\isachardoublequoteopen}z\ {\isasymin}\ space\ M{\isachardoublequoteclose}\ \isakeyword{for}\ z\isanewline
\ \ \ \ \ \ \ \ \isacommand{proof}\isamarkupfalse%
\ {\isacharparenleft}{\kern0pt}cases\ {\isachardoublequoteopen}f\ z\ {\isacharequal}{\kern0pt}\ x{\isachardoublequoteclose}{\isacharparenright}{\kern0pt}\isanewline
\ \ \ \ \ \ \ \ \ \ \isacommand{case}\isamarkupfalse%
\ True\isanewline
\ \ \ \ \ \ \ \ \ \ \isacommand{have}\isamarkupfalse%
\ {\isachardoublequoteopen}indicat{\isacharunderscore}{\kern0pt}real\ {\isacharparenleft}{\kern0pt}f\ {\isacharminus}{\kern0pt}{\isacharbackquote}{\kern0pt}\ {\isacharbraceleft}{\kern0pt}y{\isacharbraceright}{\kern0pt}\ {\isasyminter}\ space\ M{\isacharparenright}{\kern0pt}\ z\ {\isacharasterisk}{\kern0pt}\isactrlsub R\ y\ {\isacharequal}{\kern0pt}\ {\isadigit{0}}{\isachardoublequoteclose}\ \isakeyword{if}\ {\isachardoublequoteopen}y\ {\isasymin}\ F{\isachardoublequoteclose}\ \isakeyword{for}\ y\ \isacommand{using}\isamarkupfalse%
\ True\ insert{\isacharparenleft}{\kern0pt}{\isadigit{2}}{\isacharparenright}{\kern0pt}\ z\ that\ {\isadigit{1}}\ \isacommand{unfolding}\isamarkupfalse%
\ indicator{\isacharunderscore}{\kern0pt}def\ \isacommand{by}\isamarkupfalse%
\ force\isanewline
\ \ \ \ \ \ \ \ \ \ \isacommand{hence}\isamarkupfalse%
\ {\isachardoublequoteopen}{\isacharparenleft}{\kern0pt}{\isasymSum}y{\isasymin}F{\isachardot}{\kern0pt}\ indicat{\isacharunderscore}{\kern0pt}real\ {\isacharparenleft}{\kern0pt}f\ {\isacharminus}{\kern0pt}{\isacharbackquote}{\kern0pt}\ {\isacharbraceleft}{\kern0pt}y{\isacharbraceright}{\kern0pt}\ {\isasyminter}\ space\ M{\isacharparenright}{\kern0pt}\ z\ {\isacharasterisk}{\kern0pt}\isactrlsub R\ y{\isacharparenright}{\kern0pt}\ {\isacharequal}{\kern0pt}\ {\isadigit{0}}{\isachardoublequoteclose}\ \isacommand{by}\isamarkupfalse%
\ {\isacharparenleft}{\kern0pt}meson\ sum{\isachardot}{\kern0pt}neutral{\isacharparenright}{\kern0pt}\isanewline
\ \ \ \ \ \ \ \ \ \ \isacommand{then}\isamarkupfalse%
\ \isacommand{show}\isamarkupfalse%
\ {\isacharquery}{\kern0pt}thesis\ \isacommand{by}\isamarkupfalse%
\ force\isanewline
\ \ \ \ \ \ \ \ \isacommand{next}\isamarkupfalse%
\isanewline
\ \ \ \ \ \ \ \ \ \ \isacommand{case}\isamarkupfalse%
\ False\isanewline
\ \ \ \ \ \ \ \ \ \ \isacommand{then}\isamarkupfalse%
\ \isacommand{show}\isamarkupfalse%
\ {\isacharquery}{\kern0pt}thesis\ \isacommand{by}\isamarkupfalse%
\ force\isanewline
\ \ \ \ \ \ \ \ \isacommand{qed}\isamarkupfalse%
\isanewline
\ \ \ \ \ \ \ \ \isacommand{show}\isamarkupfalse%
\ {\isacharquery}{\kern0pt}thesis\ \isacommand{using}\isamarkupfalse%
\ False\ simple{\isacharunderscore}{\kern0pt}functionD{\isacharparenleft}{\kern0pt}{\isadigit{2}}{\isacharparenright}{\kern0pt}{\isacharbrackleft}{\kern0pt}OF\ f{\isacharparenleft}{\kern0pt}{\isadigit{1}}{\isacharparenright}{\kern0pt}{\isacharbrackright}{\kern0pt}\ insert{\isacharparenleft}{\kern0pt}{\isadigit{3}}{\isacharcomma}{\kern0pt}{\isadigit{5}}{\isacharparenright}{\kern0pt}{\isacharbrackleft}{\kern0pt}OF\ F{\isacharbrackright}{\kern0pt}\ simple{\isacharunderscore}{\kern0pt}function{\isacharunderscore}{\kern0pt}scaleR\ fin{\isacharunderscore}{\kern0pt}{\isadigit{0}}\ fin{\isacharunderscore}{\kern0pt}{\isadigit{1}}\ \isacommand{by}\isamarkupfalse%
\ {\isacharparenleft}{\kern0pt}subst\ {\isacharasterisk}{\kern0pt}{\isacharcomma}{\kern0pt}\ subst\ add{\isacharcomma}{\kern0pt}\ subst\ simple{\isacharunderscore}{\kern0pt}function{\isacharunderscore}{\kern0pt}sum{\isacharparenright}{\kern0pt}\ {\isacharparenleft}{\kern0pt}blast\ intro{\isacharcolon}{\kern0pt}\ norm{\isacharunderscore}{\kern0pt}argument\ indicator{\isacharparenright}{\kern0pt}{\isacharplus}{\kern0pt}\isanewline
\ \ \ \ \ \ \isacommand{qed}\isamarkupfalse%
\ \isanewline
\ \ \ \ \isacommand{next}\isamarkupfalse%
\isanewline
\ \ \ \ \ \ \isacommand{case}\isamarkupfalse%
\ {\isadigit{2}}\isanewline
\ \ \ \ \ \ \isacommand{hence}\isamarkupfalse%
\ x{\isacharcolon}{\kern0pt}\ {\isachardoublequoteopen}x\ {\isasymin}\ f\ {\isacharbackquote}{\kern0pt}\ space\ M{\isachardoublequoteclose}\ \isakeyword{and}\ F{\isacharcolon}{\kern0pt}\ {\isachardoublequoteopen}F\ {\isasymsubseteq}\ f\ {\isacharbackquote}{\kern0pt}\ space\ M{\isachardoublequoteclose}\ \isacommand{by}\isamarkupfalse%
\ auto\isanewline
\ \ \ \ \ \ \isacommand{show}\isamarkupfalse%
\ {\isacharquery}{\kern0pt}case\ \isanewline
\ \ \ \ \ \ \isacommand{proof}\isamarkupfalse%
\ {\isacharparenleft}{\kern0pt}cases\ {\isachardoublequoteopen}x\ {\isacharequal}{\kern0pt}\ {\isadigit{0}}{\isachardoublequoteclose}{\isacharparenright}{\kern0pt}\isanewline
\ \ \ \ \ \ \ \ \isacommand{case}\isamarkupfalse%
\ True\isanewline
\ \ \ \ \ \ \ \ \isacommand{then}\isamarkupfalse%
\ \isacommand{show}\isamarkupfalse%
\ {\isacharquery}{\kern0pt}thesis\ \isacommand{unfolding}\isamarkupfalse%
\ {\isacharasterisk}{\kern0pt}\ \isacommand{using}\isamarkupfalse%
\ insert{\isacharparenleft}{\kern0pt}{\isadigit{4}}{\isacharparenright}{\kern0pt}{\isacharbrackleft}{\kern0pt}OF\ F{\isacharbrackright}{\kern0pt}\ \isacommand{by}\isamarkupfalse%
\ simp\isanewline
\ \ \ \ \ \ \isacommand{next}\isamarkupfalse%
\isanewline
\ \ \ \ \ \ \ \ \isacommand{case}\isamarkupfalse%
\ False\isanewline
\ \ \ \ \ \ \ \ \isacommand{then}\isamarkupfalse%
\ \isacommand{show}\isamarkupfalse%
\ {\isacharquery}{\kern0pt}thesis\ \isacommand{unfolding}\isamarkupfalse%
\ {\isacharasterisk}{\kern0pt}\ \isacommand{using}\isamarkupfalse%
\ insert{\isacharparenleft}{\kern0pt}{\isadigit{4}}{\isacharparenright}{\kern0pt}{\isacharbrackleft}{\kern0pt}OF\ F{\isacharbrackright}{\kern0pt}\ simple{\isacharunderscore}{\kern0pt}functionD{\isacharparenleft}{\kern0pt}{\isadigit{2}}{\isacharparenright}{\kern0pt}{\isacharbrackleft}{\kern0pt}OF\ f{\isacharparenleft}{\kern0pt}{\isadigit{1}}{\isacharparenright}{\kern0pt}{\isacharbrackright}{\kern0pt}\ \isacommand{by}\isamarkupfalse%
\ fast\isanewline
\ \ \ \ \ \ \isacommand{qed}\isamarkupfalse%
\isanewline
\ \ \ \ \isacommand{next}\isamarkupfalse%
\isanewline
\ \ \ \ \ \ \isacommand{case}\isamarkupfalse%
\ {\isadigit{3}}\isanewline
\ \ \ \ \ \ \isacommand{hence}\isamarkupfalse%
\ x{\isacharcolon}{\kern0pt}\ {\isachardoublequoteopen}x\ {\isasymin}\ f\ {\isacharbackquote}{\kern0pt}\ space\ M{\isachardoublequoteclose}\ \isakeyword{and}\ F{\isacharcolon}{\kern0pt}\ {\isachardoublequoteopen}F\ {\isasymsubseteq}\ f\ {\isacharbackquote}{\kern0pt}\ space\ M{\isachardoublequoteclose}\ \isacommand{by}\isamarkupfalse%
\ auto\isanewline
\ \ \ \ \ \ \isacommand{show}\isamarkupfalse%
\ {\isacharquery}{\kern0pt}case\ \isanewline
\ \ \ \ \ \ \isacommand{proof}\isamarkupfalse%
\ {\isacharparenleft}{\kern0pt}cases\ {\isachardoublequoteopen}x\ {\isacharequal}{\kern0pt}\ {\isadigit{0}}{\isachardoublequoteclose}{\isacharparenright}{\kern0pt}\isanewline
\ \ \ \ \ \ \ \ \isacommand{case}\isamarkupfalse%
\ True\isanewline
\ \ \ \ \ \ \ \ \isacommand{then}\isamarkupfalse%
\ \isacommand{show}\isamarkupfalse%
\ {\isacharquery}{\kern0pt}thesis\ \isacommand{unfolding}\isamarkupfalse%
\ {\isacharasterisk}{\kern0pt}\ \isacommand{using}\isamarkupfalse%
\ insert{\isacharparenleft}{\kern0pt}{\isadigit{5}}{\isacharparenright}{\kern0pt}{\isacharbrackleft}{\kern0pt}OF\ F{\isacharbrackright}{\kern0pt}\ \isacommand{by}\isamarkupfalse%
\ simp\isanewline
\ \ \ \ \ \ \isacommand{next}\isamarkupfalse%
\isanewline
\ \ \ \ \ \ \ \ \isacommand{case}\isamarkupfalse%
\ False\isanewline
\ \ \ \ \ \ \ \ \isacommand{have}\isamarkupfalse%
\ {\isachardoublequoteopen}emeasure\ M\ {\isacharbraceleft}{\kern0pt}y\ {\isasymin}\ space\ M{\isachardot}{\kern0pt}\ {\isacharparenleft}{\kern0pt}{\isasymSum}x{\isasymin}insert\ x\ F{\isachardot}{\kern0pt}\ indicat{\isacharunderscore}{\kern0pt}real\ {\isacharparenleft}{\kern0pt}f\ {\isacharminus}{\kern0pt}{\isacharbackquote}{\kern0pt}\ {\isacharbraceleft}{\kern0pt}x{\isacharbraceright}{\kern0pt}\ {\isasyminter}\ space\ M{\isacharparenright}{\kern0pt}\ y\ {\isacharasterisk}{\kern0pt}\isactrlsub R\ x{\isacharparenright}{\kern0pt}\ {\isasymnoteq}\ {\isadigit{0}}{\isacharbraceright}{\kern0pt}\ {\isasymle}\ emeasure\ M\ {\isacharparenleft}{\kern0pt}{\isacharbraceleft}{\kern0pt}y\ {\isasymin}\ space\ M{\isachardot}{\kern0pt}\ {\isacharparenleft}{\kern0pt}{\isasymSum}x{\isasymin}F{\isachardot}{\kern0pt}\ indicat{\isacharunderscore}{\kern0pt}real\ {\isacharparenleft}{\kern0pt}f\ {\isacharminus}{\kern0pt}{\isacharbackquote}{\kern0pt}\ {\isacharbraceleft}{\kern0pt}x{\isacharbraceright}{\kern0pt}\ {\isasyminter}\ space\ M{\isacharparenright}{\kern0pt}\ y\ {\isacharasterisk}{\kern0pt}\isactrlsub R\ x{\isacharparenright}{\kern0pt}\ {\isasymnoteq}\ {\isadigit{0}}{\isacharbraceright}{\kern0pt}\ {\isasymunion}\ {\isacharbraceleft}{\kern0pt}y\ {\isasymin}\ space\ M{\isachardot}{\kern0pt}\ indicat{\isacharunderscore}{\kern0pt}real\ {\isacharparenleft}{\kern0pt}f\ {\isacharminus}{\kern0pt}{\isacharbackquote}{\kern0pt}\ {\isacharbraceleft}{\kern0pt}x{\isacharbraceright}{\kern0pt}\ {\isasyminter}\ space\ M{\isacharparenright}{\kern0pt}\ y\ {\isacharasterisk}{\kern0pt}\isactrlsub R\ x\ {\isasymnoteq}\ {\isadigit{0}}{\isacharbraceright}{\kern0pt}{\isacharparenright}{\kern0pt}{\isachardoublequoteclose}\isanewline
\ \ \ \ \ \ \ \ \ \ \isacommand{using}\isamarkupfalse%
\ {\isacharasterisk}{\kern0pt}{\isacharasterisk}{\kern0pt}\ simple{\isacharunderscore}{\kern0pt}functionD{\isacharparenleft}{\kern0pt}{\isadigit{2}}{\isacharparenright}{\kern0pt}{\isacharbrackleft}{\kern0pt}OF\ insert{\isacharparenleft}{\kern0pt}{\isadigit{4}}{\isacharparenright}{\kern0pt}{\isacharbrackleft}{\kern0pt}OF\ F{\isacharbrackright}{\kern0pt}{\isacharbrackright}{\kern0pt}\ simple{\isacharunderscore}{\kern0pt}functionD{\isacharparenleft}{\kern0pt}{\isadigit{2}}{\isacharparenright}{\kern0pt}{\isacharbrackleft}{\kern0pt}OF\ f{\isacharparenleft}{\kern0pt}{\isadigit{1}}{\isacharparenright}{\kern0pt}{\isacharbrackright}{\kern0pt}\ \isacommand{by}\isamarkupfalse%
\ {\isacharparenleft}{\kern0pt}intro\ emeasure{\isacharunderscore}{\kern0pt}mono{\isacharcomma}{\kern0pt}\ force{\isacharplus}{\kern0pt}{\isacharparenright}{\kern0pt}\isanewline
\ \ \ \ \ \ \ \ \isacommand{also}\isamarkupfalse%
\ \isacommand{have}\isamarkupfalse%
\ {\isachardoublequoteopen}{\isachardot}{\kern0pt}{\isachardot}{\kern0pt}{\isachardot}{\kern0pt}\ {\isasymle}\ emeasure\ M\ {\isacharbraceleft}{\kern0pt}y\ {\isasymin}\ space\ M{\isachardot}{\kern0pt}\ {\isacharparenleft}{\kern0pt}{\isasymSum}x{\isasymin}F{\isachardot}{\kern0pt}\ indicat{\isacharunderscore}{\kern0pt}real\ {\isacharparenleft}{\kern0pt}f\ {\isacharminus}{\kern0pt}{\isacharbackquote}{\kern0pt}\ {\isacharbraceleft}{\kern0pt}x{\isacharbraceright}{\kern0pt}\ {\isasyminter}\ space\ M{\isacharparenright}{\kern0pt}\ y\ {\isacharasterisk}{\kern0pt}\isactrlsub R\ x{\isacharparenright}{\kern0pt}\ {\isasymnoteq}\ {\isadigit{0}}{\isacharbraceright}{\kern0pt}\ {\isacharplus}{\kern0pt}\ emeasure\ M\ {\isacharbraceleft}{\kern0pt}y\ {\isasymin}\ space\ M{\isachardot}{\kern0pt}\ indicat{\isacharunderscore}{\kern0pt}real\ {\isacharparenleft}{\kern0pt}f\ {\isacharminus}{\kern0pt}{\isacharbackquote}{\kern0pt}\ {\isacharbraceleft}{\kern0pt}x{\isacharbraceright}{\kern0pt}\ {\isasyminter}\ space\ M{\isacharparenright}{\kern0pt}\ y\ {\isacharasterisk}{\kern0pt}\isactrlsub R\ x\ {\isasymnoteq}\ {\isadigit{0}}{\isacharbraceright}{\kern0pt}{\isachardoublequoteclose}\isanewline
\ \ \ \ \ \ \ \ \ \ \isacommand{using}\isamarkupfalse%
\ simple{\isacharunderscore}{\kern0pt}functionD{\isacharparenleft}{\kern0pt}{\isadigit{2}}{\isacharparenright}{\kern0pt}{\isacharbrackleft}{\kern0pt}OF\ insert{\isacharparenleft}{\kern0pt}{\isadigit{4}}{\isacharparenright}{\kern0pt}{\isacharbrackleft}{\kern0pt}OF\ F{\isacharbrackright}{\kern0pt}{\isacharbrackright}{\kern0pt}\ simple{\isacharunderscore}{\kern0pt}functionD{\isacharparenleft}{\kern0pt}{\isadigit{2}}{\isacharparenright}{\kern0pt}{\isacharbrackleft}{\kern0pt}OF\ f{\isacharparenleft}{\kern0pt}{\isadigit{1}}{\isacharparenright}{\kern0pt}{\isacharbrackright}{\kern0pt}\ \isacommand{by}\isamarkupfalse%
\ {\isacharparenleft}{\kern0pt}intro\ emeasure{\isacharunderscore}{\kern0pt}subadditive{\isacharcomma}{\kern0pt}\ force{\isacharplus}{\kern0pt}{\isacharparenright}{\kern0pt}\isanewline
\ \ \ \ \ \ \ \ \isacommand{also}\isamarkupfalse%
\ \isacommand{have}\isamarkupfalse%
\ {\isachardoublequoteopen}{\isachardot}{\kern0pt}{\isachardot}{\kern0pt}{\isachardot}{\kern0pt}\ {\isacharless}{\kern0pt}\ {\isasyminfinity}{\isachardoublequoteclose}\ \isacommand{using}\isamarkupfalse%
\ insert{\isacharparenleft}{\kern0pt}{\isadigit{5}}{\isacharparenright}{\kern0pt}{\isacharbrackleft}{\kern0pt}OF\ F{\isacharbrackright}{\kern0pt}\ fin{\isacharunderscore}{\kern0pt}{\isadigit{1}}{\isacharbrackleft}{\kern0pt}OF\ False{\isacharbrackright}{\kern0pt}\ \isacommand{by}\isamarkupfalse%
\ {\isacharparenleft}{\kern0pt}simp\ add{\isacharcolon}{\kern0pt}\ less{\isacharunderscore}{\kern0pt}top{\isacharparenright}{\kern0pt}\isanewline
\ \ \ \ \ \ \ \ \isacommand{finally}\isamarkupfalse%
\ \isacommand{show}\isamarkupfalse%
\ {\isacharquery}{\kern0pt}thesis\ \isacommand{by}\isamarkupfalse%
\ simp\isanewline
\ \ \ \ \ \ \isacommand{qed}\isamarkupfalse%
\isanewline
\ \ \ \ \isacommand{{\isacharbraceright}{\kern0pt}}\isamarkupfalse%
\isanewline
\ \ \isacommand{qed}\isamarkupfalse%
\isanewline
\ \ \isacommand{moreover}\isamarkupfalse%
\ \isacommand{have}\isamarkupfalse%
\ {\isachardoublequoteopen}simple{\isacharunderscore}{\kern0pt}function\ M\ {\isacharparenleft}{\kern0pt}{\isasymlambda}x{\isachardot}{\kern0pt}\ {\isasymSum}y{\isasymin}f\ {\isacharbackquote}{\kern0pt}\ space\ M{\isachardot}{\kern0pt}\ indicat{\isacharunderscore}{\kern0pt}real\ {\isacharparenleft}{\kern0pt}f\ {\isacharminus}{\kern0pt}{\isacharbackquote}{\kern0pt}\ {\isacharbraceleft}{\kern0pt}y{\isacharbraceright}{\kern0pt}\ {\isasyminter}\ space\ M{\isacharparenright}{\kern0pt}\ x\ {\isacharasterisk}{\kern0pt}\isactrlsub R\ y{\isacharparenright}{\kern0pt}{\isachardoublequoteclose}\ \isacommand{using}\isamarkupfalse%
\ calculation\ \isacommand{by}\isamarkupfalse%
\ blast\isanewline
\ \ \isacommand{moreover}\isamarkupfalse%
\ \isacommand{have}\isamarkupfalse%
\ {\isachardoublequoteopen}P\ {\isacharparenleft}{\kern0pt}{\isasymlambda}x{\isachardot}{\kern0pt}\ {\isasymSum}y{\isasymin}f\ {\isacharbackquote}{\kern0pt}\ space\ M{\isachardot}{\kern0pt}\ indicat{\isacharunderscore}{\kern0pt}real\ {\isacharparenleft}{\kern0pt}f\ {\isacharminus}{\kern0pt}{\isacharbackquote}{\kern0pt}\ {\isacharbraceleft}{\kern0pt}y{\isacharbraceright}{\kern0pt}\ {\isasyminter}\ space\ M{\isacharparenright}{\kern0pt}\ x\ {\isacharasterisk}{\kern0pt}\isactrlsub R\ y{\isacharparenright}{\kern0pt}{\isachardoublequoteclose}\ \isacommand{using}\isamarkupfalse%
\ calculation\ \isacommand{by}\isamarkupfalse%
\ blast\isanewline
\ \ \isacommand{ultimately}\isamarkupfalse%
\ \isacommand{show}\isamarkupfalse%
\ {\isacharquery}{\kern0pt}thesis\ \isacommand{by}\isamarkupfalse%
\ {\isacharparenleft}{\kern0pt}intro\ cong{\isacharbrackleft}{\kern0pt}OF\ {\isacharunderscore}{\kern0pt}\ {\isacharunderscore}{\kern0pt}\ f{\isacharparenleft}{\kern0pt}{\isadigit{1}}{\isacharcomma}{\kern0pt}{\isadigit{2}}{\isacharparenright}{\kern0pt}{\isacharbrackright}{\kern0pt}{\isacharcomma}{\kern0pt}\ blast{\isacharcomma}{\kern0pt}\ presburger{\isacharplus}{\kern0pt}{\isacharparenright}{\kern0pt}\ \isanewline
\isacommand{qed}\isamarkupfalse%
%
\endisatagproof
{\isafoldproof}%
%
\isadelimproof
\isanewline
%
\endisadelimproof
\isanewline
%
\isadelimimportant
\isanewline
%
\endisadelimimportant
%
\isatagimportant
\isacommand{lemma}\isamarkupfalse%
\ simple{\isacharunderscore}{\kern0pt}integrable{\isacharunderscore}{\kern0pt}function{\isacharunderscore}{\kern0pt}induct{\isacharunderscore}{\kern0pt}nn{\isacharbrackleft}{\kern0pt}consumes\ {\isadigit{3}}{\isacharcomma}{\kern0pt}\ case{\isacharunderscore}{\kern0pt}names\ cong\ indicator\ add{\isacharcomma}{\kern0pt}\ induct\ set{\isacharcolon}{\kern0pt}\ simple{\isacharunderscore}{\kern0pt}function{\isacharbrackright}{\kern0pt}{\isacharcolon}{\kern0pt}\isanewline
\ \ \isakeyword{fixes}\ f\ {\isacharcolon}{\kern0pt}{\isacharcolon}{\kern0pt}\ {\isachardoublequoteopen}{\isacharprime}{\kern0pt}a\ {\isasymRightarrow}\ {\isacharprime}{\kern0pt}b\ {\isacharcolon}{\kern0pt}{\isacharcolon}{\kern0pt}\ {\isacharbraceleft}{\kern0pt}second{\isacharunderscore}{\kern0pt}countable{\isacharunderscore}{\kern0pt}topology{\isacharcomma}{\kern0pt}\ banach{\isacharcomma}{\kern0pt}\ linorder{\isacharunderscore}{\kern0pt}topology{\isacharcomma}{\kern0pt}\ ordered{\isacharunderscore}{\kern0pt}real{\isacharunderscore}{\kern0pt}vector{\isacharbraceright}{\kern0pt}{\isachardoublequoteclose}\isanewline
\ \ \isakeyword{assumes}\ f{\isacharcolon}{\kern0pt}\ {\isachardoublequoteopen}simple{\isacharunderscore}{\kern0pt}function\ M\ f{\isachardoublequoteclose}\ {\isachardoublequoteopen}emeasure\ M\ {\isacharbraceleft}{\kern0pt}y\ {\isasymin}\ space\ M{\isachardot}{\kern0pt}\ f\ y\ {\isasymnoteq}\ {\isadigit{0}}{\isacharbraceright}{\kern0pt}\ {\isasymnoteq}\ {\isasyminfinity}{\isachardoublequoteclose}\ {\isachardoublequoteopen}{\isasymAnd}x{\isachardot}{\kern0pt}\ x\ {\isasymin}\ space\ M\ {\isasymlongrightarrow}\ f\ x\ {\isasymge}\ {\isadigit{0}}{\isachardoublequoteclose}\isanewline
\ \ \isakeyword{assumes}\ cong{\isacharcolon}{\kern0pt}\ {\isachardoublequoteopen}{\isasymAnd}f\ g{\isachardot}{\kern0pt}\ simple{\isacharunderscore}{\kern0pt}function\ M\ f\ {\isasymLongrightarrow}\ emeasure\ M\ {\isacharbraceleft}{\kern0pt}y\ {\isasymin}\ space\ M{\isachardot}{\kern0pt}\ f\ y\ {\isasymnoteq}\ {\isadigit{0}}{\isacharbraceright}{\kern0pt}\ {\isasymnoteq}\ {\isasyminfinity}\ {\isasymLongrightarrow}\ {\isacharparenleft}{\kern0pt}{\isasymAnd}x{\isachardot}{\kern0pt}\ x\ {\isasymin}\ space\ M\ {\isasymLongrightarrow}\ f\ x\ {\isasymge}\ {\isadigit{0}}{\isacharparenright}{\kern0pt}\ {\isasymLongrightarrow}\ simple{\isacharunderscore}{\kern0pt}function\ M\ g\ {\isasymLongrightarrow}\ emeasure\ M\ {\isacharbraceleft}{\kern0pt}y\ {\isasymin}\ space\ M{\isachardot}{\kern0pt}\ g\ y\ {\isasymnoteq}\ {\isadigit{0}}{\isacharbraceright}{\kern0pt}\ {\isasymnoteq}\ {\isasyminfinity}\ {\isasymLongrightarrow}\ {\isacharparenleft}{\kern0pt}{\isasymAnd}x{\isachardot}{\kern0pt}\ x\ {\isasymin}\ space\ M\ {\isasymLongrightarrow}\ g\ x\ {\isasymge}\ {\isadigit{0}}{\isacharparenright}{\kern0pt}\ {\isasymLongrightarrow}\ {\isacharparenleft}{\kern0pt}{\isasymAnd}x{\isachardot}{\kern0pt}\ x\ {\isasymin}\ space\ M\ {\isasymLongrightarrow}\ f\ x\ {\isacharequal}{\kern0pt}\ g\ x{\isacharparenright}{\kern0pt}\ {\isasymLongrightarrow}\ P\ f\ {\isasymLongrightarrow}\ P\ g{\isachardoublequoteclose}\isanewline
\ \ \isakeyword{assumes}\ indicator{\isacharcolon}{\kern0pt}\ {\isachardoublequoteopen}{\isasymAnd}A\ y{\isachardot}{\kern0pt}\ y\ {\isasymge}\ {\isadigit{0}}\ {\isasymLongrightarrow}\ A\ {\isasymin}\ sets\ M\ {\isasymLongrightarrow}\ emeasure\ M\ A\ {\isacharless}{\kern0pt}\ {\isasyminfinity}\ {\isasymLongrightarrow}\ P\ {\isacharparenleft}{\kern0pt}{\isasymlambda}x{\isachardot}{\kern0pt}\ indicator\ A\ x\ {\isacharasterisk}{\kern0pt}\isactrlsub R\ y{\isacharparenright}{\kern0pt}{\isachardoublequoteclose}\isanewline
\ \ \isakeyword{assumes}\ add{\isacharcolon}{\kern0pt}\ {\isachardoublequoteopen}{\isasymAnd}f\ g{\isachardot}{\kern0pt}\ {\isacharparenleft}{\kern0pt}{\isasymAnd}x{\isachardot}{\kern0pt}\ x\ {\isasymin}\ space\ M\ {\isasymLongrightarrow}\ f\ x\ {\isasymge}\ {\isadigit{0}}{\isacharparenright}{\kern0pt}\ {\isasymLongrightarrow}\ simple{\isacharunderscore}{\kern0pt}function\ M\ f\ {\isasymLongrightarrow}\ emeasure\ M\ {\isacharbraceleft}{\kern0pt}y\ {\isasymin}\ space\ M{\isachardot}{\kern0pt}\ f\ y\ {\isasymnoteq}\ {\isadigit{0}}{\isacharbraceright}{\kern0pt}\ {\isasymnoteq}\ {\isasyminfinity}\ {\isasymLongrightarrow}\ \isanewline
\ \ \ \ \ \ \ \ \ \ \ \ \ \ \ \ \ \ \ \ \ \ {\isacharparenleft}{\kern0pt}{\isasymAnd}x{\isachardot}{\kern0pt}\ x\ {\isasymin}\ space\ M\ {\isasymLongrightarrow}\ g\ x\ {\isasymge}\ {\isadigit{0}}{\isacharparenright}{\kern0pt}\ {\isasymLongrightarrow}\ simple{\isacharunderscore}{\kern0pt}function\ M\ g\ {\isasymLongrightarrow}\ emeasure\ M\ {\isacharbraceleft}{\kern0pt}y\ {\isasymin}\ space\ M{\isachardot}{\kern0pt}\ g\ y\ {\isasymnoteq}\ {\isadigit{0}}{\isacharbraceright}{\kern0pt}\ {\isasymnoteq}\ {\isasyminfinity}\ {\isasymLongrightarrow}\ \isanewline
\ \ \ \ \ \ \ \ \ \ \ \ \ \ \ \ \ \ \ \ \ \ {\isacharparenleft}{\kern0pt}{\isasymAnd}z{\isachardot}{\kern0pt}\ z\ {\isasymin}\ space\ M\ {\isasymLongrightarrow}\ norm\ {\isacharparenleft}{\kern0pt}f\ z\ {\isacharplus}{\kern0pt}\ g\ z{\isacharparenright}{\kern0pt}\ {\isacharequal}{\kern0pt}\ norm\ {\isacharparenleft}{\kern0pt}f\ z{\isacharparenright}{\kern0pt}\ {\isacharplus}{\kern0pt}\ norm\ {\isacharparenleft}{\kern0pt}g\ z{\isacharparenright}{\kern0pt}{\isacharparenright}{\kern0pt}\ {\isasymLongrightarrow}\isanewline
\ \ \ \ \ \ \ \ \ \ \ \ \ \ \ \ \ \ \ \ \ \ P\ f\ {\isasymLongrightarrow}\ P\ g\ {\isasymLongrightarrow}\ P\ {\isacharparenleft}{\kern0pt}{\isasymlambda}x{\isachardot}{\kern0pt}\ f\ x\ {\isacharplus}{\kern0pt}\ g\ x{\isacharparenright}{\kern0pt}{\isachardoublequoteclose}\isanewline
\ \ \isakeyword{shows}\ {\isachardoublequoteopen}P\ f{\isachardoublequoteclose}%
\endisatagimportant
{\isafoldimportant}%
%
\isadelimimportant
\isanewline
%
\endisadelimimportant
%
\isadelimproof
%
\endisadelimproof
%
\isatagproof
\isacommand{proof}\isamarkupfalse%
{\isacharminus}{\kern0pt}\isanewline
\ \ \isacommand{let}\isamarkupfalse%
\ {\isacharquery}{\kern0pt}f\ {\isacharequal}{\kern0pt}\ {\isachardoublequoteopen}{\isasymlambda}x{\isachardot}{\kern0pt}\ {\isacharparenleft}{\kern0pt}{\isasymSum}y{\isasymin}f\ {\isacharbackquote}{\kern0pt}\ space\ M{\isachardot}{\kern0pt}\ indicat{\isacharunderscore}{\kern0pt}real\ {\isacharparenleft}{\kern0pt}f\ {\isacharminus}{\kern0pt}{\isacharbackquote}{\kern0pt}\ {\isacharbraceleft}{\kern0pt}y{\isacharbraceright}{\kern0pt}\ {\isasyminter}\ space\ M{\isacharparenright}{\kern0pt}\ x\ {\isacharasterisk}{\kern0pt}\isactrlsub R\ y{\isacharparenright}{\kern0pt}{\isachardoublequoteclose}\isanewline
\ \ \isacommand{have}\isamarkupfalse%
\ f{\isacharunderscore}{\kern0pt}ae{\isacharunderscore}{\kern0pt}eq{\isacharcolon}{\kern0pt}\ {\isachardoublequoteopen}f\ x\ {\isacharequal}{\kern0pt}\ {\isacharquery}{\kern0pt}f\ x{\isachardoublequoteclose}\ \isakeyword{if}\ {\isachardoublequoteopen}x\ {\isasymin}\ space\ M{\isachardoublequoteclose}\ \isakeyword{for}\ x\ \isacommand{using}\isamarkupfalse%
\ simple{\isacharunderscore}{\kern0pt}function{\isacharunderscore}{\kern0pt}indicator{\isacharunderscore}{\kern0pt}representation{\isacharbrackleft}{\kern0pt}OF\ f{\isacharparenleft}{\kern0pt}{\isadigit{1}}{\isacharparenright}{\kern0pt}\ that{\isacharbrackright}{\kern0pt}\ \isacommand{{\isachardot}{\kern0pt}}\isamarkupfalse%
\isanewline
\ \ \isacommand{moreover}\isamarkupfalse%
\ \isacommand{have}\isamarkupfalse%
\ {\isachardoublequoteopen}emeasure\ M\ {\isacharbraceleft}{\kern0pt}y\ {\isasymin}\ space\ M{\isachardot}{\kern0pt}\ {\isacharquery}{\kern0pt}f\ y\ {\isasymnoteq}\ {\isadigit{0}}{\isacharbraceright}{\kern0pt}\ {\isasymnoteq}\ {\isasyminfinity}{\isachardoublequoteclose}\ \isacommand{by}\isamarkupfalse%
\ {\isacharparenleft}{\kern0pt}metis\ {\isacharparenleft}{\kern0pt}no{\isacharunderscore}{\kern0pt}types{\isacharcomma}{\kern0pt}\ lifting{\isacharparenright}{\kern0pt}\ Collect{\isacharunderscore}{\kern0pt}cong\ calculation\ f{\isacharparenleft}{\kern0pt}{\isadigit{2}}{\isacharparenright}{\kern0pt}{\isacharparenright}{\kern0pt}\isanewline
\ \ \isacommand{moreover}\isamarkupfalse%
\ \isacommand{have}\isamarkupfalse%
\ {\isachardoublequoteopen}P\ {\isacharparenleft}{\kern0pt}{\isasymlambda}x{\isachardot}{\kern0pt}\ {\isasymSum}y{\isasymin}S{\isachardot}{\kern0pt}\ indicat{\isacharunderscore}{\kern0pt}real\ {\isacharparenleft}{\kern0pt}f\ {\isacharminus}{\kern0pt}{\isacharbackquote}{\kern0pt}\ {\isacharbraceleft}{\kern0pt}y{\isacharbraceright}{\kern0pt}\ {\isasyminter}\ space\ M{\isacharparenright}{\kern0pt}\ x\ {\isacharasterisk}{\kern0pt}\isactrlsub R\ y{\isacharparenright}{\kern0pt}{\isachardoublequoteclose}\isanewline
\ \ \ \ \ \ \ \ \ \ \ \ \ \ \ \ {\isachardoublequoteopen}simple{\isacharunderscore}{\kern0pt}function\ M\ {\isacharparenleft}{\kern0pt}{\isasymlambda}x{\isachardot}{\kern0pt}\ {\isasymSum}y{\isasymin}S{\isachardot}{\kern0pt}\ indicat{\isacharunderscore}{\kern0pt}real\ {\isacharparenleft}{\kern0pt}f\ {\isacharminus}{\kern0pt}{\isacharbackquote}{\kern0pt}\ {\isacharbraceleft}{\kern0pt}y{\isacharbraceright}{\kern0pt}\ {\isasyminter}\ space\ M{\isacharparenright}{\kern0pt}\ x\ {\isacharasterisk}{\kern0pt}\isactrlsub R\ y{\isacharparenright}{\kern0pt}{\isachardoublequoteclose}\isanewline
\ \ \ \ \ \ \ \ \ \ \ \ \ \ \ \ {\isachardoublequoteopen}emeasure\ M\ {\isacharbraceleft}{\kern0pt}y\ {\isasymin}\ space\ M{\isachardot}{\kern0pt}\ {\isacharparenleft}{\kern0pt}{\isasymSum}x{\isasymin}S{\isachardot}{\kern0pt}\ indicat{\isacharunderscore}{\kern0pt}real\ {\isacharparenleft}{\kern0pt}f\ {\isacharminus}{\kern0pt}{\isacharbackquote}{\kern0pt}\ {\isacharbraceleft}{\kern0pt}x{\isacharbraceright}{\kern0pt}\ {\isasyminter}\ space\ M{\isacharparenright}{\kern0pt}\ y\ {\isacharasterisk}{\kern0pt}\isactrlsub R\ x{\isacharparenright}{\kern0pt}\ {\isasymnoteq}\ {\isadigit{0}}{\isacharbraceright}{\kern0pt}\ {\isasymnoteq}\ {\isasyminfinity}{\isachardoublequoteclose}\isanewline
\ \ \ \ \ \ \ \ \ \ \ \ \ \ \ \ {\isachardoublequoteopen}{\isasymAnd}x{\isachardot}{\kern0pt}\ x\ {\isasymin}\ space\ M\ {\isasymLongrightarrow}\ {\isadigit{0}}\ {\isasymle}\ {\isacharparenleft}{\kern0pt}{\isasymSum}y{\isasymin}S{\isachardot}{\kern0pt}\ indicat{\isacharunderscore}{\kern0pt}real\ {\isacharparenleft}{\kern0pt}f\ {\isacharminus}{\kern0pt}{\isacharbackquote}{\kern0pt}\ {\isacharbraceleft}{\kern0pt}y{\isacharbraceright}{\kern0pt}\ {\isasyminter}\ space\ M{\isacharparenright}{\kern0pt}\ x\ {\isacharasterisk}{\kern0pt}\isactrlsub R\ y{\isacharparenright}{\kern0pt}{\isachardoublequoteclose}\isanewline
\ \ \ \ \ \ \ \ \ \ \ \ \ \ \ \ \isakeyword{if}\ {\isachardoublequoteopen}S\ {\isasymsubseteq}\ f\ {\isacharbackquote}{\kern0pt}\ space\ M{\isachardoublequoteclose}\ \isakeyword{for}\ S\ \isacommand{using}\isamarkupfalse%
\ simple{\isacharunderscore}{\kern0pt}functionD{\isacharparenleft}{\kern0pt}{\isadigit{1}}{\isacharparenright}{\kern0pt}{\isacharbrackleft}{\kern0pt}OF\ assms{\isacharparenleft}{\kern0pt}{\isadigit{1}}{\isacharparenright}{\kern0pt}{\isacharcomma}{\kern0pt}\ THEN\ rev{\isacharunderscore}{\kern0pt}finite{\isacharunderscore}{\kern0pt}subset{\isacharcomma}{\kern0pt}\ OF\ that{\isacharbrackright}{\kern0pt}\ that\ \isanewline
\ \ \isacommand{proof}\isamarkupfalse%
\ {\isacharparenleft}{\kern0pt}induction\ rule{\isacharcolon}{\kern0pt}\ finite{\isacharunderscore}{\kern0pt}subset{\isacharunderscore}{\kern0pt}induct{\isacharprime}{\kern0pt}{\isacharparenright}{\kern0pt}\isanewline
\ \ \ \ \isacommand{case}\isamarkupfalse%
\ empty\isanewline
\ \ \ \ \isacommand{{\isacharbraceleft}{\kern0pt}}\isamarkupfalse%
\isanewline
\ \ \ \ \ \ \isacommand{case}\isamarkupfalse%
\ {\isadigit{1}}\isanewline
\ \ \ \ \ \ \isacommand{then}\isamarkupfalse%
\ \isacommand{show}\isamarkupfalse%
\ {\isacharquery}{\kern0pt}case\ \isacommand{using}\isamarkupfalse%
\ indicator{\isacharbrackleft}{\kern0pt}of\ {\isadigit{0}}\ {\isachardoublequoteopen}{\isacharbraceleft}{\kern0pt}{\isacharbraceright}{\kern0pt}{\isachardoublequoteclose}{\isacharbrackright}{\kern0pt}\ \isacommand{by}\isamarkupfalse%
\ force\isanewline
\ \ \ \ \isacommand{next}\isamarkupfalse%
\isanewline
\ \ \ \ \ \ \isacommand{case}\isamarkupfalse%
\ {\isadigit{2}}\isanewline
\ \ \ \ \ \ \isacommand{then}\isamarkupfalse%
\ \isacommand{show}\isamarkupfalse%
\ {\isacharquery}{\kern0pt}case\ \isacommand{by}\isamarkupfalse%
\ force\ \isanewline
\ \ \ \ \isacommand{next}\isamarkupfalse%
\isanewline
\ \ \ \ \ \ \isacommand{case}\isamarkupfalse%
\ {\isadigit{3}}\isanewline
\ \ \ \ \ \ \isacommand{then}\isamarkupfalse%
\ \isacommand{show}\isamarkupfalse%
\ {\isacharquery}{\kern0pt}case\ \isacommand{by}\isamarkupfalse%
\ force\ \isanewline
\ \ \ \ \isacommand{next}\isamarkupfalse%
\isanewline
\ \ \ \ \ \ \isacommand{case}\isamarkupfalse%
\ {\isadigit{4}}\isanewline
\ \ \ \ \ \ \isacommand{then}\isamarkupfalse%
\ \isacommand{show}\isamarkupfalse%
\ {\isacharquery}{\kern0pt}case\ \isacommand{by}\isamarkupfalse%
\ force\ \isanewline
\ \ \ \ \isacommand{{\isacharbraceright}{\kern0pt}}\isamarkupfalse%
\isanewline
\ \ \isacommand{next}\isamarkupfalse%
\isanewline
\ \ \ \ \isacommand{case}\isamarkupfalse%
\ {\isacharparenleft}{\kern0pt}insert\ x\ F{\isacharparenright}{\kern0pt}\isanewline
\ \ \ \ \isacommand{have}\isamarkupfalse%
\ {\isachardoublequoteopen}{\isacharparenleft}{\kern0pt}f\ {\isacharminus}{\kern0pt}{\isacharbackquote}{\kern0pt}\ {\isacharbraceleft}{\kern0pt}x{\isacharbraceright}{\kern0pt}\ {\isasyminter}\ space\ M{\isacharparenright}{\kern0pt}\ {\isasymsubseteq}\ {\isacharbraceleft}{\kern0pt}y\ {\isasymin}\ space\ M{\isachardot}{\kern0pt}\ f\ y\ {\isasymnoteq}\ {\isadigit{0}}{\isacharbraceright}{\kern0pt}{\isachardoublequoteclose}\ \isakeyword{if}\ {\isachardoublequoteopen}x\ {\isasymnoteq}\ {\isadigit{0}}{\isachardoublequoteclose}\ \isacommand{using}\isamarkupfalse%
\ that\ \isacommand{by}\isamarkupfalse%
\ blast\isanewline
\ \ \ \ \isacommand{moreover}\isamarkupfalse%
\ \isacommand{have}\isamarkupfalse%
\ {\isachardoublequoteopen}{\isacharbraceleft}{\kern0pt}y\ {\isasymin}\ space\ M{\isachardot}{\kern0pt}\ f\ y\ {\isasymnoteq}\ {\isadigit{0}}{\isacharbraceright}{\kern0pt}\ {\isacharequal}{\kern0pt}\ space\ M\ {\isacharminus}{\kern0pt}\ {\isacharparenleft}{\kern0pt}f\ {\isacharminus}{\kern0pt}{\isacharbackquote}{\kern0pt}\ {\isacharbraceleft}{\kern0pt}{\isadigit{0}}{\isacharbraceright}{\kern0pt}\ {\isasyminter}\ space\ M{\isacharparenright}{\kern0pt}{\isachardoublequoteclose}\ \isacommand{by}\isamarkupfalse%
\ blast\isanewline
\ \ \ \ \isacommand{moreover}\isamarkupfalse%
\ \isacommand{have}\isamarkupfalse%
\ {\isachardoublequoteopen}space\ M\ {\isacharminus}{\kern0pt}\ {\isacharparenleft}{\kern0pt}f\ {\isacharminus}{\kern0pt}{\isacharbackquote}{\kern0pt}\ {\isacharbraceleft}{\kern0pt}{\isadigit{0}}{\isacharbraceright}{\kern0pt}\ {\isasyminter}\ space\ M{\isacharparenright}{\kern0pt}\ {\isasymin}\ sets\ M{\isachardoublequoteclose}\ \isacommand{using}\isamarkupfalse%
\ simple{\isacharunderscore}{\kern0pt}functionD{\isacharparenleft}{\kern0pt}{\isadigit{2}}{\isacharparenright}{\kern0pt}{\isacharbrackleft}{\kern0pt}OF\ f{\isacharparenleft}{\kern0pt}{\isadigit{1}}{\isacharparenright}{\kern0pt}{\isacharbrackright}{\kern0pt}\ \isacommand{by}\isamarkupfalse%
\ blast\isanewline
\ \ \ \ \isacommand{ultimately}\isamarkupfalse%
\ \isacommand{have}\isamarkupfalse%
\ fin{\isacharunderscore}{\kern0pt}{\isadigit{0}}{\isacharcolon}{\kern0pt}\ {\isachardoublequoteopen}emeasure\ M\ {\isacharparenleft}{\kern0pt}f\ {\isacharminus}{\kern0pt}{\isacharbackquote}{\kern0pt}\ {\isacharbraceleft}{\kern0pt}x{\isacharbraceright}{\kern0pt}\ {\isasyminter}\ space\ M{\isacharparenright}{\kern0pt}\ {\isacharless}{\kern0pt}\ {\isasyminfinity}{\isachardoublequoteclose}\ \isakeyword{if}\ {\isachardoublequoteopen}x\ {\isasymnoteq}\ {\isadigit{0}}{\isachardoublequoteclose}\ \isacommand{using}\isamarkupfalse%
\ that\ \isacommand{by}\isamarkupfalse%
\ {\isacharparenleft}{\kern0pt}metis\ emeasure{\isacharunderscore}{\kern0pt}mono\ f{\isacharparenleft}{\kern0pt}{\isadigit{2}}{\isacharparenright}{\kern0pt}\ infinity{\isacharunderscore}{\kern0pt}ennreal{\isacharunderscore}{\kern0pt}def\ top{\isachardot}{\kern0pt}not{\isacharunderscore}{\kern0pt}eq{\isacharunderscore}{\kern0pt}extremum\ top{\isacharunderscore}{\kern0pt}unique{\isacharparenright}{\kern0pt}\isanewline
\ \ \ \ \isacommand{hence}\isamarkupfalse%
\ fin{\isacharunderscore}{\kern0pt}{\isadigit{1}}{\isacharcolon}{\kern0pt}\ {\isachardoublequoteopen}emeasure\ M\ {\isacharbraceleft}{\kern0pt}y\ {\isasymin}\ space\ M{\isachardot}{\kern0pt}\ indicat{\isacharunderscore}{\kern0pt}real\ {\isacharparenleft}{\kern0pt}f\ {\isacharminus}{\kern0pt}{\isacharbackquote}{\kern0pt}\ {\isacharbraceleft}{\kern0pt}x{\isacharbraceright}{\kern0pt}\ {\isasyminter}\ space\ M{\isacharparenright}{\kern0pt}\ y\ {\isacharasterisk}{\kern0pt}\isactrlsub R\ x\ {\isasymnoteq}\ {\isadigit{0}}{\isacharbraceright}{\kern0pt}\ {\isasymnoteq}\ {\isasyminfinity}{\isachardoublequoteclose}\ \isakeyword{if}\ {\isachardoublequoteopen}x\ {\isasymnoteq}\ {\isadigit{0}}{\isachardoublequoteclose}\ \isacommand{by}\isamarkupfalse%
\ {\isacharparenleft}{\kern0pt}metis\ {\isacharparenleft}{\kern0pt}mono{\isacharunderscore}{\kern0pt}tags{\isacharcomma}{\kern0pt}\ lifting{\isacharparenright}{\kern0pt}\ emeasure{\isacharunderscore}{\kern0pt}mono\ f{\isacharparenleft}{\kern0pt}{\isadigit{1}}{\isacharparenright}{\kern0pt}\ indicator{\isacharunderscore}{\kern0pt}simps{\isacharparenleft}{\kern0pt}{\isadigit{2}}{\isacharparenright}{\kern0pt}\ linorder{\isacharunderscore}{\kern0pt}not{\isacharunderscore}{\kern0pt}less\ mem{\isacharunderscore}{\kern0pt}Collect{\isacharunderscore}{\kern0pt}eq\ scaleR{\isacharunderscore}{\kern0pt}eq{\isacharunderscore}{\kern0pt}{\isadigit{0}}{\isacharunderscore}{\kern0pt}iff\ simple{\isacharunderscore}{\kern0pt}functionD{\isacharparenleft}{\kern0pt}{\isadigit{2}}{\isacharparenright}{\kern0pt}\ subsetI\ that{\isacharparenright}{\kern0pt}\isanewline
\isanewline
\ \ \ \ \isacommand{have}\isamarkupfalse%
\ nonneg{\isacharunderscore}{\kern0pt}x{\isacharcolon}{\kern0pt}\ {\isachardoublequoteopen}x\ {\isasymge}\ {\isadigit{0}}{\isachardoublequoteclose}\ \isacommand{using}\isamarkupfalse%
\ insert\ f\ \isacommand{by}\isamarkupfalse%
\ blast\isanewline
\ \ \ \ \isacommand{have}\isamarkupfalse%
\ {\isacharasterisk}{\kern0pt}{\isacharcolon}{\kern0pt}\ {\isachardoublequoteopen}{\isacharparenleft}{\kern0pt}{\isasymSum}y{\isasymin}insert\ x\ F{\isachardot}{\kern0pt}\ indicat{\isacharunderscore}{\kern0pt}real\ {\isacharparenleft}{\kern0pt}f\ {\isacharminus}{\kern0pt}{\isacharbackquote}{\kern0pt}\ {\isacharbraceleft}{\kern0pt}y{\isacharbraceright}{\kern0pt}\ {\isasyminter}\ space\ M{\isacharparenright}{\kern0pt}\ xa\ {\isacharasterisk}{\kern0pt}\isactrlsub R\ y{\isacharparenright}{\kern0pt}\ {\isacharequal}{\kern0pt}\ {\isacharparenleft}{\kern0pt}{\isasymSum}y{\isasymin}F{\isachardot}{\kern0pt}\ indicat{\isacharunderscore}{\kern0pt}real\ {\isacharparenleft}{\kern0pt}f\ {\isacharminus}{\kern0pt}{\isacharbackquote}{\kern0pt}\ {\isacharbraceleft}{\kern0pt}y{\isacharbraceright}{\kern0pt}\ {\isasyminter}\ space\ M{\isacharparenright}{\kern0pt}\ xa\ {\isacharasterisk}{\kern0pt}\isactrlsub R\ y{\isacharparenright}{\kern0pt}\ {\isacharplus}{\kern0pt}\ indicat{\isacharunderscore}{\kern0pt}real\ {\isacharparenleft}{\kern0pt}f\ {\isacharminus}{\kern0pt}{\isacharbackquote}{\kern0pt}\ {\isacharbraceleft}{\kern0pt}x{\isacharbraceright}{\kern0pt}\ {\isasyminter}\ space\ M{\isacharparenright}{\kern0pt}\ xa\ {\isacharasterisk}{\kern0pt}\isactrlsub R\ x{\isachardoublequoteclose}\ \isakeyword{for}\ xa\ \isacommand{by}\isamarkupfalse%
\ {\isacharparenleft}{\kern0pt}metis\ {\isacharparenleft}{\kern0pt}no{\isacharunderscore}{\kern0pt}types{\isacharcomma}{\kern0pt}\ lifting{\isacharparenright}{\kern0pt}\ add{\isachardot}{\kern0pt}commute\ insert{\isachardot}{\kern0pt}hyps{\isacharparenleft}{\kern0pt}{\isadigit{1}}{\isacharparenright}{\kern0pt}\ insert{\isachardot}{\kern0pt}hyps{\isacharparenleft}{\kern0pt}{\isadigit{4}}{\isacharparenright}{\kern0pt}\ sum{\isachardot}{\kern0pt}insert{\isacharparenright}{\kern0pt}\isanewline
\ \ \ \ \isacommand{have}\isamarkupfalse%
\ {\isacharasterisk}{\kern0pt}{\isacharasterisk}{\kern0pt}{\isacharcolon}{\kern0pt}\ {\isachardoublequoteopen}{\isacharbraceleft}{\kern0pt}y\ {\isasymin}\ space\ M{\isachardot}{\kern0pt}\ {\isacharparenleft}{\kern0pt}{\isasymSum}x{\isasymin}insert\ x\ F{\isachardot}{\kern0pt}\ indicat{\isacharunderscore}{\kern0pt}real\ {\isacharparenleft}{\kern0pt}f\ {\isacharminus}{\kern0pt}{\isacharbackquote}{\kern0pt}\ {\isacharbraceleft}{\kern0pt}x{\isacharbraceright}{\kern0pt}\ {\isasyminter}\ space\ M{\isacharparenright}{\kern0pt}\ y\ {\isacharasterisk}{\kern0pt}\isactrlsub R\ x{\isacharparenright}{\kern0pt}\ {\isasymnoteq}\ {\isadigit{0}}{\isacharbraceright}{\kern0pt}\ {\isasymsubseteq}\ {\isacharbraceleft}{\kern0pt}y\ {\isasymin}\ space\ M{\isachardot}{\kern0pt}\ {\isacharparenleft}{\kern0pt}{\isasymSum}x{\isasymin}F{\isachardot}{\kern0pt}\ indicat{\isacharunderscore}{\kern0pt}real\ {\isacharparenleft}{\kern0pt}f\ {\isacharminus}{\kern0pt}{\isacharbackquote}{\kern0pt}\ {\isacharbraceleft}{\kern0pt}x{\isacharbraceright}{\kern0pt}\ {\isasyminter}\ space\ M{\isacharparenright}{\kern0pt}\ y\ {\isacharasterisk}{\kern0pt}\isactrlsub R\ x{\isacharparenright}{\kern0pt}\ {\isasymnoteq}\ {\isadigit{0}}{\isacharbraceright}{\kern0pt}\ {\isasymunion}\ {\isacharbraceleft}{\kern0pt}y\ {\isasymin}\ space\ M{\isachardot}{\kern0pt}\ indicat{\isacharunderscore}{\kern0pt}real\ {\isacharparenleft}{\kern0pt}f\ {\isacharminus}{\kern0pt}{\isacharbackquote}{\kern0pt}\ {\isacharbraceleft}{\kern0pt}x{\isacharbraceright}{\kern0pt}\ {\isasyminter}\ space\ M{\isacharparenright}{\kern0pt}\ y\ {\isacharasterisk}{\kern0pt}\isactrlsub R\ x\ {\isasymnoteq}\ {\isadigit{0}}{\isacharbraceright}{\kern0pt}{\isachardoublequoteclose}\ \isacommand{unfolding}\isamarkupfalse%
\ {\isacharasterisk}{\kern0pt}\ \isacommand{by}\isamarkupfalse%
\ fastforce\ \ \ \ \isanewline
\ \ \ \ \isacommand{{\isacharbraceleft}{\kern0pt}}\isamarkupfalse%
\isanewline
\ \ \ \ \ \ \isacommand{case}\isamarkupfalse%
\ {\isadigit{1}}\isanewline
\ \ \ \ \ \ \isacommand{show}\isamarkupfalse%
\ {\isacharquery}{\kern0pt}case\ \isanewline
\ \ \ \ \ \ \isacommand{proof}\isamarkupfalse%
\ {\isacharparenleft}{\kern0pt}cases\ {\isachardoublequoteopen}x\ {\isacharequal}{\kern0pt}\ {\isadigit{0}}{\isachardoublequoteclose}{\isacharparenright}{\kern0pt}\isanewline
\ \ \ \ \ \ \ \ \isacommand{case}\isamarkupfalse%
\ True\isanewline
\ \ \ \ \ \ \ \ \isacommand{then}\isamarkupfalse%
\ \isacommand{show}\isamarkupfalse%
\ {\isacharquery}{\kern0pt}thesis\ \isacommand{unfolding}\isamarkupfalse%
\ {\isacharasterisk}{\kern0pt}\ \isacommand{using}\isamarkupfalse%
\ insert\ \isacommand{by}\isamarkupfalse%
\ simp\isanewline
\ \ \ \ \ \ \isacommand{next}\isamarkupfalse%
\isanewline
\ \ \ \ \ \ \ \ \isacommand{case}\isamarkupfalse%
\ False\isanewline
\ \ \ \ \ \ \ \ \isacommand{have}\isamarkupfalse%
\ norm{\isacharunderscore}{\kern0pt}argument{\isacharcolon}{\kern0pt}\ {\isachardoublequoteopen}norm\ {\isacharparenleft}{\kern0pt}{\isacharparenleft}{\kern0pt}{\isasymSum}y{\isasymin}F{\isachardot}{\kern0pt}\ indicat{\isacharunderscore}{\kern0pt}real\ {\isacharparenleft}{\kern0pt}f\ {\isacharminus}{\kern0pt}{\isacharbackquote}{\kern0pt}\ {\isacharbraceleft}{\kern0pt}y{\isacharbraceright}{\kern0pt}\ {\isasyminter}\ space\ M{\isacharparenright}{\kern0pt}\ z\ {\isacharasterisk}{\kern0pt}\isactrlsub R\ y{\isacharparenright}{\kern0pt}\ {\isacharplus}{\kern0pt}\ indicat{\isacharunderscore}{\kern0pt}real\ {\isacharparenleft}{\kern0pt}f\ {\isacharminus}{\kern0pt}{\isacharbackquote}{\kern0pt}\ {\isacharbraceleft}{\kern0pt}x{\isacharbraceright}{\kern0pt}\ {\isasyminter}\ space\ M{\isacharparenright}{\kern0pt}\ z\ {\isacharasterisk}{\kern0pt}\isactrlsub R\ x{\isacharparenright}{\kern0pt}\ {\isacharequal}{\kern0pt}\ norm\ {\isacharparenleft}{\kern0pt}{\isasymSum}y{\isasymin}F{\isachardot}{\kern0pt}\ indicat{\isacharunderscore}{\kern0pt}real\ {\isacharparenleft}{\kern0pt}f\ {\isacharminus}{\kern0pt}{\isacharbackquote}{\kern0pt}\ {\isacharbraceleft}{\kern0pt}y{\isacharbraceright}{\kern0pt}\ {\isasyminter}\ space\ M{\isacharparenright}{\kern0pt}\ z\ {\isacharasterisk}{\kern0pt}\isactrlsub R\ y{\isacharparenright}{\kern0pt}\ {\isacharplus}{\kern0pt}\ norm\ {\isacharparenleft}{\kern0pt}indicat{\isacharunderscore}{\kern0pt}real\ {\isacharparenleft}{\kern0pt}f\ {\isacharminus}{\kern0pt}{\isacharbackquote}{\kern0pt}\ {\isacharbraceleft}{\kern0pt}x{\isacharbraceright}{\kern0pt}\ {\isasyminter}\ space\ M{\isacharparenright}{\kern0pt}\ z\ {\isacharasterisk}{\kern0pt}\isactrlsub R\ x{\isacharparenright}{\kern0pt}{\isachardoublequoteclose}\ \isakeyword{if}\ z{\isacharcolon}{\kern0pt}\ {\isachardoublequoteopen}z\ {\isasymin}\ space\ M{\isachardoublequoteclose}\ \isakeyword{for}\ z\isanewline
\ \ \ \ \ \ \ \ \isacommand{proof}\isamarkupfalse%
\ {\isacharparenleft}{\kern0pt}cases\ {\isachardoublequoteopen}f\ z\ {\isacharequal}{\kern0pt}\ x{\isachardoublequoteclose}{\isacharparenright}{\kern0pt}\isanewline
\ \ \ \ \ \ \ \ \ \ \isacommand{case}\isamarkupfalse%
\ True\isanewline
\ \ \ \ \ \ \ \ \ \ \isacommand{have}\isamarkupfalse%
\ {\isachardoublequoteopen}indicat{\isacharunderscore}{\kern0pt}real\ {\isacharparenleft}{\kern0pt}f\ {\isacharminus}{\kern0pt}{\isacharbackquote}{\kern0pt}\ {\isacharbraceleft}{\kern0pt}y{\isacharbraceright}{\kern0pt}\ {\isasyminter}\ space\ M{\isacharparenright}{\kern0pt}\ z\ {\isacharasterisk}{\kern0pt}\isactrlsub R\ y\ {\isacharequal}{\kern0pt}\ {\isadigit{0}}{\isachardoublequoteclose}\ \isakeyword{if}\ {\isachardoublequoteopen}y\ {\isasymin}\ F{\isachardoublequoteclose}\ \isakeyword{for}\ y\ \isacommand{using}\isamarkupfalse%
\ True\ insert\ z\ that\ {\isadigit{1}}\ \isacommand{unfolding}\isamarkupfalse%
\ indicator{\isacharunderscore}{\kern0pt}def\ \isacommand{by}\isamarkupfalse%
\ force\isanewline
\ \ \ \ \ \ \ \ \ \ \isacommand{hence}\isamarkupfalse%
\ {\isachardoublequoteopen}{\isacharparenleft}{\kern0pt}{\isasymSum}y{\isasymin}F{\isachardot}{\kern0pt}\ indicat{\isacharunderscore}{\kern0pt}real\ {\isacharparenleft}{\kern0pt}f\ {\isacharminus}{\kern0pt}{\isacharbackquote}{\kern0pt}\ {\isacharbraceleft}{\kern0pt}y{\isacharbraceright}{\kern0pt}\ {\isasyminter}\ space\ M{\isacharparenright}{\kern0pt}\ z\ {\isacharasterisk}{\kern0pt}\isactrlsub R\ y{\isacharparenright}{\kern0pt}\ {\isacharequal}{\kern0pt}\ {\isadigit{0}}{\isachardoublequoteclose}\ \isacommand{by}\isamarkupfalse%
\ {\isacharparenleft}{\kern0pt}meson\ sum{\isachardot}{\kern0pt}neutral{\isacharparenright}{\kern0pt}\isanewline
\ \ \ \ \ \ \ \ \ \ \isacommand{thus}\isamarkupfalse%
\ {\isacharquery}{\kern0pt}thesis\ \isacommand{by}\isamarkupfalse%
\ force\isanewline
\ \ \ \ \ \ \ \ \isacommand{qed}\isamarkupfalse%
\ {\isacharparenleft}{\kern0pt}force{\isacharparenright}{\kern0pt}\isanewline
\ \ \ \ \ \ \ \ \isacommand{show}\isamarkupfalse%
\ {\isacharquery}{\kern0pt}thesis\ \isacommand{using}\isamarkupfalse%
\ False\ fin{\isacharunderscore}{\kern0pt}{\isadigit{0}}\ fin{\isacharunderscore}{\kern0pt}{\isadigit{1}}\ f\ norm{\isacharunderscore}{\kern0pt}argument\ \isacommand{by}\isamarkupfalse%
\ {\isacharparenleft}{\kern0pt}subst\ {\isacharasterisk}{\kern0pt}{\isacharcomma}{\kern0pt}\ subst\ add{\isacharcomma}{\kern0pt}\ presburger\ add{\isacharcolon}{\kern0pt}\ insert{\isacharcomma}{\kern0pt}\ intro\ insert{\isacharcomma}{\kern0pt}\ intro\ insert{\isacharcomma}{\kern0pt}\ fastforce\ simp\ add{\isacharcolon}{\kern0pt}\ indicator{\isacharunderscore}{\kern0pt}def\ intro{\isacharbang}{\kern0pt}{\isacharcolon}{\kern0pt}\ insert{\isacharparenleft}{\kern0pt}{\isadigit{2}}{\isacharparenright}{\kern0pt}\ f{\isacharparenleft}{\kern0pt}{\isadigit{3}}{\isacharparenright}{\kern0pt}{\isacharcomma}{\kern0pt}\ auto\ intro{\isacharbang}{\kern0pt}{\isacharcolon}{\kern0pt}\ indicator\ insert\ nonneg{\isacharunderscore}{\kern0pt}x{\isacharparenright}{\kern0pt}\isanewline
\ \ \ \ \ \ \isacommand{qed}\isamarkupfalse%
\ \isanewline
\ \ \ \ \isacommand{next}\isamarkupfalse%
\isanewline
\ \ \ \ \ \ \isacommand{case}\isamarkupfalse%
\ {\isadigit{2}}\isanewline
\ \ \ \ \ \ \isacommand{show}\isamarkupfalse%
\ {\isacharquery}{\kern0pt}case\ \isanewline
\ \ \ \ \ \ \isacommand{proof}\isamarkupfalse%
\ {\isacharparenleft}{\kern0pt}cases\ {\isachardoublequoteopen}x\ {\isacharequal}{\kern0pt}\ {\isadigit{0}}{\isachardoublequoteclose}{\isacharparenright}{\kern0pt}\isanewline
\ \ \ \ \ \ \ \ \isacommand{case}\isamarkupfalse%
\ True\isanewline
\ \ \ \ \ \ \ \ \isacommand{then}\isamarkupfalse%
\ \isacommand{show}\isamarkupfalse%
\ {\isacharquery}{\kern0pt}thesis\ \isacommand{unfolding}\isamarkupfalse%
\ {\isacharasterisk}{\kern0pt}\ \isacommand{using}\isamarkupfalse%
\ insert\ \isacommand{by}\isamarkupfalse%
\ simp\isanewline
\ \ \ \ \ \ \isacommand{next}\isamarkupfalse%
\isanewline
\ \ \ \ \ \ \ \ \isacommand{case}\isamarkupfalse%
\ False\isanewline
\ \ \ \ \ \ \ \ \isacommand{then}\isamarkupfalse%
\ \isacommand{show}\isamarkupfalse%
\ {\isacharquery}{\kern0pt}thesis\ \isacommand{unfolding}\isamarkupfalse%
\ {\isacharasterisk}{\kern0pt}\ \isacommand{using}\isamarkupfalse%
\ insert\ simple{\isacharunderscore}{\kern0pt}functionD{\isacharparenleft}{\kern0pt}{\isadigit{2}}{\isacharparenright}{\kern0pt}{\isacharbrackleft}{\kern0pt}OF\ f{\isacharparenleft}{\kern0pt}{\isadigit{1}}{\isacharparenright}{\kern0pt}{\isacharbrackright}{\kern0pt}\ \isacommand{by}\isamarkupfalse%
\ fast\isanewline
\ \ \ \ \ \ \isacommand{qed}\isamarkupfalse%
\isanewline
\ \ \ \ \isacommand{next}\isamarkupfalse%
\isanewline
\ \ \ \ \ \ \isacommand{case}\isamarkupfalse%
\ {\isadigit{3}}\isanewline
\ \ \ \ \ \ \isacommand{show}\isamarkupfalse%
\ {\isacharquery}{\kern0pt}case\ \isanewline
\ \ \ \ \ \ \isacommand{proof}\isamarkupfalse%
\ {\isacharparenleft}{\kern0pt}cases\ {\isachardoublequoteopen}x\ {\isacharequal}{\kern0pt}\ {\isadigit{0}}{\isachardoublequoteclose}{\isacharparenright}{\kern0pt}\isanewline
\ \ \ \ \ \ \ \ \isacommand{case}\isamarkupfalse%
\ True\isanewline
\ \ \ \ \ \ \ \ \isacommand{then}\isamarkupfalse%
\ \isacommand{show}\isamarkupfalse%
\ {\isacharquery}{\kern0pt}thesis\ \isacommand{unfolding}\isamarkupfalse%
\ {\isacharasterisk}{\kern0pt}\ \isacommand{using}\isamarkupfalse%
\ insert\ \isacommand{by}\isamarkupfalse%
\ simp\isanewline
\ \ \ \ \ \ \isacommand{next}\isamarkupfalse%
\isanewline
\ \ \ \ \ \ \ \ \isacommand{case}\isamarkupfalse%
\ False\isanewline
\ \ \ \ \ \ \ \ \isacommand{have}\isamarkupfalse%
\ {\isachardoublequoteopen}emeasure\ M\ {\isacharbraceleft}{\kern0pt}y\ {\isasymin}\ space\ M{\isachardot}{\kern0pt}\ {\isacharparenleft}{\kern0pt}{\isasymSum}x{\isasymin}insert\ x\ F{\isachardot}{\kern0pt}\ indicat{\isacharunderscore}{\kern0pt}real\ {\isacharparenleft}{\kern0pt}f\ {\isacharminus}{\kern0pt}{\isacharbackquote}{\kern0pt}\ {\isacharbraceleft}{\kern0pt}x{\isacharbraceright}{\kern0pt}\ {\isasyminter}\ space\ M{\isacharparenright}{\kern0pt}\ y\ {\isacharasterisk}{\kern0pt}\isactrlsub R\ x{\isacharparenright}{\kern0pt}\ {\isasymnoteq}\ {\isadigit{0}}{\isacharbraceright}{\kern0pt}\ {\isasymle}\ emeasure\ M\ {\isacharparenleft}{\kern0pt}{\isacharbraceleft}{\kern0pt}y\ {\isasymin}\ space\ M{\isachardot}{\kern0pt}\ {\isacharparenleft}{\kern0pt}{\isasymSum}x{\isasymin}F{\isachardot}{\kern0pt}\ indicat{\isacharunderscore}{\kern0pt}real\ {\isacharparenleft}{\kern0pt}f\ {\isacharminus}{\kern0pt}{\isacharbackquote}{\kern0pt}\ {\isacharbraceleft}{\kern0pt}x{\isacharbraceright}{\kern0pt}\ {\isasyminter}\ space\ M{\isacharparenright}{\kern0pt}\ y\ {\isacharasterisk}{\kern0pt}\isactrlsub R\ x{\isacharparenright}{\kern0pt}\ {\isasymnoteq}\ {\isadigit{0}}{\isacharbraceright}{\kern0pt}\ {\isasymunion}\ {\isacharbraceleft}{\kern0pt}y\ {\isasymin}\ space\ M{\isachardot}{\kern0pt}\ indicat{\isacharunderscore}{\kern0pt}real\ {\isacharparenleft}{\kern0pt}f\ {\isacharminus}{\kern0pt}{\isacharbackquote}{\kern0pt}\ {\isacharbraceleft}{\kern0pt}x{\isacharbraceright}{\kern0pt}\ {\isasyminter}\ space\ M{\isacharparenright}{\kern0pt}\ y\ {\isacharasterisk}{\kern0pt}\isactrlsub R\ x\ {\isasymnoteq}\ {\isadigit{0}}{\isacharbraceright}{\kern0pt}{\isacharparenright}{\kern0pt}{\isachardoublequoteclose}\isanewline
\ \ \ \ \ \ \ \ \ \ \isacommand{using}\isamarkupfalse%
\ {\isacharasterisk}{\kern0pt}{\isacharasterisk}{\kern0pt}\ simple{\isacharunderscore}{\kern0pt}functionD{\isacharparenleft}{\kern0pt}{\isadigit{2}}{\isacharparenright}{\kern0pt}{\isacharbrackleft}{\kern0pt}OF\ insert{\isacharparenleft}{\kern0pt}{\isadigit{6}}{\isacharparenright}{\kern0pt}{\isacharbrackright}{\kern0pt}\ simple{\isacharunderscore}{\kern0pt}functionD{\isacharparenleft}{\kern0pt}{\isadigit{2}}{\isacharparenright}{\kern0pt}{\isacharbrackleft}{\kern0pt}OF\ f{\isacharparenleft}{\kern0pt}{\isadigit{1}}{\isacharparenright}{\kern0pt}{\isacharbrackright}{\kern0pt}\ insert{\isachardot}{\kern0pt}IH{\isacharparenleft}{\kern0pt}{\isadigit{2}}{\isacharparenright}{\kern0pt}\ \isacommand{by}\isamarkupfalse%
\ {\isacharparenleft}{\kern0pt}intro\ emeasure{\isacharunderscore}{\kern0pt}mono{\isacharcomma}{\kern0pt}\ blast{\isacharcomma}{\kern0pt}\ simp{\isacharparenright}{\kern0pt}\ \isanewline
\ \ \ \ \ \ \ \ \isacommand{also}\isamarkupfalse%
\ \isacommand{have}\isamarkupfalse%
\ {\isachardoublequoteopen}{\isachardot}{\kern0pt}{\isachardot}{\kern0pt}{\isachardot}{\kern0pt}\ {\isasymle}\ emeasure\ M\ {\isacharbraceleft}{\kern0pt}y\ {\isasymin}\ space\ M{\isachardot}{\kern0pt}\ {\isacharparenleft}{\kern0pt}{\isasymSum}x{\isasymin}F{\isachardot}{\kern0pt}\ indicat{\isacharunderscore}{\kern0pt}real\ {\isacharparenleft}{\kern0pt}f\ {\isacharminus}{\kern0pt}{\isacharbackquote}{\kern0pt}\ {\isacharbraceleft}{\kern0pt}x{\isacharbraceright}{\kern0pt}\ {\isasyminter}\ space\ M{\isacharparenright}{\kern0pt}\ y\ {\isacharasterisk}{\kern0pt}\isactrlsub R\ x{\isacharparenright}{\kern0pt}\ {\isasymnoteq}\ {\isadigit{0}}{\isacharbraceright}{\kern0pt}\ {\isacharplus}{\kern0pt}\ emeasure\ M\ {\isacharbraceleft}{\kern0pt}y\ {\isasymin}\ space\ M{\isachardot}{\kern0pt}\ indicat{\isacharunderscore}{\kern0pt}real\ {\isacharparenleft}{\kern0pt}f\ {\isacharminus}{\kern0pt}{\isacharbackquote}{\kern0pt}\ {\isacharbraceleft}{\kern0pt}x{\isacharbraceright}{\kern0pt}\ {\isasyminter}\ space\ M{\isacharparenright}{\kern0pt}\ y\ {\isacharasterisk}{\kern0pt}\isactrlsub R\ x\ {\isasymnoteq}\ {\isadigit{0}}{\isacharbraceright}{\kern0pt}{\isachardoublequoteclose}\isanewline
\ \ \ \ \ \ \ \ \ \ \isacommand{using}\isamarkupfalse%
\ simple{\isacharunderscore}{\kern0pt}functionD{\isacharparenleft}{\kern0pt}{\isadigit{2}}{\isacharparenright}{\kern0pt}{\isacharbrackleft}{\kern0pt}OF\ insert{\isacharparenleft}{\kern0pt}{\isadigit{6}}{\isacharparenright}{\kern0pt}{\isacharbrackright}{\kern0pt}\ simple{\isacharunderscore}{\kern0pt}functionD{\isacharparenleft}{\kern0pt}{\isadigit{2}}{\isacharparenright}{\kern0pt}{\isacharbrackleft}{\kern0pt}OF\ f{\isacharparenleft}{\kern0pt}{\isadigit{1}}{\isacharparenright}{\kern0pt}{\isacharbrackright}{\kern0pt}\ \isacommand{by}\isamarkupfalse%
\ {\isacharparenleft}{\kern0pt}intro\ emeasure{\isacharunderscore}{\kern0pt}subadditive{\isacharcomma}{\kern0pt}\ force{\isacharplus}{\kern0pt}{\isacharparenright}{\kern0pt}\isanewline
\ \ \ \ \ \ \ \ \isacommand{also}\isamarkupfalse%
\ \isacommand{have}\isamarkupfalse%
\ {\isachardoublequoteopen}{\isachardot}{\kern0pt}{\isachardot}{\kern0pt}{\isachardot}{\kern0pt}\ {\isacharless}{\kern0pt}\ {\isasyminfinity}{\isachardoublequoteclose}\ \isacommand{using}\isamarkupfalse%
\ insert{\isacharparenleft}{\kern0pt}{\isadigit{7}}{\isacharparenright}{\kern0pt}\ fin{\isacharunderscore}{\kern0pt}{\isadigit{1}}{\isacharbrackleft}{\kern0pt}OF\ False{\isacharbrackright}{\kern0pt}\ \isacommand{by}\isamarkupfalse%
\ {\isacharparenleft}{\kern0pt}simp\ add{\isacharcolon}{\kern0pt}\ less{\isacharunderscore}{\kern0pt}top{\isacharparenright}{\kern0pt}\isanewline
\ \ \ \ \ \ \ \ \isacommand{finally}\isamarkupfalse%
\ \isacommand{show}\isamarkupfalse%
\ {\isacharquery}{\kern0pt}thesis\ \isacommand{by}\isamarkupfalse%
\ simp\isanewline
\ \ \ \ \ \ \isacommand{qed}\isamarkupfalse%
\isanewline
\ \ \ \ \isacommand{next}\isamarkupfalse%
\isanewline
\ \ \ \ \ \ \isacommand{case}\isamarkupfalse%
\ {\isacharparenleft}{\kern0pt}{\isadigit{4}}\ {\isasymxi}{\isacharparenright}{\kern0pt}\isanewline
\ \ \ \ \ \ \isacommand{thus}\isamarkupfalse%
\ {\isacharquery}{\kern0pt}case\ \isacommand{using}\isamarkupfalse%
\ insert\ nonneg{\isacharunderscore}{\kern0pt}x\ f{\isacharparenleft}{\kern0pt}{\isadigit{3}}{\isacharparenright}{\kern0pt}\ \isacommand{by}\isamarkupfalse%
\ {\isacharparenleft}{\kern0pt}auto\ simp\ add{\isacharcolon}{\kern0pt}\ scaleR{\isacharunderscore}{\kern0pt}nonneg{\isacharunderscore}{\kern0pt}nonneg\ intro{\isacharcolon}{\kern0pt}\ sum{\isacharunderscore}{\kern0pt}nonneg{\isacharparenright}{\kern0pt}\isanewline
\ \ \ \ \isacommand{{\isacharbraceright}{\kern0pt}}\isamarkupfalse%
\isanewline
\ \ \isacommand{qed}\isamarkupfalse%
\isanewline
\ \ \isacommand{moreover}\isamarkupfalse%
\ \isacommand{have}\isamarkupfalse%
\ {\isachardoublequoteopen}simple{\isacharunderscore}{\kern0pt}function\ M\ {\isacharparenleft}{\kern0pt}{\isasymlambda}x{\isachardot}{\kern0pt}\ {\isasymSum}y{\isasymin}f\ {\isacharbackquote}{\kern0pt}\ space\ M{\isachardot}{\kern0pt}\ indicat{\isacharunderscore}{\kern0pt}real\ {\isacharparenleft}{\kern0pt}f\ {\isacharminus}{\kern0pt}{\isacharbackquote}{\kern0pt}\ {\isacharbraceleft}{\kern0pt}y{\isacharbraceright}{\kern0pt}\ {\isasyminter}\ space\ M{\isacharparenright}{\kern0pt}\ x\ {\isacharasterisk}{\kern0pt}\isactrlsub R\ y{\isacharparenright}{\kern0pt}{\isachardoublequoteclose}\ \isacommand{using}\isamarkupfalse%
\ calculation\ \isacommand{by}\isamarkupfalse%
\ blast\isanewline
\ \ \isacommand{moreover}\isamarkupfalse%
\ \isacommand{have}\isamarkupfalse%
\ {\isachardoublequoteopen}P\ {\isacharparenleft}{\kern0pt}{\isasymlambda}x{\isachardot}{\kern0pt}\ {\isasymSum}y{\isasymin}f\ {\isacharbackquote}{\kern0pt}\ space\ M{\isachardot}{\kern0pt}\ indicat{\isacharunderscore}{\kern0pt}real\ {\isacharparenleft}{\kern0pt}f\ {\isacharminus}{\kern0pt}{\isacharbackquote}{\kern0pt}\ {\isacharbraceleft}{\kern0pt}y{\isacharbraceright}{\kern0pt}\ {\isasyminter}\ space\ M{\isacharparenright}{\kern0pt}\ x\ {\isacharasterisk}{\kern0pt}\isactrlsub R\ y{\isacharparenright}{\kern0pt}{\isachardoublequoteclose}\ \isacommand{using}\isamarkupfalse%
\ calculation\ \isacommand{by}\isamarkupfalse%
\ blast\isanewline
\ \ \isacommand{moreover}\isamarkupfalse%
\ \isacommand{have}\isamarkupfalse%
\ {\isachardoublequoteopen}{\isasymAnd}x{\isachardot}{\kern0pt}\ x\ {\isasymin}\ space\ M\ {\isasymLongrightarrow}\ {\isadigit{0}}\ {\isasymle}\ f\ x{\isachardoublequoteclose}\ \isacommand{using}\isamarkupfalse%
\ f{\isacharparenleft}{\kern0pt}{\isadigit{3}}{\isacharparenright}{\kern0pt}\ \isacommand{by}\isamarkupfalse%
\ simp\isanewline
\ \ \isacommand{ultimately}\isamarkupfalse%
\ \isacommand{show}\isamarkupfalse%
\ {\isacharquery}{\kern0pt}thesis\ \isacommand{by}\isamarkupfalse%
\ {\isacharparenleft}{\kern0pt}intro\ cong{\isacharbrackleft}{\kern0pt}OF\ {\isacharunderscore}{\kern0pt}\ {\isacharunderscore}{\kern0pt}\ {\isacharunderscore}{\kern0pt}\ f{\isacharparenleft}{\kern0pt}{\isadigit{1}}{\isacharcomma}{\kern0pt}{\isadigit{2}}{\isacharparenright}{\kern0pt}{\isacharbrackright}{\kern0pt}{\isacharcomma}{\kern0pt}\ blast{\isacharcomma}{\kern0pt}\ blast{\isacharcomma}{\kern0pt}\ fast{\isacharparenright}{\kern0pt}\ presburger{\isacharplus}{\kern0pt}\isanewline
\isacommand{qed}\isamarkupfalse%
%
\endisatagproof
{\isafoldproof}%
%
\isadelimproof
\isanewline
%
\endisadelimproof
\isanewline
\isacommand{lemma}\isamarkupfalse%
\ finite{\isacharunderscore}{\kern0pt}nn{\isacharunderscore}{\kern0pt}integral{\isacharunderscore}{\kern0pt}imp{\isacharunderscore}{\kern0pt}ae{\isacharunderscore}{\kern0pt}finite{\isacharcolon}{\kern0pt}\isanewline
\ \ \isakeyword{fixes}\ f\ {\isacharcolon}{\kern0pt}{\isacharcolon}{\kern0pt}\ {\isachardoublequoteopen}{\isacharprime}{\kern0pt}a\ {\isasymRightarrow}\ ennreal{\isachardoublequoteclose}\isanewline
\ \ \isakeyword{assumes}\ {\isachardoublequoteopen}f\ {\isasymin}\ borel{\isacharunderscore}{\kern0pt}measurable\ M{\isachardoublequoteclose}\ {\isachardoublequoteopen}{\isacharparenleft}{\kern0pt}{\isasymintegral}\isactrlsup {\isacharplus}{\kern0pt}x{\isachardot}{\kern0pt}\ f\ x\ {\isasympartial}M{\isacharparenright}{\kern0pt}\ {\isacharless}{\kern0pt}\ {\isasyminfinity}{\isachardoublequoteclose}\isanewline
\ \ \isakeyword{shows}\ {\isachardoublequoteopen}AE\ x\ in\ M{\isachardot}{\kern0pt}\ f\ x\ {\isacharless}{\kern0pt}\ {\isasyminfinity}{\isachardoublequoteclose}\isanewline
%
\isadelimproof
%
\endisadelimproof
%
\isatagproof
\isacommand{proof}\isamarkupfalse%
\ {\isacharparenleft}{\kern0pt}rule\ ccontr{\isacharcomma}{\kern0pt}\ goal{\isacharunderscore}{\kern0pt}cases{\isacharparenright}{\kern0pt}\isanewline
\ \ \isacommand{case}\isamarkupfalse%
\ {\isadigit{1}}\isanewline
\ \ \isacommand{let}\isamarkupfalse%
\ {\isacharquery}{\kern0pt}A\ {\isacharequal}{\kern0pt}\ {\isachardoublequoteopen}space\ M\ {\isasyminter}\ {\isacharbraceleft}{\kern0pt}x{\isachardot}{\kern0pt}\ f\ x\ {\isacharequal}{\kern0pt}\ {\isasyminfinity}{\isacharbraceright}{\kern0pt}{\isachardoublequoteclose}\isanewline
\ \ \isacommand{have}\isamarkupfalse%
\ {\isacharasterisk}{\kern0pt}{\isacharcolon}{\kern0pt}\ {\isachardoublequoteopen}emeasure\ M\ {\isacharquery}{\kern0pt}A\ {\isachargreater}{\kern0pt}\ {\isadigit{0}}{\isachardoublequoteclose}\ \isacommand{using}\isamarkupfalse%
\ {\isadigit{1}}\ assms{\isacharparenleft}{\kern0pt}{\isadigit{1}}{\isacharparenright}{\kern0pt}\ \isacommand{by}\isamarkupfalse%
\ {\isacharparenleft}{\kern0pt}metis\ {\isacharparenleft}{\kern0pt}mono{\isacharunderscore}{\kern0pt}tags{\isacharcomma}{\kern0pt}\ lifting{\isacharparenright}{\kern0pt}\ assms{\isacharparenleft}{\kern0pt}{\isadigit{2}}{\isacharparenright}{\kern0pt}\ eventually{\isacharunderscore}{\kern0pt}mono\ infinity{\isacharunderscore}{\kern0pt}ennreal{\isacharunderscore}{\kern0pt}def\ nn{\isacharunderscore}{\kern0pt}integral{\isacharunderscore}{\kern0pt}noteq{\isacharunderscore}{\kern0pt}infinite\ top{\isachardot}{\kern0pt}not{\isacharunderscore}{\kern0pt}eq{\isacharunderscore}{\kern0pt}extremum{\isacharparenright}{\kern0pt}\isanewline
\ \ \isacommand{have}\isamarkupfalse%
\ {\isachardoublequoteopen}{\isacharparenleft}{\kern0pt}{\isasymintegral}\isactrlsup {\isacharplus}{\kern0pt}x{\isachardot}{\kern0pt}\ f\ x\ {\isacharasterisk}{\kern0pt}\ indicator\ {\isacharquery}{\kern0pt}A\ x\ {\isasympartial}M{\isacharparenright}{\kern0pt}\ {\isacharequal}{\kern0pt}\ {\isacharparenleft}{\kern0pt}{\isasymintegral}\isactrlsup {\isacharplus}{\kern0pt}x{\isachardot}{\kern0pt}\ {\isasyminfinity}\ {\isacharasterisk}{\kern0pt}\ indicator\ {\isacharquery}{\kern0pt}A\ x\ {\isasympartial}M{\isacharparenright}{\kern0pt}{\isachardoublequoteclose}\ \isacommand{by}\isamarkupfalse%
\ {\isacharparenleft}{\kern0pt}metis\ {\isacharparenleft}{\kern0pt}mono{\isacharunderscore}{\kern0pt}tags{\isacharcomma}{\kern0pt}\ lifting{\isacharparenright}{\kern0pt}\ indicator{\isacharunderscore}{\kern0pt}inter{\isacharunderscore}{\kern0pt}arith\ indicator{\isacharunderscore}{\kern0pt}simps{\isacharparenleft}{\kern0pt}{\isadigit{2}}{\isacharparenright}{\kern0pt}\ mem{\isacharunderscore}{\kern0pt}Collect{\isacharunderscore}{\kern0pt}eq\ mult{\isacharunderscore}{\kern0pt}eq{\isacharunderscore}{\kern0pt}{\isadigit{0}}{\isacharunderscore}{\kern0pt}iff{\isacharparenright}{\kern0pt}\isanewline
\ \ \isacommand{also}\isamarkupfalse%
\ \isacommand{have}\isamarkupfalse%
\ {\isachardoublequoteopen}{\isachardot}{\kern0pt}{\isachardot}{\kern0pt}{\isachardot}{\kern0pt}\ {\isacharequal}{\kern0pt}\ {\isasyminfinity}\ {\isacharasterisk}{\kern0pt}\ emeasure\ M\ {\isacharquery}{\kern0pt}A{\isachardoublequoteclose}\ \isacommand{using}\isamarkupfalse%
\ assms{\isacharparenleft}{\kern0pt}{\isadigit{1}}{\isacharparenright}{\kern0pt}\ \isacommand{by}\isamarkupfalse%
\ {\isacharparenleft}{\kern0pt}intro\ nn{\isacharunderscore}{\kern0pt}integral{\isacharunderscore}{\kern0pt}cmult{\isacharunderscore}{\kern0pt}indicator{\isacharcomma}{\kern0pt}\ simp{\isacharparenright}{\kern0pt}\isanewline
\ \ \isacommand{also}\isamarkupfalse%
\ \isacommand{have}\isamarkupfalse%
\ {\isachardoublequoteopen}{\isachardot}{\kern0pt}{\isachardot}{\kern0pt}{\isachardot}{\kern0pt}\ {\isacharequal}{\kern0pt}\ {\isasyminfinity}{\isachardoublequoteclose}\ \isacommand{using}\isamarkupfalse%
\ {\isacharasterisk}{\kern0pt}\ \isacommand{by}\isamarkupfalse%
\ fastforce\isanewline
\ \ \isacommand{finally}\isamarkupfalse%
\ \isacommand{have}\isamarkupfalse%
\ {\isachardoublequoteopen}{\isacharparenleft}{\kern0pt}{\isasymintegral}\isactrlsup {\isacharplus}{\kern0pt}x{\isachardot}{\kern0pt}\ f\ x\ {\isacharasterisk}{\kern0pt}\ indicator\ {\isacharquery}{\kern0pt}A\ x\ {\isasympartial}M{\isacharparenright}{\kern0pt}\ {\isacharequal}{\kern0pt}\ {\isasyminfinity}{\isachardoublequoteclose}\ \isacommand{{\isachardot}{\kern0pt}}\isamarkupfalse%
\isanewline
\ \ \isacommand{moreover}\isamarkupfalse%
\ \isacommand{have}\isamarkupfalse%
\ {\isachardoublequoteopen}{\isacharparenleft}{\kern0pt}{\isasymintegral}\isactrlsup {\isacharplus}{\kern0pt}x{\isachardot}{\kern0pt}\ f\ x\ {\isacharasterisk}{\kern0pt}\ indicator\ {\isacharquery}{\kern0pt}A\ x\ {\isasympartial}M{\isacharparenright}{\kern0pt}\ {\isasymle}\ {\isacharparenleft}{\kern0pt}{\isasymintegral}\isactrlsup {\isacharplus}{\kern0pt}x{\isachardot}{\kern0pt}\ f\ x\ {\isasympartial}M{\isacharparenright}{\kern0pt}{\isachardoublequoteclose}\ \isacommand{by}\isamarkupfalse%
\ {\isacharparenleft}{\kern0pt}intro\ nn{\isacharunderscore}{\kern0pt}integral{\isacharunderscore}{\kern0pt}mono{\isacharcomma}{\kern0pt}\ simp\ add{\isacharcolon}{\kern0pt}\ indicator{\isacharunderscore}{\kern0pt}def{\isacharparenright}{\kern0pt}\isanewline
\ \ \isacommand{ultimately}\isamarkupfalse%
\ \isacommand{show}\isamarkupfalse%
\ {\isacharquery}{\kern0pt}case\ \isacommand{using}\isamarkupfalse%
\ assms{\isacharparenleft}{\kern0pt}{\isadigit{2}}{\isacharparenright}{\kern0pt}\ \isacommand{by}\isamarkupfalse%
\ simp\isanewline
\isacommand{qed}\isamarkupfalse%
%
\endisatagproof
{\isafoldproof}%
%
\isadelimproof
\isanewline
%
\endisadelimproof
\isanewline
\isacommand{lemma}\isamarkupfalse%
\ cauchy{\isacharunderscore}{\kern0pt}L{\isadigit{1}}{\isacharunderscore}{\kern0pt}AE{\isacharunderscore}{\kern0pt}cauchy{\isacharunderscore}{\kern0pt}subseq{\isacharcolon}{\kern0pt}\isanewline
\ \ \isakeyword{fixes}\ s\ {\isacharcolon}{\kern0pt}{\isacharcolon}{\kern0pt}\ {\isachardoublequoteopen}nat\ {\isasymRightarrow}\ {\isacharprime}{\kern0pt}a\ {\isasymRightarrow}\ {\isacharprime}{\kern0pt}b{\isacharcolon}{\kern0pt}{\isacharcolon}{\kern0pt}{\isacharbraceleft}{\kern0pt}banach{\isacharcomma}{\kern0pt}\ second{\isacharunderscore}{\kern0pt}countable{\isacharunderscore}{\kern0pt}topology{\isacharbraceright}{\kern0pt}{\isachardoublequoteclose}\isanewline
\ \ \isakeyword{assumes}\ {\isacharbrackleft}{\kern0pt}measurable{\isacharbrackright}{\kern0pt}{\isacharcolon}{\kern0pt}\ {\isachardoublequoteopen}{\isasymAnd}n{\isachardot}{\kern0pt}\ integrable\ M\ {\isacharparenleft}{\kern0pt}s\ n{\isacharparenright}{\kern0pt}{\isachardoublequoteclose}\isanewline
\ \ \ \ \ \ \isakeyword{and}\ {\isachardoublequoteopen}{\isasymAnd}e{\isachardot}{\kern0pt}\ e\ {\isachargreater}{\kern0pt}\ {\isadigit{0}}\ {\isasymLongrightarrow}\ {\isasymexists}N{\isachardot}{\kern0pt}\ {\isasymforall}i{\isasymge}N{\isachardot}{\kern0pt}\ {\isasymforall}j{\isasymge}N{\isachardot}{\kern0pt}\ LINT\ x{\isacharbar}{\kern0pt}M{\isachardot}{\kern0pt}\ dist\ {\isacharparenleft}{\kern0pt}s\ i\ x{\isacharparenright}{\kern0pt}\ {\isacharparenleft}{\kern0pt}s\ j\ x{\isacharparenright}{\kern0pt}\ {\isacharless}{\kern0pt}\ e{\isachardoublequoteclose}\isanewline
\ \ \isakeyword{obtains}\ r\ \isakeyword{where}\ {\isachardoublequoteopen}strict{\isacharunderscore}{\kern0pt}mono\ r{\isachardoublequoteclose}\ {\isachardoublequoteopen}AE\ x\ in\ M{\isachardot}{\kern0pt}\ Cauchy\ {\isacharparenleft}{\kern0pt}{\isasymlambda}i{\isachardot}{\kern0pt}\ s\ {\isacharparenleft}{\kern0pt}r\ i{\isacharparenright}{\kern0pt}\ x{\isacharparenright}{\kern0pt}{\isachardoublequoteclose}\isanewline
%
\isadelimproof
%
\endisadelimproof
%
\isatagproof
\isacommand{proof}\isamarkupfalse%
{\isacharminus}{\kern0pt}\isanewline
\ \ \isacommand{have}\isamarkupfalse%
\ {\isachardoublequoteopen}{\isasymexists}r{\isachardot}{\kern0pt}\ {\isasymforall}n{\isachardot}{\kern0pt}\ {\isacharparenleft}{\kern0pt}{\isasymforall}i{\isasymge}r\ n{\isachardot}{\kern0pt}\ {\isasymforall}j{\isasymge}\ r\ n{\isachardot}{\kern0pt}\ LINT\ x{\isacharbar}{\kern0pt}M{\isachardot}{\kern0pt}\ dist\ {\isacharparenleft}{\kern0pt}s\ i\ x{\isacharparenright}{\kern0pt}\ {\isacharparenleft}{\kern0pt}s\ j\ x{\isacharparenright}{\kern0pt}\ {\isacharless}{\kern0pt}\ {\isacharparenleft}{\kern0pt}{\isadigit{1}}\ {\isacharslash}{\kern0pt}\ {\isadigit{2}}{\isacharparenright}{\kern0pt}\ {\isacharcircum}{\kern0pt}\ n{\isacharparenright}{\kern0pt}\ {\isasymand}\ {\isacharparenleft}{\kern0pt}r\ {\isacharparenleft}{\kern0pt}Suc\ n{\isacharparenright}{\kern0pt}\ {\isachargreater}{\kern0pt}\ r\ n{\isacharparenright}{\kern0pt}{\isachardoublequoteclose}\isanewline
\ \ \isacommand{proof}\isamarkupfalse%
\ {\isacharparenleft}{\kern0pt}intro\ dependent{\isacharunderscore}{\kern0pt}nat{\isacharunderscore}{\kern0pt}choice{\isacharcomma}{\kern0pt}\ goal{\isacharunderscore}{\kern0pt}cases{\isacharparenright}{\kern0pt}\isanewline
\ \ \ \ \isacommand{case}\isamarkupfalse%
\ {\isadigit{1}}\isanewline
\ \ \ \ \isacommand{then}\isamarkupfalse%
\ \isacommand{show}\isamarkupfalse%
\ {\isacharquery}{\kern0pt}case\ \isacommand{using}\isamarkupfalse%
\ assms{\isacharparenleft}{\kern0pt}{\isadigit{2}}{\isacharparenright}{\kern0pt}\ \isacommand{by}\isamarkupfalse%
\ presburger\isanewline
\ \ \isacommand{next}\isamarkupfalse%
\isanewline
\ \ \ \ \isacommand{case}\isamarkupfalse%
\ {\isacharparenleft}{\kern0pt}{\isadigit{2}}\ x\ n{\isacharparenright}{\kern0pt}\isanewline
\ \ \ \ \isacommand{obtain}\isamarkupfalse%
\ N\ \isakeyword{where}\ {\isacharasterisk}{\kern0pt}{\isacharcolon}{\kern0pt}\ {\isachardoublequoteopen}LINT\ x{\isacharbar}{\kern0pt}M{\isachardot}{\kern0pt}\ dist\ {\isacharparenleft}{\kern0pt}s\ i\ x{\isacharparenright}{\kern0pt}\ {\isacharparenleft}{\kern0pt}s\ j\ x{\isacharparenright}{\kern0pt}\ {\isacharless}{\kern0pt}\ {\isacharparenleft}{\kern0pt}{\isadigit{1}}\ {\isacharslash}{\kern0pt}\ {\isadigit{2}}{\isacharparenright}{\kern0pt}\ {\isacharcircum}{\kern0pt}\ Suc\ n{\isachardoublequoteclose}\ \isakeyword{if}\ {\isachardoublequoteopen}i\ {\isasymge}\ N{\isachardoublequoteclose}\ {\isachardoublequoteopen}j\ {\isasymge}\ N{\isachardoublequoteclose}\ \isakeyword{for}\ i\ j\ \isacommand{using}\isamarkupfalse%
\ assms{\isacharparenleft}{\kern0pt}{\isadigit{2}}{\isacharparenright}{\kern0pt}{\isacharbrackleft}{\kern0pt}of\ {\isachardoublequoteopen}{\isacharparenleft}{\kern0pt}{\isadigit{1}}\ {\isacharslash}{\kern0pt}\ {\isadigit{2}}{\isacharparenright}{\kern0pt}\ {\isacharcircum}{\kern0pt}\ Suc\ n{\isachardoublequoteclose}{\isacharbrackright}{\kern0pt}\ \isacommand{by}\isamarkupfalse%
\ auto\isanewline
\ \ \ \ \isacommand{{\isacharbraceleft}{\kern0pt}}\isamarkupfalse%
\isanewline
\ \ \ \ \ \ \isacommand{fix}\isamarkupfalse%
\ i\ j\ \isacommand{assume}\isamarkupfalse%
\ {\isachardoublequoteopen}i\ {\isasymge}\ max\ N\ {\isacharparenleft}{\kern0pt}Suc\ x{\isacharparenright}{\kern0pt}{\isachardoublequoteclose}\ {\isachardoublequoteopen}j\ {\isasymge}\ max\ N\ {\isacharparenleft}{\kern0pt}Suc\ x{\isacharparenright}{\kern0pt}{\isachardoublequoteclose}\isanewline
\ \ \ \ \ \ \isacommand{hence}\isamarkupfalse%
\ {\isachardoublequoteopen}integral\isactrlsup L\ M\ {\isacharparenleft}{\kern0pt}{\isasymlambda}x{\isachardot}{\kern0pt}\ dist\ {\isacharparenleft}{\kern0pt}s\ i\ x{\isacharparenright}{\kern0pt}\ {\isacharparenleft}{\kern0pt}s\ j\ x{\isacharparenright}{\kern0pt}{\isacharparenright}{\kern0pt}\ {\isacharless}{\kern0pt}\ {\isacharparenleft}{\kern0pt}{\isadigit{1}}\ {\isacharslash}{\kern0pt}\ {\isadigit{2}}{\isacharparenright}{\kern0pt}\ {\isacharcircum}{\kern0pt}\ Suc\ n{\isachardoublequoteclose}\ \isacommand{using}\isamarkupfalse%
\ {\isacharasterisk}{\kern0pt}\ \isacommand{by}\isamarkupfalse%
\ fastforce\isanewline
\ \ \ \ \isacommand{{\isacharbraceright}{\kern0pt}}\isamarkupfalse%
\isanewline
\ \ \ \ \isacommand{then}\isamarkupfalse%
\ \isacommand{show}\isamarkupfalse%
\ {\isacharquery}{\kern0pt}case\ \isacommand{by}\isamarkupfalse%
\ fastforce\isanewline
\ \ \isacommand{qed}\isamarkupfalse%
\isanewline
\ \ \isacommand{then}\isamarkupfalse%
\ \isacommand{obtain}\isamarkupfalse%
\ r\ \isakeyword{where}\ strict{\isacharunderscore}{\kern0pt}mono{\isacharcolon}{\kern0pt}\ {\isachardoublequoteopen}strict{\isacharunderscore}{\kern0pt}mono\ r{\isachardoublequoteclose}\ \isakeyword{and}\ {\isachardoublequoteopen}{\isasymforall}i{\isasymge}r\ n{\isachardot}{\kern0pt}\ {\isasymforall}j{\isasymge}\ r\ n{\isachardot}{\kern0pt}\ LINT\ x{\isacharbar}{\kern0pt}M{\isachardot}{\kern0pt}\ dist\ {\isacharparenleft}{\kern0pt}s\ i\ x{\isacharparenright}{\kern0pt}\ {\isacharparenleft}{\kern0pt}s\ j\ x{\isacharparenright}{\kern0pt}\ {\isacharless}{\kern0pt}\ {\isacharparenleft}{\kern0pt}{\isadigit{1}}\ {\isacharslash}{\kern0pt}\ {\isadigit{2}}{\isacharparenright}{\kern0pt}\ {\isacharcircum}{\kern0pt}\ n{\isachardoublequoteclose}\ \isakeyword{for}\ n\ \isacommand{using}\isamarkupfalse%
\ strict{\isacharunderscore}{\kern0pt}mono{\isacharunderscore}{\kern0pt}Suc{\isacharunderscore}{\kern0pt}iff\ \isacommand{by}\isamarkupfalse%
\ blast\isanewline
\ \ \isacommand{hence}\isamarkupfalse%
\ r{\isacharunderscore}{\kern0pt}is{\isacharcolon}{\kern0pt}\ {\isachardoublequoteopen}LINT\ x{\isacharbar}{\kern0pt}M{\isachardot}{\kern0pt}\ dist\ {\isacharparenleft}{\kern0pt}s\ {\isacharparenleft}{\kern0pt}r\ {\isacharparenleft}{\kern0pt}Suc\ n{\isacharparenright}{\kern0pt}{\isacharparenright}{\kern0pt}\ x{\isacharparenright}{\kern0pt}\ {\isacharparenleft}{\kern0pt}s\ {\isacharparenleft}{\kern0pt}r\ n{\isacharparenright}{\kern0pt}\ x{\isacharparenright}{\kern0pt}\ {\isacharless}{\kern0pt}\ {\isacharparenleft}{\kern0pt}{\isadigit{1}}\ {\isacharslash}{\kern0pt}\ {\isadigit{2}}{\isacharparenright}{\kern0pt}\ {\isacharcircum}{\kern0pt}\ n{\isachardoublequoteclose}\ \isakeyword{for}\ n\ \isacommand{by}\isamarkupfalse%
\ {\isacharparenleft}{\kern0pt}simp\ add{\isacharcolon}{\kern0pt}\ strict{\isacharunderscore}{\kern0pt}mono{\isacharunderscore}{\kern0pt}leD{\isacharparenright}{\kern0pt}\isanewline
\isanewline
\ \ \isacommand{define}\isamarkupfalse%
\ g\ \isakeyword{where}\ {\isachardoublequoteopen}g\ {\isacharequal}{\kern0pt}\ {\isacharparenleft}{\kern0pt}{\isasymlambda}n\ x{\isachardot}{\kern0pt}\ {\isacharparenleft}{\kern0pt}{\isasymSum}i{\isasymle}n{\isachardot}{\kern0pt}\ ennreal\ {\isacharparenleft}{\kern0pt}dist\ {\isacharparenleft}{\kern0pt}s\ {\isacharparenleft}{\kern0pt}r\ {\isacharparenleft}{\kern0pt}Suc\ i{\isacharparenright}{\kern0pt}{\isacharparenright}{\kern0pt}\ x{\isacharparenright}{\kern0pt}\ {\isacharparenleft}{\kern0pt}s\ {\isacharparenleft}{\kern0pt}r\ i{\isacharparenright}{\kern0pt}\ x{\isacharparenright}{\kern0pt}{\isacharparenright}{\kern0pt}{\isacharparenright}{\kern0pt}{\isacharparenright}{\kern0pt}{\isachardoublequoteclose}\isanewline
\ \ \isacommand{define}\isamarkupfalse%
\ g{\isacharprime}{\kern0pt}\ \isakeyword{where}\ {\isachardoublequoteopen}g{\isacharprime}{\kern0pt}\ {\isacharequal}{\kern0pt}\ {\isacharparenleft}{\kern0pt}{\isasymlambda}x{\isachardot}{\kern0pt}\ {\isasymSum}i{\isachardot}{\kern0pt}\ ennreal\ {\isacharparenleft}{\kern0pt}dist\ {\isacharparenleft}{\kern0pt}s\ {\isacharparenleft}{\kern0pt}r\ {\isacharparenleft}{\kern0pt}Suc\ i{\isacharparenright}{\kern0pt}{\isacharparenright}{\kern0pt}\ x{\isacharparenright}{\kern0pt}\ {\isacharparenleft}{\kern0pt}s\ {\isacharparenleft}{\kern0pt}r\ i{\isacharparenright}{\kern0pt}\ x{\isacharparenright}{\kern0pt}{\isacharparenright}{\kern0pt}{\isacharparenright}{\kern0pt}{\isachardoublequoteclose}\isanewline
\isanewline
\ \ \isacommand{have}\isamarkupfalse%
\ integrable{\isacharunderscore}{\kern0pt}g{\isacharcolon}{\kern0pt}\ {\isachardoublequoteopen}{\isacharparenleft}{\kern0pt}{\isasymintegral}\isactrlsup {\isacharplus}{\kern0pt}\ x{\isachardot}{\kern0pt}\ g\ n\ x\ {\isasympartial}M{\isacharparenright}{\kern0pt}\ {\isacharless}{\kern0pt}\ {\isadigit{2}}{\isachardoublequoteclose}\ \isakeyword{for}\ n\isanewline
\ \ \isacommand{proof}\isamarkupfalse%
\ {\isacharminus}{\kern0pt}\isanewline
\ \ \ \ \isacommand{have}\isamarkupfalse%
\ {\isachardoublequoteopen}{\isacharparenleft}{\kern0pt}{\isasymintegral}\isactrlsup {\isacharplus}{\kern0pt}\ x{\isachardot}{\kern0pt}\ g\ n\ x\ {\isasympartial}M{\isacharparenright}{\kern0pt}\ {\isacharequal}{\kern0pt}\ {\isacharparenleft}{\kern0pt}{\isasymintegral}\isactrlsup {\isacharplus}{\kern0pt}\ x{\isachardot}{\kern0pt}\ {\isacharparenleft}{\kern0pt}{\isasymSum}i{\isasymle}n{\isachardot}{\kern0pt}\ ennreal\ {\isacharparenleft}{\kern0pt}dist\ {\isacharparenleft}{\kern0pt}s\ {\isacharparenleft}{\kern0pt}r\ {\isacharparenleft}{\kern0pt}Suc\ i{\isacharparenright}{\kern0pt}{\isacharparenright}{\kern0pt}\ x{\isacharparenright}{\kern0pt}\ {\isacharparenleft}{\kern0pt}s\ {\isacharparenleft}{\kern0pt}r\ i{\isacharparenright}{\kern0pt}\ x{\isacharparenright}{\kern0pt}{\isacharparenright}{\kern0pt}{\isacharparenright}{\kern0pt}\ {\isasympartial}M{\isacharparenright}{\kern0pt}{\isachardoublequoteclose}\ \isacommand{using}\isamarkupfalse%
\ g{\isacharunderscore}{\kern0pt}def\ \isacommand{by}\isamarkupfalse%
\ simp\isanewline
\ \ \ \ \isacommand{also}\isamarkupfalse%
\ \isacommand{have}\isamarkupfalse%
\ {\isachardoublequoteopen}{\isachardot}{\kern0pt}{\isachardot}{\kern0pt}{\isachardot}{\kern0pt}\ {\isacharequal}{\kern0pt}\ {\isacharparenleft}{\kern0pt}{\isasymSum}i{\isasymle}n{\isachardot}{\kern0pt}\ {\isacharparenleft}{\kern0pt}{\isasymintegral}\isactrlsup {\isacharplus}{\kern0pt}\ x{\isachardot}{\kern0pt}\ ennreal\ {\isacharparenleft}{\kern0pt}dist\ {\isacharparenleft}{\kern0pt}s\ {\isacharparenleft}{\kern0pt}r\ {\isacharparenleft}{\kern0pt}Suc\ i{\isacharparenright}{\kern0pt}{\isacharparenright}{\kern0pt}\ x{\isacharparenright}{\kern0pt}\ {\isacharparenleft}{\kern0pt}s\ {\isacharparenleft}{\kern0pt}r\ i{\isacharparenright}{\kern0pt}\ x{\isacharparenright}{\kern0pt}{\isacharparenright}{\kern0pt}\ {\isasympartial}M{\isacharparenright}{\kern0pt}{\isacharparenright}{\kern0pt}{\isachardoublequoteclose}\ \isacommand{by}\isamarkupfalse%
\ {\isacharparenleft}{\kern0pt}intro\ nn{\isacharunderscore}{\kern0pt}integral{\isacharunderscore}{\kern0pt}sum{\isacharcomma}{\kern0pt}\ simp{\isacharparenright}{\kern0pt}\isanewline
\ \ \ \ \isacommand{also}\isamarkupfalse%
\ \isacommand{have}\isamarkupfalse%
\ {\isachardoublequoteopen}{\isachardot}{\kern0pt}{\isachardot}{\kern0pt}{\isachardot}{\kern0pt}\ {\isacharequal}{\kern0pt}\ {\isacharparenleft}{\kern0pt}{\isasymSum}i{\isasymle}n{\isachardot}{\kern0pt}\ LINT\ x{\isacharbar}{\kern0pt}M{\isachardot}{\kern0pt}\ dist\ {\isacharparenleft}{\kern0pt}s\ {\isacharparenleft}{\kern0pt}r\ {\isacharparenleft}{\kern0pt}Suc\ i{\isacharparenright}{\kern0pt}{\isacharparenright}{\kern0pt}\ x{\isacharparenright}{\kern0pt}\ {\isacharparenleft}{\kern0pt}s\ {\isacharparenleft}{\kern0pt}r\ i{\isacharparenright}{\kern0pt}\ x{\isacharparenright}{\kern0pt}{\isacharparenright}{\kern0pt}{\isachardoublequoteclose}\ \isacommand{unfolding}\isamarkupfalse%
\ dist{\isacharunderscore}{\kern0pt}norm\ \isacommand{using}\isamarkupfalse%
\ assms{\isacharparenleft}{\kern0pt}{\isadigit{1}}{\isacharparenright}{\kern0pt}\ \isacommand{by}\isamarkupfalse%
\ {\isacharparenleft}{\kern0pt}subst\ nn{\isacharunderscore}{\kern0pt}integral{\isacharunderscore}{\kern0pt}eq{\isacharunderscore}{\kern0pt}integral{\isacharbrackleft}{\kern0pt}OF\ integrable{\isacharunderscore}{\kern0pt}norm{\isacharbrackright}{\kern0pt}{\isacharcomma}{\kern0pt}\ auto{\isacharparenright}{\kern0pt}\isanewline
\ \ \ \ \isacommand{also}\isamarkupfalse%
\ \isacommand{have}\isamarkupfalse%
\ {\isachardoublequoteopen}{\isachardot}{\kern0pt}{\isachardot}{\kern0pt}{\isachardot}{\kern0pt}\ {\isacharless}{\kern0pt}\ ennreal\ {\isacharparenleft}{\kern0pt}{\isasymSum}i{\isasymle}n{\isachardot}{\kern0pt}\ {\isacharparenleft}{\kern0pt}{\isadigit{1}}\ {\isacharslash}{\kern0pt}\ {\isadigit{2}}{\isacharparenright}{\kern0pt}\ {\isacharcircum}{\kern0pt}\ i{\isacharparenright}{\kern0pt}{\isachardoublequoteclose}\ \isacommand{by}\isamarkupfalse%
\ {\isacharparenleft}{\kern0pt}intro\ ennreal{\isacharunderscore}{\kern0pt}lessI{\isacharbrackleft}{\kern0pt}OF\ sum{\isacharunderscore}{\kern0pt}pos\ sum{\isacharunderscore}{\kern0pt}strict{\isacharunderscore}{\kern0pt}mono{\isacharbrackleft}{\kern0pt}OF\ finite{\isacharunderscore}{\kern0pt}atMost\ {\isacharunderscore}{\kern0pt}\ r{\isacharunderscore}{\kern0pt}is{\isacharbrackright}{\kern0pt}{\isacharbrackright}{\kern0pt}{\isacharcomma}{\kern0pt}\ auto{\isacharparenright}{\kern0pt}\isanewline
\ \ \ \ \isacommand{also}\isamarkupfalse%
\ \isacommand{have}\isamarkupfalse%
\ {\isachardoublequoteopen}{\isachardot}{\kern0pt}{\isachardot}{\kern0pt}{\isachardot}{\kern0pt}\ {\isasymle}\ ennreal\ {\isadigit{2}}{\isachardoublequoteclose}\ \isacommand{unfolding}\isamarkupfalse%
\ sum{\isacharunderscore}{\kern0pt}gp{\isadigit{0}}{\isacharbrackleft}{\kern0pt}of\ {\isachardoublequoteopen}{\isadigit{1}}\ {\isacharslash}{\kern0pt}\ {\isadigit{2}}{\isachardoublequoteclose}\ n{\isacharbrackright}{\kern0pt}\ \isacommand{by}\isamarkupfalse%
\ {\isacharparenleft}{\kern0pt}intro\ ennreal{\isacharunderscore}{\kern0pt}leI{\isacharcomma}{\kern0pt}\ auto{\isacharparenright}{\kern0pt}\isanewline
\ \ \ \ \isacommand{finally}\isamarkupfalse%
\ \isacommand{show}\isamarkupfalse%
\ {\isachardoublequoteopen}{\isacharparenleft}{\kern0pt}{\isasymintegral}\isactrlsup {\isacharplus}{\kern0pt}\ x{\isachardot}{\kern0pt}\ g\ n\ x\ {\isasympartial}M{\isacharparenright}{\kern0pt}\ {\isacharless}{\kern0pt}\ {\isadigit{2}}{\isachardoublequoteclose}\ \isacommand{by}\isamarkupfalse%
\ simp\isanewline
\ \ \isacommand{qed}\isamarkupfalse%
\isanewline
\isanewline
\ \ \isacommand{have}\isamarkupfalse%
\ integrable{\isacharunderscore}{\kern0pt}g{\isacharprime}{\kern0pt}{\isacharcolon}{\kern0pt}\ {\isachardoublequoteopen}{\isacharparenleft}{\kern0pt}{\isasymintegral}\isactrlsup {\isacharplus}{\kern0pt}\ x{\isachardot}{\kern0pt}\ g{\isacharprime}{\kern0pt}\ x\ {\isasympartial}M{\isacharparenright}{\kern0pt}\ {\isasymle}\ {\isadigit{2}}{\isachardoublequoteclose}\isanewline
\ \ \isacommand{proof}\isamarkupfalse%
\ {\isacharminus}{\kern0pt}\isanewline
\ \ \ \ \isacommand{have}\isamarkupfalse%
\ {\isachardoublequoteopen}incseq\ {\isacharparenleft}{\kern0pt}{\isasymlambda}n{\isachardot}{\kern0pt}\ g\ n\ x{\isacharparenright}{\kern0pt}{\isachardoublequoteclose}\ \isakeyword{for}\ x\ \isacommand{by}\isamarkupfalse%
\ {\isacharparenleft}{\kern0pt}intro\ incseq{\isacharunderscore}{\kern0pt}SucI{\isacharcomma}{\kern0pt}\ auto\ simp\ add{\isacharcolon}{\kern0pt}\ g{\isacharunderscore}{\kern0pt}def\ ennreal{\isacharunderscore}{\kern0pt}leI{\isacharparenright}{\kern0pt}\isanewline
\ \ \ \ \isacommand{hence}\isamarkupfalse%
\ {\isachardoublequoteopen}convergent\ {\isacharparenleft}{\kern0pt}{\isasymlambda}n{\isachardot}{\kern0pt}\ g\ n\ x{\isacharparenright}{\kern0pt}{\isachardoublequoteclose}\ \isakeyword{for}\ x\ \isacommand{unfolding}\isamarkupfalse%
\ convergent{\isacharunderscore}{\kern0pt}def\ \isacommand{using}\isamarkupfalse%
\ LIMSEQ{\isacharunderscore}{\kern0pt}incseq{\isacharunderscore}{\kern0pt}SUP\ \isacommand{by}\isamarkupfalse%
\ fast\isanewline
\ \ \ \ \isacommand{hence}\isamarkupfalse%
\ {\isachardoublequoteopen}{\isacharparenleft}{\kern0pt}{\isasymlambda}n{\isachardot}{\kern0pt}\ g\ n\ x{\isacharparenright}{\kern0pt}\ {\isasymlonglonglongrightarrow}\ g{\isacharprime}{\kern0pt}\ x{\isachardoublequoteclose}\ \isakeyword{for}\ x\ \isacommand{unfolding}\isamarkupfalse%
\ g{\isacharunderscore}{\kern0pt}def\ g{\isacharprime}{\kern0pt}{\isacharunderscore}{\kern0pt}def\ \isacommand{by}\isamarkupfalse%
\ {\isacharparenleft}{\kern0pt}intro\ summable{\isacharunderscore}{\kern0pt}iff{\isacharunderscore}{\kern0pt}convergent{\isacharprime}{\kern0pt}{\isacharbrackleft}{\kern0pt}THEN\ iffD{\isadigit{2}}{\isacharcomma}{\kern0pt}\ THEN\ summable{\isacharunderscore}{\kern0pt}LIMSEQ{\isacharprime}{\kern0pt}{\isacharbrackright}{\kern0pt}{\isacharcomma}{\kern0pt}\ blast{\isacharparenright}{\kern0pt}\isanewline
\ \ \ \ \isacommand{hence}\isamarkupfalse%
\ {\isachardoublequoteopen}{\isacharparenleft}{\kern0pt}{\isasymintegral}\isactrlsup {\isacharplus}{\kern0pt}\ x{\isachardot}{\kern0pt}\ g{\isacharprime}{\kern0pt}\ x\ {\isasympartial}M{\isacharparenright}{\kern0pt}\ {\isacharequal}{\kern0pt}\ {\isacharparenleft}{\kern0pt}{\isasymintegral}\isactrlsup {\isacharplus}{\kern0pt}\ x{\isachardot}{\kern0pt}\ liminf\ {\isacharparenleft}{\kern0pt}{\isasymlambda}n{\isachardot}{\kern0pt}\ g\ n\ x{\isacharparenright}{\kern0pt}\ {\isasympartial}M{\isacharparenright}{\kern0pt}{\isachardoublequoteclose}\ \isacommand{by}\isamarkupfalse%
\ {\isacharparenleft}{\kern0pt}metis\ lim{\isacharunderscore}{\kern0pt}imp{\isacharunderscore}{\kern0pt}Liminf\ trivial{\isacharunderscore}{\kern0pt}limit{\isacharunderscore}{\kern0pt}sequentially{\isacharparenright}{\kern0pt}\isanewline
\ \ \ \ \isacommand{also}\isamarkupfalse%
\ \isacommand{have}\isamarkupfalse%
\ {\isachardoublequoteopen}{\isachardot}{\kern0pt}{\isachardot}{\kern0pt}{\isachardot}{\kern0pt}\ {\isasymle}\ liminf\ {\isacharparenleft}{\kern0pt}{\isasymlambda}n{\isachardot}{\kern0pt}\ {\isasymintegral}\isactrlsup {\isacharplus}{\kern0pt}\ x{\isachardot}{\kern0pt}\ g\ n\ x\ {\isasympartial}M{\isacharparenright}{\kern0pt}{\isachardoublequoteclose}\ \isacommand{by}\isamarkupfalse%
\ {\isacharparenleft}{\kern0pt}intro\ nn{\isacharunderscore}{\kern0pt}integral{\isacharunderscore}{\kern0pt}liminf{\isacharcomma}{\kern0pt}\ simp\ add{\isacharcolon}{\kern0pt}\ g{\isacharunderscore}{\kern0pt}def{\isacharparenright}{\kern0pt}\isanewline
\ \ \ \ \isacommand{also}\isamarkupfalse%
\ \isacommand{have}\isamarkupfalse%
\ {\isachardoublequoteopen}{\isachardot}{\kern0pt}{\isachardot}{\kern0pt}{\isachardot}{\kern0pt}\ {\isasymle}\ liminf\ {\isacharparenleft}{\kern0pt}{\isasymlambda}n{\isachardot}{\kern0pt}\ {\isadigit{2}}{\isacharparenright}{\kern0pt}{\isachardoublequoteclose}\ \isacommand{using}\isamarkupfalse%
\ integrable{\isacharunderscore}{\kern0pt}g\ \isacommand{by}\isamarkupfalse%
\ {\isacharparenleft}{\kern0pt}intro\ Liminf{\isacharunderscore}{\kern0pt}mono{\isacharparenright}{\kern0pt}\ {\isacharparenleft}{\kern0pt}simp\ add{\isacharcolon}{\kern0pt}\ order{\isacharunderscore}{\kern0pt}le{\isacharunderscore}{\kern0pt}less{\isacharparenright}{\kern0pt}\isanewline
\ \ \ \ \isacommand{also}\isamarkupfalse%
\ \isacommand{have}\isamarkupfalse%
\ {\isachardoublequoteopen}{\isachardot}{\kern0pt}{\isachardot}{\kern0pt}{\isachardot}{\kern0pt}\ {\isacharequal}{\kern0pt}\ {\isadigit{2}}{\isachardoublequoteclose}\ \isacommand{using}\isamarkupfalse%
\ sequentially{\isacharunderscore}{\kern0pt}bot\ tendsto{\isacharunderscore}{\kern0pt}iff{\isacharunderscore}{\kern0pt}Liminf{\isacharunderscore}{\kern0pt}eq{\isacharunderscore}{\kern0pt}Limsup\ \isacommand{by}\isamarkupfalse%
\ blast\isanewline
\ \ \ \ \isacommand{finally}\isamarkupfalse%
\ \isacommand{show}\isamarkupfalse%
\ {\isacharquery}{\kern0pt}thesis\ \isacommand{{\isachardot}{\kern0pt}}\isamarkupfalse%
\isanewline
\ \ \isacommand{qed}\isamarkupfalse%
\isanewline
\ \ \isacommand{hence}\isamarkupfalse%
\ {\isachardoublequoteopen}AE\ x\ in\ M{\isachardot}{\kern0pt}\ g{\isacharprime}{\kern0pt}\ x\ {\isacharless}{\kern0pt}\ {\isasyminfinity}{\isachardoublequoteclose}\ \isacommand{by}\isamarkupfalse%
\ {\isacharparenleft}{\kern0pt}intro\ finite{\isacharunderscore}{\kern0pt}nn{\isacharunderscore}{\kern0pt}integral{\isacharunderscore}{\kern0pt}imp{\isacharunderscore}{\kern0pt}ae{\isacharunderscore}{\kern0pt}finite{\isacharparenright}{\kern0pt}\ {\isacharparenleft}{\kern0pt}auto\ simp\ add{\isacharcolon}{\kern0pt}\ order{\isacharunderscore}{\kern0pt}le{\isacharunderscore}{\kern0pt}less{\isacharunderscore}{\kern0pt}trans\ g{\isacharprime}{\kern0pt}{\isacharunderscore}{\kern0pt}def{\isacharparenright}{\kern0pt}\isanewline
\ \ \isacommand{moreover}\isamarkupfalse%
\ \isacommand{have}\isamarkupfalse%
\ {\isachardoublequoteopen}summable\ {\isacharparenleft}{\kern0pt}{\isasymlambda}i{\isachardot}{\kern0pt}\ dist\ {\isacharparenleft}{\kern0pt}s\ {\isacharparenleft}{\kern0pt}r\ {\isacharparenleft}{\kern0pt}Suc\ i{\isacharparenright}{\kern0pt}{\isacharparenright}{\kern0pt}\ x{\isacharparenright}{\kern0pt}\ {\isacharparenleft}{\kern0pt}s\ {\isacharparenleft}{\kern0pt}r\ i{\isacharparenright}{\kern0pt}\ x{\isacharparenright}{\kern0pt}{\isacharparenright}{\kern0pt}{\isachardoublequoteclose}\ \isakeyword{if}\ {\isachardoublequoteopen}g{\isacharprime}{\kern0pt}\ x\ {\isasymnoteq}\ {\isasyminfinity}{\isachardoublequoteclose}\ \isakeyword{for}\ x\ \isacommand{using}\isamarkupfalse%
\ that\ \isacommand{unfolding}\isamarkupfalse%
\ g{\isacharprime}{\kern0pt}{\isacharunderscore}{\kern0pt}def\ \isacommand{by}\isamarkupfalse%
\ {\isacharparenleft}{\kern0pt}intro\ summable{\isacharunderscore}{\kern0pt}suminf{\isacharunderscore}{\kern0pt}not{\isacharunderscore}{\kern0pt}top{\isacharcomma}{\kern0pt}\ intro\ zero{\isacharunderscore}{\kern0pt}le{\isacharunderscore}{\kern0pt}dist{\isacharcomma}{\kern0pt}\ fastforce{\isacharparenright}{\kern0pt}\ \isanewline
\ \ \isacommand{ultimately}\isamarkupfalse%
\ \isacommand{have}\isamarkupfalse%
\ ae{\isacharunderscore}{\kern0pt}summable{\isacharcolon}{\kern0pt}\ {\isachardoublequoteopen}AE\ x\ in\ M{\isachardot}{\kern0pt}\ summable\ {\isacharparenleft}{\kern0pt}{\isasymlambda}i{\isachardot}{\kern0pt}\ s\ {\isacharparenleft}{\kern0pt}r\ {\isacharparenleft}{\kern0pt}Suc\ i{\isacharparenright}{\kern0pt}{\isacharparenright}{\kern0pt}\ x\ {\isacharminus}{\kern0pt}\ s\ {\isacharparenleft}{\kern0pt}r\ i{\isacharparenright}{\kern0pt}\ x{\isacharparenright}{\kern0pt}{\isachardoublequoteclose}\ \isacommand{using}\isamarkupfalse%
\ summable{\isacharunderscore}{\kern0pt}norm{\isacharunderscore}{\kern0pt}cancel\ \isacommand{unfolding}\isamarkupfalse%
\ dist{\isacharunderscore}{\kern0pt}norm\ \isacommand{by}\isamarkupfalse%
\ force\isanewline
\isanewline
\ \ \isacommand{{\isacharbraceleft}{\kern0pt}}\isamarkupfalse%
\isanewline
\ \ \ \ \isacommand{fix}\isamarkupfalse%
\ x\ \isacommand{assume}\isamarkupfalse%
\ {\isachardoublequoteopen}summable\ {\isacharparenleft}{\kern0pt}{\isasymlambda}i{\isachardot}{\kern0pt}\ s\ {\isacharparenleft}{\kern0pt}r\ {\isacharparenleft}{\kern0pt}Suc\ i{\isacharparenright}{\kern0pt}{\isacharparenright}{\kern0pt}\ x\ {\isacharminus}{\kern0pt}\ s\ {\isacharparenleft}{\kern0pt}r\ i{\isacharparenright}{\kern0pt}\ x{\isacharparenright}{\kern0pt}{\isachardoublequoteclose}\isanewline
\ \ \ \ \isacommand{hence}\isamarkupfalse%
\ {\isachardoublequoteopen}{\isacharparenleft}{\kern0pt}{\isasymlambda}n{\isachardot}{\kern0pt}\ {\isasymSum}i{\isacharless}{\kern0pt}n{\isachardot}{\kern0pt}\ s\ {\isacharparenleft}{\kern0pt}r\ {\isacharparenleft}{\kern0pt}Suc\ i{\isacharparenright}{\kern0pt}{\isacharparenright}{\kern0pt}\ x\ {\isacharminus}{\kern0pt}\ s\ {\isacharparenleft}{\kern0pt}r\ i{\isacharparenright}{\kern0pt}\ x{\isacharparenright}{\kern0pt}\ {\isasymlonglonglongrightarrow}\ {\isacharparenleft}{\kern0pt}{\isasymSum}i{\isachardot}{\kern0pt}\ s\ {\isacharparenleft}{\kern0pt}r\ {\isacharparenleft}{\kern0pt}Suc\ i{\isacharparenright}{\kern0pt}{\isacharparenright}{\kern0pt}\ x\ {\isacharminus}{\kern0pt}\ s\ {\isacharparenleft}{\kern0pt}r\ i{\isacharparenright}{\kern0pt}\ x{\isacharparenright}{\kern0pt}{\isachardoublequoteclose}\ \isacommand{using}\isamarkupfalse%
\ summable{\isacharunderscore}{\kern0pt}LIMSEQ\ \isacommand{by}\isamarkupfalse%
\ blast\isanewline
\ \ \ \ \isacommand{moreover}\isamarkupfalse%
\ \isacommand{have}\isamarkupfalse%
\ {\isachardoublequoteopen}{\isacharparenleft}{\kern0pt}{\isasymlambda}n{\isachardot}{\kern0pt}\ {\isacharparenleft}{\kern0pt}{\isasymSum}i{\isacharless}{\kern0pt}n{\isachardot}{\kern0pt}\ s\ {\isacharparenleft}{\kern0pt}r\ {\isacharparenleft}{\kern0pt}Suc\ i{\isacharparenright}{\kern0pt}{\isacharparenright}{\kern0pt}\ x\ {\isacharminus}{\kern0pt}\ s\ {\isacharparenleft}{\kern0pt}r\ i{\isacharparenright}{\kern0pt}\ x{\isacharparenright}{\kern0pt}{\isacharparenright}{\kern0pt}\ {\isacharequal}{\kern0pt}\ {\isacharparenleft}{\kern0pt}{\isasymlambda}n{\isachardot}{\kern0pt}\ s\ {\isacharparenleft}{\kern0pt}r\ n{\isacharparenright}{\kern0pt}\ x\ {\isacharminus}{\kern0pt}\ s\ {\isacharparenleft}{\kern0pt}r\ {\isadigit{0}}{\isacharparenright}{\kern0pt}\ x{\isacharparenright}{\kern0pt}{\isachardoublequoteclose}\ \isacommand{using}\isamarkupfalse%
\ sum{\isacharunderscore}{\kern0pt}lessThan{\isacharunderscore}{\kern0pt}telescope\ \isacommand{by}\isamarkupfalse%
\ fastforce\isanewline
\ \ \ \ \isacommand{ultimately}\isamarkupfalse%
\ \isacommand{have}\isamarkupfalse%
\ {\isachardoublequoteopen}{\isacharparenleft}{\kern0pt}{\isasymlambda}n{\isachardot}{\kern0pt}\ s\ {\isacharparenleft}{\kern0pt}r\ n{\isacharparenright}{\kern0pt}\ x\ {\isacharminus}{\kern0pt}\ s\ {\isacharparenleft}{\kern0pt}r\ {\isadigit{0}}{\isacharparenright}{\kern0pt}\ x{\isacharparenright}{\kern0pt}\ {\isasymlonglonglongrightarrow}\ {\isacharparenleft}{\kern0pt}{\isasymSum}i{\isachardot}{\kern0pt}\ s\ {\isacharparenleft}{\kern0pt}r\ {\isacharparenleft}{\kern0pt}Suc\ i{\isacharparenright}{\kern0pt}{\isacharparenright}{\kern0pt}\ x\ {\isacharminus}{\kern0pt}\ s\ {\isacharparenleft}{\kern0pt}r\ i{\isacharparenright}{\kern0pt}\ x{\isacharparenright}{\kern0pt}{\isachardoublequoteclose}\ \isacommand{by}\isamarkupfalse%
\ argo\isanewline
\ \ \ \ \isacommand{hence}\isamarkupfalse%
\ {\isachardoublequoteopen}{\isacharparenleft}{\kern0pt}{\isasymlambda}n{\isachardot}{\kern0pt}\ s\ {\isacharparenleft}{\kern0pt}r\ n{\isacharparenright}{\kern0pt}\ x\ {\isacharminus}{\kern0pt}\ s\ {\isacharparenleft}{\kern0pt}r\ {\isadigit{0}}{\isacharparenright}{\kern0pt}\ x\ {\isacharplus}{\kern0pt}\ s\ {\isacharparenleft}{\kern0pt}r\ {\isadigit{0}}{\isacharparenright}{\kern0pt}\ x{\isacharparenright}{\kern0pt}\ {\isasymlonglonglongrightarrow}\ {\isacharparenleft}{\kern0pt}{\isasymSum}i{\isachardot}{\kern0pt}\ s\ {\isacharparenleft}{\kern0pt}r\ {\isacharparenleft}{\kern0pt}Suc\ i{\isacharparenright}{\kern0pt}{\isacharparenright}{\kern0pt}\ x\ {\isacharminus}{\kern0pt}\ s\ {\isacharparenleft}{\kern0pt}r\ i{\isacharparenright}{\kern0pt}\ x{\isacharparenright}{\kern0pt}\ {\isacharplus}{\kern0pt}\ s\ {\isacharparenleft}{\kern0pt}r\ {\isadigit{0}}{\isacharparenright}{\kern0pt}\ x{\isachardoublequoteclose}\ \isacommand{by}\isamarkupfalse%
\ {\isacharparenleft}{\kern0pt}intro\ isCont{\isacharunderscore}{\kern0pt}tendsto{\isacharunderscore}{\kern0pt}compose{\isacharbrackleft}{\kern0pt}of\ {\isacharunderscore}{\kern0pt}\ {\isachardoublequoteopen}{\isasymlambda}z{\isachardot}{\kern0pt}\ z\ {\isacharplus}{\kern0pt}\ s\ {\isacharparenleft}{\kern0pt}r\ {\isadigit{0}}{\isacharparenright}{\kern0pt}\ x{\isachardoublequoteclose}{\isacharbrackright}{\kern0pt}{\isacharcomma}{\kern0pt}\ auto{\isacharparenright}{\kern0pt}\isanewline
\ \ \ \ \isacommand{hence}\isamarkupfalse%
\ {\isachardoublequoteopen}Cauchy\ {\isacharparenleft}{\kern0pt}{\isasymlambda}n{\isachardot}{\kern0pt}\ s\ {\isacharparenleft}{\kern0pt}r\ n{\isacharparenright}{\kern0pt}\ x{\isacharparenright}{\kern0pt}{\isachardoublequoteclose}\ \isacommand{by}\isamarkupfalse%
\ {\isacharparenleft}{\kern0pt}simp\ add{\isacharcolon}{\kern0pt}\ LIMSEQ{\isacharunderscore}{\kern0pt}imp{\isacharunderscore}{\kern0pt}Cauchy{\isacharparenright}{\kern0pt}\isanewline
\ \ \isacommand{{\isacharbraceright}{\kern0pt}}\isamarkupfalse%
\isanewline
\isanewline
\ \ \isacommand{hence}\isamarkupfalse%
\ {\isachardoublequoteopen}AE\ x\ in\ M{\isachardot}{\kern0pt}\ Cauchy\ {\isacharparenleft}{\kern0pt}{\isasymlambda}i{\isachardot}{\kern0pt}\ s\ {\isacharparenleft}{\kern0pt}r\ i{\isacharparenright}{\kern0pt}\ x{\isacharparenright}{\kern0pt}{\isachardoublequoteclose}\ \isacommand{using}\isamarkupfalse%
\ ae{\isacharunderscore}{\kern0pt}summable\ \isacommand{by}\isamarkupfalse%
\ fast\isanewline
\ \ \isacommand{thus}\isamarkupfalse%
\ {\isacharquery}{\kern0pt}thesis\ \isacommand{by}\isamarkupfalse%
\ {\isacharparenleft}{\kern0pt}rule\ that{\isacharbrackleft}{\kern0pt}OF\ strict{\isacharunderscore}{\kern0pt}mono{\isacharparenleft}{\kern0pt}{\isadigit{1}}{\isacharparenright}{\kern0pt}{\isacharbrackright}{\kern0pt}{\isacharparenright}{\kern0pt}\isanewline
\isacommand{qed}\isamarkupfalse%
%
\endisatagproof
{\isafoldproof}%
%
\isadelimproof
\isanewline
%
\endisadelimproof
\isanewline
\isacommand{lemma}\isamarkupfalse%
\ integrable{\isacharunderscore}{\kern0pt}max{\isacharbrackleft}{\kern0pt}simp{\isacharcomma}{\kern0pt}\ intro{\isacharbrackright}{\kern0pt}{\isacharcolon}{\kern0pt}\isanewline
\ \ \isakeyword{fixes}\ f\ {\isacharcolon}{\kern0pt}{\isacharcolon}{\kern0pt}\ {\isachardoublequoteopen}{\isacharprime}{\kern0pt}a\ {\isasymRightarrow}\ {\isacharprime}{\kern0pt}b\ {\isacharcolon}{\kern0pt}{\isacharcolon}{\kern0pt}\ {\isacharbraceleft}{\kern0pt}second{\isacharunderscore}{\kern0pt}countable{\isacharunderscore}{\kern0pt}topology{\isacharcomma}{\kern0pt}\ banach{\isacharcomma}{\kern0pt}\ linorder{\isacharunderscore}{\kern0pt}topology{\isacharbraceright}{\kern0pt}{\isachardoublequoteclose}\isanewline
\ \ \isakeyword{assumes}\ fg{\isacharbrackleft}{\kern0pt}measurable{\isacharbrackright}{\kern0pt}{\isacharcolon}{\kern0pt}\ {\isachardoublequoteopen}integrable\ M\ f{\isachardoublequoteclose}\ {\isachardoublequoteopen}integrable\ M\ g{\isachardoublequoteclose}\isanewline
\ \ \isakeyword{shows}\ {\isachardoublequoteopen}integrable\ M\ {\isacharparenleft}{\kern0pt}{\isasymlambda}x{\isachardot}{\kern0pt}\ max\ {\isacharparenleft}{\kern0pt}f\ x{\isacharparenright}{\kern0pt}\ {\isacharparenleft}{\kern0pt}g\ x{\isacharparenright}{\kern0pt}{\isacharparenright}{\kern0pt}{\isachardoublequoteclose}\isanewline
%
\isadelimproof
%
\endisadelimproof
%
\isatagproof
\isacommand{proof}\isamarkupfalse%
\ {\isacharparenleft}{\kern0pt}rule\ Bochner{\isacharunderscore}{\kern0pt}Integration{\isachardot}{\kern0pt}integrable{\isacharunderscore}{\kern0pt}bound{\isacharparenright}{\kern0pt}\isanewline
\ \ \isacommand{{\isacharbraceleft}{\kern0pt}}\isamarkupfalse%
\isanewline
\ \ \ \ \isacommand{fix}\isamarkupfalse%
\ x\ y\ {\isacharcolon}{\kern0pt}{\isacharcolon}{\kern0pt}\ {\isacharprime}{\kern0pt}b\ \ \ \ \ \ \ \ \ \ \ \ \ \ \ \ \ \ \ \ \ \ \ \ \ \ \ \ \ \ \ \ \ \ \ \ \ \ \ \ \ \ \ \ \ \isanewline
\ \ \ \ \isacommand{have}\isamarkupfalse%
\ {\isachardoublequoteopen}norm\ {\isacharparenleft}{\kern0pt}max\ x\ y{\isacharparenright}{\kern0pt}\ {\isasymle}\ max\ {\isacharparenleft}{\kern0pt}norm\ x{\isacharparenright}{\kern0pt}\ {\isacharparenleft}{\kern0pt}norm\ y{\isacharparenright}{\kern0pt}{\isachardoublequoteclose}\ \isacommand{by}\isamarkupfalse%
\ linarith\isanewline
\ \ \ \ \isacommand{also}\isamarkupfalse%
\ \isacommand{have}\isamarkupfalse%
\ {\isachardoublequoteopen}{\isachardot}{\kern0pt}{\isachardot}{\kern0pt}{\isachardot}{\kern0pt}\ {\isasymle}\ norm\ x\ {\isacharplus}{\kern0pt}\ norm\ y{\isachardoublequoteclose}\ \isacommand{by}\isamarkupfalse%
\ simp\isanewline
\ \ \ \ \isacommand{finally}\isamarkupfalse%
\ \isacommand{have}\isamarkupfalse%
\ {\isachardoublequoteopen}norm\ {\isacharparenleft}{\kern0pt}max\ x\ y{\isacharparenright}{\kern0pt}\ {\isasymle}\ norm\ {\isacharparenleft}{\kern0pt}norm\ x\ {\isacharplus}{\kern0pt}\ norm\ y{\isacharparenright}{\kern0pt}{\isachardoublequoteclose}\ \isacommand{by}\isamarkupfalse%
\ simp\isanewline
\ \ \isacommand{{\isacharbraceright}{\kern0pt}}\isamarkupfalse%
\isanewline
\ \ \isacommand{thus}\isamarkupfalse%
\ {\isachardoublequoteopen}AE\ x\ in\ M{\isachardot}{\kern0pt}\ norm\ {\isacharparenleft}{\kern0pt}max\ {\isacharparenleft}{\kern0pt}f\ x{\isacharparenright}{\kern0pt}\ {\isacharparenleft}{\kern0pt}g\ x{\isacharparenright}{\kern0pt}{\isacharparenright}{\kern0pt}\ {\isasymle}\ norm\ {\isacharparenleft}{\kern0pt}norm\ {\isacharparenleft}{\kern0pt}f\ x{\isacharparenright}{\kern0pt}\ {\isacharplus}{\kern0pt}\ norm\ {\isacharparenleft}{\kern0pt}g\ x{\isacharparenright}{\kern0pt}{\isacharparenright}{\kern0pt}{\isachardoublequoteclose}\ \isacommand{by}\isamarkupfalse%
\ simp\isanewline
\isacommand{qed}\isamarkupfalse%
\ {\isacharparenleft}{\kern0pt}auto\ intro{\isacharcolon}{\kern0pt}\ Bochner{\isacharunderscore}{\kern0pt}Integration{\isachardot}{\kern0pt}integrable{\isacharunderscore}{\kern0pt}add{\isacharbrackleft}{\kern0pt}OF\ integrable{\isacharunderscore}{\kern0pt}norm{\isacharbrackleft}{\kern0pt}OF\ fg{\isacharparenleft}{\kern0pt}{\isadigit{1}}{\isacharparenright}{\kern0pt}{\isacharbrackright}{\kern0pt}\ integrable{\isacharunderscore}{\kern0pt}norm{\isacharbrackleft}{\kern0pt}OF\ fg{\isacharparenleft}{\kern0pt}{\isadigit{2}}{\isacharparenright}{\kern0pt}{\isacharbrackright}{\kern0pt}{\isacharbrackright}{\kern0pt}{\isacharparenright}{\kern0pt}%
\endisatagproof
{\isafoldproof}%
%
\isadelimproof
\isanewline
%
\endisadelimproof
\isanewline
\isacommand{lemma}\isamarkupfalse%
\ integrable{\isacharunderscore}{\kern0pt}min{\isacharbrackleft}{\kern0pt}simp{\isacharcomma}{\kern0pt}\ intro{\isacharbrackright}{\kern0pt}{\isacharcolon}{\kern0pt}\isanewline
\ \ \isakeyword{fixes}\ f\ {\isacharcolon}{\kern0pt}{\isacharcolon}{\kern0pt}\ {\isachardoublequoteopen}{\isacharprime}{\kern0pt}a\ {\isasymRightarrow}\ {\isacharprime}{\kern0pt}b\ {\isacharcolon}{\kern0pt}{\isacharcolon}{\kern0pt}\ {\isacharbraceleft}{\kern0pt}second{\isacharunderscore}{\kern0pt}countable{\isacharunderscore}{\kern0pt}topology{\isacharcomma}{\kern0pt}\ banach{\isacharcomma}{\kern0pt}\ linorder{\isacharunderscore}{\kern0pt}topology{\isacharbraceright}{\kern0pt}{\isachardoublequoteclose}\isanewline
\ \ \isakeyword{assumes}\ {\isacharbrackleft}{\kern0pt}measurable{\isacharbrackright}{\kern0pt}{\isacharcolon}{\kern0pt}\ {\isachardoublequoteopen}integrable\ M\ f{\isachardoublequoteclose}\ {\isachardoublequoteopen}integrable\ M\ g{\isachardoublequoteclose}\isanewline
\ \ \isakeyword{shows}\ {\isachardoublequoteopen}integrable\ M\ {\isacharparenleft}{\kern0pt}{\isasymlambda}x{\isachardot}{\kern0pt}\ min\ {\isacharparenleft}{\kern0pt}f\ x{\isacharparenright}{\kern0pt}\ {\isacharparenleft}{\kern0pt}g\ x{\isacharparenright}{\kern0pt}{\isacharparenright}{\kern0pt}{\isachardoublequoteclose}\isanewline
%
\isadelimproof
%
\endisadelimproof
%
\isatagproof
\isacommand{proof}\isamarkupfalse%
\ {\isacharminus}{\kern0pt}\isanewline
\ \ \isacommand{have}\isamarkupfalse%
\ {\isachardoublequoteopen}norm\ {\isacharparenleft}{\kern0pt}min\ {\isacharparenleft}{\kern0pt}f\ x{\isacharparenright}{\kern0pt}\ {\isacharparenleft}{\kern0pt}g\ x{\isacharparenright}{\kern0pt}{\isacharparenright}{\kern0pt}\ {\isasymle}\ norm\ {\isacharparenleft}{\kern0pt}f\ x{\isacharparenright}{\kern0pt}\ {\isasymor}\ norm\ {\isacharparenleft}{\kern0pt}min\ {\isacharparenleft}{\kern0pt}f\ x{\isacharparenright}{\kern0pt}\ {\isacharparenleft}{\kern0pt}g\ x{\isacharparenright}{\kern0pt}{\isacharparenright}{\kern0pt}\ {\isasymle}\ norm\ {\isacharparenleft}{\kern0pt}g\ x{\isacharparenright}{\kern0pt}{\isachardoublequoteclose}\ \isakeyword{for}\ x\ \isacommand{by}\isamarkupfalse%
\ linarith\isanewline
\ \ \isacommand{thus}\isamarkupfalse%
\ {\isacharquery}{\kern0pt}thesis\ \isacommand{by}\isamarkupfalse%
\ {\isacharparenleft}{\kern0pt}intro\ integrable{\isacharunderscore}{\kern0pt}bound{\isacharbrackleft}{\kern0pt}OF\ integrable{\isacharunderscore}{\kern0pt}max{\isacharbrackleft}{\kern0pt}OF\ integrable{\isacharunderscore}{\kern0pt}norm{\isacharparenleft}{\kern0pt}{\isadigit{1}}{\isacharcomma}{\kern0pt}{\isadigit{1}}{\isacharparenright}{\kern0pt}{\isacharcomma}{\kern0pt}\ OF\ assms{\isacharbrackright}{\kern0pt}{\isacharbrackright}{\kern0pt}{\isacharcomma}{\kern0pt}\ auto{\isacharparenright}{\kern0pt}\isanewline
\isacommand{qed}\isamarkupfalse%
%
\endisatagproof
{\isafoldproof}%
%
\isadelimproof
\isanewline
%
\endisadelimproof
\isanewline
\isanewline
\isanewline
\isacommand{lemma}\isamarkupfalse%
\ integral{\isacharunderscore}{\kern0pt}nonneg{\isacharunderscore}{\kern0pt}AE{\isacharunderscore}{\kern0pt}banach{\isacharcolon}{\kern0pt}\ \ \ \ \ \ \ \ \ \ \ \ \ \ \ \ \ \ \ \ \ \ \ \ \isanewline
\ \ \isakeyword{fixes}\ f\ {\isacharcolon}{\kern0pt}{\isacharcolon}{\kern0pt}\ {\isachardoublequoteopen}{\isacharprime}{\kern0pt}a\ {\isasymRightarrow}\ {\isacharprime}{\kern0pt}b\ {\isacharcolon}{\kern0pt}{\isacharcolon}{\kern0pt}\ {\isacharbraceleft}{\kern0pt}second{\isacharunderscore}{\kern0pt}countable{\isacharunderscore}{\kern0pt}topology{\isacharcomma}{\kern0pt}\ banach{\isacharcomma}{\kern0pt}\ linorder{\isacharunderscore}{\kern0pt}topology{\isacharcomma}{\kern0pt}\ ordered{\isacharunderscore}{\kern0pt}real{\isacharunderscore}{\kern0pt}vector{\isacharbraceright}{\kern0pt}{\isachardoublequoteclose}\isanewline
\ \ \isakeyword{assumes}\ {\isacharbrackleft}{\kern0pt}measurable{\isacharbrackright}{\kern0pt}{\isacharcolon}{\kern0pt}\ {\isachardoublequoteopen}f\ {\isasymin}\ borel{\isacharunderscore}{\kern0pt}measurable\ M{\isachardoublequoteclose}\ \isakeyword{and}\ nonneg{\isacharcolon}{\kern0pt}\ {\isachardoublequoteopen}AE\ x\ in\ M{\isachardot}{\kern0pt}\ {\isadigit{0}}\ {\isasymle}\ f\ x{\isachardoublequoteclose}\isanewline
\ \ \isakeyword{shows}\ {\isachardoublequoteopen}{\isadigit{0}}\ {\isasymle}\ integral\isactrlsup L\ M\ f{\isachardoublequoteclose}\isanewline
%
\isadelimproof
%
\endisadelimproof
%
\isatagproof
\isacommand{proof}\isamarkupfalse%
\ cases\isanewline
\ \ \isacommand{assume}\isamarkupfalse%
\ integrable{\isacharcolon}{\kern0pt}\ {\isachardoublequoteopen}integrable\ M\ f{\isachardoublequoteclose}\isanewline
\ \ \isacommand{hence}\isamarkupfalse%
\ max{\isacharcolon}{\kern0pt}\ {\isachardoublequoteopen}{\isacharparenleft}{\kern0pt}{\isasymlambda}x{\isachardot}{\kern0pt}\ max\ {\isadigit{0}}\ {\isacharparenleft}{\kern0pt}f\ x{\isacharparenright}{\kern0pt}{\isacharparenright}{\kern0pt}\ {\isasymin}\ borel{\isacharunderscore}{\kern0pt}measurable\ M{\isachardoublequoteclose}\ {\isachardoublequoteopen}{\isasymAnd}x{\isachardot}{\kern0pt}\ {\isadigit{0}}\ {\isasymle}\ max\ {\isadigit{0}}\ {\isacharparenleft}{\kern0pt}f\ x{\isacharparenright}{\kern0pt}{\isachardoublequoteclose}\ {\isachardoublequoteopen}integrable\ M\ {\isacharparenleft}{\kern0pt}{\isasymlambda}x{\isachardot}{\kern0pt}\ max\ {\isadigit{0}}\ {\isacharparenleft}{\kern0pt}f\ x{\isacharparenright}{\kern0pt}{\isacharparenright}{\kern0pt}{\isachardoublequoteclose}\ \isacommand{by}\isamarkupfalse%
\ auto\isanewline
\ \ \isacommand{hence}\isamarkupfalse%
\ {\isachardoublequoteopen}{\isadigit{0}}\ {\isasymle}\ integral\isactrlsup L\ M\ {\isacharparenleft}{\kern0pt}{\isasymlambda}x{\isachardot}{\kern0pt}\ max\ {\isadigit{0}}\ {\isacharparenleft}{\kern0pt}f\ x{\isacharparenright}{\kern0pt}{\isacharparenright}{\kern0pt}{\isachardoublequoteclose}\isanewline
\ \ \isacommand{proof}\isamarkupfalse%
\ {\isacharminus}{\kern0pt}\isanewline
\ \ \isacommand{obtain}\isamarkupfalse%
\ s\ \isakeyword{where}\ {\isacharasterisk}{\kern0pt}{\isacharcolon}{\kern0pt}\ {\isachardoublequoteopen}{\isasymAnd}i{\isachardot}{\kern0pt}\ simple{\isacharunderscore}{\kern0pt}function\ M\ {\isacharparenleft}{\kern0pt}s\ i{\isacharparenright}{\kern0pt}{\isachardoublequoteclose}\ \isanewline
\ \ \ \ \ \ \ \ \ \ \ \ \ \ \ \ \ \ \ \ {\isachardoublequoteopen}{\isasymAnd}i{\isachardot}{\kern0pt}\ emeasure\ M\ {\isacharbraceleft}{\kern0pt}y\ {\isasymin}\ space\ M{\isachardot}{\kern0pt}\ s\ i\ y\ {\isasymnoteq}\ {\isadigit{0}}{\isacharbraceright}{\kern0pt}\ {\isasymnoteq}\ {\isasyminfinity}{\isachardoublequoteclose}\ \isanewline
\ \ \ \ \ \ \ \ \ \ \ \ \ \ \ \ \ \ \ \ {\isachardoublequoteopen}{\isasymAnd}x{\isachardot}{\kern0pt}\ x\ {\isasymin}\ space\ M\ {\isasymLongrightarrow}\ {\isacharparenleft}{\kern0pt}{\isasymlambda}i{\isachardot}{\kern0pt}\ s\ i\ x{\isacharparenright}{\kern0pt}\ {\isasymlonglonglongrightarrow}\ f\ x{\isachardoublequoteclose}\ \isanewline
\ \ \ \ \ \ \ \ \ \ \ \ \ \ \ \ \ \ \ \ {\isachardoublequoteopen}{\isasymAnd}x\ i{\isachardot}{\kern0pt}\ x\ {\isasymin}\ space\ M\ {\isasymLongrightarrow}\ norm\ {\isacharparenleft}{\kern0pt}s\ i\ x{\isacharparenright}{\kern0pt}\ {\isasymle}\ {\isadigit{2}}\ {\isacharasterisk}{\kern0pt}\ norm\ {\isacharparenleft}{\kern0pt}f\ x{\isacharparenright}{\kern0pt}{\isachardoublequoteclose}\ \isacommand{using}\isamarkupfalse%
\ integrable{\isacharunderscore}{\kern0pt}implies{\isacharunderscore}{\kern0pt}simple{\isacharunderscore}{\kern0pt}function{\isacharunderscore}{\kern0pt}sequence{\isacharbrackleft}{\kern0pt}OF\ integrable{\isacharbrackright}{\kern0pt}\ \isacommand{by}\isamarkupfalse%
\ blast\isanewline
\ \ \ \ \isacommand{have}\isamarkupfalse%
\ simple{\isacharcolon}{\kern0pt}\ {\isachardoublequoteopen}{\isasymAnd}i{\isachardot}{\kern0pt}\ simple{\isacharunderscore}{\kern0pt}function\ M\ {\isacharparenleft}{\kern0pt}{\isasymlambda}x{\isachardot}{\kern0pt}\ max\ {\isadigit{0}}\ {\isacharparenleft}{\kern0pt}s\ i\ x{\isacharparenright}{\kern0pt}{\isacharparenright}{\kern0pt}{\isachardoublequoteclose}\ \isacommand{using}\isamarkupfalse%
\ {\isacharasterisk}{\kern0pt}\ \isacommand{by}\isamarkupfalse%
\ fast\isanewline
\ \ \ \ \isacommand{have}\isamarkupfalse%
\ {\isachardoublequoteopen}{\isasymAnd}i{\isachardot}{\kern0pt}\ {\isacharbraceleft}{\kern0pt}y\ {\isasymin}\ space\ M{\isachardot}{\kern0pt}\ max\ {\isadigit{0}}\ {\isacharparenleft}{\kern0pt}s\ i\ y{\isacharparenright}{\kern0pt}\ {\isasymnoteq}\ {\isadigit{0}}{\isacharbraceright}{\kern0pt}\ {\isasymsubseteq}\ {\isacharbraceleft}{\kern0pt}y\ {\isasymin}\ space\ M{\isachardot}{\kern0pt}\ s\ i\ y\ {\isasymnoteq}\ {\isadigit{0}}{\isacharbraceright}{\kern0pt}{\isachardoublequoteclose}\ \isacommand{unfolding}\isamarkupfalse%
\ max{\isacharunderscore}{\kern0pt}def\ \isacommand{by}\isamarkupfalse%
\ force\isanewline
\ \ \ \ \isacommand{moreover}\isamarkupfalse%
\ \isacommand{have}\isamarkupfalse%
\ {\isachardoublequoteopen}{\isasymAnd}i{\isachardot}{\kern0pt}\ {\isacharbraceleft}{\kern0pt}y\ {\isasymin}\ space\ M{\isachardot}{\kern0pt}\ s\ i\ y\ {\isasymnoteq}\ {\isadigit{0}}{\isacharbraceright}{\kern0pt}\ {\isasymin}\ sets\ M{\isachardoublequoteclose}\ \isacommand{using}\isamarkupfalse%
\ {\isacharasterisk}{\kern0pt}\ \isacommand{by}\isamarkupfalse%
\ measurable\isanewline
\ \ \ \ \isacommand{ultimately}\isamarkupfalse%
\ \isacommand{have}\isamarkupfalse%
\ {\isachardoublequoteopen}{\isasymAnd}i{\isachardot}{\kern0pt}\ emeasure\ M\ {\isacharbraceleft}{\kern0pt}y\ {\isasymin}\ space\ M{\isachardot}{\kern0pt}\ max\ {\isadigit{0}}\ {\isacharparenleft}{\kern0pt}s\ i\ y{\isacharparenright}{\kern0pt}\ {\isasymnoteq}\ {\isadigit{0}}{\isacharbraceright}{\kern0pt}\ {\isasymle}\ emeasure\ M\ {\isacharbraceleft}{\kern0pt}y\ {\isasymin}\ space\ M{\isachardot}{\kern0pt}\ s\ i\ y\ {\isasymnoteq}\ {\isadigit{0}}{\isacharbraceright}{\kern0pt}{\isachardoublequoteclose}\ \isacommand{by}\isamarkupfalse%
\ {\isacharparenleft}{\kern0pt}rule\ emeasure{\isacharunderscore}{\kern0pt}mono{\isacharparenright}{\kern0pt}\ \isanewline
\ \ \ \ \isacommand{hence}\isamarkupfalse%
\ {\isacharasterisk}{\kern0pt}{\isacharasterisk}{\kern0pt}{\isacharcolon}{\kern0pt}{\isachardoublequoteopen}{\isasymAnd}i{\isachardot}{\kern0pt}\ emeasure\ M\ {\isacharbraceleft}{\kern0pt}y\ {\isasymin}\ space\ M{\isachardot}{\kern0pt}\ max\ {\isadigit{0}}\ {\isacharparenleft}{\kern0pt}s\ i\ y{\isacharparenright}{\kern0pt}\ {\isasymnoteq}\ {\isadigit{0}}{\isacharbraceright}{\kern0pt}\ {\isasymnoteq}\ {\isasyminfinity}{\isachardoublequoteclose}\ \isacommand{using}\isamarkupfalse%
\ {\isacharasterisk}{\kern0pt}{\isacharparenleft}{\kern0pt}{\isadigit{2}}{\isacharparenright}{\kern0pt}\ \isacommand{by}\isamarkupfalse%
\ {\isacharparenleft}{\kern0pt}auto\ intro{\isacharcolon}{\kern0pt}\ order{\isachardot}{\kern0pt}strict{\isacharunderscore}{\kern0pt}trans{\isadigit{1}}\ simp\ add{\isacharcolon}{\kern0pt}\ \ top{\isachardot}{\kern0pt}not{\isacharunderscore}{\kern0pt}eq{\isacharunderscore}{\kern0pt}extremum{\isacharparenright}{\kern0pt}\isanewline
\ \ \ \ \isacommand{have}\isamarkupfalse%
\ {\isachardoublequoteopen}{\isasymAnd}x{\isachardot}{\kern0pt}\ x\ {\isasymin}\ space\ M\ {\isasymLongrightarrow}\ {\isacharparenleft}{\kern0pt}{\isasymlambda}i{\isachardot}{\kern0pt}\ max\ {\isadigit{0}}\ {\isacharparenleft}{\kern0pt}s\ i\ x{\isacharparenright}{\kern0pt}{\isacharparenright}{\kern0pt}\ {\isasymlonglonglongrightarrow}\ max\ {\isadigit{0}}\ {\isacharparenleft}{\kern0pt}f\ x{\isacharparenright}{\kern0pt}{\isachardoublequoteclose}\ \isacommand{using}\isamarkupfalse%
\ {\isacharasterisk}{\kern0pt}{\isacharparenleft}{\kern0pt}{\isadigit{3}}{\isacharparenright}{\kern0pt}\ tendsto{\isacharunderscore}{\kern0pt}max\ \isacommand{by}\isamarkupfalse%
\ blast\isanewline
\ \ \ \ \isacommand{moreover}\isamarkupfalse%
\ \isacommand{have}\isamarkupfalse%
\ {\isachardoublequoteopen}{\isasymAnd}x\ i{\isachardot}{\kern0pt}\ x\ {\isasymin}\ space\ M\ {\isasymLongrightarrow}\ norm\ {\isacharparenleft}{\kern0pt}max\ {\isadigit{0}}\ {\isacharparenleft}{\kern0pt}s\ i\ x{\isacharparenright}{\kern0pt}{\isacharparenright}{\kern0pt}\ {\isasymle}\ norm\ {\isacharparenleft}{\kern0pt}{\isadigit{2}}\ {\isacharasterisk}{\kern0pt}\isactrlsub R\ f\ x{\isacharparenright}{\kern0pt}{\isachardoublequoteclose}\ \isacommand{using}\isamarkupfalse%
\ {\isacharasterisk}{\kern0pt}{\isacharparenleft}{\kern0pt}{\isadigit{4}}{\isacharparenright}{\kern0pt}\ \isacommand{unfolding}\isamarkupfalse%
\ max{\isacharunderscore}{\kern0pt}def\ \isacommand{by}\isamarkupfalse%
\ auto\isanewline
\ \ \ \ \isacommand{ultimately}\isamarkupfalse%
\ \isacommand{have}\isamarkupfalse%
\ tendsto{\isacharcolon}{\kern0pt}\ {\isachardoublequoteopen}{\isacharparenleft}{\kern0pt}{\isasymlambda}i{\isachardot}{\kern0pt}\ integral\isactrlsup L\ M\ {\isacharparenleft}{\kern0pt}{\isasymlambda}x{\isachardot}{\kern0pt}\ max\ {\isadigit{0}}\ {\isacharparenleft}{\kern0pt}s\ i\ x{\isacharparenright}{\kern0pt}{\isacharparenright}{\kern0pt}{\isacharparenright}{\kern0pt}\ {\isasymlonglonglongrightarrow}\ integral\isactrlsup L\ M\ {\isacharparenleft}{\kern0pt}{\isasymlambda}x{\isachardot}{\kern0pt}\ max\ {\isadigit{0}}\ {\isacharparenleft}{\kern0pt}f\ x{\isacharparenright}{\kern0pt}{\isacharparenright}{\kern0pt}{\isachardoublequoteclose}\ \isanewline
\ \ \ \ \ \ \isacommand{using}\isamarkupfalse%
\ borel{\isacharunderscore}{\kern0pt}measurable{\isacharunderscore}{\kern0pt}simple{\isacharunderscore}{\kern0pt}function\ simple\ integrable\ \isacommand{by}\isamarkupfalse%
\ {\isacharparenleft}{\kern0pt}intro\ integral{\isacharunderscore}{\kern0pt}dominated{\isacharunderscore}{\kern0pt}convergence{\isacharbrackleft}{\kern0pt}OF\ max{\isacharparenleft}{\kern0pt}{\isadigit{1}}{\isacharparenright}{\kern0pt}\ {\isacharunderscore}{\kern0pt}\ integrable{\isacharunderscore}{\kern0pt}norm{\isacharbrackleft}{\kern0pt}OF\ integrable{\isacharunderscore}{\kern0pt}scaleR{\isacharunderscore}{\kern0pt}right{\isacharbrackright}{\kern0pt}{\isacharcomma}{\kern0pt}\ of\ {\isacharunderscore}{\kern0pt}\ {\isadigit{2}}\ f{\isacharbrackright}{\kern0pt}{\isacharcomma}{\kern0pt}\ blast{\isacharplus}{\kern0pt}{\isacharparenright}{\kern0pt}\isanewline
\ \ \ \ \isacommand{{\isacharbraceleft}{\kern0pt}}\isamarkupfalse%
\isanewline
\ \ \ \ \ \ \isacommand{fix}\isamarkupfalse%
\ h\ {\isacharcolon}{\kern0pt}{\isacharcolon}{\kern0pt}\ {\isachardoublequoteopen}{\isacharprime}{\kern0pt}a\ {\isasymRightarrow}\ {\isacharprime}{\kern0pt}b\ {\isacharcolon}{\kern0pt}{\isacharcolon}{\kern0pt}\ {\isacharbraceleft}{\kern0pt}second{\isacharunderscore}{\kern0pt}countable{\isacharunderscore}{\kern0pt}topology{\isacharcomma}{\kern0pt}\ banach{\isacharcomma}{\kern0pt}\ linorder{\isacharunderscore}{\kern0pt}topology{\isacharcomma}{\kern0pt}\ ordered{\isacharunderscore}{\kern0pt}real{\isacharunderscore}{\kern0pt}vector{\isacharbraceright}{\kern0pt}{\isachardoublequoteclose}\ \isanewline
\ \ \ \ \ \ \isacommand{assume}\isamarkupfalse%
\ {\isachardoublequoteopen}simple{\isacharunderscore}{\kern0pt}function\ M\ h{\isachardoublequoteclose}\ {\isachardoublequoteopen}emeasure\ M\ {\isacharbraceleft}{\kern0pt}y\ {\isasymin}\ space\ M{\isachardot}{\kern0pt}\ h\ y\ {\isasymnoteq}\ {\isadigit{0}}{\isacharbraceright}{\kern0pt}\ {\isasymnoteq}\ {\isasyminfinity}{\isachardoublequoteclose}\ {\isachardoublequoteopen}{\isasymAnd}x{\isachardot}{\kern0pt}\ x\ {\isasymin}\ space\ M\ {\isasymlongrightarrow}\ h\ x\ {\isasymge}\ {\isadigit{0}}{\isachardoublequoteclose}\isanewline
\ \ \ \ \ \ \isacommand{hence}\isamarkupfalse%
\ {\isacharasterisk}{\kern0pt}{\isacharcolon}{\kern0pt}\ {\isachardoublequoteopen}integral\isactrlsup L\ M\ h\ {\isasymge}\ {\isadigit{0}}{\isachardoublequoteclose}\isanewline
\ \ \ \ \ \ \isacommand{proof}\isamarkupfalse%
\ {\isacharparenleft}{\kern0pt}induct\ rule{\isacharcolon}{\kern0pt}\ simple{\isacharunderscore}{\kern0pt}integrable{\isacharunderscore}{\kern0pt}function{\isacharunderscore}{\kern0pt}induct{\isacharunderscore}{\kern0pt}nn{\isacharparenright}{\kern0pt}\isanewline
\ \ \ \ \ \ \ \ \isacommand{case}\isamarkupfalse%
\ {\isacharparenleft}{\kern0pt}cong\ f\ g{\isacharparenright}{\kern0pt}\ \ \ \ \ \ \ \ \ \ \ \ \ \ \ \ \ \ \ \isanewline
\ \ \ \ \ \ \ \ \isacommand{then}\isamarkupfalse%
\ \isacommand{show}\isamarkupfalse%
\ {\isacharquery}{\kern0pt}case\ \isacommand{using}\isamarkupfalse%
\ Bochner{\isacharunderscore}{\kern0pt}Integration{\isachardot}{\kern0pt}integral{\isacharunderscore}{\kern0pt}cong\ \isacommand{by}\isamarkupfalse%
\ force\isanewline
\ \ \ \ \ \ \isacommand{next}\isamarkupfalse%
\isanewline
\ \ \ \ \ \ \ \ \isacommand{case}\isamarkupfalse%
\ {\isacharparenleft}{\kern0pt}indicator\ A\ y{\isacharparenright}{\kern0pt}\isanewline
\ \ \ \ \ \ \ \ \isacommand{hence}\isamarkupfalse%
\ {\isachardoublequoteopen}A\ {\isasymnoteq}\ {\isacharbraceleft}{\kern0pt}{\isacharbraceright}{\kern0pt}\ {\isasymLongrightarrow}\ y\ {\isasymge}\ {\isadigit{0}}{\isachardoublequoteclose}\ \isacommand{using}\isamarkupfalse%
\ sets{\isachardot}{\kern0pt}sets{\isacharunderscore}{\kern0pt}into{\isacharunderscore}{\kern0pt}space\ \isacommand{by}\isamarkupfalse%
\ fastforce\isanewline
\ \ \ \ \ \ \ \ \isacommand{then}\isamarkupfalse%
\ \isacommand{show}\isamarkupfalse%
\ {\isacharquery}{\kern0pt}case\ \isacommand{using}\isamarkupfalse%
\ indicator\ \isacommand{by}\isamarkupfalse%
\ {\isacharparenleft}{\kern0pt}cases\ {\isachardoublequoteopen}A\ {\isacharequal}{\kern0pt}\ {\isacharbraceleft}{\kern0pt}{\isacharbraceright}{\kern0pt}{\isachardoublequoteclose}{\isacharcomma}{\kern0pt}\ auto\ simp\ add{\isacharcolon}{\kern0pt}\ scaleR{\isacharunderscore}{\kern0pt}nonneg{\isacharunderscore}{\kern0pt}nonneg{\isacharparenright}{\kern0pt}\isanewline
\ \ \ \ \ \ \isacommand{next}\isamarkupfalse%
\isanewline
\ \ \ \ \ \ \ \ \isacommand{case}\isamarkupfalse%
\ {\isacharparenleft}{\kern0pt}add\ f\ g{\isacharparenright}{\kern0pt}\isanewline
\ \ \ \ \ \ \ \ \isacommand{then}\isamarkupfalse%
\ \isacommand{show}\isamarkupfalse%
\ {\isacharquery}{\kern0pt}case\ \isacommand{by}\isamarkupfalse%
\ {\isacharparenleft}{\kern0pt}simp\ add{\isacharcolon}{\kern0pt}\ integrable{\isacharunderscore}{\kern0pt}simple{\isacharunderscore}{\kern0pt}function{\isacharparenright}{\kern0pt}\isanewline
\ \ \ \ \ \ \isacommand{qed}\isamarkupfalse%
\isanewline
\ \ \ \ \isacommand{{\isacharbraceright}{\kern0pt}}\isamarkupfalse%
\isanewline
\ \ \ \ \isacommand{thus}\isamarkupfalse%
\ {\isacharquery}{\kern0pt}thesis\ \isacommand{using}\isamarkupfalse%
\ LIMSEQ{\isacharunderscore}{\kern0pt}le{\isacharunderscore}{\kern0pt}const{\isacharbrackleft}{\kern0pt}OF\ tendsto{\isacharcomma}{\kern0pt}\ of\ {\isadigit{0}}{\isacharbrackright}{\kern0pt}\ {\isacharasterisk}{\kern0pt}{\isacharasterisk}{\kern0pt}\ simple\ \isacommand{by}\isamarkupfalse%
\ fastforce\isanewline
\ \ \isacommand{qed}\isamarkupfalse%
\isanewline
\ \ \isacommand{also}\isamarkupfalse%
\ \isacommand{have}\isamarkupfalse%
\ {\isachardoublequoteopen}{\isasymdots}\ {\isacharequal}{\kern0pt}\ integral\isactrlsup L\ M\ f{\isachardoublequoteclose}\ \isacommand{using}\isamarkupfalse%
\ nonneg\ \isacommand{by}\isamarkupfalse%
\ {\isacharparenleft}{\kern0pt}auto\ intro{\isacharcolon}{\kern0pt}\ integral{\isacharunderscore}{\kern0pt}cong{\isacharunderscore}{\kern0pt}AE{\isacharparenright}{\kern0pt}\isanewline
\ \ \isacommand{finally}\isamarkupfalse%
\ \isacommand{show}\isamarkupfalse%
\ {\isacharquery}{\kern0pt}thesis\ \isacommand{{\isachardot}{\kern0pt}}\isamarkupfalse%
\isanewline
\isacommand{qed}\isamarkupfalse%
\ {\isacharparenleft}{\kern0pt}simp\ add{\isacharcolon}{\kern0pt}\ not{\isacharunderscore}{\kern0pt}integrable{\isacharunderscore}{\kern0pt}integral{\isacharunderscore}{\kern0pt}eq{\isacharparenright}{\kern0pt}%
\endisatagproof
{\isafoldproof}%
%
\isadelimproof
\isanewline
%
\endisadelimproof
\isanewline
\isacommand{lemma}\isamarkupfalse%
\ integral{\isacharunderscore}{\kern0pt}mono{\isacharunderscore}{\kern0pt}AE{\isacharunderscore}{\kern0pt}banach{\isacharcolon}{\kern0pt}\isanewline
\ \ \isakeyword{fixes}\ f\ g\ {\isacharcolon}{\kern0pt}{\isacharcolon}{\kern0pt}\ {\isachardoublequoteopen}{\isacharprime}{\kern0pt}a\ {\isasymRightarrow}\ {\isacharprime}{\kern0pt}b\ {\isacharcolon}{\kern0pt}{\isacharcolon}{\kern0pt}\ {\isacharbraceleft}{\kern0pt}second{\isacharunderscore}{\kern0pt}countable{\isacharunderscore}{\kern0pt}topology{\isacharcomma}{\kern0pt}\ banach{\isacharcomma}{\kern0pt}\ linorder{\isacharunderscore}{\kern0pt}topology{\isacharcomma}{\kern0pt}\ ordered{\isacharunderscore}{\kern0pt}real{\isacharunderscore}{\kern0pt}vector{\isacharbraceright}{\kern0pt}{\isachardoublequoteclose}\isanewline
\ \ \isakeyword{assumes}\ {\isachardoublequoteopen}integrable\ M\ f{\isachardoublequoteclose}\ {\isachardoublequoteopen}integrable\ M\ g{\isachardoublequoteclose}\ {\isachardoublequoteopen}AE\ x\ in\ M{\isachardot}{\kern0pt}\ f\ x\ {\isasymle}\ g\ x{\isachardoublequoteclose}\isanewline
\ \ \isakeyword{shows}\ {\isachardoublequoteopen}integral\isactrlsup L\ M\ f\ {\isasymle}\ integral\isactrlsup L\ M\ g{\isachardoublequoteclose}\isanewline
%
\isadelimproof
\ \ %
\endisadelimproof
%
\isatagproof
\isacommand{using}\isamarkupfalse%
\ integral{\isacharunderscore}{\kern0pt}nonneg{\isacharunderscore}{\kern0pt}AE{\isacharunderscore}{\kern0pt}banach{\isacharbrackleft}{\kern0pt}of\ {\isachardoublequoteopen}{\isasymlambda}x{\isachardot}{\kern0pt}\ g\ x\ {\isacharminus}{\kern0pt}\ f\ x{\isachardoublequoteclose}{\isacharbrackright}{\kern0pt}\ assms\ Bochner{\isacharunderscore}{\kern0pt}Integration{\isachardot}{\kern0pt}integral{\isacharunderscore}{\kern0pt}diff{\isacharbrackleft}{\kern0pt}OF\ assms{\isacharparenleft}{\kern0pt}{\isadigit{1}}{\isacharcomma}{\kern0pt}{\isadigit{2}}{\isacharparenright}{\kern0pt}{\isacharbrackright}{\kern0pt}\ \isacommand{by}\isamarkupfalse%
\ force%
\endisatagproof
{\isafoldproof}%
%
\isadelimproof
\isanewline
%
\endisadelimproof
\isanewline
\isacommand{lemma}\isamarkupfalse%
\ integral{\isacharunderscore}{\kern0pt}mono{\isacharunderscore}{\kern0pt}banach{\isacharcolon}{\kern0pt}\isanewline
\ \ \isakeyword{fixes}\ f\ g\ {\isacharcolon}{\kern0pt}{\isacharcolon}{\kern0pt}\ {\isachardoublequoteopen}{\isacharprime}{\kern0pt}a\ {\isasymRightarrow}\ {\isacharprime}{\kern0pt}b\ {\isacharcolon}{\kern0pt}{\isacharcolon}{\kern0pt}\ {\isacharbraceleft}{\kern0pt}second{\isacharunderscore}{\kern0pt}countable{\isacharunderscore}{\kern0pt}topology{\isacharcomma}{\kern0pt}\ banach{\isacharcomma}{\kern0pt}\ linorder{\isacharunderscore}{\kern0pt}topology{\isacharcomma}{\kern0pt}\ ordered{\isacharunderscore}{\kern0pt}real{\isacharunderscore}{\kern0pt}vector{\isacharbraceright}{\kern0pt}{\isachardoublequoteclose}\isanewline
\ \ \isakeyword{assumes}\ {\isachardoublequoteopen}integrable\ M\ f{\isachardoublequoteclose}\ {\isachardoublequoteopen}integrable\ M\ g{\isachardoublequoteclose}\ {\isachardoublequoteopen}{\isasymAnd}x{\isachardot}{\kern0pt}\ x\ {\isasymin}\ space\ M\ {\isasymLongrightarrow}\ f\ x\ {\isasymle}\ g\ x{\isachardoublequoteclose}\isanewline
\ \ \isakeyword{shows}\ {\isachardoublequoteopen}integral\isactrlsup L\ M\ f\ {\isasymle}\ integral\isactrlsup L\ M\ g{\isachardoublequoteclose}\isanewline
%
\isadelimproof
\ \ %
\endisadelimproof
%
\isatagproof
\isacommand{using}\isamarkupfalse%
\ integral{\isacharunderscore}{\kern0pt}mono{\isacharunderscore}{\kern0pt}AE{\isacharunderscore}{\kern0pt}banach\ assms\ \isacommand{by}\isamarkupfalse%
\ blast%
\endisatagproof
{\isafoldproof}%
%
\isadelimproof
\isanewline
%
\endisadelimproof
%
\isadelimtheory
\isanewline
%
\endisadelimtheory
%
\isatagtheory
\isacommand{end}\isamarkupfalse%
%
\endisatagtheory
{\isafoldtheory}%
%
\isadelimtheory
%
\endisadelimtheory
%
\end{isabellebody}%
\endinput
%:%file=Bochner_Integration_Addendum.tex%:%
%:%10=1%:%
%:%11=1%:%
%:%12=2%:%
%:%13=3%:%
%:%27=5%:%
%:%37=7%:%
%:%38=7%:%
%:%39=8%:%
%:%40=9%:%
%:%41=10%:%
%:%42=11%:%
%:%43=12%:%
%:%44=13%:%
%:%51=14%:%
%:%52=14%:%
%:%53=15%:%
%:%54=15%:%
%:%55=15%:%
%:%56=15%:%
%:%57=15%:%
%:%58=16%:%
%:%59=16%:%
%:%60=16%:%
%:%61=16%:%
%:%62=16%:%
%:%63=17%:%
%:%64=17%:%
%:%65=18%:%
%:%66=18%:%
%:%67=19%:%
%:%68=19%:%
%:%69=19%:%
%:%70=19%:%
%:%71=20%:%
%:%72=20%:%
%:%73=20%:%
%:%74=20%:%
%:%75=20%:%
%:%76=21%:%
%:%77=21%:%
%:%78=21%:%
%:%79=21%:%
%:%80=21%:%
%:%81=22%:%
%:%82=22%:%
%:%83=22%:%
%:%84=23%:%
%:%85=23%:%
%:%86=24%:%
%:%87=24%:%
%:%88=24%:%
%:%89=24%:%
%:%90=25%:%
%:%96=25%:%
%:%99=26%:%
%:%100=27%:%
%:%101=27%:%
%:%102=28%:%
%:%103=29%:%
%:%104=30%:%
%:%105=31%:%
%:%112=32%:%
%:%113=32%:%
%:%114=33%:%
%:%115=33%:%
%:%116=34%:%
%:%117=34%:%
%:%118=35%:%
%:%119=35%:%
%:%120=35%:%
%:%121=35%:%
%:%122=35%:%
%:%123=36%:%
%:%124=36%:%
%:%125=36%:%
%:%126=36%:%
%:%127=36%:%
%:%128=37%:%
%:%129=37%:%
%:%130=37%:%
%:%131=37%:%
%:%132=38%:%
%:%138=38%:%
%:%141=39%:%
%:%142=40%:%
%:%143=40%:%
%:%144=41%:%
%:%145=42%:%
%:%146=43%:%
%:%149=44%:%
%:%153=44%:%
%:%154=44%:%
%:%159=44%:%
%:%162=45%:%
%:%163=46%:%
%:%164=46%:%
%:%165=47%:%
%:%166=47%:%
%:%167=48%:%
%:%168=49%:%
%:%171=50%:%
%:%176=51%:%
%:%177=51%:%
%:%178=52%:%
%:%179=53%:%
%:%180=54%:%
%:%182=56%:%
%:%183=57%:%
%:%184=58%:%
%:%187=61%:%
%:%188=62%:%
%:%193=62%:%
%:%202=63%:%
%:%203=63%:%
%:%204=64%:%
%:%205=64%:%
%:%206=65%:%
%:%207=65%:%
%:%208=65%:%
%:%209=65%:%
%:%210=66%:%
%:%211=66%:%
%:%212=66%:%
%:%213=66%:%
%:%214=67%:%
%:%215=67%:%
%:%216=67%:%
%:%217=68%:%
%:%218=69%:%
%:%219=70%:%
%:%220=70%:%
%:%221=71%:%
%:%222=71%:%
%:%223=72%:%
%:%224=72%:%
%:%225=73%:%
%:%226=73%:%
%:%227=74%:%
%:%228=74%:%
%:%229=75%:%
%:%230=75%:%
%:%231=75%:%
%:%232=75%:%
%:%233=75%:%
%:%234=76%:%
%:%235=76%:%
%:%236=77%:%
%:%237=77%:%
%:%238=78%:%
%:%239=78%:%
%:%240=78%:%
%:%241=78%:%
%:%242=79%:%
%:%243=79%:%
%:%244=80%:%
%:%245=80%:%
%:%246=81%:%
%:%247=81%:%
%:%248=81%:%
%:%249=81%:%
%:%250=82%:%
%:%251=82%:%
%:%252=83%:%
%:%253=83%:%
%:%254=84%:%
%:%255=84%:%
%:%256=85%:%
%:%257=85%:%
%:%258=85%:%
%:%259=85%:%
%:%260=86%:%
%:%261=86%:%
%:%262=86%:%
%:%263=86%:%
%:%264=87%:%
%:%265=87%:%
%:%266=87%:%
%:%267=87%:%
%:%268=87%:%
%:%269=88%:%
%:%270=88%:%
%:%271=88%:%
%:%272=88%:%
%:%273=88%:%
%:%274=89%:%
%:%275=89%:%
%:%276=89%:%
%:%277=90%:%
%:%278=91%:%
%:%279=91%:%
%:%280=91%:%
%:%281=92%:%
%:%282=92%:%
%:%283=92%:%
%:%284=92%:%
%:%285=93%:%
%:%286=93%:%
%:%287=94%:%
%:%288=94%:%
%:%289=95%:%
%:%290=95%:%
%:%291=95%:%
%:%292=96%:%
%:%293=96%:%
%:%294=97%:%
%:%295=97%:%
%:%296=98%:%
%:%297=98%:%
%:%298=99%:%
%:%299=99%:%
%:%300=99%:%
%:%301=99%:%
%:%302=99%:%
%:%303=99%:%
%:%304=100%:%
%:%305=100%:%
%:%306=101%:%
%:%307=101%:%
%:%308=102%:%
%:%309=102%:%
%:%310=103%:%
%:%311=103%:%
%:%312=104%:%
%:%313=104%:%
%:%314=105%:%
%:%315=105%:%
%:%316=105%:%
%:%317=105%:%
%:%318=105%:%
%:%319=106%:%
%:%320=106%:%
%:%321=106%:%
%:%322=107%:%
%:%323=107%:%
%:%324=107%:%
%:%325=107%:%
%:%326=108%:%
%:%327=108%:%
%:%328=109%:%
%:%329=109%:%
%:%330=110%:%
%:%331=110%:%
%:%332=110%:%
%:%333=110%:%
%:%334=111%:%
%:%335=111%:%
%:%336=112%:%
%:%337=112%:%
%:%338=112%:%
%:%339=112%:%
%:%340=113%:%
%:%341=113%:%
%:%342=114%:%
%:%343=114%:%
%:%344=115%:%
%:%345=115%:%
%:%346=116%:%
%:%347=116%:%
%:%348=116%:%
%:%349=117%:%
%:%350=117%:%
%:%351=118%:%
%:%352=118%:%
%:%353=119%:%
%:%354=119%:%
%:%355=120%:%
%:%356=120%:%
%:%357=120%:%
%:%358=120%:%
%:%359=120%:%
%:%360=120%:%
%:%361=121%:%
%:%362=121%:%
%:%363=122%:%
%:%364=122%:%
%:%365=123%:%
%:%366=123%:%
%:%367=123%:%
%:%368=123%:%
%:%369=123%:%
%:%370=123%:%
%:%371=124%:%
%:%372=124%:%
%:%373=125%:%
%:%374=125%:%
%:%375=126%:%
%:%376=126%:%
%:%377=127%:%
%:%378=127%:%
%:%379=127%:%
%:%380=128%:%
%:%381=128%:%
%:%382=129%:%
%:%383=129%:%
%:%384=130%:%
%:%385=130%:%
%:%386=131%:%
%:%387=131%:%
%:%388=131%:%
%:%389=131%:%
%:%390=131%:%
%:%391=131%:%
%:%392=132%:%
%:%393=132%:%
%:%394=133%:%
%:%395=133%:%
%:%396=134%:%
%:%397=134%:%
%:%398=135%:%
%:%399=135%:%
%:%400=135%:%
%:%401=136%:%
%:%402=136%:%
%:%403=136%:%
%:%404=137%:%
%:%405=137%:%
%:%406=137%:%
%:%407=138%:%
%:%408=138%:%
%:%409=138%:%
%:%410=138%:%
%:%411=138%:%
%:%412=139%:%
%:%413=139%:%
%:%414=139%:%
%:%415=139%:%
%:%416=140%:%
%:%417=140%:%
%:%418=141%:%
%:%419=141%:%
%:%420=142%:%
%:%421=142%:%
%:%422=143%:%
%:%423=143%:%
%:%424=143%:%
%:%425=143%:%
%:%426=143%:%
%:%427=144%:%
%:%428=144%:%
%:%429=144%:%
%:%430=144%:%
%:%431=144%:%
%:%432=145%:%
%:%433=145%:%
%:%434=145%:%
%:%435=145%:%
%:%436=146%:%
%:%442=146%:%
%:%445=147%:%
%:%448=148%:%
%:%453=149%:%
%:%454=149%:%
%:%455=150%:%
%:%456=151%:%
%:%457=152%:%
%:%458=153%:%
%:%459=154%:%
%:%462=157%:%
%:%463=158%:%
%:%468=158%:%
%:%477=159%:%
%:%478=159%:%
%:%479=160%:%
%:%480=160%:%
%:%481=161%:%
%:%482=161%:%
%:%483=161%:%
%:%484=161%:%
%:%485=162%:%
%:%486=162%:%
%:%487=162%:%
%:%488=162%:%
%:%489=163%:%
%:%490=163%:%
%:%491=163%:%
%:%492=164%:%
%:%493=165%:%
%:%494=166%:%
%:%495=167%:%
%:%496=167%:%
%:%497=168%:%
%:%498=168%:%
%:%499=169%:%
%:%500=169%:%
%:%501=170%:%
%:%502=170%:%
%:%503=171%:%
%:%504=171%:%
%:%505=172%:%
%:%506=172%:%
%:%507=172%:%
%:%508=172%:%
%:%509=172%:%
%:%510=173%:%
%:%511=173%:%
%:%512=174%:%
%:%513=174%:%
%:%514=175%:%
%:%515=175%:%
%:%516=175%:%
%:%517=175%:%
%:%518=176%:%
%:%519=176%:%
%:%520=177%:%
%:%521=177%:%
%:%522=178%:%
%:%523=178%:%
%:%524=178%:%
%:%525=178%:%
%:%526=179%:%
%:%527=179%:%
%:%528=180%:%
%:%529=180%:%
%:%530=181%:%
%:%531=181%:%
%:%532=181%:%
%:%533=181%:%
%:%534=182%:%
%:%535=182%:%
%:%536=183%:%
%:%537=183%:%
%:%538=184%:%
%:%539=184%:%
%:%540=185%:%
%:%541=185%:%
%:%542=185%:%
%:%543=185%:%
%:%544=186%:%
%:%545=186%:%
%:%546=186%:%
%:%547=186%:%
%:%548=187%:%
%:%549=187%:%
%:%550=187%:%
%:%551=187%:%
%:%552=187%:%
%:%553=188%:%
%:%554=188%:%
%:%555=188%:%
%:%556=188%:%
%:%557=188%:%
%:%558=189%:%
%:%559=189%:%
%:%560=189%:%
%:%561=190%:%
%:%562=191%:%
%:%563=191%:%
%:%564=191%:%
%:%565=191%:%
%:%566=192%:%
%:%567=192%:%
%:%568=192%:%
%:%569=193%:%
%:%570=193%:%
%:%571=193%:%
%:%572=193%:%
%:%573=194%:%
%:%574=194%:%
%:%575=195%:%
%:%576=195%:%
%:%577=196%:%
%:%578=196%:%
%:%579=197%:%
%:%580=197%:%
%:%581=198%:%
%:%582=198%:%
%:%583=199%:%
%:%584=199%:%
%:%585=199%:%
%:%586=199%:%
%:%587=199%:%
%:%588=199%:%
%:%589=200%:%
%:%590=200%:%
%:%591=201%:%
%:%592=201%:%
%:%593=202%:%
%:%594=202%:%
%:%595=203%:%
%:%596=203%:%
%:%597=204%:%
%:%598=204%:%
%:%599=205%:%
%:%600=205%:%
%:%601=205%:%
%:%602=205%:%
%:%603=205%:%
%:%604=206%:%
%:%605=206%:%
%:%606=206%:%
%:%607=207%:%
%:%608=207%:%
%:%609=207%:%
%:%610=208%:%
%:%611=208%:%
%:%612=209%:%
%:%613=209%:%
%:%614=209%:%
%:%615=209%:%
%:%616=210%:%
%:%617=210%:%
%:%618=211%:%
%:%619=211%:%
%:%620=212%:%
%:%621=212%:%
%:%622=213%:%
%:%623=213%:%
%:%624=214%:%
%:%625=214%:%
%:%626=215%:%
%:%627=215%:%
%:%628=216%:%
%:%629=216%:%
%:%630=216%:%
%:%631=216%:%
%:%632=216%:%
%:%633=216%:%
%:%634=217%:%
%:%635=217%:%
%:%636=218%:%
%:%637=218%:%
%:%638=219%:%
%:%639=219%:%
%:%640=219%:%
%:%641=219%:%
%:%642=219%:%
%:%643=219%:%
%:%644=220%:%
%:%645=220%:%
%:%646=221%:%
%:%647=221%:%
%:%648=222%:%
%:%649=222%:%
%:%650=223%:%
%:%651=223%:%
%:%652=224%:%
%:%653=224%:%
%:%654=225%:%
%:%655=225%:%
%:%656=226%:%
%:%657=226%:%
%:%658=226%:%
%:%659=226%:%
%:%660=226%:%
%:%661=226%:%
%:%662=227%:%
%:%663=227%:%
%:%664=228%:%
%:%665=228%:%
%:%666=229%:%
%:%667=229%:%
%:%668=230%:%
%:%669=230%:%
%:%670=230%:%
%:%671=231%:%
%:%672=231%:%
%:%673=231%:%
%:%674=232%:%
%:%675=232%:%
%:%676=232%:%
%:%677=233%:%
%:%678=233%:%
%:%679=233%:%
%:%680=233%:%
%:%681=233%:%
%:%682=234%:%
%:%683=234%:%
%:%684=234%:%
%:%685=234%:%
%:%686=235%:%
%:%687=235%:%
%:%688=236%:%
%:%689=236%:%
%:%690=237%:%
%:%691=237%:%
%:%692=238%:%
%:%693=238%:%
%:%694=238%:%
%:%695=238%:%
%:%696=239%:%
%:%697=239%:%
%:%698=240%:%
%:%699=240%:%
%:%700=241%:%
%:%701=241%:%
%:%702=241%:%
%:%703=241%:%
%:%704=241%:%
%:%705=242%:%
%:%706=242%:%
%:%707=242%:%
%:%708=242%:%
%:%709=242%:%
%:%710=243%:%
%:%711=243%:%
%:%712=243%:%
%:%713=243%:%
%:%714=243%:%
%:%715=244%:%
%:%716=244%:%
%:%717=244%:%
%:%718=244%:%
%:%719=245%:%
%:%725=245%:%
%:%728=246%:%
%:%729=247%:%
%:%730=247%:%
%:%731=248%:%
%:%732=249%:%
%:%733=250%:%
%:%740=251%:%
%:%741=251%:%
%:%742=252%:%
%:%743=252%:%
%:%744=253%:%
%:%745=253%:%
%:%746=254%:%
%:%747=254%:%
%:%748=254%:%
%:%749=254%:%
%:%750=255%:%
%:%751=255%:%
%:%752=255%:%
%:%753=256%:%
%:%754=256%:%
%:%755=256%:%
%:%756=256%:%
%:%757=256%:%
%:%758=257%:%
%:%759=257%:%
%:%760=257%:%
%:%761=257%:%
%:%762=257%:%
%:%763=258%:%
%:%764=258%:%
%:%765=258%:%
%:%766=258%:%
%:%767=259%:%
%:%768=259%:%
%:%769=259%:%
%:%770=259%:%
%:%771=260%:%
%:%772=260%:%
%:%773=260%:%
%:%774=260%:%
%:%775=260%:%
%:%776=261%:%
%:%782=261%:%
%:%785=262%:%
%:%786=263%:%
%:%787=263%:%
%:%788=264%:%
%:%789=265%:%
%:%790=266%:%
%:%791=267%:%
%:%798=268%:%
%:%799=268%:%
%:%800=269%:%
%:%801=269%:%
%:%802=270%:%
%:%803=270%:%
%:%804=271%:%
%:%805=271%:%
%:%806=272%:%
%:%807=272%:%
%:%808=272%:%
%:%809=272%:%
%:%810=272%:%
%:%811=273%:%
%:%812=273%:%
%:%813=274%:%
%:%814=274%:%
%:%815=275%:%
%:%816=275%:%
%:%817=275%:%
%:%818=275%:%
%:%819=276%:%
%:%820=276%:%
%:%821=277%:%
%:%822=277%:%
%:%823=277%:%
%:%824=278%:%
%:%825=278%:%
%:%826=278%:%
%:%827=278%:%
%:%828=279%:%
%:%829=279%:%
%:%830=280%:%
%:%831=280%:%
%:%832=280%:%
%:%833=280%:%
%:%834=281%:%
%:%835=281%:%
%:%836=282%:%
%:%837=282%:%
%:%838=282%:%
%:%839=282%:%
%:%840=282%:%
%:%841=283%:%
%:%842=283%:%
%:%843=283%:%
%:%844=284%:%
%:%845=285%:%
%:%846=285%:%
%:%847=286%:%
%:%848=286%:%
%:%849=287%:%
%:%850=288%:%
%:%851=288%:%
%:%852=289%:%
%:%853=289%:%
%:%854=290%:%
%:%855=290%:%
%:%856=290%:%
%:%857=290%:%
%:%858=291%:%
%:%859=291%:%
%:%860=291%:%
%:%861=291%:%
%:%862=292%:%
%:%863=292%:%
%:%864=292%:%
%:%865=292%:%
%:%866=292%:%
%:%867=292%:%
%:%868=293%:%
%:%869=293%:%
%:%870=293%:%
%:%871=293%:%
%:%872=294%:%
%:%873=294%:%
%:%874=294%:%
%:%875=294%:%
%:%876=294%:%
%:%877=295%:%
%:%878=295%:%
%:%879=295%:%
%:%880=295%:%
%:%881=296%:%
%:%882=296%:%
%:%883=297%:%
%:%884=298%:%
%:%885=298%:%
%:%886=299%:%
%:%887=299%:%
%:%888=300%:%
%:%889=300%:%
%:%890=300%:%
%:%891=301%:%
%:%892=301%:%
%:%893=301%:%
%:%894=301%:%
%:%895=301%:%
%:%896=302%:%
%:%897=302%:%
%:%898=302%:%
%:%899=302%:%
%:%900=303%:%
%:%901=303%:%
%:%902=303%:%
%:%903=304%:%
%:%904=304%:%
%:%905=304%:%
%:%906=304%:%
%:%907=305%:%
%:%908=305%:%
%:%909=305%:%
%:%910=305%:%
%:%911=305%:%
%:%912=306%:%
%:%913=306%:%
%:%914=306%:%
%:%915=306%:%
%:%916=306%:%
%:%917=307%:%
%:%918=307%:%
%:%919=307%:%
%:%920=307%:%
%:%921=308%:%
%:%922=308%:%
%:%923=309%:%
%:%924=309%:%
%:%925=309%:%
%:%926=310%:%
%:%927=310%:%
%:%928=310%:%
%:%929=310%:%
%:%930=310%:%
%:%931=310%:%
%:%932=311%:%
%:%933=311%:%
%:%934=311%:%
%:%935=311%:%
%:%936=311%:%
%:%937=311%:%
%:%938=312%:%
%:%939=313%:%
%:%940=313%:%
%:%941=314%:%
%:%942=314%:%
%:%943=314%:%
%:%944=315%:%
%:%945=315%:%
%:%946=315%:%
%:%947=315%:%
%:%948=316%:%
%:%949=316%:%
%:%950=316%:%
%:%951=316%:%
%:%952=316%:%
%:%953=317%:%
%:%954=317%:%
%:%955=317%:%
%:%956=317%:%
%:%957=318%:%
%:%958=318%:%
%:%959=318%:%
%:%960=319%:%
%:%961=319%:%
%:%962=319%:%
%:%963=320%:%
%:%964=320%:%
%:%965=321%:%
%:%966=322%:%
%:%967=322%:%
%:%968=322%:%
%:%969=322%:%
%:%970=323%:%
%:%971=323%:%
%:%972=323%:%
%:%973=324%:%
%:%979=324%:%
%:%982=325%:%
%:%983=326%:%
%:%984=326%:%
%:%985=327%:%
%:%986=328%:%
%:%987=329%:%
%:%994=330%:%
%:%995=330%:%
%:%996=331%:%
%:%997=331%:%
%:%998=332%:%
%:%999=332%:%
%:%1000=333%:%
%:%1001=333%:%
%:%1002=333%:%
%:%1003=334%:%
%:%1004=334%:%
%:%1005=334%:%
%:%1006=334%:%
%:%1007=335%:%
%:%1008=335%:%
%:%1009=335%:%
%:%1010=335%:%
%:%1011=336%:%
%:%1012=336%:%
%:%1013=337%:%
%:%1014=337%:%
%:%1015=337%:%
%:%1016=338%:%
%:%1017=338%:%
%:%1022=338%:%
%:%1025=339%:%
%:%1026=340%:%
%:%1027=340%:%
%:%1028=341%:%
%:%1029=342%:%
%:%1030=343%:%
%:%1037=344%:%
%:%1038=344%:%
%:%1039=345%:%
%:%1040=345%:%
%:%1041=345%:%
%:%1042=346%:%
%:%1043=346%:%
%:%1044=346%:%
%:%1045=347%:%
%:%1051=347%:%
%:%1054=348%:%
%:%1055=349%:%
%:%1056=350%:%
%:%1057=351%:%
%:%1058=351%:%
%:%1059=352%:%
%:%1060=353%:%
%:%1061=354%:%
%:%1068=355%:%
%:%1069=355%:%
%:%1070=356%:%
%:%1071=356%:%
%:%1072=357%:%
%:%1073=357%:%
%:%1074=357%:%
%:%1075=358%:%
%:%1076=358%:%
%:%1077=359%:%
%:%1078=359%:%
%:%1079=360%:%
%:%1080=360%:%
%:%1081=361%:%
%:%1082=362%:%
%:%1083=363%:%
%:%1084=363%:%
%:%1085=363%:%
%:%1086=364%:%
%:%1087=364%:%
%:%1088=364%:%
%:%1089=364%:%
%:%1090=365%:%
%:%1091=365%:%
%:%1092=365%:%
%:%1093=365%:%
%:%1094=366%:%
%:%1095=366%:%
%:%1096=366%:%
%:%1097=366%:%
%:%1098=366%:%
%:%1099=367%:%
%:%1100=367%:%
%:%1101=367%:%
%:%1102=367%:%
%:%1103=368%:%
%:%1104=368%:%
%:%1105=368%:%
%:%1106=368%:%
%:%1107=369%:%
%:%1108=369%:%
%:%1109=369%:%
%:%1110=369%:%
%:%1111=370%:%
%:%1112=370%:%
%:%1113=370%:%
%:%1114=370%:%
%:%1115=370%:%
%:%1116=370%:%
%:%1117=371%:%
%:%1118=371%:%
%:%1119=371%:%
%:%1120=372%:%
%:%1121=372%:%
%:%1122=372%:%
%:%1123=373%:%
%:%1124=373%:%
%:%1125=374%:%
%:%1126=374%:%
%:%1127=375%:%
%:%1128=375%:%
%:%1129=376%:%
%:%1130=376%:%
%:%1131=377%:%
%:%1132=377%:%
%:%1133=378%:%
%:%1134=378%:%
%:%1135=379%:%
%:%1136=379%:%
%:%1137=379%:%
%:%1138=379%:%
%:%1139=379%:%
%:%1140=380%:%
%:%1141=380%:%
%:%1142=381%:%
%:%1143=381%:%
%:%1144=382%:%
%:%1145=382%:%
%:%1146=382%:%
%:%1147=382%:%
%:%1148=383%:%
%:%1149=383%:%
%:%1150=383%:%
%:%1151=383%:%
%:%1152=383%:%
%:%1153=384%:%
%:%1154=384%:%
%:%1155=385%:%
%:%1156=385%:%
%:%1157=386%:%
%:%1158=386%:%
%:%1159=386%:%
%:%1160=386%:%
%:%1161=387%:%
%:%1162=387%:%
%:%1163=388%:%
%:%1164=388%:%
%:%1165=389%:%
%:%1166=389%:%
%:%1167=389%:%
%:%1168=389%:%
%:%1169=390%:%
%:%1170=390%:%
%:%1171=391%:%
%:%1172=391%:%
%:%1173=391%:%
%:%1174=391%:%
%:%1175=391%:%
%:%1176=392%:%
%:%1177=392%:%
%:%1178=392%:%
%:%1179=392%:%
%:%1180=393%:%
%:%1181=393%:%
%:%1186=393%:%
%:%1189=394%:%
%:%1190=395%:%
%:%1191=395%:%
%:%1192=396%:%
%:%1193=397%:%
%:%1194=398%:%
%:%1197=399%:%
%:%1201=399%:%
%:%1202=399%:%
%:%1203=399%:%
%:%1208=399%:%
%:%1211=400%:%
%:%1212=401%:%
%:%1213=401%:%
%:%1214=402%:%
%:%1215=403%:%
%:%1216=404%:%
%:%1219=405%:%
%:%1223=405%:%
%:%1224=405%:%
%:%1225=405%:%
%:%1230=405%:%
%:%1235=406%:%
%:%1240=407%:%

%
\begin{isabellebody}%
\setisabellecontext{Set{\isacharunderscore}{\kern0pt}Integral{\isacharunderscore}{\kern0pt}Addendum}%
%
\isadelimtheory
%
\endisadelimtheory
%
\isatagtheory
\isacommand{theory}\isamarkupfalse%
\ Set{\isacharunderscore}{\kern0pt}Integral{\isacharunderscore}{\kern0pt}Addendum\isanewline
\ \ \isakeyword{imports}\ {\isachardoublequoteopen}HOL{\isacharminus}{\kern0pt}Analysis{\isachardot}{\kern0pt}Set{\isacharunderscore}{\kern0pt}Integral{\isachardoublequoteclose}\ Bochner{\isacharunderscore}{\kern0pt}Integration{\isacharunderscore}{\kern0pt}Addendum\isanewline
\isakeyword{begin}%
\endisatagtheory
{\isafoldtheory}%
%
\isadelimtheory
%
\endisadelimtheory
%
\isadelimdocument
%
\endisadelimdocument
%
\isatagdocument
%
\isamarkupsection{Auxiliary Lemmas for Integrals on a Set%
}
\isamarkuptrue%
%
\endisatagdocument
{\isafolddocument}%
%
\isadelimdocument
%
\endisadelimdocument
\isacommand{lemma}\isamarkupfalse%
\ set{\isacharunderscore}{\kern0pt}integral{\isacharunderscore}{\kern0pt}scaleR{\isacharunderscore}{\kern0pt}left{\isacharcolon}{\kern0pt}\ \isanewline
\ \ \isakeyword{assumes}\ {\isachardoublequoteopen}A\ {\isasymin}\ sets\ M{\isachardoublequoteclose}\ {\isachardoublequoteopen}c\ {\isasymnoteq}\ {\isadigit{0}}\ {\isasymLongrightarrow}\ integrable\ M\ f{\isachardoublequoteclose}\isanewline
\ \ \isakeyword{shows}\ {\isachardoublequoteopen}LINT\ t{\isacharcolon}{\kern0pt}A{\isacharbar}{\kern0pt}M{\isachardot}{\kern0pt}\ f\ t\ {\isacharasterisk}{\kern0pt}\isactrlsub R\ c\ {\isacharequal}{\kern0pt}\ {\isacharparenleft}{\kern0pt}LINT\ t{\isacharcolon}{\kern0pt}A{\isacharbar}{\kern0pt}M{\isachardot}{\kern0pt}\ f\ t{\isacharparenright}{\kern0pt}\ {\isacharasterisk}{\kern0pt}\isactrlsub R\ c{\isachardoublequoteclose}\isanewline
%
\isadelimproof
\ \ %
\endisadelimproof
%
\isatagproof
\isacommand{unfolding}\isamarkupfalse%
\ set{\isacharunderscore}{\kern0pt}lebesgue{\isacharunderscore}{\kern0pt}integral{\isacharunderscore}{\kern0pt}def\ \isanewline
\ \ \isacommand{using}\isamarkupfalse%
\ integrable{\isacharunderscore}{\kern0pt}mult{\isacharunderscore}{\kern0pt}indicator{\isacharbrackleft}{\kern0pt}OF\ assms{\isacharbrackright}{\kern0pt}\isanewline
\ \ \isacommand{by}\isamarkupfalse%
\ {\isacharparenleft}{\kern0pt}subst\ integral{\isacharunderscore}{\kern0pt}scaleR{\isacharunderscore}{\kern0pt}left{\isacharbrackleft}{\kern0pt}symmetric{\isacharbrackright}{\kern0pt}{\isacharcomma}{\kern0pt}\ auto{\isacharparenright}{\kern0pt}%
\endisatagproof
{\isafoldproof}%
%
\isadelimproof
\isanewline
%
\endisadelimproof
\isanewline
\isacommand{lemma}\isamarkupfalse%
\ nn{\isacharunderscore}{\kern0pt}set{\isacharunderscore}{\kern0pt}integral{\isacharunderscore}{\kern0pt}eq{\isacharunderscore}{\kern0pt}set{\isacharunderscore}{\kern0pt}integral{\isacharcolon}{\kern0pt}\isanewline
\ \ \isakeyword{assumes}\ {\isacharbrackleft}{\kern0pt}measurable{\isacharbrackright}{\kern0pt}{\isacharcolon}{\kern0pt}{\isachardoublequoteopen}integrable\ M\ f{\isachardoublequoteclose}\isanewline
\ \ \ \ \ \ \isakeyword{and}\ {\isachardoublequoteopen}AE\ x\ {\isasymin}\ A\ in\ M{\isachardot}{\kern0pt}\ {\isadigit{0}}\ {\isasymle}\ f\ x{\isachardoublequoteclose}\ {\isachardoublequoteopen}A\ {\isasymin}\ sets\ M{\isachardoublequoteclose}\isanewline
\ \ \ \ \isakeyword{shows}\ {\isachardoublequoteopen}{\isacharparenleft}{\kern0pt}{\isasymintegral}\isactrlsup {\isacharplus}{\kern0pt}x{\isasymin}A{\isachardot}{\kern0pt}\ f\ x\ {\isasympartial}M{\isacharparenright}{\kern0pt}\ {\isacharequal}{\kern0pt}\ {\isacharparenleft}{\kern0pt}{\isasymintegral}\ x\ {\isasymin}\ A{\isachardot}{\kern0pt}\ f\ x\ {\isasympartial}M{\isacharparenright}{\kern0pt}{\isachardoublequoteclose}\isanewline
%
\isadelimproof
%
\endisadelimproof
%
\isatagproof
\isacommand{proof}\isamarkupfalse%
{\isacharminus}{\kern0pt}\isanewline
\ \ \isacommand{have}\isamarkupfalse%
\ {\isachardoublequoteopen}{\isacharparenleft}{\kern0pt}{\isasymintegral}\isactrlsup {\isacharplus}{\kern0pt}x{\isachardot}{\kern0pt}\ indicator\ A\ x\ {\isacharasterisk}{\kern0pt}\isactrlsub R\ f\ x\ {\isasympartial}M{\isacharparenright}{\kern0pt}\ {\isacharequal}{\kern0pt}\ {\isacharparenleft}{\kern0pt}{\isasymintegral}\ x\ {\isasymin}\ A{\isachardot}{\kern0pt}\ f\ x\ {\isasympartial}M{\isacharparenright}{\kern0pt}{\isachardoublequoteclose}\isanewline
\ \ \isacommand{unfolding}\isamarkupfalse%
\ set{\isacharunderscore}{\kern0pt}lebesgue{\isacharunderscore}{\kern0pt}integral{\isacharunderscore}{\kern0pt}def\ \isacommand{using}\isamarkupfalse%
\ assms{\isacharparenleft}{\kern0pt}{\isadigit{2}}{\isacharparenright}{\kern0pt}\ \isacommand{by}\isamarkupfalse%
\ {\isacharparenleft}{\kern0pt}intro\ nn{\isacharunderscore}{\kern0pt}integral{\isacharunderscore}{\kern0pt}eq{\isacharunderscore}{\kern0pt}integral{\isacharbrackleft}{\kern0pt}of\ {\isacharunderscore}{\kern0pt}\ {\isachardoublequoteopen}{\isasymlambda}x{\isachardot}{\kern0pt}\ indicat{\isacharunderscore}{\kern0pt}real\ A\ x\ {\isacharasterisk}{\kern0pt}\isactrlsub R\ f\ x{\isachardoublequoteclose}{\isacharbrackright}{\kern0pt}{\isacharcomma}{\kern0pt}\ blast\ intro{\isacharcolon}{\kern0pt}\ assms\ integrable{\isacharunderscore}{\kern0pt}mult{\isacharunderscore}{\kern0pt}indicator{\isacharcomma}{\kern0pt}\ fastforce{\isacharparenright}{\kern0pt}\isanewline
\ \ \isacommand{moreover}\isamarkupfalse%
\ \isacommand{have}\isamarkupfalse%
\ {\isachardoublequoteopen}{\isacharparenleft}{\kern0pt}{\isasymintegral}\isactrlsup {\isacharplus}{\kern0pt}x{\isachardot}{\kern0pt}\ indicator\ A\ x\ {\isacharasterisk}{\kern0pt}\isactrlsub R\ f\ x\ {\isasympartial}M{\isacharparenright}{\kern0pt}\ {\isacharequal}{\kern0pt}\ {\isacharparenleft}{\kern0pt}{\isasymintegral}\isactrlsup {\isacharplus}{\kern0pt}x{\isasymin}A{\isachardot}{\kern0pt}\ f\ x\ {\isasympartial}M{\isacharparenright}{\kern0pt}{\isachardoublequoteclose}\ \ \isacommand{by}\isamarkupfalse%
\ {\isacharparenleft}{\kern0pt}metis\ ennreal{\isacharunderscore}{\kern0pt}{\isadigit{0}}\ indicator{\isacharunderscore}{\kern0pt}simps{\isacharparenleft}{\kern0pt}{\isadigit{1}}{\isacharparenright}{\kern0pt}\ indicator{\isacharunderscore}{\kern0pt}simps{\isacharparenleft}{\kern0pt}{\isadigit{2}}{\isacharparenright}{\kern0pt}\ mult{\isachardot}{\kern0pt}commute\ mult{\isacharunderscore}{\kern0pt}{\isadigit{1}}\ mult{\isacharunderscore}{\kern0pt}zero{\isacharunderscore}{\kern0pt}left\ real{\isacharunderscore}{\kern0pt}scaleR{\isacharunderscore}{\kern0pt}def{\isacharparenright}{\kern0pt}\isanewline
\ \ \isacommand{ultimately}\isamarkupfalse%
\ \isacommand{show}\isamarkupfalse%
\ {\isacharquery}{\kern0pt}thesis\ \isacommand{by}\isamarkupfalse%
\ argo\isanewline
\isacommand{qed}\isamarkupfalse%
%
\endisatagproof
{\isafoldproof}%
%
\isadelimproof
\isanewline
%
\endisadelimproof
\isanewline
\isacommand{lemma}\isamarkupfalse%
\ set{\isacharunderscore}{\kern0pt}integral{\isacharunderscore}{\kern0pt}restrict{\isacharunderscore}{\kern0pt}space{\isacharcolon}{\kern0pt}\isanewline
\ \ \isakeyword{fixes}\ f\ {\isacharcolon}{\kern0pt}{\isacharcolon}{\kern0pt}\ {\isachardoublequoteopen}{\isacharprime}{\kern0pt}a\ {\isasymRightarrow}\ {\isacharprime}{\kern0pt}b{\isacharcolon}{\kern0pt}{\isacharcolon}{\kern0pt}{\isacharbraceleft}{\kern0pt}banach{\isacharcomma}{\kern0pt}\ second{\isacharunderscore}{\kern0pt}countable{\isacharunderscore}{\kern0pt}topology{\isacharbraceright}{\kern0pt}{\isachardoublequoteclose}\isanewline
\ \ \isakeyword{assumes}\ {\isachardoublequoteopen}{\isasymOmega}\ {\isasyminter}\ space\ M\ {\isasymin}\ sets\ M{\isachardoublequoteclose}\isanewline
\ \ \isakeyword{shows}\ {\isachardoublequoteopen}set{\isacharunderscore}{\kern0pt}lebesgue{\isacharunderscore}{\kern0pt}integral\ {\isacharparenleft}{\kern0pt}restrict{\isacharunderscore}{\kern0pt}space\ M\ {\isasymOmega}{\isacharparenright}{\kern0pt}\ A\ f\ {\isacharequal}{\kern0pt}\ set{\isacharunderscore}{\kern0pt}lebesgue{\isacharunderscore}{\kern0pt}integral\ M\ A\ {\isacharparenleft}{\kern0pt}{\isasymlambda}x{\isachardot}{\kern0pt}\ indicator\ {\isasymOmega}\ x\ {\isacharasterisk}{\kern0pt}\isactrlsub R\ f\ x{\isacharparenright}{\kern0pt}{\isachardoublequoteclose}\isanewline
%
\isadelimproof
\ \ %
\endisadelimproof
%
\isatagproof
\isacommand{unfolding}\isamarkupfalse%
\ set{\isacharunderscore}{\kern0pt}lebesgue{\isacharunderscore}{\kern0pt}integral{\isacharunderscore}{\kern0pt}def\ \isanewline
\ \ \isacommand{by}\isamarkupfalse%
\ {\isacharparenleft}{\kern0pt}subst\ integral{\isacharunderscore}{\kern0pt}restrict{\isacharunderscore}{\kern0pt}space{\isacharcomma}{\kern0pt}\ auto\ intro{\isacharbang}{\kern0pt}{\isacharcolon}{\kern0pt}\ integrable{\isacharunderscore}{\kern0pt}mult{\isacharunderscore}{\kern0pt}indicator\ assms\ simp{\isacharcolon}{\kern0pt}\ mult{\isachardot}{\kern0pt}commute{\isacharparenright}{\kern0pt}%
\endisatagproof
{\isafoldproof}%
%
\isadelimproof
\isanewline
%
\endisadelimproof
\isanewline
\isacommand{lemma}\isamarkupfalse%
\ set{\isacharunderscore}{\kern0pt}integral{\isacharunderscore}{\kern0pt}const{\isacharcolon}{\kern0pt}\isanewline
\ \ \isakeyword{fixes}\ c\ {\isacharcolon}{\kern0pt}{\isacharcolon}{\kern0pt}\ {\isachardoublequoteopen}{\isacharprime}{\kern0pt}b{\isacharcolon}{\kern0pt}{\isacharcolon}{\kern0pt}{\isacharbraceleft}{\kern0pt}banach{\isacharcomma}{\kern0pt}\ second{\isacharunderscore}{\kern0pt}countable{\isacharunderscore}{\kern0pt}topology{\isacharbraceright}{\kern0pt}{\isachardoublequoteclose}\isanewline
\ \ \isakeyword{assumes}\ {\isachardoublequoteopen}A\ {\isasymin}\ sets\ M{\isachardoublequoteclose}\ {\isachardoublequoteopen}emeasure\ M\ A\ {\isasymnoteq}\ {\isasyminfinity}{\isachardoublequoteclose}\isanewline
\ \ \isakeyword{shows}\ {\isachardoublequoteopen}set{\isacharunderscore}{\kern0pt}lebesgue{\isacharunderscore}{\kern0pt}integral\ M\ A\ {\isacharparenleft}{\kern0pt}{\isasymlambda}{\isacharunderscore}{\kern0pt}{\isachardot}{\kern0pt}\ c{\isacharparenright}{\kern0pt}\ {\isacharequal}{\kern0pt}\ measure\ M\ A\ {\isacharasterisk}{\kern0pt}\isactrlsub R\ c{\isachardoublequoteclose}\isanewline
%
\isadelimproof
\ \ %
\endisadelimproof
%
\isatagproof
\isacommand{unfolding}\isamarkupfalse%
\ set{\isacharunderscore}{\kern0pt}lebesgue{\isacharunderscore}{\kern0pt}integral{\isacharunderscore}{\kern0pt}def\ \isanewline
\ \ \isacommand{using}\isamarkupfalse%
\ assms\ \isacommand{by}\isamarkupfalse%
\ {\isacharparenleft}{\kern0pt}metis\ has{\isacharunderscore}{\kern0pt}bochner{\isacharunderscore}{\kern0pt}integral{\isacharunderscore}{\kern0pt}indicator\ has{\isacharunderscore}{\kern0pt}bochner{\isacharunderscore}{\kern0pt}integral{\isacharunderscore}{\kern0pt}integral{\isacharunderscore}{\kern0pt}eq\ infinity{\isacharunderscore}{\kern0pt}ennreal{\isacharunderscore}{\kern0pt}def\ less{\isacharunderscore}{\kern0pt}top{\isacharparenright}{\kern0pt}%
\endisatagproof
{\isafoldproof}%
%
\isadelimproof
\isanewline
%
\endisadelimproof
\isanewline
\isacommand{lemma}\isamarkupfalse%
\ set{\isacharunderscore}{\kern0pt}integral{\isacharunderscore}{\kern0pt}mono{\isacharunderscore}{\kern0pt}banach{\isacharcolon}{\kern0pt}\isanewline
\ \ \isakeyword{fixes}\ f\ g\ {\isacharcolon}{\kern0pt}{\isacharcolon}{\kern0pt}\ {\isachardoublequoteopen}{\isacharprime}{\kern0pt}a\ {\isasymRightarrow}\ {\isacharprime}{\kern0pt}b\ {\isacharcolon}{\kern0pt}{\isacharcolon}{\kern0pt}\ {\isacharbraceleft}{\kern0pt}second{\isacharunderscore}{\kern0pt}countable{\isacharunderscore}{\kern0pt}topology{\isacharcomma}{\kern0pt}\ banach{\isacharcomma}{\kern0pt}\ linorder{\isacharunderscore}{\kern0pt}topology{\isacharcomma}{\kern0pt}\ ordered{\isacharunderscore}{\kern0pt}real{\isacharunderscore}{\kern0pt}vector{\isacharbraceright}{\kern0pt}{\isachardoublequoteclose}\isanewline
\ \ \isakeyword{assumes}\ {\isachardoublequoteopen}set{\isacharunderscore}{\kern0pt}integrable\ M\ A\ f{\isachardoublequoteclose}\ {\isachardoublequoteopen}set{\isacharunderscore}{\kern0pt}integrable\ M\ A\ g{\isachardoublequoteclose}\isanewline
\ \ \ \ {\isachardoublequoteopen}{\isasymAnd}x{\isachardot}{\kern0pt}\ x\ {\isasymin}\ A\ {\isasymLongrightarrow}\ f\ x\ {\isasymle}\ g\ x{\isachardoublequoteclose}\isanewline
\ \ \isakeyword{shows}\ {\isachardoublequoteopen}{\isacharparenleft}{\kern0pt}LINT\ x{\isacharcolon}{\kern0pt}A{\isacharbar}{\kern0pt}M{\isachardot}{\kern0pt}\ f\ x{\isacharparenright}{\kern0pt}\ {\isasymle}\ {\isacharparenleft}{\kern0pt}LINT\ x{\isacharcolon}{\kern0pt}A{\isacharbar}{\kern0pt}M{\isachardot}{\kern0pt}\ g\ x{\isacharparenright}{\kern0pt}{\isachardoublequoteclose}\isanewline
%
\isadelimproof
\ \ %
\endisadelimproof
%
\isatagproof
\isacommand{using}\isamarkupfalse%
\ assms\ \isacommand{unfolding}\isamarkupfalse%
\ set{\isacharunderscore}{\kern0pt}integrable{\isacharunderscore}{\kern0pt}def\ set{\isacharunderscore}{\kern0pt}lebesgue{\isacharunderscore}{\kern0pt}integral{\isacharunderscore}{\kern0pt}def\isanewline
\ \ \isacommand{by}\isamarkupfalse%
\ {\isacharparenleft}{\kern0pt}auto\ intro{\isacharcolon}{\kern0pt}\ integral{\isacharunderscore}{\kern0pt}mono{\isacharunderscore}{\kern0pt}banach\ split{\isacharcolon}{\kern0pt}\ split{\isacharunderscore}{\kern0pt}indicator{\isacharparenright}{\kern0pt}%
\endisatagproof
{\isafoldproof}%
%
\isadelimproof
\isanewline
%
\endisadelimproof
\isanewline
\isacommand{lemma}\isamarkupfalse%
\ set{\isacharunderscore}{\kern0pt}integral{\isacharunderscore}{\kern0pt}mono{\isacharunderscore}{\kern0pt}AE{\isacharunderscore}{\kern0pt}banach{\isacharcolon}{\kern0pt}\isanewline
\ \ \isakeyword{fixes}\ f\ g\ {\isacharcolon}{\kern0pt}{\isacharcolon}{\kern0pt}\ {\isachardoublequoteopen}{\isacharprime}{\kern0pt}a\ {\isasymRightarrow}\ {\isacharprime}{\kern0pt}b\ {\isacharcolon}{\kern0pt}{\isacharcolon}{\kern0pt}\ {\isacharbraceleft}{\kern0pt}second{\isacharunderscore}{\kern0pt}countable{\isacharunderscore}{\kern0pt}topology{\isacharcomma}{\kern0pt}\ banach{\isacharcomma}{\kern0pt}\ linorder{\isacharunderscore}{\kern0pt}topology{\isacharcomma}{\kern0pt}\ ordered{\isacharunderscore}{\kern0pt}real{\isacharunderscore}{\kern0pt}vector{\isacharbraceright}{\kern0pt}{\isachardoublequoteclose}\isanewline
\ \ \isakeyword{assumes}\ {\isachardoublequoteopen}set{\isacharunderscore}{\kern0pt}integrable\ M\ A\ f{\isachardoublequoteclose}\ {\isachardoublequoteopen}set{\isacharunderscore}{\kern0pt}integrable\ M\ A\ g{\isachardoublequoteclose}\ {\isachardoublequoteopen}AE\ x{\isasymin}A\ in\ M{\isachardot}{\kern0pt}\ f\ x\ {\isasymle}\ g\ x{\isachardoublequoteclose}\isanewline
\ \ \isakeyword{shows}\ {\isachardoublequoteopen}set{\isacharunderscore}{\kern0pt}lebesgue{\isacharunderscore}{\kern0pt}integral\ M\ A\ f\ {\isasymle}\ set{\isacharunderscore}{\kern0pt}lebesgue{\isacharunderscore}{\kern0pt}integral\ M\ A\ g{\isachardoublequoteclose}%
\isadelimproof
\ %
\endisadelimproof
%
\isatagproof
\isacommand{using}\isamarkupfalse%
\ assms\ \isacommand{unfolding}\isamarkupfalse%
\ set{\isacharunderscore}{\kern0pt}lebesgue{\isacharunderscore}{\kern0pt}integral{\isacharunderscore}{\kern0pt}def\ \isacommand{by}\isamarkupfalse%
\ {\isacharparenleft}{\kern0pt}auto\ simp\ add{\isacharcolon}{\kern0pt}\ set{\isacharunderscore}{\kern0pt}integrable{\isacharunderscore}{\kern0pt}def\ intro{\isacharbang}{\kern0pt}{\isacharcolon}{\kern0pt}\ integral{\isacharunderscore}{\kern0pt}mono{\isacharunderscore}{\kern0pt}AE{\isacharunderscore}{\kern0pt}banach{\isacharbrackleft}{\kern0pt}of\ M\ {\isachardoublequoteopen}{\isasymlambda}x{\isachardot}{\kern0pt}\ indicator\ A\ x\ {\isacharasterisk}{\kern0pt}\isactrlsub R\ f\ x{\isachardoublequoteclose}\ {\isachardoublequoteopen}{\isasymlambda}x{\isachardot}{\kern0pt}\ indicator\ A\ x\ {\isacharasterisk}{\kern0pt}\isactrlsub R\ g\ x{\isachardoublequoteclose}{\isacharbrackright}{\kern0pt}{\isacharcomma}{\kern0pt}\ simp\ add{\isacharcolon}{\kern0pt}\ indicator{\isacharunderscore}{\kern0pt}def{\isacharparenright}{\kern0pt}%
\endisatagproof
{\isafoldproof}%
%
\isadelimproof
%
\endisadelimproof
\isanewline
%
\isadelimtheory
\isanewline
%
\endisadelimtheory
%
\isatagtheory
\isacommand{end}\isamarkupfalse%
%
\endisatagtheory
{\isafoldtheory}%
%
\isadelimtheory
%
\endisadelimtheory
%
\end{isabellebody}%
\endinput
%:%file=Set_Integral_Addendum.tex%:%
%:%10=1%:%
%:%11=1%:%
%:%12=2%:%
%:%13=3%:%
%:%27=5%:%
%:%37=7%:%
%:%38=7%:%
%:%39=8%:%
%:%40=9%:%
%:%43=10%:%
%:%47=10%:%
%:%48=10%:%
%:%49=11%:%
%:%50=11%:%
%:%51=12%:%
%:%52=12%:%
%:%57=12%:%
%:%60=13%:%
%:%61=14%:%
%:%62=14%:%
%:%63=15%:%
%:%64=16%:%
%:%65=17%:%
%:%72=18%:%
%:%73=18%:%
%:%74=19%:%
%:%75=19%:%
%:%76=20%:%
%:%77=20%:%
%:%78=20%:%
%:%79=20%:%
%:%80=21%:%
%:%81=21%:%
%:%82=21%:%
%:%83=21%:%
%:%84=22%:%
%:%85=22%:%
%:%86=22%:%
%:%87=22%:%
%:%88=23%:%
%:%94=23%:%
%:%97=24%:%
%:%98=25%:%
%:%99=25%:%
%:%100=26%:%
%:%101=27%:%
%:%102=28%:%
%:%105=29%:%
%:%109=29%:%
%:%110=29%:%
%:%111=30%:%
%:%112=30%:%
%:%117=30%:%
%:%120=31%:%
%:%121=32%:%
%:%122=32%:%
%:%123=33%:%
%:%124=34%:%
%:%125=35%:%
%:%128=36%:%
%:%132=36%:%
%:%133=36%:%
%:%134=37%:%
%:%135=37%:%
%:%136=37%:%
%:%141=37%:%
%:%144=38%:%
%:%145=39%:%
%:%146=39%:%
%:%147=40%:%
%:%148=41%:%
%:%149=42%:%
%:%150=43%:%
%:%153=44%:%
%:%157=44%:%
%:%158=44%:%
%:%159=44%:%
%:%160=45%:%
%:%161=45%:%
%:%166=45%:%
%:%169=46%:%
%:%170=47%:%
%:%171=47%:%
%:%172=48%:%
%:%173=49%:%
%:%174=50%:%
%:%176=50%:%
%:%180=50%:%
%:%181=50%:%
%:%182=50%:%
%:%183=50%:%
%:%190=50%:%
%:%193=51%:%
%:%198=52%:%

%
\begin{isabellebody}%
\setisabellecontext{Sigma{\isacharunderscore}{\kern0pt}Finite{\isacharunderscore}{\kern0pt}Measure{\isacharunderscore}{\kern0pt}Addendum}%
%
\isadelimtheory
%
\endisadelimtheory
%
\isatagtheory
\isacommand{theory}\isamarkupfalse%
\ Sigma{\isacharunderscore}{\kern0pt}Finite{\isacharunderscore}{\kern0pt}Measure{\isacharunderscore}{\kern0pt}Addendum\isanewline
\isakeyword{imports}\ Set{\isacharunderscore}{\kern0pt}Integral{\isacharunderscore}{\kern0pt}Addendum\isanewline
\isakeyword{begin}%
\endisatagtheory
{\isafoldtheory}%
%
\isadelimtheory
%
\endisadelimtheory
%
\isadelimdocument
%
\endisadelimdocument
%
\isatagdocument
%
\isamarkupsection{Averaging Theorem%
}
\isamarkuptrue%
%
\endisatagdocument
{\isafolddocument}%
%
\isadelimdocument
%
\endisadelimdocument
\isacommand{lemma}\isamarkupfalse%
\ balls{\isacharunderscore}{\kern0pt}countable{\isacharunderscore}{\kern0pt}basis{\isacharcolon}{\kern0pt}\isanewline
\ \ \isakeyword{obtains}\ D\ {\isacharcolon}{\kern0pt}{\isacharcolon}{\kern0pt}\ {\isachardoublequoteopen}{\isacharprime}{\kern0pt}a\ {\isacharcolon}{\kern0pt}{\isacharcolon}{\kern0pt}\ {\isacharbraceleft}{\kern0pt}metric{\isacharunderscore}{\kern0pt}space{\isacharcomma}{\kern0pt}\ second{\isacharunderscore}{\kern0pt}countable{\isacharunderscore}{\kern0pt}topology{\isacharbraceright}{\kern0pt}\ set{\isachardoublequoteclose}\ \isanewline
\ \ \isakeyword{where}\ {\isachardoublequoteopen}topological{\isacharunderscore}{\kern0pt}basis\ {\isacharparenleft}{\kern0pt}case{\isacharunderscore}{\kern0pt}prod\ ball\ {\isacharbackquote}{\kern0pt}\ {\isacharparenleft}{\kern0pt}D\ {\isasymtimes}\ {\isacharparenleft}{\kern0pt}{\isasymrat}\ {\isasyminter}\ {\isacharbraceleft}{\kern0pt}{\isadigit{0}}{\isacharless}{\kern0pt}{\isachardot}{\kern0pt}{\isachardot}{\kern0pt}{\isacharbraceright}{\kern0pt}{\isacharparenright}{\kern0pt}{\isacharparenright}{\kern0pt}{\isacharparenright}{\kern0pt}{\isachardoublequoteclose}\isanewline
\ \ \ \ \isakeyword{and}\ {\isachardoublequoteopen}countable\ D{\isachardoublequoteclose}\isanewline
\ \ \ \ \isakeyword{and}\ {\isachardoublequoteopen}D\ {\isasymnoteq}\ {\isacharbraceleft}{\kern0pt}{\isacharbraceright}{\kern0pt}{\isachardoublequoteclose}\ \ \ \ \isanewline
%
\isadelimproof
%
\endisadelimproof
%
\isatagproof
\isacommand{proof}\isamarkupfalse%
\ {\isacharminus}{\kern0pt}\isanewline
\ \ \isacommand{obtain}\isamarkupfalse%
\ D\ {\isacharcolon}{\kern0pt}{\isacharcolon}{\kern0pt}\ {\isachardoublequoteopen}{\isacharprime}{\kern0pt}a\ set{\isachardoublequoteclose}\ \isakeyword{where}\ dense{\isacharunderscore}{\kern0pt}subset{\isacharcolon}{\kern0pt}\ {\isachardoublequoteopen}countable\ D{\isachardoublequoteclose}\ {\isachardoublequoteopen}D\ {\isasymnoteq}\ {\isacharbraceleft}{\kern0pt}{\isacharbraceright}{\kern0pt}{\isachardoublequoteclose}\ {\isachardoublequoteopen}{\isasymlbrakk}open\ U{\isacharsemicolon}{\kern0pt}\ U\ {\isasymnoteq}\ {\isacharbraceleft}{\kern0pt}{\isacharbraceright}{\kern0pt}{\isasymrbrakk}\ {\isasymLongrightarrow}\ {\isasymexists}y\ {\isasymin}\ D{\isachardot}{\kern0pt}\ y\ {\isasymin}\ U{\isachardoublequoteclose}\ \isakeyword{for}\ U\ \isacommand{using}\isamarkupfalse%
\ countable{\isacharunderscore}{\kern0pt}dense{\isacharunderscore}{\kern0pt}exists\ \isacommand{by}\isamarkupfalse%
\ blast\isanewline
\ \ \isacommand{have}\isamarkupfalse%
\ {\isachardoublequoteopen}topological{\isacharunderscore}{\kern0pt}basis\ {\isacharparenleft}{\kern0pt}case{\isacharunderscore}{\kern0pt}prod\ ball\ {\isacharbackquote}{\kern0pt}\ {\isacharparenleft}{\kern0pt}D\ {\isasymtimes}\ {\isacharparenleft}{\kern0pt}{\isasymrat}\ {\isasyminter}\ {\isacharbraceleft}{\kern0pt}{\isadigit{0}}{\isacharless}{\kern0pt}{\isachardot}{\kern0pt}{\isachardot}{\kern0pt}{\isacharbraceright}{\kern0pt}{\isacharparenright}{\kern0pt}{\isacharparenright}{\kern0pt}{\isacharparenright}{\kern0pt}{\isachardoublequoteclose}\isanewline
\ \ \isacommand{proof}\isamarkupfalse%
\ {\isacharparenleft}{\kern0pt}intro\ topological{\isacharunderscore}{\kern0pt}basis{\isacharunderscore}{\kern0pt}iff{\isacharbrackleft}{\kern0pt}THEN\ iffD{\isadigit{2}}{\isacharbrackright}{\kern0pt}{\isacharcomma}{\kern0pt}\ fast{\isacharcomma}{\kern0pt}\ clarify{\isacharparenright}{\kern0pt}\isanewline
\ \ \ \ \isacommand{fix}\isamarkupfalse%
\ U\ \isakeyword{and}\ x\ {\isacharcolon}{\kern0pt}{\isacharcolon}{\kern0pt}\ {\isacharprime}{\kern0pt}a\ \isacommand{assume}\isamarkupfalse%
\ asm{\isacharcolon}{\kern0pt}\ {\isachardoublequoteopen}open\ U{\isachardoublequoteclose}\ {\isachardoublequoteopen}x\ {\isasymin}\ U{\isachardoublequoteclose}\isanewline
\ \ \ \ \isacommand{obtain}\isamarkupfalse%
\ e\ \isakeyword{where}\ e{\isacharcolon}{\kern0pt}\ {\isachardoublequoteopen}e\ {\isachargreater}{\kern0pt}\ {\isadigit{0}}{\isachardoublequoteclose}\ {\isachardoublequoteopen}ball\ x\ e\ {\isasymsubseteq}\ U{\isachardoublequoteclose}\ \isacommand{using}\isamarkupfalse%
\ asm\ openE\ \isacommand{by}\isamarkupfalse%
\ blast\isanewline
\ \ \ \ \isacommand{obtain}\isamarkupfalse%
\ y\ \isakeyword{where}\ y{\isacharcolon}{\kern0pt}\ {\isachardoublequoteopen}y\ {\isasymin}\ D{\isachardoublequoteclose}\ {\isachardoublequoteopen}y\ {\isasymin}\ ball\ x\ {\isacharparenleft}{\kern0pt}e\ {\isacharslash}{\kern0pt}\ {\isadigit{3}}{\isacharparenright}{\kern0pt}{\isachardoublequoteclose}\ \isacommand{using}\isamarkupfalse%
\ dense{\isacharunderscore}{\kern0pt}subset{\isacharparenleft}{\kern0pt}{\isadigit{3}}{\isacharparenright}{\kern0pt}{\isacharbrackleft}{\kern0pt}OF\ open{\isacharunderscore}{\kern0pt}ball{\isacharcomma}{\kern0pt}\ of\ x\ {\isachardoublequoteopen}e\ {\isacharslash}{\kern0pt}\ {\isadigit{3}}{\isachardoublequoteclose}{\isacharbrackright}{\kern0pt}\ centre{\isacharunderscore}{\kern0pt}in{\isacharunderscore}{\kern0pt}ball{\isacharbrackleft}{\kern0pt}THEN\ iffD{\isadigit{2}}{\isacharcomma}{\kern0pt}\ OF\ divide{\isacharunderscore}{\kern0pt}pos{\isacharunderscore}{\kern0pt}pos{\isacharbrackleft}{\kern0pt}OF\ e{\isacharparenleft}{\kern0pt}{\isadigit{1}}{\isacharparenright}{\kern0pt}{\isacharcomma}{\kern0pt}\ of\ {\isadigit{3}}{\isacharbrackright}{\kern0pt}{\isacharbrackright}{\kern0pt}\ \isacommand{by}\isamarkupfalse%
\ force\isanewline
\ \ \ \ \isacommand{obtain}\isamarkupfalse%
\ r\ \isakeyword{where}\ r{\isacharcolon}{\kern0pt}\ {\isachardoublequoteopen}r\ {\isasymin}\ {\isasymrat}\ {\isasyminter}\ {\isacharbraceleft}{\kern0pt}e{\isacharslash}{\kern0pt}{\isadigit{3}}{\isacharless}{\kern0pt}{\isachardot}{\kern0pt}{\isachardot}{\kern0pt}{\isacharless}{\kern0pt}e{\isacharslash}{\kern0pt}{\isadigit{2}}{\isacharbraceright}{\kern0pt}{\isachardoublequoteclose}\ \isacommand{unfolding}\isamarkupfalse%
\ Rats{\isacharunderscore}{\kern0pt}def\ \isacommand{using}\isamarkupfalse%
\ of{\isacharunderscore}{\kern0pt}rat{\isacharunderscore}{\kern0pt}dense{\isacharbrackleft}{\kern0pt}OF\ divide{\isacharunderscore}{\kern0pt}strict{\isacharunderscore}{\kern0pt}left{\isacharunderscore}{\kern0pt}mono{\isacharbrackleft}{\kern0pt}OF\ {\isacharunderscore}{\kern0pt}\ e{\isacharparenleft}{\kern0pt}{\isadigit{1}}{\isacharparenright}{\kern0pt}{\isacharbrackright}{\kern0pt}{\isacharcomma}{\kern0pt}\ of\ {\isadigit{2}}\ {\isadigit{3}}{\isacharbrackright}{\kern0pt}\ \isacommand{by}\isamarkupfalse%
\ auto\isanewline
\isanewline
\ \ \ \ \isacommand{have}\isamarkupfalse%
\ {\isacharasterisk}{\kern0pt}{\isacharcolon}{\kern0pt}\ {\isachardoublequoteopen}x\ {\isasymin}\ ball\ y\ r{\isachardoublequoteclose}\ \isacommand{using}\isamarkupfalse%
\ r\ y\ \isacommand{by}\isamarkupfalse%
\ {\isacharparenleft}{\kern0pt}simp\ add{\isacharcolon}{\kern0pt}\ dist{\isacharunderscore}{\kern0pt}commute{\isacharparenright}{\kern0pt}\isanewline
\ \ \ \ \isacommand{hence}\isamarkupfalse%
\ {\isachardoublequoteopen}ball\ y\ r\ {\isasymsubseteq}\ U{\isachardoublequoteclose}\ \isacommand{using}\isamarkupfalse%
\ r\ \isacommand{by}\isamarkupfalse%
\ {\isacharparenleft}{\kern0pt}intro\ order{\isacharunderscore}{\kern0pt}trans{\isacharbrackleft}{\kern0pt}OF\ {\isacharunderscore}{\kern0pt}\ e{\isacharparenleft}{\kern0pt}{\isadigit{2}}{\isacharparenright}{\kern0pt}{\isacharbrackright}{\kern0pt}{\isacharcomma}{\kern0pt}\ simp{\isacharcomma}{\kern0pt}\ metric{\isacharparenright}{\kern0pt}\isanewline
\ \ \ \ \isacommand{moreover}\isamarkupfalse%
\ \isacommand{have}\isamarkupfalse%
\ {\isachardoublequoteopen}ball\ y\ r\ {\isasymin}\ {\isacharparenleft}{\kern0pt}case{\isacharunderscore}{\kern0pt}prod\ ball\ {\isacharbackquote}{\kern0pt}\ {\isacharparenleft}{\kern0pt}D\ {\isasymtimes}\ {\isacharparenleft}{\kern0pt}{\isasymrat}\ {\isasyminter}\ {\isacharbraceleft}{\kern0pt}{\isadigit{0}}{\isacharless}{\kern0pt}{\isachardot}{\kern0pt}{\isachardot}{\kern0pt}{\isacharbraceright}{\kern0pt}{\isacharparenright}{\kern0pt}{\isacharparenright}{\kern0pt}{\isacharparenright}{\kern0pt}{\isachardoublequoteclose}\ \isacommand{using}\isamarkupfalse%
\ y{\isacharparenleft}{\kern0pt}{\isadigit{1}}{\isacharparenright}{\kern0pt}\ r\ \isacommand{by}\isamarkupfalse%
\ force\isanewline
\ \ \ \ \isacommand{ultimately}\isamarkupfalse%
\ \isacommand{show}\isamarkupfalse%
\ {\isachardoublequoteopen}{\isasymexists}B{\isacharprime}{\kern0pt}{\isasymin}{\isacharparenleft}{\kern0pt}case{\isacharunderscore}{\kern0pt}prod\ ball\ {\isacharbackquote}{\kern0pt}\ {\isacharparenleft}{\kern0pt}D\ {\isasymtimes}\ {\isacharparenleft}{\kern0pt}{\isasymrat}\ {\isasyminter}\ {\isacharbraceleft}{\kern0pt}{\isadigit{0}}{\isacharless}{\kern0pt}{\isachardot}{\kern0pt}{\isachardot}{\kern0pt}{\isacharbraceright}{\kern0pt}{\isacharparenright}{\kern0pt}{\isacharparenright}{\kern0pt}{\isacharparenright}{\kern0pt}{\isachardot}{\kern0pt}\ x\ {\isasymin}\ B{\isacharprime}{\kern0pt}\ {\isasymand}\ B{\isacharprime}{\kern0pt}\ {\isasymsubseteq}\ U{\isachardoublequoteclose}\ \isacommand{using}\isamarkupfalse%
\ {\isacharasterisk}{\kern0pt}\ \isacommand{by}\isamarkupfalse%
\ meson\isanewline
\ \ \isacommand{qed}\isamarkupfalse%
\isanewline
\ \ \isacommand{thus}\isamarkupfalse%
\ {\isacharquery}{\kern0pt}thesis\ \isacommand{using}\isamarkupfalse%
\ that\ dense{\isacharunderscore}{\kern0pt}subset\ \isacommand{by}\isamarkupfalse%
\ blast\isanewline
\isacommand{qed}\isamarkupfalse%
%
\endisatagproof
{\isafoldproof}%
%
\isadelimproof
\isanewline
%
\endisadelimproof
\isanewline
\isacommand{context}\isamarkupfalse%
\ sigma{\isacharunderscore}{\kern0pt}finite{\isacharunderscore}{\kern0pt}measure\isanewline
\isakeyword{begin}\ \ \ \ \ \ \ \ \ \isanewline
\isanewline
\isacommand{lemma}\isamarkupfalse%
\ sigma{\isacharunderscore}{\kern0pt}finite{\isacharunderscore}{\kern0pt}measure{\isacharunderscore}{\kern0pt}induct{\isacharbrackleft}{\kern0pt}case{\isacharunderscore}{\kern0pt}names\ finite{\isacharunderscore}{\kern0pt}measure{\isacharcomma}{\kern0pt}\ consumes\ {\isadigit{0}}{\isacharbrackright}{\kern0pt}{\isacharcolon}{\kern0pt}\isanewline
\ \ \isakeyword{assumes}\ {\isachardoublequoteopen}{\isasymAnd}{\isacharparenleft}{\kern0pt}N\ {\isacharcolon}{\kern0pt}{\isacharcolon}{\kern0pt}\ {\isacharprime}{\kern0pt}a\ measure{\isacharparenright}{\kern0pt}\ {\isasymOmega}{\isachardot}{\kern0pt}\ finite{\isacharunderscore}{\kern0pt}measure\ N\ \isanewline
\ \ \ \ \ \ \ \ \ \ \ \ \ \ \ \ \ \ \ \ \ \ \ \ \ \ \ \ \ \ {\isasymLongrightarrow}\ N\ {\isacharequal}{\kern0pt}\ restrict{\isacharunderscore}{\kern0pt}space\ M\ {\isasymOmega}\isanewline
\ \ \ \ \ \ \ \ \ \ \ \ \ \ \ \ \ \ \ \ \ \ \ \ \ \ \ \ \ \ {\isasymLongrightarrow}\ {\isasymOmega}\ {\isasymin}\ sets\ M\ \isanewline
\ \ \ \ \ \ \ \ \ \ \ \ \ \ \ \ \ \ \ \ \ \ \ \ \ \ \ \ \ \ {\isasymLongrightarrow}\ emeasure\ N\ {\isasymOmega}\ {\isasymnoteq}\ {\isasyminfinity}\ \isanewline
\ \ \ \ \ \ \ \ \ \ \ \ \ \ \ \ \ \ \ \ \ \ \ \ \ \ \ \ \ \ {\isasymLongrightarrow}\ emeasure\ N\ {\isasymOmega}\ {\isasymnoteq}\ {\isadigit{0}}\ \isanewline
\ \ \ \ \ \ \ \ \ \ \ \ \ \ \ \ \ \ \ \ \ \ \ \ \ \ \ \ \ \ {\isasymLongrightarrow}\ almost{\isacharunderscore}{\kern0pt}everywhere\ N\ Q{\isachardoublequoteclose}\isanewline
\ \ \ \ \ \ \isakeyword{and}\ {\isacharbrackleft}{\kern0pt}measurable{\isacharbrackright}{\kern0pt}{\isacharcolon}{\kern0pt}\ {\isachardoublequoteopen}Measurable{\isachardot}{\kern0pt}pred\ M\ Q{\isachardoublequoteclose}\isanewline
\ \ \isakeyword{shows}\ {\isachardoublequoteopen}almost{\isacharunderscore}{\kern0pt}everywhere\ M\ Q{\isachardoublequoteclose}\isanewline
%
\isadelimproof
%
\endisadelimproof
%
\isatagproof
\isacommand{proof}\isamarkupfalse%
\ {\isacharminus}{\kern0pt}\isanewline
\ \ \isacommand{have}\isamarkupfalse%
\ {\isacharasterisk}{\kern0pt}{\isacharcolon}{\kern0pt}\ {\isachardoublequoteopen}almost{\isacharunderscore}{\kern0pt}everywhere\ N\ Q{\isachardoublequoteclose}\ \isakeyword{if}\ {\isachardoublequoteopen}finite{\isacharunderscore}{\kern0pt}measure\ N{\isachardoublequoteclose}\ {\isachardoublequoteopen}N\ {\isacharequal}{\kern0pt}\ restrict{\isacharunderscore}{\kern0pt}space\ M\ {\isasymOmega}{\isachardoublequoteclose}\ {\isachardoublequoteopen}{\isasymOmega}\ {\isasymin}\ sets\ M{\isachardoublequoteclose}\ {\isachardoublequoteopen}emeasure\ N\ {\isasymOmega}\ {\isasymnoteq}\ {\isasyminfinity}{\isachardoublequoteclose}\ \isakeyword{for}\ N\ {\isasymOmega}\ \isacommand{using}\isamarkupfalse%
\ that\ \isacommand{by}\isamarkupfalse%
\ {\isacharparenleft}{\kern0pt}cases\ {\isachardoublequoteopen}emeasure\ N\ {\isasymOmega}\ {\isacharequal}{\kern0pt}\ {\isadigit{0}}{\isachardoublequoteclose}{\isacharcomma}{\kern0pt}\ auto\ intro{\isacharcolon}{\kern0pt}\ emeasure{\isacharunderscore}{\kern0pt}{\isadigit{0}}{\isacharunderscore}{\kern0pt}AE\ assms{\isacharparenleft}{\kern0pt}{\isadigit{1}}{\isacharparenright}{\kern0pt}{\isacharparenright}{\kern0pt}\isanewline
\isanewline
\ \ \isacommand{obtain}\isamarkupfalse%
\ A\ {\isacharcolon}{\kern0pt}{\isacharcolon}{\kern0pt}\ {\isachardoublequoteopen}nat\ {\isasymRightarrow}\ {\isacharprime}{\kern0pt}a\ set{\isachardoublequoteclose}\ \isakeyword{where}\ A{\isacharcolon}{\kern0pt}\ {\isachardoublequoteopen}range\ A\ {\isasymsubseteq}\ sets\ M{\isachardoublequoteclose}\ {\isachardoublequoteopen}{\isacharparenleft}{\kern0pt}{\isasymUnion}i{\isachardot}{\kern0pt}\ A\ i{\isacharparenright}{\kern0pt}\ {\isacharequal}{\kern0pt}\ space\ M{\isachardoublequoteclose}\ \isakeyword{and}\ emeasure{\isacharunderscore}{\kern0pt}finite{\isacharcolon}{\kern0pt}\ {\isachardoublequoteopen}emeasure\ M\ {\isacharparenleft}{\kern0pt}A\ i{\isacharparenright}{\kern0pt}\ {\isasymnoteq}\ {\isasyminfinity}{\isachardoublequoteclose}\ \isakeyword{for}\ i\ \isacommand{using}\isamarkupfalse%
\ sigma{\isacharunderscore}{\kern0pt}finite\ \isacommand{by}\isamarkupfalse%
\ metis\isanewline
\ \ \isacommand{note}\isamarkupfalse%
\ A{\isacharparenleft}{\kern0pt}{\isadigit{1}}{\isacharparenright}{\kern0pt}{\isacharbrackleft}{\kern0pt}measurable{\isacharbrackright}{\kern0pt}\isanewline
\ \ \isacommand{have}\isamarkupfalse%
\ space{\isacharunderscore}{\kern0pt}restr{\isacharcolon}{\kern0pt}\ {\isachardoublequoteopen}space\ {\isacharparenleft}{\kern0pt}restrict{\isacharunderscore}{\kern0pt}space\ M\ {\isacharparenleft}{\kern0pt}A\ i{\isacharparenright}{\kern0pt}{\isacharparenright}{\kern0pt}\ {\isacharequal}{\kern0pt}\ A\ i{\isachardoublequoteclose}\ \isakeyword{for}\ i\ \isacommand{unfolding}\isamarkupfalse%
\ space{\isacharunderscore}{\kern0pt}restrict{\isacharunderscore}{\kern0pt}space\ \isacommand{by}\isamarkupfalse%
\ simp\isanewline
\ \ \isacommand{{\isacharbraceleft}{\kern0pt}}\isamarkupfalse%
\isanewline
\ \ \ \ \isacommand{fix}\isamarkupfalse%
\ i\ \ \ \ \isanewline
\ \ \ \ \isacommand{have}\isamarkupfalse%
\ {\isacharasterisk}{\kern0pt}{\isacharcolon}{\kern0pt}\ {\isachardoublequoteopen}{\isacharbraceleft}{\kern0pt}x\ {\isasymin}\ A\ i\ {\isasyminter}\ space\ M{\isachardot}{\kern0pt}\ Q\ x{\isacharbraceright}{\kern0pt}\ {\isacharequal}{\kern0pt}\ {\isacharbraceleft}{\kern0pt}x\ {\isasymin}\ space\ M{\isachardot}{\kern0pt}\ Q\ x{\isacharbraceright}{\kern0pt}\ {\isasyminter}\ {\isacharparenleft}{\kern0pt}A\ i{\isacharparenright}{\kern0pt}{\isachardoublequoteclose}\ \isacommand{by}\isamarkupfalse%
\ fast\isanewline
\ \ \ \ \isacommand{have}\isamarkupfalse%
\ {\isachardoublequoteopen}Measurable{\isachardot}{\kern0pt}pred\ {\isacharparenleft}{\kern0pt}restrict{\isacharunderscore}{\kern0pt}space\ M\ {\isacharparenleft}{\kern0pt}A\ i{\isacharparenright}{\kern0pt}{\isacharparenright}{\kern0pt}\ Q{\isachardoublequoteclose}\ \isacommand{using}\isamarkupfalse%
\ A\ \isacommand{by}\isamarkupfalse%
\ {\isacharparenleft}{\kern0pt}intro\ measurableI{\isacharcomma}{\kern0pt}\ auto\ simp\ add{\isacharcolon}{\kern0pt}\ space{\isacharunderscore}{\kern0pt}restr\ intro{\isacharbang}{\kern0pt}{\isacharcolon}{\kern0pt}\ sets{\isacharunderscore}{\kern0pt}restrict{\isacharunderscore}{\kern0pt}space{\isacharunderscore}{\kern0pt}iff{\isacharbrackleft}{\kern0pt}THEN\ iffD{\isadigit{2}}{\isacharbrackright}{\kern0pt}{\isacharcomma}{\kern0pt}\ measurable{\isacharcomma}{\kern0pt}\ auto{\isacharparenright}{\kern0pt}\isanewline
\ \ \isacommand{{\isacharbraceright}{\kern0pt}}\isamarkupfalse%
\isanewline
\ \ \isacommand{note}\isamarkupfalse%
\ this{\isacharbrackleft}{\kern0pt}measurable{\isacharbrackright}{\kern0pt}\isanewline
\ \ \isacommand{{\isacharbraceleft}{\kern0pt}}\isamarkupfalse%
\isanewline
\ \ \ \ \isacommand{fix}\isamarkupfalse%
\ i\isanewline
\ \ \ \ \isacommand{have}\isamarkupfalse%
\ {\isachardoublequoteopen}finite{\isacharunderscore}{\kern0pt}measure\ {\isacharparenleft}{\kern0pt}restrict{\isacharunderscore}{\kern0pt}space\ M\ {\isacharparenleft}{\kern0pt}A\ i{\isacharparenright}{\kern0pt}{\isacharparenright}{\kern0pt}{\isachardoublequoteclose}\ \isacommand{using}\isamarkupfalse%
\ emeasure{\isacharunderscore}{\kern0pt}finite\ \isacommand{by}\isamarkupfalse%
\ {\isacharparenleft}{\kern0pt}intro\ finite{\isacharunderscore}{\kern0pt}measureI{\isacharcomma}{\kern0pt}\ subst\ space{\isacharunderscore}{\kern0pt}restr{\isacharcomma}{\kern0pt}\ subst\ emeasure{\isacharunderscore}{\kern0pt}restrict{\isacharunderscore}{\kern0pt}space{\isacharcomma}{\kern0pt}\ auto{\isacharparenright}{\kern0pt}\isanewline
\ \ \ \ \isacommand{hence}\isamarkupfalse%
\ {\isachardoublequoteopen}emeasure\ {\isacharparenleft}{\kern0pt}restrict{\isacharunderscore}{\kern0pt}space\ M\ {\isacharparenleft}{\kern0pt}A\ i{\isacharparenright}{\kern0pt}{\isacharparenright}{\kern0pt}\ {\isacharbraceleft}{\kern0pt}x\ {\isasymin}\ A\ i{\isachardot}{\kern0pt}\ {\isasymnot}Q\ x{\isacharbraceright}{\kern0pt}\ {\isacharequal}{\kern0pt}\ {\isadigit{0}}{\isachardoublequoteclose}\ \isacommand{using}\isamarkupfalse%
\ emeasure{\isacharunderscore}{\kern0pt}finite\ \isacommand{by}\isamarkupfalse%
\ {\isacharparenleft}{\kern0pt}intro\ AE{\isacharunderscore}{\kern0pt}iff{\isacharunderscore}{\kern0pt}measurable{\isacharbrackleft}{\kern0pt}THEN\ iffD{\isadigit{1}}{\isacharcomma}{\kern0pt}\ OF\ {\isacharunderscore}{\kern0pt}\ {\isacharunderscore}{\kern0pt}\ {\isacharasterisk}{\kern0pt}{\isacharbrackright}{\kern0pt}{\isacharcomma}{\kern0pt}\ measurable{\isacharcomma}{\kern0pt}\ subst\ space{\isacharunderscore}{\kern0pt}restr{\isacharbrackleft}{\kern0pt}symmetric{\isacharbrackright}{\kern0pt}{\isacharcomma}{\kern0pt}\ intro\ sets{\isachardot}{\kern0pt}top{\isacharcomma}{\kern0pt}\ auto\ simp\ add{\isacharcolon}{\kern0pt}\ emeasure{\isacharunderscore}{\kern0pt}restrict{\isacharunderscore}{\kern0pt}space{\isacharparenright}{\kern0pt}\isanewline
\ \ \ \ \isacommand{hence}\isamarkupfalse%
\ {\isachardoublequoteopen}emeasure\ M\ {\isacharbraceleft}{\kern0pt}x\ {\isasymin}\ A\ i{\isachardot}{\kern0pt}\ {\isasymnot}\ Q\ x{\isacharbraceright}{\kern0pt}\ {\isacharequal}{\kern0pt}\ {\isadigit{0}}{\isachardoublequoteclose}\ \isacommand{by}\isamarkupfalse%
\ {\isacharparenleft}{\kern0pt}subst\ emeasure{\isacharunderscore}{\kern0pt}restrict{\isacharunderscore}{\kern0pt}space{\isacharbrackleft}{\kern0pt}symmetric{\isacharbrackright}{\kern0pt}{\isacharcomma}{\kern0pt}\ auto{\isacharparenright}{\kern0pt}\isanewline
\ \ \isacommand{{\isacharbraceright}{\kern0pt}}\isamarkupfalse%
\isanewline
\ \ \isacommand{hence}\isamarkupfalse%
\ {\isachardoublequoteopen}emeasure\ M\ {\isacharparenleft}{\kern0pt}{\isasymUnion}i{\isachardot}{\kern0pt}\ {\isacharbraceleft}{\kern0pt}x\ {\isasymin}\ A\ i{\isachardot}{\kern0pt}\ {\isasymnot}\ Q\ x{\isacharbraceright}{\kern0pt}{\isacharparenright}{\kern0pt}\ {\isacharequal}{\kern0pt}\ {\isadigit{0}}{\isachardoublequoteclose}\ \isacommand{by}\isamarkupfalse%
\ {\isacharparenleft}{\kern0pt}intro\ emeasure{\isacharunderscore}{\kern0pt}UN{\isacharunderscore}{\kern0pt}eq{\isacharunderscore}{\kern0pt}{\isadigit{0}}{\isacharcomma}{\kern0pt}\ auto{\isacharparenright}{\kern0pt}\isanewline
\ \ \isacommand{moreover}\isamarkupfalse%
\ \isacommand{have}\isamarkupfalse%
\ {\isachardoublequoteopen}{\isacharparenleft}{\kern0pt}{\isasymUnion}i{\isachardot}{\kern0pt}\ {\isacharbraceleft}{\kern0pt}x\ {\isasymin}\ A\ i{\isachardot}{\kern0pt}\ {\isasymnot}\ Q\ x{\isacharbraceright}{\kern0pt}{\isacharparenright}{\kern0pt}\ {\isacharequal}{\kern0pt}\ {\isacharbraceleft}{\kern0pt}x\ {\isasymin}\ space\ M{\isachardot}{\kern0pt}\ {\isasymnot}\ Q\ x{\isacharbraceright}{\kern0pt}{\isachardoublequoteclose}\ \isacommand{using}\isamarkupfalse%
\ A\ \isacommand{by}\isamarkupfalse%
\ auto\isanewline
\ \ \isacommand{ultimately}\isamarkupfalse%
\ \isacommand{show}\isamarkupfalse%
\ {\isacharquery}{\kern0pt}thesis\ \isacommand{by}\isamarkupfalse%
\ {\isacharparenleft}{\kern0pt}intro\ AE{\isacharunderscore}{\kern0pt}iff{\isacharunderscore}{\kern0pt}measurable{\isacharbrackleft}{\kern0pt}THEN\ iffD{\isadigit{2}}{\isacharbrackright}{\kern0pt}{\isacharcomma}{\kern0pt}\ auto{\isacharparenright}{\kern0pt}\isanewline
\isacommand{qed}\isamarkupfalse%
%
\endisatagproof
{\isafoldproof}%
%
\isadelimproof
\isanewline
%
\endisadelimproof
\isanewline
\isanewline
\isacommand{lemma}\isamarkupfalse%
\ averaging{\isacharunderscore}{\kern0pt}theorem{\isacharcolon}{\kern0pt}\isanewline
\ \ \isakeyword{fixes}\ f{\isacharcolon}{\kern0pt}{\isacharcolon}{\kern0pt}{\isachardoublequoteopen}{\isacharunderscore}{\kern0pt}\ {\isasymRightarrow}\ {\isacharprime}{\kern0pt}b{\isacharcolon}{\kern0pt}{\isacharcolon}{\kern0pt}{\isacharbraceleft}{\kern0pt}second{\isacharunderscore}{\kern0pt}countable{\isacharunderscore}{\kern0pt}topology{\isacharcomma}{\kern0pt}\ banach{\isacharbraceright}{\kern0pt}{\isachardoublequoteclose}\isanewline
\ \ \isakeyword{assumes}\ {\isacharbrackleft}{\kern0pt}measurable{\isacharbrackright}{\kern0pt}{\isacharcolon}{\kern0pt}{\isachardoublequoteopen}integrable\ M\ f{\isachardoublequoteclose}\ \isanewline
\ \ \ \ \ \ \isakeyword{and}\ closed{\isacharcolon}{\kern0pt}\ {\isachardoublequoteopen}closed\ S{\isachardoublequoteclose}\isanewline
\ \ \ \ \ \ \isakeyword{and}\ {\isachardoublequoteopen}{\isasymAnd}A{\isachardot}{\kern0pt}\ A\ {\isasymin}\ sets\ M\ {\isasymLongrightarrow}\ measure\ M\ A\ {\isachargreater}{\kern0pt}\ {\isadigit{0}}\ {\isasymLongrightarrow}\ {\isacharparenleft}{\kern0pt}{\isadigit{1}}\ {\isacharslash}{\kern0pt}\ measure\ M\ A{\isacharparenright}{\kern0pt}\ {\isacharasterisk}{\kern0pt}\isactrlsub R\ set{\isacharunderscore}{\kern0pt}lebesgue{\isacharunderscore}{\kern0pt}integral\ M\ A\ f\ {\isasymin}\ S{\isachardoublequoteclose}\isanewline
\ \ \ \ \isakeyword{shows}\ {\isachardoublequoteopen}AE\ x\ in\ M{\isachardot}{\kern0pt}\ f\ x\ {\isasymin}\ S{\isachardoublequoteclose}\isanewline
%
\isadelimproof
%
\endisadelimproof
%
\isatagproof
\isacommand{proof}\isamarkupfalse%
\ {\isacharparenleft}{\kern0pt}induct\ rule{\isacharcolon}{\kern0pt}\ sigma{\isacharunderscore}{\kern0pt}finite{\isacharunderscore}{\kern0pt}measure{\isacharunderscore}{\kern0pt}induct{\isacharparenright}{\kern0pt}\isanewline
\ \ \isacommand{case}\isamarkupfalse%
\ {\isacharparenleft}{\kern0pt}finite{\isacharunderscore}{\kern0pt}measure\ N\ {\isasymOmega}{\isacharparenright}{\kern0pt}\isanewline
\isanewline
\ \ \isacommand{interpret}\isamarkupfalse%
\ finite{\isacharunderscore}{\kern0pt}measure\ N\ \isacommand{by}\isamarkupfalse%
\ {\isacharparenleft}{\kern0pt}rule\ finite{\isacharunderscore}{\kern0pt}measure{\isacharparenright}{\kern0pt}\isanewline
\ \ \isanewline
\ \ \isacommand{have}\isamarkupfalse%
\ integrable{\isacharbrackleft}{\kern0pt}measurable{\isacharbrackright}{\kern0pt}{\isacharcolon}{\kern0pt}\ {\isachardoublequoteopen}integrable\ N\ f{\isachardoublequoteclose}\ \isacommand{using}\isamarkupfalse%
\ assms\ finite{\isacharunderscore}{\kern0pt}measure\ \isacommand{by}\isamarkupfalse%
\ {\isacharparenleft}{\kern0pt}auto\ simp{\isacharcolon}{\kern0pt}\ integrable{\isacharunderscore}{\kern0pt}restrict{\isacharunderscore}{\kern0pt}space\ integrable{\isacharunderscore}{\kern0pt}mult{\isacharunderscore}{\kern0pt}indicator{\isacharparenright}{\kern0pt}\isanewline
\ \ \isacommand{have}\isamarkupfalse%
\ average{\isacharcolon}{\kern0pt}\ {\isachardoublequoteopen}{\isacharparenleft}{\kern0pt}{\isadigit{1}}\ {\isacharslash}{\kern0pt}\ Sigma{\isacharunderscore}{\kern0pt}Algebra{\isachardot}{\kern0pt}measure\ N\ A{\isacharparenright}{\kern0pt}\ {\isacharasterisk}{\kern0pt}\isactrlsub R\ set{\isacharunderscore}{\kern0pt}lebesgue{\isacharunderscore}{\kern0pt}integral\ N\ A\ f\ {\isasymin}\ S{\isachardoublequoteclose}\ \isakeyword{if}\ {\isachardoublequoteopen}A\ {\isasymin}\ sets\ N{\isachardoublequoteclose}\ {\isachardoublequoteopen}measure\ N\ A\ {\isachargreater}{\kern0pt}\ {\isadigit{0}}{\isachardoublequoteclose}\ \isakeyword{for}\ A\isanewline
\ \ \isacommand{proof}\isamarkupfalse%
\ {\isacharminus}{\kern0pt}\isanewline
\ \ \ \ \isacommand{have}\isamarkupfalse%
\ {\isacharasterisk}{\kern0pt}{\isacharcolon}{\kern0pt}\ {\isachardoublequoteopen}A\ {\isasymin}\ sets\ M{\isachardoublequoteclose}\ \isacommand{using}\isamarkupfalse%
\ that\ \isacommand{by}\isamarkupfalse%
\ {\isacharparenleft}{\kern0pt}simp\ add{\isacharcolon}{\kern0pt}\ sets{\isacharunderscore}{\kern0pt}restrict{\isacharunderscore}{\kern0pt}space{\isacharunderscore}{\kern0pt}iff\ finite{\isacharunderscore}{\kern0pt}measure{\isacharparenright}{\kern0pt}\isanewline
\ \ \ \ \isacommand{have}\isamarkupfalse%
\ {\isachardoublequoteopen}A\ {\isacharequal}{\kern0pt}\ A\ {\isasyminter}\ {\isasymOmega}{\isachardoublequoteclose}\ \isacommand{by}\isamarkupfalse%
\ {\isacharparenleft}{\kern0pt}metis\ finite{\isacharunderscore}{\kern0pt}measure{\isacharparenleft}{\kern0pt}{\isadigit{2}}{\isacharcomma}{\kern0pt}{\isadigit{3}}{\isacharparenright}{\kern0pt}\ inf{\isachardot}{\kern0pt}orderE\ sets{\isachardot}{\kern0pt}sets{\isacharunderscore}{\kern0pt}into{\isacharunderscore}{\kern0pt}space\ space{\isacharunderscore}{\kern0pt}restrict{\isacharunderscore}{\kern0pt}space\ that{\isacharparenleft}{\kern0pt}{\isadigit{1}}{\isacharparenright}{\kern0pt}{\isacharparenright}{\kern0pt}\isanewline
\ \ \ \ \isacommand{hence}\isamarkupfalse%
\ {\isachardoublequoteopen}set{\isacharunderscore}{\kern0pt}lebesgue{\isacharunderscore}{\kern0pt}integral\ N\ A\ f\ {\isacharequal}{\kern0pt}\ set{\isacharunderscore}{\kern0pt}lebesgue{\isacharunderscore}{\kern0pt}integral\ M\ A\ f{\isachardoublequoteclose}\ \isacommand{unfolding}\isamarkupfalse%
\ finite{\isacharunderscore}{\kern0pt}measure\ \isacommand{by}\isamarkupfalse%
\ {\isacharparenleft}{\kern0pt}subst\ set{\isacharunderscore}{\kern0pt}integral{\isacharunderscore}{\kern0pt}restrict{\isacharunderscore}{\kern0pt}space{\isacharcomma}{\kern0pt}\ auto\ simp\ add{\isacharcolon}{\kern0pt}\ finite{\isacharunderscore}{\kern0pt}measure\ set{\isacharunderscore}{\kern0pt}lebesgue{\isacharunderscore}{\kern0pt}integral{\isacharunderscore}{\kern0pt}def\ indicator{\isacharunderscore}{\kern0pt}inter{\isacharunderscore}{\kern0pt}arith{\isacharbrackleft}{\kern0pt}symmetric{\isacharbrackright}{\kern0pt}{\isacharparenright}{\kern0pt}\isanewline
\ \ \ \ \isacommand{moreover}\isamarkupfalse%
\ \isacommand{have}\isamarkupfalse%
\ {\isachardoublequoteopen}measure\ N\ A\ {\isacharequal}{\kern0pt}\ measure\ M\ A{\isachardoublequoteclose}\ \isacommand{using}\isamarkupfalse%
\ that\ \isacommand{by}\isamarkupfalse%
\ {\isacharparenleft}{\kern0pt}auto\ intro{\isacharbang}{\kern0pt}{\isacharcolon}{\kern0pt}\ measure{\isacharunderscore}{\kern0pt}restrict{\isacharunderscore}{\kern0pt}space\ simp\ add{\isacharcolon}{\kern0pt}\ finite{\isacharunderscore}{\kern0pt}measure\ sets{\isacharunderscore}{\kern0pt}restrict{\isacharunderscore}{\kern0pt}space{\isacharunderscore}{\kern0pt}iff{\isacharparenright}{\kern0pt}\isanewline
\ \ \ \ \isacommand{ultimately}\isamarkupfalse%
\ \isacommand{show}\isamarkupfalse%
\ {\isacharquery}{\kern0pt}thesis\ \isacommand{using}\isamarkupfalse%
\ that\ {\isacharasterisk}{\kern0pt}\ assms{\isacharparenleft}{\kern0pt}{\isadigit{3}}{\isacharparenright}{\kern0pt}\ \isacommand{by}\isamarkupfalse%
\ presburger\isanewline
\ \ \isacommand{qed}\isamarkupfalse%
\isanewline
\isanewline
\ \ \isacommand{obtain}\isamarkupfalse%
\ D\ {\isacharcolon}{\kern0pt}{\isacharcolon}{\kern0pt}\ {\isachardoublequoteopen}{\isacharprime}{\kern0pt}b\ set{\isachardoublequoteclose}\ \isakeyword{where}\ balls{\isacharunderscore}{\kern0pt}basis{\isacharcolon}{\kern0pt}\ {\isachardoublequoteopen}topological{\isacharunderscore}{\kern0pt}basis\ {\isacharparenleft}{\kern0pt}case{\isacharunderscore}{\kern0pt}prod\ ball\ {\isacharbackquote}{\kern0pt}\ {\isacharparenleft}{\kern0pt}D\ {\isasymtimes}\ {\isacharparenleft}{\kern0pt}{\isasymrat}\ {\isasyminter}\ {\isacharbraceleft}{\kern0pt}{\isadigit{0}}{\isacharless}{\kern0pt}{\isachardot}{\kern0pt}{\isachardot}{\kern0pt}{\isacharbraceright}{\kern0pt}{\isacharparenright}{\kern0pt}{\isacharparenright}{\kern0pt}{\isacharparenright}{\kern0pt}{\isachardoublequoteclose}\ \isakeyword{and}\ countable{\isacharunderscore}{\kern0pt}D{\isacharcolon}{\kern0pt}\ {\isachardoublequoteopen}countable\ D{\isachardoublequoteclose}\ \isacommand{using}\isamarkupfalse%
\ balls{\isacharunderscore}{\kern0pt}countable{\isacharunderscore}{\kern0pt}basis\ \isacommand{by}\isamarkupfalse%
\ blast\isanewline
\ \ \isacommand{have}\isamarkupfalse%
\ countable{\isacharunderscore}{\kern0pt}balls{\isacharcolon}{\kern0pt}\ {\isachardoublequoteopen}countable\ {\isacharparenleft}{\kern0pt}case{\isacharunderscore}{\kern0pt}prod\ ball\ {\isacharbackquote}{\kern0pt}\ {\isacharparenleft}{\kern0pt}D\ {\isasymtimes}\ {\isacharparenleft}{\kern0pt}{\isasymrat}\ {\isasyminter}\ {\isacharbraceleft}{\kern0pt}{\isadigit{0}}{\isacharless}{\kern0pt}{\isachardot}{\kern0pt}{\isachardot}{\kern0pt}{\isacharbraceright}{\kern0pt}{\isacharparenright}{\kern0pt}{\isacharparenright}{\kern0pt}{\isacharparenright}{\kern0pt}{\isachardoublequoteclose}\ \isacommand{using}\isamarkupfalse%
\ countable{\isacharunderscore}{\kern0pt}rat\ countable{\isacharunderscore}{\kern0pt}D\ \isacommand{by}\isamarkupfalse%
\ blast\isanewline
\isanewline
\ \ \isacommand{obtain}\isamarkupfalse%
\ B\ \isakeyword{where}\ B{\isacharunderscore}{\kern0pt}balls{\isacharcolon}{\kern0pt}\ {\isachardoublequoteopen}B\ {\isasymsubseteq}\ case{\isacharunderscore}{\kern0pt}prod\ ball\ {\isacharbackquote}{\kern0pt}\ {\isacharparenleft}{\kern0pt}D\ {\isasymtimes}\ {\isacharparenleft}{\kern0pt}{\isasymrat}\ {\isasyminter}\ {\isacharbraceleft}{\kern0pt}{\isadigit{0}}{\isacharless}{\kern0pt}{\isachardot}{\kern0pt}{\isachardot}{\kern0pt}{\isacharbraceright}{\kern0pt}{\isacharparenright}{\kern0pt}{\isacharparenright}{\kern0pt}{\isachardoublequoteclose}\ {\isachardoublequoteopen}{\isasymUnion}B\ {\isacharequal}{\kern0pt}\ {\isacharminus}{\kern0pt}S{\isachardoublequoteclose}\ \isacommand{using}\isamarkupfalse%
\ topological{\isacharunderscore}{\kern0pt}basis{\isacharbrackleft}{\kern0pt}THEN\ iffD{\isadigit{1}}{\isacharcomma}{\kern0pt}\ OF\ balls{\isacharunderscore}{\kern0pt}basis{\isacharbrackright}{\kern0pt}\ open{\isacharunderscore}{\kern0pt}Compl{\isacharbrackleft}{\kern0pt}OF\ assms{\isacharparenleft}{\kern0pt}{\isadigit{2}}{\isacharparenright}{\kern0pt}{\isacharbrackright}{\kern0pt}\ \isacommand{by}\isamarkupfalse%
\ meson\isanewline
\ \ \isacommand{hence}\isamarkupfalse%
\ countable{\isacharunderscore}{\kern0pt}B{\isacharcolon}{\kern0pt}\ {\isachardoublequoteopen}countable\ B{\isachardoublequoteclose}\ \isacommand{using}\isamarkupfalse%
\ countable{\isacharunderscore}{\kern0pt}balls\ countable{\isacharunderscore}{\kern0pt}subset\ \isacommand{by}\isamarkupfalse%
\ fast\isanewline
\isanewline
\ \ \isacommand{define}\isamarkupfalse%
\ b\ \isakeyword{where}\ {\isachardoublequoteopen}b\ {\isacharequal}{\kern0pt}\ from{\isacharunderscore}{\kern0pt}nat{\isacharunderscore}{\kern0pt}into\ {\isacharparenleft}{\kern0pt}B\ {\isasymunion}\ {\isacharbraceleft}{\kern0pt}{\isacharbraceleft}{\kern0pt}{\isacharbraceright}{\kern0pt}{\isacharbraceright}{\kern0pt}{\isacharparenright}{\kern0pt}{\isachardoublequoteclose}\isanewline
\ \ \isacommand{have}\isamarkupfalse%
\ {\isachardoublequoteopen}B\ {\isasymunion}\ {\isacharbraceleft}{\kern0pt}{\isacharbraceleft}{\kern0pt}{\isacharbraceright}{\kern0pt}{\isacharbraceright}{\kern0pt}\ {\isasymnoteq}\ {\isacharbraceleft}{\kern0pt}{\isacharbraceright}{\kern0pt}{\isachardoublequoteclose}\ \isacommand{by}\isamarkupfalse%
\ simp\isanewline
\ \ \isacommand{have}\isamarkupfalse%
\ range{\isacharunderscore}{\kern0pt}b{\isacharcolon}{\kern0pt}\ {\isachardoublequoteopen}range\ b\ {\isacharequal}{\kern0pt}\ B\ {\isasymunion}\ {\isacharbraceleft}{\kern0pt}{\isacharbraceleft}{\kern0pt}{\isacharbraceright}{\kern0pt}{\isacharbraceright}{\kern0pt}{\isachardoublequoteclose}\ \isacommand{using}\isamarkupfalse%
\ countable{\isacharunderscore}{\kern0pt}B\ \isacommand{by}\isamarkupfalse%
\ {\isacharparenleft}{\kern0pt}auto\ simp\ add{\isacharcolon}{\kern0pt}\ b{\isacharunderscore}{\kern0pt}def\ intro{\isacharbang}{\kern0pt}{\isacharcolon}{\kern0pt}\ range{\isacharunderscore}{\kern0pt}from{\isacharunderscore}{\kern0pt}nat{\isacharunderscore}{\kern0pt}into{\isacharparenright}{\kern0pt}\isanewline
\ \ \isacommand{have}\isamarkupfalse%
\ open{\isacharunderscore}{\kern0pt}b{\isacharcolon}{\kern0pt}\ {\isachardoublequoteopen}open\ {\isacharparenleft}{\kern0pt}b\ i{\isacharparenright}{\kern0pt}{\isachardoublequoteclose}\ \isakeyword{for}\ i\ \isacommand{unfolding}\isamarkupfalse%
\ b{\isacharunderscore}{\kern0pt}def\ \isacommand{using}\isamarkupfalse%
\ B{\isacharunderscore}{\kern0pt}balls\ open{\isacharunderscore}{\kern0pt}ball\ from{\isacharunderscore}{\kern0pt}nat{\isacharunderscore}{\kern0pt}into{\isacharbrackleft}{\kern0pt}of\ {\isachardoublequoteopen}B\ {\isasymunion}\ {\isacharbraceleft}{\kern0pt}{\isacharbraceleft}{\kern0pt}{\isacharbraceright}{\kern0pt}{\isacharbraceright}{\kern0pt}{\isachardoublequoteclose}\ i{\isacharbrackright}{\kern0pt}\ \isacommand{by}\isamarkupfalse%
\ force\isanewline
\ \ \isacommand{have}\isamarkupfalse%
\ Union{\isacharunderscore}{\kern0pt}range{\isacharunderscore}{\kern0pt}b{\isacharcolon}{\kern0pt}\ {\isachardoublequoteopen}{\isasymUnion}{\isacharparenleft}{\kern0pt}range\ b{\isacharparenright}{\kern0pt}\ {\isacharequal}{\kern0pt}\ {\isacharminus}{\kern0pt}S{\isachardoublequoteclose}\ \isacommand{using}\isamarkupfalse%
\ B{\isacharunderscore}{\kern0pt}balls\ range{\isacharunderscore}{\kern0pt}b\ \isacommand{by}\isamarkupfalse%
\ simp\isanewline
\isanewline
\ \ \isacommand{{\isacharbraceleft}{\kern0pt}}\isamarkupfalse%
\isanewline
\ \ \ \ \isacommand{fix}\isamarkupfalse%
\ v\ r\ \isacommand{assume}\isamarkupfalse%
\ ball{\isacharunderscore}{\kern0pt}in{\isacharunderscore}{\kern0pt}Compl{\isacharcolon}{\kern0pt}\ {\isachardoublequoteopen}ball\ v\ r\ {\isasymsubseteq}\ {\isacharminus}{\kern0pt}S{\isachardoublequoteclose}\isanewline
\ \ \ \ \isacommand{define}\isamarkupfalse%
\ A\ \isakeyword{where}\ {\isachardoublequoteopen}A\ {\isacharequal}{\kern0pt}\ f\ {\isacharminus}{\kern0pt}{\isacharbackquote}{\kern0pt}\ ball\ v\ r\ {\isasyminter}\ space\ N{\isachardoublequoteclose}\isanewline
\ \ \ \ \isacommand{have}\isamarkupfalse%
\ dist{\isacharunderscore}{\kern0pt}less{\isacharcolon}{\kern0pt}\ {\isachardoublequoteopen}dist\ {\isacharparenleft}{\kern0pt}f\ x{\isacharparenright}{\kern0pt}\ v\ {\isacharless}{\kern0pt}\ r{\isachardoublequoteclose}\ \isakeyword{if}\ {\isachardoublequoteopen}x\ {\isasymin}\ A{\isachardoublequoteclose}\ \isakeyword{for}\ x\ \isacommand{using}\isamarkupfalse%
\ that\ \isacommand{unfolding}\isamarkupfalse%
\ A{\isacharunderscore}{\kern0pt}def\ vimage{\isacharunderscore}{\kern0pt}def\ \isacommand{by}\isamarkupfalse%
\ {\isacharparenleft}{\kern0pt}simp\ add{\isacharcolon}{\kern0pt}\ dist{\isacharunderscore}{\kern0pt}commute{\isacharparenright}{\kern0pt}\isanewline
\ \ \ \ \isacommand{hence}\isamarkupfalse%
\ AE{\isacharunderscore}{\kern0pt}less{\isacharcolon}{\kern0pt}\ {\isachardoublequoteopen}AE\ x\ {\isasymin}\ A\ in\ N{\isachardot}{\kern0pt}\ norm\ {\isacharparenleft}{\kern0pt}f\ x\ {\isacharminus}{\kern0pt}\ v{\isacharparenright}{\kern0pt}\ {\isacharless}{\kern0pt}\ r{\isachardoublequoteclose}\ \isacommand{by}\isamarkupfalse%
\ {\isacharparenleft}{\kern0pt}auto\ simp\ add{\isacharcolon}{\kern0pt}\ dist{\isacharunderscore}{\kern0pt}norm{\isacharparenright}{\kern0pt}\isanewline
\ \ \ \ \isacommand{have}\isamarkupfalse%
\ {\isacharasterisk}{\kern0pt}{\isacharcolon}{\kern0pt}\ {\isachardoublequoteopen}A\ {\isasymin}\ sets\ N{\isachardoublequoteclose}\ \isacommand{unfolding}\isamarkupfalse%
\ A{\isacharunderscore}{\kern0pt}def\ \isacommand{by}\isamarkupfalse%
\ simp\isanewline
\ \ \ \ \isacommand{have}\isamarkupfalse%
\ {\isachardoublequoteopen}emeasure\ N\ A\ {\isacharequal}{\kern0pt}\ {\isadigit{0}}{\isachardoublequoteclose}\ \isanewline
\ \ \ \ \isacommand{proof}\isamarkupfalse%
\ {\isacharminus}{\kern0pt}\isanewline
\ \ \ \ \ \ \isacommand{{\isacharbraceleft}{\kern0pt}}\isamarkupfalse%
\isanewline
\ \ \ \ \ \ \ \ \isacommand{assume}\isamarkupfalse%
\ asm{\isacharcolon}{\kern0pt}\ {\isachardoublequoteopen}emeasure\ N\ A\ {\isachargreater}{\kern0pt}\ {\isadigit{0}}{\isachardoublequoteclose}\isanewline
\ \ \ \ \ \ \ \ \isacommand{hence}\isamarkupfalse%
\ measure{\isacharunderscore}{\kern0pt}pos{\isacharcolon}{\kern0pt}\ {\isachardoublequoteopen}measure\ N\ A\ {\isachargreater}{\kern0pt}\ {\isadigit{0}}{\isachardoublequoteclose}\ \isacommand{unfolding}\isamarkupfalse%
\ emeasure{\isacharunderscore}{\kern0pt}eq{\isacharunderscore}{\kern0pt}measure\ \isacommand{by}\isamarkupfalse%
\ simp\isanewline
\ \ \ \ \ \ \ \ \isacommand{hence}\isamarkupfalse%
\ {\isachardoublequoteopen}{\isacharparenleft}{\kern0pt}{\isadigit{1}}\ {\isacharslash}{\kern0pt}\ measure\ N\ A{\isacharparenright}{\kern0pt}\ {\isacharasterisk}{\kern0pt}\isactrlsub R\ set{\isacharunderscore}{\kern0pt}lebesgue{\isacharunderscore}{\kern0pt}integral\ N\ A\ f\ {\isacharminus}{\kern0pt}\ v\ {\isacharequal}{\kern0pt}\ {\isacharparenleft}{\kern0pt}{\isadigit{1}}\ {\isacharslash}{\kern0pt}\ measure\ N\ A{\isacharparenright}{\kern0pt}\ {\isacharasterisk}{\kern0pt}\isactrlsub R\ set{\isacharunderscore}{\kern0pt}lebesgue{\isacharunderscore}{\kern0pt}integral\ N\ A\ {\isacharparenleft}{\kern0pt}{\isasymlambda}x{\isachardot}{\kern0pt}\ f\ x\ {\isacharminus}{\kern0pt}\ v{\isacharparenright}{\kern0pt}{\isachardoublequoteclose}\ \isacommand{using}\isamarkupfalse%
\ integrable\ integrable{\isacharunderscore}{\kern0pt}const\ {\isacharasterisk}{\kern0pt}\ \isacommand{by}\isamarkupfalse%
\ {\isacharparenleft}{\kern0pt}subst\ set{\isacharunderscore}{\kern0pt}integral{\isacharunderscore}{\kern0pt}diff{\isacharparenleft}{\kern0pt}{\isadigit{2}}{\isacharparenright}{\kern0pt}{\isacharcomma}{\kern0pt}\ auto\ simp\ add{\isacharcolon}{\kern0pt}\ set{\isacharunderscore}{\kern0pt}integrable{\isacharunderscore}{\kern0pt}def\ set{\isacharunderscore}{\kern0pt}integral{\isacharunderscore}{\kern0pt}const{\isacharbrackleft}{\kern0pt}OF\ {\isacharasterisk}{\kern0pt}{\isacharbrackright}{\kern0pt}\ algebra{\isacharunderscore}{\kern0pt}simps\ intro{\isacharbang}{\kern0pt}{\isacharcolon}{\kern0pt}\ integrable{\isacharunderscore}{\kern0pt}mult{\isacharunderscore}{\kern0pt}indicator{\isacharparenright}{\kern0pt}\isanewline
\ \ \ \ \ \ \ \ \isacommand{moreover}\isamarkupfalse%
\ \isacommand{have}\isamarkupfalse%
\ {\isachardoublequoteopen}norm\ {\isacharparenleft}{\kern0pt}{\isasymintegral}x{\isasymin}A{\isachardot}{\kern0pt}\ {\isacharparenleft}{\kern0pt}f\ x\ {\isacharminus}{\kern0pt}\ v{\isacharparenright}{\kern0pt}{\isasympartial}N{\isacharparenright}{\kern0pt}\ {\isasymle}\ {\isacharparenleft}{\kern0pt}{\isasymintegral}x{\isasymin}A{\isachardot}{\kern0pt}\ norm\ {\isacharparenleft}{\kern0pt}f\ x\ {\isacharminus}{\kern0pt}\ v{\isacharparenright}{\kern0pt}{\isasympartial}N{\isacharparenright}{\kern0pt}{\isachardoublequoteclose}\ \isacommand{using}\isamarkupfalse%
\ {\isacharasterisk}{\kern0pt}\ \isacommand{by}\isamarkupfalse%
\ {\isacharparenleft}{\kern0pt}auto\ intro{\isacharbang}{\kern0pt}{\isacharcolon}{\kern0pt}\ integral{\isacharunderscore}{\kern0pt}norm{\isacharunderscore}{\kern0pt}bound{\isacharbrackleft}{\kern0pt}of\ N\ {\isachardoublequoteopen}{\isasymlambda}x{\isachardot}{\kern0pt}\ indicator\ A\ x\ {\isacharasterisk}{\kern0pt}\isactrlsub R\ {\isacharparenleft}{\kern0pt}f\ x\ {\isacharminus}{\kern0pt}\ v{\isacharparenright}{\kern0pt}{\isachardoublequoteclose}{\isacharcomma}{\kern0pt}\ THEN\ order{\isacharunderscore}{\kern0pt}trans{\isacharbrackright}{\kern0pt}\ integrable{\isacharunderscore}{\kern0pt}mult{\isacharunderscore}{\kern0pt}indicator\ integrable\ simp\ add{\isacharcolon}{\kern0pt}\ set{\isacharunderscore}{\kern0pt}lebesgue{\isacharunderscore}{\kern0pt}integral{\isacharunderscore}{\kern0pt}def{\isacharparenright}{\kern0pt}\isanewline
\ \ \ \ \ \ \ \ \isacommand{ultimately}\isamarkupfalse%
\ \isacommand{have}\isamarkupfalse%
\ {\isachardoublequoteopen}norm\ {\isacharparenleft}{\kern0pt}{\isacharparenleft}{\kern0pt}{\isadigit{1}}\ {\isacharslash}{\kern0pt}\ measure\ N\ A{\isacharparenright}{\kern0pt}\ {\isacharasterisk}{\kern0pt}\isactrlsub R\ set{\isacharunderscore}{\kern0pt}lebesgue{\isacharunderscore}{\kern0pt}integral\ N\ A\ f\ {\isacharminus}{\kern0pt}\ v{\isacharparenright}{\kern0pt}\ {\isasymle}\ \ set{\isacharunderscore}{\kern0pt}lebesgue{\isacharunderscore}{\kern0pt}integral\ N\ A\ {\isacharparenleft}{\kern0pt}{\isasymlambda}x{\isachardot}{\kern0pt}\ norm\ {\isacharparenleft}{\kern0pt}f\ x\ {\isacharminus}{\kern0pt}\ v{\isacharparenright}{\kern0pt}{\isacharparenright}{\kern0pt}\ {\isacharslash}{\kern0pt}\ measure\ N\ A{\isachardoublequoteclose}\ \isacommand{using}\isamarkupfalse%
\ asm\ \isacommand{by}\isamarkupfalse%
\ {\isacharparenleft}{\kern0pt}auto\ intro{\isacharcolon}{\kern0pt}\ divide{\isacharunderscore}{\kern0pt}right{\isacharunderscore}{\kern0pt}mono{\isacharparenright}{\kern0pt}\isanewline
\ \ \ \ \ \ \ \ \isacommand{also}\isamarkupfalse%
\ \isacommand{have}\isamarkupfalse%
\ {\isachardoublequoteopen}{\isachardot}{\kern0pt}{\isachardot}{\kern0pt}{\isachardot}{\kern0pt}\ {\isacharless}{\kern0pt}\ set{\isacharunderscore}{\kern0pt}lebesgue{\isacharunderscore}{\kern0pt}integral\ N\ A\ {\isacharparenleft}{\kern0pt}{\isasymlambda}x{\isachardot}{\kern0pt}\ r{\isacharparenright}{\kern0pt}\ {\isacharslash}{\kern0pt}\ measure\ N\ A{\isachardoublequoteclose}\ \isanewline
\ \ \ \ \ \ \ \ \ \ \isacommand{unfolding}\isamarkupfalse%
\ set{\isacharunderscore}{\kern0pt}lebesgue{\isacharunderscore}{\kern0pt}integral{\isacharunderscore}{\kern0pt}def\ \isanewline
\ \ \ \ \ \ \ \ \ \ \isacommand{using}\isamarkupfalse%
\ asm\ {\isacharasterisk}{\kern0pt}\ integrable\ integrable{\isacharunderscore}{\kern0pt}const\ AE{\isacharunderscore}{\kern0pt}less\ measure{\isacharunderscore}{\kern0pt}pos\isanewline
\ \ \ \ \ \ \ \ \ \ \isacommand{by}\isamarkupfalse%
\ {\isacharparenleft}{\kern0pt}intro\ divide{\isacharunderscore}{\kern0pt}strict{\isacharunderscore}{\kern0pt}right{\isacharunderscore}{\kern0pt}mono\ integral{\isacharunderscore}{\kern0pt}less{\isacharunderscore}{\kern0pt}AE{\isacharbrackleft}{\kern0pt}of\ {\isacharunderscore}{\kern0pt}\ {\isacharunderscore}{\kern0pt}\ A{\isacharbrackright}{\kern0pt}\ integrable{\isacharunderscore}{\kern0pt}mult{\isacharunderscore}{\kern0pt}indicator{\isacharparenright}{\kern0pt}\isanewline
\ \ \ \ \ \ \ \ \ \ \ \ {\isacharparenleft}{\kern0pt}fastforce\ simp\ add{\isacharcolon}{\kern0pt}\ dist{\isacharunderscore}{\kern0pt}less\ dist{\isacharunderscore}{\kern0pt}norm\ indicator{\isacharunderscore}{\kern0pt}def{\isacharparenright}{\kern0pt}{\isacharplus}{\kern0pt}\isanewline
\ \ \ \ \ \ \ \ \isacommand{also}\isamarkupfalse%
\ \isacommand{have}\isamarkupfalse%
\ {\isachardoublequoteopen}{\isachardot}{\kern0pt}{\isachardot}{\kern0pt}{\isachardot}{\kern0pt}\ {\isacharequal}{\kern0pt}\ r{\isachardoublequoteclose}\ \isacommand{using}\isamarkupfalse%
\ {\isacharasterisk}{\kern0pt}\ measure{\isacharunderscore}{\kern0pt}pos\ \isacommand{by}\isamarkupfalse%
\ {\isacharparenleft}{\kern0pt}simp\ add{\isacharcolon}{\kern0pt}\ set{\isacharunderscore}{\kern0pt}integral{\isacharunderscore}{\kern0pt}const{\isacharparenright}{\kern0pt}\isanewline
\ \ \ \ \ \ \ \ \isacommand{finally}\isamarkupfalse%
\ \isacommand{have}\isamarkupfalse%
\ {\isachardoublequoteopen}dist\ {\isacharparenleft}{\kern0pt}{\isacharparenleft}{\kern0pt}{\isadigit{1}}\ {\isacharslash}{\kern0pt}\ measure\ N\ A{\isacharparenright}{\kern0pt}\ {\isacharasterisk}{\kern0pt}\isactrlsub R\ set{\isacharunderscore}{\kern0pt}lebesgue{\isacharunderscore}{\kern0pt}integral\ N\ A\ f{\isacharparenright}{\kern0pt}\ v\ {\isacharless}{\kern0pt}\ r{\isachardoublequoteclose}\ \isacommand{by}\isamarkupfalse%
\ {\isacharparenleft}{\kern0pt}subst\ dist{\isacharunderscore}{\kern0pt}norm{\isacharparenright}{\kern0pt}\isanewline
\ \ \ \ \ \ \ \ \isacommand{hence}\isamarkupfalse%
\ {\isachardoublequoteopen}False{\isachardoublequoteclose}\ \isacommand{using}\isamarkupfalse%
\ average{\isacharbrackleft}{\kern0pt}OF\ {\isacharasterisk}{\kern0pt}\ measure{\isacharunderscore}{\kern0pt}pos{\isacharbrackright}{\kern0pt}\ \isacommand{by}\isamarkupfalse%
\ {\isacharparenleft}{\kern0pt}metis\ ComplD\ dist{\isacharunderscore}{\kern0pt}commute\ in{\isacharunderscore}{\kern0pt}mono\ mem{\isacharunderscore}{\kern0pt}ball\ ball{\isacharunderscore}{\kern0pt}in{\isacharunderscore}{\kern0pt}Compl{\isacharparenright}{\kern0pt}\isanewline
\ \ \ \ \ \ \isacommand{{\isacharbraceright}{\kern0pt}}\isamarkupfalse%
\isanewline
\ \ \ \ \ \ \isacommand{thus}\isamarkupfalse%
\ {\isacharquery}{\kern0pt}thesis\ \isacommand{by}\isamarkupfalse%
\ fastforce\isanewline
\ \ \ \ \isacommand{qed}\isamarkupfalse%
\isanewline
\ \ \isacommand{{\isacharbraceright}{\kern0pt}}\isamarkupfalse%
\isanewline
\ \ \isacommand{note}\isamarkupfalse%
\ {\isacharasterisk}{\kern0pt}\ {\isacharequal}{\kern0pt}\ this\isanewline
\ \ \isacommand{{\isacharbraceleft}{\kern0pt}}\isamarkupfalse%
\isanewline
\ \ \ \ \isacommand{fix}\isamarkupfalse%
\ b{\isacharprime}{\kern0pt}\ \isacommand{assume}\isamarkupfalse%
\ {\isachardoublequoteopen}b{\isacharprime}{\kern0pt}\ {\isasymin}\ B{\isachardoublequoteclose}\isanewline
\ \ \ \ \isacommand{hence}\isamarkupfalse%
\ ball{\isacharunderscore}{\kern0pt}subset{\isacharunderscore}{\kern0pt}Compl{\isacharcolon}{\kern0pt}\ {\isachardoublequoteopen}b{\isacharprime}{\kern0pt}\ {\isasymsubseteq}\ {\isacharminus}{\kern0pt}S{\isachardoublequoteclose}\ \isakeyword{and}\ ball{\isacharunderscore}{\kern0pt}radius{\isacharunderscore}{\kern0pt}pos{\isacharcolon}{\kern0pt}\ {\isachardoublequoteopen}{\isasymexists}v\ {\isasymin}\ D{\isachardot}{\kern0pt}\ {\isasymexists}r{\isachargreater}{\kern0pt}{\isadigit{0}}{\isachardot}{\kern0pt}\ b{\isacharprime}{\kern0pt}\ {\isacharequal}{\kern0pt}\ ball\ v\ r{\isachardoublequoteclose}\ \isacommand{using}\isamarkupfalse%
\ B{\isacharunderscore}{\kern0pt}balls\ \isacommand{by}\isamarkupfalse%
\ {\isacharparenleft}{\kern0pt}blast{\isacharcomma}{\kern0pt}\ fast{\isacharparenright}{\kern0pt}\isanewline
\ \ \isacommand{{\isacharbraceright}{\kern0pt}}\isamarkupfalse%
\isanewline
\ \ \isacommand{note}\isamarkupfalse%
\ {\isacharasterisk}{\kern0pt}{\isacharasterisk}{\kern0pt}\ {\isacharequal}{\kern0pt}\ this\isanewline
\ \ \isacommand{hence}\isamarkupfalse%
\ {\isachardoublequoteopen}emeasure\ N\ {\isacharparenleft}{\kern0pt}f\ {\isacharminus}{\kern0pt}{\isacharbackquote}{\kern0pt}\ b\ i\ {\isasyminter}\ space\ N{\isacharparenright}{\kern0pt}\ {\isacharequal}{\kern0pt}\ {\isadigit{0}}{\isachardoublequoteclose}\ \isakeyword{for}\ i\ \isacommand{by}\isamarkupfalse%
\ {\isacharparenleft}{\kern0pt}cases\ {\isachardoublequoteopen}b\ i\ {\isacharequal}{\kern0pt}\ {\isacharbraceleft}{\kern0pt}{\isacharbraceright}{\kern0pt}{\isachardoublequoteclose}{\isacharcomma}{\kern0pt}\ simp{\isacharparenright}{\kern0pt}\ {\isacharparenleft}{\kern0pt}metis\ UnE\ singletonD\ \ {\isacharasterisk}{\kern0pt}\ range{\isacharunderscore}{\kern0pt}b{\isacharbrackleft}{\kern0pt}THEN\ eq{\isacharunderscore}{\kern0pt}refl{\isacharcomma}{\kern0pt}\ THEN\ range{\isacharunderscore}{\kern0pt}subsetD{\isacharbrackright}{\kern0pt}{\isacharparenright}{\kern0pt}\isanewline
\ \ \isacommand{hence}\isamarkupfalse%
\ {\isachardoublequoteopen}emeasure\ N\ {\isacharparenleft}{\kern0pt}{\isasymUnion}i{\isachardot}{\kern0pt}\ f\ {\isacharminus}{\kern0pt}{\isacharbackquote}{\kern0pt}\ b\ i\ {\isasyminter}\ space\ N{\isacharparenright}{\kern0pt}\ {\isacharequal}{\kern0pt}\ {\isadigit{0}}{\isachardoublequoteclose}\ \isacommand{using}\isamarkupfalse%
\ open{\isacharunderscore}{\kern0pt}b\ \isacommand{by}\isamarkupfalse%
\ {\isacharparenleft}{\kern0pt}intro\ emeasure{\isacharunderscore}{\kern0pt}UN{\isacharunderscore}{\kern0pt}eq{\isacharunderscore}{\kern0pt}{\isadigit{0}}{\isacharparenright}{\kern0pt}\ fastforce{\isacharplus}{\kern0pt}\isanewline
\ \ \isacommand{moreover}\isamarkupfalse%
\ \isacommand{have}\isamarkupfalse%
\ {\isachardoublequoteopen}{\isacharparenleft}{\kern0pt}{\isasymUnion}i{\isachardot}{\kern0pt}\ f\ {\isacharminus}{\kern0pt}{\isacharbackquote}{\kern0pt}\ b\ i\ {\isasyminter}\ space\ N{\isacharparenright}{\kern0pt}\ {\isacharequal}{\kern0pt}\ f\ {\isacharminus}{\kern0pt}{\isacharbackquote}{\kern0pt}\ {\isacharparenleft}{\kern0pt}{\isasymUnion}{\isacharparenleft}{\kern0pt}range\ b{\isacharparenright}{\kern0pt}{\isacharparenright}{\kern0pt}\ {\isasyminter}\ space\ N{\isachardoublequoteclose}\ \isacommand{by}\isamarkupfalse%
\ blast\isanewline
\ \ \isacommand{ultimately}\isamarkupfalse%
\ \isacommand{have}\isamarkupfalse%
\ {\isachardoublequoteopen}emeasure\ N\ {\isacharparenleft}{\kern0pt}f\ {\isacharminus}{\kern0pt}{\isacharbackquote}{\kern0pt}\ {\isacharparenleft}{\kern0pt}{\isacharminus}{\kern0pt}S{\isacharparenright}{\kern0pt}\ {\isasyminter}\ space\ N{\isacharparenright}{\kern0pt}\ {\isacharequal}{\kern0pt}\ {\isadigit{0}}{\isachardoublequoteclose}\ \isacommand{using}\isamarkupfalse%
\ Union{\isacharunderscore}{\kern0pt}range{\isacharunderscore}{\kern0pt}b\ \isacommand{by}\isamarkupfalse%
\ argo\isanewline
\ \ \isacommand{hence}\isamarkupfalse%
\ {\isachardoublequoteopen}AE\ x\ in\ N{\isachardot}{\kern0pt}\ f\ x\ {\isasymnotin}\ {\isacharminus}{\kern0pt}S{\isachardoublequoteclose}\ \isacommand{using}\isamarkupfalse%
\ open{\isacharunderscore}{\kern0pt}Compl{\isacharbrackleft}{\kern0pt}OF\ assms{\isacharparenleft}{\kern0pt}{\isadigit{2}}{\isacharparenright}{\kern0pt}{\isacharbrackright}{\kern0pt}\ \isacommand{by}\isamarkupfalse%
\ {\isacharparenleft}{\kern0pt}intro\ AE{\isacharunderscore}{\kern0pt}iff{\isacharunderscore}{\kern0pt}measurable{\isacharbrackleft}{\kern0pt}THEN\ iffD{\isadigit{2}}{\isacharbrackright}{\kern0pt}{\isacharcomma}{\kern0pt}\ auto{\isacharparenright}{\kern0pt}\isanewline
\ \ \isacommand{thus}\isamarkupfalse%
\ {\isacharquery}{\kern0pt}case\ \isacommand{by}\isamarkupfalse%
\ force\isanewline
\isacommand{qed}\isamarkupfalse%
\ {\isacharparenleft}{\kern0pt}simp\ add{\isacharcolon}{\kern0pt}\ pred{\isacharunderscore}{\kern0pt}sets{\isadigit{2}}{\isacharbrackleft}{\kern0pt}OF\ borel{\isacharunderscore}{\kern0pt}closed{\isacharbrackright}{\kern0pt}\ assms{\isacharparenleft}{\kern0pt}{\isadigit{2}}{\isacharparenright}{\kern0pt}{\isacharparenright}{\kern0pt}%
\endisatagproof
{\isafoldproof}%
%
\isadelimproof
\isanewline
%
\endisadelimproof
\ \ \isanewline
\isacommand{lemma}\isamarkupfalse%
\ density{\isacharunderscore}{\kern0pt}zero{\isacharcolon}{\kern0pt}\isanewline
\ \ \isakeyword{fixes}\ f{\isacharcolon}{\kern0pt}{\isacharcolon}{\kern0pt}{\isachardoublequoteopen}{\isacharprime}{\kern0pt}a\ {\isasymRightarrow}\ {\isacharprime}{\kern0pt}b{\isacharcolon}{\kern0pt}{\isacharcolon}{\kern0pt}{\isacharbraceleft}{\kern0pt}second{\isacharunderscore}{\kern0pt}countable{\isacharunderscore}{\kern0pt}topology{\isacharcomma}{\kern0pt}\ banach{\isacharbraceright}{\kern0pt}{\isachardoublequoteclose}\isanewline
\ \ \isakeyword{assumes}\ {\isachardoublequoteopen}integrable\ M\ f{\isachardoublequoteclose}\isanewline
\ \ \ \ \ \ \isakeyword{and}\ density{\isacharunderscore}{\kern0pt}{\isadigit{0}}{\isacharcolon}{\kern0pt}\ {\isachardoublequoteopen}{\isasymAnd}A{\isachardot}{\kern0pt}\ A\ {\isasymin}\ sets\ M\ {\isasymLongrightarrow}\ set{\isacharunderscore}{\kern0pt}lebesgue{\isacharunderscore}{\kern0pt}integral\ M\ A\ f\ {\isacharequal}{\kern0pt}\ {\isadigit{0}}{\isachardoublequoteclose}\isanewline
\ \ \isakeyword{shows}\ {\isachardoublequoteopen}AE\ x\ in\ M{\isachardot}{\kern0pt}\ f\ x\ {\isacharequal}{\kern0pt}\ {\isadigit{0}}{\isachardoublequoteclose}\isanewline
%
\isadelimproof
\ \ %
\endisadelimproof
%
\isatagproof
\isacommand{using}\isamarkupfalse%
\ averaging{\isacharunderscore}{\kern0pt}theorem{\isacharbrackleft}{\kern0pt}OF\ assms{\isacharparenleft}{\kern0pt}{\isadigit{1}}{\isacharparenright}{\kern0pt}{\isacharcomma}{\kern0pt}\ of\ {\isachardoublequoteopen}{\isacharbraceleft}{\kern0pt}{\isadigit{0}}{\isacharbraceright}{\kern0pt}{\isachardoublequoteclose}{\isacharbrackright}{\kern0pt}\ assms{\isacharparenleft}{\kern0pt}{\isadigit{2}}{\isacharparenright}{\kern0pt}\isanewline
\ \ \isacommand{by}\isamarkupfalse%
\ {\isacharparenleft}{\kern0pt}simp\ add{\isacharcolon}{\kern0pt}\ scaleR{\isacharunderscore}{\kern0pt}nonneg{\isacharunderscore}{\kern0pt}nonneg{\isacharparenright}{\kern0pt}%
\endisatagproof
{\isafoldproof}%
%
\isadelimproof
\isanewline
%
\endisadelimproof
\isanewline
\isacommand{lemma}\isamarkupfalse%
\ density{\isacharunderscore}{\kern0pt}unique{\isacharcolon}{\kern0pt}\isanewline
\ \ \isakeyword{fixes}\ f\ f{\isacharprime}{\kern0pt}{\isacharcolon}{\kern0pt}{\isacharcolon}{\kern0pt}{\isachardoublequoteopen}{\isacharprime}{\kern0pt}a\ {\isasymRightarrow}\ {\isacharprime}{\kern0pt}b{\isacharcolon}{\kern0pt}{\isacharcolon}{\kern0pt}{\isacharbraceleft}{\kern0pt}second{\isacharunderscore}{\kern0pt}countable{\isacharunderscore}{\kern0pt}topology{\isacharcomma}{\kern0pt}\ banach{\isacharbraceright}{\kern0pt}{\isachardoublequoteclose}\isanewline
\ \ \isakeyword{assumes}\ {\isachardoublequoteopen}integrable\ M\ f{\isachardoublequoteclose}\ {\isachardoublequoteopen}integrable\ M\ f{\isacharprime}{\kern0pt}{\isachardoublequoteclose}\isanewline
\ \ \ \ \ \ \isakeyword{and}\ density{\isacharunderscore}{\kern0pt}eq{\isacharcolon}{\kern0pt}\ {\isachardoublequoteopen}{\isasymAnd}A{\isachardot}{\kern0pt}\ A\ {\isasymin}\ sets\ M\ {\isasymLongrightarrow}\ set{\isacharunderscore}{\kern0pt}lebesgue{\isacharunderscore}{\kern0pt}integral\ M\ A\ f\ {\isacharequal}{\kern0pt}\ set{\isacharunderscore}{\kern0pt}lebesgue{\isacharunderscore}{\kern0pt}integral\ M\ A\ f{\isacharprime}{\kern0pt}{\isachardoublequoteclose}\isanewline
\ \ \isakeyword{shows}\ {\isachardoublequoteopen}AE\ x\ in\ M{\isachardot}{\kern0pt}\ f\ x\ {\isacharequal}{\kern0pt}\ f{\isacharprime}{\kern0pt}\ x{\isachardoublequoteclose}\isanewline
%
\isadelimproof
%
\endisadelimproof
%
\isatagproof
\isacommand{proof}\isamarkupfalse%
{\isacharminus}{\kern0pt}\isanewline
\ \ \isacommand{{\isacharbraceleft}{\kern0pt}}\isamarkupfalse%
\ \isanewline
\ \ \ \ \isacommand{fix}\isamarkupfalse%
\ A\ \isacommand{assume}\isamarkupfalse%
\ asm{\isacharcolon}{\kern0pt}\ {\isachardoublequoteopen}A\ {\isasymin}\ sets\ M{\isachardoublequoteclose}\isanewline
\ \ \ \ \isacommand{hence}\isamarkupfalse%
\ {\isachardoublequoteopen}LINT\ x{\isacharbar}{\kern0pt}M{\isachardot}{\kern0pt}\ indicat{\isacharunderscore}{\kern0pt}real\ A\ x\ {\isacharasterisk}{\kern0pt}\isactrlsub R\ {\isacharparenleft}{\kern0pt}f\ x\ {\isacharminus}{\kern0pt}\ f{\isacharprime}{\kern0pt}\ x{\isacharparenright}{\kern0pt}\ {\isacharequal}{\kern0pt}\ {\isadigit{0}}{\isachardoublequoteclose}\ \isacommand{using}\isamarkupfalse%
\ density{\isacharunderscore}{\kern0pt}eq\ assms{\isacharparenleft}{\kern0pt}{\isadigit{1}}{\isacharcomma}{\kern0pt}{\isadigit{2}}{\isacharparenright}{\kern0pt}\ \isacommand{by}\isamarkupfalse%
\ {\isacharparenleft}{\kern0pt}simp\ add{\isacharcolon}{\kern0pt}\ set{\isacharunderscore}{\kern0pt}lebesgue{\isacharunderscore}{\kern0pt}integral{\isacharunderscore}{\kern0pt}def\ algebra{\isacharunderscore}{\kern0pt}simps\ Bochner{\isacharunderscore}{\kern0pt}Integration{\isachardot}{\kern0pt}integral{\isacharunderscore}{\kern0pt}diff{\isacharbrackleft}{\kern0pt}OF\ integrable{\isacharunderscore}{\kern0pt}mult{\isacharunderscore}{\kern0pt}indicator{\isacharparenleft}{\kern0pt}{\isadigit{1}}{\isacharcomma}{\kern0pt}{\isadigit{1}}{\isacharparenright}{\kern0pt}{\isacharbrackright}{\kern0pt}{\isacharparenright}{\kern0pt}\isanewline
\ \ \isacommand{{\isacharbraceright}{\kern0pt}}\isamarkupfalse%
\isanewline
\ \ \isacommand{thus}\isamarkupfalse%
\ {\isacharquery}{\kern0pt}thesis\ \isacommand{using}\isamarkupfalse%
\ density{\isacharunderscore}{\kern0pt}zero{\isacharbrackleft}{\kern0pt}OF\ Bochner{\isacharunderscore}{\kern0pt}Integration{\isachardot}{\kern0pt}integrable{\isacharunderscore}{\kern0pt}diff{\isacharbrackleft}{\kern0pt}OF\ assms{\isacharparenleft}{\kern0pt}{\isadigit{1}}{\isacharcomma}{\kern0pt}{\isadigit{2}}{\isacharparenright}{\kern0pt}{\isacharbrackright}{\kern0pt}{\isacharbrackright}{\kern0pt}\ \isacommand{by}\isamarkupfalse%
\ {\isacharparenleft}{\kern0pt}simp\ add{\isacharcolon}{\kern0pt}\ set{\isacharunderscore}{\kern0pt}lebesgue{\isacharunderscore}{\kern0pt}integral{\isacharunderscore}{\kern0pt}def{\isacharparenright}{\kern0pt}\isanewline
\isacommand{qed}\isamarkupfalse%
%
\endisatagproof
{\isafoldproof}%
%
\isadelimproof
\isanewline
%
\endisadelimproof
\isanewline
\isacommand{lemma}\isamarkupfalse%
\ density{\isacharunderscore}{\kern0pt}nonneg{\isacharcolon}{\kern0pt}\isanewline
\ \ \isakeyword{fixes}\ f{\isacharcolon}{\kern0pt}{\isacharcolon}{\kern0pt}{\isachardoublequoteopen}{\isacharunderscore}{\kern0pt}\ {\isasymRightarrow}\ {\isacharprime}{\kern0pt}b{\isacharcolon}{\kern0pt}{\isacharcolon}{\kern0pt}{\isacharbraceleft}{\kern0pt}second{\isacharunderscore}{\kern0pt}countable{\isacharunderscore}{\kern0pt}topology{\isacharcomma}{\kern0pt}\ banach{\isacharcomma}{\kern0pt}\ linorder{\isacharunderscore}{\kern0pt}topology{\isacharcomma}{\kern0pt}\ ordered{\isacharunderscore}{\kern0pt}real{\isacharunderscore}{\kern0pt}vector{\isacharbraceright}{\kern0pt}{\isachardoublequoteclose}\isanewline
\ \ \isakeyword{assumes}\ {\isachardoublequoteopen}integrable\ M\ f{\isachardoublequoteclose}\ \isanewline
\ \ \ \ \ \ \isakeyword{and}\ {\isachardoublequoteopen}{\isasymAnd}A{\isachardot}{\kern0pt}\ A\ {\isasymin}\ sets\ M\ {\isasymLongrightarrow}\ set{\isacharunderscore}{\kern0pt}lebesgue{\isacharunderscore}{\kern0pt}integral\ M\ A\ f\ {\isasymge}\ {\isadigit{0}}{\isachardoublequoteclose}\isanewline
\ \ \ \ \isakeyword{shows}\ {\isachardoublequoteopen}AE\ x\ in\ M{\isachardot}{\kern0pt}\ f\ x\ {\isasymge}\ {\isadigit{0}}{\isachardoublequoteclose}\isanewline
%
\isadelimproof
\ \ %
\endisadelimproof
%
\isatagproof
\isacommand{using}\isamarkupfalse%
\ averaging{\isacharunderscore}{\kern0pt}theorem{\isacharbrackleft}{\kern0pt}OF\ assms{\isacharparenleft}{\kern0pt}{\isadigit{1}}{\isacharparenright}{\kern0pt}{\isacharcomma}{\kern0pt}\ of\ {\isachardoublequoteopen}{\isacharbraceleft}{\kern0pt}{\isadigit{0}}{\isachardot}{\kern0pt}{\isachardot}{\kern0pt}{\isacharbraceright}{\kern0pt}{\isachardoublequoteclose}{\isacharcomma}{\kern0pt}\ OF\ closed{\isacharunderscore}{\kern0pt}atLeast{\isacharbrackright}{\kern0pt}\ assms{\isacharparenleft}{\kern0pt}{\isadigit{2}}{\isacharparenright}{\kern0pt}\isanewline
\ \ \isacommand{by}\isamarkupfalse%
\ {\isacharparenleft}{\kern0pt}simp\ add{\isacharcolon}{\kern0pt}\ scaleR{\isacharunderscore}{\kern0pt}nonneg{\isacharunderscore}{\kern0pt}nonneg{\isacharparenright}{\kern0pt}%
\endisatagproof
{\isafoldproof}%
%
\isadelimproof
\isanewline
%
\endisadelimproof
\isanewline
\isacommand{corollary}\isamarkupfalse%
\ integral{\isacharunderscore}{\kern0pt}nonneg{\isacharunderscore}{\kern0pt}AE{\isacharunderscore}{\kern0pt}eq{\isacharunderscore}{\kern0pt}{\isadigit{0}}{\isacharunderscore}{\kern0pt}iff{\isacharunderscore}{\kern0pt}AE{\isacharcolon}{\kern0pt}\isanewline
\ \ \isakeyword{fixes}\ f\ {\isacharcolon}{\kern0pt}{\isacharcolon}{\kern0pt}\ {\isachardoublequoteopen}{\isacharprime}{\kern0pt}a\ {\isasymRightarrow}\ {\isacharprime}{\kern0pt}b\ {\isacharcolon}{\kern0pt}{\isacharcolon}{\kern0pt}\ {\isacharbraceleft}{\kern0pt}second{\isacharunderscore}{\kern0pt}countable{\isacharunderscore}{\kern0pt}topology{\isacharcomma}{\kern0pt}\ banach{\isacharcomma}{\kern0pt}\ linorder{\isacharunderscore}{\kern0pt}topology{\isacharcomma}{\kern0pt}\ ordered{\isacharunderscore}{\kern0pt}real{\isacharunderscore}{\kern0pt}vector{\isacharbraceright}{\kern0pt}{\isachardoublequoteclose}\isanewline
\ \ \isakeyword{assumes}\ f{\isacharbrackleft}{\kern0pt}measurable{\isacharbrackright}{\kern0pt}{\isacharcolon}{\kern0pt}\ {\isachardoublequoteopen}integrable\ M\ f{\isachardoublequoteclose}\ \isakeyword{and}\ nonneg{\isacharcolon}{\kern0pt}\ {\isachardoublequoteopen}AE\ x\ in\ M{\isachardot}{\kern0pt}\ {\isadigit{0}}\ {\isasymle}\ f\ x{\isachardoublequoteclose}\isanewline
\ \ \isakeyword{shows}\ {\isachardoublequoteopen}integral\isactrlsup L\ M\ f\ {\isacharequal}{\kern0pt}\ {\isadigit{0}}\ {\isasymlongleftrightarrow}\ {\isacharparenleft}{\kern0pt}AE\ x\ in\ M{\isachardot}{\kern0pt}\ f\ x\ {\isacharequal}{\kern0pt}\ {\isadigit{0}}{\isacharparenright}{\kern0pt}{\isachardoublequoteclose}\isanewline
%
\isadelimproof
%
\endisadelimproof
%
\isatagproof
\isacommand{proof}\isamarkupfalse%
\ \isanewline
\ \ \isacommand{assume}\isamarkupfalse%
\ {\isacharasterisk}{\kern0pt}{\isacharcolon}{\kern0pt}\ {\isachardoublequoteopen}integral\isactrlsup L\ M\ f\ {\isacharequal}{\kern0pt}\ {\isadigit{0}}{\isachardoublequoteclose}\isanewline
\ \ \isacommand{{\isacharbraceleft}{\kern0pt}}\isamarkupfalse%
\isanewline
\ \ \ \ \isacommand{fix}\isamarkupfalse%
\ A\ \isacommand{assume}\isamarkupfalse%
\ asm{\isacharcolon}{\kern0pt}\ {\isachardoublequoteopen}A\ {\isasymin}\ sets\ M{\isachardoublequoteclose}\isanewline
\ \ \ \ \isacommand{have}\isamarkupfalse%
\ {\isachardoublequoteopen}{\isadigit{0}}\ {\isasymle}\ integral\isactrlsup L\ M\ {\isacharparenleft}{\kern0pt}{\isasymlambda}x{\isachardot}{\kern0pt}\ indicator\ A\ x\ {\isacharasterisk}{\kern0pt}\isactrlsub R\ f\ x{\isacharparenright}{\kern0pt}{\isachardoublequoteclose}\ \isacommand{using}\isamarkupfalse%
\ nonneg\ \isacommand{by}\isamarkupfalse%
\ {\isacharparenleft}{\kern0pt}subst\ integral{\isacharunderscore}{\kern0pt}zero{\isacharbrackleft}{\kern0pt}of\ M{\isacharcomma}{\kern0pt}\ symmetric{\isacharbrackright}{\kern0pt}{\isacharcomma}{\kern0pt}\ intro\ integral{\isacharunderscore}{\kern0pt}mono{\isacharunderscore}{\kern0pt}AE{\isacharunderscore}{\kern0pt}banach\ integrable{\isacharunderscore}{\kern0pt}mult{\isacharunderscore}{\kern0pt}indicator\ asm\ f\ integrable{\isacharunderscore}{\kern0pt}zero{\isacharcomma}{\kern0pt}\ auto\ simp\ add{\isacharcolon}{\kern0pt}\ indicator{\isacharunderscore}{\kern0pt}def{\isacharparenright}{\kern0pt}\isanewline
\ \ \ \ \isacommand{moreover}\isamarkupfalse%
\ \isacommand{have}\isamarkupfalse%
\ {\isachardoublequoteopen}{\isachardot}{\kern0pt}{\isachardot}{\kern0pt}{\isachardot}{\kern0pt}\ {\isasymle}\ integral\isactrlsup L\ M\ f{\isachardoublequoteclose}\ \isacommand{using}\isamarkupfalse%
\ nonneg\ \isacommand{by}\isamarkupfalse%
\ {\isacharparenleft}{\kern0pt}intro\ integral{\isacharunderscore}{\kern0pt}mono{\isacharunderscore}{\kern0pt}AE{\isacharunderscore}{\kern0pt}banach\ integrable{\isacharunderscore}{\kern0pt}mult{\isacharunderscore}{\kern0pt}indicator\ asm\ f{\isacharcomma}{\kern0pt}\ auto\ simp\ add{\isacharcolon}{\kern0pt}\ indicator{\isacharunderscore}{\kern0pt}def{\isacharparenright}{\kern0pt}\isanewline
\ \ \ \ \isacommand{ultimately}\isamarkupfalse%
\ \isacommand{have}\isamarkupfalse%
\ {\isachardoublequoteopen}set{\isacharunderscore}{\kern0pt}lebesgue{\isacharunderscore}{\kern0pt}integral\ M\ A\ f\ {\isacharequal}{\kern0pt}\ {\isadigit{0}}{\isachardoublequoteclose}\ \isacommand{unfolding}\isamarkupfalse%
\ set{\isacharunderscore}{\kern0pt}lebesgue{\isacharunderscore}{\kern0pt}integral{\isacharunderscore}{\kern0pt}def\ \isacommand{using}\isamarkupfalse%
\ {\isacharasterisk}{\kern0pt}\ \isacommand{by}\isamarkupfalse%
\ force\isanewline
\ \ \isacommand{{\isacharbraceright}{\kern0pt}}\isamarkupfalse%
\isanewline
\ \ \isacommand{thus}\isamarkupfalse%
\ {\isachardoublequoteopen}AE\ x\ in\ M{\isachardot}{\kern0pt}\ f\ x\ {\isacharequal}{\kern0pt}\ {\isadigit{0}}{\isachardoublequoteclose}\ \isacommand{by}\isamarkupfalse%
\ {\isacharparenleft}{\kern0pt}intro\ density{\isacharunderscore}{\kern0pt}zero\ f{\isacharcomma}{\kern0pt}\ blast{\isacharparenright}{\kern0pt}\isanewline
\isacommand{qed}\isamarkupfalse%
\ {\isacharparenleft}{\kern0pt}auto\ simp\ add{\isacharcolon}{\kern0pt}\ integral{\isacharunderscore}{\kern0pt}eq{\isacharunderscore}{\kern0pt}zero{\isacharunderscore}{\kern0pt}AE{\isacharparenright}{\kern0pt}%
\endisatagproof
{\isafoldproof}%
%
\isadelimproof
\isanewline
%
\endisadelimproof
\isanewline
\isacommand{corollary}\isamarkupfalse%
\ integral{\isacharunderscore}{\kern0pt}eq{\isacharunderscore}{\kern0pt}mono{\isacharunderscore}{\kern0pt}AE{\isacharunderscore}{\kern0pt}eq{\isacharunderscore}{\kern0pt}AE{\isacharcolon}{\kern0pt}\isanewline
\ \ \isakeyword{fixes}\ f\ g\ {\isacharcolon}{\kern0pt}{\isacharcolon}{\kern0pt}\ {\isachardoublequoteopen}{\isacharprime}{\kern0pt}a\ {\isasymRightarrow}\ {\isacharprime}{\kern0pt}b\ {\isacharcolon}{\kern0pt}{\isacharcolon}{\kern0pt}\ {\isacharbraceleft}{\kern0pt}second{\isacharunderscore}{\kern0pt}countable{\isacharunderscore}{\kern0pt}topology{\isacharcomma}{\kern0pt}\ banach{\isacharcomma}{\kern0pt}\ linorder{\isacharunderscore}{\kern0pt}topology{\isacharcomma}{\kern0pt}\ ordered{\isacharunderscore}{\kern0pt}real{\isacharunderscore}{\kern0pt}vector{\isacharbraceright}{\kern0pt}{\isachardoublequoteclose}\isanewline
\ \ \isakeyword{assumes}\ {\isachardoublequoteopen}integrable\ M\ f{\isachardoublequoteclose}\ {\isachardoublequoteopen}integrable\ M\ g{\isachardoublequoteclose}\ {\isachardoublequoteopen}integral\isactrlsup L\ M\ f\ {\isacharequal}{\kern0pt}\ integral\isactrlsup L\ M\ g{\isachardoublequoteclose}\ {\isachardoublequoteopen}AE\ x\ in\ M{\isachardot}{\kern0pt}\ f\ x\ {\isasymle}\ g\ x{\isachardoublequoteclose}\ \isanewline
\ \ \isakeyword{shows}\ {\isachardoublequoteopen}AE\ x\ in\ M{\isachardot}{\kern0pt}\ f\ x\ {\isacharequal}{\kern0pt}\ g\ x{\isachardoublequoteclose}\isanewline
%
\isadelimproof
%
\endisadelimproof
%
\isatagproof
\isacommand{proof}\isamarkupfalse%
\ {\isacharminus}{\kern0pt}\isanewline
\ \ \isacommand{define}\isamarkupfalse%
\ h\ \isakeyword{where}\ {\isachardoublequoteopen}h\ {\isacharequal}{\kern0pt}\ {\isacharparenleft}{\kern0pt}{\isasymlambda}x{\isachardot}{\kern0pt}\ g\ x\ {\isacharminus}{\kern0pt}\ f\ x{\isacharparenright}{\kern0pt}{\isachardoublequoteclose}\isanewline
\ \ \isacommand{have}\isamarkupfalse%
\ {\isachardoublequoteopen}AE\ x\ in\ M{\isachardot}{\kern0pt}\ h\ x\ {\isacharequal}{\kern0pt}\ {\isadigit{0}}{\isachardoublequoteclose}\ \isacommand{unfolding}\isamarkupfalse%
\ h{\isacharunderscore}{\kern0pt}def\ \isacommand{using}\isamarkupfalse%
\ assms\ \isacommand{by}\isamarkupfalse%
\ {\isacharparenleft}{\kern0pt}subst\ integral{\isacharunderscore}{\kern0pt}nonneg{\isacharunderscore}{\kern0pt}AE{\isacharunderscore}{\kern0pt}eq{\isacharunderscore}{\kern0pt}{\isadigit{0}}{\isacharunderscore}{\kern0pt}iff{\isacharunderscore}{\kern0pt}AE{\isacharbrackleft}{\kern0pt}symmetric{\isacharbrackright}{\kern0pt}{\isacharparenright}{\kern0pt}\ auto\isanewline
\ \ \isacommand{then}\isamarkupfalse%
\ \isacommand{show}\isamarkupfalse%
\ {\isacharquery}{\kern0pt}thesis\ \isacommand{unfolding}\isamarkupfalse%
\ h{\isacharunderscore}{\kern0pt}def\ \isacommand{by}\isamarkupfalse%
\ auto\isanewline
\isacommand{qed}\isamarkupfalse%
%
\endisatagproof
{\isafoldproof}%
%
\isadelimproof
\isanewline
%
\endisadelimproof
\isanewline
\isacommand{end}\isamarkupfalse%
\isanewline
%
\isadelimtheory
\isanewline
%
\endisadelimtheory
%
\isatagtheory
\isacommand{end}\isamarkupfalse%
%
\endisatagtheory
{\isafoldtheory}%
%
\isadelimtheory
%
\endisadelimtheory
%
\end{isabellebody}%
\endinput
%:%file=Sigma_Finite_Measure_Addendum.tex%:%
%:%10=1%:%
%:%11=1%:%
%:%12=2%:%
%:%13=3%:%
%:%27=5%:%
%:%37=8%:%
%:%38=8%:%
%:%39=9%:%
%:%40=10%:%
%:%41=11%:%
%:%42=12%:%
%:%49=13%:%
%:%50=13%:%
%:%51=14%:%
%:%52=14%:%
%:%53=14%:%
%:%54=14%:%
%:%55=15%:%
%:%56=15%:%
%:%57=16%:%
%:%58=16%:%
%:%59=17%:%
%:%60=17%:%
%:%61=17%:%
%:%62=18%:%
%:%63=18%:%
%:%64=18%:%
%:%65=18%:%
%:%66=19%:%
%:%67=19%:%
%:%68=19%:%
%:%69=19%:%
%:%70=20%:%
%:%71=20%:%
%:%72=20%:%
%:%73=20%:%
%:%74=20%:%
%:%75=21%:%
%:%76=22%:%
%:%77=22%:%
%:%78=22%:%
%:%79=22%:%
%:%80=23%:%
%:%81=23%:%
%:%82=23%:%
%:%83=23%:%
%:%84=24%:%
%:%85=24%:%
%:%86=24%:%
%:%87=24%:%
%:%88=24%:%
%:%89=25%:%
%:%90=25%:%
%:%91=25%:%
%:%92=25%:%
%:%93=25%:%
%:%94=26%:%
%:%95=26%:%
%:%96=27%:%
%:%97=27%:%
%:%98=27%:%
%:%99=27%:%
%:%100=28%:%
%:%106=28%:%
%:%109=29%:%
%:%110=30%:%
%:%111=30%:%
%:%112=31%:%
%:%113=32%:%
%:%114=33%:%
%:%115=33%:%
%:%116=34%:%
%:%121=39%:%
%:%122=40%:%
%:%123=41%:%
%:%130=42%:%
%:%131=42%:%
%:%132=43%:%
%:%133=43%:%
%:%134=43%:%
%:%135=43%:%
%:%136=44%:%
%:%137=45%:%
%:%138=45%:%
%:%139=45%:%
%:%140=45%:%
%:%141=46%:%
%:%142=46%:%
%:%143=47%:%
%:%144=47%:%
%:%145=47%:%
%:%146=47%:%
%:%147=48%:%
%:%148=48%:%
%:%149=49%:%
%:%150=49%:%
%:%151=50%:%
%:%152=50%:%
%:%153=50%:%
%:%154=51%:%
%:%155=51%:%
%:%156=51%:%
%:%157=51%:%
%:%158=52%:%
%:%159=52%:%
%:%160=53%:%
%:%161=53%:%
%:%162=54%:%
%:%163=54%:%
%:%164=55%:%
%:%165=55%:%
%:%166=56%:%
%:%167=56%:%
%:%168=56%:%
%:%169=56%:%
%:%170=57%:%
%:%171=57%:%
%:%172=57%:%
%:%173=57%:%
%:%174=58%:%
%:%175=58%:%
%:%176=58%:%
%:%177=59%:%
%:%178=59%:%
%:%179=60%:%
%:%180=60%:%
%:%181=60%:%
%:%182=61%:%
%:%183=61%:%
%:%184=61%:%
%:%185=61%:%
%:%186=61%:%
%:%187=62%:%
%:%188=62%:%
%:%189=62%:%
%:%190=62%:%
%:%191=63%:%
%:%197=63%:%
%:%200=64%:%
%:%201=65%:%
%:%202=66%:%
%:%203=66%:%
%:%204=67%:%
%:%205=68%:%
%:%206=69%:%
%:%207=70%:%
%:%208=71%:%
%:%215=72%:%
%:%216=72%:%
%:%217=73%:%
%:%218=73%:%
%:%219=74%:%
%:%220=75%:%
%:%221=75%:%
%:%222=75%:%
%:%223=76%:%
%:%224=77%:%
%:%225=77%:%
%:%226=77%:%
%:%227=77%:%
%:%228=78%:%
%:%229=78%:%
%:%230=79%:%
%:%231=79%:%
%:%232=80%:%
%:%233=80%:%
%:%234=80%:%
%:%235=80%:%
%:%236=81%:%
%:%237=81%:%
%:%238=81%:%
%:%239=82%:%
%:%240=82%:%
%:%241=82%:%
%:%242=82%:%
%:%243=83%:%
%:%244=83%:%
%:%245=83%:%
%:%246=83%:%
%:%247=83%:%
%:%248=84%:%
%:%249=84%:%
%:%250=84%:%
%:%251=84%:%
%:%252=84%:%
%:%253=85%:%
%:%254=85%:%
%:%255=86%:%
%:%256=87%:%
%:%257=87%:%
%:%258=87%:%
%:%259=87%:%
%:%260=88%:%
%:%261=88%:%
%:%262=88%:%
%:%263=88%:%
%:%264=89%:%
%:%265=90%:%
%:%266=90%:%
%:%267=90%:%
%:%268=90%:%
%:%269=91%:%
%:%270=91%:%
%:%271=91%:%
%:%272=91%:%
%:%273=92%:%
%:%274=93%:%
%:%275=93%:%
%:%276=94%:%
%:%277=94%:%
%:%278=94%:%
%:%279=95%:%
%:%280=95%:%
%:%281=95%:%
%:%282=95%:%
%:%283=96%:%
%:%284=96%:%
%:%285=96%:%
%:%286=96%:%
%:%287=96%:%
%:%288=97%:%
%:%289=97%:%
%:%290=97%:%
%:%291=97%:%
%:%292=98%:%
%:%293=99%:%
%:%294=99%:%
%:%295=100%:%
%:%296=100%:%
%:%297=100%:%
%:%298=101%:%
%:%299=101%:%
%:%300=102%:%
%:%301=102%:%
%:%302=102%:%
%:%303=102%:%
%:%304=102%:%
%:%305=103%:%
%:%306=103%:%
%:%307=103%:%
%:%308=104%:%
%:%309=104%:%
%:%310=104%:%
%:%311=104%:%
%:%312=105%:%
%:%313=105%:%
%:%314=106%:%
%:%315=106%:%
%:%316=107%:%
%:%317=107%:%
%:%318=108%:%
%:%319=108%:%
%:%320=109%:%
%:%321=109%:%
%:%322=109%:%
%:%323=109%:%
%:%324=110%:%
%:%325=110%:%
%:%326=110%:%
%:%327=110%:%
%:%328=111%:%
%:%329=111%:%
%:%330=111%:%
%:%331=111%:%
%:%332=111%:%
%:%333=112%:%
%:%334=112%:%
%:%335=112%:%
%:%336=112%:%
%:%337=112%:%
%:%338=113%:%
%:%339=113%:%
%:%340=113%:%
%:%341=114%:%
%:%342=114%:%
%:%343=115%:%
%:%344=115%:%
%:%345=116%:%
%:%346=116%:%
%:%347=117%:%
%:%348=118%:%
%:%349=118%:%
%:%350=118%:%
%:%351=118%:%
%:%352=118%:%
%:%353=119%:%
%:%354=119%:%
%:%355=119%:%
%:%356=119%:%
%:%357=120%:%
%:%358=120%:%
%:%359=120%:%
%:%360=120%:%
%:%361=121%:%
%:%362=121%:%
%:%363=122%:%
%:%364=122%:%
%:%365=122%:%
%:%366=123%:%
%:%367=123%:%
%:%368=124%:%
%:%369=124%:%
%:%370=125%:%
%:%371=125%:%
%:%372=126%:%
%:%373=126%:%
%:%374=127%:%
%:%375=127%:%
%:%376=127%:%
%:%377=128%:%
%:%378=128%:%
%:%379=128%:%
%:%380=128%:%
%:%381=129%:%
%:%382=129%:%
%:%383=130%:%
%:%384=130%:%
%:%385=131%:%
%:%386=131%:%
%:%387=131%:%
%:%388=132%:%
%:%389=132%:%
%:%390=132%:%
%:%391=132%:%
%:%392=133%:%
%:%393=133%:%
%:%394=133%:%
%:%395=133%:%
%:%396=134%:%
%:%397=134%:%
%:%398=134%:%
%:%399=134%:%
%:%400=134%:%
%:%401=135%:%
%:%402=135%:%
%:%403=135%:%
%:%404=135%:%
%:%405=136%:%
%:%406=136%:%
%:%407=136%:%
%:%408=137%:%
%:%409=137%:%
%:%414=137%:%
%:%417=138%:%
%:%418=139%:%
%:%419=139%:%
%:%420=140%:%
%:%421=141%:%
%:%422=142%:%
%:%423=143%:%
%:%426=144%:%
%:%430=144%:%
%:%431=144%:%
%:%432=145%:%
%:%433=145%:%
%:%438=145%:%
%:%441=146%:%
%:%442=147%:%
%:%443=147%:%
%:%444=148%:%
%:%445=149%:%
%:%446=150%:%
%:%447=151%:%
%:%454=152%:%
%:%455=152%:%
%:%456=153%:%
%:%457=153%:%
%:%458=154%:%
%:%459=154%:%
%:%460=154%:%
%:%461=155%:%
%:%462=155%:%
%:%463=155%:%
%:%464=155%:%
%:%465=156%:%
%:%466=156%:%
%:%467=157%:%
%:%468=157%:%
%:%469=157%:%
%:%470=157%:%
%:%471=158%:%
%:%477=158%:%
%:%480=159%:%
%:%481=160%:%
%:%482=160%:%
%:%483=161%:%
%:%484=162%:%
%:%485=163%:%
%:%486=164%:%
%:%489=165%:%
%:%493=165%:%
%:%494=165%:%
%:%495=166%:%
%:%496=166%:%
%:%501=166%:%
%:%504=167%:%
%:%505=168%:%
%:%506=168%:%
%:%507=169%:%
%:%508=170%:%
%:%509=171%:%
%:%516=172%:%
%:%517=172%:%
%:%518=173%:%
%:%519=173%:%
%:%520=174%:%
%:%521=174%:%
%:%522=175%:%
%:%523=175%:%
%:%524=175%:%
%:%525=176%:%
%:%526=176%:%
%:%527=176%:%
%:%528=176%:%
%:%529=177%:%
%:%530=177%:%
%:%531=177%:%
%:%532=177%:%
%:%533=177%:%
%:%534=178%:%
%:%535=178%:%
%:%536=178%:%
%:%537=178%:%
%:%538=178%:%
%:%539=178%:%
%:%540=179%:%
%:%541=179%:%
%:%542=180%:%
%:%543=180%:%
%:%544=180%:%
%:%545=181%:%
%:%546=181%:%
%:%551=181%:%
%:%554=182%:%
%:%555=183%:%
%:%556=183%:%
%:%557=184%:%
%:%558=185%:%
%:%559=186%:%
%:%566=187%:%
%:%567=187%:%
%:%568=188%:%
%:%569=188%:%
%:%570=189%:%
%:%571=189%:%
%:%572=189%:%
%:%573=189%:%
%:%574=189%:%
%:%575=190%:%
%:%576=190%:%
%:%577=190%:%
%:%578=190%:%
%:%579=190%:%
%:%580=191%:%
%:%586=191%:%
%:%589=192%:%
%:%590=193%:%
%:%591=193%:%
%:%594=194%:%
%:%599=195%:%

%
\begin{isabellebody}%
\setisabellecontext{Filtration}%
%
\isadelimtheory
%
\endisadelimtheory
%
\isatagtheory
\isacommand{theory}\isamarkupfalse%
\ Filtration\isanewline
\isakeyword{imports}\ {\isachardoublequoteopen}HOL{\isacharminus}{\kern0pt}Probability{\isachardot}{\kern0pt}Conditional{\isacharunderscore}{\kern0pt}Expectation{\isachardoublequoteclose}\ {\isachardoublequoteopen}HOL{\isacharminus}{\kern0pt}Probability{\isachardot}{\kern0pt}Stopping{\isacharunderscore}{\kern0pt}Time{\isachardoublequoteclose}\ Measure{\isacharunderscore}{\kern0pt}Space{\isacharunderscore}{\kern0pt}Addendum\isanewline
\isakeyword{begin}%
\endisatagtheory
{\isafoldtheory}%
%
\isadelimtheory
%
\endisadelimtheory
%
\isadelimdocument
%
\endisadelimdocument
%
\isatagdocument
%
\isamarkupsubsection{Filtered Sigma Finite Measure%
}
\isamarkuptrue%
%
\endisatagdocument
{\isafolddocument}%
%
\isadelimdocument
%
\endisadelimdocument
\isacommand{locale}\isamarkupfalse%
\ filtered{\isacharunderscore}{\kern0pt}sigma{\isacharunderscore}{\kern0pt}finite{\isacharunderscore}{\kern0pt}measure\ {\isacharequal}{\kern0pt}\ sigma{\isacharunderscore}{\kern0pt}finite{\isacharunderscore}{\kern0pt}measure\ M\ {\isacharplus}{\kern0pt}\ filtration\ {\isachardoublequoteopen}space\ M{\isachardoublequoteclose}\ F\ \isakeyword{for}\ M\ \isakeyword{and}\ F\ {\isacharcolon}{\kern0pt}{\isacharcolon}{\kern0pt}\ {\isachardoublequoteopen}{\isacharprime}{\kern0pt}t\ {\isacharcolon}{\kern0pt}{\isacharcolon}{\kern0pt}\ {\isacharbraceleft}{\kern0pt}second{\isacharunderscore}{\kern0pt}countable{\isacharunderscore}{\kern0pt}topology{\isacharcomma}{\kern0pt}\ linorder{\isacharunderscore}{\kern0pt}topology{\isacharcomma}{\kern0pt}\ order{\isacharunderscore}{\kern0pt}bot{\isacharbraceright}{\kern0pt}\ {\isasymRightarrow}\ {\isacharprime}{\kern0pt}a\ measure{\isachardoublequoteclose}\ {\isacharplus}{\kern0pt}\isanewline
\ \ \isakeyword{assumes}\ subalgebra{\isacharcolon}{\kern0pt}\ {\isachardoublequoteopen}{\isasymAnd}i{\isachardot}{\kern0pt}\ subalgebra\ M\ {\isacharparenleft}{\kern0pt}F\ i{\isacharparenright}{\kern0pt}{\isachardoublequoteclose}\isanewline
\ \ \ \ \ \ \isakeyword{and}\ sigma{\isacharunderscore}{\kern0pt}finite{\isacharcolon}{\kern0pt}\ {\isachardoublequoteopen}sigma{\isacharunderscore}{\kern0pt}finite{\isacharunderscore}{\kern0pt}measure\ {\isacharparenleft}{\kern0pt}restr{\isacharunderscore}{\kern0pt}to{\isacharunderscore}{\kern0pt}subalg\ M\ {\isacharparenleft}{\kern0pt}F\ bot{\isacharparenright}{\kern0pt}{\isacharparenright}{\kern0pt}{\isachardoublequoteclose}\isanewline
\isanewline
\isacommand{locale}\isamarkupfalse%
\ ennreal{\isacharunderscore}{\kern0pt}filtered{\isacharunderscore}{\kern0pt}sigma{\isacharunderscore}{\kern0pt}finite{\isacharunderscore}{\kern0pt}measure\ {\isacharequal}{\kern0pt}\ filtered{\isacharunderscore}{\kern0pt}sigma{\isacharunderscore}{\kern0pt}finite{\isacharunderscore}{\kern0pt}measure\ M\ F\ \isakeyword{for}\ M\ \isakeyword{and}\ F\ {\isacharcolon}{\kern0pt}{\isacharcolon}{\kern0pt}\ {\isachardoublequoteopen}ennreal\ {\isasymRightarrow}\ {\isacharunderscore}{\kern0pt}{\isachardoublequoteclose}\isanewline
\isacommand{locale}\isamarkupfalse%
\ nat{\isacharunderscore}{\kern0pt}filtered{\isacharunderscore}{\kern0pt}sigma{\isacharunderscore}{\kern0pt}finite{\isacharunderscore}{\kern0pt}measure\ {\isacharequal}{\kern0pt}\ filtered{\isacharunderscore}{\kern0pt}sigma{\isacharunderscore}{\kern0pt}finite{\isacharunderscore}{\kern0pt}measure\ M\ F\ \isakeyword{for}\ M\ \isakeyword{and}\ F\ {\isacharcolon}{\kern0pt}{\isacharcolon}{\kern0pt}\ {\isachardoublequoteopen}nat\ {\isasymRightarrow}\ {\isacharunderscore}{\kern0pt}{\isachardoublequoteclose}\isanewline
\isanewline
\isacommand{sublocale}\isamarkupfalse%
\ filtered{\isacharunderscore}{\kern0pt}sigma{\isacharunderscore}{\kern0pt}finite{\isacharunderscore}{\kern0pt}measure\ {\isasymsubseteq}\ sigma{\isacharunderscore}{\kern0pt}finite{\isacharunderscore}{\kern0pt}subalgebra\ M\ {\isachardoublequoteopen}F\ i{\isachardoublequoteclose}%
\isadelimproof
\ %
\endisadelimproof
%
\isatagproof
\isacommand{by}\isamarkupfalse%
\ {\isacharparenleft}{\kern0pt}metis\ bot{\isachardot}{\kern0pt}extremum\ sigma{\isacharunderscore}{\kern0pt}finite\ sigma{\isacharunderscore}{\kern0pt}finite{\isacharunderscore}{\kern0pt}subalgebra{\isachardot}{\kern0pt}intro\ subalgebra\ sets{\isacharunderscore}{\kern0pt}F{\isacharunderscore}{\kern0pt}mono\ sigma{\isacharunderscore}{\kern0pt}finite{\isacharunderscore}{\kern0pt}subalgebra{\isachardot}{\kern0pt}nested{\isacharunderscore}{\kern0pt}subalg{\isacharunderscore}{\kern0pt}is{\isacharunderscore}{\kern0pt}sigma{\isacharunderscore}{\kern0pt}finite\ subalgebra{\isacharunderscore}{\kern0pt}def{\isacharparenright}{\kern0pt}%
\endisatagproof
{\isafoldproof}%
%
\isadelimproof
%
\endisadelimproof
%
\isadelimdocument
%
\endisadelimdocument
%
\isatagdocument
%
\isamarkupsubsection{Natural Filtration%
}
\isamarkuptrue%
%
\endisatagdocument
{\isafolddocument}%
%
\isadelimdocument
%
\endisadelimdocument
\isacommand{definition}\isamarkupfalse%
\ natural{\isacharunderscore}{\kern0pt}filtration\ {\isacharcolon}{\kern0pt}{\isacharcolon}{\kern0pt}\ {\isachardoublequoteopen}{\isacharprime}{\kern0pt}a\ measure\ {\isasymRightarrow}\ {\isacharprime}{\kern0pt}s\ measure\ {\isasymRightarrow}\ {\isacharparenleft}{\kern0pt}{\isacharprime}{\kern0pt}t\ {\isasymRightarrow}\ {\isacharprime}{\kern0pt}a\ {\isasymRightarrow}\ {\isacharprime}{\kern0pt}s{\isacharparenright}{\kern0pt}\ {\isasymRightarrow}\ {\isacharprime}{\kern0pt}t\ {\isacharcolon}{\kern0pt}{\isacharcolon}{\kern0pt}\ {\isacharbraceleft}{\kern0pt}second{\isacharunderscore}{\kern0pt}countable{\isacharunderscore}{\kern0pt}topology{\isacharcomma}{\kern0pt}\ linorder{\isacharunderscore}{\kern0pt}topology{\isacharcomma}{\kern0pt}\ order{\isacharunderscore}{\kern0pt}bot{\isacharbraceright}{\kern0pt}\ {\isasymRightarrow}\ {\isacharprime}{\kern0pt}a\ measure{\isachardoublequoteclose}\ \isakeyword{where}\isanewline
\ \ {\isachardoublequoteopen}natural{\isacharunderscore}{\kern0pt}filtration\ M\ N\ Y\ {\isacharequal}{\kern0pt}\ {\isacharparenleft}{\kern0pt}{\isasymlambda}t{\isachardot}{\kern0pt}\ restr{\isacharunderscore}{\kern0pt}to{\isacharunderscore}{\kern0pt}subalg\ M\ {\isacharparenleft}{\kern0pt}sigma{\isacharunderscore}{\kern0pt}gen\ {\isacharparenleft}{\kern0pt}space\ M{\isacharparenright}{\kern0pt}\ N\ {\isacharbraceleft}{\kern0pt}Y\ i\ {\isacharbar}{\kern0pt}\ i{\isachardot}{\kern0pt}\ i\ {\isasymle}\ t{\isacharbraceright}{\kern0pt}{\isacharparenright}{\kern0pt}{\isacharparenright}{\kern0pt}{\isachardoublequoteclose}\isanewline
\isanewline
\isacommand{lemma}\isamarkupfalse%
\ \isanewline
\ \ \isakeyword{assumes}\ {\isachardoublequoteopen}{\isasymAnd}i{\isachardot}{\kern0pt}\ Y\ i\ {\isasymin}\ M\ {\isasymrightarrow}\isactrlsub M\ N{\isachardoublequoteclose}\isanewline
\ \ \isakeyword{shows}\ sets{\isacharunderscore}{\kern0pt}natural{\isacharunderscore}{\kern0pt}filtration{\isacharbrackleft}{\kern0pt}simp{\isacharbrackright}{\kern0pt}{\isacharcolon}{\kern0pt}\ {\isachardoublequoteopen}sets\ {\isacharparenleft}{\kern0pt}natural{\isacharunderscore}{\kern0pt}filtration\ M\ N\ Y\ t{\isacharparenright}{\kern0pt}\ {\isacharequal}{\kern0pt}\ sigma{\isacharunderscore}{\kern0pt}sets\ {\isacharparenleft}{\kern0pt}space\ M{\isacharparenright}{\kern0pt}\ {\isacharparenleft}{\kern0pt}{\isasymUnion}i\ {\isasymle}\ t{\isachardot}{\kern0pt}\ {\isacharbraceleft}{\kern0pt}Y\ i\ {\isacharminus}{\kern0pt}{\isacharbackquote}{\kern0pt}\ A\ {\isasyminter}\ space\ M\ {\isacharbar}{\kern0pt}\ A{\isachardot}{\kern0pt}\ A\ {\isasymin}\ N{\isacharbraceright}{\kern0pt}{\isacharparenright}{\kern0pt}{\isachardoublequoteclose}\ \isanewline
\ \ \ \ \isakeyword{and}\ space{\isacharunderscore}{\kern0pt}natural{\isacharunderscore}{\kern0pt}filtration{\isacharbrackleft}{\kern0pt}simp{\isacharbrackright}{\kern0pt}{\isacharcolon}{\kern0pt}\ {\isachardoublequoteopen}space\ {\isacharparenleft}{\kern0pt}natural{\isacharunderscore}{\kern0pt}filtration\ M\ N\ Y\ t{\isacharparenright}{\kern0pt}\ {\isacharequal}{\kern0pt}\ space\ M{\isachardoublequoteclose}\isanewline
%
\isadelimproof
\ \ %
\endisadelimproof
%
\isatagproof
\isacommand{by}\isamarkupfalse%
\ {\isacharparenleft}{\kern0pt}standard{\isacharsemicolon}{\kern0pt}\ {\isacharparenleft}{\kern0pt}subst\ natural{\isacharunderscore}{\kern0pt}filtration{\isacharunderscore}{\kern0pt}def{\isacharcomma}{\kern0pt}\ subst\ sets{\isacharunderscore}{\kern0pt}restr{\isacharunderscore}{\kern0pt}to{\isacharunderscore}{\kern0pt}subalg{\isacharparenright}{\kern0pt}{\isacharparenright}{\kern0pt}\ {\isacharparenleft}{\kern0pt}auto\ simp\ add{\isacharcolon}{\kern0pt}\ natural{\isacharunderscore}{\kern0pt}filtration{\isacharunderscore}{\kern0pt}def\ space{\isacharunderscore}{\kern0pt}restr{\isacharunderscore}{\kern0pt}to{\isacharunderscore}{\kern0pt}subalg\ subalgebra{\isacharunderscore}{\kern0pt}def\ intro{\isacharbang}{\kern0pt}{\isacharcolon}{\kern0pt}\ sets{\isachardot}{\kern0pt}sigma{\isacharunderscore}{\kern0pt}sets{\isacharunderscore}{\kern0pt}subset\ measurable{\isacharunderscore}{\kern0pt}sets{\isacharbrackleft}{\kern0pt}OF\ assms{\isacharbrackright}{\kern0pt}\ sigma{\isacharunderscore}{\kern0pt}sets{\isacharunderscore}{\kern0pt}mono{\isacharparenright}{\kern0pt}%
\endisatagproof
{\isafoldproof}%
%
\isadelimproof
\isanewline
%
\endisadelimproof
%
\isadelimtheory
\isanewline
%
\endisadelimtheory
%
\isatagtheory
\isacommand{end}\isamarkupfalse%
%
\endisatagtheory
{\isafoldtheory}%
%
\isadelimtheory
%
\endisadelimtheory
%
\end{isabellebody}%
\endinput
%:%file=Filtration.tex%:%
%:%10=1%:%
%:%11=1%:%
%:%12=2%:%
%:%13=3%:%
%:%27=5%:%
%:%37=7%:%
%:%38=7%:%
%:%39=8%:%
%:%40=9%:%
%:%41=10%:%
%:%42=11%:%
%:%43=11%:%
%:%44=12%:%
%:%45=12%:%
%:%46=13%:%
%:%47=14%:%
%:%48=14%:%
%:%50=14%:%
%:%54=14%:%
%:%55=14%:%
%:%69=16%:%
%:%79=18%:%
%:%80=18%:%
%:%81=19%:%
%:%82=20%:%
%:%83=21%:%
%:%84=21%:%
%:%85=22%:%
%:%86=23%:%
%:%87=24%:%
%:%90=25%:%
%:%94=25%:%
%:%95=25%:%
%:%100=25%:%
%:%105=26%:%
%:%110=27%:%

%
\begin{isabellebody}%
\setisabellecontext{Banach{\isacharunderscore}{\kern0pt}Conditional{\isacharunderscore}{\kern0pt}Expectation}%
%
\isadelimtheory
%
\endisadelimtheory
%
\isatagtheory
\isacommand{theory}\isamarkupfalse%
\ Banach{\isacharunderscore}{\kern0pt}Conditional{\isacharunderscore}{\kern0pt}Expectation\ \ \ \ \ \ \ \ \ \ \ \ \ \ \ \ \ \ \ \ \ \ \ \ \ \ \ \ \ \ \ \ \ \ \ \ \ \ \ \ \ \ \ \ \ \ \ \ \ \ \ \ \ \ \ \ \ \ \ \ \ \ \ \ \ \ \ \ \ \ \ \ \ \ \ \ \ \ \ \ \ \ \isanewline
\isakeyword{imports}\ {\isachardoublequoteopen}HOL{\isacharminus}{\kern0pt}Probability{\isachardot}{\kern0pt}Conditional{\isacharunderscore}{\kern0pt}Expectation{\isachardoublequoteclose}\isanewline
\ \ \ \ \ \ \ \ Sigma{\isacharunderscore}{\kern0pt}Finite{\isacharunderscore}{\kern0pt}Measure{\isacharunderscore}{\kern0pt}Addendum\ Bochner{\isacharunderscore}{\kern0pt}Integration{\isacharunderscore}{\kern0pt}Addendum\ Elementary{\isacharunderscore}{\kern0pt}Metric{\isacharunderscore}{\kern0pt}Spaces{\isacharunderscore}{\kern0pt}Addendum\isanewline
\isakeyword{begin}%
\endisatagtheory
{\isafoldtheory}%
%
\isadelimtheory
\ \ \ \ \ \ \ \ \ \ \ \ \ \ \ \ \ \ \ \ \ \ \ \ \ \ \ \ \ \ \ \ \ \ \ \ \ \ \ \ \ \ \ \isanewline
%
\endisadelimtheory
\isanewline
\isacommand{definition}\isamarkupfalse%
\ has{\isacharunderscore}{\kern0pt}cond{\isacharunderscore}{\kern0pt}exp\ {\isacharcolon}{\kern0pt}{\isacharcolon}{\kern0pt}\ {\isachardoublequoteopen}{\isacharprime}{\kern0pt}a\ measure\ {\isasymRightarrow}\ {\isacharprime}{\kern0pt}a\ measure\ {\isasymRightarrow}\ {\isacharparenleft}{\kern0pt}{\isacharprime}{\kern0pt}a\ {\isasymRightarrow}\ {\isacharprime}{\kern0pt}b{\isacharparenright}{\kern0pt}\ {\isasymRightarrow}\ {\isacharparenleft}{\kern0pt}{\isacharprime}{\kern0pt}a\ {\isasymRightarrow}\ {\isacharprime}{\kern0pt}b{\isacharcolon}{\kern0pt}{\isacharcolon}{\kern0pt}{\isacharbraceleft}{\kern0pt}real{\isacharunderscore}{\kern0pt}normed{\isacharunderscore}{\kern0pt}vector{\isacharcomma}{\kern0pt}\ second{\isacharunderscore}{\kern0pt}countable{\isacharunderscore}{\kern0pt}topology{\isacharbraceright}{\kern0pt}{\isacharparenright}{\kern0pt}\ {\isasymRightarrow}\ bool{\isachardoublequoteclose}\ \isakeyword{where}\ \isanewline
\ \ {\isachardoublequoteopen}has{\isacharunderscore}{\kern0pt}cond{\isacharunderscore}{\kern0pt}exp\ M\ F\ f\ g\ {\isacharequal}{\kern0pt}\ {\isacharparenleft}{\kern0pt}{\isacharparenleft}{\kern0pt}{\isasymforall}A\ {\isasymin}\ sets\ F{\isachardot}{\kern0pt}\ {\isacharparenleft}{\kern0pt}{\isasymintegral}\ x\ {\isasymin}\ A{\isachardot}{\kern0pt}\ f\ x\ {\isasympartial}M{\isacharparenright}{\kern0pt}\ {\isacharequal}{\kern0pt}\ {\isacharparenleft}{\kern0pt}{\isasymintegral}\ x\ {\isasymin}\ A{\isachardot}{\kern0pt}\ g\ x\ {\isasympartial}M{\isacharparenright}{\kern0pt}{\isacharparenright}{\kern0pt}\isanewline
\ \ \ \ \ \ \ \ \ \ \ \ \ \ \ \ \ \ \ \ \ \ \ \ {\isasymand}\ integrable\ M\ f\ \isanewline
\ \ \ \ \ \ \ \ \ \ \ \ \ \ \ \ \ \ \ \ \ \ \ \ {\isasymand}\ integrable\ M\ g\ \isanewline
\ \ \ \ \ \ \ \ \ \ \ \ \ \ \ \ \ \ \ \ \ \ \ \ {\isasymand}\ g\ {\isasymin}\ borel{\isacharunderscore}{\kern0pt}measurable\ F{\isacharparenright}{\kern0pt}{\isachardoublequoteclose}\isanewline
\isanewline
\isacommand{lemma}\isamarkupfalse%
\ has{\isacharunderscore}{\kern0pt}cond{\isacharunderscore}{\kern0pt}expI{\isacharprime}{\kern0pt}{\isacharbrackleft}{\kern0pt}intro{\isacharbrackright}{\kern0pt}{\isacharcolon}{\kern0pt}\isanewline
\ \ \isakeyword{assumes}\ {\isachardoublequoteopen}{\isasymAnd}A{\isachardot}{\kern0pt}\ A\ {\isasymin}\ sets\ F\ {\isasymLongrightarrow}\ {\isacharparenleft}{\kern0pt}{\isasymintegral}\ x\ {\isasymin}\ A{\isachardot}{\kern0pt}\ f\ x\ {\isasympartial}M{\isacharparenright}{\kern0pt}\ {\isacharequal}{\kern0pt}\ {\isacharparenleft}{\kern0pt}{\isasymintegral}\ x\ {\isasymin}\ A{\isachardot}{\kern0pt}\ g\ x\ {\isasympartial}M{\isacharparenright}{\kern0pt}{\isachardoublequoteclose}\isanewline
\ \ \ \ \ \ \ \ \ \ {\isachardoublequoteopen}integrable\ M\ f{\isachardoublequoteclose}\isanewline
\ \ \ \ \ \ \ \ \ \ {\isachardoublequoteopen}integrable\ M\ g{\isachardoublequoteclose}\isanewline
\ \ \ \ \ \ \ \ \ \ {\isachardoublequoteopen}g\ {\isasymin}\ borel{\isacharunderscore}{\kern0pt}measurable\ F{\isachardoublequoteclose}\isanewline
\ \ \isakeyword{shows}\ {\isachardoublequoteopen}has{\isacharunderscore}{\kern0pt}cond{\isacharunderscore}{\kern0pt}exp\ M\ F\ f\ g{\isachardoublequoteclose}\isanewline
%
\isadelimproof
\ \ %
\endisadelimproof
%
\isatagproof
\isacommand{using}\isamarkupfalse%
\ assms\ \isacommand{unfolding}\isamarkupfalse%
\ has{\isacharunderscore}{\kern0pt}cond{\isacharunderscore}{\kern0pt}exp{\isacharunderscore}{\kern0pt}def\ \isacommand{by}\isamarkupfalse%
\ simp%
\endisatagproof
{\isafoldproof}%
%
\isadelimproof
\isanewline
%
\endisadelimproof
\isanewline
\isacommand{lemma}\isamarkupfalse%
\ has{\isacharunderscore}{\kern0pt}cond{\isacharunderscore}{\kern0pt}expD{\isacharcolon}{\kern0pt}\isanewline
\ \ \isakeyword{assumes}\ {\isachardoublequoteopen}has{\isacharunderscore}{\kern0pt}cond{\isacharunderscore}{\kern0pt}exp\ M\ F\ f\ g{\isachardoublequoteclose}\isanewline
\ \ \isakeyword{shows}\ {\isachardoublequoteopen}{\isasymAnd}A{\isachardot}{\kern0pt}\ A\ {\isasymin}\ sets\ F\ {\isasymLongrightarrow}\ {\isacharparenleft}{\kern0pt}{\isasymintegral}\ x\ {\isasymin}\ A{\isachardot}{\kern0pt}\ f\ x\ {\isasympartial}M{\isacharparenright}{\kern0pt}\ {\isacharequal}{\kern0pt}\ {\isacharparenleft}{\kern0pt}{\isasymintegral}\ x\ {\isasymin}\ A{\isachardot}{\kern0pt}\ g\ x\ {\isasympartial}M{\isacharparenright}{\kern0pt}{\isachardoublequoteclose}\isanewline
\ \ \ \ \ \ \ \ {\isachardoublequoteopen}integrable\ M\ f{\isachardoublequoteclose}\isanewline
\ \ \ \ \ \ \ \ {\isachardoublequoteopen}integrable\ M\ g{\isachardoublequoteclose}\isanewline
\ \ \ \ \ \ \ \ {\isachardoublequoteopen}g\ {\isasymin}\ borel{\isacharunderscore}{\kern0pt}measurable\ F{\isachardoublequoteclose}\isanewline
%
\isadelimproof
\ \ %
\endisadelimproof
%
\isatagproof
\isacommand{using}\isamarkupfalse%
\ assms\ \isacommand{unfolding}\isamarkupfalse%
\ has{\isacharunderscore}{\kern0pt}cond{\isacharunderscore}{\kern0pt}exp{\isacharunderscore}{\kern0pt}def\ \isacommand{by}\isamarkupfalse%
\ simp{\isacharplus}{\kern0pt}%
\endisatagproof
{\isafoldproof}%
%
\isadelimproof
\isanewline
%
\endisadelimproof
\isanewline
\isanewline
\isanewline
\isacommand{lemma}\isamarkupfalse%
\ has{\isacharunderscore}{\kern0pt}cond{\isacharunderscore}{\kern0pt}exp{\isacharunderscore}{\kern0pt}nested{\isacharunderscore}{\kern0pt}subalg{\isacharcolon}{\kern0pt}\isanewline
\ \ \isakeyword{fixes}\ f\ {\isacharcolon}{\kern0pt}{\isacharcolon}{\kern0pt}\ {\isachardoublequoteopen}{\isacharprime}{\kern0pt}a\ {\isasymRightarrow}\ {\isacharprime}{\kern0pt}b{\isacharcolon}{\kern0pt}{\isacharcolon}{\kern0pt}{\isacharbraceleft}{\kern0pt}second{\isacharunderscore}{\kern0pt}countable{\isacharunderscore}{\kern0pt}topology{\isacharcomma}{\kern0pt}\ banach{\isacharbraceright}{\kern0pt}{\isachardoublequoteclose}\isanewline
\ \ \isakeyword{assumes}\ {\isachardoublequoteopen}subalgebra\ M\ G{\isachardoublequoteclose}\ {\isachardoublequoteopen}subalgebra\ G\ F{\isachardoublequoteclose}\ {\isachardoublequoteopen}integrable\ M\ f{\isachardoublequoteclose}\ {\isachardoublequoteopen}has{\isacharunderscore}{\kern0pt}cond{\isacharunderscore}{\kern0pt}exp\ M\ F\ f\ h{\isachardoublequoteclose}\ {\isachardoublequoteopen}has{\isacharunderscore}{\kern0pt}cond{\isacharunderscore}{\kern0pt}exp\ M\ G\ f\ h{\isacharprime}{\kern0pt}{\isachardoublequoteclose}\isanewline
\ \ \isakeyword{shows}\ {\isachardoublequoteopen}has{\isacharunderscore}{\kern0pt}cond{\isacharunderscore}{\kern0pt}exp\ M\ F\ h{\isacharprime}{\kern0pt}\ h{\isachardoublequoteclose}\isanewline
%
\isadelimproof
%
\endisadelimproof
%
\isatagproof
\isacommand{proof}\isamarkupfalse%
\ {\isacharminus}{\kern0pt}\isanewline
\ \ \isacommand{show}\isamarkupfalse%
\ {\isacharquery}{\kern0pt}thesis\isanewline
\ \ \isacommand{proof}\isamarkupfalse%
\ {\isacharparenleft}{\kern0pt}standard{\isacharcomma}{\kern0pt}\ goal{\isacharunderscore}{\kern0pt}cases{\isacharparenright}{\kern0pt}\isanewline
\ \ \ \ \isacommand{case}\isamarkupfalse%
\ {\isacharparenleft}{\kern0pt}{\isadigit{1}}\ A{\isacharparenright}{\kern0pt}\isanewline
\ \ \ \ \isacommand{show}\isamarkupfalse%
\ {\isacharquery}{\kern0pt}case\ \isacommand{by}\isamarkupfalse%
\ {\isacharparenleft}{\kern0pt}metis\ {\isadigit{1}}\ assms{\isacharparenleft}{\kern0pt}{\isadigit{2}}{\isacharcomma}{\kern0pt}{\isadigit{4}}{\isacharcomma}{\kern0pt}{\isadigit{5}}{\isacharparenright}{\kern0pt}\ has{\isacharunderscore}{\kern0pt}cond{\isacharunderscore}{\kern0pt}expD{\isacharparenleft}{\kern0pt}{\isadigit{1}}{\isacharparenright}{\kern0pt}\ in{\isacharunderscore}{\kern0pt}mono\ subalgebra{\isacharunderscore}{\kern0pt}def{\isacharparenright}{\kern0pt}\isanewline
\ \ \isacommand{next}\isamarkupfalse%
\isanewline
\ \ \ \ \isacommand{case}\isamarkupfalse%
\ {\isadigit{2}}\isanewline
\ \ \ \ \isacommand{then}\isamarkupfalse%
\ \isacommand{show}\isamarkupfalse%
\ {\isacharquery}{\kern0pt}case\ \isacommand{using}\isamarkupfalse%
\ has{\isacharunderscore}{\kern0pt}cond{\isacharunderscore}{\kern0pt}expD{\isacharparenleft}{\kern0pt}{\isadigit{3}}{\isacharparenright}{\kern0pt}{\isacharbrackleft}{\kern0pt}OF\ assms{\isacharparenleft}{\kern0pt}{\isadigit{5}}{\isacharparenright}{\kern0pt}{\isacharbrackright}{\kern0pt}\ \isacommand{by}\isamarkupfalse%
\ blast\isanewline
\ \ \isacommand{next}\isamarkupfalse%
\isanewline
\ \ \ \ \isacommand{case}\isamarkupfalse%
\ {\isadigit{3}}\isanewline
\ \ \ \ \isacommand{then}\isamarkupfalse%
\ \isacommand{show}\isamarkupfalse%
\ {\isacharquery}{\kern0pt}case\ \isacommand{using}\isamarkupfalse%
\ has{\isacharunderscore}{\kern0pt}cond{\isacharunderscore}{\kern0pt}expD{\isacharparenleft}{\kern0pt}{\isadigit{3}}{\isacharparenright}{\kern0pt}{\isacharbrackleft}{\kern0pt}OF\ assms{\isacharparenleft}{\kern0pt}{\isadigit{4}}{\isacharparenright}{\kern0pt}{\isacharbrackright}{\kern0pt}\ \isacommand{{\isachardot}{\kern0pt}}\isamarkupfalse%
\isanewline
\ \ \isacommand{next}\isamarkupfalse%
\isanewline
\ \ \ \ \isacommand{case}\isamarkupfalse%
\ {\isadigit{4}}\isanewline
\ \ \ \ \isacommand{then}\isamarkupfalse%
\ \isacommand{show}\isamarkupfalse%
\ {\isacharquery}{\kern0pt}case\ \isacommand{using}\isamarkupfalse%
\ has{\isacharunderscore}{\kern0pt}cond{\isacharunderscore}{\kern0pt}expD{\isacharparenleft}{\kern0pt}{\isadigit{4}}{\isacharparenright}{\kern0pt}{\isacharbrackleft}{\kern0pt}OF\ assms{\isacharparenleft}{\kern0pt}{\isadigit{4}}{\isacharparenright}{\kern0pt}{\isacharbrackright}{\kern0pt}\ \isacommand{{\isachardot}{\kern0pt}}\isamarkupfalse%
\isanewline
\ \ \isacommand{qed}\isamarkupfalse%
\isanewline
\isacommand{qed}\isamarkupfalse%
%
\endisatagproof
{\isafoldproof}%
%
\isadelimproof
\isanewline
%
\endisadelimproof
\isanewline
\isacommand{definition}\isamarkupfalse%
\ cond{\isacharunderscore}{\kern0pt}exp\ {\isacharcolon}{\kern0pt}{\isacharcolon}{\kern0pt}\ {\isachardoublequoteopen}{\isacharprime}{\kern0pt}a\ measure\ {\isasymRightarrow}\ {\isacharprime}{\kern0pt}a\ measure\ {\isasymRightarrow}\ {\isacharparenleft}{\kern0pt}{\isacharprime}{\kern0pt}a\ {\isasymRightarrow}\ {\isacharprime}{\kern0pt}b{\isacharparenright}{\kern0pt}\ {\isasymRightarrow}\ {\isacharparenleft}{\kern0pt}{\isacharprime}{\kern0pt}a\ {\isasymRightarrow}\ {\isacharprime}{\kern0pt}b{\isacharcolon}{\kern0pt}{\isacharcolon}{\kern0pt}{\isacharbraceleft}{\kern0pt}banach{\isacharcomma}{\kern0pt}\ second{\isacharunderscore}{\kern0pt}countable{\isacharunderscore}{\kern0pt}topology{\isacharbraceright}{\kern0pt}{\isacharparenright}{\kern0pt}{\isachardoublequoteclose}\ \isakeyword{where}\isanewline
\ \ {\isachardoublequoteopen}cond{\isacharunderscore}{\kern0pt}exp\ M\ F\ f\ {\isacharequal}{\kern0pt}\ {\isacharparenleft}{\kern0pt}if\ {\isasymexists}g{\isachardot}{\kern0pt}\ has{\isacharunderscore}{\kern0pt}cond{\isacharunderscore}{\kern0pt}exp\ M\ F\ f\ g\ then\ {\isacharparenleft}{\kern0pt}SOME\ g{\isachardot}{\kern0pt}\ has{\isacharunderscore}{\kern0pt}cond{\isacharunderscore}{\kern0pt}exp\ M\ F\ f\ g{\isacharparenright}{\kern0pt}\ else\ {\isacharparenleft}{\kern0pt}{\isasymlambda}{\isacharunderscore}{\kern0pt}{\isachardot}{\kern0pt}\ {\isadigit{0}}{\isacharparenright}{\kern0pt}{\isacharparenright}{\kern0pt}{\isachardoublequoteclose}\isanewline
\isanewline
\isacommand{lemma}\isamarkupfalse%
\ borel{\isacharunderscore}{\kern0pt}measurable{\isacharunderscore}{\kern0pt}cond{\isacharunderscore}{\kern0pt}exp{\isacharbrackleft}{\kern0pt}measurable{\isacharbrackright}{\kern0pt}{\isacharcolon}{\kern0pt}\ {\isachardoublequoteopen}cond{\isacharunderscore}{\kern0pt}exp\ M\ F\ f\ {\isasymin}\ borel{\isacharunderscore}{\kern0pt}measurable\ F{\isachardoublequoteclose}\ \isanewline
%
\isadelimproof
\ \ %
\endisadelimproof
%
\isatagproof
\isacommand{by}\isamarkupfalse%
\ {\isacharparenleft}{\kern0pt}metis\ cond{\isacharunderscore}{\kern0pt}exp{\isacharunderscore}{\kern0pt}def\ someI\ has{\isacharunderscore}{\kern0pt}cond{\isacharunderscore}{\kern0pt}exp{\isacharunderscore}{\kern0pt}def\ borel{\isacharunderscore}{\kern0pt}measurable{\isacharunderscore}{\kern0pt}const{\isacharparenright}{\kern0pt}%
\endisatagproof
{\isafoldproof}%
%
\isadelimproof
\isanewline
%
\endisadelimproof
\isanewline
\isacommand{lemma}\isamarkupfalse%
\ integrable{\isacharunderscore}{\kern0pt}cond{\isacharunderscore}{\kern0pt}exp{\isacharbrackleft}{\kern0pt}intro{\isacharbrackright}{\kern0pt}{\isacharcolon}{\kern0pt}\ {\isachardoublequoteopen}integrable\ M\ {\isacharparenleft}{\kern0pt}cond{\isacharunderscore}{\kern0pt}exp\ M\ F\ f{\isacharparenright}{\kern0pt}{\isachardoublequoteclose}\ \isanewline
%
\isadelimproof
\ \ %
\endisadelimproof
%
\isatagproof
\isacommand{by}\isamarkupfalse%
\ {\isacharparenleft}{\kern0pt}metis\ cond{\isacharunderscore}{\kern0pt}exp{\isacharunderscore}{\kern0pt}def\ has{\isacharunderscore}{\kern0pt}cond{\isacharunderscore}{\kern0pt}expD{\isacharparenleft}{\kern0pt}{\isadigit{3}}{\isacharparenright}{\kern0pt}\ integrable{\isacharunderscore}{\kern0pt}zero\ someI{\isacharparenright}{\kern0pt}%
\endisatagproof
{\isafoldproof}%
%
\isadelimproof
\isanewline
%
\endisadelimproof
\isanewline
\isacommand{lemma}\isamarkupfalse%
\ set{\isacharunderscore}{\kern0pt}integrable{\isacharunderscore}{\kern0pt}cond{\isacharunderscore}{\kern0pt}exp{\isacharbrackleft}{\kern0pt}intro{\isacharbrackright}{\kern0pt}{\isacharcolon}{\kern0pt}\isanewline
\ \ \isakeyword{assumes}\ {\isachardoublequoteopen}A\ {\isasymin}\ sets\ M{\isachardoublequoteclose}\isanewline
\isakeyword{shows}\ {\isachardoublequoteopen}set{\isacharunderscore}{\kern0pt}integrable\ M\ A\ {\isacharparenleft}{\kern0pt}cond{\isacharunderscore}{\kern0pt}exp\ M\ F\ f{\isacharparenright}{\kern0pt}{\isachardoublequoteclose}%
\isadelimproof
\ %
\endisadelimproof
%
\isatagproof
\isacommand{using}\isamarkupfalse%
\ integrable{\isacharunderscore}{\kern0pt}mult{\isacharunderscore}{\kern0pt}indicator{\isacharbrackleft}{\kern0pt}OF\ assms\ integrable{\isacharunderscore}{\kern0pt}cond{\isacharunderscore}{\kern0pt}exp{\isacharcomma}{\kern0pt}\ of\ F\ f{\isacharbrackright}{\kern0pt}\ \isacommand{by}\isamarkupfalse%
\ {\isacharparenleft}{\kern0pt}auto\ simp\ add{\isacharcolon}{\kern0pt}\ set{\isacharunderscore}{\kern0pt}integrable{\isacharunderscore}{\kern0pt}def\ intro{\isacharbang}{\kern0pt}{\isacharcolon}{\kern0pt}\ integrable{\isacharunderscore}{\kern0pt}mult{\isacharunderscore}{\kern0pt}indicator{\isacharbrackleft}{\kern0pt}OF\ assms\ integrable{\isacharunderscore}{\kern0pt}cond{\isacharunderscore}{\kern0pt}exp{\isacharbrackright}{\kern0pt}{\isacharparenright}{\kern0pt}%
\endisatagproof
{\isafoldproof}%
%
\isadelimproof
%
\endisadelimproof
\isanewline
\isanewline
\isacommand{context}\isamarkupfalse%
\ sigma{\isacharunderscore}{\kern0pt}finite{\isacharunderscore}{\kern0pt}subalgebra\isanewline
\isakeyword{begin}\isanewline
\isanewline
\isacommand{lemma}\isamarkupfalse%
\ borel{\isacharunderscore}{\kern0pt}measurable{\isacharunderscore}{\kern0pt}cond{\isacharunderscore}{\kern0pt}exp{\isacharprime}{\kern0pt}{\isacharbrackleft}{\kern0pt}measurable{\isacharbrackright}{\kern0pt}{\isacharcolon}{\kern0pt}\ {\isachardoublequoteopen}cond{\isacharunderscore}{\kern0pt}exp\ M\ F\ f\ {\isasymin}\ borel{\isacharunderscore}{\kern0pt}measurable\ M{\isachardoublequoteclose}\isanewline
%
\isadelimproof
\ \ %
\endisadelimproof
%
\isatagproof
\isacommand{by}\isamarkupfalse%
\ {\isacharparenleft}{\kern0pt}metis\ cond{\isacharunderscore}{\kern0pt}exp{\isacharunderscore}{\kern0pt}def\ someI\ has{\isacharunderscore}{\kern0pt}cond{\isacharunderscore}{\kern0pt}exp{\isacharunderscore}{\kern0pt}def\ borel{\isacharunderscore}{\kern0pt}measurable{\isacharunderscore}{\kern0pt}const\ subalg\ measurable{\isacharunderscore}{\kern0pt}from{\isacharunderscore}{\kern0pt}subalg{\isacharparenright}{\kern0pt}%
\endisatagproof
{\isafoldproof}%
%
\isadelimproof
\isanewline
%
\endisadelimproof
\isanewline
\isacommand{lemma}\isamarkupfalse%
\ cond{\isacharunderscore}{\kern0pt}exp{\isacharunderscore}{\kern0pt}null{\isacharcolon}{\kern0pt}\ \isanewline
\ \ \isakeyword{assumes}\ {\isachardoublequoteopen}{\isasymnexists}g{\isachardot}{\kern0pt}\ has{\isacharunderscore}{\kern0pt}cond{\isacharunderscore}{\kern0pt}exp\ M\ F\ f\ g{\isachardoublequoteclose}\ \isanewline
\ \ \isakeyword{shows}\ {\isachardoublequoteopen}cond{\isacharunderscore}{\kern0pt}exp\ M\ F\ f\ {\isacharequal}{\kern0pt}\ {\isacharparenleft}{\kern0pt}{\isasymlambda}{\isacharunderscore}{\kern0pt}{\isachardot}{\kern0pt}\ {\isadigit{0}}{\isacharparenright}{\kern0pt}{\isachardoublequoteclose}\isanewline
%
\isadelimproof
\ \ %
\endisadelimproof
%
\isatagproof
\isacommand{unfolding}\isamarkupfalse%
\ cond{\isacharunderscore}{\kern0pt}exp{\isacharunderscore}{\kern0pt}def\ \isacommand{using}\isamarkupfalse%
\ assms\ \isacommand{by}\isamarkupfalse%
\ argo%
\endisatagproof
{\isafoldproof}%
%
\isadelimproof
\isanewline
%
\endisadelimproof
\isanewline
\isacommand{lemma}\isamarkupfalse%
\ has{\isacharunderscore}{\kern0pt}cond{\isacharunderscore}{\kern0pt}exp{\isacharunderscore}{\kern0pt}charact{\isacharcolon}{\kern0pt}\isanewline
\ \ \isakeyword{fixes}\ f\ {\isacharcolon}{\kern0pt}{\isacharcolon}{\kern0pt}\ {\isachardoublequoteopen}{\isacharprime}{\kern0pt}a\ {\isasymRightarrow}\ {\isacharprime}{\kern0pt}b{\isacharcolon}{\kern0pt}{\isacharcolon}{\kern0pt}{\isacharbraceleft}{\kern0pt}second{\isacharunderscore}{\kern0pt}countable{\isacharunderscore}{\kern0pt}topology{\isacharcomma}{\kern0pt}\ banach{\isacharbraceright}{\kern0pt}{\isachardoublequoteclose}\isanewline
\ \ \isakeyword{assumes}\ {\isachardoublequoteopen}has{\isacharunderscore}{\kern0pt}cond{\isacharunderscore}{\kern0pt}exp\ M\ F\ f\ g{\isachardoublequoteclose}\isanewline
\ \ \isakeyword{shows}\ {\isachardoublequoteopen}has{\isacharunderscore}{\kern0pt}cond{\isacharunderscore}{\kern0pt}exp\ M\ F\ f\ {\isacharparenleft}{\kern0pt}cond{\isacharunderscore}{\kern0pt}exp\ M\ F\ f{\isacharparenright}{\kern0pt}{\isachardoublequoteclose}\isanewline
\ \ \ \ \ \ \ \ {\isachardoublequoteopen}AE\ x\ in\ M{\isachardot}{\kern0pt}\ cond{\isacharunderscore}{\kern0pt}exp\ M\ F\ f\ x\ {\isacharequal}{\kern0pt}\ g\ x{\isachardoublequoteclose}\isanewline
%
\isadelimproof
%
\endisadelimproof
%
\isatagproof
\isacommand{proof}\isamarkupfalse%
\ {\isacharminus}{\kern0pt}\isanewline
\ \ \isacommand{show}\isamarkupfalse%
\ cond{\isacharunderscore}{\kern0pt}exp{\isacharcolon}{\kern0pt}\ {\isachardoublequoteopen}has{\isacharunderscore}{\kern0pt}cond{\isacharunderscore}{\kern0pt}exp\ M\ F\ f\ {\isacharparenleft}{\kern0pt}cond{\isacharunderscore}{\kern0pt}exp\ M\ F\ f{\isacharparenright}{\kern0pt}{\isachardoublequoteclose}\ \isacommand{using}\isamarkupfalse%
\ assms\ someI\ cond{\isacharunderscore}{\kern0pt}exp{\isacharunderscore}{\kern0pt}def\ \isacommand{by}\isamarkupfalse%
\ metis\isanewline
\ \ \isacommand{let}\isamarkupfalse%
\ {\isacharquery}{\kern0pt}MF\ {\isacharequal}{\kern0pt}\ {\isachardoublequoteopen}restr{\isacharunderscore}{\kern0pt}to{\isacharunderscore}{\kern0pt}subalg\ M\ F{\isachardoublequoteclose}\isanewline
\ \ \isacommand{interpret}\isamarkupfalse%
\ sigma{\isacharunderscore}{\kern0pt}finite{\isacharunderscore}{\kern0pt}measure\ {\isacharquery}{\kern0pt}MF\ \isacommand{by}\isamarkupfalse%
\ {\isacharparenleft}{\kern0pt}rule\ sigma{\isacharunderscore}{\kern0pt}fin{\isacharunderscore}{\kern0pt}subalg{\isacharparenright}{\kern0pt}\isanewline
\ \ \isacommand{{\isacharbraceleft}{\kern0pt}}\isamarkupfalse%
\isanewline
\ \ \ \ \isacommand{fix}\isamarkupfalse%
\ A\ \isacommand{assume}\isamarkupfalse%
\ {\isachardoublequoteopen}A\ {\isasymin}\ sets\ {\isacharquery}{\kern0pt}MF{\isachardoublequoteclose}\isanewline
\ \ \ \ \isacommand{then}\isamarkupfalse%
\ \isacommand{have}\isamarkupfalse%
\ {\isacharbrackleft}{\kern0pt}measurable{\isacharbrackright}{\kern0pt}{\isacharcolon}{\kern0pt}\ {\isachardoublequoteopen}A\ {\isasymin}\ sets\ F{\isachardoublequoteclose}\ \isacommand{using}\isamarkupfalse%
\ sets{\isacharunderscore}{\kern0pt}restr{\isacharunderscore}{\kern0pt}to{\isacharunderscore}{\kern0pt}subalg{\isacharbrackleft}{\kern0pt}OF\ subalg{\isacharbrackright}{\kern0pt}\ \isacommand{by}\isamarkupfalse%
\ simp\isanewline
\ \ \ \ \isacommand{have}\isamarkupfalse%
\ {\isachardoublequoteopen}{\isacharparenleft}{\kern0pt}{\isasymintegral}x\ {\isasymin}\ A{\isachardot}{\kern0pt}\ g\ x\ {\isasympartial}{\isacharquery}{\kern0pt}MF{\isacharparenright}{\kern0pt}\ {\isacharequal}{\kern0pt}\ {\isacharparenleft}{\kern0pt}{\isasymintegral}x\ {\isasymin}\ A{\isachardot}{\kern0pt}\ g\ x\ {\isasympartial}M{\isacharparenright}{\kern0pt}{\isachardoublequoteclose}\ \isacommand{using}\isamarkupfalse%
\ assms\ subalg\ \isacommand{by}\isamarkupfalse%
\ {\isacharparenleft}{\kern0pt}auto\ simp\ add{\isacharcolon}{\kern0pt}\ integral{\isacharunderscore}{\kern0pt}subalgebra{\isadigit{2}}\ set{\isacharunderscore}{\kern0pt}lebesgue{\isacharunderscore}{\kern0pt}integral{\isacharunderscore}{\kern0pt}def\ dest{\isacharbang}{\kern0pt}{\isacharcolon}{\kern0pt}\ has{\isacharunderscore}{\kern0pt}cond{\isacharunderscore}{\kern0pt}expD{\isacharparenright}{\kern0pt}\isanewline
\ \ \ \ \isacommand{also}\isamarkupfalse%
\ \isacommand{have}\isamarkupfalse%
\ {\isachardoublequoteopen}{\isachardot}{\kern0pt}{\isachardot}{\kern0pt}{\isachardot}{\kern0pt}\ {\isacharequal}{\kern0pt}\ {\isacharparenleft}{\kern0pt}{\isasymintegral}x\ {\isasymin}\ A{\isachardot}{\kern0pt}\ cond{\isacharunderscore}{\kern0pt}exp\ M\ F\ f\ x\ {\isasympartial}M{\isacharparenright}{\kern0pt}{\isachardoublequoteclose}\ \isacommand{using}\isamarkupfalse%
\ assms\ cond{\isacharunderscore}{\kern0pt}exp\ \isacommand{by}\isamarkupfalse%
\ {\isacharparenleft}{\kern0pt}simp\ add{\isacharcolon}{\kern0pt}\ has{\isacharunderscore}{\kern0pt}cond{\isacharunderscore}{\kern0pt}exp{\isacharunderscore}{\kern0pt}def{\isacharparenright}{\kern0pt}\isanewline
\ \ \ \ \isacommand{also}\isamarkupfalse%
\ \isacommand{have}\isamarkupfalse%
\ {\isachardoublequoteopen}{\isachardot}{\kern0pt}{\isachardot}{\kern0pt}{\isachardot}{\kern0pt}\ {\isacharequal}{\kern0pt}\ {\isacharparenleft}{\kern0pt}{\isasymintegral}x\ {\isasymin}\ A{\isachardot}{\kern0pt}\ cond{\isacharunderscore}{\kern0pt}exp\ M\ F\ f\ x\ {\isasympartial}{\isacharquery}{\kern0pt}MF{\isacharparenright}{\kern0pt}{\isachardoublequoteclose}\ \isacommand{using}\isamarkupfalse%
\ subalg\ \isacommand{by}\isamarkupfalse%
\ {\isacharparenleft}{\kern0pt}auto\ simp\ add{\isacharcolon}{\kern0pt}\ integral{\isacharunderscore}{\kern0pt}subalgebra{\isadigit{2}}\ set{\isacharunderscore}{\kern0pt}lebesgue{\isacharunderscore}{\kern0pt}integral{\isacharunderscore}{\kern0pt}def{\isacharparenright}{\kern0pt}\isanewline
\ \ \ \ \isacommand{finally}\isamarkupfalse%
\ \isacommand{have}\isamarkupfalse%
\ {\isachardoublequoteopen}{\isacharparenleft}{\kern0pt}{\isasymintegral}x\ {\isasymin}\ A{\isachardot}{\kern0pt}\ g\ x\ {\isasympartial}{\isacharquery}{\kern0pt}MF{\isacharparenright}{\kern0pt}\ {\isacharequal}{\kern0pt}\ {\isacharparenleft}{\kern0pt}{\isasymintegral}x\ {\isasymin}\ A{\isachardot}{\kern0pt}\ cond{\isacharunderscore}{\kern0pt}exp\ M\ F\ f\ x\ {\isasympartial}{\isacharquery}{\kern0pt}MF{\isacharparenright}{\kern0pt}{\isachardoublequoteclose}\ \isacommand{by}\isamarkupfalse%
\ simp\isanewline
\ \ \isacommand{{\isacharbraceright}{\kern0pt}}\isamarkupfalse%
\isanewline
\ \ \isacommand{hence}\isamarkupfalse%
\ {\isachardoublequoteopen}AE\ x\ in\ {\isacharquery}{\kern0pt}MF{\isachardot}{\kern0pt}\ cond{\isacharunderscore}{\kern0pt}exp\ M\ F\ f\ x\ {\isacharequal}{\kern0pt}\ g\ x{\isachardoublequoteclose}\ \isacommand{using}\isamarkupfalse%
\ cond{\isacharunderscore}{\kern0pt}exp\ assms\ subalg\ \isacommand{by}\isamarkupfalse%
\ {\isacharparenleft}{\kern0pt}intro\ density{\isacharunderscore}{\kern0pt}unique{\isacharcomma}{\kern0pt}\ auto\ dest{\isacharcolon}{\kern0pt}\ has{\isacharunderscore}{\kern0pt}cond{\isacharunderscore}{\kern0pt}expD\ intro{\isacharbang}{\kern0pt}{\isacharcolon}{\kern0pt}\ integrable{\isacharunderscore}{\kern0pt}in{\isacharunderscore}{\kern0pt}subalg{\isacharparenright}{\kern0pt}\isanewline
\ \ \isacommand{then}\isamarkupfalse%
\ \isacommand{show}\isamarkupfalse%
\ {\isachardoublequoteopen}AE\ x\ in\ M{\isachardot}{\kern0pt}\ cond{\isacharunderscore}{\kern0pt}exp\ M\ F\ f\ x\ {\isacharequal}{\kern0pt}\ g\ x{\isachardoublequoteclose}\ \isacommand{using}\isamarkupfalse%
\ AE{\isacharunderscore}{\kern0pt}restr{\isacharunderscore}{\kern0pt}to{\isacharunderscore}{\kern0pt}subalg{\isacharbrackleft}{\kern0pt}OF\ subalg{\isacharbrackright}{\kern0pt}\ \isacommand{by}\isamarkupfalse%
\ simp\isanewline
\isacommand{qed}\isamarkupfalse%
%
\endisatagproof
{\isafoldproof}%
%
\isadelimproof
\isanewline
%
\endisadelimproof
\isanewline
\isacommand{lemma}\isamarkupfalse%
\ cond{\isacharunderscore}{\kern0pt}exp{\isacharunderscore}{\kern0pt}F{\isacharunderscore}{\kern0pt}meas{\isacharbrackleft}{\kern0pt}intro{\isacharcomma}{\kern0pt}\ simp{\isacharbrackright}{\kern0pt}{\isacharcolon}{\kern0pt}\isanewline
\ \ \isakeyword{fixes}\ f\ {\isacharcolon}{\kern0pt}{\isacharcolon}{\kern0pt}\ {\isachardoublequoteopen}{\isacharprime}{\kern0pt}a\ {\isasymRightarrow}\ {\isacharprime}{\kern0pt}b{\isacharcolon}{\kern0pt}{\isacharcolon}{\kern0pt}{\isacharbraceleft}{\kern0pt}second{\isacharunderscore}{\kern0pt}countable{\isacharunderscore}{\kern0pt}topology{\isacharcomma}{\kern0pt}\ banach{\isacharbraceright}{\kern0pt}{\isachardoublequoteclose}\isanewline
\ \ \isakeyword{assumes}\ {\isachardoublequoteopen}integrable\ M\ f{\isachardoublequoteclose}\isanewline
\ \ \ \ \ \ \ \ \ \ {\isachardoublequoteopen}f\ {\isasymin}\ borel{\isacharunderscore}{\kern0pt}measurable\ F{\isachardoublequoteclose}\isanewline
\ \ \ \ \isakeyword{shows}\ {\isachardoublequoteopen}AE\ x\ in\ M{\isachardot}{\kern0pt}\ cond{\isacharunderscore}{\kern0pt}exp\ M\ F\ f\ x\ {\isacharequal}{\kern0pt}\ f\ x{\isachardoublequoteclose}\isanewline
%
\isadelimproof
\ \ %
\endisadelimproof
%
\isatagproof
\isacommand{by}\isamarkupfalse%
\ {\isacharparenleft}{\kern0pt}rule\ has{\isacharunderscore}{\kern0pt}cond{\isacharunderscore}{\kern0pt}exp{\isacharunderscore}{\kern0pt}charact{\isacharparenleft}{\kern0pt}{\isadigit{2}}{\isacharparenright}{\kern0pt}{\isacharcomma}{\kern0pt}\ auto\ intro{\isacharcolon}{\kern0pt}\ assms{\isacharparenright}{\kern0pt}%
\endisatagproof
{\isafoldproof}%
%
\isadelimproof
%
\endisadelimproof
%
\begin{isamarkuptext}%
Congruence%
\end{isamarkuptext}\isamarkuptrue%
\isacommand{lemma}\isamarkupfalse%
\ has{\isacharunderscore}{\kern0pt}cond{\isacharunderscore}{\kern0pt}exp{\isacharunderscore}{\kern0pt}cong{\isacharcolon}{\kern0pt}\isanewline
\ \ \isakeyword{assumes}\ {\isachardoublequoteopen}integrable\ M\ f{\isachardoublequoteclose}\ {\isachardoublequoteopen}{\isasymAnd}x{\isachardot}{\kern0pt}\ x\ {\isasymin}\ space\ M\ {\isasymLongrightarrow}\ f\ x\ {\isacharequal}{\kern0pt}\ g\ x{\isachardoublequoteclose}\ {\isachardoublequoteopen}has{\isacharunderscore}{\kern0pt}cond{\isacharunderscore}{\kern0pt}exp\ M\ F\ g\ h{\isachardoublequoteclose}\isanewline
\ \ \isakeyword{shows}\ {\isachardoublequoteopen}has{\isacharunderscore}{\kern0pt}cond{\isacharunderscore}{\kern0pt}exp\ M\ F\ f\ h{\isachardoublequoteclose}\isanewline
%
\isadelimproof
%
\endisadelimproof
%
\isatagproof
\isacommand{proof}\isamarkupfalse%
\ {\isacharparenleft}{\kern0pt}intro\ has{\isacharunderscore}{\kern0pt}cond{\isacharunderscore}{\kern0pt}expI{\isacharprime}{\kern0pt}{\isacharbrackleft}{\kern0pt}OF\ {\isacharunderscore}{\kern0pt}\ assms{\isacharparenleft}{\kern0pt}{\isadigit{1}}{\isacharparenright}{\kern0pt}{\isacharbrackright}{\kern0pt}{\isacharcomma}{\kern0pt}\ goal{\isacharunderscore}{\kern0pt}cases{\isacharparenright}{\kern0pt}\isanewline
\ \ \isacommand{case}\isamarkupfalse%
\ {\isacharparenleft}{\kern0pt}{\isadigit{1}}\ A{\isacharparenright}{\kern0pt}\isanewline
\ \ \isacommand{hence}\isamarkupfalse%
\ {\isachardoublequoteopen}set{\isacharunderscore}{\kern0pt}lebesgue{\isacharunderscore}{\kern0pt}integral\ M\ A\ f\ {\isacharequal}{\kern0pt}\ set{\isacharunderscore}{\kern0pt}lebesgue{\isacharunderscore}{\kern0pt}integral\ M\ A\ g{\isachardoublequoteclose}\ \isacommand{by}\isamarkupfalse%
\ {\isacharparenleft}{\kern0pt}intro\ set{\isacharunderscore}{\kern0pt}lebesgue{\isacharunderscore}{\kern0pt}integral{\isacharunderscore}{\kern0pt}cong{\isacharparenright}{\kern0pt}\ {\isacharparenleft}{\kern0pt}meson\ assms{\isacharparenleft}{\kern0pt}{\isadigit{2}}{\isacharparenright}{\kern0pt}\ subalg\ in{\isacharunderscore}{\kern0pt}mono\ subalgebra{\isacharunderscore}{\kern0pt}def\ sets{\isachardot}{\kern0pt}sets{\isacharunderscore}{\kern0pt}into{\isacharunderscore}{\kern0pt}space\ subalgebra{\isacharunderscore}{\kern0pt}def\ subsetD{\isacharparenright}{\kern0pt}{\isacharplus}{\kern0pt}\isanewline
\ \ \isacommand{then}\isamarkupfalse%
\ \isacommand{show}\isamarkupfalse%
\ {\isacharquery}{\kern0pt}case\ \isacommand{using}\isamarkupfalse%
\ {\isadigit{1}}\ assms{\isacharparenleft}{\kern0pt}{\isadigit{3}}{\isacharparenright}{\kern0pt}\ \isacommand{by}\isamarkupfalse%
\ {\isacharparenleft}{\kern0pt}simp\ add{\isacharcolon}{\kern0pt}\ has{\isacharunderscore}{\kern0pt}cond{\isacharunderscore}{\kern0pt}exp{\isacharunderscore}{\kern0pt}def{\isacharparenright}{\kern0pt}\isanewline
\isacommand{qed}\isamarkupfalse%
\ {\isacharparenleft}{\kern0pt}auto\ simp\ add{\isacharcolon}{\kern0pt}\ has{\isacharunderscore}{\kern0pt}cond{\isacharunderscore}{\kern0pt}expD{\isacharbrackleft}{\kern0pt}OF\ assms{\isacharparenleft}{\kern0pt}{\isadigit{3}}{\isacharparenright}{\kern0pt}{\isacharbrackright}{\kern0pt}{\isacharparenright}{\kern0pt}%
\endisatagproof
{\isafoldproof}%
%
\isadelimproof
\isanewline
%
\endisadelimproof
\isanewline
\isacommand{lemma}\isamarkupfalse%
\ cond{\isacharunderscore}{\kern0pt}exp{\isacharunderscore}{\kern0pt}cong{\isacharcolon}{\kern0pt}\isanewline
\ \ \isakeyword{fixes}\ f\ {\isacharcolon}{\kern0pt}{\isacharcolon}{\kern0pt}\ {\isachardoublequoteopen}{\isacharprime}{\kern0pt}a\ {\isasymRightarrow}\ {\isacharprime}{\kern0pt}b{\isacharcolon}{\kern0pt}{\isacharcolon}{\kern0pt}{\isacharbraceleft}{\kern0pt}second{\isacharunderscore}{\kern0pt}countable{\isacharunderscore}{\kern0pt}topology{\isacharcomma}{\kern0pt}banach{\isacharbraceright}{\kern0pt}{\isachardoublequoteclose}\isanewline
\ \ \isakeyword{assumes}\ {\isachardoublequoteopen}integrable\ M\ f{\isachardoublequoteclose}\ {\isachardoublequoteopen}integrable\ M\ g{\isachardoublequoteclose}\ {\isachardoublequoteopen}{\isasymAnd}x{\isachardot}{\kern0pt}\ x\ {\isasymin}\ space\ M\ {\isasymLongrightarrow}\ f\ x\ {\isacharequal}{\kern0pt}\ g\ x{\isachardoublequoteclose}\isanewline
\ \ \isakeyword{shows}\ {\isachardoublequoteopen}AE\ x\ in\ M{\isachardot}{\kern0pt}\ cond{\isacharunderscore}{\kern0pt}exp\ M\ F\ f\ x\ {\isacharequal}{\kern0pt}\ cond{\isacharunderscore}{\kern0pt}exp\ M\ F\ g\ x{\isachardoublequoteclose}\isanewline
%
\isadelimproof
%
\endisadelimproof
%
\isatagproof
\isacommand{proof}\isamarkupfalse%
\ {\isacharparenleft}{\kern0pt}cases\ {\isachardoublequoteopen}{\isasymexists}h{\isachardot}{\kern0pt}\ has{\isacharunderscore}{\kern0pt}cond{\isacharunderscore}{\kern0pt}exp\ M\ F\ f\ h{\isachardoublequoteclose}{\isacharparenright}{\kern0pt}\isanewline
\ \ \isacommand{case}\isamarkupfalse%
\ True\isanewline
\ \ \isacommand{then}\isamarkupfalse%
\ \isacommand{obtain}\isamarkupfalse%
\ h\ \isakeyword{where}\ h{\isacharcolon}{\kern0pt}\ {\isachardoublequoteopen}has{\isacharunderscore}{\kern0pt}cond{\isacharunderscore}{\kern0pt}exp\ M\ F\ f\ h{\isachardoublequoteclose}\ {\isachardoublequoteopen}has{\isacharunderscore}{\kern0pt}cond{\isacharunderscore}{\kern0pt}exp\ M\ F\ g\ h{\isachardoublequoteclose}\ \isacommand{using}\isamarkupfalse%
\ has{\isacharunderscore}{\kern0pt}cond{\isacharunderscore}{\kern0pt}exp{\isacharunderscore}{\kern0pt}cong\ assms\ \isacommand{by}\isamarkupfalse%
\ metis\ \isanewline
\ \ \isacommand{show}\isamarkupfalse%
\ {\isacharquery}{\kern0pt}thesis\ \isacommand{using}\isamarkupfalse%
\ h{\isacharbrackleft}{\kern0pt}THEN\ has{\isacharunderscore}{\kern0pt}cond{\isacharunderscore}{\kern0pt}exp{\isacharunderscore}{\kern0pt}charact{\isacharparenleft}{\kern0pt}{\isadigit{2}}{\isacharparenright}{\kern0pt}{\isacharbrackright}{\kern0pt}\ \isacommand{by}\isamarkupfalse%
\ fastforce\isanewline
\isacommand{next}\isamarkupfalse%
\isanewline
\ \ \isacommand{case}\isamarkupfalse%
\ False\isanewline
\ \ \isacommand{moreover}\isamarkupfalse%
\ \isacommand{have}\isamarkupfalse%
\ {\isachardoublequoteopen}{\isasymnexists}h{\isachardot}{\kern0pt}\ has{\isacharunderscore}{\kern0pt}cond{\isacharunderscore}{\kern0pt}exp\ M\ F\ g\ h{\isachardoublequoteclose}\ \isacommand{using}\isamarkupfalse%
\ False\ has{\isacharunderscore}{\kern0pt}cond{\isacharunderscore}{\kern0pt}exp{\isacharunderscore}{\kern0pt}cong\ assms\ \isacommand{by}\isamarkupfalse%
\ auto\isanewline
\ \ \isacommand{ultimately}\isamarkupfalse%
\ \isacommand{show}\isamarkupfalse%
\ {\isacharquery}{\kern0pt}thesis\ \isacommand{unfolding}\isamarkupfalse%
\ cond{\isacharunderscore}{\kern0pt}exp{\isacharunderscore}{\kern0pt}def\ \isacommand{by}\isamarkupfalse%
\ auto\isanewline
\isacommand{qed}\isamarkupfalse%
%
\endisatagproof
{\isafoldproof}%
%
\isadelimproof
\isanewline
%
\endisadelimproof
\isanewline
\isacommand{lemma}\isamarkupfalse%
\ has{\isacharunderscore}{\kern0pt}cond{\isacharunderscore}{\kern0pt}exp{\isacharunderscore}{\kern0pt}cong{\isacharunderscore}{\kern0pt}AE{\isacharcolon}{\kern0pt}\isanewline
\ \ \isakeyword{assumes}\ {\isachardoublequoteopen}integrable\ M\ f{\isachardoublequoteclose}\ {\isachardoublequoteopen}AE\ x\ in\ M{\isachardot}{\kern0pt}\ f\ x\ {\isacharequal}{\kern0pt}\ g\ x{\isachardoublequoteclose}\ {\isachardoublequoteopen}has{\isacharunderscore}{\kern0pt}cond{\isacharunderscore}{\kern0pt}exp\ M\ F\ g\ h{\isachardoublequoteclose}\isanewline
\ \ \isakeyword{shows}\ {\isachardoublequoteopen}has{\isacharunderscore}{\kern0pt}cond{\isacharunderscore}{\kern0pt}exp\ M\ F\ f\ h{\isachardoublequoteclose}\isanewline
%
\isadelimproof
\ \ %
\endisadelimproof
%
\isatagproof
\isacommand{using}\isamarkupfalse%
\ assms{\isacharparenleft}{\kern0pt}{\isadigit{1}}{\isacharcomma}{\kern0pt}{\isadigit{2}}{\isacharparenright}{\kern0pt}\ subalg\ subalgebra{\isacharunderscore}{\kern0pt}def\ subset{\isacharunderscore}{\kern0pt}iff\ \isanewline
\ \ \isacommand{by}\isamarkupfalse%
\ {\isacharparenleft}{\kern0pt}intro\ has{\isacharunderscore}{\kern0pt}cond{\isacharunderscore}{\kern0pt}expI{\isacharprime}{\kern0pt}{\isacharcomma}{\kern0pt}\ subst\ set{\isacharunderscore}{\kern0pt}lebesgue{\isacharunderscore}{\kern0pt}integral{\isacharunderscore}{\kern0pt}cong{\isacharunderscore}{\kern0pt}AE{\isacharbrackleft}{\kern0pt}OF\ {\isacharunderscore}{\kern0pt}\ assms{\isacharparenleft}{\kern0pt}{\isadigit{1}}{\isacharparenright}{\kern0pt}{\isacharbrackleft}{\kern0pt}THEN\ borel{\isacharunderscore}{\kern0pt}measurable{\isacharunderscore}{\kern0pt}integrable{\isacharbrackright}{\kern0pt}\ borel{\isacharunderscore}{\kern0pt}measurable{\isacharunderscore}{\kern0pt}integrable{\isacharparenleft}{\kern0pt}{\isadigit{1}}{\isacharparenright}{\kern0pt}{\isacharbrackleft}{\kern0pt}OF\ has{\isacharunderscore}{\kern0pt}cond{\isacharunderscore}{\kern0pt}expD{\isacharparenleft}{\kern0pt}{\isadigit{2}}{\isacharparenright}{\kern0pt}{\isacharbrackleft}{\kern0pt}OF\ assms{\isacharparenleft}{\kern0pt}{\isadigit{3}}{\isacharparenright}{\kern0pt}{\isacharbrackright}{\kern0pt}{\isacharbrackright}{\kern0pt}{\isacharbrackright}{\kern0pt}{\isacharparenright}{\kern0pt}\ \isanewline
\ \ \ \ \ {\isacharparenleft}{\kern0pt}fast\ intro{\isacharcolon}{\kern0pt}\ has{\isacharunderscore}{\kern0pt}cond{\isacharunderscore}{\kern0pt}expD{\isacharbrackleft}{\kern0pt}OF\ assms{\isacharparenleft}{\kern0pt}{\isadigit{3}}{\isacharparenright}{\kern0pt}{\isacharbrackright}{\kern0pt}\ integrable{\isacharunderscore}{\kern0pt}cong{\isacharunderscore}{\kern0pt}AE{\isacharunderscore}{\kern0pt}imp{\isacharbrackleft}{\kern0pt}OF\ {\isacharunderscore}{\kern0pt}\ {\isacharunderscore}{\kern0pt}\ AE{\isacharunderscore}{\kern0pt}symmetric{\isacharbrackright}{\kern0pt}{\isacharparenright}{\kern0pt}{\isacharplus}{\kern0pt}%
\endisatagproof
{\isafoldproof}%
%
\isadelimproof
\isanewline
%
\endisadelimproof
\isanewline
\isacommand{lemma}\isamarkupfalse%
\ has{\isacharunderscore}{\kern0pt}cond{\isacharunderscore}{\kern0pt}exp{\isacharunderscore}{\kern0pt}cong{\isacharunderscore}{\kern0pt}AE{\isacharprime}{\kern0pt}{\isacharcolon}{\kern0pt}\isanewline
\ \ \isakeyword{assumes}\ {\isachardoublequoteopen}h\ {\isasymin}\ borel{\isacharunderscore}{\kern0pt}measurable\ F{\isachardoublequoteclose}\ {\isachardoublequoteopen}AE\ x\ in\ M{\isachardot}{\kern0pt}\ h\ x\ {\isacharequal}{\kern0pt}\ h{\isacharprime}{\kern0pt}\ x{\isachardoublequoteclose}\ {\isachardoublequoteopen}has{\isacharunderscore}{\kern0pt}cond{\isacharunderscore}{\kern0pt}exp\ M\ F\ f\ h{\isacharprime}{\kern0pt}{\isachardoublequoteclose}\isanewline
\ \ \isakeyword{shows}\ {\isachardoublequoteopen}has{\isacharunderscore}{\kern0pt}cond{\isacharunderscore}{\kern0pt}exp\ M\ F\ f\ h{\isachardoublequoteclose}\isanewline
%
\isadelimproof
\ \ %
\endisadelimproof
%
\isatagproof
\isacommand{using}\isamarkupfalse%
\ assms{\isacharparenleft}{\kern0pt}{\isadigit{1}}{\isacharcomma}{\kern0pt}\ {\isadigit{2}}{\isacharparenright}{\kern0pt}\ subalg\ subalgebra{\isacharunderscore}{\kern0pt}def\ subset{\isacharunderscore}{\kern0pt}iff\isanewline
\ \ \isacommand{using}\isamarkupfalse%
\ AE{\isacharunderscore}{\kern0pt}restr{\isacharunderscore}{\kern0pt}to{\isacharunderscore}{\kern0pt}subalg{\isadigit{2}}{\isacharbrackleft}{\kern0pt}OF\ subalg\ assms{\isacharparenleft}{\kern0pt}{\isadigit{2}}{\isacharparenright}{\kern0pt}{\isacharbrackright}{\kern0pt}\ measurable{\isacharunderscore}{\kern0pt}from{\isacharunderscore}{\kern0pt}subalg\isanewline
\ \ \isacommand{by}\isamarkupfalse%
\ {\isacharparenleft}{\kern0pt}intro\ has{\isacharunderscore}{\kern0pt}cond{\isacharunderscore}{\kern0pt}expI{\isacharprime}{\kern0pt}\ {\isacharcomma}{\kern0pt}\ subst\ set{\isacharunderscore}{\kern0pt}lebesgue{\isacharunderscore}{\kern0pt}integral{\isacharunderscore}{\kern0pt}cong{\isacharunderscore}{\kern0pt}AE{\isacharbrackleft}{\kern0pt}OF\ {\isacharunderscore}{\kern0pt}\ measurable{\isacharunderscore}{\kern0pt}from{\isacharunderscore}{\kern0pt}subalg{\isacharparenleft}{\kern0pt}{\isadigit{1}}{\isacharcomma}{\kern0pt}{\isadigit{1}}{\isacharparenright}{\kern0pt}{\isacharbrackleft}{\kern0pt}OF\ subalg{\isacharbrackright}{\kern0pt}{\isacharcomma}{\kern0pt}\ OF\ {\isacharunderscore}{\kern0pt}\ assms{\isacharparenleft}{\kern0pt}{\isadigit{1}}{\isacharparenright}{\kern0pt}\ has{\isacharunderscore}{\kern0pt}cond{\isacharunderscore}{\kern0pt}expD{\isacharparenleft}{\kern0pt}{\isadigit{4}}{\isacharparenright}{\kern0pt}{\isacharbrackleft}{\kern0pt}OF\ assms{\isacharparenleft}{\kern0pt}{\isadigit{3}}{\isacharparenright}{\kern0pt}{\isacharbrackright}{\kern0pt}{\isacharbrackright}{\kern0pt}{\isacharparenright}{\kern0pt}\isanewline
\ \ \ \ \ {\isacharparenleft}{\kern0pt}fast\ intro{\isacharcolon}{\kern0pt}\ has{\isacharunderscore}{\kern0pt}cond{\isacharunderscore}{\kern0pt}expD{\isacharbrackleft}{\kern0pt}OF\ assms{\isacharparenleft}{\kern0pt}{\isadigit{3}}{\isacharparenright}{\kern0pt}{\isacharbrackright}{\kern0pt}\ integrable{\isacharunderscore}{\kern0pt}cong{\isacharunderscore}{\kern0pt}AE{\isacharunderscore}{\kern0pt}imp{\isacharbrackleft}{\kern0pt}OF\ {\isacharunderscore}{\kern0pt}\ {\isacharunderscore}{\kern0pt}\ AE{\isacharunderscore}{\kern0pt}symmetric{\isacharbrackright}{\kern0pt}{\isacharparenright}{\kern0pt}{\isacharplus}{\kern0pt}%
\endisatagproof
{\isafoldproof}%
%
\isadelimproof
\isanewline
%
\endisadelimproof
\isanewline
\isacommand{lemma}\isamarkupfalse%
\ cond{\isacharunderscore}{\kern0pt}exp{\isacharunderscore}{\kern0pt}cong{\isacharunderscore}{\kern0pt}AE{\isacharcolon}{\kern0pt}\isanewline
\ \ \isakeyword{fixes}\ f\ {\isacharcolon}{\kern0pt}{\isacharcolon}{\kern0pt}\ {\isachardoublequoteopen}{\isacharprime}{\kern0pt}a\ {\isasymRightarrow}\ {\isacharprime}{\kern0pt}b{\isacharcolon}{\kern0pt}{\isacharcolon}{\kern0pt}{\isacharbraceleft}{\kern0pt}second{\isacharunderscore}{\kern0pt}countable{\isacharunderscore}{\kern0pt}topology{\isacharcomma}{\kern0pt}banach{\isacharbraceright}{\kern0pt}{\isachardoublequoteclose}\isanewline
\ \ \isakeyword{assumes}\ {\isachardoublequoteopen}integrable\ M\ f{\isachardoublequoteclose}\ {\isachardoublequoteopen}integrable\ M\ g{\isachardoublequoteclose}\ {\isachardoublequoteopen}AE\ x\ in\ M{\isachardot}{\kern0pt}\ f\ x\ {\isacharequal}{\kern0pt}\ g\ x{\isachardoublequoteclose}\isanewline
\ \ \isakeyword{shows}\ {\isachardoublequoteopen}AE\ x\ in\ M{\isachardot}{\kern0pt}\ cond{\isacharunderscore}{\kern0pt}exp\ M\ F\ f\ x\ {\isacharequal}{\kern0pt}\ cond{\isacharunderscore}{\kern0pt}exp\ M\ F\ g\ x{\isachardoublequoteclose}\isanewline
%
\isadelimproof
%
\endisadelimproof
%
\isatagproof
\isacommand{proof}\isamarkupfalse%
\ {\isacharparenleft}{\kern0pt}cases\ {\isachardoublequoteopen}{\isasymexists}h{\isachardot}{\kern0pt}\ has{\isacharunderscore}{\kern0pt}cond{\isacharunderscore}{\kern0pt}exp\ M\ F\ f\ h{\isachardoublequoteclose}{\isacharparenright}{\kern0pt}\isanewline
\ \ \isacommand{case}\isamarkupfalse%
\ True\isanewline
\ \ \isacommand{then}\isamarkupfalse%
\ \isacommand{obtain}\isamarkupfalse%
\ h\ \isakeyword{where}\ h{\isacharcolon}{\kern0pt}\ {\isachardoublequoteopen}has{\isacharunderscore}{\kern0pt}cond{\isacharunderscore}{\kern0pt}exp\ M\ F\ f\ h{\isachardoublequoteclose}\ {\isachardoublequoteopen}has{\isacharunderscore}{\kern0pt}cond{\isacharunderscore}{\kern0pt}exp\ M\ F\ g\ h{\isachardoublequoteclose}\ \isacommand{using}\isamarkupfalse%
\ has{\isacharunderscore}{\kern0pt}cond{\isacharunderscore}{\kern0pt}exp{\isacharunderscore}{\kern0pt}cong{\isacharunderscore}{\kern0pt}AE\ assms\ \isacommand{by}\isamarkupfalse%
\ {\isacharparenleft}{\kern0pt}metis\ {\isacharparenleft}{\kern0pt}mono{\isacharunderscore}{\kern0pt}tags{\isacharcomma}{\kern0pt}\ lifting{\isacharparenright}{\kern0pt}\ eventually{\isacharunderscore}{\kern0pt}mono{\isacharparenright}{\kern0pt}\isanewline
\ \ \isacommand{show}\isamarkupfalse%
\ {\isacharquery}{\kern0pt}thesis\ \isacommand{using}\isamarkupfalse%
\ h{\isacharbrackleft}{\kern0pt}THEN\ has{\isacharunderscore}{\kern0pt}cond{\isacharunderscore}{\kern0pt}exp{\isacharunderscore}{\kern0pt}charact{\isacharparenleft}{\kern0pt}{\isadigit{2}}{\isacharparenright}{\kern0pt}{\isacharbrackright}{\kern0pt}\ \isacommand{by}\isamarkupfalse%
\ fastforce\isanewline
\isacommand{next}\isamarkupfalse%
\isanewline
\ \ \isacommand{case}\isamarkupfalse%
\ False\isanewline
\ \ \isacommand{moreover}\isamarkupfalse%
\ \isacommand{have}\isamarkupfalse%
\ {\isachardoublequoteopen}{\isasymnexists}h{\isachardot}{\kern0pt}\ has{\isacharunderscore}{\kern0pt}cond{\isacharunderscore}{\kern0pt}exp\ M\ F\ g\ h{\isachardoublequoteclose}\ \isacommand{using}\isamarkupfalse%
\ False\ has{\isacharunderscore}{\kern0pt}cond{\isacharunderscore}{\kern0pt}exp{\isacharunderscore}{\kern0pt}cong{\isacharunderscore}{\kern0pt}AE\ assms\ \isacommand{by}\isamarkupfalse%
\ auto\isanewline
\ \ \isacommand{ultimately}\isamarkupfalse%
\ \isacommand{show}\isamarkupfalse%
\ {\isacharquery}{\kern0pt}thesis\ \isacommand{unfolding}\isamarkupfalse%
\ cond{\isacharunderscore}{\kern0pt}exp{\isacharunderscore}{\kern0pt}def\ \isacommand{by}\isamarkupfalse%
\ auto\isanewline
\isacommand{qed}\isamarkupfalse%
%
\endisatagproof
{\isafoldproof}%
%
\isadelimproof
\isanewline
%
\endisadelimproof
\isanewline
\isacommand{lemma}\isamarkupfalse%
\ has{\isacharunderscore}{\kern0pt}cond{\isacharunderscore}{\kern0pt}exp{\isacharunderscore}{\kern0pt}real{\isacharbrackleft}{\kern0pt}intro{\isacharbrackright}{\kern0pt}{\isacharcolon}{\kern0pt}\isanewline
\ \ \isakeyword{fixes}\ f\ {\isacharcolon}{\kern0pt}{\isacharcolon}{\kern0pt}\ {\isachardoublequoteopen}{\isacharprime}{\kern0pt}a\ {\isasymRightarrow}\ real{\isachardoublequoteclose}\isanewline
\ \ \isakeyword{assumes}\ {\isachardoublequoteopen}integrable\ M\ f{\isachardoublequoteclose}\isanewline
\ \ \isakeyword{shows}\ {\isachardoublequoteopen}has{\isacharunderscore}{\kern0pt}cond{\isacharunderscore}{\kern0pt}exp\ M\ F\ f\ {\isacharparenleft}{\kern0pt}real{\isacharunderscore}{\kern0pt}cond{\isacharunderscore}{\kern0pt}exp\ M\ F\ f{\isacharparenright}{\kern0pt}{\isachardoublequoteclose}\isanewline
%
\isadelimproof
\ \ %
\endisadelimproof
%
\isatagproof
\isacommand{by}\isamarkupfalse%
\ {\isacharparenleft}{\kern0pt}standard{\isacharcomma}{\kern0pt}\ auto\ intro{\isacharbang}{\kern0pt}{\isacharcolon}{\kern0pt}\ real{\isacharunderscore}{\kern0pt}cond{\isacharunderscore}{\kern0pt}exp{\isacharunderscore}{\kern0pt}intA\ assms{\isacharparenright}{\kern0pt}%
\endisatagproof
{\isafoldproof}%
%
\isadelimproof
\isanewline
%
\endisadelimproof
\isanewline
\isacommand{lemma}\isamarkupfalse%
\ cond{\isacharunderscore}{\kern0pt}exp{\isacharunderscore}{\kern0pt}real{\isacharbrackleft}{\kern0pt}intro{\isacharbrackright}{\kern0pt}{\isacharcolon}{\kern0pt}\isanewline
\ \ \isakeyword{fixes}\ f\ {\isacharcolon}{\kern0pt}{\isacharcolon}{\kern0pt}\ {\isachardoublequoteopen}{\isacharprime}{\kern0pt}a\ {\isasymRightarrow}\ real{\isachardoublequoteclose}\isanewline
\ \ \isakeyword{assumes}\ {\isachardoublequoteopen}integrable\ M\ f{\isachardoublequoteclose}\isanewline
\ \ \isakeyword{shows}\ {\isachardoublequoteopen}AE\ x\ in\ M{\isachardot}{\kern0pt}\ cond{\isacharunderscore}{\kern0pt}exp\ M\ F\ f\ x\ {\isacharequal}{\kern0pt}\ real{\isacharunderscore}{\kern0pt}cond{\isacharunderscore}{\kern0pt}exp\ M\ F\ f\ x{\isachardoublequoteclose}\ \isanewline
%
\isadelimproof
\ \ %
\endisadelimproof
%
\isatagproof
\isacommand{using}\isamarkupfalse%
\ has{\isacharunderscore}{\kern0pt}cond{\isacharunderscore}{\kern0pt}exp{\isacharunderscore}{\kern0pt}charact\ assms\ \isacommand{by}\isamarkupfalse%
\ blast%
\endisatagproof
{\isafoldproof}%
%
\isadelimproof
\isanewline
%
\endisadelimproof
\isanewline
\isacommand{lemma}\isamarkupfalse%
\ cond{\isacharunderscore}{\kern0pt}exp{\isacharunderscore}{\kern0pt}cmult{\isacharcolon}{\kern0pt}\isanewline
\ \ \isakeyword{fixes}\ f\ {\isacharcolon}{\kern0pt}{\isacharcolon}{\kern0pt}\ {\isachardoublequoteopen}{\isacharprime}{\kern0pt}a\ {\isasymRightarrow}\ real{\isachardoublequoteclose}\isanewline
\ \ \isakeyword{assumes}\ {\isachardoublequoteopen}integrable\ M\ f{\isachardoublequoteclose}\isanewline
\ \ \isakeyword{shows}\ {\isachardoublequoteopen}AE\ x\ in\ M{\isachardot}{\kern0pt}\ cond{\isacharunderscore}{\kern0pt}exp\ M\ F\ {\isacharparenleft}{\kern0pt}{\isasymlambda}x{\isachardot}{\kern0pt}\ c\ {\isacharasterisk}{\kern0pt}\ f\ x{\isacharparenright}{\kern0pt}\ x\ {\isacharequal}{\kern0pt}\ c\ {\isacharasterisk}{\kern0pt}\ cond{\isacharunderscore}{\kern0pt}exp\ M\ F\ f\ x{\isachardoublequoteclose}\isanewline
%
\isadelimproof
\ \ %
\endisadelimproof
%
\isatagproof
\isacommand{using}\isamarkupfalse%
\ real{\isacharunderscore}{\kern0pt}cond{\isacharunderscore}{\kern0pt}exp{\isacharunderscore}{\kern0pt}cmult{\isacharbrackleft}{\kern0pt}OF\ assms{\isacharparenleft}{\kern0pt}{\isadigit{1}}{\isacharparenright}{\kern0pt}{\isacharcomma}{\kern0pt}\ of\ c{\isacharbrackright}{\kern0pt}\ assms{\isacharparenleft}{\kern0pt}{\isadigit{1}}{\isacharparenright}{\kern0pt}{\isacharbrackleft}{\kern0pt}THEN\ cond{\isacharunderscore}{\kern0pt}exp{\isacharunderscore}{\kern0pt}real{\isacharbrackright}{\kern0pt}\ assms{\isacharparenleft}{\kern0pt}{\isadigit{1}}{\isacharparenright}{\kern0pt}{\isacharbrackleft}{\kern0pt}THEN\ integrable{\isacharunderscore}{\kern0pt}mult{\isacharunderscore}{\kern0pt}right{\isacharcomma}{\kern0pt}\ THEN\ cond{\isacharunderscore}{\kern0pt}exp{\isacharunderscore}{\kern0pt}real{\isacharcomma}{\kern0pt}\ of\ c{\isacharbrackright}{\kern0pt}\ \isacommand{by}\isamarkupfalse%
\ fastforce%
\endisatagproof
{\isafoldproof}%
%
\isadelimproof
%
\endisadelimproof
%
\begin{isamarkuptext}%
Indicator functions%
\end{isamarkuptext}\isamarkuptrue%
\isacommand{lemma}\isamarkupfalse%
\ has{\isacharunderscore}{\kern0pt}cond{\isacharunderscore}{\kern0pt}exp{\isacharunderscore}{\kern0pt}indicator{\isacharcolon}{\kern0pt}\isanewline
\ \ \isakeyword{assumes}\ {\isachardoublequoteopen}A\ {\isasymin}\ sets\ M{\isachardoublequoteclose}\ {\isachardoublequoteopen}emeasure\ M\ A\ {\isacharless}{\kern0pt}\ {\isasyminfinity}{\isachardoublequoteclose}\isanewline
\ \ \isakeyword{shows}\ {\isachardoublequoteopen}has{\isacharunderscore}{\kern0pt}cond{\isacharunderscore}{\kern0pt}exp\ M\ F\ {\isacharparenleft}{\kern0pt}{\isasymlambda}x{\isachardot}{\kern0pt}\ indicat{\isacharunderscore}{\kern0pt}real\ A\ x\ {\isacharasterisk}{\kern0pt}\isactrlsub R\ y{\isacharparenright}{\kern0pt}\ {\isacharparenleft}{\kern0pt}{\isasymlambda}x{\isachardot}{\kern0pt}\ real{\isacharunderscore}{\kern0pt}cond{\isacharunderscore}{\kern0pt}exp\ M\ F\ {\isacharparenleft}{\kern0pt}indicator\ A{\isacharparenright}{\kern0pt}\ x\ {\isacharasterisk}{\kern0pt}\isactrlsub R\ y{\isacharparenright}{\kern0pt}{\isachardoublequoteclose}\isanewline
%
\isadelimproof
%
\endisadelimproof
%
\isatagproof
\isacommand{proof}\isamarkupfalse%
\ {\isacharparenleft}{\kern0pt}intro\ has{\isacharunderscore}{\kern0pt}cond{\isacharunderscore}{\kern0pt}expI{\isacharprime}{\kern0pt}{\isacharcomma}{\kern0pt}\ goal{\isacharunderscore}{\kern0pt}cases{\isacharparenright}{\kern0pt}\isanewline
\ \ \isacommand{case}\isamarkupfalse%
\ {\isacharparenleft}{\kern0pt}{\isadigit{1}}\ B{\isacharparenright}{\kern0pt}\isanewline
\ \ \isacommand{have}\isamarkupfalse%
\ {\isachardoublequoteopen}{\isasymintegral}x{\isasymin}B{\isachardot}{\kern0pt}\ {\isacharparenleft}{\kern0pt}indicat{\isacharunderscore}{\kern0pt}real\ A\ x\ {\isacharasterisk}{\kern0pt}\isactrlsub R\ y{\isacharparenright}{\kern0pt}\ {\isasympartial}M\ \ {\isacharequal}{\kern0pt}\ {\isacharparenleft}{\kern0pt}{\isasymintegral}x{\isasymin}B{\isachardot}{\kern0pt}\ indicat{\isacharunderscore}{\kern0pt}real\ A\ x\ {\isasympartial}M{\isacharparenright}{\kern0pt}\ {\isacharasterisk}{\kern0pt}\isactrlsub R\ y{\isachardoublequoteclose}\ \isacommand{using}\isamarkupfalse%
\ assms\ \isacommand{by}\isamarkupfalse%
\ {\isacharparenleft}{\kern0pt}intro\ set{\isacharunderscore}{\kern0pt}integral{\isacharunderscore}{\kern0pt}scaleR{\isacharunderscore}{\kern0pt}left{\isacharcomma}{\kern0pt}\ meson\ {\isadigit{1}}\ in{\isacharunderscore}{\kern0pt}mono\ subalg\ subalgebra{\isacharunderscore}{\kern0pt}def{\isacharcomma}{\kern0pt}\ blast{\isacharparenright}{\kern0pt}\isanewline
\ \ \isacommand{also}\isamarkupfalse%
\ \isacommand{have}\isamarkupfalse%
\ {\isachardoublequoteopen}{\isachardot}{\kern0pt}{\isachardot}{\kern0pt}{\isachardot}{\kern0pt}\ {\isacharequal}{\kern0pt}\ {\isacharparenleft}{\kern0pt}{\isasymintegral}x{\isasymin}B{\isachardot}{\kern0pt}\ real{\isacharunderscore}{\kern0pt}cond{\isacharunderscore}{\kern0pt}exp\ M\ F\ {\isacharparenleft}{\kern0pt}indicator\ A{\isacharparenright}{\kern0pt}\ x\ {\isasympartial}M{\isacharparenright}{\kern0pt}\ {\isacharasterisk}{\kern0pt}\isactrlsub R\ y{\isachardoublequoteclose}\ \isacommand{using}\isamarkupfalse%
\ {\isadigit{1}}\ assms\ \isacommand{by}\isamarkupfalse%
\ {\isacharparenleft}{\kern0pt}subst\ real{\isacharunderscore}{\kern0pt}cond{\isacharunderscore}{\kern0pt}exp{\isacharunderscore}{\kern0pt}intA{\isacharcomma}{\kern0pt}\ auto{\isacharparenright}{\kern0pt}\isanewline
\ \ \isacommand{also}\isamarkupfalse%
\ \isacommand{have}\isamarkupfalse%
\ {\isachardoublequoteopen}{\isachardot}{\kern0pt}{\isachardot}{\kern0pt}{\isachardot}{\kern0pt}\ {\isacharequal}{\kern0pt}\ {\isasymintegral}x{\isasymin}B{\isachardot}{\kern0pt}\ {\isacharparenleft}{\kern0pt}real{\isacharunderscore}{\kern0pt}cond{\isacharunderscore}{\kern0pt}exp\ M\ F\ {\isacharparenleft}{\kern0pt}indicator\ A{\isacharparenright}{\kern0pt}\ x\ {\isacharasterisk}{\kern0pt}\isactrlsub R\ y{\isacharparenright}{\kern0pt}\ {\isasympartial}M{\isachardoublequoteclose}\ \isacommand{using}\isamarkupfalse%
\ assms\ \isacommand{by}\isamarkupfalse%
\ {\isacharparenleft}{\kern0pt}intro\ set{\isacharunderscore}{\kern0pt}integral{\isacharunderscore}{\kern0pt}scaleR{\isacharunderscore}{\kern0pt}left{\isacharbrackleft}{\kern0pt}symmetric{\isacharbrackright}{\kern0pt}{\isacharcomma}{\kern0pt}\ meson\ {\isadigit{1}}\ in{\isacharunderscore}{\kern0pt}mono\ subalg\ subalgebra{\isacharunderscore}{\kern0pt}def{\isacharcomma}{\kern0pt}\ blast{\isacharparenright}{\kern0pt}\isanewline
\ \ \isacommand{finally}\isamarkupfalse%
\ \isacommand{show}\isamarkupfalse%
\ {\isacharquery}{\kern0pt}case\ \isacommand{{\isachardot}{\kern0pt}}\isamarkupfalse%
\isanewline
\isacommand{next}\isamarkupfalse%
\isanewline
\ \ \isacommand{case}\isamarkupfalse%
\ {\isadigit{2}}\isanewline
\ \ \isacommand{then}\isamarkupfalse%
\ \isacommand{show}\isamarkupfalse%
\ {\isacharquery}{\kern0pt}case\ \isacommand{using}\isamarkupfalse%
\ integrable{\isacharunderscore}{\kern0pt}scaleR{\isacharunderscore}{\kern0pt}left\ integrable{\isacharunderscore}{\kern0pt}real{\isacharunderscore}{\kern0pt}indicator\ assms\ \isacommand{by}\isamarkupfalse%
\ blast\isanewline
\isacommand{next}\isamarkupfalse%
\isanewline
\ \ \isacommand{case}\isamarkupfalse%
\ {\isadigit{3}}\isanewline
\ \ \isacommand{show}\isamarkupfalse%
\ {\isacharquery}{\kern0pt}case\ \isacommand{using}\isamarkupfalse%
\ assms\ \isacommand{by}\isamarkupfalse%
\ {\isacharparenleft}{\kern0pt}intro\ integrable{\isacharunderscore}{\kern0pt}scaleR{\isacharunderscore}{\kern0pt}left{\isacharcomma}{\kern0pt}\ intro\ real{\isacharunderscore}{\kern0pt}cond{\isacharunderscore}{\kern0pt}exp{\isacharunderscore}{\kern0pt}int{\isacharcomma}{\kern0pt}\ blast{\isacharplus}{\kern0pt}{\isacharparenright}{\kern0pt}\isanewline
\isacommand{next}\isamarkupfalse%
\isanewline
\ \ \isacommand{case}\isamarkupfalse%
\ {\isadigit{4}}\isanewline
\ \ \isacommand{then}\isamarkupfalse%
\ \isacommand{show}\isamarkupfalse%
\ {\isacharquery}{\kern0pt}case\ \isacommand{by}\isamarkupfalse%
\ {\isacharparenleft}{\kern0pt}intro\ borel{\isacharunderscore}{\kern0pt}measurable{\isacharunderscore}{\kern0pt}scaleR{\isacharcomma}{\kern0pt}\ intro\ Conditional{\isacharunderscore}{\kern0pt}Expectation{\isachardot}{\kern0pt}borel{\isacharunderscore}{\kern0pt}measurable{\isacharunderscore}{\kern0pt}cond{\isacharunderscore}{\kern0pt}exp{\isacharcomma}{\kern0pt}\ simp{\isacharparenright}{\kern0pt}\isanewline
\isacommand{qed}\isamarkupfalse%
%
\endisatagproof
{\isafoldproof}%
%
\isadelimproof
\isanewline
%
\endisadelimproof
\isanewline
\isacommand{lemma}\isamarkupfalse%
\ cond{\isacharunderscore}{\kern0pt}exp{\isacharunderscore}{\kern0pt}indicator{\isacharbrackleft}{\kern0pt}intro{\isacharbrackright}{\kern0pt}{\isacharcolon}{\kern0pt}\isanewline
\ \ \isakeyword{fixes}\ y\ {\isacharcolon}{\kern0pt}{\isacharcolon}{\kern0pt}\ {\isachardoublequoteopen}{\isacharprime}{\kern0pt}b{\isacharcolon}{\kern0pt}{\isacharcolon}{\kern0pt}{\isacharbraceleft}{\kern0pt}second{\isacharunderscore}{\kern0pt}countable{\isacharunderscore}{\kern0pt}topology{\isacharcomma}{\kern0pt}banach{\isacharbraceright}{\kern0pt}{\isachardoublequoteclose}\isanewline
\ \ \isakeyword{assumes}\ {\isacharbrackleft}{\kern0pt}measurable{\isacharbrackright}{\kern0pt}{\isacharcolon}{\kern0pt}\ {\isachardoublequoteopen}A\ {\isasymin}\ sets\ M{\isachardoublequoteclose}\ {\isachardoublequoteopen}emeasure\ M\ A\ {\isacharless}{\kern0pt}\ {\isasyminfinity}{\isachardoublequoteclose}\isanewline
\ \ \isakeyword{shows}\ {\isachardoublequoteopen}AE\ x\ in\ M{\isachardot}{\kern0pt}\ cond{\isacharunderscore}{\kern0pt}exp\ M\ F\ {\isacharparenleft}{\kern0pt}{\isasymlambda}x{\isachardot}{\kern0pt}\ indicat{\isacharunderscore}{\kern0pt}real\ A\ x\ {\isacharasterisk}{\kern0pt}\isactrlsub R\ y{\isacharparenright}{\kern0pt}\ x\ {\isacharequal}{\kern0pt}\ cond{\isacharunderscore}{\kern0pt}exp\ M\ F\ {\isacharparenleft}{\kern0pt}indicator\ A{\isacharparenright}{\kern0pt}\ x\ {\isacharasterisk}{\kern0pt}\isactrlsub R\ y{\isachardoublequoteclose}\isanewline
%
\isadelimproof
%
\endisadelimproof
%
\isatagproof
\isacommand{proof}\isamarkupfalse%
\ {\isacharminus}{\kern0pt}\isanewline
\ \ \isacommand{have}\isamarkupfalse%
\ {\isachardoublequoteopen}AE\ x\ in\ M{\isachardot}{\kern0pt}\ cond{\isacharunderscore}{\kern0pt}exp\ M\ F\ {\isacharparenleft}{\kern0pt}{\isasymlambda}x{\isachardot}{\kern0pt}\ indicat{\isacharunderscore}{\kern0pt}real\ A\ x\ {\isacharasterisk}{\kern0pt}\isactrlsub R\ y{\isacharparenright}{\kern0pt}\ x\ {\isacharequal}{\kern0pt}\ real{\isacharunderscore}{\kern0pt}cond{\isacharunderscore}{\kern0pt}exp\ M\ F\ {\isacharparenleft}{\kern0pt}indicator\ A{\isacharparenright}{\kern0pt}\ x\ {\isacharasterisk}{\kern0pt}\isactrlsub R\ y{\isachardoublequoteclose}\ \isacommand{using}\isamarkupfalse%
\ has{\isacharunderscore}{\kern0pt}cond{\isacharunderscore}{\kern0pt}exp{\isacharunderscore}{\kern0pt}indicator{\isacharbrackleft}{\kern0pt}OF\ assms{\isacharbrackright}{\kern0pt}\ has{\isacharunderscore}{\kern0pt}cond{\isacharunderscore}{\kern0pt}exp{\isacharunderscore}{\kern0pt}charact\ \isacommand{by}\isamarkupfalse%
\ blast\isanewline
\ \ \isacommand{thus}\isamarkupfalse%
\ {\isacharquery}{\kern0pt}thesis\ \isacommand{using}\isamarkupfalse%
\ cond{\isacharunderscore}{\kern0pt}exp{\isacharunderscore}{\kern0pt}real{\isacharbrackleft}{\kern0pt}OF\ integrable{\isacharunderscore}{\kern0pt}real{\isacharunderscore}{\kern0pt}indicator{\isacharcomma}{\kern0pt}\ OF\ assms{\isacharbrackright}{\kern0pt}\ \isacommand{by}\isamarkupfalse%
\ fastforce\isanewline
\isacommand{qed}\isamarkupfalse%
%
\endisatagproof
{\isafoldproof}%
%
\isadelimproof
%
\endisadelimproof
%
\begin{isamarkuptext}%
Addition%
\end{isamarkuptext}\isamarkuptrue%
\isacommand{lemma}\isamarkupfalse%
\ has{\isacharunderscore}{\kern0pt}cond{\isacharunderscore}{\kern0pt}exp{\isacharunderscore}{\kern0pt}add{\isacharcolon}{\kern0pt}\isanewline
\ \ \isakeyword{fixes}\ f\ g\ {\isacharcolon}{\kern0pt}{\isacharcolon}{\kern0pt}\ {\isachardoublequoteopen}{\isacharprime}{\kern0pt}a\ {\isasymRightarrow}\ {\isacharprime}{\kern0pt}b{\isacharcolon}{\kern0pt}{\isacharcolon}{\kern0pt}{\isacharbraceleft}{\kern0pt}second{\isacharunderscore}{\kern0pt}countable{\isacharunderscore}{\kern0pt}topology{\isacharcomma}{\kern0pt}banach{\isacharbraceright}{\kern0pt}{\isachardoublequoteclose}\isanewline
\ \ \isakeyword{assumes}\ {\isachardoublequoteopen}has{\isacharunderscore}{\kern0pt}cond{\isacharunderscore}{\kern0pt}exp\ M\ F\ f\ f{\isacharprime}{\kern0pt}{\isachardoublequoteclose}\ {\isachardoublequoteopen}has{\isacharunderscore}{\kern0pt}cond{\isacharunderscore}{\kern0pt}exp\ M\ F\ g\ g{\isacharprime}{\kern0pt}{\isachardoublequoteclose}\isanewline
\ \ \isakeyword{shows}\ {\isachardoublequoteopen}has{\isacharunderscore}{\kern0pt}cond{\isacharunderscore}{\kern0pt}exp\ M\ F\ {\isacharparenleft}{\kern0pt}{\isasymlambda}x{\isachardot}{\kern0pt}\ f\ x\ {\isacharplus}{\kern0pt}\ g\ x{\isacharparenright}{\kern0pt}\ {\isacharparenleft}{\kern0pt}{\isasymlambda}x{\isachardot}{\kern0pt}\ f{\isacharprime}{\kern0pt}\ x\ {\isacharplus}{\kern0pt}\ g{\isacharprime}{\kern0pt}\ x{\isacharparenright}{\kern0pt}{\isachardoublequoteclose}\isanewline
%
\isadelimproof
%
\endisadelimproof
%
\isatagproof
\isacommand{proof}\isamarkupfalse%
\ {\isacharparenleft}{\kern0pt}intro\ has{\isacharunderscore}{\kern0pt}cond{\isacharunderscore}{\kern0pt}expI{\isacharprime}{\kern0pt}{\isacharcomma}{\kern0pt}\ goal{\isacharunderscore}{\kern0pt}cases{\isacharparenright}{\kern0pt}\isanewline
\ \ \isacommand{case}\isamarkupfalse%
\ {\isacharparenleft}{\kern0pt}{\isadigit{1}}\ A{\isacharparenright}{\kern0pt}\isanewline
\ \ \isacommand{have}\isamarkupfalse%
\ {\isachardoublequoteopen}{\isasymintegral}x{\isasymin}A{\isachardot}{\kern0pt}\ {\isacharparenleft}{\kern0pt}f\ x\ {\isacharplus}{\kern0pt}\ g\ x{\isacharparenright}{\kern0pt}{\isasympartial}M\ {\isacharequal}{\kern0pt}\ {\isacharparenleft}{\kern0pt}{\isasymintegral}x{\isasymin}A{\isachardot}{\kern0pt}\ f\ x\ {\isasympartial}M{\isacharparenright}{\kern0pt}\ {\isacharplus}{\kern0pt}\ {\isacharparenleft}{\kern0pt}{\isasymintegral}x{\isasymin}A{\isachardot}{\kern0pt}\ g\ x\ {\isasympartial}M{\isacharparenright}{\kern0pt}{\isachardoublequoteclose}\ \isacommand{using}\isamarkupfalse%
\ assms{\isacharbrackleft}{\kern0pt}THEN\ has{\isacharunderscore}{\kern0pt}cond{\isacharunderscore}{\kern0pt}expD{\isacharparenleft}{\kern0pt}{\isadigit{2}}{\isacharparenright}{\kern0pt}{\isacharbrackright}{\kern0pt}\ subalg\ {\isadigit{1}}\ \isacommand{by}\isamarkupfalse%
\ {\isacharparenleft}{\kern0pt}intro\ set{\isacharunderscore}{\kern0pt}integral{\isacharunderscore}{\kern0pt}add{\isacharparenleft}{\kern0pt}{\isadigit{2}}{\isacharparenright}{\kern0pt}{\isacharcomma}{\kern0pt}\ auto\ simp\ add{\isacharcolon}{\kern0pt}\ subalgebra{\isacharunderscore}{\kern0pt}def\ set{\isacharunderscore}{\kern0pt}integrable{\isacharunderscore}{\kern0pt}def\ intro{\isacharcolon}{\kern0pt}\ integrable{\isacharunderscore}{\kern0pt}mult{\isacharunderscore}{\kern0pt}indicator{\isacharparenright}{\kern0pt}\isanewline
\ \ \isacommand{also}\isamarkupfalse%
\ \isacommand{have}\isamarkupfalse%
\ {\isachardoublequoteopen}{\isachardot}{\kern0pt}{\isachardot}{\kern0pt}{\isachardot}{\kern0pt}\ {\isacharequal}{\kern0pt}\ {\isacharparenleft}{\kern0pt}{\isasymintegral}x{\isasymin}A{\isachardot}{\kern0pt}\ f{\isacharprime}{\kern0pt}\ x\ {\isasympartial}M{\isacharparenright}{\kern0pt}\ {\isacharplus}{\kern0pt}\ {\isacharparenleft}{\kern0pt}{\isasymintegral}x{\isasymin}A{\isachardot}{\kern0pt}\ g{\isacharprime}{\kern0pt}\ x\ {\isasympartial}M{\isacharparenright}{\kern0pt}{\isachardoublequoteclose}\ \isacommand{using}\isamarkupfalse%
\ assms{\isacharbrackleft}{\kern0pt}THEN\ has{\isacharunderscore}{\kern0pt}cond{\isacharunderscore}{\kern0pt}expD{\isacharparenleft}{\kern0pt}{\isadigit{1}}{\isacharparenright}{\kern0pt}{\isacharbrackleft}{\kern0pt}OF\ {\isacharunderscore}{\kern0pt}\ {\isadigit{1}}{\isacharbrackright}{\kern0pt}{\isacharbrackright}{\kern0pt}\ \isacommand{by}\isamarkupfalse%
\ argo\isanewline
\ \ \isacommand{also}\isamarkupfalse%
\ \isacommand{have}\isamarkupfalse%
\ {\isachardoublequoteopen}{\isachardot}{\kern0pt}{\isachardot}{\kern0pt}{\isachardot}{\kern0pt}\ {\isacharequal}{\kern0pt}\ {\isasymintegral}x{\isasymin}A{\isachardot}{\kern0pt}\ {\isacharparenleft}{\kern0pt}f{\isacharprime}{\kern0pt}\ x\ {\isacharplus}{\kern0pt}\ g{\isacharprime}{\kern0pt}\ x{\isacharparenright}{\kern0pt}{\isasympartial}M{\isachardoublequoteclose}\ \isacommand{using}\isamarkupfalse%
\ assms{\isacharbrackleft}{\kern0pt}THEN\ has{\isacharunderscore}{\kern0pt}cond{\isacharunderscore}{\kern0pt}expD{\isacharparenleft}{\kern0pt}{\isadigit{3}}{\isacharparenright}{\kern0pt}{\isacharbrackright}{\kern0pt}\ subalg\ {\isadigit{1}}\ \isacommand{by}\isamarkupfalse%
\ {\isacharparenleft}{\kern0pt}intro\ set{\isacharunderscore}{\kern0pt}integral{\isacharunderscore}{\kern0pt}add{\isacharparenleft}{\kern0pt}{\isadigit{2}}{\isacharparenright}{\kern0pt}{\isacharbrackleft}{\kern0pt}symmetric{\isacharbrackright}{\kern0pt}{\isacharcomma}{\kern0pt}\ auto\ simp\ add{\isacharcolon}{\kern0pt}\ subalgebra{\isacharunderscore}{\kern0pt}def\ set{\isacharunderscore}{\kern0pt}integrable{\isacharunderscore}{\kern0pt}def\ intro{\isacharcolon}{\kern0pt}\ integrable{\isacharunderscore}{\kern0pt}mult{\isacharunderscore}{\kern0pt}indicator{\isacharparenright}{\kern0pt}\isanewline
\ \ \isacommand{finally}\isamarkupfalse%
\ \isacommand{show}\isamarkupfalse%
\ {\isacharquery}{\kern0pt}case\ \isacommand{{\isachardot}{\kern0pt}}\isamarkupfalse%
\isanewline
\isacommand{next}\isamarkupfalse%
\isanewline
\ \ \isacommand{case}\isamarkupfalse%
\ {\isadigit{2}}\isanewline
\ \ \isacommand{then}\isamarkupfalse%
\ \isacommand{show}\isamarkupfalse%
\ {\isacharquery}{\kern0pt}case\ \isacommand{by}\isamarkupfalse%
\ {\isacharparenleft}{\kern0pt}metis\ Bochner{\isacharunderscore}{\kern0pt}Integration{\isachardot}{\kern0pt}integrable{\isacharunderscore}{\kern0pt}add\ assms\ has{\isacharunderscore}{\kern0pt}cond{\isacharunderscore}{\kern0pt}expD{\isacharparenleft}{\kern0pt}{\isadigit{2}}{\isacharparenright}{\kern0pt}{\isacharparenright}{\kern0pt}\isanewline
\isacommand{next}\isamarkupfalse%
\isanewline
\ \ \isacommand{case}\isamarkupfalse%
\ {\isadigit{3}}\isanewline
\ \ \isacommand{then}\isamarkupfalse%
\ \isacommand{show}\isamarkupfalse%
\ {\isacharquery}{\kern0pt}case\ \isacommand{by}\isamarkupfalse%
\ {\isacharparenleft}{\kern0pt}metis\ Bochner{\isacharunderscore}{\kern0pt}Integration{\isachardot}{\kern0pt}integrable{\isacharunderscore}{\kern0pt}add\ assms\ has{\isacharunderscore}{\kern0pt}cond{\isacharunderscore}{\kern0pt}expD{\isacharparenleft}{\kern0pt}{\isadigit{3}}{\isacharparenright}{\kern0pt}{\isacharparenright}{\kern0pt}\isanewline
\isacommand{next}\isamarkupfalse%
\isanewline
\ \ \isacommand{case}\isamarkupfalse%
\ {\isadigit{4}}\isanewline
\ \ \isacommand{then}\isamarkupfalse%
\ \isacommand{show}\isamarkupfalse%
\ {\isacharquery}{\kern0pt}case\ \isacommand{using}\isamarkupfalse%
\ assms\ borel{\isacharunderscore}{\kern0pt}measurable{\isacharunderscore}{\kern0pt}add\ has{\isacharunderscore}{\kern0pt}cond{\isacharunderscore}{\kern0pt}expD{\isacharparenleft}{\kern0pt}{\isadigit{4}}{\isacharparenright}{\kern0pt}\ \isacommand{by}\isamarkupfalse%
\ blast\isanewline
\isacommand{qed}\isamarkupfalse%
%
\endisatagproof
{\isafoldproof}%
%
\isadelimproof
\isanewline
%
\endisadelimproof
\isanewline
\isacommand{lemma}\isamarkupfalse%
\ has{\isacharunderscore}{\kern0pt}cond{\isacharunderscore}{\kern0pt}exp{\isacharunderscore}{\kern0pt}scaleR{\isacharunderscore}{\kern0pt}right{\isacharcolon}{\kern0pt}\isanewline
\ \ \isakeyword{fixes}\ f\ {\isacharcolon}{\kern0pt}{\isacharcolon}{\kern0pt}\ {\isachardoublequoteopen}{\isacharprime}{\kern0pt}a\ {\isasymRightarrow}\ {\isacharprime}{\kern0pt}b{\isacharcolon}{\kern0pt}{\isacharcolon}{\kern0pt}{\isacharbraceleft}{\kern0pt}second{\isacharunderscore}{\kern0pt}countable{\isacharunderscore}{\kern0pt}topology{\isacharcomma}{\kern0pt}banach{\isacharbraceright}{\kern0pt}{\isachardoublequoteclose}\isanewline
\ \ \isakeyword{assumes}\ {\isachardoublequoteopen}has{\isacharunderscore}{\kern0pt}cond{\isacharunderscore}{\kern0pt}exp\ M\ F\ f\ f{\isacharprime}{\kern0pt}{\isachardoublequoteclose}\isanewline
\ \ \isakeyword{shows}\ {\isachardoublequoteopen}has{\isacharunderscore}{\kern0pt}cond{\isacharunderscore}{\kern0pt}exp\ M\ F\ {\isacharparenleft}{\kern0pt}{\isasymlambda}x{\isachardot}{\kern0pt}\ c\ {\isacharasterisk}{\kern0pt}\isactrlsub R\ f\ x{\isacharparenright}{\kern0pt}\ {\isacharparenleft}{\kern0pt}{\isasymlambda}x{\isachardot}{\kern0pt}\ c\ {\isacharasterisk}{\kern0pt}\isactrlsub R\ f{\isacharprime}{\kern0pt}\ x{\isacharparenright}{\kern0pt}{\isachardoublequoteclose}\isanewline
%
\isadelimproof
\ \ %
\endisadelimproof
%
\isatagproof
\isacommand{using}\isamarkupfalse%
\ has{\isacharunderscore}{\kern0pt}cond{\isacharunderscore}{\kern0pt}expD{\isacharbrackleft}{\kern0pt}OF\ assms{\isacharbrackright}{\kern0pt}\ \isacommand{by}\isamarkupfalse%
\ {\isacharparenleft}{\kern0pt}intro\ has{\isacharunderscore}{\kern0pt}cond{\isacharunderscore}{\kern0pt}expI{\isacharprime}{\kern0pt}{\isacharcomma}{\kern0pt}\ auto{\isacharparenright}{\kern0pt}%
\endisatagproof
{\isafoldproof}%
%
\isadelimproof
\isanewline
%
\endisadelimproof
\isanewline
\isacommand{lemma}\isamarkupfalse%
\ cond{\isacharunderscore}{\kern0pt}exp{\isacharunderscore}{\kern0pt}scaleR{\isacharunderscore}{\kern0pt}right{\isacharcolon}{\kern0pt}\isanewline
\ \ \isakeyword{fixes}\ f\ {\isacharcolon}{\kern0pt}{\isacharcolon}{\kern0pt}\ {\isachardoublequoteopen}{\isacharprime}{\kern0pt}a\ {\isasymRightarrow}\ {\isacharprime}{\kern0pt}b{\isacharcolon}{\kern0pt}{\isacharcolon}{\kern0pt}{\isacharbraceleft}{\kern0pt}second{\isacharunderscore}{\kern0pt}countable{\isacharunderscore}{\kern0pt}topology{\isacharcomma}{\kern0pt}banach{\isacharbraceright}{\kern0pt}{\isachardoublequoteclose}\isanewline
\ \ \isakeyword{assumes}\ {\isachardoublequoteopen}integrable\ M\ f{\isachardoublequoteclose}\isanewline
\ \ \isakeyword{shows}\ {\isachardoublequoteopen}AE\ x\ in\ M{\isachardot}{\kern0pt}\ cond{\isacharunderscore}{\kern0pt}exp\ M\ F\ {\isacharparenleft}{\kern0pt}{\isasymlambda}x{\isachardot}{\kern0pt}\ c\ {\isacharasterisk}{\kern0pt}\isactrlsub R\ f\ x{\isacharparenright}{\kern0pt}\ x\ {\isacharequal}{\kern0pt}\ c\ {\isacharasterisk}{\kern0pt}\isactrlsub R\ cond{\isacharunderscore}{\kern0pt}exp\ M\ F\ f\ x{\isachardoublequoteclose}\isanewline
%
\isadelimproof
%
\endisadelimproof
%
\isatagproof
\isacommand{proof}\isamarkupfalse%
\ {\isacharparenleft}{\kern0pt}cases\ {\isachardoublequoteopen}{\isasymexists}f{\isacharprime}{\kern0pt}{\isachardot}{\kern0pt}\ has{\isacharunderscore}{\kern0pt}cond{\isacharunderscore}{\kern0pt}exp\ M\ F\ f\ f{\isacharprime}{\kern0pt}{\isachardoublequoteclose}{\isacharparenright}{\kern0pt}\isanewline
\ \ \isacommand{case}\isamarkupfalse%
\ True\isanewline
\ \ \isacommand{then}\isamarkupfalse%
\ \isacommand{show}\isamarkupfalse%
\ {\isacharquery}{\kern0pt}thesis\ \isacommand{using}\isamarkupfalse%
\ assms\ has{\isacharunderscore}{\kern0pt}cond{\isacharunderscore}{\kern0pt}exp{\isacharunderscore}{\kern0pt}charact\ has{\isacharunderscore}{\kern0pt}cond{\isacharunderscore}{\kern0pt}exp{\isacharunderscore}{\kern0pt}scaleR{\isacharunderscore}{\kern0pt}right\ \isacommand{by}\isamarkupfalse%
\ metis\isanewline
\isacommand{next}\isamarkupfalse%
\isanewline
\ \ \isacommand{case}\isamarkupfalse%
\ False\isanewline
\ \ \isacommand{show}\isamarkupfalse%
\ {\isacharquery}{\kern0pt}thesis\isanewline
\ \ \isacommand{proof}\isamarkupfalse%
\ {\isacharparenleft}{\kern0pt}cases\ {\isachardoublequoteopen}c\ {\isacharequal}{\kern0pt}\ {\isadigit{0}}{\isachardoublequoteclose}{\isacharparenright}{\kern0pt}\isanewline
\ \ \ \ \isacommand{case}\isamarkupfalse%
\ True\isanewline
\ \ \ \ \isacommand{then}\isamarkupfalse%
\ \isacommand{show}\isamarkupfalse%
\ {\isacharquery}{\kern0pt}thesis\ \isacommand{by}\isamarkupfalse%
\ simp\isanewline
\ \ \isacommand{next}\isamarkupfalse%
\isanewline
\ \ \ \ \isacommand{case}\isamarkupfalse%
\ c{\isacharunderscore}{\kern0pt}nonzero{\isacharcolon}{\kern0pt}\ False\isanewline
\ \ \ \ \isacommand{have}\isamarkupfalse%
\ {\isachardoublequoteopen}{\isasymnexists}f{\isacharprime}{\kern0pt}{\isachardot}{\kern0pt}\ has{\isacharunderscore}{\kern0pt}cond{\isacharunderscore}{\kern0pt}exp\ M\ F\ {\isacharparenleft}{\kern0pt}{\isasymlambda}x{\isachardot}{\kern0pt}\ c\ {\isacharasterisk}{\kern0pt}\isactrlsub R\ f\ x{\isacharparenright}{\kern0pt}\ f{\isacharprime}{\kern0pt}{\isachardoublequoteclose}\isanewline
\ \ \ \ \isacommand{proof}\isamarkupfalse%
\ {\isacharparenleft}{\kern0pt}standard{\isacharcomma}{\kern0pt}\ goal{\isacharunderscore}{\kern0pt}cases{\isacharparenright}{\kern0pt}\isanewline
\ \ \ \ \ \ \isacommand{case}\isamarkupfalse%
\ {\isadigit{1}}\isanewline
\ \ \ \ \ \ \isacommand{then}\isamarkupfalse%
\ \isacommand{obtain}\isamarkupfalse%
\ f{\isacharprime}{\kern0pt}\ \isakeyword{where}\ f{\isacharprime}{\kern0pt}{\isacharcolon}{\kern0pt}\ {\isachardoublequoteopen}has{\isacharunderscore}{\kern0pt}cond{\isacharunderscore}{\kern0pt}exp\ M\ F\ {\isacharparenleft}{\kern0pt}{\isasymlambda}x{\isachardot}{\kern0pt}\ c\ {\isacharasterisk}{\kern0pt}\isactrlsub R\ f\ x{\isacharparenright}{\kern0pt}\ f{\isacharprime}{\kern0pt}{\isachardoublequoteclose}\ \isacommand{by}\isamarkupfalse%
\ blast\isanewline
\ \ \ \ \ \ \isacommand{have}\isamarkupfalse%
\ {\isachardoublequoteopen}has{\isacharunderscore}{\kern0pt}cond{\isacharunderscore}{\kern0pt}exp\ M\ F\ f\ {\isacharparenleft}{\kern0pt}{\isasymlambda}x{\isachardot}{\kern0pt}\ inverse\ c\ {\isacharasterisk}{\kern0pt}\isactrlsub R\ f{\isacharprime}{\kern0pt}\ x{\isacharparenright}{\kern0pt}{\isachardoublequoteclose}\ \isacommand{using}\isamarkupfalse%
\ has{\isacharunderscore}{\kern0pt}cond{\isacharunderscore}{\kern0pt}expD{\isacharbrackleft}{\kern0pt}OF\ f{\isacharprime}{\kern0pt}{\isacharbrackright}{\kern0pt}\ divideR{\isacharunderscore}{\kern0pt}right{\isacharbrackleft}{\kern0pt}OF\ c{\isacharunderscore}{\kern0pt}nonzero{\isacharbrackright}{\kern0pt}\ assms\ \isacommand{by}\isamarkupfalse%
\ {\isacharparenleft}{\kern0pt}intro\ has{\isacharunderscore}{\kern0pt}cond{\isacharunderscore}{\kern0pt}expI{\isacharprime}{\kern0pt}{\isacharcomma}{\kern0pt}\ auto{\isacharparenright}{\kern0pt}\isanewline
\ \ \ \ \ \ \isacommand{then}\isamarkupfalse%
\ \isacommand{show}\isamarkupfalse%
\ {\isacharquery}{\kern0pt}case\ \isacommand{using}\isamarkupfalse%
\ False\ \isacommand{by}\isamarkupfalse%
\ blast\isanewline
\ \ \ \ \isacommand{qed}\isamarkupfalse%
\isanewline
\ \ \ \ \isacommand{then}\isamarkupfalse%
\ \isacommand{show}\isamarkupfalse%
\ {\isacharquery}{\kern0pt}thesis\ \isacommand{using}\isamarkupfalse%
\ cond{\isacharunderscore}{\kern0pt}exp{\isacharunderscore}{\kern0pt}null{\isacharbrackleft}{\kern0pt}OF\ False{\isacharbrackright}{\kern0pt}\ cond{\isacharunderscore}{\kern0pt}exp{\isacharunderscore}{\kern0pt}null\ \isacommand{by}\isamarkupfalse%
\ force\isanewline
\ \ \isacommand{qed}\isamarkupfalse%
\ \isanewline
\isacommand{qed}\isamarkupfalse%
%
\endisatagproof
{\isafoldproof}%
%
\isadelimproof
\isanewline
%
\endisadelimproof
\isanewline
\isacommand{lemma}\isamarkupfalse%
\ cond{\isacharunderscore}{\kern0pt}exp{\isacharunderscore}{\kern0pt}uminus{\isacharcolon}{\kern0pt}\isanewline
\ \ \isakeyword{fixes}\ f\ {\isacharcolon}{\kern0pt}{\isacharcolon}{\kern0pt}\ {\isachardoublequoteopen}{\isacharprime}{\kern0pt}a\ {\isasymRightarrow}\ {\isacharprime}{\kern0pt}b{\isacharcolon}{\kern0pt}{\isacharcolon}{\kern0pt}{\isacharbraceleft}{\kern0pt}second{\isacharunderscore}{\kern0pt}countable{\isacharunderscore}{\kern0pt}topology{\isacharcomma}{\kern0pt}banach{\isacharbraceright}{\kern0pt}{\isachardoublequoteclose}\isanewline
\ \ \isakeyword{assumes}\ {\isachardoublequoteopen}integrable\ M\ f{\isachardoublequoteclose}\isanewline
\ \ \isakeyword{shows}\ {\isachardoublequoteopen}AE\ x\ in\ M{\isachardot}{\kern0pt}\ cond{\isacharunderscore}{\kern0pt}exp\ M\ F\ {\isacharparenleft}{\kern0pt}{\isasymlambda}x{\isachardot}{\kern0pt}\ {\isacharminus}{\kern0pt}\ f\ x{\isacharparenright}{\kern0pt}\ x\ {\isacharequal}{\kern0pt}\ {\isacharminus}{\kern0pt}\ cond{\isacharunderscore}{\kern0pt}exp\ M\ F\ f\ x{\isachardoublequoteclose}\isanewline
%
\isadelimproof
\ \ %
\endisadelimproof
%
\isatagproof
\isacommand{using}\isamarkupfalse%
\ cond{\isacharunderscore}{\kern0pt}exp{\isacharunderscore}{\kern0pt}scaleR{\isacharunderscore}{\kern0pt}right{\isacharbrackleft}{\kern0pt}OF\ assms{\isacharcomma}{\kern0pt}\ of\ {\isachardoublequoteopen}{\isacharminus}{\kern0pt}{\isadigit{1}}{\isachardoublequoteclose}{\isacharbrackright}{\kern0pt}\ \isacommand{by}\isamarkupfalse%
\ force%
\endisatagproof
{\isafoldproof}%
%
\isadelimproof
\isanewline
%
\endisadelimproof
\isanewline
\isacommand{lemma}\isamarkupfalse%
\ has{\isacharunderscore}{\kern0pt}cond{\isacharunderscore}{\kern0pt}exp{\isacharunderscore}{\kern0pt}simple{\isacharcolon}{\kern0pt}\isanewline
\ \ \isakeyword{fixes}\ f\ {\isacharcolon}{\kern0pt}{\isacharcolon}{\kern0pt}\ {\isachardoublequoteopen}{\isacharprime}{\kern0pt}a\ {\isasymRightarrow}\ {\isacharprime}{\kern0pt}b{\isacharcolon}{\kern0pt}{\isacharcolon}{\kern0pt}{\isacharbraceleft}{\kern0pt}second{\isacharunderscore}{\kern0pt}countable{\isacharunderscore}{\kern0pt}topology{\isacharcomma}{\kern0pt}banach{\isacharbraceright}{\kern0pt}{\isachardoublequoteclose}\isanewline
\ \ \isakeyword{assumes}\ {\isachardoublequoteopen}simple{\isacharunderscore}{\kern0pt}function\ M\ f{\isachardoublequoteclose}\ {\isachardoublequoteopen}emeasure\ M\ {\isacharbraceleft}{\kern0pt}y\ {\isasymin}\ space\ M{\isachardot}{\kern0pt}\ f\ y\ {\isasymnoteq}\ {\isadigit{0}}{\isacharbraceright}{\kern0pt}\ {\isasymnoteq}\ {\isasyminfinity}{\isachardoublequoteclose}\isanewline
\ \ \isakeyword{shows}\ {\isachardoublequoteopen}has{\isacharunderscore}{\kern0pt}cond{\isacharunderscore}{\kern0pt}exp\ M\ F\ f\ {\isacharparenleft}{\kern0pt}cond{\isacharunderscore}{\kern0pt}exp\ M\ F\ f{\isacharparenright}{\kern0pt}{\isachardoublequoteclose}\isanewline
%
\isadelimproof
\ \ %
\endisadelimproof
%
\isatagproof
\isacommand{using}\isamarkupfalse%
\ assms\isanewline
\isacommand{proof}\isamarkupfalse%
\ {\isacharparenleft}{\kern0pt}induction\ rule{\isacharcolon}{\kern0pt}\ simple{\isacharunderscore}{\kern0pt}integrable{\isacharunderscore}{\kern0pt}function{\isacharunderscore}{\kern0pt}induct{\isacharparenright}{\kern0pt}\isanewline
\ \ \isacommand{case}\isamarkupfalse%
\ {\isacharparenleft}{\kern0pt}cong\ f\ g{\isacharparenright}{\kern0pt}\isanewline
\ \ \isacommand{then}\isamarkupfalse%
\ \isacommand{show}\isamarkupfalse%
\ {\isacharquery}{\kern0pt}case\ \isacommand{using}\isamarkupfalse%
\ has{\isacharunderscore}{\kern0pt}cond{\isacharunderscore}{\kern0pt}exp{\isacharunderscore}{\kern0pt}cong\ \isacommand{by}\isamarkupfalse%
\ {\isacharparenleft}{\kern0pt}metis\ {\isacharparenleft}{\kern0pt}no{\isacharunderscore}{\kern0pt}types{\isacharcomma}{\kern0pt}\ opaque{\isacharunderscore}{\kern0pt}lifting{\isacharparenright}{\kern0pt}\ Bochner{\isacharunderscore}{\kern0pt}Integration{\isachardot}{\kern0pt}integrable{\isacharunderscore}{\kern0pt}cong\ has{\isacharunderscore}{\kern0pt}cond{\isacharunderscore}{\kern0pt}expD{\isacharparenleft}{\kern0pt}{\isadigit{2}}{\isacharparenright}{\kern0pt}\ has{\isacharunderscore}{\kern0pt}cond{\isacharunderscore}{\kern0pt}exp{\isacharunderscore}{\kern0pt}charact{\isacharparenleft}{\kern0pt}{\isadigit{1}}{\isacharparenright}{\kern0pt}{\isacharparenright}{\kern0pt}\isanewline
\isacommand{next}\isamarkupfalse%
\isanewline
\ \ \isacommand{case}\isamarkupfalse%
\ {\isacharparenleft}{\kern0pt}indicator\ A\ y{\isacharparenright}{\kern0pt}\isanewline
\ \ \isacommand{then}\isamarkupfalse%
\ \isacommand{show}\isamarkupfalse%
\ {\isacharquery}{\kern0pt}case\ \isacommand{using}\isamarkupfalse%
\ has{\isacharunderscore}{\kern0pt}cond{\isacharunderscore}{\kern0pt}exp{\isacharunderscore}{\kern0pt}charact{\isacharbrackleft}{\kern0pt}OF\ has{\isacharunderscore}{\kern0pt}cond{\isacharunderscore}{\kern0pt}exp{\isacharunderscore}{\kern0pt}indicator{\isacharbrackright}{\kern0pt}\ \isacommand{by}\isamarkupfalse%
\ fast\isanewline
\isacommand{next}\isamarkupfalse%
\isanewline
\ \ \isacommand{case}\isamarkupfalse%
\ {\isacharparenleft}{\kern0pt}add\ u\ v{\isacharparenright}{\kern0pt}\isanewline
\ \ \isacommand{then}\isamarkupfalse%
\ \isacommand{show}\isamarkupfalse%
\ {\isacharquery}{\kern0pt}case\ \isacommand{using}\isamarkupfalse%
\ has{\isacharunderscore}{\kern0pt}cond{\isacharunderscore}{\kern0pt}exp{\isacharunderscore}{\kern0pt}add\ has{\isacharunderscore}{\kern0pt}cond{\isacharunderscore}{\kern0pt}exp{\isacharunderscore}{\kern0pt}charact{\isacharparenleft}{\kern0pt}{\isadigit{1}}{\isacharparenright}{\kern0pt}\ \isacommand{by}\isamarkupfalse%
\ blast\isanewline
\isacommand{qed}\isamarkupfalse%
%
\endisatagproof
{\isafoldproof}%
%
\isadelimproof
\isanewline
%
\endisadelimproof
\isanewline
\isacommand{lemma}\isamarkupfalse%
\ cond{\isacharunderscore}{\kern0pt}exp{\isacharunderscore}{\kern0pt}contraction{\isacharunderscore}{\kern0pt}real{\isacharcolon}{\kern0pt}\isanewline
\ \ \isakeyword{fixes}\ f\ {\isacharcolon}{\kern0pt}{\isacharcolon}{\kern0pt}\ {\isachardoublequoteopen}{\isacharprime}{\kern0pt}a\ {\isasymRightarrow}\ real{\isachardoublequoteclose}\isanewline
\ \ \isakeyword{assumes}\ integrable{\isacharbrackleft}{\kern0pt}measurable{\isacharbrackright}{\kern0pt}{\isacharcolon}{\kern0pt}\ {\isachardoublequoteopen}integrable\ M\ f{\isachardoublequoteclose}\isanewline
\ \ \isakeyword{shows}\ {\isachardoublequoteopen}AE\ x\ in\ M{\isachardot}{\kern0pt}\ norm\ {\isacharparenleft}{\kern0pt}cond{\isacharunderscore}{\kern0pt}exp\ M\ F\ f\ x{\isacharparenright}{\kern0pt}\ {\isasymle}\ cond{\isacharunderscore}{\kern0pt}exp\ M\ F\ {\isacharparenleft}{\kern0pt}{\isasymlambda}x{\isachardot}{\kern0pt}\ norm\ {\isacharparenleft}{\kern0pt}f\ x{\isacharparenright}{\kern0pt}{\isacharparenright}{\kern0pt}\ x{\isachardoublequoteclose}\isanewline
%
\isadelimproof
%
\endisadelimproof
%
\isatagproof
\isacommand{proof}\isamarkupfalse%
{\isacharminus}{\kern0pt}\isanewline
\ \ \isacommand{have}\isamarkupfalse%
\ int{\isacharcolon}{\kern0pt}\ {\isachardoublequoteopen}integrable\ M\ {\isacharparenleft}{\kern0pt}{\isasymlambda}x{\isachardot}{\kern0pt}\ norm\ {\isacharparenleft}{\kern0pt}f\ x{\isacharparenright}{\kern0pt}{\isacharparenright}{\kern0pt}{\isachardoublequoteclose}\ \isacommand{using}\isamarkupfalse%
\ assms\ \isacommand{by}\isamarkupfalse%
\ blast\isanewline
\ \ \isacommand{have}\isamarkupfalse%
\ {\isacharasterisk}{\kern0pt}{\isacharcolon}{\kern0pt}\ {\isachardoublequoteopen}AE\ x\ in\ M{\isachardot}{\kern0pt}\ {\isadigit{0}}\ {\isasymle}\ cond{\isacharunderscore}{\kern0pt}exp\ M\ F\ {\isacharparenleft}{\kern0pt}{\isasymlambda}x{\isachardot}{\kern0pt}\ norm\ {\isacharparenleft}{\kern0pt}f\ x{\isacharparenright}{\kern0pt}{\isacharparenright}{\kern0pt}\ x{\isachardoublequoteclose}\ \isacommand{using}\isamarkupfalse%
\ cond{\isacharunderscore}{\kern0pt}exp{\isacharunderscore}{\kern0pt}real{\isacharbrackleft}{\kern0pt}THEN\ AE{\isacharunderscore}{\kern0pt}symmetric{\isacharcomma}{\kern0pt}\ OF\ integrable{\isacharunderscore}{\kern0pt}norm{\isacharbrackleft}{\kern0pt}OF\ integrable{\isacharbrackright}{\kern0pt}{\isacharbrackright}{\kern0pt}\ real{\isacharunderscore}{\kern0pt}cond{\isacharunderscore}{\kern0pt}exp{\isacharunderscore}{\kern0pt}ge{\isacharunderscore}{\kern0pt}c{\isacharbrackleft}{\kern0pt}OF\ integrable{\isacharunderscore}{\kern0pt}norm{\isacharbrackleft}{\kern0pt}OF\ integrable{\isacharbrackright}{\kern0pt}{\isacharcomma}{\kern0pt}\ of\ {\isadigit{0}}{\isacharbrackright}{\kern0pt}\ norm{\isacharunderscore}{\kern0pt}ge{\isacharunderscore}{\kern0pt}zero\ \isacommand{by}\isamarkupfalse%
\ fastforce\isanewline
\ \ \isacommand{have}\isamarkupfalse%
\ {\isacharasterisk}{\kern0pt}{\isacharasterisk}{\kern0pt}{\isacharcolon}{\kern0pt}\ {\isachardoublequoteopen}A\ {\isasymin}\ sets\ F\ {\isasymLongrightarrow}\ {\isasymintegral}x{\isasymin}A{\isachardot}{\kern0pt}\ {\isasymbar}f\ x{\isasymbar}\ {\isasympartial}M\ {\isacharequal}{\kern0pt}\ {\isasymintegral}x{\isasymin}A{\isachardot}{\kern0pt}\ real{\isacharunderscore}{\kern0pt}cond{\isacharunderscore}{\kern0pt}exp\ M\ F\ {\isacharparenleft}{\kern0pt}{\isasymlambda}x{\isachardot}{\kern0pt}\ norm\ {\isacharparenleft}{\kern0pt}f\ x{\isacharparenright}{\kern0pt}{\isacharparenright}{\kern0pt}\ x\ {\isasympartial}M{\isachardoublequoteclose}\ \isakeyword{for}\ A\ \isacommand{unfolding}\isamarkupfalse%
\ real{\isacharunderscore}{\kern0pt}norm{\isacharunderscore}{\kern0pt}def\ \isacommand{using}\isamarkupfalse%
\ assms\ integrable{\isacharunderscore}{\kern0pt}abs\ real{\isacharunderscore}{\kern0pt}cond{\isacharunderscore}{\kern0pt}exp{\isacharunderscore}{\kern0pt}intA\ \isacommand{by}\isamarkupfalse%
\ blast\isanewline
\ \ \isanewline
\ \ \isacommand{have}\isamarkupfalse%
\ norm{\isacharunderscore}{\kern0pt}int{\isacharcolon}{\kern0pt}\ {\isachardoublequoteopen}A\ {\isasymin}\ sets\ F\ {\isasymLongrightarrow}\ {\isacharparenleft}{\kern0pt}{\isasymintegral}x{\isasymin}A{\isachardot}{\kern0pt}\ {\isasymbar}f\ x{\isasymbar}\ {\isasympartial}M{\isacharparenright}{\kern0pt}\ {\isacharequal}{\kern0pt}\ {\isacharparenleft}{\kern0pt}{\isasymintegral}\isactrlsup {\isacharplus}{\kern0pt}x{\isasymin}A{\isachardot}{\kern0pt}\ {\isasymbar}f\ x{\isasymbar}\ {\isasympartial}M{\isacharparenright}{\kern0pt}{\isachardoublequoteclose}\ \isakeyword{for}\ A\ \isacommand{using}\isamarkupfalse%
\ assms\ \isacommand{by}\isamarkupfalse%
\ {\isacharparenleft}{\kern0pt}intro\ nn{\isacharunderscore}{\kern0pt}set{\isacharunderscore}{\kern0pt}integral{\isacharunderscore}{\kern0pt}eq{\isacharunderscore}{\kern0pt}set{\isacharunderscore}{\kern0pt}integral{\isacharbrackleft}{\kern0pt}symmetric{\isacharbrackright}{\kern0pt}{\isacharcomma}{\kern0pt}\ blast{\isacharcomma}{\kern0pt}\ fastforce{\isacharparenright}{\kern0pt}\ {\isacharparenleft}{\kern0pt}meson\ subalg\ subalgebra{\isacharunderscore}{\kern0pt}def\ subsetD{\isacharparenright}{\kern0pt}\isanewline
\ \ \isanewline
\ \ \isacommand{have}\isamarkupfalse%
\ {\isachardoublequoteopen}AE\ x\ in\ M{\isachardot}{\kern0pt}\ real{\isacharunderscore}{\kern0pt}cond{\isacharunderscore}{\kern0pt}exp\ M\ F\ {\isacharparenleft}{\kern0pt}{\isasymlambda}x{\isachardot}{\kern0pt}\ norm\ {\isacharparenleft}{\kern0pt}f\ x{\isacharparenright}{\kern0pt}{\isacharparenright}{\kern0pt}\ x\ {\isasymge}\ {\isadigit{0}}{\isachardoublequoteclose}\ \isacommand{using}\isamarkupfalse%
\ int\ real{\isacharunderscore}{\kern0pt}cond{\isacharunderscore}{\kern0pt}exp{\isacharunderscore}{\kern0pt}ge{\isacharunderscore}{\kern0pt}c\ \isacommand{by}\isamarkupfalse%
\ force\isanewline
\ \ \isacommand{hence}\isamarkupfalse%
\ cond{\isacharunderscore}{\kern0pt}exp{\isacharunderscore}{\kern0pt}norm{\isacharunderscore}{\kern0pt}int{\isacharcolon}{\kern0pt}\ {\isachardoublequoteopen}A\ {\isasymin}\ sets\ F\ {\isasymLongrightarrow}\ {\isacharparenleft}{\kern0pt}{\isasymintegral}x{\isasymin}A{\isachardot}{\kern0pt}\ real{\isacharunderscore}{\kern0pt}cond{\isacharunderscore}{\kern0pt}exp\ M\ F\ {\isacharparenleft}{\kern0pt}{\isasymlambda}x{\isachardot}{\kern0pt}\ norm\ {\isacharparenleft}{\kern0pt}f\ x{\isacharparenright}{\kern0pt}{\isacharparenright}{\kern0pt}\ x\ {\isasympartial}M{\isacharparenright}{\kern0pt}\ {\isacharequal}{\kern0pt}\ {\isacharparenleft}{\kern0pt}{\isasymintegral}\isactrlsup {\isacharplus}{\kern0pt}x{\isasymin}A{\isachardot}{\kern0pt}\ real{\isacharunderscore}{\kern0pt}cond{\isacharunderscore}{\kern0pt}exp\ M\ F\ {\isacharparenleft}{\kern0pt}{\isasymlambda}x{\isachardot}{\kern0pt}\ norm\ {\isacharparenleft}{\kern0pt}f\ x{\isacharparenright}{\kern0pt}{\isacharparenright}{\kern0pt}\ x\ {\isasympartial}M{\isacharparenright}{\kern0pt}{\isachardoublequoteclose}\ \isakeyword{for}\ A\ \isacommand{using}\isamarkupfalse%
\ assms\ \isacommand{by}\isamarkupfalse%
\ {\isacharparenleft}{\kern0pt}intro\ nn{\isacharunderscore}{\kern0pt}set{\isacharunderscore}{\kern0pt}integral{\isacharunderscore}{\kern0pt}eq{\isacharunderscore}{\kern0pt}set{\isacharunderscore}{\kern0pt}integral{\isacharbrackleft}{\kern0pt}symmetric{\isacharbrackright}{\kern0pt}{\isacharcomma}{\kern0pt}\ blast{\isacharcomma}{\kern0pt}\ fastforce{\isacharparenright}{\kern0pt}\ {\isacharparenleft}{\kern0pt}meson\ subalg\ subalgebra{\isacharunderscore}{\kern0pt}def\ subsetD{\isacharparenright}{\kern0pt}\isanewline
\ \ \isanewline
\ \ \isacommand{have}\isamarkupfalse%
\ {\isachardoublequoteopen}A\ {\isasymin}\ sets\ F\ {\isasymLongrightarrow}\ {\isasymintegral}\isactrlsup {\isacharplus}{\kern0pt}x{\isasymin}A{\isachardot}{\kern0pt}\ {\isasymbar}f\ x{\isasymbar}{\isasympartial}M\ {\isacharequal}{\kern0pt}\ {\isasymintegral}\isactrlsup {\isacharplus}{\kern0pt}x{\isasymin}A{\isachardot}{\kern0pt}\ real{\isacharunderscore}{\kern0pt}cond{\isacharunderscore}{\kern0pt}exp\ M\ F\ {\isacharparenleft}{\kern0pt}{\isasymlambda}x{\isachardot}{\kern0pt}\ norm\ {\isacharparenleft}{\kern0pt}f\ x{\isacharparenright}{\kern0pt}{\isacharparenright}{\kern0pt}\ x\ {\isasympartial}M{\isachardoublequoteclose}\ \isakeyword{for}\ A\ \isacommand{using}\isamarkupfalse%
\ {\isacharasterisk}{\kern0pt}{\isacharasterisk}{\kern0pt}\ norm{\isacharunderscore}{\kern0pt}int\ cond{\isacharunderscore}{\kern0pt}exp{\isacharunderscore}{\kern0pt}norm{\isacharunderscore}{\kern0pt}int\ \isacommand{by}\isamarkupfalse%
\ {\isacharparenleft}{\kern0pt}auto\ simp\ add{\isacharcolon}{\kern0pt}\ nn{\isacharunderscore}{\kern0pt}integral{\isacharunderscore}{\kern0pt}set{\isacharunderscore}{\kern0pt}ennreal{\isacharparenright}{\kern0pt}\isanewline
\ \ \isacommand{moreover}\isamarkupfalse%
\ \isacommand{have}\isamarkupfalse%
\ {\isachardoublequoteopen}{\isacharparenleft}{\kern0pt}{\isasymlambda}x{\isachardot}{\kern0pt}\ ennreal\ {\isasymbar}f\ x{\isasymbar}{\isacharparenright}{\kern0pt}\ {\isasymin}\ borel{\isacharunderscore}{\kern0pt}measurable\ M{\isachardoublequoteclose}\ \isacommand{by}\isamarkupfalse%
\ measurable\isanewline
\ \ \isacommand{moreover}\isamarkupfalse%
\ \isacommand{have}\isamarkupfalse%
\ {\isachardoublequoteopen}{\isacharparenleft}{\kern0pt}{\isasymlambda}x{\isachardot}{\kern0pt}\ ennreal\ {\isacharparenleft}{\kern0pt}real{\isacharunderscore}{\kern0pt}cond{\isacharunderscore}{\kern0pt}exp\ M\ F\ {\isacharparenleft}{\kern0pt}{\isasymlambda}x{\isachardot}{\kern0pt}\ norm\ {\isacharparenleft}{\kern0pt}f\ x{\isacharparenright}{\kern0pt}{\isacharparenright}{\kern0pt}\ x{\isacharparenright}{\kern0pt}{\isacharparenright}{\kern0pt}\ {\isasymin}\ borel{\isacharunderscore}{\kern0pt}measurable\ F{\isachardoublequoteclose}\ \isacommand{by}\isamarkupfalse%
\ measurable\isanewline
\ \ \isacommand{ultimately}\isamarkupfalse%
\ \isacommand{have}\isamarkupfalse%
\ {\isachardoublequoteopen}AE\ x\ in\ M{\isachardot}{\kern0pt}\ nn{\isacharunderscore}{\kern0pt}cond{\isacharunderscore}{\kern0pt}exp\ M\ F\ {\isacharparenleft}{\kern0pt}{\isasymlambda}x{\isachardot}{\kern0pt}\ ennreal\ {\isasymbar}f\ x{\isasymbar}{\isacharparenright}{\kern0pt}\ x\ {\isacharequal}{\kern0pt}\ real{\isacharunderscore}{\kern0pt}cond{\isacharunderscore}{\kern0pt}exp\ M\ F\ {\isacharparenleft}{\kern0pt}{\isasymlambda}x{\isachardot}{\kern0pt}\ norm\ {\isacharparenleft}{\kern0pt}f\ x{\isacharparenright}{\kern0pt}{\isacharparenright}{\kern0pt}\ x{\isachardoublequoteclose}\ \isacommand{by}\isamarkupfalse%
\ {\isacharparenleft}{\kern0pt}intro\ nn{\isacharunderscore}{\kern0pt}cond{\isacharunderscore}{\kern0pt}exp{\isacharunderscore}{\kern0pt}charact{\isacharbrackleft}{\kern0pt}THEN\ AE{\isacharunderscore}{\kern0pt}symmetric{\isacharbrackright}{\kern0pt}{\isacharcomma}{\kern0pt}\ auto{\isacharparenright}{\kern0pt}\isanewline
\ \ \ \ \isacommand{hence}\isamarkupfalse%
\ {\isachardoublequoteopen}AE\ x\ in\ M{\isachardot}{\kern0pt}\ nn{\isacharunderscore}{\kern0pt}cond{\isacharunderscore}{\kern0pt}exp\ M\ F\ {\isacharparenleft}{\kern0pt}{\isasymlambda}x{\isachardot}{\kern0pt}\ ennreal\ {\isasymbar}f\ x{\isasymbar}{\isacharparenright}{\kern0pt}\ x\ {\isasymle}\ cond{\isacharunderscore}{\kern0pt}exp\ M\ F\ {\isacharparenleft}{\kern0pt}{\isasymlambda}x{\isachardot}{\kern0pt}\ norm\ {\isacharparenleft}{\kern0pt}f\ x{\isacharparenright}{\kern0pt}{\isacharparenright}{\kern0pt}\ x{\isachardoublequoteclose}\ \isacommand{using}\isamarkupfalse%
\ cond{\isacharunderscore}{\kern0pt}exp{\isacharunderscore}{\kern0pt}real{\isacharbrackleft}{\kern0pt}OF\ int{\isacharbrackright}{\kern0pt}\ \isacommand{by}\isamarkupfalse%
\ force\isanewline
\ \ \isacommand{moreover}\isamarkupfalse%
\ \isacommand{have}\isamarkupfalse%
\ {\isachardoublequoteopen}AE\ x\ in\ M{\isachardot}{\kern0pt}\ {\isasymbar}real{\isacharunderscore}{\kern0pt}cond{\isacharunderscore}{\kern0pt}exp\ M\ F\ f\ x{\isasymbar}\ {\isacharequal}{\kern0pt}\ norm\ {\isacharparenleft}{\kern0pt}cond{\isacharunderscore}{\kern0pt}exp\ M\ F\ f\ x{\isacharparenright}{\kern0pt}{\isachardoublequoteclose}\ \isacommand{unfolding}\isamarkupfalse%
\ real{\isacharunderscore}{\kern0pt}norm{\isacharunderscore}{\kern0pt}def\ \isacommand{using}\isamarkupfalse%
\ cond{\isacharunderscore}{\kern0pt}exp{\isacharunderscore}{\kern0pt}real{\isacharbrackleft}{\kern0pt}OF\ assms{\isacharbrackright}{\kern0pt}\ {\isacharasterisk}{\kern0pt}\ \isacommand{by}\isamarkupfalse%
\ force\isanewline
\ \ \isacommand{ultimately}\isamarkupfalse%
\ \isacommand{have}\isamarkupfalse%
\ {\isachardoublequoteopen}AE\ x\ in\ M{\isachardot}{\kern0pt}\ ennreal\ {\isacharparenleft}{\kern0pt}norm\ {\isacharparenleft}{\kern0pt}cond{\isacharunderscore}{\kern0pt}exp\ M\ F\ f\ x{\isacharparenright}{\kern0pt}{\isacharparenright}{\kern0pt}\ {\isasymle}\ cond{\isacharunderscore}{\kern0pt}exp\ M\ F\ {\isacharparenleft}{\kern0pt}{\isasymlambda}x{\isachardot}{\kern0pt}\ norm\ {\isacharparenleft}{\kern0pt}f\ x{\isacharparenright}{\kern0pt}{\isacharparenright}{\kern0pt}\ x{\isachardoublequoteclose}\ \isacommand{using}\isamarkupfalse%
\ real{\isacharunderscore}{\kern0pt}cond{\isacharunderscore}{\kern0pt}exp{\isacharunderscore}{\kern0pt}abs{\isacharbrackleft}{\kern0pt}OF\ assms{\isacharbrackleft}{\kern0pt}THEN\ borel{\isacharunderscore}{\kern0pt}measurable{\isacharunderscore}{\kern0pt}integrable{\isacharbrackright}{\kern0pt}{\isacharbrackright}{\kern0pt}\ \isacommand{by}\isamarkupfalse%
\ fastforce\isanewline
\ \ \isacommand{hence}\isamarkupfalse%
\ {\isachardoublequoteopen}AE\ x\ in\ M{\isachardot}{\kern0pt}\ enn{\isadigit{2}}real\ {\isacharparenleft}{\kern0pt}ennreal\ {\isacharparenleft}{\kern0pt}norm\ {\isacharparenleft}{\kern0pt}cond{\isacharunderscore}{\kern0pt}exp\ M\ F\ f\ x{\isacharparenright}{\kern0pt}{\isacharparenright}{\kern0pt}{\isacharparenright}{\kern0pt}\ {\isasymle}\ enn{\isadigit{2}}real\ {\isacharparenleft}{\kern0pt}cond{\isacharunderscore}{\kern0pt}exp\ M\ F\ {\isacharparenleft}{\kern0pt}{\isasymlambda}x{\isachardot}{\kern0pt}\ norm\ {\isacharparenleft}{\kern0pt}f\ x{\isacharparenright}{\kern0pt}{\isacharparenright}{\kern0pt}\ x{\isacharparenright}{\kern0pt}{\isachardoublequoteclose}\ \isacommand{using}\isamarkupfalse%
\ ennreal{\isacharunderscore}{\kern0pt}le{\isacharunderscore}{\kern0pt}iff{\isadigit{2}}\ \isacommand{by}\isamarkupfalse%
\ force\isanewline
\ \ \isacommand{thus}\isamarkupfalse%
\ {\isacharquery}{\kern0pt}thesis\ \isacommand{using}\isamarkupfalse%
\ {\isacharasterisk}{\kern0pt}\ \isacommand{by}\isamarkupfalse%
\ fastforce\isanewline
\isacommand{qed}\isamarkupfalse%
%
\endisatagproof
{\isafoldproof}%
%
\isadelimproof
\isanewline
%
\endisadelimproof
\isanewline
\isacommand{lemma}\isamarkupfalse%
\ cond{\isacharunderscore}{\kern0pt}exp{\isacharunderscore}{\kern0pt}contraction{\isacharunderscore}{\kern0pt}simple{\isacharcolon}{\kern0pt}\isanewline
\ \ \isakeyword{fixes}\ f\ {\isacharcolon}{\kern0pt}{\isacharcolon}{\kern0pt}\ {\isachardoublequoteopen}{\isacharprime}{\kern0pt}a\ {\isasymRightarrow}\ {\isacharprime}{\kern0pt}b{\isacharcolon}{\kern0pt}{\isacharcolon}{\kern0pt}{\isacharbraceleft}{\kern0pt}second{\isacharunderscore}{\kern0pt}countable{\isacharunderscore}{\kern0pt}topology{\isacharcomma}{\kern0pt}\ banach{\isacharbraceright}{\kern0pt}{\isachardoublequoteclose}\isanewline
\ \ \isakeyword{assumes}\ {\isachardoublequoteopen}simple{\isacharunderscore}{\kern0pt}function\ M\ f{\isachardoublequoteclose}\ {\isachardoublequoteopen}emeasure\ M\ {\isacharbraceleft}{\kern0pt}y\ {\isasymin}\ space\ M{\isachardot}{\kern0pt}\ f\ y\ {\isasymnoteq}\ {\isadigit{0}}{\isacharbraceright}{\kern0pt}\ {\isasymnoteq}\ {\isasyminfinity}{\isachardoublequoteclose}\isanewline
\ \ \isakeyword{shows}\ {\isachardoublequoteopen}AE\ x\ in\ M{\isachardot}{\kern0pt}\ norm\ {\isacharparenleft}{\kern0pt}cond{\isacharunderscore}{\kern0pt}exp\ M\ F\ f\ x{\isacharparenright}{\kern0pt}\ {\isasymle}\ cond{\isacharunderscore}{\kern0pt}exp\ M\ F\ {\isacharparenleft}{\kern0pt}{\isasymlambda}x{\isachardot}{\kern0pt}\ norm\ {\isacharparenleft}{\kern0pt}f\ x{\isacharparenright}{\kern0pt}{\isacharparenright}{\kern0pt}\ x{\isachardoublequoteclose}\isanewline
%
\isadelimproof
\ \ %
\endisadelimproof
%
\isatagproof
\isacommand{using}\isamarkupfalse%
\ assms\isanewline
\isacommand{proof}\isamarkupfalse%
\ {\isacharparenleft}{\kern0pt}induction\ rule{\isacharcolon}{\kern0pt}\ simple{\isacharunderscore}{\kern0pt}integrable{\isacharunderscore}{\kern0pt}function{\isacharunderscore}{\kern0pt}induct{\isacharparenright}{\kern0pt}\isanewline
\ \ \isacommand{case}\isamarkupfalse%
\ {\isacharparenleft}{\kern0pt}cong\ f\ g{\isacharparenright}{\kern0pt}\isanewline
\ \ \isacommand{hence}\isamarkupfalse%
\ ae{\isacharcolon}{\kern0pt}\ {\isachardoublequoteopen}AE\ x\ in\ M{\isachardot}{\kern0pt}\ f\ x\ {\isacharequal}{\kern0pt}\ g\ x{\isachardoublequoteclose}\ \isacommand{by}\isamarkupfalse%
\ blast\isanewline
\ \ \isacommand{hence}\isamarkupfalse%
\ {\isachardoublequoteopen}AE\ x\ in\ M{\isachardot}{\kern0pt}\ cond{\isacharunderscore}{\kern0pt}exp\ M\ F\ f\ x\ {\isacharequal}{\kern0pt}\ cond{\isacharunderscore}{\kern0pt}exp\ M\ F\ g\ x{\isachardoublequoteclose}\ \isacommand{using}\isamarkupfalse%
\ cong\ has{\isacharunderscore}{\kern0pt}cond{\isacharunderscore}{\kern0pt}exp{\isacharunderscore}{\kern0pt}simple\ \isacommand{by}\isamarkupfalse%
\ {\isacharparenleft}{\kern0pt}subst\ cond{\isacharunderscore}{\kern0pt}exp{\isacharunderscore}{\kern0pt}cong{\isacharunderscore}{\kern0pt}AE{\isacharparenright}{\kern0pt}\ {\isacharparenleft}{\kern0pt}auto\ intro{\isacharbang}{\kern0pt}{\isacharcolon}{\kern0pt}\ has{\isacharunderscore}{\kern0pt}cond{\isacharunderscore}{\kern0pt}expD{\isacharparenleft}{\kern0pt}{\isadigit{2}}{\isacharparenright}{\kern0pt}{\isacharparenright}{\kern0pt}\isanewline
\ \ \isacommand{hence}\isamarkupfalse%
\ {\isachardoublequoteopen}AE\ x\ in\ M{\isachardot}{\kern0pt}\ norm\ {\isacharparenleft}{\kern0pt}cond{\isacharunderscore}{\kern0pt}exp\ M\ F\ f\ x{\isacharparenright}{\kern0pt}\ {\isacharequal}{\kern0pt}\ norm\ {\isacharparenleft}{\kern0pt}cond{\isacharunderscore}{\kern0pt}exp\ M\ F\ g\ x{\isacharparenright}{\kern0pt}{\isachardoublequoteclose}\ \isacommand{by}\isamarkupfalse%
\ force\isanewline
\ \ \isacommand{moreover}\isamarkupfalse%
\ \isacommand{have}\isamarkupfalse%
\ {\isachardoublequoteopen}AE\ x\ in\ M{\isachardot}{\kern0pt}\ cond{\isacharunderscore}{\kern0pt}exp\ M\ F\ {\isacharparenleft}{\kern0pt}{\isasymlambda}x{\isachardot}{\kern0pt}\ norm\ {\isacharparenleft}{\kern0pt}f\ x{\isacharparenright}{\kern0pt}{\isacharparenright}{\kern0pt}\ x\ {\isacharequal}{\kern0pt}\ cond{\isacharunderscore}{\kern0pt}exp\ M\ F\ {\isacharparenleft}{\kern0pt}{\isasymlambda}x{\isachardot}{\kern0pt}\ norm\ {\isacharparenleft}{\kern0pt}g\ x{\isacharparenright}{\kern0pt}{\isacharparenright}{\kern0pt}\ x{\isachardoublequoteclose}\ \ \isacommand{using}\isamarkupfalse%
\ ae\ cong\ has{\isacharunderscore}{\kern0pt}cond{\isacharunderscore}{\kern0pt}exp{\isacharunderscore}{\kern0pt}simple\ \isacommand{by}\isamarkupfalse%
\ {\isacharparenleft}{\kern0pt}subst\ cond{\isacharunderscore}{\kern0pt}exp{\isacharunderscore}{\kern0pt}cong{\isacharunderscore}{\kern0pt}AE{\isacharparenright}{\kern0pt}\ {\isacharparenleft}{\kern0pt}auto\ dest{\isacharcolon}{\kern0pt}\ has{\isacharunderscore}{\kern0pt}cond{\isacharunderscore}{\kern0pt}expD{\isacharparenright}{\kern0pt}\isanewline
\ \ \isacommand{ultimately}\isamarkupfalse%
\ \isacommand{show}\isamarkupfalse%
\ {\isacharquery}{\kern0pt}case\ \isacommand{using}\isamarkupfalse%
\ cong{\isacharparenleft}{\kern0pt}{\isadigit{6}}{\isacharparenright}{\kern0pt}\ \isacommand{by}\isamarkupfalse%
\ fastforce\isanewline
\isacommand{next}\isamarkupfalse%
\isanewline
\ \ \isacommand{case}\isamarkupfalse%
\ {\isacharparenleft}{\kern0pt}indicator\ A\ y{\isacharparenright}{\kern0pt}\isanewline
\ \ \isacommand{hence}\isamarkupfalse%
\ {\isachardoublequoteopen}AE\ x\ in\ M{\isachardot}{\kern0pt}\ cond{\isacharunderscore}{\kern0pt}exp\ M\ F\ {\isacharparenleft}{\kern0pt}{\isasymlambda}a{\isachardot}{\kern0pt}\ indicator\ A\ a\ {\isacharasterisk}{\kern0pt}\isactrlsub R\ y{\isacharparenright}{\kern0pt}\ x\ {\isacharequal}{\kern0pt}\ cond{\isacharunderscore}{\kern0pt}exp\ M\ F\ {\isacharparenleft}{\kern0pt}indicator\ A{\isacharparenright}{\kern0pt}\ x\ {\isacharasterisk}{\kern0pt}\isactrlsub R\ y{\isachardoublequoteclose}\ \isacommand{by}\isamarkupfalse%
\ blast\isanewline
\ \ \isacommand{hence}\isamarkupfalse%
\ {\isacharasterisk}{\kern0pt}{\isacharcolon}{\kern0pt}\ {\isachardoublequoteopen}AE\ x\ in\ M{\isachardot}{\kern0pt}\ norm\ {\isacharparenleft}{\kern0pt}cond{\isacharunderscore}{\kern0pt}exp\ M\ F\ {\isacharparenleft}{\kern0pt}{\isasymlambda}a{\isachardot}{\kern0pt}\ indicat{\isacharunderscore}{\kern0pt}real\ A\ a\ {\isacharasterisk}{\kern0pt}\isactrlsub R\ y{\isacharparenright}{\kern0pt}\ x{\isacharparenright}{\kern0pt}\ {\isasymle}\ norm\ y\ {\isacharasterisk}{\kern0pt}\ cond{\isacharunderscore}{\kern0pt}exp\ M\ F\ {\isacharparenleft}{\kern0pt}{\isasymlambda}x{\isachardot}{\kern0pt}\ norm\ {\isacharparenleft}{\kern0pt}indicat{\isacharunderscore}{\kern0pt}real\ A\ x{\isacharparenright}{\kern0pt}{\isacharparenright}{\kern0pt}\ x{\isachardoublequoteclose}\ \isacommand{using}\isamarkupfalse%
\ cond{\isacharunderscore}{\kern0pt}exp{\isacharunderscore}{\kern0pt}contraction{\isacharunderscore}{\kern0pt}real{\isacharbrackleft}{\kern0pt}OF\ integrable{\isacharunderscore}{\kern0pt}real{\isacharunderscore}{\kern0pt}indicator{\isacharcomma}{\kern0pt}\ OF\ indicator{\isacharbrackright}{\kern0pt}\ \isacommand{by}\isamarkupfalse%
\ fastforce\isanewline
\isanewline
\ \ \isacommand{have}\isamarkupfalse%
\ {\isachardoublequoteopen}AE\ x\ in\ M{\isachardot}{\kern0pt}\ norm\ y\ {\isacharasterisk}{\kern0pt}\ cond{\isacharunderscore}{\kern0pt}exp\ M\ F\ {\isacharparenleft}{\kern0pt}{\isasymlambda}x{\isachardot}{\kern0pt}\ norm\ {\isacharparenleft}{\kern0pt}indicat{\isacharunderscore}{\kern0pt}real\ A\ x{\isacharparenright}{\kern0pt}{\isacharparenright}{\kern0pt}\ x\ {\isacharequal}{\kern0pt}\ norm\ y\ {\isacharasterisk}{\kern0pt}\ real{\isacharunderscore}{\kern0pt}cond{\isacharunderscore}{\kern0pt}exp\ M\ F\ {\isacharparenleft}{\kern0pt}{\isasymlambda}x{\isachardot}{\kern0pt}\ norm\ {\isacharparenleft}{\kern0pt}indicat{\isacharunderscore}{\kern0pt}real\ A\ x{\isacharparenright}{\kern0pt}{\isacharparenright}{\kern0pt}\ x{\isachardoublequoteclose}\ \isacommand{using}\isamarkupfalse%
\ cond{\isacharunderscore}{\kern0pt}exp{\isacharunderscore}{\kern0pt}real{\isacharbrackleft}{\kern0pt}OF\ integrable{\isacharunderscore}{\kern0pt}real{\isacharunderscore}{\kern0pt}indicator{\isacharcomma}{\kern0pt}\ OF\ indicator{\isacharbrackright}{\kern0pt}\ \isacommand{by}\isamarkupfalse%
\ fastforce\isanewline
\ \ \isacommand{moreover}\isamarkupfalse%
\ \isacommand{have}\isamarkupfalse%
\ {\isachardoublequoteopen}AE\ x\ in\ M{\isachardot}{\kern0pt}\ cond{\isacharunderscore}{\kern0pt}exp\ M\ F\ {\isacharparenleft}{\kern0pt}{\isasymlambda}x{\isachardot}{\kern0pt}\ norm\ y\ {\isacharasterisk}{\kern0pt}\ norm\ {\isacharparenleft}{\kern0pt}indicat{\isacharunderscore}{\kern0pt}real\ A\ x{\isacharparenright}{\kern0pt}{\isacharparenright}{\kern0pt}\ x\ {\isacharequal}{\kern0pt}\ real{\isacharunderscore}{\kern0pt}cond{\isacharunderscore}{\kern0pt}exp\ M\ F\ {\isacharparenleft}{\kern0pt}{\isasymlambda}x{\isachardot}{\kern0pt}\ norm\ y\ {\isacharasterisk}{\kern0pt}\ norm\ {\isacharparenleft}{\kern0pt}indicat{\isacharunderscore}{\kern0pt}real\ A\ x{\isacharparenright}{\kern0pt}{\isacharparenright}{\kern0pt}\ x{\isachardoublequoteclose}\ \isacommand{using}\isamarkupfalse%
\ indicator\ \isacommand{by}\isamarkupfalse%
\ {\isacharparenleft}{\kern0pt}intro\ cond{\isacharunderscore}{\kern0pt}exp{\isacharunderscore}{\kern0pt}real{\isacharcomma}{\kern0pt}\ auto{\isacharparenright}{\kern0pt}\isanewline
\ \ \isacommand{ultimately}\isamarkupfalse%
\ \isacommand{have}\isamarkupfalse%
\ {\isachardoublequoteopen}AE\ x\ in\ M{\isachardot}{\kern0pt}\ norm\ y\ {\isacharasterisk}{\kern0pt}\ cond{\isacharunderscore}{\kern0pt}exp\ M\ F\ {\isacharparenleft}{\kern0pt}{\isasymlambda}x{\isachardot}{\kern0pt}\ norm\ {\isacharparenleft}{\kern0pt}indicat{\isacharunderscore}{\kern0pt}real\ A\ x{\isacharparenright}{\kern0pt}{\isacharparenright}{\kern0pt}\ x\ {\isacharequal}{\kern0pt}\ cond{\isacharunderscore}{\kern0pt}exp\ M\ F\ {\isacharparenleft}{\kern0pt}{\isasymlambda}x{\isachardot}{\kern0pt}\ norm\ y\ {\isacharasterisk}{\kern0pt}\ norm\ {\isacharparenleft}{\kern0pt}indicat{\isacharunderscore}{\kern0pt}real\ A\ x{\isacharparenright}{\kern0pt}{\isacharparenright}{\kern0pt}\ x{\isachardoublequoteclose}\ \isacommand{using}\isamarkupfalse%
\ real{\isacharunderscore}{\kern0pt}cond{\isacharunderscore}{\kern0pt}exp{\isacharunderscore}{\kern0pt}cmult{\isacharbrackleft}{\kern0pt}of\ {\isachardoublequoteopen}{\isasymlambda}x{\isachardot}{\kern0pt}\ norm\ {\isacharparenleft}{\kern0pt}indicat{\isacharunderscore}{\kern0pt}real\ A\ x{\isacharparenright}{\kern0pt}{\isachardoublequoteclose}\ {\isachardoublequoteopen}norm\ y{\isachardoublequoteclose}{\isacharbrackright}{\kern0pt}\ indicator\ \isacommand{by}\isamarkupfalse%
\ fastforce\isanewline
\ \ \isacommand{moreover}\isamarkupfalse%
\ \isacommand{have}\isamarkupfalse%
\ {\isachardoublequoteopen}{\isacharparenleft}{\kern0pt}{\isasymlambda}x{\isachardot}{\kern0pt}\ norm\ y\ {\isacharasterisk}{\kern0pt}\ norm\ {\isacharparenleft}{\kern0pt}indicat{\isacharunderscore}{\kern0pt}real\ A\ x{\isacharparenright}{\kern0pt}{\isacharparenright}{\kern0pt}\ {\isacharequal}{\kern0pt}\ {\isacharparenleft}{\kern0pt}{\isasymlambda}x{\isachardot}{\kern0pt}\ norm\ {\isacharparenleft}{\kern0pt}indicat{\isacharunderscore}{\kern0pt}real\ A\ x\ {\isacharasterisk}{\kern0pt}\isactrlsub R\ y{\isacharparenright}{\kern0pt}{\isacharparenright}{\kern0pt}{\isachardoublequoteclose}\ \isacommand{by}\isamarkupfalse%
\ force\isanewline
\ \ \isacommand{ultimately}\isamarkupfalse%
\ \isacommand{show}\isamarkupfalse%
\ {\isacharquery}{\kern0pt}case\ \isacommand{using}\isamarkupfalse%
\ {\isacharasterisk}{\kern0pt}\ \isacommand{by}\isamarkupfalse%
\ force\isanewline
\isacommand{next}\isamarkupfalse%
\isanewline
\ \ \isacommand{case}\isamarkupfalse%
\ {\isacharparenleft}{\kern0pt}add\ u\ v{\isacharparenright}{\kern0pt}\isanewline
\ \ \isacommand{have}\isamarkupfalse%
\ {\isachardoublequoteopen}AE\ x\ in\ M{\isachardot}{\kern0pt}\ norm\ {\isacharparenleft}{\kern0pt}cond{\isacharunderscore}{\kern0pt}exp\ M\ F\ {\isacharparenleft}{\kern0pt}{\isasymlambda}a{\isachardot}{\kern0pt}\ u\ a\ {\isacharplus}{\kern0pt}\ v\ a{\isacharparenright}{\kern0pt}\ x{\isacharparenright}{\kern0pt}\ {\isacharequal}{\kern0pt}\ norm\ {\isacharparenleft}{\kern0pt}cond{\isacharunderscore}{\kern0pt}exp\ M\ F\ u\ x\ {\isacharplus}{\kern0pt}\ cond{\isacharunderscore}{\kern0pt}exp\ M\ F\ v\ x{\isacharparenright}{\kern0pt}{\isachardoublequoteclose}\ \isacommand{using}\isamarkupfalse%
\ has{\isacharunderscore}{\kern0pt}cond{\isacharunderscore}{\kern0pt}exp{\isacharunderscore}{\kern0pt}charact{\isacharparenleft}{\kern0pt}{\isadigit{2}}{\isacharparenright}{\kern0pt}{\isacharbrackleft}{\kern0pt}OF\ has{\isacharunderscore}{\kern0pt}cond{\isacharunderscore}{\kern0pt}exp{\isacharunderscore}{\kern0pt}add{\isacharcomma}{\kern0pt}\ OF\ has{\isacharunderscore}{\kern0pt}cond{\isacharunderscore}{\kern0pt}exp{\isacharunderscore}{\kern0pt}simple{\isacharparenleft}{\kern0pt}{\isadigit{1}}{\isacharcomma}{\kern0pt}{\isadigit{1}}{\isacharparenright}{\kern0pt}{\isacharcomma}{\kern0pt}\ OF\ add{\isacharparenleft}{\kern0pt}{\isadigit{1}}{\isacharcomma}{\kern0pt}{\isadigit{2}}{\isacharcomma}{\kern0pt}{\isadigit{3}}{\isacharcomma}{\kern0pt}{\isadigit{4}}{\isacharparenright}{\kern0pt}{\isacharbrackright}{\kern0pt}\ \isacommand{by}\isamarkupfalse%
\ fastforce\isanewline
\ \ \isacommand{moreover}\isamarkupfalse%
\ \isacommand{have}\isamarkupfalse%
\ {\isachardoublequoteopen}AE\ x\ in\ M{\isachardot}{\kern0pt}\ norm\ {\isacharparenleft}{\kern0pt}cond{\isacharunderscore}{\kern0pt}exp\ M\ F\ u\ x\ {\isacharplus}{\kern0pt}\ cond{\isacharunderscore}{\kern0pt}exp\ M\ F\ v\ x{\isacharparenright}{\kern0pt}\ {\isasymle}\ norm\ {\isacharparenleft}{\kern0pt}cond{\isacharunderscore}{\kern0pt}exp\ M\ F\ u\ x{\isacharparenright}{\kern0pt}\ {\isacharplus}{\kern0pt}\ norm\ {\isacharparenleft}{\kern0pt}cond{\isacharunderscore}{\kern0pt}exp\ M\ F\ v\ x{\isacharparenright}{\kern0pt}{\isachardoublequoteclose}\ \isacommand{using}\isamarkupfalse%
\ norm{\isacharunderscore}{\kern0pt}triangle{\isacharunderscore}{\kern0pt}ineq\ \isacommand{by}\isamarkupfalse%
\ blast\isanewline
\ \ \isacommand{moreover}\isamarkupfalse%
\ \isacommand{have}\isamarkupfalse%
\ {\isachardoublequoteopen}AE\ x\ in\ M{\isachardot}{\kern0pt}\ norm\ {\isacharparenleft}{\kern0pt}cond{\isacharunderscore}{\kern0pt}exp\ M\ F\ u\ x{\isacharparenright}{\kern0pt}\ {\isacharplus}{\kern0pt}\ norm\ {\isacharparenleft}{\kern0pt}cond{\isacharunderscore}{\kern0pt}exp\ M\ F\ v\ x{\isacharparenright}{\kern0pt}\ {\isasymle}\ cond{\isacharunderscore}{\kern0pt}exp\ M\ F\ {\isacharparenleft}{\kern0pt}{\isasymlambda}x{\isachardot}{\kern0pt}\ norm\ {\isacharparenleft}{\kern0pt}u\ x{\isacharparenright}{\kern0pt}{\isacharparenright}{\kern0pt}\ x\ {\isacharplus}{\kern0pt}\ cond{\isacharunderscore}{\kern0pt}exp\ M\ F\ {\isacharparenleft}{\kern0pt}{\isasymlambda}x{\isachardot}{\kern0pt}\ norm\ {\isacharparenleft}{\kern0pt}v\ x{\isacharparenright}{\kern0pt}{\isacharparenright}{\kern0pt}\ x{\isachardoublequoteclose}\ \isacommand{using}\isamarkupfalse%
\ add{\isacharparenleft}{\kern0pt}{\isadigit{6}}{\isacharcomma}{\kern0pt}{\isadigit{7}}{\isacharparenright}{\kern0pt}\ \isacommand{by}\isamarkupfalse%
\ fastforce\isanewline
\ \ \isacommand{moreover}\isamarkupfalse%
\ \isacommand{have}\isamarkupfalse%
\ {\isachardoublequoteopen}AE\ x\ in\ M{\isachardot}{\kern0pt}\ cond{\isacharunderscore}{\kern0pt}exp\ M\ F\ {\isacharparenleft}{\kern0pt}{\isasymlambda}x{\isachardot}{\kern0pt}\ norm\ {\isacharparenleft}{\kern0pt}u\ x{\isacharparenright}{\kern0pt}{\isacharparenright}{\kern0pt}\ x\ {\isacharplus}{\kern0pt}\ cond{\isacharunderscore}{\kern0pt}exp\ M\ F\ {\isacharparenleft}{\kern0pt}{\isasymlambda}x{\isachardot}{\kern0pt}\ norm\ {\isacharparenleft}{\kern0pt}v\ x{\isacharparenright}{\kern0pt}{\isacharparenright}{\kern0pt}\ x\ {\isacharequal}{\kern0pt}\ cond{\isacharunderscore}{\kern0pt}exp\ M\ F\ {\isacharparenleft}{\kern0pt}{\isasymlambda}x{\isachardot}{\kern0pt}\ norm\ {\isacharparenleft}{\kern0pt}u\ x{\isacharparenright}{\kern0pt}\ {\isacharplus}{\kern0pt}\ norm\ {\isacharparenleft}{\kern0pt}v\ x{\isacharparenright}{\kern0pt}{\isacharparenright}{\kern0pt}\ x{\isachardoublequoteclose}\ \isacommand{using}\isamarkupfalse%
\ integrable{\isacharunderscore}{\kern0pt}simple{\isacharunderscore}{\kern0pt}function{\isacharbrackleft}{\kern0pt}OF\ add{\isacharparenleft}{\kern0pt}{\isadigit{1}}{\isacharcomma}{\kern0pt}{\isadigit{2}}{\isacharparenright}{\kern0pt}{\isacharbrackright}{\kern0pt}\ integrable{\isacharunderscore}{\kern0pt}simple{\isacharunderscore}{\kern0pt}function{\isacharbrackleft}{\kern0pt}OF\ add{\isacharparenleft}{\kern0pt}{\isadigit{3}}{\isacharcomma}{\kern0pt}{\isadigit{4}}{\isacharparenright}{\kern0pt}{\isacharbrackright}{\kern0pt}\ \isacommand{by}\isamarkupfalse%
\ {\isacharparenleft}{\kern0pt}intro\ has{\isacharunderscore}{\kern0pt}cond{\isacharunderscore}{\kern0pt}exp{\isacharunderscore}{\kern0pt}charact{\isacharparenleft}{\kern0pt}{\isadigit{2}}{\isacharparenright}{\kern0pt}{\isacharbrackleft}{\kern0pt}OF\ has{\isacharunderscore}{\kern0pt}cond{\isacharunderscore}{\kern0pt}exp{\isacharunderscore}{\kern0pt}add{\isacharbrackleft}{\kern0pt}OF\ has{\isacharunderscore}{\kern0pt}cond{\isacharunderscore}{\kern0pt}exp{\isacharunderscore}{\kern0pt}charact{\isacharparenleft}{\kern0pt}{\isadigit{1}}{\isacharcomma}{\kern0pt}{\isadigit{1}}{\isacharparenright}{\kern0pt}{\isacharbrackright}{\kern0pt}{\isacharcomma}{\kern0pt}\ THEN\ AE{\isacharunderscore}{\kern0pt}symmetric{\isacharbrackright}{\kern0pt}{\isacharcomma}{\kern0pt}\ auto{\isacharparenright}{\kern0pt}\isanewline
\ \ \isacommand{moreover}\isamarkupfalse%
\ \isacommand{have}\isamarkupfalse%
\ {\isachardoublequoteopen}AE\ x\ in\ M{\isachardot}{\kern0pt}\ cond{\isacharunderscore}{\kern0pt}exp\ M\ F\ {\isacharparenleft}{\kern0pt}{\isasymlambda}x{\isachardot}{\kern0pt}\ norm\ {\isacharparenleft}{\kern0pt}u\ x{\isacharparenright}{\kern0pt}\ {\isacharplus}{\kern0pt}\ norm\ {\isacharparenleft}{\kern0pt}v\ x{\isacharparenright}{\kern0pt}{\isacharparenright}{\kern0pt}\ x\ {\isacharequal}{\kern0pt}\ cond{\isacharunderscore}{\kern0pt}exp\ M\ F\ {\isacharparenleft}{\kern0pt}{\isasymlambda}x{\isachardot}{\kern0pt}\ norm\ {\isacharparenleft}{\kern0pt}u\ x\ {\isacharplus}{\kern0pt}\ v\ x{\isacharparenright}{\kern0pt}{\isacharparenright}{\kern0pt}\ x{\isachardoublequoteclose}\ \isacommand{using}\isamarkupfalse%
\ add{\isacharparenleft}{\kern0pt}{\isadigit{5}}{\isacharparenright}{\kern0pt}\ integrable{\isacharunderscore}{\kern0pt}simple{\isacharunderscore}{\kern0pt}function{\isacharbrackleft}{\kern0pt}OF\ add{\isacharparenleft}{\kern0pt}{\isadigit{1}}{\isacharcomma}{\kern0pt}{\isadigit{2}}{\isacharparenright}{\kern0pt}{\isacharbrackright}{\kern0pt}\ integrable{\isacharunderscore}{\kern0pt}simple{\isacharunderscore}{\kern0pt}function{\isacharbrackleft}{\kern0pt}OF\ add{\isacharparenleft}{\kern0pt}{\isadigit{3}}{\isacharcomma}{\kern0pt}{\isadigit{4}}{\isacharparenright}{\kern0pt}{\isacharbrackright}{\kern0pt}\ \isacommand{by}\isamarkupfalse%
\ {\isacharparenleft}{\kern0pt}intro\ cond{\isacharunderscore}{\kern0pt}exp{\isacharunderscore}{\kern0pt}cong{\isacharcomma}{\kern0pt}\ auto{\isacharparenright}{\kern0pt}\isanewline
\ \ \isacommand{ultimately}\isamarkupfalse%
\ \isacommand{show}\isamarkupfalse%
\ {\isacharquery}{\kern0pt}case\ \isacommand{by}\isamarkupfalse%
\ force\isanewline
\isacommand{qed}\isamarkupfalse%
%
\endisatagproof
{\isafoldproof}%
%
\isadelimproof
\isanewline
%
\endisadelimproof
\isanewline
\isacommand{lemma}\isamarkupfalse%
\ has{\isacharunderscore}{\kern0pt}cond{\isacharunderscore}{\kern0pt}exp{\isacharunderscore}{\kern0pt}lim{\isacharcolon}{\kern0pt}\isanewline
\ \ \ \ \isakeyword{fixes}\ f\ {\isacharcolon}{\kern0pt}{\isacharcolon}{\kern0pt}\ {\isachardoublequoteopen}{\isacharprime}{\kern0pt}a\ {\isasymRightarrow}\ {\isacharprime}{\kern0pt}b{\isacharcolon}{\kern0pt}{\isacharcolon}{\kern0pt}{\isacharbraceleft}{\kern0pt}second{\isacharunderscore}{\kern0pt}countable{\isacharunderscore}{\kern0pt}topology{\isacharcomma}{\kern0pt}\ banach{\isacharbraceright}{\kern0pt}{\isachardoublequoteclose}\isanewline
\ \ \isakeyword{assumes}\ integrable{\isacharbrackleft}{\kern0pt}measurable{\isacharbrackright}{\kern0pt}{\isacharcolon}{\kern0pt}\ {\isachardoublequoteopen}integrable\ M\ f{\isachardoublequoteclose}\isanewline
\ \ \ \ \ \ \isakeyword{and}\ {\isachardoublequoteopen}{\isasymAnd}i{\isachardot}{\kern0pt}\ simple{\isacharunderscore}{\kern0pt}function\ M\ {\isacharparenleft}{\kern0pt}s\ i{\isacharparenright}{\kern0pt}{\isachardoublequoteclose}\isanewline
\ \ \ \ \ \ \isakeyword{and}\ {\isachardoublequoteopen}{\isasymAnd}i{\isachardot}{\kern0pt}\ emeasure\ M\ {\isacharbraceleft}{\kern0pt}y\ {\isasymin}\ space\ M{\isachardot}{\kern0pt}\ s\ i\ y\ {\isasymnoteq}\ {\isadigit{0}}{\isacharbraceright}{\kern0pt}\ {\isasymnoteq}\ {\isasyminfinity}{\isachardoublequoteclose}\isanewline
\ \ \ \ \ \ \isakeyword{and}\ {\isachardoublequoteopen}{\isasymAnd}x{\isachardot}{\kern0pt}\ x\ {\isasymin}\ space\ M\ {\isasymLongrightarrow}\ {\isacharparenleft}{\kern0pt}{\isasymlambda}i{\isachardot}{\kern0pt}\ s\ i\ x{\isacharparenright}{\kern0pt}\ {\isasymlonglonglongrightarrow}\ f\ x{\isachardoublequoteclose}\isanewline
\ \ \ \ \ \ \isakeyword{and}\ {\isachardoublequoteopen}{\isasymAnd}x\ i{\isachardot}{\kern0pt}\ x\ {\isasymin}\ space\ M\ {\isasymLongrightarrow}\ norm\ {\isacharparenleft}{\kern0pt}s\ i\ x{\isacharparenright}{\kern0pt}\ {\isasymle}\ {\isadigit{2}}\ {\isacharasterisk}{\kern0pt}\ norm\ {\isacharparenleft}{\kern0pt}f\ x{\isacharparenright}{\kern0pt}{\isachardoublequoteclose}\isanewline
\ \ \isakeyword{obtains}\ r\ \isanewline
\ \ \isakeyword{where}\ {\isachardoublequoteopen}has{\isacharunderscore}{\kern0pt}cond{\isacharunderscore}{\kern0pt}exp\ M\ F\ f\ {\isacharparenleft}{\kern0pt}{\isasymlambda}x{\isachardot}{\kern0pt}\ lim\ {\isacharparenleft}{\kern0pt}{\isasymlambda}i{\isachardot}{\kern0pt}\ cond{\isacharunderscore}{\kern0pt}exp\ M\ F\ {\isacharparenleft}{\kern0pt}s\ {\isacharparenleft}{\kern0pt}r\ i{\isacharparenright}{\kern0pt}{\isacharparenright}{\kern0pt}\ x{\isacharparenright}{\kern0pt}{\isacharparenright}{\kern0pt}{\isachardoublequoteclose}\ \isanewline
\ \ \ \ \ \ \ \ {\isachardoublequoteopen}AE\ x\ in\ M{\isachardot}{\kern0pt}\ convergent\ {\isacharparenleft}{\kern0pt}{\isasymlambda}i{\isachardot}{\kern0pt}\ cond{\isacharunderscore}{\kern0pt}exp\ M\ F\ {\isacharparenleft}{\kern0pt}s\ {\isacharparenleft}{\kern0pt}r\ i{\isacharparenright}{\kern0pt}{\isacharparenright}{\kern0pt}\ x{\isacharparenright}{\kern0pt}{\isachardoublequoteclose}\isanewline
\ \ \ \ \ \ \ \ {\isachardoublequoteopen}strict{\isacharunderscore}{\kern0pt}mono\ r{\isachardoublequoteclose}\isanewline
%
\isadelimproof
%
\endisadelimproof
%
\isatagproof
\isacommand{proof}\isamarkupfalse%
\ {\isacharminus}{\kern0pt}\isanewline
\ \ \isacommand{have}\isamarkupfalse%
\ {\isacharbrackleft}{\kern0pt}measurable{\isacharbrackright}{\kern0pt}{\isacharcolon}{\kern0pt}\ {\isachardoublequoteopen}{\isacharparenleft}{\kern0pt}s\ i{\isacharparenright}{\kern0pt}\ {\isasymin}\ borel{\isacharunderscore}{\kern0pt}measurable\ M{\isachardoublequoteclose}\ \isakeyword{for}\ i\ \isacommand{using}\isamarkupfalse%
\ assms{\isacharparenleft}{\kern0pt}{\isadigit{2}}{\isacharparenright}{\kern0pt}\ \isacommand{by}\isamarkupfalse%
\ {\isacharparenleft}{\kern0pt}simp\ add{\isacharcolon}{\kern0pt}\ borel{\isacharunderscore}{\kern0pt}measurable{\isacharunderscore}{\kern0pt}simple{\isacharunderscore}{\kern0pt}function{\isacharparenright}{\kern0pt}\isanewline
\ \ \isacommand{have}\isamarkupfalse%
\ integrable{\isacharunderscore}{\kern0pt}s{\isacharcolon}{\kern0pt}\ {\isachardoublequoteopen}integrable\ M\ {\isacharparenleft}{\kern0pt}{\isasymlambda}x{\isachardot}{\kern0pt}\ s\ i\ x{\isacharparenright}{\kern0pt}{\isachardoublequoteclose}\ \isakeyword{for}\ i\ \isacommand{using}\isamarkupfalse%
\ assms{\isacharparenleft}{\kern0pt}{\isadigit{2}}{\isacharparenright}{\kern0pt}\ assms{\isacharparenleft}{\kern0pt}{\isadigit{3}}{\isacharparenright}{\kern0pt}\ integrable{\isacharunderscore}{\kern0pt}simple{\isacharunderscore}{\kern0pt}function\ \isacommand{by}\isamarkupfalse%
\ blast\isanewline
\ \ \isacommand{have}\isamarkupfalse%
\ integrable{\isacharunderscore}{\kern0pt}{\isadigit{4}}f{\isacharcolon}{\kern0pt}\ {\isachardoublequoteopen}integrable\ M\ {\isacharparenleft}{\kern0pt}{\isasymlambda}x{\isachardot}{\kern0pt}\ {\isadigit{4}}\ {\isacharasterisk}{\kern0pt}\ norm\ {\isacharparenleft}{\kern0pt}f\ x{\isacharparenright}{\kern0pt}{\isacharparenright}{\kern0pt}{\isachardoublequoteclose}\ \isacommand{using}\isamarkupfalse%
\ assms{\isacharparenleft}{\kern0pt}{\isadigit{1}}{\isacharparenright}{\kern0pt}\ \isacommand{by}\isamarkupfalse%
\ simp\isanewline
\ \ \isacommand{have}\isamarkupfalse%
\ integrable{\isacharunderscore}{\kern0pt}{\isadigit{2}}f{\isacharcolon}{\kern0pt}\ {\isachardoublequoteopen}integrable\ M\ {\isacharparenleft}{\kern0pt}{\isasymlambda}x{\isachardot}{\kern0pt}\ {\isadigit{2}}\ {\isacharasterisk}{\kern0pt}\ norm\ {\isacharparenleft}{\kern0pt}f\ x{\isacharparenright}{\kern0pt}{\isacharparenright}{\kern0pt}{\isachardoublequoteclose}\ \isacommand{using}\isamarkupfalse%
\ assms{\isacharparenleft}{\kern0pt}{\isadigit{1}}{\isacharparenright}{\kern0pt}\ \isacommand{by}\isamarkupfalse%
\ simp\isanewline
\ \ \isacommand{have}\isamarkupfalse%
\ integrable{\isacharunderscore}{\kern0pt}{\isadigit{2}}{\isacharunderscore}{\kern0pt}cond{\isacharunderscore}{\kern0pt}exp{\isacharunderscore}{\kern0pt}norm{\isacharunderscore}{\kern0pt}f{\isacharcolon}{\kern0pt}\ {\isachardoublequoteopen}integrable\ M\ {\isacharparenleft}{\kern0pt}{\isasymlambda}x{\isachardot}{\kern0pt}\ {\isadigit{2}}\ {\isacharasterisk}{\kern0pt}\ cond{\isacharunderscore}{\kern0pt}exp\ M\ F\ {\isacharparenleft}{\kern0pt}{\isasymlambda}x{\isachardot}{\kern0pt}\ norm\ {\isacharparenleft}{\kern0pt}f\ x{\isacharparenright}{\kern0pt}{\isacharparenright}{\kern0pt}\ x{\isacharparenright}{\kern0pt}{\isachardoublequoteclose}\ \isacommand{by}\isamarkupfalse%
\ fast\isanewline
\isanewline
\ \ \isacommand{have}\isamarkupfalse%
\ {\isachardoublequoteopen}emeasure\ M\ {\isacharbraceleft}{\kern0pt}y\ {\isasymin}\ space\ M{\isachardot}{\kern0pt}\ s\ i\ y\ {\isacharminus}{\kern0pt}\ s\ j\ y\ {\isasymnoteq}\ {\isadigit{0}}{\isacharbraceright}{\kern0pt}\ {\isasymle}\ \ emeasure\ M\ {\isacharbraceleft}{\kern0pt}y\ {\isasymin}\ space\ M{\isachardot}{\kern0pt}\ s\ i\ y\ {\isasymnoteq}\ {\isadigit{0}}{\isacharbraceright}{\kern0pt}\ {\isacharplus}{\kern0pt}\ emeasure\ M\ {\isacharbraceleft}{\kern0pt}y\ {\isasymin}\ space\ M{\isachardot}{\kern0pt}\ s\ j\ y\ {\isasymnoteq}\ {\isadigit{0}}{\isacharbraceright}{\kern0pt}{\isachardoublequoteclose}\ \isakeyword{for}\ i\ j\ \isacommand{using}\isamarkupfalse%
\ simple{\isacharunderscore}{\kern0pt}functionD{\isacharparenleft}{\kern0pt}{\isadigit{2}}{\isacharparenright}{\kern0pt}{\isacharbrackleft}{\kern0pt}OF\ assms{\isacharparenleft}{\kern0pt}{\isadigit{2}}{\isacharparenright}{\kern0pt}{\isacharbrackright}{\kern0pt}\ \isacommand{by}\isamarkupfalse%
\ {\isacharparenleft}{\kern0pt}intro\ order{\isacharunderscore}{\kern0pt}trans{\isacharbrackleft}{\kern0pt}OF\ emeasure{\isacharunderscore}{\kern0pt}mono\ emeasure{\isacharunderscore}{\kern0pt}subadditive{\isacharbrackright}{\kern0pt}{\isacharcomma}{\kern0pt}\ auto{\isacharparenright}{\kern0pt}\isanewline
\ \ \isacommand{hence}\isamarkupfalse%
\ fin{\isacharunderscore}{\kern0pt}sup{\isacharcolon}{\kern0pt}\ {\isachardoublequoteopen}emeasure\ M\ {\isacharbraceleft}{\kern0pt}y\ {\isasymin}\ space\ M{\isachardot}{\kern0pt}\ s\ i\ y\ {\isacharminus}{\kern0pt}\ s\ j\ y\ {\isasymnoteq}\ {\isadigit{0}}{\isacharbraceright}{\kern0pt}\ {\isasymnoteq}\ {\isasyminfinity}{\isachardoublequoteclose}\ \isakeyword{for}\ i\ j\ \isacommand{using}\isamarkupfalse%
\ assms{\isacharparenleft}{\kern0pt}{\isadigit{3}}{\isacharparenright}{\kern0pt}\ \isacommand{by}\isamarkupfalse%
\ {\isacharparenleft}{\kern0pt}metis\ {\isacharparenleft}{\kern0pt}mono{\isacharunderscore}{\kern0pt}tags{\isacharparenright}{\kern0pt}\ ennreal{\isacharunderscore}{\kern0pt}add{\isacharunderscore}{\kern0pt}eq{\isacharunderscore}{\kern0pt}top\ linorder{\isacharunderscore}{\kern0pt}not{\isacharunderscore}{\kern0pt}less\ top{\isachardot}{\kern0pt}not{\isacharunderscore}{\kern0pt}eq{\isacharunderscore}{\kern0pt}extremum\ infinity{\isacharunderscore}{\kern0pt}ennreal{\isacharunderscore}{\kern0pt}def{\isacharparenright}{\kern0pt}\isanewline
\isanewline
\ \ \isacommand{have}\isamarkupfalse%
\ {\isachardoublequoteopen}emeasure\ M\ {\isacharbraceleft}{\kern0pt}y\ {\isasymin}\ space\ M{\isachardot}{\kern0pt}\ norm\ {\isacharparenleft}{\kern0pt}s\ i\ y\ {\isacharminus}{\kern0pt}\ s\ j\ y{\isacharparenright}{\kern0pt}\ {\isasymnoteq}\ {\isadigit{0}}{\isacharbraceright}{\kern0pt}\ {\isasymle}\ \ emeasure\ M\ {\isacharbraceleft}{\kern0pt}y\ {\isasymin}\ space\ M{\isachardot}{\kern0pt}\ s\ i\ y\ {\isasymnoteq}\ {\isadigit{0}}{\isacharbraceright}{\kern0pt}\ {\isacharplus}{\kern0pt}\ emeasure\ M\ {\isacharbraceleft}{\kern0pt}y\ {\isasymin}\ space\ M{\isachardot}{\kern0pt}\ s\ j\ y\ {\isasymnoteq}\ {\isadigit{0}}{\isacharbraceright}{\kern0pt}{\isachardoublequoteclose}\ \isakeyword{for}\ i\ j\ \isacommand{using}\isamarkupfalse%
\ simple{\isacharunderscore}{\kern0pt}functionD{\isacharparenleft}{\kern0pt}{\isadigit{2}}{\isacharparenright}{\kern0pt}{\isacharbrackleft}{\kern0pt}OF\ assms{\isacharparenleft}{\kern0pt}{\isadigit{2}}{\isacharparenright}{\kern0pt}{\isacharbrackright}{\kern0pt}\ \isacommand{by}\isamarkupfalse%
\ {\isacharparenleft}{\kern0pt}intro\ order{\isacharunderscore}{\kern0pt}trans{\isacharbrackleft}{\kern0pt}OF\ emeasure{\isacharunderscore}{\kern0pt}mono\ emeasure{\isacharunderscore}{\kern0pt}subadditive{\isacharbrackright}{\kern0pt}{\isacharcomma}{\kern0pt}\ auto{\isacharparenright}{\kern0pt}\isanewline
\ \ \isacommand{hence}\isamarkupfalse%
\ fin{\isacharunderscore}{\kern0pt}sup{\isacharunderscore}{\kern0pt}norm{\isacharcolon}{\kern0pt}\ {\isachardoublequoteopen}emeasure\ M\ {\isacharbraceleft}{\kern0pt}y\ {\isasymin}\ space\ M{\isachardot}{\kern0pt}\ norm\ {\isacharparenleft}{\kern0pt}s\ i\ y\ {\isacharminus}{\kern0pt}\ s\ j\ y{\isacharparenright}{\kern0pt}\ {\isasymnoteq}\ {\isadigit{0}}{\isacharbraceright}{\kern0pt}\ {\isasymnoteq}\ {\isasyminfinity}{\isachardoublequoteclose}\ \isakeyword{for}\ i\ j\ \isacommand{using}\isamarkupfalse%
\ assms{\isacharparenleft}{\kern0pt}{\isadigit{3}}{\isacharparenright}{\kern0pt}\ \isacommand{by}\isamarkupfalse%
\ {\isacharparenleft}{\kern0pt}metis\ {\isacharparenleft}{\kern0pt}mono{\isacharunderscore}{\kern0pt}tags{\isacharparenright}{\kern0pt}\ ennreal{\isacharunderscore}{\kern0pt}add{\isacharunderscore}{\kern0pt}eq{\isacharunderscore}{\kern0pt}top\ linorder{\isacharunderscore}{\kern0pt}not{\isacharunderscore}{\kern0pt}less\ top{\isachardot}{\kern0pt}not{\isacharunderscore}{\kern0pt}eq{\isacharunderscore}{\kern0pt}extremum\ infinity{\isacharunderscore}{\kern0pt}ennreal{\isacharunderscore}{\kern0pt}def{\isacharparenright}{\kern0pt}\isanewline
\isanewline
\ \ \isacommand{have}\isamarkupfalse%
\ Cauchy{\isacharcolon}{\kern0pt}\ {\isachardoublequoteopen}Cauchy\ {\isacharparenleft}{\kern0pt}{\isasymlambda}n{\isachardot}{\kern0pt}\ s\ n\ x{\isacharparenright}{\kern0pt}{\isachardoublequoteclose}\ \isakeyword{if}\ {\isachardoublequoteopen}x\ {\isasymin}\ space\ M{\isachardoublequoteclose}\ \isakeyword{for}\ x\ \isacommand{using}\isamarkupfalse%
\ assms{\isacharparenleft}{\kern0pt}{\isadigit{4}}{\isacharparenright}{\kern0pt}\ LIMSEQ{\isacharunderscore}{\kern0pt}imp{\isacharunderscore}{\kern0pt}Cauchy\ that\ \isacommand{by}\isamarkupfalse%
\ blast\isanewline
\ \ \isacommand{hence}\isamarkupfalse%
\ bounded{\isacharunderscore}{\kern0pt}range{\isacharunderscore}{\kern0pt}s{\isacharcolon}{\kern0pt}\ {\isachardoublequoteopen}bounded\ {\isacharparenleft}{\kern0pt}range\ {\isacharparenleft}{\kern0pt}{\isasymlambda}n{\isachardot}{\kern0pt}\ s\ n\ x{\isacharparenright}{\kern0pt}{\isacharparenright}{\kern0pt}{\isachardoublequoteclose}\ \isakeyword{if}\ {\isachardoublequoteopen}x\ {\isasymin}\ space\ M{\isachardoublequoteclose}\ \isakeyword{for}\ x\ \isacommand{using}\isamarkupfalse%
\ that\ cauchy{\isacharunderscore}{\kern0pt}imp{\isacharunderscore}{\kern0pt}bounded\ \isacommand{by}\isamarkupfalse%
\ fast\isanewline
\isanewline
\ \ \isacommand{have}\isamarkupfalse%
\ {\isachardoublequoteopen}AE\ x\ in\ M{\isachardot}{\kern0pt}\ {\isacharparenleft}{\kern0pt}{\isasymlambda}n{\isachardot}{\kern0pt}\ diameter\ {\isacharbraceleft}{\kern0pt}s\ i\ x\ {\isacharbar}{\kern0pt}\ i{\isachardot}{\kern0pt}\ n\ {\isasymle}\ i{\isacharbraceright}{\kern0pt}{\isacharparenright}{\kern0pt}\ {\isasymlonglonglongrightarrow}\ {\isadigit{0}}{\isachardoublequoteclose}\ \isacommand{using}\isamarkupfalse%
\ Cauchy\ cauchy{\isacharunderscore}{\kern0pt}iff{\isacharunderscore}{\kern0pt}diameter{\isacharunderscore}{\kern0pt}tends{\isacharunderscore}{\kern0pt}to{\isacharunderscore}{\kern0pt}zero{\isacharunderscore}{\kern0pt}and{\isacharunderscore}{\kern0pt}bounded\ \isacommand{by}\isamarkupfalse%
\ fast\isanewline
\ \ \isacommand{moreover}\isamarkupfalse%
\ \isacommand{have}\isamarkupfalse%
\ {\isachardoublequoteopen}{\isacharparenleft}{\kern0pt}{\isasymlambda}x{\isachardot}{\kern0pt}\ diameter\ {\isacharbraceleft}{\kern0pt}s\ i\ x\ {\isacharbar}{\kern0pt}i{\isachardot}{\kern0pt}\ n\ {\isasymle}\ i{\isacharbraceright}{\kern0pt}{\isacharparenright}{\kern0pt}\ {\isasymin}\ borel{\isacharunderscore}{\kern0pt}measurable\ M{\isachardoublequoteclose}\ \isakeyword{for}\ n\ \isacommand{using}\isamarkupfalse%
\ bounded{\isacharunderscore}{\kern0pt}range{\isacharunderscore}{\kern0pt}s\ borel{\isacharunderscore}{\kern0pt}measurable{\isacharunderscore}{\kern0pt}diameter\ \isacommand{by}\isamarkupfalse%
\ measurable\isanewline
\ \ \isacommand{moreover}\isamarkupfalse%
\ \isacommand{have}\isamarkupfalse%
\ {\isachardoublequoteopen}AE\ x\ in\ M{\isachardot}{\kern0pt}\ norm\ {\isacharparenleft}{\kern0pt}diameter\ {\isacharbraceleft}{\kern0pt}s\ i\ x\ {\isacharbar}{\kern0pt}i{\isachardot}{\kern0pt}\ n\ {\isasymle}\ i{\isacharbraceright}{\kern0pt}{\isacharparenright}{\kern0pt}\ {\isasymle}\ {\isadigit{4}}\ {\isacharasterisk}{\kern0pt}\ norm\ {\isacharparenleft}{\kern0pt}f\ x{\isacharparenright}{\kern0pt}{\isachardoublequoteclose}\ \isakeyword{for}\ n\isanewline
\ \ \isacommand{proof}\isamarkupfalse%
\ {\isacharminus}{\kern0pt}\ \isanewline
\ \ \ \ \isacommand{{\isacharbraceleft}{\kern0pt}}\isamarkupfalse%
\isanewline
\ \ \ \ \ \ \isacommand{fix}\isamarkupfalse%
\ x\ \isacommand{assume}\isamarkupfalse%
\ x{\isacharcolon}{\kern0pt}\ {\isachardoublequoteopen}x\ {\isasymin}\ space\ M{\isachardoublequoteclose}\isanewline
\ \ \ \ \ \ \isacommand{have}\isamarkupfalse%
\ {\isachardoublequoteopen}diameter\ {\isacharbraceleft}{\kern0pt}s\ i\ x\ {\isacharbar}{\kern0pt}i{\isachardot}{\kern0pt}\ n\ {\isasymle}\ i{\isacharbraceright}{\kern0pt}\ {\isasymle}\ {\isadigit{2}}\ {\isacharasterisk}{\kern0pt}\ norm\ {\isacharparenleft}{\kern0pt}f\ x{\isacharparenright}{\kern0pt}\ {\isacharplus}{\kern0pt}\ {\isadigit{2}}\ {\isacharasterisk}{\kern0pt}\ norm\ {\isacharparenleft}{\kern0pt}f\ x{\isacharparenright}{\kern0pt}{\isachardoublequoteclose}\ \isacommand{by}\isamarkupfalse%
\ {\isacharparenleft}{\kern0pt}intro\ diameter{\isacharunderscore}{\kern0pt}le{\isacharcomma}{\kern0pt}\ blast{\isacharcomma}{\kern0pt}\ subst\ dist{\isacharunderscore}{\kern0pt}norm{\isacharbrackleft}{\kern0pt}symmetric{\isacharbrackright}{\kern0pt}{\isacharcomma}{\kern0pt}\ intro\ dist{\isacharunderscore}{\kern0pt}triangle{\isadigit{3}}{\isacharbrackleft}{\kern0pt}THEN\ order{\isacharunderscore}{\kern0pt}trans{\isacharcomma}{\kern0pt}\ of\ {\isadigit{0}}{\isacharbrackright}{\kern0pt}{\isacharcomma}{\kern0pt}\ intro\ add{\isacharunderscore}{\kern0pt}mono{\isacharparenright}{\kern0pt}\ {\isacharparenleft}{\kern0pt}auto\ intro{\isacharcolon}{\kern0pt}\ assms{\isacharparenleft}{\kern0pt}{\isadigit{5}}{\isacharparenright}{\kern0pt}{\isacharbrackleft}{\kern0pt}OF\ x{\isacharbrackright}{\kern0pt}{\isacharparenright}{\kern0pt}\isanewline
\ \ \ \ \ \ \isacommand{hence}\isamarkupfalse%
\ {\isachardoublequoteopen}norm\ {\isacharparenleft}{\kern0pt}diameter\ {\isacharbraceleft}{\kern0pt}s\ i\ x\ {\isacharbar}{\kern0pt}i{\isachardot}{\kern0pt}\ n\ {\isasymle}\ i{\isacharbraceright}{\kern0pt}{\isacharparenright}{\kern0pt}\ {\isasymle}\ {\isadigit{4}}\ {\isacharasterisk}{\kern0pt}\ norm\ {\isacharparenleft}{\kern0pt}f\ x{\isacharparenright}{\kern0pt}{\isachardoublequoteclose}\ \isacommand{using}\isamarkupfalse%
\ diameter{\isacharunderscore}{\kern0pt}ge{\isacharunderscore}{\kern0pt}{\isadigit{0}}{\isacharbrackleft}{\kern0pt}OF\ bounded{\isacharunderscore}{\kern0pt}subset{\isacharbrackleft}{\kern0pt}OF\ bounded{\isacharunderscore}{\kern0pt}range{\isacharunderscore}{\kern0pt}s{\isacharbrackright}{\kern0pt}{\isacharcomma}{\kern0pt}\ OF\ x{\isacharcomma}{\kern0pt}\ of\ {\isachardoublequoteopen}{\isacharbraceleft}{\kern0pt}s\ i\ x\ {\isacharbar}{\kern0pt}i{\isachardot}{\kern0pt}\ n\ {\isasymle}\ i{\isacharbraceright}{\kern0pt}{\isachardoublequoteclose}{\isacharbrackright}{\kern0pt}\ \isacommand{by}\isamarkupfalse%
\ force\isanewline
\ \ \ \ \isacommand{{\isacharbraceright}{\kern0pt}}\isamarkupfalse%
\isanewline
\ \ \ \ \isacommand{thus}\isamarkupfalse%
\ {\isacharquery}{\kern0pt}thesis\ \isacommand{by}\isamarkupfalse%
\ fast\isanewline
\ \ \isacommand{qed}\isamarkupfalse%
\isanewline
\ \ \isacommand{ultimately}\isamarkupfalse%
\ \isacommand{have}\isamarkupfalse%
\ diameter{\isacharunderscore}{\kern0pt}tendsto{\isacharunderscore}{\kern0pt}zero{\isacharcolon}{\kern0pt}\ {\isachardoublequoteopen}{\isacharparenleft}{\kern0pt}{\isasymlambda}n{\isachardot}{\kern0pt}\ LINT\ x{\isacharbar}{\kern0pt}M{\isachardot}{\kern0pt}\ diameter\ {\isacharbraceleft}{\kern0pt}s\ i\ x\ {\isacharbar}{\kern0pt}\ i{\isachardot}{\kern0pt}\ n\ {\isasymle}\ i{\isacharbraceright}{\kern0pt}{\isacharparenright}{\kern0pt}\ {\isasymlonglonglongrightarrow}\ {\isadigit{0}}{\isachardoublequoteclose}\ \isacommand{by}\isamarkupfalse%
\ {\isacharparenleft}{\kern0pt}intro\ integral{\isacharunderscore}{\kern0pt}dominated{\isacharunderscore}{\kern0pt}convergence{\isacharbrackleft}{\kern0pt}OF\ borel{\isacharunderscore}{\kern0pt}measurable{\isacharunderscore}{\kern0pt}const{\isacharbrackleft}{\kern0pt}of\ {\isadigit{0}}{\isacharbrackright}{\kern0pt}\ {\isacharunderscore}{\kern0pt}\ integrable{\isacharunderscore}{\kern0pt}{\isadigit{4}}f{\isacharcomma}{\kern0pt}\ simplified{\isacharbrackright}{\kern0pt}{\isacharparenright}{\kern0pt}\ {\isacharparenleft}{\kern0pt}fast{\isacharplus}{\kern0pt}{\isacharparenright}{\kern0pt}\isanewline
\ \ \isanewline
\ \ \isacommand{have}\isamarkupfalse%
\ diameter{\isacharunderscore}{\kern0pt}integrable{\isacharcolon}{\kern0pt}\ {\isachardoublequoteopen}integrable\ M\ {\isacharparenleft}{\kern0pt}{\isasymlambda}x{\isachardot}{\kern0pt}\ diameter\ {\isacharbraceleft}{\kern0pt}s\ i\ x\ {\isacharbar}{\kern0pt}\ i{\isachardot}{\kern0pt}\ n\ {\isasymle}\ i{\isacharbraceright}{\kern0pt}{\isacharparenright}{\kern0pt}{\isachardoublequoteclose}\ \isakeyword{for}\ n\ \isacommand{using}\isamarkupfalse%
\ assms{\isacharparenleft}{\kern0pt}{\isadigit{1}}{\isacharcomma}{\kern0pt}{\isadigit{5}}{\isacharparenright}{\kern0pt}\ \isacommand{by}\isamarkupfalse%
\ {\isacharparenleft}{\kern0pt}intro\ integrable{\isacharunderscore}{\kern0pt}bound{\isacharunderscore}{\kern0pt}diameter{\isacharbrackleft}{\kern0pt}OF\ bounded{\isacharunderscore}{\kern0pt}range{\isacharunderscore}{\kern0pt}s\ integrable{\isacharunderscore}{\kern0pt}{\isadigit{2}}f{\isacharbrackright}{\kern0pt}{\isacharcomma}{\kern0pt}\ auto{\isacharparenright}{\kern0pt}\isanewline
\isanewline
\ \ \isacommand{have}\isamarkupfalse%
\ dist{\isacharunderscore}{\kern0pt}integrable{\isacharcolon}{\kern0pt}\ {\isachardoublequoteopen}integrable\ M\ {\isacharparenleft}{\kern0pt}{\isasymlambda}x{\isachardot}{\kern0pt}\ dist\ {\isacharparenleft}{\kern0pt}s\ i\ x{\isacharparenright}{\kern0pt}\ {\isacharparenleft}{\kern0pt}s\ j\ x{\isacharparenright}{\kern0pt}{\isacharparenright}{\kern0pt}{\isachardoublequoteclose}\ \isakeyword{for}\ i\ j\ \isanewline
\ \ \ \ \isacommand{using}\isamarkupfalse%
\ assms{\isacharparenleft}{\kern0pt}{\isadigit{5}}{\isacharparenright}{\kern0pt}\ dist{\isacharunderscore}{\kern0pt}triangle{\isadigit{3}}{\isacharbrackleft}{\kern0pt}of\ {\isachardoublequoteopen}s\ i\ {\isacharunderscore}{\kern0pt}{\isachardoublequoteclose}\ {\isacharunderscore}{\kern0pt}\ {\isadigit{0}}{\isacharcomma}{\kern0pt}\ THEN\ order{\isacharunderscore}{\kern0pt}trans{\isacharcomma}{\kern0pt}\ OF\ add{\isacharunderscore}{\kern0pt}mono{\isacharcomma}{\kern0pt}\ of\ {\isacharunderscore}{\kern0pt}\ {\isachardoublequoteopen}{\isadigit{2}}\ {\isacharasterisk}{\kern0pt}\ norm\ {\isacharparenleft}{\kern0pt}f\ {\isacharunderscore}{\kern0pt}{\isacharparenright}{\kern0pt}{\isachardoublequoteclose}{\isacharbrackright}{\kern0pt}\isanewline
\ \ \ \ \isacommand{by}\isamarkupfalse%
\ {\isacharparenleft}{\kern0pt}intro\ Bochner{\isacharunderscore}{\kern0pt}Integration{\isachardot}{\kern0pt}integrable{\isacharunderscore}{\kern0pt}bound{\isacharbrackleft}{\kern0pt}OF\ integrable{\isacharunderscore}{\kern0pt}{\isadigit{4}}f{\isacharbrackright}{\kern0pt}{\isacharparenright}{\kern0pt}\ fastforce{\isacharplus}{\kern0pt}\isanewline
\ \ \ \isanewline
\ \ \isacommand{hence}\isamarkupfalse%
\ dist{\isacharunderscore}{\kern0pt}norm{\isacharunderscore}{\kern0pt}integrable{\isacharcolon}{\kern0pt}\ {\isachardoublequoteopen}integrable\ M\ {\isacharparenleft}{\kern0pt}{\isasymlambda}x{\isachardot}{\kern0pt}\ norm\ {\isacharparenleft}{\kern0pt}s\ i\ x\ {\isacharminus}{\kern0pt}\ s\ j\ x{\isacharparenright}{\kern0pt}{\isacharparenright}{\kern0pt}{\isachardoublequoteclose}\ \isakeyword{for}\ i\ j\ \isacommand{unfolding}\isamarkupfalse%
\ dist{\isacharunderscore}{\kern0pt}norm\ \isacommand{by}\isamarkupfalse%
\ presburger\isanewline
\isanewline
\ \ \isacommand{have}\isamarkupfalse%
\ {\isachardoublequoteopen}{\isasymexists}N{\isachardot}{\kern0pt}\ {\isasymforall}i{\isasymge}N{\isachardot}{\kern0pt}\ {\isasymforall}j{\isasymge}N{\isachardot}{\kern0pt}\ LINT\ x{\isacharbar}{\kern0pt}M{\isachardot}{\kern0pt}\ dist\ {\isacharparenleft}{\kern0pt}cond{\isacharunderscore}{\kern0pt}exp\ M\ F\ {\isacharparenleft}{\kern0pt}s\ i{\isacharparenright}{\kern0pt}\ x{\isacharparenright}{\kern0pt}\ {\isacharparenleft}{\kern0pt}cond{\isacharunderscore}{\kern0pt}exp\ M\ F\ {\isacharparenleft}{\kern0pt}s\ j{\isacharparenright}{\kern0pt}\ x{\isacharparenright}{\kern0pt}\ {\isacharless}{\kern0pt}\ e{\isachardoublequoteclose}\ \isakeyword{if}\ {\isachardoublequoteopen}e\ {\isachargreater}{\kern0pt}\ {\isadigit{0}}{\isachardoublequoteclose}\ \isakeyword{for}\ e\isanewline
\ \ \isacommand{proof}\isamarkupfalse%
\ {\isacharminus}{\kern0pt}\isanewline
\ \ \ \ \isacommand{obtain}\isamarkupfalse%
\ N\ \isakeyword{where}\ {\isacharasterisk}{\kern0pt}{\isacharcolon}{\kern0pt}\ {\isachardoublequoteopen}LINT\ x{\isacharbar}{\kern0pt}M{\isachardot}{\kern0pt}\ diameter\ {\isacharbraceleft}{\kern0pt}s\ i\ x\ {\isacharbar}{\kern0pt}\ i{\isachardot}{\kern0pt}\ n\ {\isasymle}\ i{\isacharbraceright}{\kern0pt}\ {\isacharless}{\kern0pt}\ e{\isachardoublequoteclose}\ \isakeyword{if}\ {\isachardoublequoteopen}n\ {\isasymge}\ N{\isachardoublequoteclose}\ \isakeyword{for}\ n\ \isacommand{using}\isamarkupfalse%
\ diameter{\isacharunderscore}{\kern0pt}tendsto{\isacharunderscore}{\kern0pt}zero\ \isacommand{by}\isamarkupfalse%
\ {\isacharparenleft}{\kern0pt}smt\ {\isacharparenleft}{\kern0pt}verit{\isacharcomma}{\kern0pt}\ del{\isacharunderscore}{\kern0pt}insts{\isacharparenright}{\kern0pt}\ {\isacartoucheopen}{\isadigit{0}}\ {\isacharless}{\kern0pt}\ e{\isacartoucheclose}\ eventually{\isacharunderscore}{\kern0pt}sequentially\ order{\isacharunderscore}{\kern0pt}tendsto{\isacharunderscore}{\kern0pt}iff{\isacharparenright}{\kern0pt}\isanewline
\ \ \ \ \isacommand{{\isacharbraceleft}{\kern0pt}}\isamarkupfalse%
\isanewline
\ \ \ \ \ \ \isacommand{fix}\isamarkupfalse%
\ i\ j\ x\ \isacommand{assume}\isamarkupfalse%
\ asm{\isacharcolon}{\kern0pt}\ {\isachardoublequoteopen}i\ {\isasymge}\ N{\isachardoublequoteclose}\ {\isachardoublequoteopen}j\ {\isasymge}\ N{\isachardoublequoteclose}\ {\isachardoublequoteopen}x\ {\isasymin}\ space\ M{\isachardoublequoteclose}\isanewline
\ \ \ \ \ \ \isacommand{have}\isamarkupfalse%
\ {\isachardoublequoteopen}case{\isacharunderscore}{\kern0pt}prod\ dist\ {\isacharbackquote}{\kern0pt}\ {\isacharparenleft}{\kern0pt}{\isacharbraceleft}{\kern0pt}s\ i\ x\ {\isacharbar}{\kern0pt}i{\isachardot}{\kern0pt}\ N\ {\isasymle}\ i{\isacharbraceright}{\kern0pt}\ {\isasymtimes}\ {\isacharbraceleft}{\kern0pt}s\ i\ x\ {\isacharbar}{\kern0pt}i{\isachardot}{\kern0pt}\ N\ {\isasymle}\ i{\isacharbraceright}{\kern0pt}{\isacharparenright}{\kern0pt}\ {\isacharequal}{\kern0pt}\ case{\isacharunderscore}{\kern0pt}prod\ {\isacharparenleft}{\kern0pt}{\isasymlambda}i\ j{\isachardot}{\kern0pt}\ dist\ {\isacharparenleft}{\kern0pt}s\ i\ x{\isacharparenright}{\kern0pt}\ {\isacharparenleft}{\kern0pt}s\ j\ x{\isacharparenright}{\kern0pt}{\isacharparenright}{\kern0pt}\ {\isacharbackquote}{\kern0pt}\ {\isacharparenleft}{\kern0pt}{\isacharbraceleft}{\kern0pt}N{\isachardot}{\kern0pt}{\isachardot}{\kern0pt}{\isacharbraceright}{\kern0pt}\ {\isasymtimes}\ {\isacharbraceleft}{\kern0pt}N{\isachardot}{\kern0pt}{\isachardot}{\kern0pt}{\isacharbraceright}{\kern0pt}{\isacharparenright}{\kern0pt}{\isachardoublequoteclose}\ \isacommand{by}\isamarkupfalse%
\ fast\isanewline
\ \ \ \ \ \ \isacommand{hence}\isamarkupfalse%
\ {\isachardoublequoteopen}diameter\ {\isacharbraceleft}{\kern0pt}s\ i\ x\ {\isacharbar}{\kern0pt}\ i{\isachardot}{\kern0pt}\ N\ {\isasymle}\ i{\isacharbraceright}{\kern0pt}\ {\isacharequal}{\kern0pt}\ {\isacharparenleft}{\kern0pt}SUP\ {\isacharparenleft}{\kern0pt}i{\isacharcomma}{\kern0pt}\ j{\isacharparenright}{\kern0pt}\ {\isasymin}\ {\isacharbraceleft}{\kern0pt}N{\isachardot}{\kern0pt}{\isachardot}{\kern0pt}{\isacharbraceright}{\kern0pt}\ {\isasymtimes}\ {\isacharbraceleft}{\kern0pt}N{\isachardot}{\kern0pt}{\isachardot}{\kern0pt}{\isacharbraceright}{\kern0pt}{\isachardot}{\kern0pt}\ dist\ {\isacharparenleft}{\kern0pt}s\ i\ x{\isacharparenright}{\kern0pt}\ {\isacharparenleft}{\kern0pt}s\ j\ x{\isacharparenright}{\kern0pt}{\isacharparenright}{\kern0pt}{\isachardoublequoteclose}\ \isacommand{unfolding}\isamarkupfalse%
\ diameter{\isacharunderscore}{\kern0pt}def\ \isacommand{by}\isamarkupfalse%
\ auto\isanewline
\ \ \ \ \ \ \isacommand{moreover}\isamarkupfalse%
\ \isacommand{have}\isamarkupfalse%
\ {\isachardoublequoteopen}{\isacharparenleft}{\kern0pt}SUP\ {\isacharparenleft}{\kern0pt}i{\isacharcomma}{\kern0pt}\ j{\isacharparenright}{\kern0pt}\ {\isasymin}\ {\isacharbraceleft}{\kern0pt}N{\isachardot}{\kern0pt}{\isachardot}{\kern0pt}{\isacharbraceright}{\kern0pt}\ {\isasymtimes}\ {\isacharbraceleft}{\kern0pt}N{\isachardot}{\kern0pt}{\isachardot}{\kern0pt}{\isacharbraceright}{\kern0pt}{\isachardot}{\kern0pt}\ dist\ {\isacharparenleft}{\kern0pt}s\ i\ x{\isacharparenright}{\kern0pt}\ {\isacharparenleft}{\kern0pt}s\ j\ x{\isacharparenright}{\kern0pt}{\isacharparenright}{\kern0pt}\ {\isasymge}\ dist\ {\isacharparenleft}{\kern0pt}s\ i\ x{\isacharparenright}{\kern0pt}\ {\isacharparenleft}{\kern0pt}s\ j\ x{\isacharparenright}{\kern0pt}{\isachardoublequoteclose}\ \isacommand{using}\isamarkupfalse%
\ asm\ bounded{\isacharunderscore}{\kern0pt}imp{\isacharunderscore}{\kern0pt}bdd{\isacharunderscore}{\kern0pt}above{\isacharbrackleft}{\kern0pt}OF\ bounded{\isacharunderscore}{\kern0pt}imp{\isacharunderscore}{\kern0pt}dist{\isacharunderscore}{\kern0pt}bounded{\isacharcomma}{\kern0pt}\ OF\ bounded{\isacharunderscore}{\kern0pt}range{\isacharunderscore}{\kern0pt}s{\isacharbrackright}{\kern0pt}\ \isacommand{by}\isamarkupfalse%
\ {\isacharparenleft}{\kern0pt}intro\ cSup{\isacharunderscore}{\kern0pt}upper{\isacharcomma}{\kern0pt}\ auto{\isacharparenright}{\kern0pt}\isanewline
\ \ \ \ \ \ \isacommand{ultimately}\isamarkupfalse%
\ \isacommand{have}\isamarkupfalse%
\ {\isachardoublequoteopen}diameter\ {\isacharbraceleft}{\kern0pt}s\ i\ x\ {\isacharbar}{\kern0pt}\ i{\isachardot}{\kern0pt}\ N\ {\isasymle}\ i{\isacharbraceright}{\kern0pt}\ {\isasymge}\ dist\ {\isacharparenleft}{\kern0pt}s\ i\ x{\isacharparenright}{\kern0pt}\ {\isacharparenleft}{\kern0pt}s\ j\ x{\isacharparenright}{\kern0pt}{\isachardoublequoteclose}\ \isacommand{by}\isamarkupfalse%
\ presburger\isanewline
\ \ \ \ \isacommand{{\isacharbraceright}{\kern0pt}}\isamarkupfalse%
\isanewline
\ \ \ \ \isacommand{hence}\isamarkupfalse%
\ {\isachardoublequoteopen}LINT\ x{\isacharbar}{\kern0pt}M{\isachardot}{\kern0pt}\ dist\ {\isacharparenleft}{\kern0pt}s\ i\ x{\isacharparenright}{\kern0pt}\ {\isacharparenleft}{\kern0pt}s\ j\ x{\isacharparenright}{\kern0pt}\ {\isacharless}{\kern0pt}\ e{\isachardoublequoteclose}\ \isakeyword{if}\ {\isachardoublequoteopen}i\ {\isasymge}\ N{\isachardoublequoteclose}\ {\isachardoublequoteopen}j\ {\isasymge}\ N{\isachardoublequoteclose}\ \isakeyword{for}\ i\ j\ \isacommand{using}\isamarkupfalse%
\ that\ {\isacharasterisk}{\kern0pt}\ \isacommand{by}\isamarkupfalse%
\ {\isacharparenleft}{\kern0pt}intro\ integral{\isacharunderscore}{\kern0pt}mono{\isacharbrackleft}{\kern0pt}OF\ dist{\isacharunderscore}{\kern0pt}integrable\ diameter{\isacharunderscore}{\kern0pt}integrable{\isacharcomma}{\kern0pt}\ THEN\ order{\isachardot}{\kern0pt}strict{\isacharunderscore}{\kern0pt}trans{\isadigit{1}}{\isacharbrackright}{\kern0pt}{\isacharcomma}{\kern0pt}\ blast{\isacharplus}{\kern0pt}{\isacharparenright}{\kern0pt}\isanewline
\ \ \ \ \isacommand{moreover}\isamarkupfalse%
\ \isacommand{have}\isamarkupfalse%
\ {\isachardoublequoteopen}LINT\ x{\isacharbar}{\kern0pt}M{\isachardot}{\kern0pt}\ dist\ {\isacharparenleft}{\kern0pt}cond{\isacharunderscore}{\kern0pt}exp\ M\ F\ {\isacharparenleft}{\kern0pt}s\ i{\isacharparenright}{\kern0pt}\ x{\isacharparenright}{\kern0pt}\ {\isacharparenleft}{\kern0pt}cond{\isacharunderscore}{\kern0pt}exp\ M\ F\ {\isacharparenleft}{\kern0pt}s\ j{\isacharparenright}{\kern0pt}\ x{\isacharparenright}{\kern0pt}\ {\isasymle}\ LINT\ x{\isacharbar}{\kern0pt}M{\isachardot}{\kern0pt}\ dist\ {\isacharparenleft}{\kern0pt}s\ i\ x{\isacharparenright}{\kern0pt}\ {\isacharparenleft}{\kern0pt}s\ j\ x{\isacharparenright}{\kern0pt}{\isachardoublequoteclose}\ \isakeyword{for}\ i\ j\isanewline
\ \ \ \ \isacommand{proof}\isamarkupfalse%
{\isacharminus}{\kern0pt}\isanewline
\ \ \ \ \ \ \isacommand{have}\isamarkupfalse%
\ {\isachardoublequoteopen}LINT\ x{\isacharbar}{\kern0pt}M{\isachardot}{\kern0pt}\ dist\ {\isacharparenleft}{\kern0pt}cond{\isacharunderscore}{\kern0pt}exp\ M\ F\ {\isacharparenleft}{\kern0pt}s\ i{\isacharparenright}{\kern0pt}\ x{\isacharparenright}{\kern0pt}\ {\isacharparenleft}{\kern0pt}cond{\isacharunderscore}{\kern0pt}exp\ M\ F\ {\isacharparenleft}{\kern0pt}s\ j{\isacharparenright}{\kern0pt}\ x{\isacharparenright}{\kern0pt}\ {\isacharequal}{\kern0pt}\ LINT\ x{\isacharbar}{\kern0pt}M{\isachardot}{\kern0pt}\ norm\ {\isacharparenleft}{\kern0pt}cond{\isacharunderscore}{\kern0pt}exp\ M\ F\ {\isacharparenleft}{\kern0pt}s\ i{\isacharparenright}{\kern0pt}\ x\ {\isacharplus}{\kern0pt}\ {\isacharminus}{\kern0pt}\ {\isadigit{1}}\ {\isacharasterisk}{\kern0pt}\isactrlsub R\ cond{\isacharunderscore}{\kern0pt}exp\ M\ F\ {\isacharparenleft}{\kern0pt}s\ j{\isacharparenright}{\kern0pt}\ x{\isacharparenright}{\kern0pt}{\isachardoublequoteclose}\ \isacommand{unfolding}\isamarkupfalse%
\ dist{\isacharunderscore}{\kern0pt}norm\ \isacommand{by}\isamarkupfalse%
\ simp\isanewline
\ \ \ \ \ \ \isacommand{also}\isamarkupfalse%
\ \isacommand{have}\isamarkupfalse%
\ {\isachardoublequoteopen}{\isachardot}{\kern0pt}{\isachardot}{\kern0pt}{\isachardot}{\kern0pt}\ {\isacharequal}{\kern0pt}\ LINT\ x{\isacharbar}{\kern0pt}M{\isachardot}{\kern0pt}\ norm\ {\isacharparenleft}{\kern0pt}cond{\isacharunderscore}{\kern0pt}exp\ M\ F\ {\isacharparenleft}{\kern0pt}{\isasymlambda}x{\isachardot}{\kern0pt}\ s\ i\ x\ {\isacharminus}{\kern0pt}\ s\ j\ x{\isacharparenright}{\kern0pt}\ x{\isacharparenright}{\kern0pt}{\isachardoublequoteclose}\ \isacommand{using}\isamarkupfalse%
\ has{\isacharunderscore}{\kern0pt}cond{\isacharunderscore}{\kern0pt}exp{\isacharunderscore}{\kern0pt}charact{\isacharparenleft}{\kern0pt}{\isadigit{2}}{\isacharparenright}{\kern0pt}{\isacharbrackleft}{\kern0pt}OF\ has{\isacharunderscore}{\kern0pt}cond{\isacharunderscore}{\kern0pt}exp{\isacharunderscore}{\kern0pt}add{\isacharbrackleft}{\kern0pt}OF\ {\isacharunderscore}{\kern0pt}\ has{\isacharunderscore}{\kern0pt}cond{\isacharunderscore}{\kern0pt}exp{\isacharunderscore}{\kern0pt}scaleR{\isacharunderscore}{\kern0pt}right{\isacharcomma}{\kern0pt}\ OF\ has{\isacharunderscore}{\kern0pt}cond{\isacharunderscore}{\kern0pt}exp{\isacharunderscore}{\kern0pt}charact{\isacharparenleft}{\kern0pt}{\isadigit{1}}{\isacharcomma}{\kern0pt}{\isadigit{1}}{\isacharparenright}{\kern0pt}{\isacharcomma}{\kern0pt}\ OF\ has{\isacharunderscore}{\kern0pt}cond{\isacharunderscore}{\kern0pt}exp{\isacharunderscore}{\kern0pt}simple{\isacharparenleft}{\kern0pt}{\isadigit{1}}{\isacharcomma}{\kern0pt}{\isadigit{1}}{\isacharparenright}{\kern0pt}{\isacharbrackleft}{\kern0pt}OF\ assms{\isacharparenleft}{\kern0pt}{\isadigit{2}}{\isacharcomma}{\kern0pt}{\isadigit{3}}{\isacharparenright}{\kern0pt}{\isacharbrackright}{\kern0pt}{\isacharbrackright}{\kern0pt}{\isacharcomma}{\kern0pt}\ THEN\ AE{\isacharunderscore}{\kern0pt}symmetric{\isacharcomma}{\kern0pt}\ of\ i\ {\isachardoublequoteopen}{\isacharminus}{\kern0pt}{\isadigit{1}}{\isachardoublequoteclose}\ j{\isacharbrackright}{\kern0pt}\ \isacommand{by}\isamarkupfalse%
\ {\isacharparenleft}{\kern0pt}intro\ integral{\isacharunderscore}{\kern0pt}cong{\isacharunderscore}{\kern0pt}AE{\isacharparenright}{\kern0pt}\ force{\isacharplus}{\kern0pt}\ \ \ \ \ \ \isanewline
\ \ \ \ \ \ \isacommand{also}\isamarkupfalse%
\ \isacommand{have}\isamarkupfalse%
\ {\isachardoublequoteopen}{\isachardot}{\kern0pt}{\isachardot}{\kern0pt}{\isachardot}{\kern0pt}\ {\isasymle}\ LINT\ x{\isacharbar}{\kern0pt}M{\isachardot}{\kern0pt}\ cond{\isacharunderscore}{\kern0pt}exp\ M\ F\ {\isacharparenleft}{\kern0pt}{\isasymlambda}x{\isachardot}{\kern0pt}\ norm\ {\isacharparenleft}{\kern0pt}s\ i\ x\ {\isacharminus}{\kern0pt}\ s\ j\ x{\isacharparenright}{\kern0pt}{\isacharparenright}{\kern0pt}\ x{\isachardoublequoteclose}\ \isacommand{using}\isamarkupfalse%
\ cond{\isacharunderscore}{\kern0pt}exp{\isacharunderscore}{\kern0pt}contraction{\isacharunderscore}{\kern0pt}simple{\isacharbrackleft}{\kern0pt}OF\ {\isacharunderscore}{\kern0pt}\ fin{\isacharunderscore}{\kern0pt}sup{\isacharcomma}{\kern0pt}\ of\ i\ j{\isacharbrackright}{\kern0pt}\ integrable{\isacharunderscore}{\kern0pt}cond{\isacharunderscore}{\kern0pt}exp\ assms{\isacharparenleft}{\kern0pt}{\isadigit{2}}{\isacharparenright}{\kern0pt}\ \isacommand{by}\isamarkupfalse%
\ {\isacharparenleft}{\kern0pt}intro\ integral{\isacharunderscore}{\kern0pt}mono{\isacharunderscore}{\kern0pt}AE{\isacharcomma}{\kern0pt}\ fast{\isacharplus}{\kern0pt}{\isacharparenright}{\kern0pt}\isanewline
\ \ \ \ \ \ \isacommand{also}\isamarkupfalse%
\ \isacommand{have}\isamarkupfalse%
\ {\isachardoublequoteopen}{\isachardot}{\kern0pt}{\isachardot}{\kern0pt}{\isachardot}{\kern0pt}\ {\isacharequal}{\kern0pt}\ LINT\ x{\isacharbar}{\kern0pt}M{\isachardot}{\kern0pt}\ norm\ {\isacharparenleft}{\kern0pt}s\ i\ x\ {\isacharminus}{\kern0pt}\ s\ j\ x{\isacharparenright}{\kern0pt}{\isachardoublequoteclose}\ \isacommand{unfolding}\isamarkupfalse%
\ set{\isacharunderscore}{\kern0pt}integral{\isacharunderscore}{\kern0pt}space{\isacharparenleft}{\kern0pt}{\isadigit{1}}{\isacharparenright}{\kern0pt}{\isacharbrackleft}{\kern0pt}OF\ integrable{\isacharunderscore}{\kern0pt}cond{\isacharunderscore}{\kern0pt}exp{\isacharcomma}{\kern0pt}\ symmetric{\isacharbrackright}{\kern0pt}\ set{\isacharunderscore}{\kern0pt}integral{\isacharunderscore}{\kern0pt}space{\isacharbrackleft}{\kern0pt}OF\ dist{\isacharunderscore}{\kern0pt}norm{\isacharunderscore}{\kern0pt}integrable{\isacharcomma}{\kern0pt}\ symmetric{\isacharbrackright}{\kern0pt}\ \isacommand{by}\isamarkupfalse%
\ {\isacharparenleft}{\kern0pt}intro\ has{\isacharunderscore}{\kern0pt}cond{\isacharunderscore}{\kern0pt}expD{\isacharparenleft}{\kern0pt}{\isadigit{1}}{\isacharparenright}{\kern0pt}{\isacharbrackleft}{\kern0pt}OF\ has{\isacharunderscore}{\kern0pt}cond{\isacharunderscore}{\kern0pt}exp{\isacharunderscore}{\kern0pt}simple{\isacharbrackleft}{\kern0pt}OF\ {\isacharunderscore}{\kern0pt}\ fin{\isacharunderscore}{\kern0pt}sup{\isacharunderscore}{\kern0pt}norm{\isacharbrackright}{\kern0pt}{\isacharcomma}{\kern0pt}\ symmetric{\isacharbrackright}{\kern0pt}{\isacharparenright}{\kern0pt}\ {\isacharparenleft}{\kern0pt}metis\ assms{\isacharparenleft}{\kern0pt}{\isadigit{2}}{\isacharparenright}{\kern0pt}\ simple{\isacharunderscore}{\kern0pt}function{\isacharunderscore}{\kern0pt}compose{\isadigit{1}}\ simple{\isacharunderscore}{\kern0pt}function{\isacharunderscore}{\kern0pt}diff{\isacharcomma}{\kern0pt}\ metis\ sets{\isachardot}{\kern0pt}top\ subalg\ subalgebra{\isacharunderscore}{\kern0pt}def{\isacharparenright}{\kern0pt}\isanewline
\ \ \ \ \ \ \isacommand{finally}\isamarkupfalse%
\ \isacommand{show}\isamarkupfalse%
\ {\isacharquery}{\kern0pt}thesis\ \isacommand{unfolding}\isamarkupfalse%
\ dist{\isacharunderscore}{\kern0pt}norm\ \isacommand{{\isachardot}{\kern0pt}}\isamarkupfalse%
\ \ \isanewline
\ \ \ \ \isacommand{qed}\isamarkupfalse%
\isanewline
\ \ \ \ \isacommand{ultimately}\isamarkupfalse%
\ \isacommand{show}\isamarkupfalse%
\ {\isacharquery}{\kern0pt}thesis\ \isacommand{using}\isamarkupfalse%
\ order{\isachardot}{\kern0pt}strict{\isacharunderscore}{\kern0pt}trans{\isadigit{1}}\ \isacommand{by}\isamarkupfalse%
\ meson\isanewline
\ \ \isacommand{qed}\isamarkupfalse%
\isanewline
\ \ \isacommand{then}\isamarkupfalse%
\ \isacommand{obtain}\isamarkupfalse%
\ r\ \isakeyword{where}\ strict{\isacharunderscore}{\kern0pt}mono{\isacharunderscore}{\kern0pt}r{\isacharcolon}{\kern0pt}\ {\isachardoublequoteopen}strict{\isacharunderscore}{\kern0pt}mono\ r{\isachardoublequoteclose}\ \isakeyword{and}\ AE{\isacharunderscore}{\kern0pt}Cauchy{\isacharcolon}{\kern0pt}\ {\isachardoublequoteopen}AE\ x\ in\ M{\isachardot}{\kern0pt}\ Cauchy\ {\isacharparenleft}{\kern0pt}{\isasymlambda}i{\isachardot}{\kern0pt}\ cond{\isacharunderscore}{\kern0pt}exp\ M\ F\ {\isacharparenleft}{\kern0pt}s\ {\isacharparenleft}{\kern0pt}r\ i{\isacharparenright}{\kern0pt}{\isacharparenright}{\kern0pt}\ x{\isacharparenright}{\kern0pt}{\isachardoublequoteclose}\ \isacommand{by}\isamarkupfalse%
\ {\isacharparenleft}{\kern0pt}rule\ cauchy{\isacharunderscore}{\kern0pt}L{\isadigit{1}}{\isacharunderscore}{\kern0pt}AE{\isacharunderscore}{\kern0pt}cauchy{\isacharunderscore}{\kern0pt}subseq{\isacharbrackleft}{\kern0pt}OF\ integrable{\isacharunderscore}{\kern0pt}cond{\isacharunderscore}{\kern0pt}exp{\isacharbrackright}{\kern0pt}{\isacharcomma}{\kern0pt}\ auto{\isacharparenright}{\kern0pt}\isanewline
\ \ \isacommand{hence}\isamarkupfalse%
\ ae{\isacharunderscore}{\kern0pt}lim{\isacharunderscore}{\kern0pt}cond{\isacharunderscore}{\kern0pt}exp{\isacharcolon}{\kern0pt}\ {\isachardoublequoteopen}AE\ x\ in\ M{\isachardot}{\kern0pt}\ {\isacharparenleft}{\kern0pt}{\isasymlambda}n{\isachardot}{\kern0pt}\ cond{\isacharunderscore}{\kern0pt}exp\ M\ F\ {\isacharparenleft}{\kern0pt}s\ {\isacharparenleft}{\kern0pt}r\ n{\isacharparenright}{\kern0pt}{\isacharparenright}{\kern0pt}\ x{\isacharparenright}{\kern0pt}\ {\isasymlonglonglongrightarrow}\ lim\ {\isacharparenleft}{\kern0pt}{\isasymlambda}n{\isachardot}{\kern0pt}\ cond{\isacharunderscore}{\kern0pt}exp\ M\ F\ {\isacharparenleft}{\kern0pt}s\ {\isacharparenleft}{\kern0pt}r\ n{\isacharparenright}{\kern0pt}{\isacharparenright}{\kern0pt}\ x{\isacharparenright}{\kern0pt}{\isachardoublequoteclose}\ \isacommand{using}\isamarkupfalse%
\ Cauchy{\isacharunderscore}{\kern0pt}convergent{\isacharunderscore}{\kern0pt}iff\ convergent{\isacharunderscore}{\kern0pt}LIMSEQ{\isacharunderscore}{\kern0pt}iff\ \isacommand{by}\isamarkupfalse%
\ fastforce\isanewline
\isanewline
\ \ \isacommand{have}\isamarkupfalse%
\ cond{\isacharunderscore}{\kern0pt}exp{\isacharunderscore}{\kern0pt}bounded{\isacharcolon}{\kern0pt}\ {\isachardoublequoteopen}AE\ x\ in\ M{\isachardot}{\kern0pt}\ norm\ {\isacharparenleft}{\kern0pt}cond{\isacharunderscore}{\kern0pt}exp\ M\ F\ {\isacharparenleft}{\kern0pt}s\ {\isacharparenleft}{\kern0pt}r\ n{\isacharparenright}{\kern0pt}{\isacharparenright}{\kern0pt}\ x{\isacharparenright}{\kern0pt}\ {\isasymle}\ cond{\isacharunderscore}{\kern0pt}exp\ M\ F\ {\isacharparenleft}{\kern0pt}{\isasymlambda}x{\isachardot}{\kern0pt}\ {\isadigit{2}}\ {\isacharasterisk}{\kern0pt}\ norm\ {\isacharparenleft}{\kern0pt}f\ x{\isacharparenright}{\kern0pt}{\isacharparenright}{\kern0pt}\ x{\isachardoublequoteclose}\ \isakeyword{for}\ n\isanewline
\ \ \isacommand{proof}\isamarkupfalse%
\ {\isacharminus}{\kern0pt}\isanewline
\ \ \ \ \isacommand{have}\isamarkupfalse%
\ {\isachardoublequoteopen}AE\ x\ in\ M{\isachardot}{\kern0pt}\ norm\ {\isacharparenleft}{\kern0pt}cond{\isacharunderscore}{\kern0pt}exp\ M\ F\ {\isacharparenleft}{\kern0pt}s\ {\isacharparenleft}{\kern0pt}r\ n{\isacharparenright}{\kern0pt}{\isacharparenright}{\kern0pt}\ x{\isacharparenright}{\kern0pt}\ {\isasymle}\ cond{\isacharunderscore}{\kern0pt}exp\ M\ F\ {\isacharparenleft}{\kern0pt}{\isasymlambda}x{\isachardot}{\kern0pt}\ norm\ {\isacharparenleft}{\kern0pt}s\ {\isacharparenleft}{\kern0pt}r\ n{\isacharparenright}{\kern0pt}\ x{\isacharparenright}{\kern0pt}{\isacharparenright}{\kern0pt}\ x{\isachardoublequoteclose}\ \isacommand{by}\isamarkupfalse%
\ {\isacharparenleft}{\kern0pt}rule\ cond{\isacharunderscore}{\kern0pt}exp{\isacharunderscore}{\kern0pt}contraction{\isacharunderscore}{\kern0pt}simple{\isacharbrackleft}{\kern0pt}OF\ assms{\isacharparenleft}{\kern0pt}{\isadigit{2}}{\isacharcomma}{\kern0pt}{\isadigit{3}}{\isacharparenright}{\kern0pt}{\isacharbrackright}{\kern0pt}{\isacharparenright}{\kern0pt}\isanewline
\ \ \ \ \isacommand{moreover}\isamarkupfalse%
\ \isacommand{have}\isamarkupfalse%
\ {\isachardoublequoteopen}AE\ x\ in\ M{\isachardot}{\kern0pt}\ real{\isacharunderscore}{\kern0pt}cond{\isacharunderscore}{\kern0pt}exp\ M\ F\ {\isacharparenleft}{\kern0pt}{\isasymlambda}x{\isachardot}{\kern0pt}\ norm\ {\isacharparenleft}{\kern0pt}s\ {\isacharparenleft}{\kern0pt}r\ n{\isacharparenright}{\kern0pt}\ x{\isacharparenright}{\kern0pt}{\isacharparenright}{\kern0pt}\ x\ {\isasymle}\ real{\isacharunderscore}{\kern0pt}cond{\isacharunderscore}{\kern0pt}exp\ M\ F\ {\isacharparenleft}{\kern0pt}{\isasymlambda}x{\isachardot}{\kern0pt}\ {\isadigit{2}}\ {\isacharasterisk}{\kern0pt}\ norm\ {\isacharparenleft}{\kern0pt}f\ x{\isacharparenright}{\kern0pt}{\isacharparenright}{\kern0pt}\ x{\isachardoublequoteclose}\ \isacommand{using}\isamarkupfalse%
\ integrable{\isacharunderscore}{\kern0pt}s\ integrable{\isacharunderscore}{\kern0pt}{\isadigit{2}}f\ assms{\isacharparenleft}{\kern0pt}{\isadigit{5}}{\isacharparenright}{\kern0pt}\ \isacommand{by}\isamarkupfalse%
\ {\isacharparenleft}{\kern0pt}intro\ real{\isacharunderscore}{\kern0pt}cond{\isacharunderscore}{\kern0pt}exp{\isacharunderscore}{\kern0pt}mono{\isacharcomma}{\kern0pt}\ auto{\isacharparenright}{\kern0pt}\ \isanewline
\ \ \ \ \isacommand{ultimately}\isamarkupfalse%
\ \isacommand{show}\isamarkupfalse%
\ {\isacharquery}{\kern0pt}thesis\ \isacommand{using}\isamarkupfalse%
\ cond{\isacharunderscore}{\kern0pt}exp{\isacharunderscore}{\kern0pt}real{\isacharbrackleft}{\kern0pt}OF\ integrable{\isacharunderscore}{\kern0pt}norm{\isacharcomma}{\kern0pt}\ OF\ integrable{\isacharunderscore}{\kern0pt}s{\isacharcomma}{\kern0pt}\ of\ {\isachardoublequoteopen}r\ n{\isachardoublequoteclose}{\isacharbrackright}{\kern0pt}\ cond{\isacharunderscore}{\kern0pt}exp{\isacharunderscore}{\kern0pt}real{\isacharbrackleft}{\kern0pt}OF\ integrable{\isacharunderscore}{\kern0pt}{\isadigit{2}}f{\isacharbrackright}{\kern0pt}\ \isacommand{by}\isamarkupfalse%
\ force\isanewline
\ \ \isacommand{qed}\isamarkupfalse%
\isanewline
\ \ \isacommand{have}\isamarkupfalse%
\ lim{\isacharunderscore}{\kern0pt}integrable{\isacharcolon}{\kern0pt}\ {\isachardoublequoteopen}integrable\ M\ {\isacharparenleft}{\kern0pt}{\isasymlambda}x{\isachardot}{\kern0pt}\ lim\ {\isacharparenleft}{\kern0pt}{\isasymlambda}i{\isachardot}{\kern0pt}\ cond{\isacharunderscore}{\kern0pt}exp\ M\ F\ {\isacharparenleft}{\kern0pt}s\ {\isacharparenleft}{\kern0pt}r\ i{\isacharparenright}{\kern0pt}{\isacharparenright}{\kern0pt}\ x{\isacharparenright}{\kern0pt}{\isacharparenright}{\kern0pt}{\isachardoublequoteclose}\ \isacommand{by}\isamarkupfalse%
\ {\isacharparenleft}{\kern0pt}intro\ integrable{\isacharunderscore}{\kern0pt}dominated{\isacharunderscore}{\kern0pt}convergence{\isacharbrackleft}{\kern0pt}OF\ {\isacharunderscore}{\kern0pt}\ borel{\isacharunderscore}{\kern0pt}measurable{\isacharunderscore}{\kern0pt}cond{\isacharunderscore}{\kern0pt}exp{\isacharprime}{\kern0pt}\ integrable{\isacharunderscore}{\kern0pt}cond{\isacharunderscore}{\kern0pt}exp\ ae{\isacharunderscore}{\kern0pt}lim{\isacharunderscore}{\kern0pt}cond{\isacharunderscore}{\kern0pt}exp\ cond{\isacharunderscore}{\kern0pt}exp{\isacharunderscore}{\kern0pt}bounded{\isacharbrackright}{\kern0pt}{\isacharcomma}{\kern0pt}\ simp{\isacharparenright}{\kern0pt}\isanewline
\isanewline
\ \ \isacommand{{\isacharbraceleft}{\kern0pt}}\isamarkupfalse%
\isanewline
\ \ \ \ \isacommand{fix}\isamarkupfalse%
\ A\ \isacommand{assume}\isamarkupfalse%
\ A{\isacharunderscore}{\kern0pt}in{\isacharunderscore}{\kern0pt}sets{\isacharunderscore}{\kern0pt}F{\isacharcolon}{\kern0pt}\ {\isachardoublequoteopen}A\ {\isasymin}\ sets\ F{\isachardoublequoteclose}\isanewline
\ \ \ \ \isacommand{have}\isamarkupfalse%
\ {\isachardoublequoteopen}AE\ x\ in\ M{\isachardot}{\kern0pt}\ norm\ {\isacharparenleft}{\kern0pt}indicator\ A\ x\ {\isacharasterisk}{\kern0pt}\isactrlsub R\ cond{\isacharunderscore}{\kern0pt}exp\ M\ F\ {\isacharparenleft}{\kern0pt}s\ {\isacharparenleft}{\kern0pt}r\ n{\isacharparenright}{\kern0pt}{\isacharparenright}{\kern0pt}\ x{\isacharparenright}{\kern0pt}\ {\isasymle}\ cond{\isacharunderscore}{\kern0pt}exp\ M\ F\ {\isacharparenleft}{\kern0pt}{\isasymlambda}x{\isachardot}{\kern0pt}\ {\isadigit{2}}\ {\isacharasterisk}{\kern0pt}\ norm\ {\isacharparenleft}{\kern0pt}f\ x{\isacharparenright}{\kern0pt}{\isacharparenright}{\kern0pt}\ x{\isachardoublequoteclose}\ \isakeyword{for}\ n\isanewline
\ \ \ \ \isacommand{proof}\isamarkupfalse%
\ {\isacharminus}{\kern0pt}\isanewline
\ \ \ \ \ \ \isacommand{have}\isamarkupfalse%
\ {\isachardoublequoteopen}AE\ x\ in\ M{\isachardot}{\kern0pt}\ norm\ {\isacharparenleft}{\kern0pt}indicator\ A\ x\ {\isacharasterisk}{\kern0pt}\isactrlsub R\ cond{\isacharunderscore}{\kern0pt}exp\ M\ F\ {\isacharparenleft}{\kern0pt}s\ {\isacharparenleft}{\kern0pt}r\ n{\isacharparenright}{\kern0pt}{\isacharparenright}{\kern0pt}\ x{\isacharparenright}{\kern0pt}\ {\isasymle}\ norm\ {\isacharparenleft}{\kern0pt}cond{\isacharunderscore}{\kern0pt}exp\ M\ F\ {\isacharparenleft}{\kern0pt}s\ {\isacharparenleft}{\kern0pt}r\ n{\isacharparenright}{\kern0pt}{\isacharparenright}{\kern0pt}\ x{\isacharparenright}{\kern0pt}{\isachardoublequoteclose}\ \isacommand{unfolding}\isamarkupfalse%
\ indicator{\isacharunderscore}{\kern0pt}def\ \isacommand{by}\isamarkupfalse%
\ simp\isanewline
\ \ \ \ \ \ \isacommand{thus}\isamarkupfalse%
\ {\isacharquery}{\kern0pt}thesis\ \isacommand{using}\isamarkupfalse%
\ cond{\isacharunderscore}{\kern0pt}exp{\isacharunderscore}{\kern0pt}bounded{\isacharbrackleft}{\kern0pt}of\ n{\isacharbrackright}{\kern0pt}\ \isacommand{by}\isamarkupfalse%
\ force\isanewline
\ \ \ \ \isacommand{qed}\isamarkupfalse%
\isanewline
\ \ \ \ \isacommand{hence}\isamarkupfalse%
\ lim{\isacharunderscore}{\kern0pt}cond{\isacharunderscore}{\kern0pt}exp{\isacharunderscore}{\kern0pt}int{\isacharcolon}{\kern0pt}\ {\isachardoublequoteopen}{\isacharparenleft}{\kern0pt}{\isasymlambda}n{\isachardot}{\kern0pt}\ LINT\ x{\isacharcolon}{\kern0pt}A{\isacharbar}{\kern0pt}M{\isachardot}{\kern0pt}\ cond{\isacharunderscore}{\kern0pt}exp\ M\ F\ {\isacharparenleft}{\kern0pt}s\ {\isacharparenleft}{\kern0pt}r\ n{\isacharparenright}{\kern0pt}{\isacharparenright}{\kern0pt}\ x{\isacharparenright}{\kern0pt}\ {\isasymlonglonglongrightarrow}\ LINT\ x{\isacharcolon}{\kern0pt}A{\isacharbar}{\kern0pt}M{\isachardot}{\kern0pt}\ lim\ {\isacharparenleft}{\kern0pt}{\isasymlambda}n{\isachardot}{\kern0pt}\ cond{\isacharunderscore}{\kern0pt}exp\ M\ F\ {\isacharparenleft}{\kern0pt}s\ {\isacharparenleft}{\kern0pt}r\ n{\isacharparenright}{\kern0pt}{\isacharparenright}{\kern0pt}\ x{\isacharparenright}{\kern0pt}{\isachardoublequoteclose}\ \isanewline
\ \ \ \ \ \ \isacommand{using}\isamarkupfalse%
\ ae{\isacharunderscore}{\kern0pt}lim{\isacharunderscore}{\kern0pt}cond{\isacharunderscore}{\kern0pt}exp\ measurable{\isacharunderscore}{\kern0pt}from{\isacharunderscore}{\kern0pt}subalg{\isacharbrackleft}{\kern0pt}OF\ subalg\ borel{\isacharunderscore}{\kern0pt}measurable{\isacharunderscore}{\kern0pt}indicator{\isacharcomma}{\kern0pt}\ OF\ A{\isacharunderscore}{\kern0pt}in{\isacharunderscore}{\kern0pt}sets{\isacharunderscore}{\kern0pt}F{\isacharbrackright}{\kern0pt}\ cond{\isacharunderscore}{\kern0pt}exp{\isacharunderscore}{\kern0pt}bounded\isanewline
\ \ \ \ \ \ \isacommand{unfolding}\isamarkupfalse%
\ set{\isacharunderscore}{\kern0pt}lebesgue{\isacharunderscore}{\kern0pt}integral{\isacharunderscore}{\kern0pt}def\isanewline
\ \ \ \ \ \ \isacommand{by}\isamarkupfalse%
\ {\isacharparenleft}{\kern0pt}intro\ integral{\isacharunderscore}{\kern0pt}dominated{\isacharunderscore}{\kern0pt}convergence{\isacharbrackleft}{\kern0pt}OF\ borel{\isacharunderscore}{\kern0pt}measurable{\isacharunderscore}{\kern0pt}scaleR\ borel{\isacharunderscore}{\kern0pt}measurable{\isacharunderscore}{\kern0pt}scaleR\ integrable{\isacharunderscore}{\kern0pt}cond{\isacharunderscore}{\kern0pt}exp{\isacharbrackright}{\kern0pt}{\isacharparenright}{\kern0pt}\ {\isacharparenleft}{\kern0pt}fastforce\ simp\ add{\isacharcolon}{\kern0pt}\ tendsto{\isacharunderscore}{\kern0pt}scaleR{\isacharparenright}{\kern0pt}{\isacharplus}{\kern0pt}\isanewline
\isanewline
\ \ \ \ \isacommand{have}\isamarkupfalse%
\ {\isachardoublequoteopen}AE\ x\ in\ M{\isachardot}{\kern0pt}\ norm\ {\isacharparenleft}{\kern0pt}indicator\ A\ x\ {\isacharasterisk}{\kern0pt}\isactrlsub R\ s\ {\isacharparenleft}{\kern0pt}r\ n{\isacharparenright}{\kern0pt}\ x{\isacharparenright}{\kern0pt}\ {\isasymle}\ {\isadigit{2}}\ {\isacharasterisk}{\kern0pt}\ norm\ {\isacharparenleft}{\kern0pt}f\ x{\isacharparenright}{\kern0pt}{\isachardoublequoteclose}\ \isakeyword{for}\ n\isanewline
\ \ \ \ \isacommand{proof}\isamarkupfalse%
\ {\isacharminus}{\kern0pt}\isanewline
\ \ \ \ \ \ \isacommand{have}\isamarkupfalse%
\ {\isachardoublequoteopen}AE\ x\ in\ M{\isachardot}{\kern0pt}\ norm\ {\isacharparenleft}{\kern0pt}indicator\ A\ x\ {\isacharasterisk}{\kern0pt}\isactrlsub R\ s\ {\isacharparenleft}{\kern0pt}r\ n{\isacharparenright}{\kern0pt}\ x{\isacharparenright}{\kern0pt}\ {\isasymle}\ norm\ {\isacharparenleft}{\kern0pt}s\ {\isacharparenleft}{\kern0pt}r\ n{\isacharparenright}{\kern0pt}\ x{\isacharparenright}{\kern0pt}{\isachardoublequoteclose}\ \isacommand{unfolding}\isamarkupfalse%
\ indicator{\isacharunderscore}{\kern0pt}def\ \isacommand{by}\isamarkupfalse%
\ simp\isanewline
\ \ \ \ \ \ \isacommand{thus}\isamarkupfalse%
\ {\isacharquery}{\kern0pt}thesis\ \isacommand{using}\isamarkupfalse%
\ assms{\isacharparenleft}{\kern0pt}{\isadigit{5}}{\isacharparenright}{\kern0pt}{\isacharbrackleft}{\kern0pt}of\ {\isacharunderscore}{\kern0pt}\ {\isachardoublequoteopen}r\ n{\isachardoublequoteclose}{\isacharbrackright}{\kern0pt}\ \isacommand{by}\isamarkupfalse%
\ fastforce\isanewline
\ \ \ \ \isacommand{qed}\isamarkupfalse%
\isanewline
\ \ \ \ \isacommand{hence}\isamarkupfalse%
\ lim{\isacharunderscore}{\kern0pt}s{\isacharunderscore}{\kern0pt}int{\isacharcolon}{\kern0pt}\ {\isachardoublequoteopen}{\isacharparenleft}{\kern0pt}{\isasymlambda}n{\isachardot}{\kern0pt}\ LINT\ x{\isacharcolon}{\kern0pt}A{\isacharbar}{\kern0pt}M{\isachardot}{\kern0pt}\ s\ {\isacharparenleft}{\kern0pt}r\ n{\isacharparenright}{\kern0pt}\ x{\isacharparenright}{\kern0pt}\ {\isasymlonglonglongrightarrow}\ LINT\ x{\isacharcolon}{\kern0pt}A{\isacharbar}{\kern0pt}M{\isachardot}{\kern0pt}\ f\ x{\isachardoublequoteclose}\isanewline
\ \ \ \ \ \ \isacommand{using}\isamarkupfalse%
\ measurable{\isacharunderscore}{\kern0pt}from{\isacharunderscore}{\kern0pt}subalg{\isacharbrackleft}{\kern0pt}OF\ subalg\ borel{\isacharunderscore}{\kern0pt}measurable{\isacharunderscore}{\kern0pt}indicator{\isacharcomma}{\kern0pt}\ OF\ A{\isacharunderscore}{\kern0pt}in{\isacharunderscore}{\kern0pt}sets{\isacharunderscore}{\kern0pt}F{\isacharbrackright}{\kern0pt}\ LIMSEQ{\isacharunderscore}{\kern0pt}subseq{\isacharunderscore}{\kern0pt}LIMSEQ{\isacharbrackleft}{\kern0pt}OF\ assms{\isacharparenleft}{\kern0pt}{\isadigit{4}}{\isacharparenright}{\kern0pt}\ strict{\isacharunderscore}{\kern0pt}mono{\isacharunderscore}{\kern0pt}r{\isacharbrackright}{\kern0pt}\ assms{\isacharparenleft}{\kern0pt}{\isadigit{5}}{\isacharparenright}{\kern0pt}\isanewline
\ \ \ \ \ \ \isacommand{unfolding}\isamarkupfalse%
\ set{\isacharunderscore}{\kern0pt}lebesgue{\isacharunderscore}{\kern0pt}integral{\isacharunderscore}{\kern0pt}def\ comp{\isacharunderscore}{\kern0pt}def\isanewline
\ \ \ \ \ \ \isacommand{by}\isamarkupfalse%
\ {\isacharparenleft}{\kern0pt}intro\ integral{\isacharunderscore}{\kern0pt}dominated{\isacharunderscore}{\kern0pt}convergence{\isacharbrackleft}{\kern0pt}OF\ borel{\isacharunderscore}{\kern0pt}measurable{\isacharunderscore}{\kern0pt}scaleR\ borel{\isacharunderscore}{\kern0pt}measurable{\isacharunderscore}{\kern0pt}scaleR\ integrable{\isacharunderscore}{\kern0pt}{\isadigit{2}}f{\isacharbrackright}{\kern0pt}{\isacharparenright}{\kern0pt}\ {\isacharparenleft}{\kern0pt}fastforce\ simp\ add{\isacharcolon}{\kern0pt}\ tendsto{\isacharunderscore}{\kern0pt}scaleR{\isacharparenright}{\kern0pt}{\isacharplus}{\kern0pt}\isanewline
\isanewline
\ \ \ \ \isacommand{have}\isamarkupfalse%
\ {\isachardoublequoteopen}LINT\ x{\isacharcolon}{\kern0pt}A{\isacharbar}{\kern0pt}M{\isachardot}{\kern0pt}\ lim\ {\isacharparenleft}{\kern0pt}{\isasymlambda}n{\isachardot}{\kern0pt}\ cond{\isacharunderscore}{\kern0pt}exp\ M\ F\ {\isacharparenleft}{\kern0pt}s\ {\isacharparenleft}{\kern0pt}r\ n{\isacharparenright}{\kern0pt}{\isacharparenright}{\kern0pt}\ x{\isacharparenright}{\kern0pt}\ {\isacharequal}{\kern0pt}\ lim\ {\isacharparenleft}{\kern0pt}{\isasymlambda}n{\isachardot}{\kern0pt}\ LINT\ x{\isacharcolon}{\kern0pt}A{\isacharbar}{\kern0pt}M{\isachardot}{\kern0pt}\ cond{\isacharunderscore}{\kern0pt}exp\ M\ F\ {\isacharparenleft}{\kern0pt}s\ {\isacharparenleft}{\kern0pt}r\ n{\isacharparenright}{\kern0pt}{\isacharparenright}{\kern0pt}\ x{\isacharparenright}{\kern0pt}{\isachardoublequoteclose}\ \isacommand{using}\isamarkupfalse%
\ limI{\isacharbrackleft}{\kern0pt}OF\ lim{\isacharunderscore}{\kern0pt}cond{\isacharunderscore}{\kern0pt}exp{\isacharunderscore}{\kern0pt}int{\isacharbrackright}{\kern0pt}\ \isacommand{by}\isamarkupfalse%
\ argo\isanewline
\ \ \ \ \isacommand{also}\isamarkupfalse%
\ \isacommand{have}\isamarkupfalse%
\ {\isachardoublequoteopen}{\isachardot}{\kern0pt}{\isachardot}{\kern0pt}{\isachardot}{\kern0pt}\ {\isacharequal}{\kern0pt}\ lim\ {\isacharparenleft}{\kern0pt}{\isasymlambda}n{\isachardot}{\kern0pt}\ LINT\ x{\isacharcolon}{\kern0pt}A{\isacharbar}{\kern0pt}M{\isachardot}{\kern0pt}\ s\ {\isacharparenleft}{\kern0pt}r\ n{\isacharparenright}{\kern0pt}\ x{\isacharparenright}{\kern0pt}{\isachardoublequoteclose}\ \isacommand{using}\isamarkupfalse%
\ has{\isacharunderscore}{\kern0pt}cond{\isacharunderscore}{\kern0pt}expD{\isacharparenleft}{\kern0pt}{\isadigit{1}}{\isacharparenright}{\kern0pt}{\isacharbrackleft}{\kern0pt}OF\ has{\isacharunderscore}{\kern0pt}cond{\isacharunderscore}{\kern0pt}exp{\isacharunderscore}{\kern0pt}simple{\isacharbrackleft}{\kern0pt}OF\ assms{\isacharparenleft}{\kern0pt}{\isadigit{2}}{\isacharcomma}{\kern0pt}{\isadigit{3}}{\isacharparenright}{\kern0pt}{\isacharbrackright}{\kern0pt}\ A{\isacharunderscore}{\kern0pt}in{\isacharunderscore}{\kern0pt}sets{\isacharunderscore}{\kern0pt}F{\isacharcomma}{\kern0pt}\ symmetric{\isacharbrackright}{\kern0pt}\ \isacommand{by}\isamarkupfalse%
\ presburger\isanewline
\ \ \ \ \isacommand{also}\isamarkupfalse%
\ \isacommand{have}\isamarkupfalse%
\ {\isachardoublequoteopen}{\isachardot}{\kern0pt}{\isachardot}{\kern0pt}{\isachardot}{\kern0pt}\ {\isacharequal}{\kern0pt}\ LINT\ x{\isacharcolon}{\kern0pt}A{\isacharbar}{\kern0pt}M{\isachardot}{\kern0pt}\ f\ x{\isachardoublequoteclose}\ \isacommand{using}\isamarkupfalse%
\ limI{\isacharbrackleft}{\kern0pt}OF\ lim{\isacharunderscore}{\kern0pt}s{\isacharunderscore}{\kern0pt}int{\isacharbrackright}{\kern0pt}\ \isacommand{by}\isamarkupfalse%
\ argo\isanewline
\ \ \ \ \isacommand{finally}\isamarkupfalse%
\ \isacommand{have}\isamarkupfalse%
\ {\isachardoublequoteopen}LINT\ x{\isacharcolon}{\kern0pt}A{\isacharbar}{\kern0pt}M{\isachardot}{\kern0pt}\ lim\ {\isacharparenleft}{\kern0pt}{\isasymlambda}n{\isachardot}{\kern0pt}\ cond{\isacharunderscore}{\kern0pt}exp\ M\ F\ {\isacharparenleft}{\kern0pt}s\ {\isacharparenleft}{\kern0pt}r\ n{\isacharparenright}{\kern0pt}{\isacharparenright}{\kern0pt}\ x{\isacharparenright}{\kern0pt}\ {\isacharequal}{\kern0pt}\ LINT\ x{\isacharcolon}{\kern0pt}A{\isacharbar}{\kern0pt}M{\isachardot}{\kern0pt}\ f\ x{\isachardoublequoteclose}\ \isacommand{{\isachardot}{\kern0pt}}\isamarkupfalse%
\isanewline
\ \ \isacommand{{\isacharbraceright}{\kern0pt}}\isamarkupfalse%
\isanewline
\ \ \isacommand{hence}\isamarkupfalse%
\ {\isachardoublequoteopen}has{\isacharunderscore}{\kern0pt}cond{\isacharunderscore}{\kern0pt}exp\ M\ F\ f\ {\isacharparenleft}{\kern0pt}{\isasymlambda}x{\isachardot}{\kern0pt}\ lim\ {\isacharparenleft}{\kern0pt}{\isasymlambda}i{\isachardot}{\kern0pt}\ cond{\isacharunderscore}{\kern0pt}exp\ M\ F\ {\isacharparenleft}{\kern0pt}s\ {\isacharparenleft}{\kern0pt}r\ i{\isacharparenright}{\kern0pt}{\isacharparenright}{\kern0pt}\ x{\isacharparenright}{\kern0pt}{\isacharparenright}{\kern0pt}{\isachardoublequoteclose}\ \isacommand{using}\isamarkupfalse%
\ assms{\isacharparenleft}{\kern0pt}{\isadigit{1}}{\isacharparenright}{\kern0pt}\ lim{\isacharunderscore}{\kern0pt}integrable\ \isacommand{by}\isamarkupfalse%
\ {\isacharparenleft}{\kern0pt}intro\ has{\isacharunderscore}{\kern0pt}cond{\isacharunderscore}{\kern0pt}expI{\isacharprime}{\kern0pt}{\isacharcomma}{\kern0pt}\ auto{\isacharparenright}{\kern0pt}\ \isanewline
\ \ \isacommand{thus}\isamarkupfalse%
\ thesis\ \isacommand{using}\isamarkupfalse%
\ AE{\isacharunderscore}{\kern0pt}Cauchy\ Cauchy{\isacharunderscore}{\kern0pt}convergent\ strict{\isacharunderscore}{\kern0pt}mono{\isacharunderscore}{\kern0pt}r\ \isacommand{by}\isamarkupfalse%
\ {\isacharparenleft}{\kern0pt}auto\ intro{\isacharbang}{\kern0pt}{\isacharcolon}{\kern0pt}\ that{\isacharparenright}{\kern0pt}\isanewline
\isacommand{qed}\isamarkupfalse%
%
\endisatagproof
{\isafoldproof}%
%
\isadelimproof
\isanewline
%
\endisadelimproof
\isanewline
\isacommand{lemma}\isamarkupfalse%
\ cond{\isacharunderscore}{\kern0pt}exp{\isacharunderscore}{\kern0pt}lim{\isacharcolon}{\kern0pt}\isanewline
\ \ \ \ \isakeyword{fixes}\ f\ {\isacharcolon}{\kern0pt}{\isacharcolon}{\kern0pt}\ {\isachardoublequoteopen}{\isacharprime}{\kern0pt}a\ {\isasymRightarrow}\ {\isacharprime}{\kern0pt}b{\isacharcolon}{\kern0pt}{\isacharcolon}{\kern0pt}{\isacharbraceleft}{\kern0pt}second{\isacharunderscore}{\kern0pt}countable{\isacharunderscore}{\kern0pt}topology{\isacharcomma}{\kern0pt}\ banach{\isacharbraceright}{\kern0pt}{\isachardoublequoteclose}\isanewline
\ \ \isakeyword{assumes}\ {\isacharbrackleft}{\kern0pt}measurable{\isacharbrackright}{\kern0pt}{\isacharcolon}{\kern0pt}{\isachardoublequoteopen}integrable\ M\ f{\isachardoublequoteclose}\isanewline
\ \ \ \ \ \ \isakeyword{and}\ {\isachardoublequoteopen}{\isasymAnd}i{\isachardot}{\kern0pt}\ simple{\isacharunderscore}{\kern0pt}function\ M\ {\isacharparenleft}{\kern0pt}s\ i{\isacharparenright}{\kern0pt}{\isachardoublequoteclose}\isanewline
\ \ \ \ \ \ \isakeyword{and}\ {\isachardoublequoteopen}{\isasymAnd}i{\isachardot}{\kern0pt}\ emeasure\ M\ {\isacharbraceleft}{\kern0pt}y\ {\isasymin}\ space\ M{\isachardot}{\kern0pt}\ s\ i\ y\ {\isasymnoteq}\ {\isadigit{0}}{\isacharbraceright}{\kern0pt}\ {\isasymnoteq}\ {\isasyminfinity}{\isachardoublequoteclose}\isanewline
\ \ \ \ \ \ \isakeyword{and}\ {\isachardoublequoteopen}{\isasymAnd}x{\isachardot}{\kern0pt}\ x\ {\isasymin}\ space\ M\ {\isasymLongrightarrow}\ {\isacharparenleft}{\kern0pt}{\isasymlambda}i{\isachardot}{\kern0pt}\ s\ i\ x{\isacharparenright}{\kern0pt}\ {\isasymlonglonglongrightarrow}\ f\ x{\isachardoublequoteclose}\isanewline
\ \ \ \ \ \ \isakeyword{and}\ {\isachardoublequoteopen}{\isasymAnd}x\ i{\isachardot}{\kern0pt}\ x\ {\isasymin}\ space\ M\ {\isasymLongrightarrow}\ norm\ {\isacharparenleft}{\kern0pt}s\ i\ x{\isacharparenright}{\kern0pt}\ {\isasymle}\ {\isadigit{2}}\ {\isacharasterisk}{\kern0pt}\ norm\ {\isacharparenleft}{\kern0pt}f\ x{\isacharparenright}{\kern0pt}{\isachardoublequoteclose}\isanewline
\ \ \isakeyword{obtains}\ r\ \isakeyword{where}\ {\isachardoublequoteopen}AE\ x\ in\ M{\isachardot}{\kern0pt}\ {\isacharparenleft}{\kern0pt}{\isasymlambda}i{\isachardot}{\kern0pt}\ cond{\isacharunderscore}{\kern0pt}exp\ M\ F\ {\isacharparenleft}{\kern0pt}s\ {\isacharparenleft}{\kern0pt}r\ i{\isacharparenright}{\kern0pt}{\isacharparenright}{\kern0pt}\ x{\isacharparenright}{\kern0pt}\ {\isasymlonglonglongrightarrow}\ cond{\isacharunderscore}{\kern0pt}exp\ M\ F\ f\ x{\isachardoublequoteclose}\ {\isachardoublequoteopen}strict{\isacharunderscore}{\kern0pt}mono\ r{\isachardoublequoteclose}\isanewline
%
\isadelimproof
%
\endisadelimproof
%
\isatagproof
\isacommand{proof}\isamarkupfalse%
\ {\isacharminus}{\kern0pt}\isanewline
\ \ \isacommand{obtain}\isamarkupfalse%
\ r\ \isakeyword{where}\ {\isachardoublequoteopen}AE\ x\ in\ M{\isachardot}{\kern0pt}\ cond{\isacharunderscore}{\kern0pt}exp\ M\ F\ f\ x\ {\isacharequal}{\kern0pt}\ lim\ {\isacharparenleft}{\kern0pt}{\isasymlambda}i{\isachardot}{\kern0pt}\ cond{\isacharunderscore}{\kern0pt}exp\ M\ F\ {\isacharparenleft}{\kern0pt}s\ {\isacharparenleft}{\kern0pt}r\ i{\isacharparenright}{\kern0pt}{\isacharparenright}{\kern0pt}\ x{\isacharparenright}{\kern0pt}{\isachardoublequoteclose}\ {\isachardoublequoteopen}AE\ x\ in\ M{\isachardot}{\kern0pt}\ convergent\ {\isacharparenleft}{\kern0pt}{\isasymlambda}i{\isachardot}{\kern0pt}\ cond{\isacharunderscore}{\kern0pt}exp\ M\ F\ {\isacharparenleft}{\kern0pt}s\ {\isacharparenleft}{\kern0pt}r\ i{\isacharparenright}{\kern0pt}{\isacharparenright}{\kern0pt}\ x{\isacharparenright}{\kern0pt}{\isachardoublequoteclose}\ {\isachardoublequoteopen}strict{\isacharunderscore}{\kern0pt}mono\ r{\isachardoublequoteclose}\ \isacommand{using}\isamarkupfalse%
\ has{\isacharunderscore}{\kern0pt}cond{\isacharunderscore}{\kern0pt}exp{\isacharunderscore}{\kern0pt}charact{\isacharparenleft}{\kern0pt}{\isadigit{2}}{\isacharparenright}{\kern0pt}\ \isacommand{by}\isamarkupfalse%
\ {\isacharparenleft}{\kern0pt}auto\ intro{\isacharcolon}{\kern0pt}\ has{\isacharunderscore}{\kern0pt}cond{\isacharunderscore}{\kern0pt}exp{\isacharunderscore}{\kern0pt}lim{\isacharbrackleft}{\kern0pt}OF\ assms{\isacharbrackright}{\kern0pt}{\isacharparenright}{\kern0pt}\isanewline
\ \ \isacommand{thus}\isamarkupfalse%
\ {\isacharquery}{\kern0pt}thesis\ \isacommand{by}\isamarkupfalse%
\ {\isacharparenleft}{\kern0pt}auto\ intro{\isacharbang}{\kern0pt}{\isacharcolon}{\kern0pt}\ that{\isacharbrackleft}{\kern0pt}of\ r{\isacharbrackright}{\kern0pt}\ simp{\isacharcolon}{\kern0pt}\ convergent{\isacharunderscore}{\kern0pt}LIMSEQ{\isacharunderscore}{\kern0pt}iff{\isacharparenright}{\kern0pt}\isanewline
\isacommand{qed}\isamarkupfalse%
%
\endisatagproof
{\isafoldproof}%
%
\isadelimproof
\isanewline
%
\endisadelimproof
\ \ \isanewline
\isacommand{lemma}\isamarkupfalse%
\ has{\isacharunderscore}{\kern0pt}cond{\isacharunderscore}{\kern0pt}expI{\isacharcolon}{\kern0pt}\isanewline
\ \ \isakeyword{fixes}\ f\ {\isacharcolon}{\kern0pt}{\isacharcolon}{\kern0pt}\ {\isachardoublequoteopen}{\isacharprime}{\kern0pt}a\ {\isasymRightarrow}\ {\isacharprime}{\kern0pt}b{\isacharcolon}{\kern0pt}{\isacharcolon}{\kern0pt}{\isacharbraceleft}{\kern0pt}second{\isacharunderscore}{\kern0pt}countable{\isacharunderscore}{\kern0pt}topology{\isacharcomma}{\kern0pt}banach{\isacharbraceright}{\kern0pt}{\isachardoublequoteclose}\isanewline
\ \ \isakeyword{assumes}\ {\isachardoublequoteopen}integrable\ M\ f{\isachardoublequoteclose}\isanewline
\ \ \isakeyword{shows}\ {\isachardoublequoteopen}has{\isacharunderscore}{\kern0pt}cond{\isacharunderscore}{\kern0pt}exp\ M\ F\ f\ {\isacharparenleft}{\kern0pt}cond{\isacharunderscore}{\kern0pt}exp\ M\ F\ f{\isacharparenright}{\kern0pt}{\isachardoublequoteclose}\isanewline
%
\isadelimproof
\ \ %
\endisadelimproof
%
\isatagproof
\isacommand{using}\isamarkupfalse%
\ assms\isanewline
\isacommand{proof}\isamarkupfalse%
\ {\isacharparenleft}{\kern0pt}induction\ rule{\isacharcolon}{\kern0pt}\ integrable{\isacharunderscore}{\kern0pt}induct{\isacharprime}{\kern0pt}{\isacharparenright}{\kern0pt}\isanewline
\ \ \isacommand{case}\isamarkupfalse%
\ {\isacharparenleft}{\kern0pt}base\ A\ c{\isacharparenright}{\kern0pt}\isanewline
\ \ \isacommand{show}\isamarkupfalse%
\ {\isacharquery}{\kern0pt}case\ \isacommand{using}\isamarkupfalse%
\ has{\isacharunderscore}{\kern0pt}cond{\isacharunderscore}{\kern0pt}exp{\isacharunderscore}{\kern0pt}indicator{\isacharbrackleft}{\kern0pt}OF\ base{\isacharparenleft}{\kern0pt}{\isadigit{1}}{\isacharcomma}{\kern0pt}{\isadigit{2}}{\isacharparenright}{\kern0pt}{\isacharbrackright}{\kern0pt}\ has{\isacharunderscore}{\kern0pt}cond{\isacharunderscore}{\kern0pt}exp{\isacharunderscore}{\kern0pt}charact{\isacharparenleft}{\kern0pt}{\isadigit{1}}{\isacharparenright}{\kern0pt}\ \isacommand{by}\isamarkupfalse%
\ blast\isanewline
\isacommand{next}\isamarkupfalse%
\isanewline
\ \ \isacommand{case}\isamarkupfalse%
\ {\isacharparenleft}{\kern0pt}add\ u\ v{\isacharparenright}{\kern0pt}\isanewline
\ \ \isacommand{show}\isamarkupfalse%
\ {\isacharquery}{\kern0pt}case\ \isacommand{using}\isamarkupfalse%
\ has{\isacharunderscore}{\kern0pt}cond{\isacharunderscore}{\kern0pt}exp{\isacharunderscore}{\kern0pt}add{\isacharbrackleft}{\kern0pt}OF\ add{\isacharparenleft}{\kern0pt}{\isadigit{3}}{\isacharcomma}{\kern0pt}{\isadigit{4}}{\isacharparenright}{\kern0pt}{\isacharbrackright}{\kern0pt}\ has{\isacharunderscore}{\kern0pt}cond{\isacharunderscore}{\kern0pt}exp{\isacharunderscore}{\kern0pt}charact{\isacharparenleft}{\kern0pt}{\isadigit{1}}{\isacharparenright}{\kern0pt}\ \isacommand{by}\isamarkupfalse%
\ blast\isanewline
\isacommand{next}\isamarkupfalse%
\isanewline
\ \ \isacommand{case}\isamarkupfalse%
\ {\isacharparenleft}{\kern0pt}lim\ f\ s{\isacharparenright}{\kern0pt}\isanewline
\ \ \isacommand{show}\isamarkupfalse%
\ {\isacharquery}{\kern0pt}case\ \isacommand{using}\isamarkupfalse%
\ has{\isacharunderscore}{\kern0pt}cond{\isacharunderscore}{\kern0pt}exp{\isacharunderscore}{\kern0pt}lim{\isacharbrackleft}{\kern0pt}OF\ lim{\isacharparenleft}{\kern0pt}{\isadigit{1}}{\isacharcomma}{\kern0pt}{\isadigit{3}}{\isacharcomma}{\kern0pt}{\isadigit{4}}{\isacharcomma}{\kern0pt}{\isadigit{5}}{\isacharcomma}{\kern0pt}{\isadigit{6}}{\isacharparenright}{\kern0pt}{\isacharbrackright}{\kern0pt}\ has{\isacharunderscore}{\kern0pt}cond{\isacharunderscore}{\kern0pt}exp{\isacharunderscore}{\kern0pt}charact{\isacharparenleft}{\kern0pt}{\isadigit{1}}{\isacharparenright}{\kern0pt}\ \isacommand{by}\isamarkupfalse%
\ meson\isanewline
\isacommand{qed}\isamarkupfalse%
%
\endisatagproof
{\isafoldproof}%
%
\isadelimproof
\isanewline
%
\endisadelimproof
\isanewline
\isacommand{lemma}\isamarkupfalse%
\ cond{\isacharunderscore}{\kern0pt}exp{\isacharunderscore}{\kern0pt}nested{\isacharunderscore}{\kern0pt}subalg{\isacharcolon}{\kern0pt}\isanewline
\ \ \isakeyword{fixes}\ f\ {\isacharcolon}{\kern0pt}{\isacharcolon}{\kern0pt}\ {\isachardoublequoteopen}{\isacharprime}{\kern0pt}a\ {\isasymRightarrow}\ {\isacharprime}{\kern0pt}b{\isacharcolon}{\kern0pt}{\isacharcolon}{\kern0pt}{\isacharbraceleft}{\kern0pt}second{\isacharunderscore}{\kern0pt}countable{\isacharunderscore}{\kern0pt}topology{\isacharcomma}{\kern0pt}banach{\isacharbraceright}{\kern0pt}{\isachardoublequoteclose}\isanewline
\ \ \isakeyword{assumes}\ {\isachardoublequoteopen}integrable\ M\ f{\isachardoublequoteclose}\ {\isachardoublequoteopen}subalgebra\ M\ G{\isachardoublequoteclose}\ {\isachardoublequoteopen}subalgebra\ G\ F{\isachardoublequoteclose}\isanewline
\ \ \isakeyword{shows}\ {\isachardoublequoteopen}AE\ {\isasymxi}\ in\ M{\isachardot}{\kern0pt}\ cond{\isacharunderscore}{\kern0pt}exp\ M\ F\ f\ {\isasymxi}\ {\isacharequal}{\kern0pt}\ cond{\isacharunderscore}{\kern0pt}exp\ M\ F\ {\isacharparenleft}{\kern0pt}cond{\isacharunderscore}{\kern0pt}exp\ M\ G\ f{\isacharparenright}{\kern0pt}\ {\isasymxi}{\isachardoublequoteclose}\isanewline
%
\isadelimproof
\ \ %
\endisadelimproof
%
\isatagproof
\isacommand{using}\isamarkupfalse%
\ has{\isacharunderscore}{\kern0pt}cond{\isacharunderscore}{\kern0pt}expI\ assms\ sigma{\isacharunderscore}{\kern0pt}finite{\isacharunderscore}{\kern0pt}subalgebra{\isacharunderscore}{\kern0pt}def\ \isacommand{by}\isamarkupfalse%
\ {\isacharparenleft}{\kern0pt}auto\ intro{\isacharbang}{\kern0pt}{\isacharcolon}{\kern0pt}\ has{\isacharunderscore}{\kern0pt}cond{\isacharunderscore}{\kern0pt}exp{\isacharunderscore}{\kern0pt}nested{\isacharunderscore}{\kern0pt}subalg{\isacharbrackleft}{\kern0pt}THEN\ has{\isacharunderscore}{\kern0pt}cond{\isacharunderscore}{\kern0pt}exp{\isacharunderscore}{\kern0pt}charact{\isacharparenleft}{\kern0pt}{\isadigit{2}}{\isacharparenright}{\kern0pt}{\isacharcomma}{\kern0pt}\ THEN\ AE{\isacharunderscore}{\kern0pt}symmetric{\isacharbrackright}{\kern0pt}\ sigma{\isacharunderscore}{\kern0pt}finite{\isacharunderscore}{\kern0pt}subalgebra{\isachardot}{\kern0pt}has{\isacharunderscore}{\kern0pt}cond{\isacharunderscore}{\kern0pt}expI{\isacharbrackleft}{\kern0pt}OF\ sigma{\isacharunderscore}{\kern0pt}finite{\isacharunderscore}{\kern0pt}subalgebra{\isachardot}{\kern0pt}intro{\isacharbrackleft}{\kern0pt}OF\ assms{\isacharparenleft}{\kern0pt}{\isadigit{2}}{\isacharparenright}{\kern0pt}{\isacharbrackright}{\kern0pt}{\isacharbrackright}{\kern0pt}\ nested{\isacharunderscore}{\kern0pt}subalg{\isacharunderscore}{\kern0pt}is{\isacharunderscore}{\kern0pt}sigma{\isacharunderscore}{\kern0pt}finite\ {\isacharparenright}{\kern0pt}%
\endisatagproof
{\isafoldproof}%
%
\isadelimproof
\isanewline
%
\endisadelimproof
\isanewline
\isacommand{lemma}\isamarkupfalse%
\ cond{\isacharunderscore}{\kern0pt}exp{\isacharunderscore}{\kern0pt}set{\isacharunderscore}{\kern0pt}integral{\isacharcolon}{\kern0pt}\isanewline
\ \ \isakeyword{fixes}\ f\ {\isacharcolon}{\kern0pt}{\isacharcolon}{\kern0pt}\ {\isachardoublequoteopen}{\isacharprime}{\kern0pt}a\ {\isasymRightarrow}\ {\isacharprime}{\kern0pt}b{\isacharcolon}{\kern0pt}{\isacharcolon}{\kern0pt}{\isacharbraceleft}{\kern0pt}second{\isacharunderscore}{\kern0pt}countable{\isacharunderscore}{\kern0pt}topology{\isacharcomma}{\kern0pt}banach{\isacharbraceright}{\kern0pt}{\isachardoublequoteclose}\isanewline
\ \ \isakeyword{assumes}\ {\isachardoublequoteopen}integrable\ M\ f{\isachardoublequoteclose}\ {\isachardoublequoteopen}A\ {\isasymin}\ sets\ F{\isachardoublequoteclose}\isanewline
\ \ \isakeyword{shows}\ {\isachardoublequoteopen}{\isacharparenleft}{\kern0pt}{\isasymintegral}\ x\ {\isasymin}\ A{\isachardot}{\kern0pt}\ f\ x\ {\isasympartial}M{\isacharparenright}{\kern0pt}\ {\isacharequal}{\kern0pt}\ {\isacharparenleft}{\kern0pt}{\isasymintegral}\ x\ {\isasymin}\ A{\isachardot}{\kern0pt}\ cond{\isacharunderscore}{\kern0pt}exp\ M\ F\ f\ x\ {\isasympartial}M{\isacharparenright}{\kern0pt}{\isachardoublequoteclose}\isanewline
%
\isadelimproof
\ \ %
\endisadelimproof
%
\isatagproof
\isacommand{using}\isamarkupfalse%
\ has{\isacharunderscore}{\kern0pt}cond{\isacharunderscore}{\kern0pt}expD{\isacharparenleft}{\kern0pt}{\isadigit{1}}{\isacharparenright}{\kern0pt}{\isacharbrackleft}{\kern0pt}OF\ has{\isacharunderscore}{\kern0pt}cond{\isacharunderscore}{\kern0pt}expI{\isacharcomma}{\kern0pt}\ OF\ assms{\isacharbrackright}{\kern0pt}\ \isacommand{by}\isamarkupfalse%
\ argo%
\endisatagproof
{\isafoldproof}%
%
\isadelimproof
\isanewline
%
\endisadelimproof
\isanewline
\isacommand{lemma}\isamarkupfalse%
\ cond{\isacharunderscore}{\kern0pt}exp{\isacharunderscore}{\kern0pt}add{\isacharcolon}{\kern0pt}\isanewline
\ \ \isakeyword{fixes}\ f\ {\isacharcolon}{\kern0pt}{\isacharcolon}{\kern0pt}\ {\isachardoublequoteopen}{\isacharprime}{\kern0pt}a\ {\isasymRightarrow}\ {\isacharprime}{\kern0pt}b{\isacharcolon}{\kern0pt}{\isacharcolon}{\kern0pt}{\isacharbraceleft}{\kern0pt}second{\isacharunderscore}{\kern0pt}countable{\isacharunderscore}{\kern0pt}topology{\isacharcomma}{\kern0pt}banach{\isacharbraceright}{\kern0pt}{\isachardoublequoteclose}\isanewline
\ \ \isakeyword{assumes}\ {\isachardoublequoteopen}integrable\ M\ f{\isachardoublequoteclose}\ {\isachardoublequoteopen}integrable\ M\ g{\isachardoublequoteclose}\isanewline
\ \ \isakeyword{shows}\ {\isachardoublequoteopen}AE\ x\ in\ M{\isachardot}{\kern0pt}\ cond{\isacharunderscore}{\kern0pt}exp\ M\ F\ {\isacharparenleft}{\kern0pt}{\isasymlambda}x{\isachardot}{\kern0pt}\ f\ x\ {\isacharplus}{\kern0pt}\ g\ x{\isacharparenright}{\kern0pt}\ x\ {\isacharequal}{\kern0pt}\ cond{\isacharunderscore}{\kern0pt}exp\ M\ F\ f\ x\ {\isacharplus}{\kern0pt}\ cond{\isacharunderscore}{\kern0pt}exp\ M\ F\ g\ x{\isachardoublequoteclose}\isanewline
%
\isadelimproof
\ \ %
\endisadelimproof
%
\isatagproof
\isacommand{using}\isamarkupfalse%
\ has{\isacharunderscore}{\kern0pt}cond{\isacharunderscore}{\kern0pt}exp{\isacharunderscore}{\kern0pt}add{\isacharbrackleft}{\kern0pt}OF\ has{\isacharunderscore}{\kern0pt}cond{\isacharunderscore}{\kern0pt}expI{\isacharparenleft}{\kern0pt}{\isadigit{1}}{\isacharcomma}{\kern0pt}{\isadigit{1}}{\isacharparenright}{\kern0pt}{\isacharcomma}{\kern0pt}\ OF\ assms{\isacharcomma}{\kern0pt}\ THEN\ has{\isacharunderscore}{\kern0pt}cond{\isacharunderscore}{\kern0pt}exp{\isacharunderscore}{\kern0pt}charact{\isacharparenleft}{\kern0pt}{\isadigit{2}}{\isacharparenright}{\kern0pt}{\isacharbrackright}{\kern0pt}\ \isacommand{{\isachardot}{\kern0pt}}\isamarkupfalse%
%
\endisatagproof
{\isafoldproof}%
%
\isadelimproof
\isanewline
%
\endisadelimproof
\isanewline
\isacommand{lemma}\isamarkupfalse%
\ cond{\isacharunderscore}{\kern0pt}exp{\isacharunderscore}{\kern0pt}diff{\isacharcolon}{\kern0pt}\isanewline
\ \ \isakeyword{fixes}\ f\ {\isacharcolon}{\kern0pt}{\isacharcolon}{\kern0pt}\ {\isachardoublequoteopen}{\isacharprime}{\kern0pt}a\ {\isasymRightarrow}\ {\isacharprime}{\kern0pt}b\ {\isacharcolon}{\kern0pt}{\isacharcolon}{\kern0pt}\ {\isacharbraceleft}{\kern0pt}second{\isacharunderscore}{\kern0pt}countable{\isacharunderscore}{\kern0pt}topology{\isacharcomma}{\kern0pt}\ banach{\isacharbraceright}{\kern0pt}{\isachardoublequoteclose}\isanewline
\ \ \isakeyword{assumes}\ {\isachardoublequoteopen}integrable\ M\ f{\isachardoublequoteclose}\ {\isachardoublequoteopen}integrable\ M\ g{\isachardoublequoteclose}\isanewline
\ \ \isakeyword{shows}\ {\isachardoublequoteopen}AE\ x\ in\ M{\isachardot}{\kern0pt}\ cond{\isacharunderscore}{\kern0pt}exp\ M\ F\ {\isacharparenleft}{\kern0pt}{\isasymlambda}x{\isachardot}{\kern0pt}\ f\ x\ {\isacharminus}{\kern0pt}\ g\ x{\isacharparenright}{\kern0pt}\ x\ {\isacharequal}{\kern0pt}\ cond{\isacharunderscore}{\kern0pt}exp\ M\ F\ f\ x\ {\isacharminus}{\kern0pt}\ cond{\isacharunderscore}{\kern0pt}exp\ M\ F\ g\ x{\isachardoublequoteclose}\isanewline
%
\isadelimproof
\ \ %
\endisadelimproof
%
\isatagproof
\isacommand{using}\isamarkupfalse%
\ has{\isacharunderscore}{\kern0pt}cond{\isacharunderscore}{\kern0pt}exp{\isacharunderscore}{\kern0pt}add{\isacharbrackleft}{\kern0pt}OF\ {\isacharunderscore}{\kern0pt}\ has{\isacharunderscore}{\kern0pt}cond{\isacharunderscore}{\kern0pt}exp{\isacharunderscore}{\kern0pt}scaleR{\isacharunderscore}{\kern0pt}right{\isacharcomma}{\kern0pt}\ OF\ has{\isacharunderscore}{\kern0pt}cond{\isacharunderscore}{\kern0pt}expI{\isacharparenleft}{\kern0pt}{\isadigit{1}}{\isacharcomma}{\kern0pt}{\isadigit{1}}{\isacharparenright}{\kern0pt}{\isacharcomma}{\kern0pt}\ OF\ assms{\isacharcomma}{\kern0pt}\ THEN\ has{\isacharunderscore}{\kern0pt}cond{\isacharunderscore}{\kern0pt}exp{\isacharunderscore}{\kern0pt}charact{\isacharparenleft}{\kern0pt}{\isadigit{2}}{\isacharparenright}{\kern0pt}{\isacharcomma}{\kern0pt}\ of\ {\isachardoublequoteopen}{\isacharminus}{\kern0pt}{\isadigit{1}}{\isachardoublequoteclose}{\isacharbrackright}{\kern0pt}\ \isacommand{by}\isamarkupfalse%
\ simp%
\endisatagproof
{\isafoldproof}%
%
\isadelimproof
\isanewline
%
\endisadelimproof
\isanewline
\isacommand{lemma}\isamarkupfalse%
\ cond{\isacharunderscore}{\kern0pt}exp{\isacharunderscore}{\kern0pt}diff{\isacharprime}{\kern0pt}{\isacharcolon}{\kern0pt}\isanewline
\ \ \isakeyword{fixes}\ f\ {\isacharcolon}{\kern0pt}{\isacharcolon}{\kern0pt}\ {\isachardoublequoteopen}{\isacharprime}{\kern0pt}a\ {\isasymRightarrow}\ {\isacharprime}{\kern0pt}b\ {\isacharcolon}{\kern0pt}{\isacharcolon}{\kern0pt}\ {\isacharbraceleft}{\kern0pt}second{\isacharunderscore}{\kern0pt}countable{\isacharunderscore}{\kern0pt}topology{\isacharcomma}{\kern0pt}\ banach{\isacharbraceright}{\kern0pt}{\isachardoublequoteclose}\isanewline
\ \ \isakeyword{assumes}\ {\isachardoublequoteopen}integrable\ M\ f{\isachardoublequoteclose}\ {\isachardoublequoteopen}integrable\ M\ g{\isachardoublequoteclose}\isanewline
\ \ \isakeyword{shows}\ {\isachardoublequoteopen}AE\ x\ in\ M{\isachardot}{\kern0pt}\ cond{\isacharunderscore}{\kern0pt}exp\ M\ F\ {\isacharparenleft}{\kern0pt}f\ {\isacharminus}{\kern0pt}\ g{\isacharparenright}{\kern0pt}\ x\ {\isacharequal}{\kern0pt}\ cond{\isacharunderscore}{\kern0pt}exp\ M\ F\ f\ x\ {\isacharminus}{\kern0pt}\ cond{\isacharunderscore}{\kern0pt}exp\ M\ F\ g\ x{\isachardoublequoteclose}\isanewline
%
\isadelimproof
\ \ %
\endisadelimproof
%
\isatagproof
\isacommand{unfolding}\isamarkupfalse%
\ fun{\isacharunderscore}{\kern0pt}diff{\isacharunderscore}{\kern0pt}def\ \isacommand{using}\isamarkupfalse%
\ assms\ \isacommand{by}\isamarkupfalse%
\ {\isacharparenleft}{\kern0pt}rule\ cond{\isacharunderscore}{\kern0pt}exp{\isacharunderscore}{\kern0pt}diff{\isacharparenright}{\kern0pt}%
\endisatagproof
{\isafoldproof}%
%
\isadelimproof
\isanewline
%
\endisadelimproof
\isanewline
\isacommand{lemma}\isamarkupfalse%
\ cond{\isacharunderscore}{\kern0pt}exp{\isacharunderscore}{\kern0pt}contraction{\isacharcolon}{\kern0pt}\isanewline
\ \ \isakeyword{fixes}\ f\ {\isacharcolon}{\kern0pt}{\isacharcolon}{\kern0pt}\ {\isachardoublequoteopen}{\isacharprime}{\kern0pt}a\ {\isasymRightarrow}\ {\isacharprime}{\kern0pt}b{\isacharcolon}{\kern0pt}{\isacharcolon}{\kern0pt}{\isacharbraceleft}{\kern0pt}second{\isacharunderscore}{\kern0pt}countable{\isacharunderscore}{\kern0pt}topology{\isacharcomma}{\kern0pt}\ banach{\isacharbraceright}{\kern0pt}{\isachardoublequoteclose}\isanewline
\ \ \isakeyword{assumes}\ {\isachardoublequoteopen}integrable\ M\ f{\isachardoublequoteclose}\isanewline
\ \ \isakeyword{shows}\ {\isachardoublequoteopen}AE\ x\ in\ M{\isachardot}{\kern0pt}\ norm\ {\isacharparenleft}{\kern0pt}cond{\isacharunderscore}{\kern0pt}exp\ M\ F\ f\ x{\isacharparenright}{\kern0pt}\ {\isasymle}\ cond{\isacharunderscore}{\kern0pt}exp\ M\ F\ {\isacharparenleft}{\kern0pt}{\isasymlambda}x{\isachardot}{\kern0pt}\ norm\ {\isacharparenleft}{\kern0pt}f\ x{\isacharparenright}{\kern0pt}{\isacharparenright}{\kern0pt}\ x{\isachardoublequoteclose}\ \isanewline
%
\isadelimproof
%
\endisadelimproof
%
\isatagproof
\isacommand{proof}\isamarkupfalse%
\ {\isacharminus}{\kern0pt}\isanewline
\ \ \isacommand{obtain}\isamarkupfalse%
\ s\ \isakeyword{where}\ s{\isacharcolon}{\kern0pt}\ {\isachardoublequoteopen}{\isasymAnd}i{\isachardot}{\kern0pt}\ simple{\isacharunderscore}{\kern0pt}function\ M\ {\isacharparenleft}{\kern0pt}s\ i{\isacharparenright}{\kern0pt}{\isachardoublequoteclose}\ {\isachardoublequoteopen}{\isasymAnd}i{\isachardot}{\kern0pt}\ emeasure\ M\ {\isacharbraceleft}{\kern0pt}y\ {\isasymin}\ space\ M{\isachardot}{\kern0pt}\ s\ i\ y\ {\isasymnoteq}\ {\isadigit{0}}{\isacharbraceright}{\kern0pt}\ {\isasymnoteq}\ {\isasyminfinity}{\isachardoublequoteclose}\ {\isachardoublequoteopen}{\isasymAnd}x{\isachardot}{\kern0pt}\ x\ {\isasymin}\ space\ M\ {\isasymLongrightarrow}\ {\isacharparenleft}{\kern0pt}{\isasymlambda}i{\isachardot}{\kern0pt}\ s\ i\ x{\isacharparenright}{\kern0pt}\ {\isasymlonglonglongrightarrow}\ f\ x{\isachardoublequoteclose}\ {\isachardoublequoteopen}{\isasymAnd}i\ x{\isachardot}{\kern0pt}\ x\ {\isasymin}\ space\ M\ {\isasymLongrightarrow}\ norm\ {\isacharparenleft}{\kern0pt}s\ i\ x{\isacharparenright}{\kern0pt}\ {\isasymle}\ {\isadigit{2}}\ {\isacharasterisk}{\kern0pt}\ norm\ {\isacharparenleft}{\kern0pt}f\ x{\isacharparenright}{\kern0pt}{\isachardoublequoteclose}\ \isanewline
\ \ \ \ \isacommand{by}\isamarkupfalse%
\ {\isacharparenleft}{\kern0pt}blast\ intro{\isacharcolon}{\kern0pt}\ integrable{\isacharunderscore}{\kern0pt}implies{\isacharunderscore}{\kern0pt}simple{\isacharunderscore}{\kern0pt}function{\isacharunderscore}{\kern0pt}sequence{\isacharbrackleft}{\kern0pt}OF\ assms{\isacharbrackright}{\kern0pt}{\isacharparenright}{\kern0pt}\isanewline
\isanewline
\ \ \isacommand{obtain}\isamarkupfalse%
\ r\ \isakeyword{where}\ r{\isacharcolon}{\kern0pt}\ {\isachardoublequoteopen}AE\ x\ in\ M{\isachardot}{\kern0pt}\ {\isacharparenleft}{\kern0pt}{\isasymlambda}i{\isachardot}{\kern0pt}\ cond{\isacharunderscore}{\kern0pt}exp\ M\ F\ {\isacharparenleft}{\kern0pt}s\ {\isacharparenleft}{\kern0pt}r\ i{\isacharparenright}{\kern0pt}{\isacharparenright}{\kern0pt}\ x{\isacharparenright}{\kern0pt}\ {\isasymlonglonglongrightarrow}\ cond{\isacharunderscore}{\kern0pt}exp\ M\ F\ f\ x{\isachardoublequoteclose}\ {\isachardoublequoteopen}strict{\isacharunderscore}{\kern0pt}mono\ r{\isachardoublequoteclose}\ \isacommand{using}\isamarkupfalse%
\ cond{\isacharunderscore}{\kern0pt}exp{\isacharunderscore}{\kern0pt}lim{\isacharbrackleft}{\kern0pt}OF\ assms\ s{\isacharbrackright}{\kern0pt}\ \isacommand{by}\isamarkupfalse%
\ blast\isanewline
\isanewline
\ \ \isacommand{have}\isamarkupfalse%
\ norm{\isacharunderscore}{\kern0pt}s{\isacharunderscore}{\kern0pt}r{\isacharcolon}{\kern0pt}\ {\isachardoublequoteopen}{\isasymAnd}i{\isachardot}{\kern0pt}\ simple{\isacharunderscore}{\kern0pt}function\ M\ {\isacharparenleft}{\kern0pt}{\isasymlambda}x{\isachardot}{\kern0pt}\ norm\ {\isacharparenleft}{\kern0pt}s\ {\isacharparenleft}{\kern0pt}r\ i{\isacharparenright}{\kern0pt}\ x{\isacharparenright}{\kern0pt}{\isacharparenright}{\kern0pt}{\isachardoublequoteclose}\ {\isachardoublequoteopen}{\isasymAnd}i{\isachardot}{\kern0pt}\ emeasure\ M\ {\isacharbraceleft}{\kern0pt}y\ {\isasymin}\ space\ M{\isachardot}{\kern0pt}\ norm\ {\isacharparenleft}{\kern0pt}s\ {\isacharparenleft}{\kern0pt}r\ i{\isacharparenright}{\kern0pt}\ y{\isacharparenright}{\kern0pt}\ {\isasymnoteq}\ {\isadigit{0}}{\isacharbraceright}{\kern0pt}\ {\isasymnoteq}\ {\isasyminfinity}{\isachardoublequoteclose}\ {\isachardoublequoteopen}{\isasymAnd}x{\isachardot}{\kern0pt}\ x\ {\isasymin}\ space\ M\ {\isasymLongrightarrow}\ {\isacharparenleft}{\kern0pt}{\isasymlambda}i{\isachardot}{\kern0pt}\ norm\ {\isacharparenleft}{\kern0pt}s\ {\isacharparenleft}{\kern0pt}r\ i{\isacharparenright}{\kern0pt}\ x{\isacharparenright}{\kern0pt}{\isacharparenright}{\kern0pt}\ {\isasymlonglonglongrightarrow}\ norm\ {\isacharparenleft}{\kern0pt}f\ x{\isacharparenright}{\kern0pt}{\isachardoublequoteclose}\ {\isachardoublequoteopen}{\isasymAnd}i\ x{\isachardot}{\kern0pt}\ x\ {\isasymin}\ space\ M\ {\isasymLongrightarrow}\ norm\ {\isacharparenleft}{\kern0pt}norm\ {\isacharparenleft}{\kern0pt}s\ {\isacharparenleft}{\kern0pt}r\ i{\isacharparenright}{\kern0pt}\ x{\isacharparenright}{\kern0pt}{\isacharparenright}{\kern0pt}\ {\isasymle}\ {\isadigit{2}}\ {\isacharasterisk}{\kern0pt}\ norm\ {\isacharparenleft}{\kern0pt}norm\ {\isacharparenleft}{\kern0pt}f\ x{\isacharparenright}{\kern0pt}{\isacharparenright}{\kern0pt}{\isachardoublequoteclose}\ \isanewline
\ \ \ \ \isacommand{using}\isamarkupfalse%
\ s\ \isacommand{by}\isamarkupfalse%
\ {\isacharparenleft}{\kern0pt}auto\ intro{\isacharcolon}{\kern0pt}\ LIMSEQ{\isacharunderscore}{\kern0pt}subseq{\isacharunderscore}{\kern0pt}LIMSEQ{\isacharbrackleft}{\kern0pt}OF\ tendsto{\isacharunderscore}{\kern0pt}norm\ r{\isacharparenleft}{\kern0pt}{\isadigit{2}}{\isacharparenright}{\kern0pt}{\isacharcomma}{\kern0pt}\ unfolded\ comp{\isacharunderscore}{\kern0pt}def{\isacharbrackright}{\kern0pt}\ simple{\isacharunderscore}{\kern0pt}function{\isacharunderscore}{\kern0pt}compose{\isadigit{1}}{\isacharparenright}{\kern0pt}\ \isanewline
\ \ \isanewline
\ \ \isacommand{obtain}\isamarkupfalse%
\ r{\isacharprime}{\kern0pt}\ \isakeyword{where}\ r{\isacharprime}{\kern0pt}{\isacharcolon}{\kern0pt}\ {\isachardoublequoteopen}AE\ x\ in\ M{\isachardot}{\kern0pt}\ {\isacharparenleft}{\kern0pt}{\isasymlambda}i{\isachardot}{\kern0pt}\ {\isacharparenleft}{\kern0pt}cond{\isacharunderscore}{\kern0pt}exp\ M\ F\ {\isacharparenleft}{\kern0pt}{\isasymlambda}x{\isachardot}{\kern0pt}\ norm\ {\isacharparenleft}{\kern0pt}s\ {\isacharparenleft}{\kern0pt}r\ {\isacharparenleft}{\kern0pt}r{\isacharprime}{\kern0pt}\ i{\isacharparenright}{\kern0pt}{\isacharparenright}{\kern0pt}\ x{\isacharparenright}{\kern0pt}{\isacharparenright}{\kern0pt}\ x{\isacharparenright}{\kern0pt}{\isacharparenright}{\kern0pt}\ {\isasymlonglonglongrightarrow}\ cond{\isacharunderscore}{\kern0pt}exp\ M\ F\ {\isacharparenleft}{\kern0pt}{\isasymlambda}x{\isachardot}{\kern0pt}\ norm\ {\isacharparenleft}{\kern0pt}f\ x{\isacharparenright}{\kern0pt}{\isacharparenright}{\kern0pt}\ x{\isachardoublequoteclose}\ {\isachardoublequoteopen}strict{\isacharunderscore}{\kern0pt}mono\ r{\isacharprime}{\kern0pt}{\isachardoublequoteclose}\ \isacommand{using}\isamarkupfalse%
\ cond{\isacharunderscore}{\kern0pt}exp{\isacharunderscore}{\kern0pt}lim{\isacharbrackleft}{\kern0pt}OF\ integrable{\isacharunderscore}{\kern0pt}norm\ norm{\isacharunderscore}{\kern0pt}s{\isacharunderscore}{\kern0pt}r{\isacharcomma}{\kern0pt}\ OF\ assms{\isacharbrackright}{\kern0pt}\ \isacommand{by}\isamarkupfalse%
\ blast\isanewline
\isanewline
\ \ \isacommand{have}\isamarkupfalse%
\ {\isachardoublequoteopen}AE\ x\ in\ M{\isachardot}{\kern0pt}\ {\isasymforall}i{\isachardot}{\kern0pt}\ norm\ {\isacharparenleft}{\kern0pt}cond{\isacharunderscore}{\kern0pt}exp\ M\ F\ {\isacharparenleft}{\kern0pt}s\ {\isacharparenleft}{\kern0pt}r\ {\isacharparenleft}{\kern0pt}r{\isacharprime}{\kern0pt}\ i{\isacharparenright}{\kern0pt}{\isacharparenright}{\kern0pt}{\isacharparenright}{\kern0pt}\ x{\isacharparenright}{\kern0pt}\ {\isasymle}\ cond{\isacharunderscore}{\kern0pt}exp\ M\ F\ {\isacharparenleft}{\kern0pt}{\isasymlambda}x{\isachardot}{\kern0pt}\ norm\ {\isacharparenleft}{\kern0pt}s\ {\isacharparenleft}{\kern0pt}r\ {\isacharparenleft}{\kern0pt}r{\isacharprime}{\kern0pt}\ i{\isacharparenright}{\kern0pt}{\isacharparenright}{\kern0pt}\ x{\isacharparenright}{\kern0pt}{\isacharparenright}{\kern0pt}\ x{\isachardoublequoteclose}\ \isacommand{using}\isamarkupfalse%
\ s\ \isacommand{by}\isamarkupfalse%
\ {\isacharparenleft}{\kern0pt}auto\ intro{\isacharcolon}{\kern0pt}\ cond{\isacharunderscore}{\kern0pt}exp{\isacharunderscore}{\kern0pt}contraction{\isacharunderscore}{\kern0pt}simple\ simp\ add{\isacharcolon}{\kern0pt}\ AE{\isacharunderscore}{\kern0pt}all{\isacharunderscore}{\kern0pt}countable{\isacharparenright}{\kern0pt}\isanewline
\ \ \isacommand{moreover}\isamarkupfalse%
\ \isacommand{have}\isamarkupfalse%
\ {\isachardoublequoteopen}AE\ x\ in\ M{\isachardot}{\kern0pt}\ {\isacharparenleft}{\kern0pt}{\isasymlambda}i{\isachardot}{\kern0pt}\ norm\ {\isacharparenleft}{\kern0pt}cond{\isacharunderscore}{\kern0pt}exp\ M\ F\ {\isacharparenleft}{\kern0pt}s\ {\isacharparenleft}{\kern0pt}r\ {\isacharparenleft}{\kern0pt}r{\isacharprime}{\kern0pt}\ i{\isacharparenright}{\kern0pt}{\isacharparenright}{\kern0pt}{\isacharparenright}{\kern0pt}\ x{\isacharparenright}{\kern0pt}{\isacharparenright}{\kern0pt}\ {\isasymlonglonglongrightarrow}\ norm\ {\isacharparenleft}{\kern0pt}cond{\isacharunderscore}{\kern0pt}exp\ M\ F\ f\ x{\isacharparenright}{\kern0pt}{\isachardoublequoteclose}\ \isacommand{using}\isamarkupfalse%
\ r\ LIMSEQ{\isacharunderscore}{\kern0pt}subseq{\isacharunderscore}{\kern0pt}LIMSEQ{\isacharbrackleft}{\kern0pt}OF\ tendsto{\isacharunderscore}{\kern0pt}norm\ r{\isacharprime}{\kern0pt}{\isacharparenleft}{\kern0pt}{\isadigit{2}}{\isacharparenright}{\kern0pt}{\isacharcomma}{\kern0pt}\ unfolded\ comp{\isacharunderscore}{\kern0pt}def{\isacharbrackright}{\kern0pt}\ \isacommand{by}\isamarkupfalse%
\ fast\isanewline
\ \ \isacommand{ultimately}\isamarkupfalse%
\ \isacommand{show}\isamarkupfalse%
\ {\isacharquery}{\kern0pt}thesis\ \isacommand{using}\isamarkupfalse%
\ LIMSEQ{\isacharunderscore}{\kern0pt}le\ r{\isacharprime}{\kern0pt}{\isacharparenleft}{\kern0pt}{\isadigit{1}}{\isacharparenright}{\kern0pt}\ \isacommand{by}\isamarkupfalse%
\ fast\isanewline
\isacommand{qed}\isamarkupfalse%
%
\endisatagproof
{\isafoldproof}%
%
\isadelimproof
\isanewline
%
\endisadelimproof
\ \ \isanewline
\isacommand{lemma}\isamarkupfalse%
\ cond{\isacharunderscore}{\kern0pt}exp{\isacharunderscore}{\kern0pt}sum\ {\isacharbrackleft}{\kern0pt}intro{\isacharcomma}{\kern0pt}\ simp{\isacharbrackright}{\kern0pt}{\isacharcolon}{\kern0pt}\isanewline
\ \ \isakeyword{fixes}\ f\ {\isacharcolon}{\kern0pt}{\isacharcolon}{\kern0pt}\ {\isachardoublequoteopen}{\isacharprime}{\kern0pt}t\ {\isasymRightarrow}\ {\isacharprime}{\kern0pt}a\ {\isasymRightarrow}\ {\isacharprime}{\kern0pt}b\ {\isacharcolon}{\kern0pt}{\isacharcolon}{\kern0pt}\ {\isacharbraceleft}{\kern0pt}second{\isacharunderscore}{\kern0pt}countable{\isacharunderscore}{\kern0pt}topology{\isacharcomma}{\kern0pt}banach{\isacharbraceright}{\kern0pt}{\isachardoublequoteclose}\isanewline
\ \ \isakeyword{assumes}\ {\isacharbrackleft}{\kern0pt}measurable{\isacharbrackright}{\kern0pt}{\isacharcolon}{\kern0pt}\ {\isachardoublequoteopen}{\isasymAnd}i{\isachardot}{\kern0pt}\ integrable\ M\ {\isacharparenleft}{\kern0pt}f\ i{\isacharparenright}{\kern0pt}{\isachardoublequoteclose}\isanewline
\ \ \isakeyword{shows}\ {\isachardoublequoteopen}AE\ x\ in\ M{\isachardot}{\kern0pt}\ cond{\isacharunderscore}{\kern0pt}exp\ M\ F\ {\isacharparenleft}{\kern0pt}{\isasymlambda}x{\isachardot}{\kern0pt}\ {\isasymSum}i{\isasymin}I{\isachardot}{\kern0pt}\ f\ i\ x{\isacharparenright}{\kern0pt}\ x\ {\isacharequal}{\kern0pt}\ {\isacharparenleft}{\kern0pt}{\isasymSum}i{\isasymin}I{\isachardot}{\kern0pt}\ cond{\isacharunderscore}{\kern0pt}exp\ M\ F\ {\isacharparenleft}{\kern0pt}f\ i{\isacharparenright}{\kern0pt}\ x{\isacharparenright}{\kern0pt}{\isachardoublequoteclose}\isanewline
%
\isadelimproof
%
\endisadelimproof
%
\isatagproof
\isacommand{proof}\isamarkupfalse%
\ {\isacharparenleft}{\kern0pt}rule\ has{\isacharunderscore}{\kern0pt}cond{\isacharunderscore}{\kern0pt}exp{\isacharunderscore}{\kern0pt}charact{\isacharcomma}{\kern0pt}\ intro\ has{\isacharunderscore}{\kern0pt}cond{\isacharunderscore}{\kern0pt}expI{\isacharprime}{\kern0pt}{\isacharparenright}{\kern0pt}\isanewline
\ \ \isacommand{fix}\isamarkupfalse%
\ A\ \isacommand{assume}\isamarkupfalse%
\ {\isacharbrackleft}{\kern0pt}measurable{\isacharbrackright}{\kern0pt}{\isacharcolon}{\kern0pt}\ {\isachardoublequoteopen}A\ {\isasymin}\ sets\ F{\isachardoublequoteclose}\isanewline
\ \ \isacommand{then}\isamarkupfalse%
\ \isacommand{have}\isamarkupfalse%
\ A{\isacharunderscore}{\kern0pt}meas\ {\isacharbrackleft}{\kern0pt}measurable{\isacharbrackright}{\kern0pt}{\isacharcolon}{\kern0pt}\ {\isachardoublequoteopen}A\ {\isasymin}\ sets\ M{\isachardoublequoteclose}\ \isacommand{by}\isamarkupfalse%
\ {\isacharparenleft}{\kern0pt}meson\ subsetD\ subalg\ subalgebra{\isacharunderscore}{\kern0pt}def{\isacharparenright}{\kern0pt}\isanewline
\isanewline
\ \ \isacommand{have}\isamarkupfalse%
\ {\isachardoublequoteopen}{\isacharparenleft}{\kern0pt}{\isasymintegral}x{\isasymin}A{\isachardot}{\kern0pt}\ {\isacharparenleft}{\kern0pt}{\isasymSum}i{\isasymin}I{\isachardot}{\kern0pt}\ f\ i\ x{\isacharparenright}{\kern0pt}{\isasympartial}M{\isacharparenright}{\kern0pt}\ {\isacharequal}{\kern0pt}\ {\isacharparenleft}{\kern0pt}{\isasymintegral}x{\isachardot}{\kern0pt}\ {\isacharparenleft}{\kern0pt}{\isasymSum}i{\isasymin}I{\isachardot}{\kern0pt}\ indicator\ A\ x\ {\isacharasterisk}{\kern0pt}\isactrlsub R\ f\ i\ x{\isacharparenright}{\kern0pt}{\isasympartial}M{\isacharparenright}{\kern0pt}{\isachardoublequoteclose}\ \isacommand{unfolding}\isamarkupfalse%
\ set{\isacharunderscore}{\kern0pt}lebesgue{\isacharunderscore}{\kern0pt}integral{\isacharunderscore}{\kern0pt}def\ \isacommand{by}\isamarkupfalse%
\ {\isacharparenleft}{\kern0pt}simp\ add{\isacharcolon}{\kern0pt}\ scaleR{\isacharunderscore}{\kern0pt}sum{\isacharunderscore}{\kern0pt}right{\isacharparenright}{\kern0pt}\isanewline
\ \ \isacommand{also}\isamarkupfalse%
\ \isacommand{have}\isamarkupfalse%
\ {\isachardoublequoteopen}{\isachardot}{\kern0pt}{\isachardot}{\kern0pt}{\isachardot}{\kern0pt}\ {\isacharequal}{\kern0pt}\ {\isacharparenleft}{\kern0pt}{\isasymSum}i{\isasymin}I{\isachardot}{\kern0pt}\ {\isacharparenleft}{\kern0pt}{\isasymintegral}x{\isachardot}{\kern0pt}\ indicator\ A\ x\ {\isacharasterisk}{\kern0pt}\isactrlsub R\ f\ i\ x\ {\isasympartial}M{\isacharparenright}{\kern0pt}{\isacharparenright}{\kern0pt}{\isachardoublequoteclose}\ \isacommand{using}\isamarkupfalse%
\ assms\ \isacommand{by}\isamarkupfalse%
\ {\isacharparenleft}{\kern0pt}auto\ intro{\isacharbang}{\kern0pt}{\isacharcolon}{\kern0pt}\ Bochner{\isacharunderscore}{\kern0pt}Integration{\isachardot}{\kern0pt}integral{\isacharunderscore}{\kern0pt}sum\ integrable{\isacharunderscore}{\kern0pt}mult{\isacharunderscore}{\kern0pt}indicator{\isacharparenright}{\kern0pt}\isanewline
\ \ \isacommand{also}\isamarkupfalse%
\ \isacommand{have}\isamarkupfalse%
\ {\isachardoublequoteopen}{\isachardot}{\kern0pt}{\isachardot}{\kern0pt}{\isachardot}{\kern0pt}\ {\isacharequal}{\kern0pt}\ {\isacharparenleft}{\kern0pt}{\isasymSum}i{\isasymin}I{\isachardot}{\kern0pt}\ {\isacharparenleft}{\kern0pt}{\isasymintegral}x{\isachardot}{\kern0pt}\ indicator\ A\ x\ {\isacharasterisk}{\kern0pt}\isactrlsub R\ cond{\isacharunderscore}{\kern0pt}exp\ M\ F\ {\isacharparenleft}{\kern0pt}f\ i{\isacharparenright}{\kern0pt}\ x\ {\isasympartial}M{\isacharparenright}{\kern0pt}{\isacharparenright}{\kern0pt}{\isachardoublequoteclose}\ \isacommand{using}\isamarkupfalse%
\ cond{\isacharunderscore}{\kern0pt}exp{\isacharunderscore}{\kern0pt}set{\isacharunderscore}{\kern0pt}integral{\isacharbrackleft}{\kern0pt}OF\ assms{\isacharbrackright}{\kern0pt}\ \isacommand{by}\isamarkupfalse%
\ {\isacharparenleft}{\kern0pt}simp\ add{\isacharcolon}{\kern0pt}\ set{\isacharunderscore}{\kern0pt}lebesgue{\isacharunderscore}{\kern0pt}integral{\isacharunderscore}{\kern0pt}def{\isacharparenright}{\kern0pt}\isanewline
\ \ \isacommand{also}\isamarkupfalse%
\ \isacommand{have}\isamarkupfalse%
\ {\isachardoublequoteopen}{\isachardot}{\kern0pt}{\isachardot}{\kern0pt}{\isachardot}{\kern0pt}\ {\isacharequal}{\kern0pt}\ {\isacharparenleft}{\kern0pt}{\isasymintegral}x{\isachardot}{\kern0pt}\ {\isacharparenleft}{\kern0pt}{\isasymSum}i{\isasymin}I{\isachardot}{\kern0pt}\ indicator\ A\ x\ {\isacharasterisk}{\kern0pt}\isactrlsub R\ cond{\isacharunderscore}{\kern0pt}exp\ M\ F\ {\isacharparenleft}{\kern0pt}f\ i{\isacharparenright}{\kern0pt}\ x{\isacharparenright}{\kern0pt}{\isasympartial}M{\isacharparenright}{\kern0pt}{\isachardoublequoteclose}\ \isacommand{using}\isamarkupfalse%
\ assms\ \isacommand{by}\isamarkupfalse%
\ {\isacharparenleft}{\kern0pt}auto\ intro{\isacharbang}{\kern0pt}{\isacharcolon}{\kern0pt}\ Bochner{\isacharunderscore}{\kern0pt}Integration{\isachardot}{\kern0pt}integral{\isacharunderscore}{\kern0pt}sum{\isacharbrackleft}{\kern0pt}symmetric{\isacharbrackright}{\kern0pt}\ integrable{\isacharunderscore}{\kern0pt}mult{\isacharunderscore}{\kern0pt}indicator{\isacharparenright}{\kern0pt}\isanewline
\ \ \isacommand{also}\isamarkupfalse%
\ \isacommand{have}\isamarkupfalse%
\ {\isachardoublequoteopen}{\isachardot}{\kern0pt}{\isachardot}{\kern0pt}{\isachardot}{\kern0pt}\ {\isacharequal}{\kern0pt}\ {\isacharparenleft}{\kern0pt}{\isasymintegral}x{\isasymin}A{\isachardot}{\kern0pt}\ {\isacharparenleft}{\kern0pt}{\isasymSum}i{\isasymin}I{\isachardot}{\kern0pt}\ cond{\isacharunderscore}{\kern0pt}exp\ M\ F\ {\isacharparenleft}{\kern0pt}f\ i{\isacharparenright}{\kern0pt}\ x{\isacharparenright}{\kern0pt}{\isasympartial}M{\isacharparenright}{\kern0pt}{\isachardoublequoteclose}\ \isacommand{unfolding}\isamarkupfalse%
\ set{\isacharunderscore}{\kern0pt}lebesgue{\isacharunderscore}{\kern0pt}integral{\isacharunderscore}{\kern0pt}def\ \isacommand{by}\isamarkupfalse%
\ {\isacharparenleft}{\kern0pt}simp\ add{\isacharcolon}{\kern0pt}\ scaleR{\isacharunderscore}{\kern0pt}sum{\isacharunderscore}{\kern0pt}right{\isacharparenright}{\kern0pt}\isanewline
\ \ \isacommand{finally}\isamarkupfalse%
\ \isacommand{show}\isamarkupfalse%
\ {\isachardoublequoteopen}{\isacharparenleft}{\kern0pt}{\isasymintegral}x{\isasymin}A{\isachardot}{\kern0pt}\ {\isacharparenleft}{\kern0pt}{\isasymSum}i{\isasymin}I{\isachardot}{\kern0pt}\ f\ i\ x{\isacharparenright}{\kern0pt}{\isasympartial}M{\isacharparenright}{\kern0pt}\ {\isacharequal}{\kern0pt}\ {\isacharparenleft}{\kern0pt}{\isasymintegral}x{\isasymin}A{\isachardot}{\kern0pt}\ {\isacharparenleft}{\kern0pt}{\isasymSum}i{\isasymin}I{\isachardot}{\kern0pt}\ cond{\isacharunderscore}{\kern0pt}exp\ M\ F\ {\isacharparenleft}{\kern0pt}f\ i{\isacharparenright}{\kern0pt}\ x{\isacharparenright}{\kern0pt}{\isasympartial}M{\isacharparenright}{\kern0pt}{\isachardoublequoteclose}\ \isacommand{by}\isamarkupfalse%
\ auto\isanewline
\isacommand{qed}\isamarkupfalse%
\ {\isacharparenleft}{\kern0pt}auto\ simp\ add{\isacharcolon}{\kern0pt}\ assms\ integrable{\isacharunderscore}{\kern0pt}cond{\isacharunderscore}{\kern0pt}exp{\isacharparenright}{\kern0pt}%
\endisatagproof
{\isafoldproof}%
%
\isadelimproof
%
\endisadelimproof
%
\isadelimdocument
%
\endisadelimdocument
%
\isatagdocument
%
\isamarkupsubsection{Ordered Real Vectors%
}
\isamarkuptrue%
%
\endisatagdocument
{\isafolddocument}%
%
\isadelimdocument
%
\endisadelimdocument
\isacommand{lemma}\isamarkupfalse%
\ cond{\isacharunderscore}{\kern0pt}exp{\isacharunderscore}{\kern0pt}gr{\isacharunderscore}{\kern0pt}c{\isacharcolon}{\kern0pt}\isanewline
\ \ \isakeyword{fixes}\ f\ {\isacharcolon}{\kern0pt}{\isacharcolon}{\kern0pt}\ {\isachardoublequoteopen}{\isacharprime}{\kern0pt}a\ {\isasymRightarrow}\ {\isacharprime}{\kern0pt}b\ {\isacharcolon}{\kern0pt}{\isacharcolon}{\kern0pt}\ {\isacharbraceleft}{\kern0pt}second{\isacharunderscore}{\kern0pt}countable{\isacharunderscore}{\kern0pt}topology{\isacharcomma}{\kern0pt}\ banach{\isacharcomma}{\kern0pt}\ linorder{\isacharunderscore}{\kern0pt}topology{\isacharcomma}{\kern0pt}\ ordered{\isacharunderscore}{\kern0pt}real{\isacharunderscore}{\kern0pt}vector{\isacharbraceright}{\kern0pt}{\isachardoublequoteclose}\isanewline
\ \ \isakeyword{assumes}\ {\isachardoublequoteopen}integrable\ M\ f{\isachardoublequoteclose}\ \ {\isachardoublequoteopen}AE\ x\ in\ M{\isachardot}{\kern0pt}\ f\ x\ {\isachargreater}{\kern0pt}\ c{\isachardoublequoteclose}\isanewline
\ \ \isakeyword{shows}\ {\isachardoublequoteopen}AE\ x\ in\ M{\isachardot}{\kern0pt}\ cond{\isacharunderscore}{\kern0pt}exp\ M\ F\ f\ x\ {\isachargreater}{\kern0pt}\ c{\isachardoublequoteclose}\isanewline
%
\isadelimproof
%
\endisadelimproof
%
\isatagproof
\isacommand{proof}\isamarkupfalse%
\ {\isacharminus}{\kern0pt}\isanewline
\ \ \isacommand{define}\isamarkupfalse%
\ X\ \isakeyword{where}\ {\isachardoublequoteopen}X\ {\isacharequal}{\kern0pt}\ {\isacharbraceleft}{\kern0pt}x\ {\isasymin}\ space\ M{\isachardot}{\kern0pt}\ cond{\isacharunderscore}{\kern0pt}exp\ M\ F\ f\ x\ {\isasymle}\ c{\isacharbraceright}{\kern0pt}{\isachardoublequoteclose}\isanewline
\ \ \isacommand{have}\isamarkupfalse%
\ {\isacharbrackleft}{\kern0pt}measurable{\isacharbrackright}{\kern0pt}{\isacharcolon}{\kern0pt}\ {\isachardoublequoteopen}X\ {\isasymin}\ sets\ F{\isachardoublequoteclose}\ \isacommand{unfolding}\isamarkupfalse%
\ X{\isacharunderscore}{\kern0pt}def\ \isacommand{by}\isamarkupfalse%
\ measurable\ {\isacharparenleft}{\kern0pt}metis\ sets{\isachardot}{\kern0pt}top\ subalg\ subalgebra{\isacharunderscore}{\kern0pt}def{\isacharparenright}{\kern0pt}\isanewline
\ \ \isacommand{hence}\isamarkupfalse%
\ X{\isacharunderscore}{\kern0pt}in{\isacharunderscore}{\kern0pt}M{\isacharcolon}{\kern0pt}\ {\isachardoublequoteopen}X\ {\isasymin}\ sets\ M{\isachardoublequoteclose}\ \isacommand{using}\isamarkupfalse%
\ sets{\isacharunderscore}{\kern0pt}restr{\isacharunderscore}{\kern0pt}to{\isacharunderscore}{\kern0pt}subalg\ subalg\ subalgebra{\isacharunderscore}{\kern0pt}def\ \isacommand{by}\isamarkupfalse%
\ blast\isanewline
\ \ \isacommand{have}\isamarkupfalse%
\ {\isachardoublequoteopen}emeasure\ M\ X\ {\isacharequal}{\kern0pt}\ {\isadigit{0}}{\isachardoublequoteclose}\isanewline
\ \ \isacommand{proof}\isamarkupfalse%
\ {\isacharparenleft}{\kern0pt}rule\ ccontr{\isacharparenright}{\kern0pt}\isanewline
\ \ \ \ \isacommand{assume}\isamarkupfalse%
\ {\isachardoublequoteopen}emeasure\ M\ X\ {\isasymnoteq}\ {\isadigit{0}}{\isachardoublequoteclose}\isanewline
\ \ \ \ \isacommand{have}\isamarkupfalse%
\ {\isachardoublequoteopen}emeasure\ {\isacharparenleft}{\kern0pt}restr{\isacharunderscore}{\kern0pt}to{\isacharunderscore}{\kern0pt}subalg\ M\ F{\isacharparenright}{\kern0pt}\ X\ {\isacharequal}{\kern0pt}\ emeasure\ M\ X{\isachardoublequoteclose}\ \isacommand{by}\isamarkupfalse%
\ {\isacharparenleft}{\kern0pt}simp\ add{\isacharcolon}{\kern0pt}\ emeasure{\isacharunderscore}{\kern0pt}restr{\isacharunderscore}{\kern0pt}to{\isacharunderscore}{\kern0pt}subalg\ subalg{\isacharparenright}{\kern0pt}\isanewline
\ \ \ \ \isacommand{hence}\isamarkupfalse%
\ {\isachardoublequoteopen}emeasure\ {\isacharparenleft}{\kern0pt}restr{\isacharunderscore}{\kern0pt}to{\isacharunderscore}{\kern0pt}subalg\ M\ F{\isacharparenright}{\kern0pt}\ X\ {\isachargreater}{\kern0pt}\ {\isadigit{0}}{\isachardoublequoteclose}\ \isacommand{using}\isamarkupfalse%
\ {\isacartoucheopen}{\isasymnot}{\isacharparenleft}{\kern0pt}emeasure\ M\ X{\isacharparenright}{\kern0pt}\ {\isacharequal}{\kern0pt}\ {\isadigit{0}}{\isacartoucheclose}\ gr{\isacharunderscore}{\kern0pt}zeroI\ \isacommand{by}\isamarkupfalse%
\ auto\isanewline
\ \ \ \ \isacommand{then}\isamarkupfalse%
\ \isacommand{obtain}\isamarkupfalse%
\ A\ \isakeyword{where}\ A{\isacharcolon}{\kern0pt}\ {\isachardoublequoteopen}A\ {\isasymin}\ sets\ {\isacharparenleft}{\kern0pt}restr{\isacharunderscore}{\kern0pt}to{\isacharunderscore}{\kern0pt}subalg\ M\ F{\isacharparenright}{\kern0pt}{\isachardoublequoteclose}\ {\isachardoublequoteopen}A\ {\isasymsubseteq}\ X{\isachardoublequoteclose}\ {\isachardoublequoteopen}emeasure\ {\isacharparenleft}{\kern0pt}restr{\isacharunderscore}{\kern0pt}to{\isacharunderscore}{\kern0pt}subalg\ M\ F{\isacharparenright}{\kern0pt}\ A\ {\isachargreater}{\kern0pt}\ {\isadigit{0}}{\isachardoublequoteclose}\ {\isachardoublequoteopen}emeasure\ {\isacharparenleft}{\kern0pt}restr{\isacharunderscore}{\kern0pt}to{\isacharunderscore}{\kern0pt}subalg\ M\ F{\isacharparenright}{\kern0pt}\ A\ {\isacharless}{\kern0pt}\ {\isasyminfinity}{\isachardoublequoteclose}\isanewline
\ \ \ \ \ \ \isacommand{using}\isamarkupfalse%
\ sigma{\isacharunderscore}{\kern0pt}fin{\isacharunderscore}{\kern0pt}subalg\ \isacommand{by}\isamarkupfalse%
\ {\isacharparenleft}{\kern0pt}metis\ emeasure{\isacharunderscore}{\kern0pt}notin{\isacharunderscore}{\kern0pt}sets\ ennreal{\isacharunderscore}{\kern0pt}{\isadigit{0}}\ infinity{\isacharunderscore}{\kern0pt}ennreal{\isacharunderscore}{\kern0pt}def\ le{\isacharunderscore}{\kern0pt}less{\isacharunderscore}{\kern0pt}linear\ neq{\isacharunderscore}{\kern0pt}top{\isacharunderscore}{\kern0pt}trans\ not{\isacharunderscore}{\kern0pt}gr{\isacharunderscore}{\kern0pt}zero\ order{\isacharunderscore}{\kern0pt}refl\ sigma{\isacharunderscore}{\kern0pt}finite{\isacharunderscore}{\kern0pt}measure{\isachardot}{\kern0pt}approx{\isacharunderscore}{\kern0pt}PInf{\isacharunderscore}{\kern0pt}emeasure{\isacharunderscore}{\kern0pt}with{\isacharunderscore}{\kern0pt}finite{\isacharparenright}{\kern0pt}\isanewline
\ \ \ \ \isacommand{hence}\isamarkupfalse%
\ {\isacharbrackleft}{\kern0pt}simp{\isacharbrackright}{\kern0pt}{\isacharcolon}{\kern0pt}\ {\isachardoublequoteopen}A\ {\isasymin}\ sets\ F{\isachardoublequoteclose}\ \isacommand{using}\isamarkupfalse%
\ subalg\ sets{\isacharunderscore}{\kern0pt}restr{\isacharunderscore}{\kern0pt}to{\isacharunderscore}{\kern0pt}subalg\ \isacommand{by}\isamarkupfalse%
\ blast\isanewline
\ \ \ \ \isacommand{hence}\isamarkupfalse%
\ {\isacharbrackleft}{\kern0pt}simp{\isacharbrackright}{\kern0pt}{\isacharcolon}{\kern0pt}\ {\isachardoublequoteopen}A\ {\isasymin}\ sets\ M{\isachardoublequoteclose}\ \isacommand{using}\isamarkupfalse%
\ sets{\isacharunderscore}{\kern0pt}restr{\isacharunderscore}{\kern0pt}to{\isacharunderscore}{\kern0pt}subalg\ subalg\ subalgebra{\isacharunderscore}{\kern0pt}def\ \isacommand{by}\isamarkupfalse%
\ blast\isanewline
\ \ \ \ \isacommand{have}\isamarkupfalse%
\ {\isacharbrackleft}{\kern0pt}simp{\isacharbrackright}{\kern0pt}{\isacharcolon}{\kern0pt}\ {\isachardoublequoteopen}set{\isacharunderscore}{\kern0pt}integrable\ M\ A\ {\isacharparenleft}{\kern0pt}{\isasymlambda}x{\isachardot}{\kern0pt}\ c{\isacharparenright}{\kern0pt}{\isachardoublequoteclose}\ \isacommand{using}\isamarkupfalse%
\ A\ subalg\ \isacommand{by}\isamarkupfalse%
\ {\isacharparenleft}{\kern0pt}auto\ simp\ add{\isacharcolon}{\kern0pt}\ set{\isacharunderscore}{\kern0pt}integrable{\isacharunderscore}{\kern0pt}def\ emeasure{\isacharunderscore}{\kern0pt}restr{\isacharunderscore}{\kern0pt}to{\isacharunderscore}{\kern0pt}subalg{\isacharparenright}{\kern0pt}\ \isanewline
\ \ \ \ \isacommand{have}\isamarkupfalse%
\ {\isacharbrackleft}{\kern0pt}simp{\isacharbrackright}{\kern0pt}{\isacharcolon}{\kern0pt}\ {\isachardoublequoteopen}set{\isacharunderscore}{\kern0pt}integrable\ M\ A\ f{\isachardoublequoteclose}\ \isacommand{unfolding}\isamarkupfalse%
\ set{\isacharunderscore}{\kern0pt}integrable{\isacharunderscore}{\kern0pt}def\ \isacommand{by}\isamarkupfalse%
\ {\isacharparenleft}{\kern0pt}rule\ integrable{\isacharunderscore}{\kern0pt}mult{\isacharunderscore}{\kern0pt}indicator{\isacharcomma}{\kern0pt}\ auto\ simp\ add{\isacharcolon}{\kern0pt}\ assms{\isacharparenleft}{\kern0pt}{\isadigit{1}}{\isacharparenright}{\kern0pt}{\isacharparenright}{\kern0pt}\isanewline
\ \ \ \ \isacommand{have}\isamarkupfalse%
\ {\isachardoublequoteopen}AE\ x\ in\ M{\isachardot}{\kern0pt}\ indicator\ A\ x\ {\isacharasterisk}{\kern0pt}\isactrlsub R\ c\ {\isacharequal}{\kern0pt}\ indicator\ A\ x\ {\isacharasterisk}{\kern0pt}\isactrlsub R\ f\ x{\isachardoublequoteclose}\isanewline
\ \ \ \ \isacommand{proof}\isamarkupfalse%
\ {\isacharparenleft}{\kern0pt}rule\ integral{\isacharunderscore}{\kern0pt}eq{\isacharunderscore}{\kern0pt}mono{\isacharunderscore}{\kern0pt}AE{\isacharunderscore}{\kern0pt}eq{\isacharunderscore}{\kern0pt}AE{\isacharparenright}{\kern0pt}\isanewline
\ \ \ \ \ \ \isacommand{show}\isamarkupfalse%
\ {\isachardoublequoteopen}LINT\ x{\isacharbar}{\kern0pt}M{\isachardot}{\kern0pt}\ indicator\ A\ x\ {\isacharasterisk}{\kern0pt}\isactrlsub R\ c\ {\isacharequal}{\kern0pt}\ LINT\ x{\isacharbar}{\kern0pt}M{\isachardot}{\kern0pt}\ indicator\ A\ x\ {\isacharasterisk}{\kern0pt}\isactrlsub R\ f\ x{\isachardoublequoteclose}\ \isanewline
\ \ \ \ \ \ \isacommand{proof}\isamarkupfalse%
\ {\isacharparenleft}{\kern0pt}simp\ only{\isacharcolon}{\kern0pt}\ set{\isacharunderscore}{\kern0pt}lebesgue{\isacharunderscore}{\kern0pt}integral{\isacharunderscore}{\kern0pt}def{\isacharbrackleft}{\kern0pt}symmetric{\isacharbrackright}{\kern0pt}{\isacharcomma}{\kern0pt}\ rule\ antisym{\isacharparenright}{\kern0pt}\isanewline
\ \ \ \ \ \ \ \ \isacommand{show}\isamarkupfalse%
\ {\isachardoublequoteopen}{\isacharparenleft}{\kern0pt}{\isasymintegral}x{\isasymin}A{\isachardot}{\kern0pt}\ c\ {\isasympartial}M{\isacharparenright}{\kern0pt}\ {\isasymle}\ {\isacharparenleft}{\kern0pt}{\isasymintegral}x{\isasymin}A{\isachardot}{\kern0pt}\ f\ x\ {\isasympartial}M{\isacharparenright}{\kern0pt}{\isachardoublequoteclose}\ \isacommand{using}\isamarkupfalse%
\ assms{\isacharparenleft}{\kern0pt}{\isadigit{2}}{\isacharparenright}{\kern0pt}\ \isacommand{by}\isamarkupfalse%
\ {\isacharparenleft}{\kern0pt}intro\ set{\isacharunderscore}{\kern0pt}integral{\isacharunderscore}{\kern0pt}mono{\isacharunderscore}{\kern0pt}AE{\isacharunderscore}{\kern0pt}banach{\isacharparenright}{\kern0pt}\ auto\isanewline
\ \ \ \ \ \ \ \ \isacommand{have}\isamarkupfalse%
\ {\isachardoublequoteopen}{\isacharparenleft}{\kern0pt}{\isasymintegral}x{\isasymin}A{\isachardot}{\kern0pt}\ f\ x\ {\isasympartial}M{\isacharparenright}{\kern0pt}\ {\isacharequal}{\kern0pt}\ {\isacharparenleft}{\kern0pt}{\isasymintegral}x{\isasymin}A{\isachardot}{\kern0pt}\ cond{\isacharunderscore}{\kern0pt}exp\ M\ F\ f\ x\ {\isasympartial}M{\isacharparenright}{\kern0pt}{\isachardoublequoteclose}\ \isacommand{by}\isamarkupfalse%
\ {\isacharparenleft}{\kern0pt}rule\ cond{\isacharunderscore}{\kern0pt}exp{\isacharunderscore}{\kern0pt}set{\isacharunderscore}{\kern0pt}integral{\isacharcomma}{\kern0pt}\ auto\ simp\ add{\isacharcolon}{\kern0pt}\ {\isacartoucheopen}integrable\ M\ f{\isacartoucheclose}{\isacharparenright}{\kern0pt}\isanewline
\ \ \ \ \ \ \ \ \isacommand{also}\isamarkupfalse%
\ \isacommand{have}\isamarkupfalse%
\ {\isachardoublequoteopen}{\isachardot}{\kern0pt}{\isachardot}{\kern0pt}{\isachardot}{\kern0pt}\ {\isasymle}\ {\isacharparenleft}{\kern0pt}{\isasymintegral}x{\isasymin}A{\isachardot}{\kern0pt}\ c\ {\isasympartial}M{\isacharparenright}{\kern0pt}{\isachardoublequoteclose}\ \isacommand{using}\isamarkupfalse%
\ A\ \isacommand{by}\isamarkupfalse%
\ {\isacharparenleft}{\kern0pt}auto\ intro{\isacharbang}{\kern0pt}{\isacharcolon}{\kern0pt}\ set{\isacharunderscore}{\kern0pt}integral{\isacharunderscore}{\kern0pt}mono{\isacharunderscore}{\kern0pt}banach\ simp\ add{\isacharcolon}{\kern0pt}\ X{\isacharunderscore}{\kern0pt}def{\isacharparenright}{\kern0pt}\isanewline
\ \ \ \ \ \ \ \ \isacommand{finally}\isamarkupfalse%
\ \isacommand{show}\isamarkupfalse%
\ {\isachardoublequoteopen}{\isacharparenleft}{\kern0pt}{\isasymintegral}x{\isasymin}A{\isachardot}{\kern0pt}\ f\ x\ {\isasympartial}M{\isacharparenright}{\kern0pt}\ {\isasymle}\ {\isacharparenleft}{\kern0pt}{\isasymintegral}x{\isasymin}A{\isachardot}{\kern0pt}\ c\ {\isasympartial}M{\isacharparenright}{\kern0pt}{\isachardoublequoteclose}\ \isacommand{by}\isamarkupfalse%
\ simp\isanewline
\ \ \ \ \ \ \isacommand{qed}\isamarkupfalse%
\isanewline
\ \ \ \ \ \ \isacommand{then}\isamarkupfalse%
\ \isacommand{have}\isamarkupfalse%
\ {\isachardoublequoteopen}measure\ M\ A\ {\isacharasterisk}{\kern0pt}\isactrlsub R\ c\ {\isacharequal}{\kern0pt}\ LINT\ x{\isacharbar}{\kern0pt}M{\isachardot}{\kern0pt}\ indicator\ A\ x\ {\isacharasterisk}{\kern0pt}\isactrlsub R\ f\ x{\isachardoublequoteclose}\ \isacommand{using}\isamarkupfalse%
\ A\ \isacommand{by}\isamarkupfalse%
\ {\isacharparenleft}{\kern0pt}auto\ simp{\isacharcolon}{\kern0pt}\ set{\isacharunderscore}{\kern0pt}lebesgue{\isacharunderscore}{\kern0pt}integral{\isacharunderscore}{\kern0pt}def\ emeasure{\isacharunderscore}{\kern0pt}restr{\isacharunderscore}{\kern0pt}to{\isacharunderscore}{\kern0pt}subalg\ subalg{\isacharparenright}{\kern0pt}\isanewline
\ \ \ \ \ \ \isacommand{show}\isamarkupfalse%
\ {\isachardoublequoteopen}AE\ x\ in\ M{\isachardot}{\kern0pt}\ indicator\ A\ x\ {\isacharasterisk}{\kern0pt}\isactrlsub R\ c\ {\isasymle}\ indicator\ A\ x\ {\isacharasterisk}{\kern0pt}\isactrlsub R\ f\ x{\isachardoublequoteclose}\ \isacommand{using}\isamarkupfalse%
\ assms\ \isacommand{by}\isamarkupfalse%
\ {\isacharparenleft}{\kern0pt}auto\ simp\ add{\isacharcolon}{\kern0pt}\ X{\isacharunderscore}{\kern0pt}def\ indicator{\isacharunderscore}{\kern0pt}def{\isacharparenright}{\kern0pt}\isanewline
\ \ \ \ \isacommand{qed}\isamarkupfalse%
\ {\isacharparenleft}{\kern0pt}auto\ simp\ add{\isacharcolon}{\kern0pt}\ set{\isacharunderscore}{\kern0pt}integrable{\isacharunderscore}{\kern0pt}def{\isacharbrackleft}{\kern0pt}symmetric{\isacharbrackright}{\kern0pt}{\isacharparenright}{\kern0pt}\isanewline
\ \ \ \ \isacommand{then}\isamarkupfalse%
\ \isacommand{have}\isamarkupfalse%
\ {\isachardoublequoteopen}AE\ x{\isasymin}A\ in\ M{\isachardot}{\kern0pt}\ c\ {\isacharequal}{\kern0pt}\ f\ x{\isachardoublequoteclose}\ \isacommand{by}\isamarkupfalse%
\ auto\isanewline
\ \ \ \ \isacommand{then}\isamarkupfalse%
\ \isacommand{have}\isamarkupfalse%
\ {\isachardoublequoteopen}AE\ x{\isasymin}A\ in\ M{\isachardot}{\kern0pt}\ False{\isachardoublequoteclose}\ \isacommand{using}\isamarkupfalse%
\ assms{\isacharparenleft}{\kern0pt}{\isadigit{2}}{\isacharparenright}{\kern0pt}\ \isacommand{by}\isamarkupfalse%
\ auto\isanewline
\ \ \ \ \isacommand{have}\isamarkupfalse%
\ {\isachardoublequoteopen}A\ {\isasymin}\ null{\isacharunderscore}{\kern0pt}sets\ M{\isachardoublequoteclose}\ \isacommand{unfolding}\isamarkupfalse%
\ ae{\isacharunderscore}{\kern0pt}filter{\isacharunderscore}{\kern0pt}def\ \isacommand{by}\isamarkupfalse%
\ {\isacharparenleft}{\kern0pt}meson\ AE{\isacharunderscore}{\kern0pt}iff{\isacharunderscore}{\kern0pt}null{\isacharunderscore}{\kern0pt}sets\ {\isacartoucheopen}A\ {\isasymin}\ sets\ M{\isacartoucheclose}\ {\isacartoucheopen}AE\ x{\isasymin}A\ in\ M{\isachardot}{\kern0pt}\ False{\isacartoucheclose}{\isacharparenright}{\kern0pt}\isanewline
\ \ \ \ \isacommand{then}\isamarkupfalse%
\ \isacommand{show}\isamarkupfalse%
\ False\ \isacommand{using}\isamarkupfalse%
\ A{\isacharparenleft}{\kern0pt}{\isadigit{3}}{\isacharparenright}{\kern0pt}\isacommand{by}\isamarkupfalse%
\ {\isacharparenleft}{\kern0pt}simp\ add{\isacharcolon}{\kern0pt}\ emeasure{\isacharunderscore}{\kern0pt}restr{\isacharunderscore}{\kern0pt}to{\isacharunderscore}{\kern0pt}subalg\ null{\isacharunderscore}{\kern0pt}setsD{\isadigit{1}}\ subalg{\isacharparenright}{\kern0pt}\isanewline
\ \ \isacommand{qed}\isamarkupfalse%
\isanewline
\ \ \isacommand{then}\isamarkupfalse%
\ \isacommand{show}\isamarkupfalse%
\ {\isacharquery}{\kern0pt}thesis\ \isacommand{using}\isamarkupfalse%
\ AE{\isacharunderscore}{\kern0pt}iff{\isacharunderscore}{\kern0pt}null{\isacharunderscore}{\kern0pt}sets{\isacharbrackleft}{\kern0pt}OF\ X{\isacharunderscore}{\kern0pt}in{\isacharunderscore}{\kern0pt}M{\isacharbrackright}{\kern0pt}\ \isacommand{unfolding}\isamarkupfalse%
\ X{\isacharunderscore}{\kern0pt}def\ \isacommand{by}\isamarkupfalse%
\ auto\isanewline
\isacommand{qed}\isamarkupfalse%
%
\endisatagproof
{\isafoldproof}%
%
\isadelimproof
\isanewline
%
\endisadelimproof
\isanewline
\isacommand{lemma}\isamarkupfalse%
\ cond{\isacharunderscore}{\kern0pt}exp{\isacharunderscore}{\kern0pt}less{\isacharunderscore}{\kern0pt}c{\isacharcolon}{\kern0pt}\isanewline
\ \ \isakeyword{fixes}\ f\ {\isacharcolon}{\kern0pt}{\isacharcolon}{\kern0pt}\ {\isachardoublequoteopen}{\isacharprime}{\kern0pt}a\ {\isasymRightarrow}\ {\isacharprime}{\kern0pt}b\ {\isacharcolon}{\kern0pt}{\isacharcolon}{\kern0pt}\ {\isacharbraceleft}{\kern0pt}second{\isacharunderscore}{\kern0pt}countable{\isacharunderscore}{\kern0pt}topology{\isacharcomma}{\kern0pt}\ banach{\isacharcomma}{\kern0pt}\ linorder{\isacharunderscore}{\kern0pt}topology{\isacharcomma}{\kern0pt}\ ordered{\isacharunderscore}{\kern0pt}real{\isacharunderscore}{\kern0pt}vector{\isacharbraceright}{\kern0pt}{\isachardoublequoteclose}\isanewline
\ \ \isakeyword{assumes}\ {\isachardoublequoteopen}integrable\ M\ f{\isachardoublequoteclose}\ {\isachardoublequoteopen}AE\ x\ in\ M{\isachardot}{\kern0pt}\ f\ x\ {\isacharless}{\kern0pt}\ c{\isachardoublequoteclose}\isanewline
\ \ \isakeyword{shows}\ {\isachardoublequoteopen}AE\ x\ in\ M{\isachardot}{\kern0pt}\ cond{\isacharunderscore}{\kern0pt}exp\ M\ F\ f\ x\ {\isacharless}{\kern0pt}\ c{\isachardoublequoteclose}\isanewline
%
\isadelimproof
%
\endisadelimproof
%
\isatagproof
\isacommand{proof}\isamarkupfalse%
\ {\isacharminus}{\kern0pt}\isanewline
\ \ \isacommand{have}\isamarkupfalse%
\ {\isachardoublequoteopen}AE\ x\ in\ M{\isachardot}{\kern0pt}\ cond{\isacharunderscore}{\kern0pt}exp\ M\ F\ f\ x\ {\isacharequal}{\kern0pt}\ {\isacharminus}{\kern0pt}\ cond{\isacharunderscore}{\kern0pt}exp\ M\ F\ {\isacharparenleft}{\kern0pt}{\isasymlambda}x{\isachardot}{\kern0pt}\ {\isacharminus}{\kern0pt}\ f\ x{\isacharparenright}{\kern0pt}\ x{\isachardoublequoteclose}\ \isacommand{using}\isamarkupfalse%
\ cond{\isacharunderscore}{\kern0pt}exp{\isacharunderscore}{\kern0pt}uminus{\isacharbrackleft}{\kern0pt}OF\ assms{\isacharparenleft}{\kern0pt}{\isadigit{1}}{\isacharparenright}{\kern0pt}{\isacharbrackright}{\kern0pt}\ \isacommand{by}\isamarkupfalse%
\ auto\isanewline
\ \ \isacommand{moreover}\isamarkupfalse%
\ \isacommand{have}\isamarkupfalse%
\ {\isachardoublequoteopen}AE\ x\ in\ M{\isachardot}{\kern0pt}\ cond{\isacharunderscore}{\kern0pt}exp\ M\ F\ {\isacharparenleft}{\kern0pt}{\isasymlambda}x{\isachardot}{\kern0pt}\ {\isacharminus}{\kern0pt}f\ x{\isacharparenright}{\kern0pt}\ x\ {\isachargreater}{\kern0pt}\ {\isacharminus}{\kern0pt}c{\isachardoublequoteclose}\ \ \isacommand{using}\isamarkupfalse%
\ assms\ \isacommand{by}\isamarkupfalse%
\ {\isacharparenleft}{\kern0pt}intro\ cond{\isacharunderscore}{\kern0pt}exp{\isacharunderscore}{\kern0pt}gr{\isacharunderscore}{\kern0pt}c{\isacharparenright}{\kern0pt}\ auto\isanewline
\ \ \isacommand{ultimately}\isamarkupfalse%
\ \isacommand{show}\isamarkupfalse%
\ {\isacharquery}{\kern0pt}thesis\ \isacommand{by}\isamarkupfalse%
\ {\isacharparenleft}{\kern0pt}force\ simp\ add{\isacharcolon}{\kern0pt}\ minus{\isacharunderscore}{\kern0pt}less{\isacharunderscore}{\kern0pt}iff{\isacharparenright}{\kern0pt}\isanewline
\isacommand{qed}\isamarkupfalse%
%
\endisatagproof
{\isafoldproof}%
%
\isadelimproof
\isanewline
%
\endisadelimproof
\isanewline
\isacommand{lemma}\isamarkupfalse%
\ cond{\isacharunderscore}{\kern0pt}exp{\isacharunderscore}{\kern0pt}mono{\isacharunderscore}{\kern0pt}strict{\isacharcolon}{\kern0pt}\isanewline
\ \ \isakeyword{fixes}\ f\ {\isacharcolon}{\kern0pt}{\isacharcolon}{\kern0pt}\ {\isachardoublequoteopen}{\isacharprime}{\kern0pt}a\ {\isasymRightarrow}\ {\isacharprime}{\kern0pt}b\ {\isacharcolon}{\kern0pt}{\isacharcolon}{\kern0pt}\ {\isacharbraceleft}{\kern0pt}second{\isacharunderscore}{\kern0pt}countable{\isacharunderscore}{\kern0pt}topology{\isacharcomma}{\kern0pt}\ banach{\isacharcomma}{\kern0pt}\ linorder{\isacharunderscore}{\kern0pt}topology{\isacharcomma}{\kern0pt}\ ordered{\isacharunderscore}{\kern0pt}real{\isacharunderscore}{\kern0pt}vector{\isacharbraceright}{\kern0pt}{\isachardoublequoteclose}\isanewline
\ \ \isakeyword{assumes}\ {\isachardoublequoteopen}integrable\ M\ f{\isachardoublequoteclose}\ {\isachardoublequoteopen}integrable\ M\ g{\isachardoublequoteclose}\ {\isachardoublequoteopen}AE\ x\ in\ M{\isachardot}{\kern0pt}\ f\ x\ {\isacharless}{\kern0pt}\ g\ x{\isachardoublequoteclose}\isanewline
\ \ \isakeyword{shows}\ {\isachardoublequoteopen}AE\ x\ in\ M{\isachardot}{\kern0pt}\ cond{\isacharunderscore}{\kern0pt}exp\ M\ F\ f\ x\ {\isacharless}{\kern0pt}\ cond{\isacharunderscore}{\kern0pt}exp\ M\ F\ g\ x{\isachardoublequoteclose}\isanewline
%
\isadelimproof
\ \ %
\endisadelimproof
%
\isatagproof
\isacommand{using}\isamarkupfalse%
\ cond{\isacharunderscore}{\kern0pt}exp{\isacharunderscore}{\kern0pt}less{\isacharunderscore}{\kern0pt}c{\isacharbrackleft}{\kern0pt}OF\ Bochner{\isacharunderscore}{\kern0pt}Integration{\isachardot}{\kern0pt}integrable{\isacharunderscore}{\kern0pt}diff{\isacharcomma}{\kern0pt}\ OF\ assms{\isacharparenleft}{\kern0pt}{\isadigit{1}}{\isacharcomma}{\kern0pt}{\isadigit{2}}{\isacharparenright}{\kern0pt}{\isacharcomma}{\kern0pt}\ of\ {\isadigit{0}}{\isacharbrackright}{\kern0pt}\ \isanewline
\ \ \ \ \ \ \ \ cond{\isacharunderscore}{\kern0pt}exp{\isacharunderscore}{\kern0pt}diff{\isacharbrackleft}{\kern0pt}OF\ assms{\isacharparenleft}{\kern0pt}{\isadigit{1}}{\isacharcomma}{\kern0pt}{\isadigit{2}}{\isacharparenright}{\kern0pt}{\isacharbrackright}{\kern0pt}\ assms{\isacharparenleft}{\kern0pt}{\isadigit{3}}{\isacharparenright}{\kern0pt}\ \isacommand{by}\isamarkupfalse%
\ auto%
\endisatagproof
{\isafoldproof}%
%
\isadelimproof
\isanewline
%
\endisadelimproof
\isanewline
\isacommand{lemma}\isamarkupfalse%
\ cond{\isacharunderscore}{\kern0pt}exp{\isacharunderscore}{\kern0pt}ge{\isacharunderscore}{\kern0pt}c{\isacharcolon}{\kern0pt}\isanewline
\ \ \isakeyword{fixes}\ f\ {\isacharcolon}{\kern0pt}{\isacharcolon}{\kern0pt}\ {\isachardoublequoteopen}{\isacharprime}{\kern0pt}a\ {\isasymRightarrow}\ {\isacharprime}{\kern0pt}b\ {\isacharcolon}{\kern0pt}{\isacharcolon}{\kern0pt}\ {\isacharbraceleft}{\kern0pt}second{\isacharunderscore}{\kern0pt}countable{\isacharunderscore}{\kern0pt}topology{\isacharcomma}{\kern0pt}\ banach{\isacharcomma}{\kern0pt}\ linorder{\isacharunderscore}{\kern0pt}topology{\isacharcomma}{\kern0pt}\ ordered{\isacharunderscore}{\kern0pt}real{\isacharunderscore}{\kern0pt}vector{\isacharbraceright}{\kern0pt}{\isachardoublequoteclose}\isanewline
\ \ \isakeyword{assumes}\ {\isacharbrackleft}{\kern0pt}measurable{\isacharbrackright}{\kern0pt}{\isacharcolon}{\kern0pt}\ {\isachardoublequoteopen}integrable\ M\ f{\isachardoublequoteclose}\ \ \ \ \ \ \ \ \ \ \ \ \ \ \ \ \ \ \ \ \ \ \ \ \ \ \ \ \ \ \ \ \ \ \ \ \ \ \ \ \ \ \ \ \ \ \ \ \ \ \ \ \ \ \ \ \ \ \ \ \ \ \ \isanewline
\ \ \ \ \ \ \isakeyword{and}\ {\isachardoublequoteopen}AE\ x\ in\ M{\isachardot}{\kern0pt}\ f\ x\ {\isasymge}\ c{\isachardoublequoteclose}\isanewline
\ \ \isakeyword{shows}\ {\isachardoublequoteopen}AE\ x\ in\ M{\isachardot}{\kern0pt}\ cond{\isacharunderscore}{\kern0pt}exp\ M\ F\ f\ x\ {\isasymge}\ c{\isachardoublequoteclose}\isanewline
%
\isadelimproof
%
\endisadelimproof
%
\isatagproof
\isacommand{proof}\isamarkupfalse%
\ {\isacharminus}{\kern0pt}\isanewline
\ \ \isacommand{let}\isamarkupfalse%
\ {\isacharquery}{\kern0pt}F\ {\isacharequal}{\kern0pt}\ {\isachardoublequoteopen}restr{\isacharunderscore}{\kern0pt}to{\isacharunderscore}{\kern0pt}subalg\ M\ F{\isachardoublequoteclose}\isanewline
\ \ \isacommand{interpret}\isamarkupfalse%
\ sigma{\isacharunderscore}{\kern0pt}finite{\isacharunderscore}{\kern0pt}measure\ {\isachardoublequoteopen}restr{\isacharunderscore}{\kern0pt}to{\isacharunderscore}{\kern0pt}subalg\ M\ F{\isachardoublequoteclose}\ \isacommand{using}\isamarkupfalse%
\ sigma{\isacharunderscore}{\kern0pt}fin{\isacharunderscore}{\kern0pt}subalg\ \isacommand{by}\isamarkupfalse%
\ auto\isanewline
\ \ \isacommand{{\isacharbraceleft}{\kern0pt}}\isamarkupfalse%
\ \isanewline
\ \ \ \ \isacommand{fix}\isamarkupfalse%
\ A\ \isacommand{assume}\isamarkupfalse%
\ asm{\isacharcolon}{\kern0pt}\ {\isachardoublequoteopen}A\ {\isasymin}\ sets\ {\isacharquery}{\kern0pt}F{\isachardoublequoteclose}\ {\isachardoublequoteopen}{\isadigit{0}}\ {\isacharless}{\kern0pt}\ measure\ {\isacharquery}{\kern0pt}F\ A{\isachardoublequoteclose}\isanewline
\ \ \ \ \isacommand{have}\isamarkupfalse%
\ {\isacharbrackleft}{\kern0pt}simp{\isacharbrackright}{\kern0pt}{\isacharcolon}{\kern0pt}\ {\isachardoublequoteopen}sets\ {\isacharquery}{\kern0pt}F\ {\isacharequal}{\kern0pt}\ sets\ F{\isachardoublequoteclose}\ {\isachardoublequoteopen}measure\ {\isacharquery}{\kern0pt}F\ A\ {\isacharequal}{\kern0pt}\ measure\ M\ A{\isachardoublequoteclose}\ \isacommand{using}\isamarkupfalse%
\ asm\ \isacommand{by}\isamarkupfalse%
\ {\isacharparenleft}{\kern0pt}auto\ simp\ add{\isacharcolon}{\kern0pt}\ measure{\isacharunderscore}{\kern0pt}def\ sets{\isacharunderscore}{\kern0pt}restr{\isacharunderscore}{\kern0pt}to{\isacharunderscore}{\kern0pt}subalg{\isacharbrackleft}{\kern0pt}OF\ subalg{\isacharbrackright}{\kern0pt}\ emeasure{\isacharunderscore}{\kern0pt}restr{\isacharunderscore}{\kern0pt}to{\isacharunderscore}{\kern0pt}subalg{\isacharbrackleft}{\kern0pt}OF\ subalg{\isacharbrackright}{\kern0pt}{\isacharparenright}{\kern0pt}\isanewline
\ \ \ \ \isacommand{have}\isamarkupfalse%
\ M{\isacharunderscore}{\kern0pt}A{\isacharcolon}{\kern0pt}\ {\isachardoublequoteopen}emeasure\ M\ A\ {\isacharless}{\kern0pt}\ {\isasyminfinity}{\isachardoublequoteclose}\ \isacommand{using}\isamarkupfalse%
\ measure{\isacharunderscore}{\kern0pt}zero{\isacharunderscore}{\kern0pt}top\ asm\ \isacommand{by}\isamarkupfalse%
\ {\isacharparenleft}{\kern0pt}force\ simp\ add{\isacharcolon}{\kern0pt}\ top{\isachardot}{\kern0pt}not{\isacharunderscore}{\kern0pt}eq{\isacharunderscore}{\kern0pt}extremum{\isacharparenright}{\kern0pt}\isanewline
\ \ \ \ \isacommand{hence}\isamarkupfalse%
\ F{\isacharunderscore}{\kern0pt}A{\isacharcolon}{\kern0pt}\ {\isachardoublequoteopen}emeasure\ {\isacharquery}{\kern0pt}F\ A\ {\isacharless}{\kern0pt}\ {\isasyminfinity}{\isachardoublequoteclose}\ \isacommand{using}\isamarkupfalse%
\ asm{\isacharparenleft}{\kern0pt}{\isadigit{1}}{\isacharparenright}{\kern0pt}\ emeasure{\isacharunderscore}{\kern0pt}restr{\isacharunderscore}{\kern0pt}to{\isacharunderscore}{\kern0pt}subalg\ subalg\ \isacommand{by}\isamarkupfalse%
\ fastforce\isanewline
\ \ \ \ \isacommand{have}\isamarkupfalse%
\ {\isachardoublequoteopen}set{\isacharunderscore}{\kern0pt}lebesgue{\isacharunderscore}{\kern0pt}integral\ M\ A\ {\isacharparenleft}{\kern0pt}{\isasymlambda}{\isacharunderscore}{\kern0pt}{\isachardot}{\kern0pt}\ c{\isacharparenright}{\kern0pt}\ {\isasymle}\ set{\isacharunderscore}{\kern0pt}lebesgue{\isacharunderscore}{\kern0pt}integral\ M\ A\ f{\isachardoublequoteclose}\ \isacommand{using}\isamarkupfalse%
\ assms\ asm\ M{\isacharunderscore}{\kern0pt}A\ subalg\ \isacommand{by}\isamarkupfalse%
\ {\isacharparenleft}{\kern0pt}intro\ set{\isacharunderscore}{\kern0pt}integral{\isacharunderscore}{\kern0pt}mono{\isacharunderscore}{\kern0pt}AE{\isacharunderscore}{\kern0pt}banach{\isacharcomma}{\kern0pt}\ auto\ simp\ add{\isacharcolon}{\kern0pt}\ set{\isacharunderscore}{\kern0pt}integrable{\isacharunderscore}{\kern0pt}def\ integrable{\isacharunderscore}{\kern0pt}mult{\isacharunderscore}{\kern0pt}indicator\ subalgebra{\isacharunderscore}{\kern0pt}def\ sets{\isacharunderscore}{\kern0pt}restr{\isacharunderscore}{\kern0pt}to{\isacharunderscore}{\kern0pt}subalg{\isacharparenright}{\kern0pt}\isanewline
\ \ \ \ \isacommand{also}\isamarkupfalse%
\ \isacommand{have}\isamarkupfalse%
\ {\isachardoublequoteopen}{\isachardot}{\kern0pt}{\isachardot}{\kern0pt}{\isachardot}{\kern0pt}\ {\isacharequal}{\kern0pt}\ set{\isacharunderscore}{\kern0pt}lebesgue{\isacharunderscore}{\kern0pt}integral\ M\ A\ {\isacharparenleft}{\kern0pt}cond{\isacharunderscore}{\kern0pt}exp\ M\ F\ f{\isacharparenright}{\kern0pt}{\isachardoublequoteclose}\ \isacommand{using}\isamarkupfalse%
\ cond{\isacharunderscore}{\kern0pt}exp{\isacharunderscore}{\kern0pt}set{\isacharunderscore}{\kern0pt}integral{\isacharbrackleft}{\kern0pt}OF\ assms{\isacharparenleft}{\kern0pt}{\isadigit{1}}{\isacharparenright}{\kern0pt}{\isacharbrackright}{\kern0pt}\ asm\ \isacommand{by}\isamarkupfalse%
\ auto\isanewline
\ \ \ \ \isacommand{also}\isamarkupfalse%
\ \isacommand{have}\isamarkupfalse%
\ {\isachardoublequoteopen}{\isachardot}{\kern0pt}{\isachardot}{\kern0pt}{\isachardot}{\kern0pt}\ {\isacharequal}{\kern0pt}\ set{\isacharunderscore}{\kern0pt}lebesgue{\isacharunderscore}{\kern0pt}integral\ {\isacharquery}{\kern0pt}F\ A\ {\isacharparenleft}{\kern0pt}cond{\isacharunderscore}{\kern0pt}exp\ M\ F\ f{\isacharparenright}{\kern0pt}{\isachardoublequoteclose}\ \isacommand{unfolding}\isamarkupfalse%
\ set{\isacharunderscore}{\kern0pt}lebesgue{\isacharunderscore}{\kern0pt}integral{\isacharunderscore}{\kern0pt}def\ \isacommand{using}\isamarkupfalse%
\ asm\ borel{\isacharunderscore}{\kern0pt}measurable{\isacharunderscore}{\kern0pt}cond{\isacharunderscore}{\kern0pt}exp\ \isacommand{by}\isamarkupfalse%
\ {\isacharparenleft}{\kern0pt}intro\ integral{\isacharunderscore}{\kern0pt}subalgebra{\isadigit{2}}{\isacharbrackleft}{\kern0pt}OF\ subalg{\isacharcomma}{\kern0pt}\ symmetric{\isacharbrackright}{\kern0pt}{\isacharcomma}{\kern0pt}\ simp{\isacharparenright}{\kern0pt}\isanewline
\ \ \ \ \isacommand{finally}\isamarkupfalse%
\ \isacommand{have}\isamarkupfalse%
\ {\isachardoublequoteopen}{\isacharparenleft}{\kern0pt}{\isadigit{1}}\ {\isacharslash}{\kern0pt}\ measure\ {\isacharquery}{\kern0pt}F\ A{\isacharparenright}{\kern0pt}\ {\isacharasterisk}{\kern0pt}\isactrlsub R\ set{\isacharunderscore}{\kern0pt}lebesgue{\isacharunderscore}{\kern0pt}integral\ {\isacharquery}{\kern0pt}F\ A\ {\isacharparenleft}{\kern0pt}cond{\isacharunderscore}{\kern0pt}exp\ M\ F\ f{\isacharparenright}{\kern0pt}\ {\isasymin}\ {\isacharbraceleft}{\kern0pt}c{\isachardot}{\kern0pt}{\isachardot}{\kern0pt}{\isacharbraceright}{\kern0pt}{\isachardoublequoteclose}\ \isacommand{using}\isamarkupfalse%
\ asm\ subalg\ M{\isacharunderscore}{\kern0pt}A\ \isacommand{by}\isamarkupfalse%
\ {\isacharparenleft}{\kern0pt}auto\ simp\ add{\isacharcolon}{\kern0pt}\ set{\isacharunderscore}{\kern0pt}integral{\isacharunderscore}{\kern0pt}const\ subalgebra{\isacharunderscore}{\kern0pt}def\ intro{\isacharbang}{\kern0pt}{\isacharcolon}{\kern0pt}\ pos{\isacharunderscore}{\kern0pt}divideR{\isacharunderscore}{\kern0pt}le{\isacharunderscore}{\kern0pt}eq{\isacharbrackleft}{\kern0pt}THEN\ iffD{\isadigit{1}}{\isacharbrackright}{\kern0pt}{\isacharparenright}{\kern0pt}\ \isanewline
\ \ \isacommand{{\isacharbraceright}{\kern0pt}}\isamarkupfalse%
\isanewline
\ \ \isacommand{thus}\isamarkupfalse%
\ {\isacharquery}{\kern0pt}thesis\ \isacommand{using}\isamarkupfalse%
\ AE{\isacharunderscore}{\kern0pt}restr{\isacharunderscore}{\kern0pt}to{\isacharunderscore}{\kern0pt}subalg{\isacharbrackleft}{\kern0pt}OF\ subalg{\isacharbrackright}{\kern0pt}\ averaging{\isacharunderscore}{\kern0pt}theorem{\isacharbrackleft}{\kern0pt}OF\ integrable{\isacharunderscore}{\kern0pt}in{\isacharunderscore}{\kern0pt}subalg\ closed{\isacharunderscore}{\kern0pt}atLeast{\isacharcomma}{\kern0pt}\ OF\ subalg\ borel{\isacharunderscore}{\kern0pt}measurable{\isacharunderscore}{\kern0pt}cond{\isacharunderscore}{\kern0pt}exp\ integrable{\isacharunderscore}{\kern0pt}cond{\isacharunderscore}{\kern0pt}exp{\isacharbrackright}{\kern0pt}\ \isacommand{by}\isamarkupfalse%
\ auto\isanewline
\isacommand{qed}\isamarkupfalse%
%
\endisatagproof
{\isafoldproof}%
%
\isadelimproof
\isanewline
%
\endisadelimproof
\isanewline
\isacommand{lemma}\isamarkupfalse%
\ cond{\isacharunderscore}{\kern0pt}exp{\isacharunderscore}{\kern0pt}le{\isacharunderscore}{\kern0pt}c{\isacharcolon}{\kern0pt}\isanewline
\ \ \isakeyword{fixes}\ f\ {\isacharcolon}{\kern0pt}{\isacharcolon}{\kern0pt}\ {\isachardoublequoteopen}{\isacharprime}{\kern0pt}a\ {\isasymRightarrow}\ {\isacharprime}{\kern0pt}b\ {\isacharcolon}{\kern0pt}{\isacharcolon}{\kern0pt}\ {\isacharbraceleft}{\kern0pt}second{\isacharunderscore}{\kern0pt}countable{\isacharunderscore}{\kern0pt}topology{\isacharcomma}{\kern0pt}\ banach{\isacharcomma}{\kern0pt}\ linorder{\isacharunderscore}{\kern0pt}topology{\isacharcomma}{\kern0pt}\ ordered{\isacharunderscore}{\kern0pt}real{\isacharunderscore}{\kern0pt}vector{\isacharbraceright}{\kern0pt}{\isachardoublequoteclose}\isanewline
\ \ \isakeyword{assumes}\ {\isachardoublequoteopen}integrable\ M\ f{\isachardoublequoteclose}\isanewline
\ \ \ \ \ \ \isakeyword{and}\ {\isachardoublequoteopen}AE\ x\ in\ M{\isachardot}{\kern0pt}\ f\ x\ {\isasymle}\ c{\isachardoublequoteclose}\isanewline
\ \ \isakeyword{shows}\ {\isachardoublequoteopen}AE\ x\ in\ M{\isachardot}{\kern0pt}\ cond{\isacharunderscore}{\kern0pt}exp\ M\ F\ f\ x\ {\isasymle}\ c{\isachardoublequoteclose}\isanewline
%
\isadelimproof
%
\endisadelimproof
%
\isatagproof
\isacommand{proof}\isamarkupfalse%
\ {\isacharminus}{\kern0pt}\isanewline
\ \ \isacommand{have}\isamarkupfalse%
\ {\isachardoublequoteopen}AE\ x\ in\ M{\isachardot}{\kern0pt}\ cond{\isacharunderscore}{\kern0pt}exp\ M\ F\ f\ x\ {\isacharequal}{\kern0pt}\ {\isacharminus}{\kern0pt}\ cond{\isacharunderscore}{\kern0pt}exp\ M\ F\ {\isacharparenleft}{\kern0pt}{\isasymlambda}x{\isachardot}{\kern0pt}\ {\isacharminus}{\kern0pt}\ f\ x{\isacharparenright}{\kern0pt}\ x{\isachardoublequoteclose}\ \isacommand{using}\isamarkupfalse%
\ cond{\isacharunderscore}{\kern0pt}exp{\isacharunderscore}{\kern0pt}uminus{\isacharbrackleft}{\kern0pt}OF\ assms{\isacharparenleft}{\kern0pt}{\isadigit{1}}{\isacharparenright}{\kern0pt}{\isacharbrackright}{\kern0pt}\ \isacommand{by}\isamarkupfalse%
\ force\isanewline
\ \ \isacommand{moreover}\isamarkupfalse%
\ \isacommand{have}\isamarkupfalse%
\ {\isachardoublequoteopen}AE\ x\ in\ M{\isachardot}{\kern0pt}\ cond{\isacharunderscore}{\kern0pt}exp\ M\ F\ {\isacharparenleft}{\kern0pt}{\isasymlambda}x{\isachardot}{\kern0pt}\ {\isacharminus}{\kern0pt}\ f\ x{\isacharparenright}{\kern0pt}\ x\ {\isasymge}\ {\isacharminus}{\kern0pt}\ c{\isachardoublequoteclose}\ \isacommand{using}\isamarkupfalse%
\ assms\ \isacommand{by}\isamarkupfalse%
\ {\isacharparenleft}{\kern0pt}intro\ cond{\isacharunderscore}{\kern0pt}exp{\isacharunderscore}{\kern0pt}ge{\isacharunderscore}{\kern0pt}c{\isacharparenright}{\kern0pt}\ auto\isanewline
\ \ \isacommand{ultimately}\isamarkupfalse%
\ \isacommand{show}\isamarkupfalse%
\ {\isacharquery}{\kern0pt}thesis\ \isacommand{by}\isamarkupfalse%
\ {\isacharparenleft}{\kern0pt}force\ simp\ add{\isacharcolon}{\kern0pt}\ minus{\isacharunderscore}{\kern0pt}le{\isacharunderscore}{\kern0pt}iff{\isacharparenright}{\kern0pt}\isanewline
\isacommand{qed}\isamarkupfalse%
%
\endisatagproof
{\isafoldproof}%
%
\isadelimproof
\isanewline
%
\endisadelimproof
\isanewline
\isacommand{lemma}\isamarkupfalse%
\ cond{\isacharunderscore}{\kern0pt}exp{\isacharunderscore}{\kern0pt}mono{\isacharcolon}{\kern0pt}\isanewline
\ \ \isakeyword{fixes}\ f\ {\isacharcolon}{\kern0pt}{\isacharcolon}{\kern0pt}\ {\isachardoublequoteopen}{\isacharprime}{\kern0pt}a\ {\isasymRightarrow}\ {\isacharprime}{\kern0pt}b\ {\isacharcolon}{\kern0pt}{\isacharcolon}{\kern0pt}\ {\isacharbraceleft}{\kern0pt}second{\isacharunderscore}{\kern0pt}countable{\isacharunderscore}{\kern0pt}topology{\isacharcomma}{\kern0pt}\ banach{\isacharcomma}{\kern0pt}\ linorder{\isacharunderscore}{\kern0pt}topology{\isacharcomma}{\kern0pt}\ ordered{\isacharunderscore}{\kern0pt}real{\isacharunderscore}{\kern0pt}vector{\isacharbraceright}{\kern0pt}{\isachardoublequoteclose}\isanewline
\ \ \isakeyword{assumes}\ {\isachardoublequoteopen}integrable\ M\ f{\isachardoublequoteclose}\ {\isachardoublequoteopen}integrable\ M\ g{\isachardoublequoteclose}\ {\isachardoublequoteopen}AE\ x\ in\ M{\isachardot}{\kern0pt}\ f\ x\ {\isasymle}\ g\ x{\isachardoublequoteclose}\isanewline
\ \ \isakeyword{shows}\ {\isachardoublequoteopen}AE\ x\ in\ M{\isachardot}{\kern0pt}\ cond{\isacharunderscore}{\kern0pt}exp\ M\ F\ f\ x\ {\isasymle}\ cond{\isacharunderscore}{\kern0pt}exp\ M\ F\ g\ x{\isachardoublequoteclose}\isanewline
%
\isadelimproof
\ \ %
\endisadelimproof
%
\isatagproof
\isacommand{using}\isamarkupfalse%
\ cond{\isacharunderscore}{\kern0pt}exp{\isacharunderscore}{\kern0pt}le{\isacharunderscore}{\kern0pt}c{\isacharbrackleft}{\kern0pt}OF\ Bochner{\isacharunderscore}{\kern0pt}Integration{\isachardot}{\kern0pt}integrable{\isacharunderscore}{\kern0pt}diff{\isacharcomma}{\kern0pt}\ OF\ assms{\isacharparenleft}{\kern0pt}{\isadigit{1}}{\isacharcomma}{\kern0pt}{\isadigit{2}}{\isacharparenright}{\kern0pt}{\isacharcomma}{\kern0pt}\ of\ {\isadigit{0}}{\isacharbrackright}{\kern0pt}\ \isanewline
\ \ \ \ \ \ \ \ cond{\isacharunderscore}{\kern0pt}exp{\isacharunderscore}{\kern0pt}diff{\isacharbrackleft}{\kern0pt}OF\ assms{\isacharparenleft}{\kern0pt}{\isadigit{1}}{\isacharcomma}{\kern0pt}{\isadigit{2}}{\isacharparenright}{\kern0pt}{\isacharbrackright}{\kern0pt}\ assms{\isacharparenleft}{\kern0pt}{\isadigit{3}}{\isacharparenright}{\kern0pt}\ \isacommand{by}\isamarkupfalse%
\ auto%
\endisatagproof
{\isafoldproof}%
%
\isadelimproof
\isanewline
%
\endisadelimproof
\ \ \ \ \ \ \ \ \ \ \ \ \ \ \ \ \ \ \ \ \ \ \ \ \ \ \ \ \ \ \ \ \ \ \ \ \ \ \isanewline
\isacommand{lemma}\isamarkupfalse%
\ cond{\isacharunderscore}{\kern0pt}exp{\isacharunderscore}{\kern0pt}min{\isacharcolon}{\kern0pt}\isanewline
\ \ \isakeyword{fixes}\ f\ {\isacharcolon}{\kern0pt}{\isacharcolon}{\kern0pt}\ {\isachardoublequoteopen}{\isacharprime}{\kern0pt}a\ {\isasymRightarrow}\ {\isacharprime}{\kern0pt}b\ {\isacharcolon}{\kern0pt}{\isacharcolon}{\kern0pt}\ {\isacharbraceleft}{\kern0pt}second{\isacharunderscore}{\kern0pt}countable{\isacharunderscore}{\kern0pt}topology{\isacharcomma}{\kern0pt}\ banach{\isacharcomma}{\kern0pt}\ linorder{\isacharunderscore}{\kern0pt}topology{\isacharcomma}{\kern0pt}\ ordered{\isacharunderscore}{\kern0pt}real{\isacharunderscore}{\kern0pt}vector{\isacharbraceright}{\kern0pt}{\isachardoublequoteclose}\isanewline
\ \ \isakeyword{assumes}\ {\isachardoublequoteopen}integrable\ M\ f{\isachardoublequoteclose}\ {\isachardoublequoteopen}integrable\ M\ g{\isachardoublequoteclose}\isanewline
\ \ \isakeyword{shows}\ {\isachardoublequoteopen}AE\ {\isasymxi}\ in\ M{\isachardot}{\kern0pt}\ cond{\isacharunderscore}{\kern0pt}exp\ M\ F\ {\isacharparenleft}{\kern0pt}{\isasymlambda}x{\isachardot}{\kern0pt}\ min\ {\isacharparenleft}{\kern0pt}f\ x{\isacharparenright}{\kern0pt}\ {\isacharparenleft}{\kern0pt}g\ x{\isacharparenright}{\kern0pt}{\isacharparenright}{\kern0pt}\ {\isasymxi}\ {\isasymle}\ min\ {\isacharparenleft}{\kern0pt}cond{\isacharunderscore}{\kern0pt}exp\ M\ F\ f\ {\isasymxi}{\isacharparenright}{\kern0pt}\ {\isacharparenleft}{\kern0pt}cond{\isacharunderscore}{\kern0pt}exp\ M\ F\ g\ {\isasymxi}{\isacharparenright}{\kern0pt}{\isachardoublequoteclose}\isanewline
%
\isadelimproof
%
\endisadelimproof
%
\isatagproof
\isacommand{proof}\isamarkupfalse%
\ {\isacharminus}{\kern0pt}\isanewline
\ \ \isacommand{have}\isamarkupfalse%
\ {\isachardoublequoteopen}AE\ {\isasymxi}\ in\ M{\isachardot}{\kern0pt}\ cond{\isacharunderscore}{\kern0pt}exp\ M\ F\ {\isacharparenleft}{\kern0pt}{\isasymlambda}x{\isachardot}{\kern0pt}\ min\ {\isacharparenleft}{\kern0pt}f\ x{\isacharparenright}{\kern0pt}\ {\isacharparenleft}{\kern0pt}g\ x{\isacharparenright}{\kern0pt}{\isacharparenright}{\kern0pt}\ {\isasymxi}\ {\isasymle}\ cond{\isacharunderscore}{\kern0pt}exp\ M\ F\ f\ {\isasymxi}{\isachardoublequoteclose}\ \isacommand{by}\isamarkupfalse%
\ {\isacharparenleft}{\kern0pt}intro\ cond{\isacharunderscore}{\kern0pt}exp{\isacharunderscore}{\kern0pt}mono\ integrable{\isacharunderscore}{\kern0pt}min\ assms{\isacharcomma}{\kern0pt}\ simp{\isacharparenright}{\kern0pt}\isanewline
\ \ \isacommand{moreover}\isamarkupfalse%
\ \isacommand{have}\isamarkupfalse%
\ {\isachardoublequoteopen}AE\ {\isasymxi}\ in\ M{\isachardot}{\kern0pt}\ cond{\isacharunderscore}{\kern0pt}exp\ M\ F\ {\isacharparenleft}{\kern0pt}{\isasymlambda}x{\isachardot}{\kern0pt}\ min\ {\isacharparenleft}{\kern0pt}f\ x{\isacharparenright}{\kern0pt}\ {\isacharparenleft}{\kern0pt}g\ x{\isacharparenright}{\kern0pt}{\isacharparenright}{\kern0pt}\ {\isasymxi}\ {\isasymle}\ cond{\isacharunderscore}{\kern0pt}exp\ M\ F\ g\ {\isasymxi}{\isachardoublequoteclose}\ \isacommand{by}\isamarkupfalse%
\ {\isacharparenleft}{\kern0pt}intro\ cond{\isacharunderscore}{\kern0pt}exp{\isacharunderscore}{\kern0pt}mono\ integrable{\isacharunderscore}{\kern0pt}min\ assms{\isacharcomma}{\kern0pt}\ simp{\isacharparenright}{\kern0pt}\isanewline
\ \ \isacommand{ultimately}\isamarkupfalse%
\ \isacommand{show}\isamarkupfalse%
\ {\isachardoublequoteopen}AE\ {\isasymxi}\ in\ M{\isachardot}{\kern0pt}\ cond{\isacharunderscore}{\kern0pt}exp\ M\ F\ {\isacharparenleft}{\kern0pt}{\isasymlambda}x{\isachardot}{\kern0pt}\ min\ {\isacharparenleft}{\kern0pt}f\ x{\isacharparenright}{\kern0pt}\ {\isacharparenleft}{\kern0pt}g\ x{\isacharparenright}{\kern0pt}{\isacharparenright}{\kern0pt}\ {\isasymxi}\ {\isasymle}\ min\ {\isacharparenleft}{\kern0pt}cond{\isacharunderscore}{\kern0pt}exp\ M\ F\ f\ {\isasymxi}{\isacharparenright}{\kern0pt}\ {\isacharparenleft}{\kern0pt}cond{\isacharunderscore}{\kern0pt}exp\ M\ F\ g\ {\isasymxi}{\isacharparenright}{\kern0pt}{\isachardoublequoteclose}\ \isacommand{by}\isamarkupfalse%
\ fastforce\isanewline
\isacommand{qed}\isamarkupfalse%
%
\endisatagproof
{\isafoldproof}%
%
\isadelimproof
\isanewline
%
\endisadelimproof
\isanewline
\isacommand{lemma}\isamarkupfalse%
\ cond{\isacharunderscore}{\kern0pt}exp{\isacharunderscore}{\kern0pt}max{\isacharcolon}{\kern0pt}\isanewline
\ \ \isakeyword{fixes}\ f\ {\isacharcolon}{\kern0pt}{\isacharcolon}{\kern0pt}\ {\isachardoublequoteopen}{\isacharprime}{\kern0pt}a\ {\isasymRightarrow}\ {\isacharprime}{\kern0pt}b\ {\isacharcolon}{\kern0pt}{\isacharcolon}{\kern0pt}\ {\isacharbraceleft}{\kern0pt}second{\isacharunderscore}{\kern0pt}countable{\isacharunderscore}{\kern0pt}topology{\isacharcomma}{\kern0pt}\ banach{\isacharcomma}{\kern0pt}\ linorder{\isacharunderscore}{\kern0pt}topology{\isacharcomma}{\kern0pt}\ ordered{\isacharunderscore}{\kern0pt}real{\isacharunderscore}{\kern0pt}vector{\isacharbraceright}{\kern0pt}{\isachardoublequoteclose}\isanewline
\ \ \isakeyword{assumes}\ {\isachardoublequoteopen}integrable\ M\ f{\isachardoublequoteclose}\ {\isachardoublequoteopen}integrable\ M\ g{\isachardoublequoteclose}\isanewline
\ \ \isakeyword{shows}\ {\isachardoublequoteopen}AE\ {\isasymxi}\ in\ M{\isachardot}{\kern0pt}\ cond{\isacharunderscore}{\kern0pt}exp\ M\ F\ {\isacharparenleft}{\kern0pt}{\isasymlambda}x{\isachardot}{\kern0pt}\ max\ {\isacharparenleft}{\kern0pt}f\ x{\isacharparenright}{\kern0pt}\ {\isacharparenleft}{\kern0pt}g\ x{\isacharparenright}{\kern0pt}{\isacharparenright}{\kern0pt}\ {\isasymxi}\ {\isasymge}\ max\ {\isacharparenleft}{\kern0pt}cond{\isacharunderscore}{\kern0pt}exp\ M\ F\ f\ {\isasymxi}{\isacharparenright}{\kern0pt}\ {\isacharparenleft}{\kern0pt}cond{\isacharunderscore}{\kern0pt}exp\ M\ F\ g\ {\isasymxi}{\isacharparenright}{\kern0pt}{\isachardoublequoteclose}\isanewline
%
\isadelimproof
%
\endisadelimproof
%
\isatagproof
\isacommand{proof}\isamarkupfalse%
\ {\isacharminus}{\kern0pt}\isanewline
\ \ \isacommand{have}\isamarkupfalse%
\ {\isachardoublequoteopen}AE\ {\isasymxi}\ in\ M{\isachardot}{\kern0pt}\ cond{\isacharunderscore}{\kern0pt}exp\ M\ F\ {\isacharparenleft}{\kern0pt}{\isasymlambda}x{\isachardot}{\kern0pt}\ max\ {\isacharparenleft}{\kern0pt}f\ x{\isacharparenright}{\kern0pt}\ {\isacharparenleft}{\kern0pt}g\ x{\isacharparenright}{\kern0pt}{\isacharparenright}{\kern0pt}\ {\isasymxi}\ {\isasymge}\ cond{\isacharunderscore}{\kern0pt}exp\ M\ F\ f\ {\isasymxi}{\isachardoublequoteclose}\ \isacommand{by}\isamarkupfalse%
\ {\isacharparenleft}{\kern0pt}intro\ cond{\isacharunderscore}{\kern0pt}exp{\isacharunderscore}{\kern0pt}mono\ integrable{\isacharunderscore}{\kern0pt}max\ assms{\isacharcomma}{\kern0pt}\ simp{\isacharparenright}{\kern0pt}\isanewline
\ \ \isacommand{moreover}\isamarkupfalse%
\ \isacommand{have}\isamarkupfalse%
\ {\isachardoublequoteopen}AE\ {\isasymxi}\ in\ M{\isachardot}{\kern0pt}\ cond{\isacharunderscore}{\kern0pt}exp\ M\ F\ {\isacharparenleft}{\kern0pt}{\isasymlambda}x{\isachardot}{\kern0pt}\ max\ {\isacharparenleft}{\kern0pt}f\ x{\isacharparenright}{\kern0pt}\ {\isacharparenleft}{\kern0pt}g\ x{\isacharparenright}{\kern0pt}{\isacharparenright}{\kern0pt}\ {\isasymxi}\ {\isasymge}\ cond{\isacharunderscore}{\kern0pt}exp\ M\ F\ g\ {\isasymxi}{\isachardoublequoteclose}\ \isacommand{by}\isamarkupfalse%
\ {\isacharparenleft}{\kern0pt}intro\ cond{\isacharunderscore}{\kern0pt}exp{\isacharunderscore}{\kern0pt}mono\ integrable{\isacharunderscore}{\kern0pt}max\ assms{\isacharcomma}{\kern0pt}\ simp{\isacharparenright}{\kern0pt}\isanewline
\ \ \isacommand{ultimately}\isamarkupfalse%
\ \isacommand{show}\isamarkupfalse%
\ {\isachardoublequoteopen}AE\ {\isasymxi}\ in\ M{\isachardot}{\kern0pt}\ cond{\isacharunderscore}{\kern0pt}exp\ M\ F\ {\isacharparenleft}{\kern0pt}{\isasymlambda}x{\isachardot}{\kern0pt}\ max\ {\isacharparenleft}{\kern0pt}f\ x{\isacharparenright}{\kern0pt}\ {\isacharparenleft}{\kern0pt}g\ x{\isacharparenright}{\kern0pt}{\isacharparenright}{\kern0pt}\ {\isasymxi}\ {\isasymge}\ max\ {\isacharparenleft}{\kern0pt}cond{\isacharunderscore}{\kern0pt}exp\ M\ F\ f\ {\isasymxi}{\isacharparenright}{\kern0pt}\ {\isacharparenleft}{\kern0pt}cond{\isacharunderscore}{\kern0pt}exp\ M\ F\ g\ {\isasymxi}{\isacharparenright}{\kern0pt}{\isachardoublequoteclose}\ \isacommand{by}\isamarkupfalse%
\ fastforce\isanewline
\isacommand{qed}\isamarkupfalse%
%
\endisatagproof
{\isafoldproof}%
%
\isadelimproof
\isanewline
%
\endisadelimproof
\isanewline
\isacommand{lemma}\isamarkupfalse%
\ cond{\isacharunderscore}{\kern0pt}exp{\isacharunderscore}{\kern0pt}inf{\isacharcolon}{\kern0pt}\isanewline
\ \ \isakeyword{fixes}\ f\ {\isacharcolon}{\kern0pt}{\isacharcolon}{\kern0pt}\ {\isachardoublequoteopen}{\isacharprime}{\kern0pt}a\ {\isasymRightarrow}\ {\isacharprime}{\kern0pt}b\ {\isacharcolon}{\kern0pt}{\isacharcolon}{\kern0pt}\ {\isacharbraceleft}{\kern0pt}second{\isacharunderscore}{\kern0pt}countable{\isacharunderscore}{\kern0pt}topology{\isacharcomma}{\kern0pt}\ banach{\isacharcomma}{\kern0pt}\ linorder{\isacharunderscore}{\kern0pt}topology{\isacharcomma}{\kern0pt}\ ordered{\isacharunderscore}{\kern0pt}real{\isacharunderscore}{\kern0pt}vector{\isacharcomma}{\kern0pt}\ lattice{\isacharbraceright}{\kern0pt}{\isachardoublequoteclose}\isanewline
\ \ \isakeyword{assumes}\ {\isachardoublequoteopen}integrable\ M\ f{\isachardoublequoteclose}\ {\isachardoublequoteopen}integrable\ M\ g{\isachardoublequoteclose}\isanewline
\ \ \isakeyword{shows}\ {\isachardoublequoteopen}AE\ {\isasymxi}\ in\ M{\isachardot}{\kern0pt}\ cond{\isacharunderscore}{\kern0pt}exp\ M\ F\ {\isacharparenleft}{\kern0pt}{\isasymlambda}x{\isachardot}{\kern0pt}\ inf\ {\isacharparenleft}{\kern0pt}f\ x{\isacharparenright}{\kern0pt}\ {\isacharparenleft}{\kern0pt}g\ x{\isacharparenright}{\kern0pt}{\isacharparenright}{\kern0pt}\ {\isasymxi}\ {\isasymle}\ inf\ {\isacharparenleft}{\kern0pt}cond{\isacharunderscore}{\kern0pt}exp\ M\ F\ f\ {\isasymxi}{\isacharparenright}{\kern0pt}\ {\isacharparenleft}{\kern0pt}cond{\isacharunderscore}{\kern0pt}exp\ M\ F\ g\ {\isasymxi}{\isacharparenright}{\kern0pt}{\isachardoublequoteclose}\isanewline
%
\isadelimproof
\ \ %
\endisadelimproof
%
\isatagproof
\isacommand{unfolding}\isamarkupfalse%
\ inf{\isacharunderscore}{\kern0pt}min\ \isacommand{using}\isamarkupfalse%
\ assms\ \isacommand{by}\isamarkupfalse%
\ {\isacharparenleft}{\kern0pt}rule\ cond{\isacharunderscore}{\kern0pt}exp{\isacharunderscore}{\kern0pt}min{\isacharparenright}{\kern0pt}%
\endisatagproof
{\isafoldproof}%
%
\isadelimproof
\isanewline
%
\endisadelimproof
\isanewline
\isacommand{lemma}\isamarkupfalse%
\ cond{\isacharunderscore}{\kern0pt}exp{\isacharunderscore}{\kern0pt}sup{\isacharcolon}{\kern0pt}\isanewline
\ \ \isakeyword{fixes}\ f\ {\isacharcolon}{\kern0pt}{\isacharcolon}{\kern0pt}\ {\isachardoublequoteopen}{\isacharprime}{\kern0pt}a\ {\isasymRightarrow}\ {\isacharprime}{\kern0pt}b\ {\isacharcolon}{\kern0pt}{\isacharcolon}{\kern0pt}\ {\isacharbraceleft}{\kern0pt}second{\isacharunderscore}{\kern0pt}countable{\isacharunderscore}{\kern0pt}topology{\isacharcomma}{\kern0pt}\ banach{\isacharcomma}{\kern0pt}\ linorder{\isacharunderscore}{\kern0pt}topology{\isacharcomma}{\kern0pt}\ ordered{\isacharunderscore}{\kern0pt}real{\isacharunderscore}{\kern0pt}vector{\isacharcomma}{\kern0pt}\ lattice{\isacharbraceright}{\kern0pt}{\isachardoublequoteclose}\isanewline
\ \ \isakeyword{assumes}\ {\isachardoublequoteopen}integrable\ M\ f{\isachardoublequoteclose}\ {\isachardoublequoteopen}integrable\ M\ g{\isachardoublequoteclose}\isanewline
\ \ \isakeyword{shows}\ {\isachardoublequoteopen}AE\ {\isasymxi}\ in\ M{\isachardot}{\kern0pt}\ cond{\isacharunderscore}{\kern0pt}exp\ M\ F\ {\isacharparenleft}{\kern0pt}{\isasymlambda}x{\isachardot}{\kern0pt}\ sup\ {\isacharparenleft}{\kern0pt}f\ x{\isacharparenright}{\kern0pt}\ {\isacharparenleft}{\kern0pt}g\ x{\isacharparenright}{\kern0pt}{\isacharparenright}{\kern0pt}\ {\isasymxi}\ {\isasymge}\ sup\ {\isacharparenleft}{\kern0pt}cond{\isacharunderscore}{\kern0pt}exp\ M\ F\ f\ {\isasymxi}{\isacharparenright}{\kern0pt}\ {\isacharparenleft}{\kern0pt}cond{\isacharunderscore}{\kern0pt}exp\ M\ F\ g\ {\isasymxi}{\isacharparenright}{\kern0pt}{\isachardoublequoteclose}\isanewline
%
\isadelimproof
\ \ %
\endisadelimproof
%
\isatagproof
\isacommand{unfolding}\isamarkupfalse%
\ sup{\isacharunderscore}{\kern0pt}max\ \isacommand{using}\isamarkupfalse%
\ assms\ \isacommand{by}\isamarkupfalse%
\ {\isacharparenleft}{\kern0pt}rule\ cond{\isacharunderscore}{\kern0pt}exp{\isacharunderscore}{\kern0pt}max{\isacharparenright}{\kern0pt}%
\endisatagproof
{\isafoldproof}%
%
\isadelimproof
\isanewline
%
\endisadelimproof
\isanewline
\isacommand{end}\isamarkupfalse%
\isanewline
%
\isadelimtheory
\isanewline
%
\endisadelimtheory
%
\isatagtheory
\isacommand{end}\isamarkupfalse%
%
\endisatagtheory
{\isafoldtheory}%
%
\isadelimtheory
%
\endisadelimtheory
%
\end{isabellebody}%
\endinput
%:%file=Banach_Conditional_Expectation.tex%:%
%:%10=1%:%
%:%11=1%:%
%:%12=2%:%
%:%13=3%:%
%:%14=4%:%
%:%19=4%:%
%:%22=5%:%
%:%23=6%:%
%:%24=6%:%
%:%25=7%:%
%:%28=10%:%
%:%29=11%:%
%:%30=12%:%
%:%31=12%:%
%:%32=13%:%
%:%33=14%:%
%:%34=15%:%
%:%35=16%:%
%:%36=17%:%
%:%39=18%:%
%:%43=18%:%
%:%44=18%:%
%:%45=18%:%
%:%46=18%:%
%:%51=18%:%
%:%54=19%:%
%:%55=20%:%
%:%56=20%:%
%:%57=21%:%
%:%58=22%:%
%:%59=23%:%
%:%60=24%:%
%:%61=25%:%
%:%64=26%:%
%:%68=26%:%
%:%69=26%:%
%:%70=26%:%
%:%71=26%:%
%:%76=26%:%
%:%79=27%:%
%:%80=28%:%
%:%81=29%:%
%:%82=30%:%
%:%83=30%:%
%:%84=31%:%
%:%85=32%:%
%:%86=33%:%
%:%93=34%:%
%:%94=34%:%
%:%95=35%:%
%:%96=35%:%
%:%97=36%:%
%:%98=36%:%
%:%99=37%:%
%:%100=37%:%
%:%101=38%:%
%:%102=38%:%
%:%103=38%:%
%:%104=39%:%
%:%105=39%:%
%:%106=40%:%
%:%107=40%:%
%:%108=41%:%
%:%109=41%:%
%:%110=41%:%
%:%111=41%:%
%:%112=41%:%
%:%113=42%:%
%:%114=42%:%
%:%115=43%:%
%:%116=43%:%
%:%117=44%:%
%:%118=44%:%
%:%119=44%:%
%:%120=44%:%
%:%121=44%:%
%:%122=45%:%
%:%123=45%:%
%:%124=46%:%
%:%125=46%:%
%:%126=47%:%
%:%127=47%:%
%:%128=47%:%
%:%129=47%:%
%:%130=47%:%
%:%131=48%:%
%:%132=48%:%
%:%133=49%:%
%:%139=49%:%
%:%142=50%:%
%:%143=51%:%
%:%144=51%:%
%:%145=52%:%
%:%146=53%:%
%:%147=54%:%
%:%148=54%:%
%:%151=55%:%
%:%155=55%:%
%:%156=55%:%
%:%161=55%:%
%:%164=56%:%
%:%165=57%:%
%:%166=57%:%
%:%169=58%:%
%:%173=58%:%
%:%174=58%:%
%:%179=58%:%
%:%182=59%:%
%:%183=60%:%
%:%184=60%:%
%:%185=61%:%
%:%186=62%:%
%:%188=62%:%
%:%192=62%:%
%:%193=62%:%
%:%194=62%:%
%:%201=62%:%
%:%202=63%:%
%:%203=64%:%
%:%204=64%:%
%:%205=65%:%
%:%206=66%:%
%:%207=67%:%
%:%208=67%:%
%:%211=68%:%
%:%215=68%:%
%:%216=68%:%
%:%221=68%:%
%:%224=69%:%
%:%225=70%:%
%:%226=70%:%
%:%227=71%:%
%:%228=72%:%
%:%231=73%:%
%:%235=73%:%
%:%236=73%:%
%:%237=73%:%
%:%238=73%:%
%:%243=73%:%
%:%246=74%:%
%:%247=75%:%
%:%248=75%:%
%:%249=76%:%
%:%250=77%:%
%:%251=78%:%
%:%252=79%:%
%:%259=80%:%
%:%260=80%:%
%:%261=81%:%
%:%262=81%:%
%:%263=81%:%
%:%264=81%:%
%:%265=82%:%
%:%266=82%:%
%:%267=83%:%
%:%268=83%:%
%:%269=83%:%
%:%270=84%:%
%:%271=84%:%
%:%272=85%:%
%:%273=85%:%
%:%274=85%:%
%:%275=86%:%
%:%276=86%:%
%:%277=86%:%
%:%278=86%:%
%:%279=86%:%
%:%280=87%:%
%:%281=87%:%
%:%282=87%:%
%:%283=87%:%
%:%284=88%:%
%:%285=88%:%
%:%286=88%:%
%:%287=88%:%
%:%288=88%:%
%:%289=89%:%
%:%290=89%:%
%:%291=89%:%
%:%292=89%:%
%:%293=89%:%
%:%294=90%:%
%:%295=90%:%
%:%296=90%:%
%:%297=90%:%
%:%298=91%:%
%:%299=91%:%
%:%300=92%:%
%:%301=92%:%
%:%302=92%:%
%:%303=92%:%
%:%304=93%:%
%:%305=93%:%
%:%306=93%:%
%:%307=93%:%
%:%308=93%:%
%:%309=94%:%
%:%315=94%:%
%:%318=95%:%
%:%319=96%:%
%:%320=96%:%
%:%321=97%:%
%:%322=98%:%
%:%323=99%:%
%:%324=100%:%
%:%327=101%:%
%:%331=101%:%
%:%332=101%:%
%:%341=103%:%
%:%343=105%:%
%:%344=105%:%
%:%345=106%:%
%:%346=107%:%
%:%353=108%:%
%:%354=108%:%
%:%355=109%:%
%:%356=109%:%
%:%357=110%:%
%:%358=110%:%
%:%359=110%:%
%:%360=111%:%
%:%361=111%:%
%:%362=111%:%
%:%363=111%:%
%:%364=111%:%
%:%365=112%:%
%:%366=112%:%
%:%371=112%:%
%:%374=113%:%
%:%375=114%:%
%:%376=114%:%
%:%377=115%:%
%:%378=116%:%
%:%379=117%:%
%:%386=118%:%
%:%387=118%:%
%:%388=119%:%
%:%389=119%:%
%:%390=120%:%
%:%391=120%:%
%:%392=120%:%
%:%393=120%:%
%:%394=120%:%
%:%395=121%:%
%:%396=121%:%
%:%397=121%:%
%:%398=121%:%
%:%399=122%:%
%:%400=122%:%
%:%401=123%:%
%:%402=123%:%
%:%403=124%:%
%:%404=124%:%
%:%405=124%:%
%:%406=124%:%
%:%407=124%:%
%:%408=125%:%
%:%409=125%:%
%:%410=125%:%
%:%411=125%:%
%:%412=125%:%
%:%413=126%:%
%:%419=126%:%
%:%422=127%:%
%:%423=128%:%
%:%424=128%:%
%:%425=129%:%
%:%426=130%:%
%:%429=131%:%
%:%433=131%:%
%:%434=131%:%
%:%435=132%:%
%:%436=132%:%
%:%437=133%:%
%:%442=133%:%
%:%445=134%:%
%:%446=135%:%
%:%447=135%:%
%:%448=136%:%
%:%449=137%:%
%:%452=138%:%
%:%456=138%:%
%:%457=138%:%
%:%458=139%:%
%:%459=139%:%
%:%460=140%:%
%:%461=140%:%
%:%462=141%:%
%:%467=141%:%
%:%470=142%:%
%:%471=143%:%
%:%472=143%:%
%:%473=144%:%
%:%474=145%:%
%:%475=146%:%
%:%482=147%:%
%:%483=147%:%
%:%484=148%:%
%:%485=148%:%
%:%486=149%:%
%:%487=149%:%
%:%488=149%:%
%:%489=149%:%
%:%490=149%:%
%:%491=150%:%
%:%492=150%:%
%:%493=150%:%
%:%494=150%:%
%:%495=151%:%
%:%496=151%:%
%:%497=152%:%
%:%498=152%:%
%:%499=153%:%
%:%500=153%:%
%:%501=153%:%
%:%502=153%:%
%:%503=153%:%
%:%504=154%:%
%:%505=154%:%
%:%506=154%:%
%:%507=154%:%
%:%508=154%:%
%:%509=155%:%
%:%515=155%:%
%:%518=156%:%
%:%519=157%:%
%:%520=157%:%
%:%521=158%:%
%:%522=159%:%
%:%523=160%:%
%:%526=161%:%
%:%530=161%:%
%:%531=161%:%
%:%536=161%:%
%:%539=162%:%
%:%540=163%:%
%:%541=163%:%
%:%542=164%:%
%:%543=165%:%
%:%544=166%:%
%:%547=167%:%
%:%551=167%:%
%:%552=167%:%
%:%553=167%:%
%:%558=167%:%
%:%561=168%:%
%:%562=169%:%
%:%563=169%:%
%:%564=170%:%
%:%565=171%:%
%:%566=172%:%
%:%569=173%:%
%:%573=173%:%
%:%574=173%:%
%:%575=173%:%
%:%584=175%:%
%:%586=177%:%
%:%587=177%:%
%:%588=178%:%
%:%589=179%:%
%:%596=180%:%
%:%597=180%:%
%:%598=181%:%
%:%599=181%:%
%:%600=182%:%
%:%601=182%:%
%:%602=182%:%
%:%603=182%:%
%:%604=183%:%
%:%605=183%:%
%:%606=183%:%
%:%607=183%:%
%:%608=183%:%
%:%609=184%:%
%:%610=184%:%
%:%611=184%:%
%:%612=184%:%
%:%613=184%:%
%:%614=185%:%
%:%615=185%:%
%:%616=185%:%
%:%617=185%:%
%:%618=186%:%
%:%619=186%:%
%:%620=187%:%
%:%621=187%:%
%:%622=188%:%
%:%623=188%:%
%:%624=188%:%
%:%625=188%:%
%:%626=188%:%
%:%627=189%:%
%:%628=189%:%
%:%629=190%:%
%:%630=190%:%
%:%631=191%:%
%:%632=191%:%
%:%633=191%:%
%:%634=191%:%
%:%635=192%:%
%:%636=192%:%
%:%637=193%:%
%:%638=193%:%
%:%639=194%:%
%:%640=194%:%
%:%641=194%:%
%:%642=194%:%
%:%643=195%:%
%:%649=195%:%
%:%652=196%:%
%:%653=197%:%
%:%654=197%:%
%:%655=198%:%
%:%656=199%:%
%:%657=200%:%
%:%664=201%:%
%:%665=201%:%
%:%666=202%:%
%:%667=202%:%
%:%668=202%:%
%:%669=202%:%
%:%670=203%:%
%:%671=203%:%
%:%672=203%:%
%:%673=203%:%
%:%674=204%:%
%:%684=206%:%
%:%686=208%:%
%:%687=208%:%
%:%688=209%:%
%:%689=210%:%
%:%690=211%:%
%:%697=212%:%
%:%698=212%:%
%:%699=213%:%
%:%700=213%:%
%:%701=214%:%
%:%702=214%:%
%:%703=214%:%
%:%704=214%:%
%:%705=215%:%
%:%706=215%:%
%:%707=215%:%
%:%708=215%:%
%:%709=215%:%
%:%710=216%:%
%:%711=216%:%
%:%712=216%:%
%:%713=216%:%
%:%714=216%:%
%:%715=217%:%
%:%716=217%:%
%:%717=217%:%
%:%718=217%:%
%:%719=218%:%
%:%720=218%:%
%:%721=219%:%
%:%722=219%:%
%:%723=220%:%
%:%724=220%:%
%:%725=220%:%
%:%726=220%:%
%:%727=221%:%
%:%728=221%:%
%:%729=222%:%
%:%730=222%:%
%:%731=223%:%
%:%732=223%:%
%:%733=223%:%
%:%734=223%:%
%:%735=224%:%
%:%736=224%:%
%:%737=225%:%
%:%738=225%:%
%:%739=226%:%
%:%740=226%:%
%:%741=226%:%
%:%742=226%:%
%:%743=226%:%
%:%744=227%:%
%:%750=227%:%
%:%753=228%:%
%:%754=229%:%
%:%755=229%:%
%:%756=230%:%
%:%757=231%:%
%:%758=232%:%
%:%761=233%:%
%:%765=233%:%
%:%766=233%:%
%:%767=233%:%
%:%772=233%:%
%:%775=234%:%
%:%776=235%:%
%:%777=235%:%
%:%778=236%:%
%:%779=237%:%
%:%780=238%:%
%:%787=239%:%
%:%788=239%:%
%:%789=240%:%
%:%790=240%:%
%:%791=241%:%
%:%792=241%:%
%:%793=241%:%
%:%794=241%:%
%:%795=241%:%
%:%796=242%:%
%:%797=242%:%
%:%798=243%:%
%:%799=243%:%
%:%800=244%:%
%:%801=244%:%
%:%802=245%:%
%:%803=245%:%
%:%804=246%:%
%:%805=246%:%
%:%806=247%:%
%:%807=247%:%
%:%808=247%:%
%:%809=247%:%
%:%810=248%:%
%:%811=248%:%
%:%812=249%:%
%:%813=249%:%
%:%814=250%:%
%:%815=250%:%
%:%816=251%:%
%:%817=251%:%
%:%818=252%:%
%:%819=252%:%
%:%820=253%:%
%:%821=253%:%
%:%822=253%:%
%:%823=253%:%
%:%824=254%:%
%:%825=254%:%
%:%826=254%:%
%:%827=254%:%
%:%828=255%:%
%:%829=255%:%
%:%830=255%:%
%:%831=255%:%
%:%832=255%:%
%:%833=256%:%
%:%834=256%:%
%:%835=257%:%
%:%836=257%:%
%:%837=257%:%
%:%838=257%:%
%:%839=257%:%
%:%840=258%:%
%:%841=258%:%
%:%842=259%:%
%:%848=259%:%
%:%851=260%:%
%:%852=261%:%
%:%853=261%:%
%:%854=262%:%
%:%855=263%:%
%:%856=264%:%
%:%859=265%:%
%:%863=265%:%
%:%864=265%:%
%:%865=265%:%
%:%870=265%:%
%:%873=266%:%
%:%874=267%:%
%:%875=267%:%
%:%876=268%:%
%:%877=269%:%
%:%878=270%:%
%:%881=271%:%
%:%885=271%:%
%:%886=271%:%
%:%887=272%:%
%:%888=272%:%
%:%889=273%:%
%:%890=273%:%
%:%891=274%:%
%:%892=274%:%
%:%893=274%:%
%:%894=274%:%
%:%895=274%:%
%:%896=275%:%
%:%897=275%:%
%:%898=276%:%
%:%899=276%:%
%:%900=277%:%
%:%901=277%:%
%:%902=277%:%
%:%903=277%:%
%:%904=277%:%
%:%905=278%:%
%:%906=278%:%
%:%907=279%:%
%:%908=279%:%
%:%909=280%:%
%:%910=280%:%
%:%911=280%:%
%:%912=280%:%
%:%913=280%:%
%:%914=281%:%
%:%920=281%:%
%:%923=282%:%
%:%924=283%:%
%:%925=283%:%
%:%926=284%:%
%:%927=285%:%
%:%928=286%:%
%:%935=287%:%
%:%936=287%:%
%:%937=288%:%
%:%938=288%:%
%:%939=288%:%
%:%940=288%:%
%:%941=289%:%
%:%942=289%:%
%:%943=289%:%
%:%944=289%:%
%:%945=290%:%
%:%946=290%:%
%:%947=290%:%
%:%948=290%:%
%:%949=290%:%
%:%950=291%:%
%:%951=292%:%
%:%952=292%:%
%:%953=292%:%
%:%954=292%:%
%:%955=293%:%
%:%956=294%:%
%:%957=294%:%
%:%958=294%:%
%:%959=294%:%
%:%960=295%:%
%:%961=295%:%
%:%962=295%:%
%:%963=295%:%
%:%964=296%:%
%:%965=297%:%
%:%966=297%:%
%:%967=297%:%
%:%968=297%:%
%:%969=298%:%
%:%970=298%:%
%:%971=298%:%
%:%972=298%:%
%:%973=299%:%
%:%974=299%:%
%:%975=299%:%
%:%976=299%:%
%:%977=300%:%
%:%978=300%:%
%:%979=300%:%
%:%980=300%:%
%:%981=301%:%
%:%982=301%:%
%:%983=301%:%
%:%984=301%:%
%:%985=302%:%
%:%986=302%:%
%:%987=302%:%
%:%988=302%:%
%:%989=302%:%
%:%990=302%:%
%:%991=303%:%
%:%992=303%:%
%:%993=303%:%
%:%994=303%:%
%:%995=303%:%
%:%996=304%:%
%:%997=304%:%
%:%998=304%:%
%:%999=304%:%
%:%1000=305%:%
%:%1001=305%:%
%:%1002=305%:%
%:%1003=305%:%
%:%1004=306%:%
%:%1010=306%:%
%:%1013=307%:%
%:%1014=308%:%
%:%1015=308%:%
%:%1016=309%:%
%:%1017=310%:%
%:%1018=311%:%
%:%1021=312%:%
%:%1025=312%:%
%:%1026=312%:%
%:%1027=313%:%
%:%1028=313%:%
%:%1029=314%:%
%:%1030=314%:%
%:%1031=315%:%
%:%1032=315%:%
%:%1033=315%:%
%:%1034=316%:%
%:%1035=316%:%
%:%1036=316%:%
%:%1037=316%:%
%:%1038=317%:%
%:%1039=317%:%
%:%1040=317%:%
%:%1041=318%:%
%:%1042=318%:%
%:%1043=318%:%
%:%1044=318%:%
%:%1045=318%:%
%:%1046=319%:%
%:%1047=319%:%
%:%1048=319%:%
%:%1049=319%:%
%:%1050=319%:%
%:%1051=320%:%
%:%1052=320%:%
%:%1053=321%:%
%:%1054=321%:%
%:%1055=322%:%
%:%1056=322%:%
%:%1057=322%:%
%:%1058=323%:%
%:%1059=323%:%
%:%1060=323%:%
%:%1061=323%:%
%:%1062=324%:%
%:%1063=325%:%
%:%1064=325%:%
%:%1065=325%:%
%:%1066=325%:%
%:%1067=326%:%
%:%1068=326%:%
%:%1069=326%:%
%:%1070=326%:%
%:%1071=326%:%
%:%1072=327%:%
%:%1073=327%:%
%:%1074=327%:%
%:%1075=327%:%
%:%1076=327%:%
%:%1077=328%:%
%:%1078=328%:%
%:%1079=328%:%
%:%1080=328%:%
%:%1081=329%:%
%:%1082=329%:%
%:%1083=329%:%
%:%1084=329%:%
%:%1085=329%:%
%:%1086=330%:%
%:%1087=330%:%
%:%1088=331%:%
%:%1089=331%:%
%:%1090=332%:%
%:%1091=332%:%
%:%1092=332%:%
%:%1093=332%:%
%:%1094=333%:%
%:%1095=333%:%
%:%1096=333%:%
%:%1097=333%:%
%:%1098=333%:%
%:%1099=334%:%
%:%1100=334%:%
%:%1101=334%:%
%:%1102=334%:%
%:%1103=334%:%
%:%1104=335%:%
%:%1105=335%:%
%:%1106=335%:%
%:%1107=335%:%
%:%1108=335%:%
%:%1109=336%:%
%:%1110=336%:%
%:%1111=336%:%
%:%1112=336%:%
%:%1113=336%:%
%:%1114=337%:%
%:%1115=337%:%
%:%1116=337%:%
%:%1117=337%:%
%:%1118=338%:%
%:%1124=338%:%
%:%1127=339%:%
%:%1128=340%:%
%:%1129=340%:%
%:%1130=341%:%
%:%1131=342%:%
%:%1132=343%:%
%:%1133=344%:%
%:%1134=345%:%
%:%1135=346%:%
%:%1136=347%:%
%:%1137=348%:%
%:%1138=349%:%
%:%1139=350%:%
%:%1146=351%:%
%:%1147=351%:%
%:%1148=352%:%
%:%1149=352%:%
%:%1150=352%:%
%:%1151=352%:%
%:%1152=353%:%
%:%1153=353%:%
%:%1154=353%:%
%:%1155=353%:%
%:%1156=354%:%
%:%1157=354%:%
%:%1158=354%:%
%:%1159=354%:%
%:%1160=355%:%
%:%1161=355%:%
%:%1162=355%:%
%:%1163=355%:%
%:%1164=356%:%
%:%1165=356%:%
%:%1166=356%:%
%:%1167=357%:%
%:%1168=358%:%
%:%1169=358%:%
%:%1170=358%:%
%:%1171=358%:%
%:%1172=359%:%
%:%1173=359%:%
%:%1174=359%:%
%:%1175=359%:%
%:%1176=360%:%
%:%1177=361%:%
%:%1178=361%:%
%:%1179=361%:%
%:%1180=361%:%
%:%1181=362%:%
%:%1182=362%:%
%:%1183=362%:%
%:%1184=362%:%
%:%1185=363%:%
%:%1186=364%:%
%:%1187=364%:%
%:%1188=364%:%
%:%1189=364%:%
%:%1190=365%:%
%:%1191=365%:%
%:%1192=365%:%
%:%1193=365%:%
%:%1194=366%:%
%:%1195=367%:%
%:%1196=367%:%
%:%1197=367%:%
%:%1198=367%:%
%:%1199=368%:%
%:%1200=368%:%
%:%1201=368%:%
%:%1202=368%:%
%:%1203=368%:%
%:%1204=369%:%
%:%1205=369%:%
%:%1206=369%:%
%:%1207=370%:%
%:%1208=370%:%
%:%1209=371%:%
%:%1210=371%:%
%:%1211=372%:%
%:%1212=372%:%
%:%1213=372%:%
%:%1214=373%:%
%:%1215=373%:%
%:%1216=373%:%
%:%1217=374%:%
%:%1218=374%:%
%:%1219=374%:%
%:%1220=374%:%
%:%1221=375%:%
%:%1222=375%:%
%:%1223=376%:%
%:%1224=376%:%
%:%1225=376%:%
%:%1226=377%:%
%:%1227=377%:%
%:%1228=378%:%
%:%1229=378%:%
%:%1230=378%:%
%:%1231=378%:%
%:%1232=379%:%
%:%1233=380%:%
%:%1234=380%:%
%:%1235=380%:%
%:%1236=380%:%
%:%1237=381%:%
%:%1238=382%:%
%:%1239=382%:%
%:%1240=383%:%
%:%1241=383%:%
%:%1242=384%:%
%:%1243=384%:%
%:%1244=385%:%
%:%1245=386%:%
%:%1246=386%:%
%:%1247=386%:%
%:%1248=386%:%
%:%1249=387%:%
%:%1250=388%:%
%:%1251=388%:%
%:%1252=389%:%
%:%1253=389%:%
%:%1254=390%:%
%:%1255=390%:%
%:%1256=390%:%
%:%1257=390%:%
%:%1258=391%:%
%:%1259=391%:%
%:%1260=392%:%
%:%1261=392%:%
%:%1262=392%:%
%:%1263=393%:%
%:%1264=393%:%
%:%1265=393%:%
%:%1266=394%:%
%:%1267=394%:%
%:%1268=394%:%
%:%1269=394%:%
%:%1270=395%:%
%:%1271=395%:%
%:%1272=395%:%
%:%1273=395%:%
%:%1274=395%:%
%:%1275=396%:%
%:%1276=396%:%
%:%1277=396%:%
%:%1278=396%:%
%:%1279=397%:%
%:%1280=397%:%
%:%1281=398%:%
%:%1282=398%:%
%:%1283=398%:%
%:%1284=398%:%
%:%1285=399%:%
%:%1286=399%:%
%:%1287=399%:%
%:%1288=400%:%
%:%1289=400%:%
%:%1290=401%:%
%:%1291=401%:%
%:%1292=401%:%
%:%1293=401%:%
%:%1294=402%:%
%:%1295=402%:%
%:%1296=402%:%
%:%1297=402%:%
%:%1298=402%:%
%:%1299=403%:%
%:%1300=403%:%
%:%1301=403%:%
%:%1302=403%:%
%:%1303=403%:%
%:%1304=404%:%
%:%1305=404%:%
%:%1306=404%:%
%:%1307=404%:%
%:%1308=404%:%
%:%1309=405%:%
%:%1310=405%:%
%:%1311=405%:%
%:%1312=405%:%
%:%1313=405%:%
%:%1314=406%:%
%:%1315=406%:%
%:%1316=407%:%
%:%1317=407%:%
%:%1318=407%:%
%:%1319=407%:%
%:%1320=407%:%
%:%1321=408%:%
%:%1322=408%:%
%:%1323=409%:%
%:%1324=409%:%
%:%1325=409%:%
%:%1326=409%:%
%:%1327=410%:%
%:%1328=410%:%
%:%1329=410%:%
%:%1330=410%:%
%:%1331=411%:%
%:%1332=412%:%
%:%1333=412%:%
%:%1334=413%:%
%:%1335=413%:%
%:%1336=414%:%
%:%1337=414%:%
%:%1338=414%:%
%:%1339=415%:%
%:%1340=415%:%
%:%1341=415%:%
%:%1342=415%:%
%:%1343=415%:%
%:%1344=416%:%
%:%1345=416%:%
%:%1346=416%:%
%:%1347=416%:%
%:%1348=416%:%
%:%1349=417%:%
%:%1350=417%:%
%:%1351=418%:%
%:%1352=418%:%
%:%1353=418%:%
%:%1354=419%:%
%:%1355=420%:%
%:%1356=420%:%
%:%1357=421%:%
%:%1358=421%:%
%:%1359=421%:%
%:%1360=422%:%
%:%1361=422%:%
%:%1362=423%:%
%:%1363=423%:%
%:%1364=424%:%
%:%1365=424%:%
%:%1366=424%:%
%:%1367=424%:%
%:%1368=425%:%
%:%1369=425%:%
%:%1370=425%:%
%:%1371=425%:%
%:%1372=426%:%
%:%1373=426%:%
%:%1374=427%:%
%:%1375=427%:%
%:%1376=428%:%
%:%1377=428%:%
%:%1378=429%:%
%:%1379=429%:%
%:%1380=430%:%
%:%1381=430%:%
%:%1382=431%:%
%:%1383=432%:%
%:%1384=432%:%
%:%1385=433%:%
%:%1386=433%:%
%:%1387=434%:%
%:%1388=434%:%
%:%1389=434%:%
%:%1390=434%:%
%:%1391=435%:%
%:%1392=435%:%
%:%1393=435%:%
%:%1394=435%:%
%:%1395=436%:%
%:%1396=436%:%
%:%1397=437%:%
%:%1398=437%:%
%:%1399=438%:%
%:%1400=438%:%
%:%1401=439%:%
%:%1402=439%:%
%:%1403=440%:%
%:%1404=440%:%
%:%1405=441%:%
%:%1406=442%:%
%:%1407=442%:%
%:%1408=442%:%
%:%1409=442%:%
%:%1410=443%:%
%:%1411=443%:%
%:%1412=443%:%
%:%1413=443%:%
%:%1414=443%:%
%:%1415=444%:%
%:%1416=444%:%
%:%1417=444%:%
%:%1418=444%:%
%:%1419=444%:%
%:%1420=445%:%
%:%1421=445%:%
%:%1422=445%:%
%:%1423=445%:%
%:%1424=446%:%
%:%1425=446%:%
%:%1426=447%:%
%:%1427=447%:%
%:%1428=447%:%
%:%1429=447%:%
%:%1430=448%:%
%:%1431=448%:%
%:%1432=448%:%
%:%1433=448%:%
%:%1434=449%:%
%:%1440=449%:%
%:%1443=450%:%
%:%1444=451%:%
%:%1445=451%:%
%:%1446=452%:%
%:%1447=453%:%
%:%1448=454%:%
%:%1449=455%:%
%:%1450=456%:%
%:%1451=457%:%
%:%1452=458%:%
%:%1459=459%:%
%:%1460=459%:%
%:%1461=460%:%
%:%1462=460%:%
%:%1463=460%:%
%:%1464=460%:%
%:%1465=461%:%
%:%1466=461%:%
%:%1467=461%:%
%:%1468=462%:%
%:%1474=462%:%
%:%1477=463%:%
%:%1478=464%:%
%:%1479=464%:%
%:%1480=465%:%
%:%1481=466%:%
%:%1482=467%:%
%:%1485=468%:%
%:%1489=468%:%
%:%1490=468%:%
%:%1491=469%:%
%:%1492=469%:%
%:%1493=470%:%
%:%1494=470%:%
%:%1495=471%:%
%:%1496=471%:%
%:%1497=471%:%
%:%1498=471%:%
%:%1499=472%:%
%:%1500=472%:%
%:%1501=473%:%
%:%1502=473%:%
%:%1503=474%:%
%:%1504=474%:%
%:%1505=474%:%
%:%1506=474%:%
%:%1507=475%:%
%:%1508=475%:%
%:%1509=476%:%
%:%1510=476%:%
%:%1511=477%:%
%:%1512=477%:%
%:%1513=477%:%
%:%1514=477%:%
%:%1515=478%:%
%:%1521=478%:%
%:%1524=479%:%
%:%1525=480%:%
%:%1526=480%:%
%:%1527=481%:%
%:%1528=482%:%
%:%1529=483%:%
%:%1532=484%:%
%:%1536=484%:%
%:%1537=484%:%
%:%1538=484%:%
%:%1543=484%:%
%:%1546=485%:%
%:%1547=486%:%
%:%1548=486%:%
%:%1549=487%:%
%:%1550=488%:%
%:%1551=489%:%
%:%1554=490%:%
%:%1558=490%:%
%:%1559=490%:%
%:%1560=490%:%
%:%1565=490%:%
%:%1568=491%:%
%:%1569=492%:%
%:%1570=492%:%
%:%1571=493%:%
%:%1572=494%:%
%:%1573=495%:%
%:%1576=496%:%
%:%1580=496%:%
%:%1581=496%:%
%:%1587=496%:%
%:%1590=497%:%
%:%1591=498%:%
%:%1592=498%:%
%:%1593=499%:%
%:%1594=500%:%
%:%1595=501%:%
%:%1598=502%:%
%:%1602=502%:%
%:%1603=502%:%
%:%1604=502%:%
%:%1609=502%:%
%:%1612=503%:%
%:%1613=504%:%
%:%1614=504%:%
%:%1615=505%:%
%:%1616=506%:%
%:%1617=507%:%
%:%1620=508%:%
%:%1624=508%:%
%:%1625=508%:%
%:%1626=508%:%
%:%1627=508%:%
%:%1632=508%:%
%:%1635=509%:%
%:%1636=510%:%
%:%1637=510%:%
%:%1638=511%:%
%:%1639=512%:%
%:%1640=513%:%
%:%1647=514%:%
%:%1648=514%:%
%:%1649=515%:%
%:%1650=515%:%
%:%1651=516%:%
%:%1652=516%:%
%:%1653=517%:%
%:%1654=518%:%
%:%1655=518%:%
%:%1656=518%:%
%:%1657=518%:%
%:%1658=519%:%
%:%1659=520%:%
%:%1660=520%:%
%:%1661=521%:%
%:%1662=521%:%
%:%1663=521%:%
%:%1664=522%:%
%:%1665=523%:%
%:%1666=523%:%
%:%1667=523%:%
%:%1668=523%:%
%:%1669=524%:%
%:%1670=525%:%
%:%1671=525%:%
%:%1672=525%:%
%:%1673=525%:%
%:%1674=526%:%
%:%1675=526%:%
%:%1676=526%:%
%:%1677=526%:%
%:%1678=526%:%
%:%1679=527%:%
%:%1680=527%:%
%:%1681=527%:%
%:%1682=527%:%
%:%1683=527%:%
%:%1684=528%:%
%:%1690=528%:%
%:%1693=529%:%
%:%1694=530%:%
%:%1695=530%:%
%:%1696=531%:%
%:%1697=532%:%
%:%1698=533%:%
%:%1705=534%:%
%:%1706=534%:%
%:%1707=535%:%
%:%1708=535%:%
%:%1709=535%:%
%:%1710=536%:%
%:%1711=536%:%
%:%1712=536%:%
%:%1713=536%:%
%:%1714=537%:%
%:%1715=538%:%
%:%1716=538%:%
%:%1717=538%:%
%:%1718=538%:%
%:%1719=539%:%
%:%1720=539%:%
%:%1721=539%:%
%:%1722=539%:%
%:%1723=539%:%
%:%1724=540%:%
%:%1725=540%:%
%:%1726=540%:%
%:%1727=540%:%
%:%1728=540%:%
%:%1729=541%:%
%:%1730=541%:%
%:%1731=541%:%
%:%1732=541%:%
%:%1733=541%:%
%:%1734=542%:%
%:%1735=542%:%
%:%1736=542%:%
%:%1737=542%:%
%:%1738=542%:%
%:%1739=543%:%
%:%1740=543%:%
%:%1741=543%:%
%:%1742=543%:%
%:%1743=544%:%
%:%1744=544%:%
%:%1758=546%:%
%:%1768=548%:%
%:%1769=548%:%
%:%1770=549%:%
%:%1771=550%:%
%:%1772=551%:%
%:%1779=552%:%
%:%1780=552%:%
%:%1781=553%:%
%:%1782=553%:%
%:%1783=554%:%
%:%1784=554%:%
%:%1785=554%:%
%:%1786=554%:%
%:%1787=555%:%
%:%1788=555%:%
%:%1789=555%:%
%:%1790=555%:%
%:%1791=556%:%
%:%1792=556%:%
%:%1793=557%:%
%:%1794=557%:%
%:%1795=558%:%
%:%1796=558%:%
%:%1797=559%:%
%:%1798=559%:%
%:%1799=559%:%
%:%1800=560%:%
%:%1801=560%:%
%:%1802=560%:%
%:%1803=560%:%
%:%1804=561%:%
%:%1805=561%:%
%:%1806=561%:%
%:%1807=562%:%
%:%1808=562%:%
%:%1809=562%:%
%:%1810=563%:%
%:%1811=563%:%
%:%1812=563%:%
%:%1813=563%:%
%:%1814=564%:%
%:%1815=564%:%
%:%1816=564%:%
%:%1817=564%:%
%:%1818=565%:%
%:%1819=565%:%
%:%1820=565%:%
%:%1821=565%:%
%:%1822=566%:%
%:%1823=566%:%
%:%1824=566%:%
%:%1825=566%:%
%:%1826=567%:%
%:%1827=567%:%
%:%1828=568%:%
%:%1829=568%:%
%:%1830=569%:%
%:%1831=569%:%
%:%1832=570%:%
%:%1833=570%:%
%:%1834=571%:%
%:%1835=571%:%
%:%1836=571%:%
%:%1837=571%:%
%:%1838=572%:%
%:%1839=572%:%
%:%1840=572%:%
%:%1841=573%:%
%:%1842=573%:%
%:%1843=573%:%
%:%1844=573%:%
%:%1845=573%:%
%:%1846=574%:%
%:%1847=574%:%
%:%1848=574%:%
%:%1849=574%:%
%:%1850=575%:%
%:%1851=575%:%
%:%1852=576%:%
%:%1853=576%:%
%:%1854=576%:%
%:%1855=576%:%
%:%1856=576%:%
%:%1857=577%:%
%:%1858=577%:%
%:%1859=577%:%
%:%1860=577%:%
%:%1861=578%:%
%:%1862=578%:%
%:%1863=579%:%
%:%1864=579%:%
%:%1865=579%:%
%:%1866=579%:%
%:%1867=580%:%
%:%1868=580%:%
%:%1869=580%:%
%:%1870=580%:%
%:%1871=580%:%
%:%1872=581%:%
%:%1873=581%:%
%:%1874=581%:%
%:%1875=581%:%
%:%1876=582%:%
%:%1877=582%:%
%:%1878=582%:%
%:%1879=582%:%
%:%1880=582%:%
%:%1881=583%:%
%:%1882=583%:%
%:%1883=584%:%
%:%1884=584%:%
%:%1885=584%:%
%:%1886=584%:%
%:%1887=584%:%
%:%1888=584%:%
%:%1889=585%:%
%:%1895=585%:%
%:%1898=586%:%
%:%1899=587%:%
%:%1900=587%:%
%:%1901=588%:%
%:%1902=589%:%
%:%1903=590%:%
%:%1910=591%:%
%:%1911=591%:%
%:%1912=592%:%
%:%1913=592%:%
%:%1914=592%:%
%:%1915=592%:%
%:%1916=593%:%
%:%1917=593%:%
%:%1918=593%:%
%:%1919=593%:%
%:%1920=593%:%
%:%1921=594%:%
%:%1922=594%:%
%:%1923=594%:%
%:%1924=594%:%
%:%1925=595%:%
%:%1931=595%:%
%:%1934=596%:%
%:%1935=597%:%
%:%1936=597%:%
%:%1937=598%:%
%:%1938=599%:%
%:%1939=600%:%
%:%1942=601%:%
%:%1946=601%:%
%:%1947=601%:%
%:%1948=602%:%
%:%1949=602%:%
%:%1954=602%:%
%:%1957=603%:%
%:%1958=604%:%
%:%1959=604%:%
%:%1960=605%:%
%:%1961=606%:%
%:%1962=607%:%
%:%1963=608%:%
%:%1970=609%:%
%:%1971=609%:%
%:%1972=610%:%
%:%1973=610%:%
%:%1974=611%:%
%:%1975=611%:%
%:%1976=611%:%
%:%1977=611%:%
%:%1978=612%:%
%:%1979=612%:%
%:%1980=613%:%
%:%1981=613%:%
%:%1982=613%:%
%:%1983=614%:%
%:%1984=614%:%
%:%1985=614%:%
%:%1986=614%:%
%:%1987=615%:%
%:%1988=615%:%
%:%1989=615%:%
%:%1990=615%:%
%:%1991=616%:%
%:%1992=616%:%
%:%1993=616%:%
%:%1994=616%:%
%:%1995=617%:%
%:%1996=617%:%
%:%1997=617%:%
%:%1998=617%:%
%:%1999=618%:%
%:%2000=618%:%
%:%2001=618%:%
%:%2002=618%:%
%:%2003=618%:%
%:%2004=619%:%
%:%2005=619%:%
%:%2006=619%:%
%:%2007=619%:%
%:%2008=619%:%
%:%2009=619%:%
%:%2010=620%:%
%:%2011=620%:%
%:%2012=620%:%
%:%2013=620%:%
%:%2014=620%:%
%:%2015=621%:%
%:%2016=621%:%
%:%2017=622%:%
%:%2018=622%:%
%:%2019=622%:%
%:%2020=622%:%
%:%2021=623%:%
%:%2027=623%:%
%:%2030=624%:%
%:%2031=625%:%
%:%2032=625%:%
%:%2033=626%:%
%:%2034=627%:%
%:%2035=628%:%
%:%2036=629%:%
%:%2043=630%:%
%:%2044=630%:%
%:%2045=631%:%
%:%2046=631%:%
%:%2047=631%:%
%:%2048=631%:%
%:%2049=632%:%
%:%2050=632%:%
%:%2051=632%:%
%:%2052=632%:%
%:%2053=632%:%
%:%2054=633%:%
%:%2055=633%:%
%:%2056=633%:%
%:%2057=633%:%
%:%2058=634%:%
%:%2064=634%:%
%:%2067=635%:%
%:%2068=636%:%
%:%2069=636%:%
%:%2070=637%:%
%:%2071=638%:%
%:%2072=639%:%
%:%2075=640%:%
%:%2079=640%:%
%:%2080=640%:%
%:%2081=641%:%
%:%2082=641%:%
%:%2087=641%:%
%:%2090=642%:%
%:%2091=643%:%
%:%2092=643%:%
%:%2093=644%:%
%:%2094=645%:%
%:%2095=646%:%
%:%2102=647%:%
%:%2103=647%:%
%:%2104=648%:%
%:%2105=648%:%
%:%2106=648%:%
%:%2107=649%:%
%:%2108=649%:%
%:%2109=649%:%
%:%2110=649%:%
%:%2111=650%:%
%:%2112=650%:%
%:%2113=650%:%
%:%2114=650%:%
%:%2115=651%:%
%:%2121=651%:%
%:%2124=652%:%
%:%2125=653%:%
%:%2126=653%:%
%:%2127=654%:%
%:%2128=655%:%
%:%2129=656%:%
%:%2136=657%:%
%:%2137=657%:%
%:%2138=658%:%
%:%2139=658%:%
%:%2140=658%:%
%:%2141=659%:%
%:%2142=659%:%
%:%2143=659%:%
%:%2144=659%:%
%:%2145=660%:%
%:%2146=660%:%
%:%2147=660%:%
%:%2148=660%:%
%:%2149=661%:%
%:%2155=661%:%
%:%2158=662%:%
%:%2159=663%:%
%:%2160=663%:%
%:%2161=664%:%
%:%2162=665%:%
%:%2163=666%:%
%:%2166=667%:%
%:%2170=667%:%
%:%2171=667%:%
%:%2172=667%:%
%:%2173=667%:%
%:%2178=667%:%
%:%2181=668%:%
%:%2182=669%:%
%:%2183=669%:%
%:%2184=670%:%
%:%2185=671%:%
%:%2186=672%:%
%:%2189=673%:%
%:%2193=673%:%
%:%2194=673%:%
%:%2195=673%:%
%:%2196=673%:%
%:%2201=673%:%
%:%2204=674%:%
%:%2205=675%:%
%:%2206=675%:%
%:%2209=676%:%
%:%2214=677%:%

%
\begin{isabellebody}%
\setisabellecontext{Stochastic{\isacharunderscore}{\kern0pt}Process}%
%
\isadelimtheory
%
\endisadelimtheory
%
\isatagtheory
\isacommand{theory}\isamarkupfalse%
\ Stochastic{\isacharunderscore}{\kern0pt}Process\isanewline
\isakeyword{imports}\ Filtration\isanewline
\isakeyword{begin}%
\endisatagtheory
{\isafoldtheory}%
%
\isadelimtheory
%
\endisadelimtheory
%
\isadelimdocument
%
\endisadelimdocument
%
\isatagdocument
%
\isamarkupsubsection{Stochastic Process%
}
\isamarkuptrue%
%
\endisatagdocument
{\isafolddocument}%
%
\isadelimdocument
%
\endisadelimdocument
\isacommand{locale}\isamarkupfalse%
\ stochastic{\isacharunderscore}{\kern0pt}process\ {\isacharequal}{\kern0pt}\ sigma{\isacharunderscore}{\kern0pt}finite{\isacharunderscore}{\kern0pt}measure\ M\ \isakeyword{for}\ M\ {\isacharplus}{\kern0pt}\isanewline
\ \ \isakeyword{fixes}\ X\ {\isacharcolon}{\kern0pt}{\isacharcolon}{\kern0pt}\ {\isachardoublequoteopen}{\isacharprime}{\kern0pt}t\ {\isacharcolon}{\kern0pt}{\isacharcolon}{\kern0pt}\ {\isacharbraceleft}{\kern0pt}second{\isacharunderscore}{\kern0pt}countable{\isacharunderscore}{\kern0pt}topology{\isacharcomma}{\kern0pt}linorder{\isacharunderscore}{\kern0pt}topology{\isacharbraceright}{\kern0pt}\ {\isasymRightarrow}\ {\isacharprime}{\kern0pt}a\ {\isasymRightarrow}\ {\isacharprime}{\kern0pt}b{\isacharcolon}{\kern0pt}{\isacharcolon}{\kern0pt}{\isacharbraceleft}{\kern0pt}real{\isacharunderscore}{\kern0pt}normed{\isacharunderscore}{\kern0pt}vector{\isacharcomma}{\kern0pt}\ second{\isacharunderscore}{\kern0pt}countable{\isacharunderscore}{\kern0pt}topology{\isacharbraceright}{\kern0pt}{\isachardoublequoteclose}\isanewline
\ \ \isakeyword{assumes}\ random{\isacharunderscore}{\kern0pt}variable{\isacharbrackleft}{\kern0pt}measurable{\isacharbrackright}{\kern0pt}{\isacharcolon}{\kern0pt}\ {\isachardoublequoteopen}{\isasymAnd}i{\isachardot}{\kern0pt}\ X\ i\ {\isasymin}\ borel{\isacharunderscore}{\kern0pt}measurable\ M{\isachardoublequoteclose}\isanewline
\isakeyword{begin}\isanewline
\isanewline
\isacommand{definition}\isamarkupfalse%
\ left{\isacharunderscore}{\kern0pt}continuous\ \isakeyword{where}\ {\isachardoublequoteopen}left{\isacharunderscore}{\kern0pt}continuous\ {\isacharequal}{\kern0pt}\ {\isacharparenleft}{\kern0pt}AE\ {\isasymxi}\ in\ M{\isachardot}{\kern0pt}\ {\isasymforall}i{\isachardot}{\kern0pt}\ continuous\ {\isacharparenleft}{\kern0pt}at{\isacharunderscore}{\kern0pt}left\ i{\isacharparenright}{\kern0pt}\ {\isacharparenleft}{\kern0pt}{\isasymlambda}i{\isachardot}{\kern0pt}\ X\ i\ {\isasymxi}{\isacharparenright}{\kern0pt}{\isacharparenright}{\kern0pt}{\isachardoublequoteclose}\isanewline
\isacommand{definition}\isamarkupfalse%
\ right{\isacharunderscore}{\kern0pt}continuous\ \isakeyword{where}\ {\isachardoublequoteopen}right{\isacharunderscore}{\kern0pt}continuous\ {\isacharequal}{\kern0pt}\ {\isacharparenleft}{\kern0pt}AE\ {\isasymxi}\ in\ M{\isachardot}{\kern0pt}\ {\isasymforall}i{\isachardot}{\kern0pt}\ continuous\ {\isacharparenleft}{\kern0pt}at{\isacharunderscore}{\kern0pt}right\ i{\isacharparenright}{\kern0pt}\ {\isacharparenleft}{\kern0pt}{\isasymlambda}i{\isachardot}{\kern0pt}\ X\ i\ {\isasymxi}{\isacharparenright}{\kern0pt}{\isacharparenright}{\kern0pt}{\isachardoublequoteclose}\isanewline
\isanewline
\isacommand{lemma}\isamarkupfalse%
\ compose{\isacharcolon}{\kern0pt}\isanewline
\ \ \isakeyword{assumes}\ {\isachardoublequoteopen}{\isasymAnd}i{\isachardot}{\kern0pt}\ f\ i\ {\isasymin}\ borel{\isacharunderscore}{\kern0pt}measurable\ borel{\isachardoublequoteclose}\isanewline
\ \ \isakeyword{shows}\ {\isachardoublequoteopen}stochastic{\isacharunderscore}{\kern0pt}process\ M\ {\isacharparenleft}{\kern0pt}{\isasymlambda}i\ {\isasymxi}{\isachardot}{\kern0pt}\ {\isacharparenleft}{\kern0pt}f\ i{\isacharparenright}{\kern0pt}\ {\isacharparenleft}{\kern0pt}X\ i\ {\isasymxi}{\isacharparenright}{\kern0pt}{\isacharparenright}{\kern0pt}{\isachardoublequoteclose}\isanewline
%
\isadelimproof
\ \ %
\endisadelimproof
%
\isatagproof
\isacommand{by}\isamarkupfalse%
\ {\isacharparenleft}{\kern0pt}unfold{\isacharunderscore}{\kern0pt}locales{\isacharcomma}{\kern0pt}\ intro\ measurable{\isacharunderscore}{\kern0pt}compose{\isacharbrackleft}{\kern0pt}OF\ random{\isacharunderscore}{\kern0pt}variable\ assms{\isacharbrackright}{\kern0pt}{\isacharparenright}{\kern0pt}%
\endisatagproof
{\isafoldproof}%
%
\isadelimproof
\ \isanewline
%
\endisadelimproof
\isanewline
\isacommand{lemma}\isamarkupfalse%
\ norm{\isacharcolon}{\kern0pt}\ {\isachardoublequoteopen}stochastic{\isacharunderscore}{\kern0pt}process\ M\ {\isacharparenleft}{\kern0pt}{\isasymlambda}i\ {\isasymxi}{\isachardot}{\kern0pt}\ norm\ {\isacharparenleft}{\kern0pt}X\ i\ {\isasymxi}{\isacharparenright}{\kern0pt}{\isacharparenright}{\kern0pt}{\isachardoublequoteclose}%
\isadelimproof
\ %
\endisadelimproof
%
\isatagproof
\isacommand{by}\isamarkupfalse%
\ {\isacharparenleft}{\kern0pt}auto\ intro{\isacharcolon}{\kern0pt}\ compose\ borel{\isacharunderscore}{\kern0pt}measurable{\isacharunderscore}{\kern0pt}norm{\isacharparenright}{\kern0pt}%
\endisatagproof
{\isafoldproof}%
%
\isadelimproof
%
\endisadelimproof
\isanewline
\isanewline
\isacommand{lemma}\isamarkupfalse%
\ scaleR{\isacharcolon}{\kern0pt}\isanewline
\ \ \isakeyword{assumes}\ {\isachardoublequoteopen}stochastic{\isacharunderscore}{\kern0pt}process\ M\ R{\isachardoublequoteclose}\isanewline
\ \ \isakeyword{shows}\ {\isachardoublequoteopen}stochastic{\isacharunderscore}{\kern0pt}process\ M\ {\isacharparenleft}{\kern0pt}{\isasymlambda}i\ {\isasymxi}{\isachardot}{\kern0pt}\ {\isacharparenleft}{\kern0pt}R\ i\ {\isasymxi}{\isacharparenright}{\kern0pt}\ {\isacharasterisk}{\kern0pt}\isactrlsub R\ {\isacharparenleft}{\kern0pt}X\ i\ {\isasymxi}{\isacharparenright}{\kern0pt}{\isacharparenright}{\kern0pt}{\isachardoublequoteclose}\isanewline
%
\isadelimproof
\ \ %
\endisadelimproof
%
\isatagproof
\isacommand{by}\isamarkupfalse%
\ {\isacharparenleft}{\kern0pt}unfold{\isacharunderscore}{\kern0pt}locales{\isacharparenright}{\kern0pt}\ {\isacharparenleft}{\kern0pt}simp\ add{\isacharcolon}{\kern0pt}\ borel{\isacharunderscore}{\kern0pt}measurable{\isacharunderscore}{\kern0pt}scaleR\ random{\isacharunderscore}{\kern0pt}variable\ assms\ stochastic{\isacharunderscore}{\kern0pt}process{\isachardot}{\kern0pt}random{\isacharunderscore}{\kern0pt}variable{\isacharparenright}{\kern0pt}%
\endisatagproof
{\isafoldproof}%
%
\isadelimproof
\isanewline
%
\endisadelimproof
\isanewline
\isacommand{lemma}\isamarkupfalse%
\ scaleR{\isacharunderscore}{\kern0pt}const{\isacharunderscore}{\kern0pt}fun{\isacharcolon}{\kern0pt}\ \isanewline
\ \ \isakeyword{assumes}\ {\isachardoublequoteopen}f\ {\isasymin}\ borel{\isacharunderscore}{\kern0pt}measurable\ M{\isachardoublequoteclose}\ \isanewline
\ \ \isakeyword{shows}\ {\isachardoublequoteopen}stochastic{\isacharunderscore}{\kern0pt}process\ M\ {\isacharparenleft}{\kern0pt}{\isasymlambda}i\ {\isasymxi}{\isachardot}{\kern0pt}\ f\ {\isasymxi}\ {\isacharasterisk}{\kern0pt}\isactrlsub R\ {\isacharparenleft}{\kern0pt}X\ i\ {\isasymxi}{\isacharparenright}{\kern0pt}{\isacharparenright}{\kern0pt}{\isachardoublequoteclose}\isanewline
%
\isadelimproof
\ \ %
\endisadelimproof
%
\isatagproof
\isacommand{by}\isamarkupfalse%
\ {\isacharparenleft}{\kern0pt}unfold{\isacharunderscore}{\kern0pt}locales{\isacharcomma}{\kern0pt}\ intro\ borel{\isacharunderscore}{\kern0pt}measurable{\isacharunderscore}{\kern0pt}scaleR\ assms\ random{\isacharunderscore}{\kern0pt}variable{\isacharparenright}{\kern0pt}%
\endisatagproof
{\isafoldproof}%
%
\isadelimproof
\isanewline
%
\endisadelimproof
\isanewline
\isacommand{lemma}\isamarkupfalse%
\ scaleR{\isacharunderscore}{\kern0pt}const{\isacharcolon}{\kern0pt}\ {\isachardoublequoteopen}stochastic{\isacharunderscore}{\kern0pt}process\ M\ {\isacharparenleft}{\kern0pt}{\isasymlambda}i\ {\isasymxi}{\isachardot}{\kern0pt}\ c\ {\isacharasterisk}{\kern0pt}\isactrlsub R\ {\isacharparenleft}{\kern0pt}X\ i\ {\isasymxi}{\isacharparenright}{\kern0pt}{\isacharparenright}{\kern0pt}{\isachardoublequoteclose}%
\isadelimproof
\ %
\endisadelimproof
%
\isatagproof
\isacommand{by}\isamarkupfalse%
\ {\isacharparenleft}{\kern0pt}auto\ intro{\isacharcolon}{\kern0pt}\ scaleR{\isacharunderscore}{\kern0pt}const{\isacharunderscore}{\kern0pt}fun\ borel{\isacharunderscore}{\kern0pt}measurable{\isacharunderscore}{\kern0pt}const{\isacharparenright}{\kern0pt}%
\endisatagproof
{\isafoldproof}%
%
\isadelimproof
%
\endisadelimproof
\isanewline
\isanewline
\isacommand{lemma}\isamarkupfalse%
\ add{\isacharcolon}{\kern0pt}\isanewline
\ \ \isakeyword{assumes}\ {\isachardoublequoteopen}stochastic{\isacharunderscore}{\kern0pt}process\ M\ Y{\isachardoublequoteclose}\isanewline
\ \ \isakeyword{shows}\ {\isachardoublequoteopen}stochastic{\isacharunderscore}{\kern0pt}process\ M\ {\isacharparenleft}{\kern0pt}{\isasymlambda}i\ {\isasymxi}{\isachardot}{\kern0pt}\ X\ i\ {\isasymxi}\ {\isacharplus}{\kern0pt}\ Y\ i\ {\isasymxi}{\isacharparenright}{\kern0pt}{\isachardoublequoteclose}\isanewline
%
\isadelimproof
\ \ %
\endisadelimproof
%
\isatagproof
\isacommand{by}\isamarkupfalse%
\ {\isacharparenleft}{\kern0pt}unfold{\isacharunderscore}{\kern0pt}locales{\isacharparenright}{\kern0pt}\ {\isacharparenleft}{\kern0pt}simp\ add{\isacharcolon}{\kern0pt}\ borel{\isacharunderscore}{\kern0pt}measurable{\isacharunderscore}{\kern0pt}add\ random{\isacharunderscore}{\kern0pt}variable\ assms\ stochastic{\isacharunderscore}{\kern0pt}process{\isachardot}{\kern0pt}random{\isacharunderscore}{\kern0pt}variable{\isacharparenright}{\kern0pt}%
\endisatagproof
{\isafoldproof}%
%
\isadelimproof
\isanewline
%
\endisadelimproof
\isanewline
\isacommand{lemma}\isamarkupfalse%
\ diff{\isacharcolon}{\kern0pt}\isanewline
\ \ \isakeyword{assumes}\ {\isachardoublequoteopen}stochastic{\isacharunderscore}{\kern0pt}process\ M\ Y{\isachardoublequoteclose}\isanewline
\ \ \isakeyword{shows}\ {\isachardoublequoteopen}stochastic{\isacharunderscore}{\kern0pt}process\ M\ {\isacharparenleft}{\kern0pt}{\isasymlambda}i\ {\isasymxi}{\isachardot}{\kern0pt}\ X\ i\ {\isasymxi}\ {\isacharminus}{\kern0pt}\ Y\ i\ {\isasymxi}{\isacharparenright}{\kern0pt}{\isachardoublequoteclose}\isanewline
%
\isadelimproof
\ \ %
\endisadelimproof
%
\isatagproof
\isacommand{by}\isamarkupfalse%
\ {\isacharparenleft}{\kern0pt}unfold{\isacharunderscore}{\kern0pt}locales{\isacharparenright}{\kern0pt}\ {\isacharparenleft}{\kern0pt}simp\ add{\isacharcolon}{\kern0pt}\ borel{\isacharunderscore}{\kern0pt}measurable{\isacharunderscore}{\kern0pt}diff\ random{\isacharunderscore}{\kern0pt}variable\ assms\ stochastic{\isacharunderscore}{\kern0pt}process{\isachardot}{\kern0pt}random{\isacharunderscore}{\kern0pt}variable{\isacharparenright}{\kern0pt}%
\endisatagproof
{\isafoldproof}%
%
\isadelimproof
\isanewline
%
\endisadelimproof
\isanewline
\isacommand{lemma}\isamarkupfalse%
\ uminus{\isacharcolon}{\kern0pt}\ {\isachardoublequoteopen}stochastic{\isacharunderscore}{\kern0pt}process\ M\ {\isacharparenleft}{\kern0pt}{\isacharminus}{\kern0pt}X{\isacharparenright}{\kern0pt}{\isachardoublequoteclose}%
\isadelimproof
\ %
\endisadelimproof
%
\isatagproof
\isacommand{using}\isamarkupfalse%
\ scaleR{\isacharunderscore}{\kern0pt}const{\isacharbrackleft}{\kern0pt}of\ {\isachardoublequoteopen}{\isacharminus}{\kern0pt}{\isadigit{1}}{\isachardoublequoteclose}{\isacharbrackright}{\kern0pt}\ \isacommand{by}\isamarkupfalse%
\ {\isacharparenleft}{\kern0pt}simp\ add{\isacharcolon}{\kern0pt}\ fun{\isacharunderscore}{\kern0pt}Compl{\isacharunderscore}{\kern0pt}def{\isacharparenright}{\kern0pt}%
\endisatagproof
{\isafoldproof}%
%
\isadelimproof
%
\endisadelimproof
\isanewline
\isanewline
\isacommand{end}\isamarkupfalse%
%
\isadelimdocument
%
\endisadelimdocument
%
\isatagdocument
%
\isamarkupsubsection{Adapted Process%
}
\isamarkuptrue%
%
\endisatagdocument
{\isafolddocument}%
%
\isadelimdocument
%
\endisadelimdocument
\isacommand{locale}\isamarkupfalse%
\ adapted{\isacharunderscore}{\kern0pt}process\ {\isacharequal}{\kern0pt}\ filtered{\isacharunderscore}{\kern0pt}sigma{\isacharunderscore}{\kern0pt}finite{\isacharunderscore}{\kern0pt}measure\ M\ F\ {\isacharplus}{\kern0pt}\ stochastic{\isacharunderscore}{\kern0pt}process\ M\ X\ \isakeyword{for}\ M\ \isakeyword{and}\ F\ {\isacharcolon}{\kern0pt}{\isacharcolon}{\kern0pt}\ {\isachardoublequoteopen}{\isacharprime}{\kern0pt}t\ {\isacharcolon}{\kern0pt}{\isacharcolon}{\kern0pt}\ {\isacharbraceleft}{\kern0pt}second{\isacharunderscore}{\kern0pt}countable{\isacharunderscore}{\kern0pt}topology{\isacharcomma}{\kern0pt}\ linorder{\isacharunderscore}{\kern0pt}topology{\isacharcomma}{\kern0pt}\ order{\isacharunderscore}{\kern0pt}bot{\isacharbraceright}{\kern0pt}\ {\isasymRightarrow}\ {\isacharunderscore}{\kern0pt}{\isachardoublequoteclose}\ \isakeyword{and}\ X\ {\isacharcolon}{\kern0pt}{\isacharcolon}{\kern0pt}\ {\isachardoublequoteopen}{\isacharprime}{\kern0pt}t\ {\isasymRightarrow}\ {\isacharunderscore}{\kern0pt}\ {\isasymRightarrow}\ {\isacharunderscore}{\kern0pt}\ {\isacharcolon}{\kern0pt}{\isacharcolon}{\kern0pt}\ {\isacharbraceleft}{\kern0pt}second{\isacharunderscore}{\kern0pt}countable{\isacharunderscore}{\kern0pt}topology{\isacharcomma}{\kern0pt}\ banach{\isacharbraceright}{\kern0pt}{\isachardoublequoteclose}\ {\isacharplus}{\kern0pt}\isanewline
\ \ \isakeyword{assumes}\ adapted{\isacharbrackleft}{\kern0pt}measurable{\isacharbrackright}{\kern0pt}{\isacharcolon}{\kern0pt}\ {\isachardoublequoteopen}{\isasymAnd}i{\isachardot}{\kern0pt}\ X\ i\ {\isasymin}\ borel{\isacharunderscore}{\kern0pt}measurable\ {\isacharparenleft}{\kern0pt}F\ i{\isacharparenright}{\kern0pt}{\isachardoublequoteclose}\isanewline
\isakeyword{begin}\isanewline
\isanewline
\isacommand{lemma}\isamarkupfalse%
\ const{\isacharunderscore}{\kern0pt}fun{\isacharcolon}{\kern0pt}\isanewline
\ \ \isakeyword{assumes}\ {\isachardoublequoteopen}f\ {\isasymin}\ borel{\isacharunderscore}{\kern0pt}measurable\ {\isacharparenleft}{\kern0pt}F\ bot{\isacharparenright}{\kern0pt}{\isachardoublequoteclose}\isanewline
\ \ \isakeyword{shows}\ {\isachardoublequoteopen}adapted{\isacharunderscore}{\kern0pt}process\ M\ F\ {\isacharparenleft}{\kern0pt}{\isasymlambda}{\isacharunderscore}{\kern0pt}{\isachardot}{\kern0pt}\ f{\isacharparenright}{\kern0pt}{\isachardoublequoteclose}\isanewline
%
\isadelimproof
\ \ %
\endisadelimproof
%
\isatagproof
\isacommand{using}\isamarkupfalse%
\ assms\ \isacommand{by}\isamarkupfalse%
\ {\isacharparenleft}{\kern0pt}unfold{\isacharunderscore}{\kern0pt}locales{\isacharparenright}{\kern0pt}\ {\isacharparenleft}{\kern0pt}blast\ intro{\isacharcolon}{\kern0pt}\ measurable{\isacharunderscore}{\kern0pt}from{\isacharunderscore}{\kern0pt}subalg\ subalgebra{\isacharcomma}{\kern0pt}\ metis\ borel{\isacharunderscore}{\kern0pt}measurable{\isacharunderscore}{\kern0pt}subalgebra\ bot{\isachardot}{\kern0pt}extremum\ sets{\isacharunderscore}{\kern0pt}F{\isacharunderscore}{\kern0pt}mono\ space{\isacharunderscore}{\kern0pt}F{\isacharparenright}{\kern0pt}%
\endisatagproof
{\isafoldproof}%
%
\isadelimproof
\isanewline
%
\endisadelimproof
\isanewline
\isacommand{lemma}\isamarkupfalse%
\ compose{\isacharcolon}{\kern0pt}\isanewline
\ \ \isakeyword{assumes}\ {\isachardoublequoteopen}{\isasymAnd}i{\isachardot}{\kern0pt}\ f\ i\ {\isasymin}\ borel{\isacharunderscore}{\kern0pt}measurable\ borel{\isachardoublequoteclose}\isanewline
\ \ \isakeyword{shows}\ {\isachardoublequoteopen}adapted{\isacharunderscore}{\kern0pt}process\ M\ F\ {\isacharparenleft}{\kern0pt}{\isasymlambda}i\ {\isasymxi}{\isachardot}{\kern0pt}\ {\isacharparenleft}{\kern0pt}f\ i{\isacharparenright}{\kern0pt}\ {\isacharparenleft}{\kern0pt}X\ i\ {\isasymxi}{\isacharparenright}{\kern0pt}{\isacharparenright}{\kern0pt}{\isachardoublequoteclose}\isanewline
%
\isadelimproof
\ \ %
\endisadelimproof
%
\isatagproof
\isacommand{by}\isamarkupfalse%
\ {\isacharparenleft}{\kern0pt}unfold{\isacharunderscore}{\kern0pt}locales{\isacharcomma}{\kern0pt}\ intro\ measurable{\isacharunderscore}{\kern0pt}compose{\isacharbrackleft}{\kern0pt}OF\ random{\isacharunderscore}{\kern0pt}variable\ assms{\isacharbrackright}{\kern0pt}{\isacharcomma}{\kern0pt}\ intro\ measurable{\isacharunderscore}{\kern0pt}compose{\isacharbrackleft}{\kern0pt}OF\ adapted\ assms{\isacharbrackright}{\kern0pt}{\isacharparenright}{\kern0pt}%
\endisatagproof
{\isafoldproof}%
%
\isadelimproof
\isanewline
%
\endisadelimproof
\isanewline
\isacommand{lemma}\isamarkupfalse%
\ norm{\isacharcolon}{\kern0pt}\ {\isachardoublequoteopen}adapted{\isacharunderscore}{\kern0pt}process\ M\ F\ {\isacharparenleft}{\kern0pt}{\isasymlambda}i\ {\isasymxi}{\isachardot}{\kern0pt}\ norm\ {\isacharparenleft}{\kern0pt}X\ i\ {\isasymxi}{\isacharparenright}{\kern0pt}{\isacharparenright}{\kern0pt}{\isachardoublequoteclose}%
\isadelimproof
\ %
\endisadelimproof
%
\isatagproof
\isacommand{by}\isamarkupfalse%
\ {\isacharparenleft}{\kern0pt}auto\ intro{\isacharcolon}{\kern0pt}\ compose\ borel{\isacharunderscore}{\kern0pt}measurable{\isacharunderscore}{\kern0pt}norm{\isacharparenright}{\kern0pt}%
\endisatagproof
{\isafoldproof}%
%
\isadelimproof
%
\endisadelimproof
\isanewline
\isanewline
\isacommand{lemma}\isamarkupfalse%
\ scaleR{\isacharcolon}{\kern0pt}\isanewline
\ \ \isakeyword{assumes}\ {\isachardoublequoteopen}adapted{\isacharunderscore}{\kern0pt}process\ M\ F\ R{\isachardoublequoteclose}\isanewline
\ \ \isakeyword{shows}\ {\isachardoublequoteopen}adapted{\isacharunderscore}{\kern0pt}process\ M\ F\ {\isacharparenleft}{\kern0pt}{\isasymlambda}i\ {\isasymxi}{\isachardot}{\kern0pt}\ {\isacharparenleft}{\kern0pt}R\ i\ {\isasymxi}{\isacharparenright}{\kern0pt}\ {\isacharasterisk}{\kern0pt}\isactrlsub R\ {\isacharparenleft}{\kern0pt}X\ i\ {\isasymxi}{\isacharparenright}{\kern0pt}{\isacharparenright}{\kern0pt}{\isachardoublequoteclose}\isanewline
%
\isadelimproof
%
\endisadelimproof
%
\isatagproof
\isacommand{proof}\isamarkupfalse%
\ {\isacharminus}{\kern0pt}\isanewline
\ \ \isacommand{interpret}\isamarkupfalse%
\ R{\isacharcolon}{\kern0pt}\ adapted{\isacharunderscore}{\kern0pt}process\ M\ F\ R\ \isacommand{by}\isamarkupfalse%
\ {\isacharparenleft}{\kern0pt}rule\ assms{\isacharparenright}{\kern0pt}\isanewline
\ \ \isacommand{show}\isamarkupfalse%
\ {\isacharquery}{\kern0pt}thesis\ \isacommand{by}\isamarkupfalse%
\ {\isacharparenleft}{\kern0pt}unfold{\isacharunderscore}{\kern0pt}locales{\isacharparenright}{\kern0pt}\ {\isacharparenleft}{\kern0pt}auto\ simp\ add{\isacharcolon}{\kern0pt}\ borel{\isacharunderscore}{\kern0pt}measurable{\isacharunderscore}{\kern0pt}scaleR\ adapted\ random{\isacharunderscore}{\kern0pt}variable\ assms\ R{\isachardot}{\kern0pt}random{\isacharunderscore}{\kern0pt}variable\ R{\isachardot}{\kern0pt}adapted{\isacharparenright}{\kern0pt}\isanewline
\isacommand{qed}\isamarkupfalse%
%
\endisatagproof
{\isafoldproof}%
%
\isadelimproof
\isanewline
%
\endisadelimproof
\ \ \isanewline
\isacommand{lemma}\isamarkupfalse%
\ scaleR{\isacharunderscore}{\kern0pt}const{\isacharunderscore}{\kern0pt}fun{\isacharcolon}{\kern0pt}\ \isanewline
\ \ \isakeyword{assumes}\ {\isachardoublequoteopen}f\ {\isasymin}\ borel{\isacharunderscore}{\kern0pt}measurable\ {\isacharparenleft}{\kern0pt}F\ bot{\isacharparenright}{\kern0pt}{\isachardoublequoteclose}\ \isanewline
\ \ \isakeyword{shows}\ {\isachardoublequoteopen}adapted{\isacharunderscore}{\kern0pt}process\ M\ F\ {\isacharparenleft}{\kern0pt}{\isasymlambda}i\ {\isasymxi}{\isachardot}{\kern0pt}\ f\ {\isasymxi}\ {\isacharasterisk}{\kern0pt}\isactrlsub R\ {\isacharparenleft}{\kern0pt}X\ i\ {\isasymxi}{\isacharparenright}{\kern0pt}{\isacharparenright}{\kern0pt}{\isachardoublequoteclose}\isanewline
%
\isadelimproof
\ \ %
\endisadelimproof
%
\isatagproof
\isacommand{using}\isamarkupfalse%
\ assms\ \isacommand{by}\isamarkupfalse%
\ {\isacharparenleft}{\kern0pt}fast\ intro{\isacharcolon}{\kern0pt}\ scaleR\ const{\isacharunderscore}{\kern0pt}fun{\isacharparenright}{\kern0pt}%
\endisatagproof
{\isafoldproof}%
%
\isadelimproof
\isanewline
%
\endisadelimproof
\isanewline
\isacommand{lemma}\isamarkupfalse%
\ scaleR{\isacharunderscore}{\kern0pt}const{\isacharcolon}{\kern0pt}\ {\isachardoublequoteopen}adapted{\isacharunderscore}{\kern0pt}process\ M\ F\ {\isacharparenleft}{\kern0pt}{\isasymlambda}i\ {\isasymxi}{\isachardot}{\kern0pt}\ c\ {\isacharasterisk}{\kern0pt}\isactrlsub R\ {\isacharparenleft}{\kern0pt}X\ i\ {\isasymxi}{\isacharparenright}{\kern0pt}{\isacharparenright}{\kern0pt}{\isachardoublequoteclose}%
\isadelimproof
\ %
\endisadelimproof
%
\isatagproof
\isacommand{by}\isamarkupfalse%
\ {\isacharparenleft}{\kern0pt}auto\ intro{\isacharcolon}{\kern0pt}\ scaleR{\isacharunderscore}{\kern0pt}const{\isacharunderscore}{\kern0pt}fun\ borel{\isacharunderscore}{\kern0pt}measurable{\isacharunderscore}{\kern0pt}const{\isacharparenright}{\kern0pt}%
\endisatagproof
{\isafoldproof}%
%
\isadelimproof
%
\endisadelimproof
\isanewline
\isanewline
\isacommand{lemma}\isamarkupfalse%
\ add{\isacharcolon}{\kern0pt}\isanewline
\ \ \isakeyword{assumes}\ {\isachardoublequoteopen}adapted{\isacharunderscore}{\kern0pt}process\ M\ F\ Y{\isachardoublequoteclose}\isanewline
\ \ \isakeyword{shows}\ {\isachardoublequoteopen}adapted{\isacharunderscore}{\kern0pt}process\ M\ F\ {\isacharparenleft}{\kern0pt}{\isasymlambda}i\ {\isasymxi}{\isachardot}{\kern0pt}\ X\ i\ {\isasymxi}\ {\isacharplus}{\kern0pt}\ Y\ i\ {\isasymxi}{\isacharparenright}{\kern0pt}{\isachardoublequoteclose}\isanewline
%
\isadelimproof
%
\endisadelimproof
%
\isatagproof
\isacommand{proof}\isamarkupfalse%
\ {\isacharminus}{\kern0pt}\isanewline
\ \ \isacommand{interpret}\isamarkupfalse%
\ Y{\isacharcolon}{\kern0pt}\ adapted{\isacharunderscore}{\kern0pt}process\ M\ F\ Y\ \isacommand{by}\isamarkupfalse%
\ {\isacharparenleft}{\kern0pt}rule\ assms{\isacharparenright}{\kern0pt}\isanewline
\ \ \isacommand{show}\isamarkupfalse%
\ {\isacharquery}{\kern0pt}thesis\ \isacommand{by}\isamarkupfalse%
\ {\isacharparenleft}{\kern0pt}unfold{\isacharunderscore}{\kern0pt}locales{\isacharparenright}{\kern0pt}\ {\isacharparenleft}{\kern0pt}auto\ simp\ add{\isacharcolon}{\kern0pt}\ borel{\isacharunderscore}{\kern0pt}measurable{\isacharunderscore}{\kern0pt}add\ adapted\ random{\isacharunderscore}{\kern0pt}variable\ Y{\isachardot}{\kern0pt}random{\isacharunderscore}{\kern0pt}variable\ Y{\isachardot}{\kern0pt}adapted{\isacharparenright}{\kern0pt}\isanewline
\isacommand{qed}\isamarkupfalse%
%
\endisatagproof
{\isafoldproof}%
%
\isadelimproof
\isanewline
%
\endisadelimproof
\isanewline
\isacommand{lemma}\isamarkupfalse%
\ diff{\isacharcolon}{\kern0pt}\isanewline
\ \ \isakeyword{assumes}\ {\isachardoublequoteopen}adapted{\isacharunderscore}{\kern0pt}process\ M\ F\ Y{\isachardoublequoteclose}\isanewline
\ \ \isakeyword{shows}\ {\isachardoublequoteopen}adapted{\isacharunderscore}{\kern0pt}process\ M\ F\ {\isacharparenleft}{\kern0pt}{\isasymlambda}i\ {\isasymxi}{\isachardot}{\kern0pt}\ X\ i\ {\isasymxi}\ {\isacharminus}{\kern0pt}\ Y\ i\ {\isasymxi}{\isacharparenright}{\kern0pt}{\isachardoublequoteclose}\isanewline
%
\isadelimproof
%
\endisadelimproof
%
\isatagproof
\isacommand{proof}\isamarkupfalse%
\ {\isacharminus}{\kern0pt}\isanewline
\ \ \isacommand{interpret}\isamarkupfalse%
\ Y{\isacharcolon}{\kern0pt}\ adapted{\isacharunderscore}{\kern0pt}process\ M\ F\ Y\ \isacommand{by}\isamarkupfalse%
\ {\isacharparenleft}{\kern0pt}rule\ assms{\isacharparenright}{\kern0pt}\isanewline
\ \ \isacommand{show}\isamarkupfalse%
\ {\isacharquery}{\kern0pt}thesis\ \isacommand{by}\isamarkupfalse%
\ {\isacharparenleft}{\kern0pt}unfold{\isacharunderscore}{\kern0pt}locales{\isacharparenright}{\kern0pt}\ {\isacharparenleft}{\kern0pt}auto\ simp\ add{\isacharcolon}{\kern0pt}\ borel{\isacharunderscore}{\kern0pt}measurable{\isacharunderscore}{\kern0pt}diff\ adapted\ random{\isacharunderscore}{\kern0pt}variable\ Y{\isachardot}{\kern0pt}random{\isacharunderscore}{\kern0pt}variable\ Y{\isachardot}{\kern0pt}adapted{\isacharparenright}{\kern0pt}\isanewline
\isacommand{qed}\isamarkupfalse%
%
\endisatagproof
{\isafoldproof}%
%
\isadelimproof
\isanewline
%
\endisadelimproof
\isanewline
\isacommand{lemma}\isamarkupfalse%
\ uminus{\isacharcolon}{\kern0pt}\ {\isachardoublequoteopen}adapted{\isacharunderscore}{\kern0pt}process\ M\ F\ {\isacharparenleft}{\kern0pt}{\isacharminus}{\kern0pt}X{\isacharparenright}{\kern0pt}{\isachardoublequoteclose}%
\isadelimproof
\ %
\endisadelimproof
%
\isatagproof
\isacommand{using}\isamarkupfalse%
\ scaleR{\isacharunderscore}{\kern0pt}const{\isacharbrackleft}{\kern0pt}of\ {\isachardoublequoteopen}{\isacharminus}{\kern0pt}{\isadigit{1}}{\isachardoublequoteclose}{\isacharbrackright}{\kern0pt}\ \isacommand{by}\isamarkupfalse%
\ {\isacharparenleft}{\kern0pt}simp\ add{\isacharcolon}{\kern0pt}\ fun{\isacharunderscore}{\kern0pt}Compl{\isacharunderscore}{\kern0pt}def{\isacharparenright}{\kern0pt}%
\endisatagproof
{\isafoldproof}%
%
\isadelimproof
%
\endisadelimproof
\isanewline
\isanewline
\isacommand{end}\isamarkupfalse%
\isanewline
\isanewline
\isacommand{locale}\isamarkupfalse%
\ adapted{\isacharunderscore}{\kern0pt}process{\isacharunderscore}{\kern0pt}order\ {\isacharequal}{\kern0pt}\ adapted{\isacharunderscore}{\kern0pt}process\ M\ F\ X\ \isakeyword{for}\ M\ F\ \isakeyword{and}\ X\ {\isacharcolon}{\kern0pt}{\isacharcolon}{\kern0pt}\ {\isachardoublequoteopen}{\isacharprime}{\kern0pt}t\ {\isacharcolon}{\kern0pt}{\isacharcolon}{\kern0pt}\ {\isacharbraceleft}{\kern0pt}second{\isacharunderscore}{\kern0pt}countable{\isacharunderscore}{\kern0pt}topology{\isacharcomma}{\kern0pt}\ linorder{\isacharunderscore}{\kern0pt}topology{\isacharcomma}{\kern0pt}\ order{\isacharunderscore}{\kern0pt}bot{\isacharbraceright}{\kern0pt}\ {\isasymRightarrow}\ {\isacharunderscore}{\kern0pt}\ {\isasymRightarrow}\ {\isacharunderscore}{\kern0pt}\ {\isacharcolon}{\kern0pt}{\isacharcolon}{\kern0pt}\ {\isacharbraceleft}{\kern0pt}linorder{\isacharunderscore}{\kern0pt}topology{\isacharcomma}{\kern0pt}\ ordered{\isacharunderscore}{\kern0pt}real{\isacharunderscore}{\kern0pt}vector{\isacharbraceright}{\kern0pt}{\isachardoublequoteclose}%
\isadelimdocument
%
\endisadelimdocument
%
\isatagdocument
%
\isamarkupsubsection{Discrete-Time Processes%
}
\isamarkuptrue%
%
\endisatagdocument
{\isafolddocument}%
%
\isadelimdocument
%
\endisadelimdocument
\isacommand{locale}\isamarkupfalse%
\ discrete{\isacharunderscore}{\kern0pt}time{\isacharunderscore}{\kern0pt}stochastic{\isacharunderscore}{\kern0pt}process\ {\isacharequal}{\kern0pt}\ stochastic{\isacharunderscore}{\kern0pt}process\ M\ X\ \isakeyword{for}\ M\ \isakeyword{and}\ X\ {\isacharcolon}{\kern0pt}{\isacharcolon}{\kern0pt}\ {\isachardoublequoteopen}nat\ {\isasymRightarrow}\ {\isacharunderscore}{\kern0pt}\ {\isasymRightarrow}\ {\isacharunderscore}{\kern0pt}{\isachardoublequoteclose}\isanewline
\isacommand{locale}\isamarkupfalse%
\ discrete{\isacharunderscore}{\kern0pt}time{\isacharunderscore}{\kern0pt}adapted{\isacharunderscore}{\kern0pt}process\ {\isacharequal}{\kern0pt}\ adapted{\isacharunderscore}{\kern0pt}process\ M\ F\ X\ \isakeyword{for}\ M\ F\ \isakeyword{and}\ X\ {\isacharcolon}{\kern0pt}{\isacharcolon}{\kern0pt}\ {\isachardoublequoteopen}nat\ {\isasymRightarrow}\ {\isacharunderscore}{\kern0pt}\ {\isasymRightarrow}\ {\isacharunderscore}{\kern0pt}{\isachardoublequoteclose}\isanewline
\isacommand{locale}\isamarkupfalse%
\ discrete{\isacharunderscore}{\kern0pt}time{\isacharunderscore}{\kern0pt}adapted{\isacharunderscore}{\kern0pt}process{\isacharunderscore}{\kern0pt}order\ {\isacharequal}{\kern0pt}\ adapted{\isacharunderscore}{\kern0pt}process{\isacharunderscore}{\kern0pt}order\ M\ F\ X\ \isakeyword{for}\ M\ F\ \isakeyword{and}\ X\ {\isacharcolon}{\kern0pt}{\isacharcolon}{\kern0pt}\ {\isachardoublequoteopen}nat\ {\isasymRightarrow}\ {\isacharunderscore}{\kern0pt}\ {\isasymRightarrow}\ {\isacharunderscore}{\kern0pt}{\isachardoublequoteclose}\isanewline
\isanewline
\isacommand{sublocale}\isamarkupfalse%
\ discrete{\isacharunderscore}{\kern0pt}time{\isacharunderscore}{\kern0pt}adapted{\isacharunderscore}{\kern0pt}process{\isacharunderscore}{\kern0pt}order\ {\isasymsubseteq}\ discrete{\isacharunderscore}{\kern0pt}time{\isacharunderscore}{\kern0pt}adapted{\isacharunderscore}{\kern0pt}process%
\isadelimproof
\ %
\endisadelimproof
%
\isatagproof
\isacommand{by}\isamarkupfalse%
\ {\isacharparenleft}{\kern0pt}unfold{\isacharunderscore}{\kern0pt}locales{\isacharparenright}{\kern0pt}%
\endisatagproof
{\isafoldproof}%
%
\isadelimproof
%
\endisadelimproof
\isanewline
\isacommand{sublocale}\isamarkupfalse%
\ discrete{\isacharunderscore}{\kern0pt}time{\isacharunderscore}{\kern0pt}adapted{\isacharunderscore}{\kern0pt}process\ {\isasymsubseteq}\ discrete{\isacharunderscore}{\kern0pt}time{\isacharunderscore}{\kern0pt}stochastic{\isacharunderscore}{\kern0pt}process%
\isadelimproof
\ %
\endisadelimproof
%
\isatagproof
\isacommand{by}\isamarkupfalse%
\ {\isacharparenleft}{\kern0pt}unfold{\isacharunderscore}{\kern0pt}locales{\isacharparenright}{\kern0pt}%
\endisatagproof
{\isafoldproof}%
%
\isadelimproof
%
\endisadelimproof
\isanewline
\isacommand{sublocale}\isamarkupfalse%
\ discrete{\isacharunderscore}{\kern0pt}time{\isacharunderscore}{\kern0pt}adapted{\isacharunderscore}{\kern0pt}process\ {\isasymsubseteq}\ nat{\isacharunderscore}{\kern0pt}filtered{\isacharunderscore}{\kern0pt}sigma{\isacharunderscore}{\kern0pt}finite{\isacharunderscore}{\kern0pt}measure%
\isadelimproof
\ %
\endisadelimproof
%
\isatagproof
\isacommand{by}\isamarkupfalse%
\ {\isacharparenleft}{\kern0pt}unfold{\isacharunderscore}{\kern0pt}locales{\isacharparenright}{\kern0pt}%
\endisatagproof
{\isafoldproof}%
%
\isadelimproof
%
\endisadelimproof
\isanewline
\isanewline
\isacommand{context}\isamarkupfalse%
\ filtered{\isacharunderscore}{\kern0pt}sigma{\isacharunderscore}{\kern0pt}finite{\isacharunderscore}{\kern0pt}measure\isanewline
\isakeyword{begin}\isanewline
\isanewline
\isacommand{definition}\isamarkupfalse%
\ predictable{\isacharunderscore}{\kern0pt}sigma\ {\isacharcolon}{\kern0pt}{\isacharcolon}{\kern0pt}\ {\isachardoublequoteopen}{\isacharparenleft}{\kern0pt}{\isacharprime}{\kern0pt}t\ {\isasymtimes}\ {\isacharprime}{\kern0pt}a{\isacharparenright}{\kern0pt}\ measure{\isachardoublequoteclose}\ \isakeyword{where}\isanewline
\ \ {\isachardoublequoteopen}predictable{\isacharunderscore}{\kern0pt}sigma\ {\isacharequal}{\kern0pt}\ sigma\ {\isacharparenleft}{\kern0pt}UNIV\ {\isasymtimes}\ space\ M{\isacharparenright}{\kern0pt}\ {\isacharparenleft}{\kern0pt}{\isacharbraceleft}{\kern0pt}{\isacharbraceleft}{\kern0pt}s{\isacharless}{\kern0pt}{\isachardot}{\kern0pt}{\isachardot}{\kern0pt}t{\isacharbraceright}{\kern0pt}\ {\isasymtimes}\ A\ {\isacharbar}{\kern0pt}\ A\ s\ t{\isachardot}{\kern0pt}\ A\ {\isasymin}\ F\ s\ {\isasymand}\ s\ {\isacharless}{\kern0pt}\ t{\isacharbraceright}{\kern0pt}\ {\isasymunion}\ {\isacharbraceleft}{\kern0pt}{\isacharbraceleft}{\kern0pt}bot{\isacharbraceright}{\kern0pt}\ {\isasymtimes}\ A\ {\isacharbar}{\kern0pt}\ A{\isachardot}{\kern0pt}\ A\ {\isasymin}\ F\ bot{\isacharbraceright}{\kern0pt}{\isacharparenright}{\kern0pt}{\isachardoublequoteclose}\isanewline
\isanewline
\isacommand{lemma}\isamarkupfalse%
\ space{\isacharunderscore}{\kern0pt}predictable{\isacharunderscore}{\kern0pt}sigma{\isacharbrackleft}{\kern0pt}simp{\isacharbrackright}{\kern0pt}{\isacharcolon}{\kern0pt}\ {\isachardoublequoteopen}space\ predictable{\isacharunderscore}{\kern0pt}sigma\ {\isacharequal}{\kern0pt}\ {\isacharparenleft}{\kern0pt}UNIV\ {\isasymtimes}\ space\ M{\isacharparenright}{\kern0pt}{\isachardoublequoteclose}%
\isadelimproof
\ %
\endisadelimproof
%
\isatagproof
\isacommand{unfolding}\isamarkupfalse%
\ predictable{\isacharunderscore}{\kern0pt}sigma{\isacharunderscore}{\kern0pt}def\ space{\isacharunderscore}{\kern0pt}measure{\isacharunderscore}{\kern0pt}of{\isacharunderscore}{\kern0pt}conv\ \isacommand{by}\isamarkupfalse%
\ blast%
\endisatagproof
{\isafoldproof}%
%
\isadelimproof
%
\endisadelimproof
\isanewline
\isanewline
\isacommand{lemma}\isamarkupfalse%
\ sets{\isacharunderscore}{\kern0pt}predictable{\isacharunderscore}{\kern0pt}sigma{\isacharbrackleft}{\kern0pt}simp{\isacharbrackright}{\kern0pt}{\isacharcolon}{\kern0pt}\ {\isachardoublequoteopen}sets\ predictable{\isacharunderscore}{\kern0pt}sigma\ {\isacharequal}{\kern0pt}\ sigma{\isacharunderscore}{\kern0pt}sets\ {\isacharparenleft}{\kern0pt}UNIV\ {\isasymtimes}\ space\ M{\isacharparenright}{\kern0pt}\ {\isacharparenleft}{\kern0pt}{\isacharbraceleft}{\kern0pt}{\isacharbraceleft}{\kern0pt}s{\isacharless}{\kern0pt}{\isachardot}{\kern0pt}{\isachardot}{\kern0pt}t{\isacharbraceright}{\kern0pt}\ {\isasymtimes}\ A\ {\isacharbar}{\kern0pt}\ A\ s\ t{\isachardot}{\kern0pt}\ A\ {\isasymin}\ F\ s\ {\isasymand}\ s\ {\isacharless}{\kern0pt}\ t{\isacharbraceright}{\kern0pt}\ {\isasymunion}\ {\isacharbraceleft}{\kern0pt}{\isacharbraceleft}{\kern0pt}bot{\isacharbraceright}{\kern0pt}\ {\isasymtimes}\ A\ {\isacharbar}{\kern0pt}\ A{\isachardot}{\kern0pt}\ A\ {\isasymin}\ F\ bot{\isacharbraceright}{\kern0pt}{\isacharparenright}{\kern0pt}{\isachardoublequoteclose}\ \isanewline
%
\isadelimproof
\ \ %
\endisadelimproof
%
\isatagproof
\isacommand{unfolding}\isamarkupfalse%
\ predictable{\isacharunderscore}{\kern0pt}sigma{\isacharunderscore}{\kern0pt}def\ sets{\isacharunderscore}{\kern0pt}measure{\isacharunderscore}{\kern0pt}of{\isacharunderscore}{\kern0pt}conv\ \isanewline
\ \ \isacommand{using}\isamarkupfalse%
\ space{\isacharunderscore}{\kern0pt}F\ sets{\isachardot}{\kern0pt}sets{\isacharunderscore}{\kern0pt}into{\isacharunderscore}{\kern0pt}space\isanewline
\ \ \isacommand{by}\isamarkupfalse%
\ {\isacharparenleft}{\kern0pt}fastforce\ intro{\isacharbang}{\kern0pt}{\isacharcolon}{\kern0pt}\ if{\isacharunderscore}{\kern0pt}P{\isacharparenright}{\kern0pt}%
\endisatagproof
{\isafoldproof}%
%
\isadelimproof
\isanewline
%
\endisadelimproof
\isanewline
\isacommand{lemma}\isamarkupfalse%
\ in{\isacharunderscore}{\kern0pt}predictable{\isacharunderscore}{\kern0pt}sigmaI{\isacharcolon}{\kern0pt}\isanewline
\ \ \isakeyword{assumes}\ {\isachardoublequoteopen}I\ {\isacharequal}{\kern0pt}\ {\isacharbraceleft}{\kern0pt}bot{\isacharbraceright}{\kern0pt}\ {\isasymLongrightarrow}\ S\ {\isasymin}\ sets\ {\isacharparenleft}{\kern0pt}F\ bot{\isacharparenright}{\kern0pt}{\isachardoublequoteclose}\ {\isachardoublequoteopen}I\ {\isasymnoteq}\ {\isacharbraceleft}{\kern0pt}bot{\isacharbraceright}{\kern0pt}\ {\isasymLongrightarrow}\ I\ {\isacharequal}{\kern0pt}\ {\isacharparenleft}{\kern0pt}{\isasymUnion}i\ {\isacharcolon}{\kern0pt}{\isacharcolon}{\kern0pt}\ nat{\isachardot}{\kern0pt}\ {\isasymInter}\ j\ {\isacharcolon}{\kern0pt}{\isacharcolon}{\kern0pt}\ nat{\isachardot}{\kern0pt}\ {\isacharbraceleft}{\kern0pt}{\isacharparenleft}{\kern0pt}ss\ i\ j{\isacharparenright}{\kern0pt}{\isacharless}{\kern0pt}{\isachardot}{\kern0pt}{\isachardot}{\kern0pt}{\isacharparenleft}{\kern0pt}ts\ i\ j{\isacharparenright}{\kern0pt}{\isacharbraceright}{\kern0pt}{\isacharparenright}{\kern0pt}\ {\isasymand}\ {\isacharparenleft}{\kern0pt}{\isasymforall}i\ j{\isachardot}{\kern0pt}\ S\ {\isasymin}\ sets\ {\isacharparenleft}{\kern0pt}F\ {\isacharparenleft}{\kern0pt}ss\ i\ j\ {\isacharcolon}{\kern0pt}{\isacharcolon}{\kern0pt}\ {\isacharprime}{\kern0pt}t{\isacharparenright}{\kern0pt}{\isacharparenright}{\kern0pt}\ {\isasymand}\ ss\ i\ j\ {\isacharless}{\kern0pt}\ ts\ i\ j{\isacharparenright}{\kern0pt}{\isachardoublequoteclose}\isanewline
\ \ \isakeyword{shows}\ {\isachardoublequoteopen}I\ {\isasymtimes}\ S\ {\isasymin}\ predictable{\isacharunderscore}{\kern0pt}sigma{\isachardoublequoteclose}\isanewline
%
\isadelimproof
%
\endisadelimproof
%
\isatagproof
\isacommand{proof}\isamarkupfalse%
\ {\isacharminus}{\kern0pt}\isanewline
\ \ \isacommand{have}\isamarkupfalse%
\ {\isacharasterisk}{\kern0pt}{\isacharcolon}{\kern0pt}\ {\isachardoublequoteopen}{\isacharbraceleft}{\kern0pt}{\isacharbraceleft}{\kern0pt}s{\isacharless}{\kern0pt}{\isachardot}{\kern0pt}{\isachardot}{\kern0pt}t{\isacharbraceright}{\kern0pt}\ {\isasymtimes}\ A\ {\isacharbar}{\kern0pt}A\ s\ t{\isachardot}{\kern0pt}\ A\ {\isasymin}\ sets\ {\isacharparenleft}{\kern0pt}F\ s{\isacharparenright}{\kern0pt}\ {\isasymand}\ s\ {\isacharless}{\kern0pt}\ t{\isacharbraceright}{\kern0pt}\ {\isasymunion}\ {\isacharbraceleft}{\kern0pt}{\isacharbraceleft}{\kern0pt}bot{\isacharbraceright}{\kern0pt}\ {\isasymtimes}\ A\ {\isacharbar}{\kern0pt}A{\isachardot}{\kern0pt}\ A\ {\isasymin}\ sets\ {\isacharparenleft}{\kern0pt}F\ bot{\isacharparenright}{\kern0pt}{\isacharbraceright}{\kern0pt}\ {\isasymsubseteq}\ Pow\ {\isacharparenleft}{\kern0pt}UNIV\ {\isasymtimes}\ space\ M{\isacharparenright}{\kern0pt}{\isachardoublequoteclose}\ \isanewline
\ \ \ \ \isacommand{using}\isamarkupfalse%
\ filtration{\isachardot}{\kern0pt}space{\isacharunderscore}{\kern0pt}F\ filtration{\isacharunderscore}{\kern0pt}axioms\ sets{\isachardot}{\kern0pt}sets{\isacharunderscore}{\kern0pt}into{\isacharunderscore}{\kern0pt}space\ \isacommand{by}\isamarkupfalse%
\ blast\isanewline
\ \ \isacommand{show}\isamarkupfalse%
\ {\isacharquery}{\kern0pt}thesis\isanewline
\ \ \isacommand{proof}\isamarkupfalse%
\ {\isacharparenleft}{\kern0pt}cases\ {\isachardoublequoteopen}I\ {\isacharequal}{\kern0pt}\ {\isacharbraceleft}{\kern0pt}bot{\isacharbraceright}{\kern0pt}{\isachardoublequoteclose}{\isacharparenright}{\kern0pt}\isanewline
\ \ \ \ \isacommand{case}\isamarkupfalse%
\ True\isanewline
\ \ \ \ \isacommand{have}\isamarkupfalse%
\ {\isachardoublequoteopen}I\ {\isasymtimes}\ S\ {\isasymin}\ {\isacharbraceleft}{\kern0pt}{\isacharbraceleft}{\kern0pt}bot{\isacharbraceright}{\kern0pt}\ {\isasymtimes}\ A\ {\isacharbar}{\kern0pt}A{\isachardot}{\kern0pt}\ A\ {\isasymin}\ sets\ {\isacharparenleft}{\kern0pt}F\ bot{\isacharparenright}{\kern0pt}{\isacharbraceright}{\kern0pt}{\isachardoublequoteclose}\ \isacommand{using}\isamarkupfalse%
\ assms\ True\ \isacommand{by}\isamarkupfalse%
\ blast\isanewline
\ \ \ \ \isacommand{then}\isamarkupfalse%
\ \isacommand{show}\isamarkupfalse%
\ {\isacharquery}{\kern0pt}thesis\ \isacommand{using}\isamarkupfalse%
\ {\isacharasterisk}{\kern0pt}\ \isacommand{by}\isamarkupfalse%
\ simp\isanewline
\ \ \isacommand{next}\isamarkupfalse%
\isanewline
\ \ \ \ \isacommand{case}\isamarkupfalse%
\ False\isanewline
\ \ \ \ \isacommand{define}\isamarkupfalse%
\ {\isasymSS}\ \isakeyword{where}\ {\isachardoublequoteopen}{\isasymSS}\ {\isacharequal}{\kern0pt}\ {\isacharbraceleft}{\kern0pt}{\isacharparenleft}{\kern0pt}{\isasymUnion}i\ {\isacharcolon}{\kern0pt}{\isacharcolon}{\kern0pt}\ nat{\isachardot}{\kern0pt}\ {\isasymInter}\ j\ {\isacharcolon}{\kern0pt}{\isacharcolon}{\kern0pt}\ nat{\isachardot}{\kern0pt}\ {\isacharbraceleft}{\kern0pt}ss\ i\ j{\isacharless}{\kern0pt}{\isachardot}{\kern0pt}{\isachardot}{\kern0pt}ts\ i\ j\ {\isacharcolon}{\kern0pt}{\isacharcolon}{\kern0pt}\ {\isacharprime}{\kern0pt}t{\isacharbraceright}{\kern0pt}{\isacharparenright}{\kern0pt}\ {\isasymtimes}\ A\ {\isacharbar}{\kern0pt}A\ ss\ ts{\isachardot}{\kern0pt}\ {\isasymforall}i\ j{\isachardot}{\kern0pt}\ A\ {\isasymin}\ sets\ {\isacharparenleft}{\kern0pt}F\ {\isacharparenleft}{\kern0pt}ss\ i\ j{\isacharparenright}{\kern0pt}{\isacharparenright}{\kern0pt}\ {\isasymand}\ ss\ i\ j\ {\isacharless}{\kern0pt}\ ts\ i\ j{\isacharbraceright}{\kern0pt}{\isachardoublequoteclose}\isanewline
\ \ \ \ \isacommand{have}\isamarkupfalse%
\ S{\isacharunderscore}{\kern0pt}in{\isacharunderscore}{\kern0pt}sets{\isacharunderscore}{\kern0pt}F{\isacharcolon}{\kern0pt}\ {\isachardoublequoteopen}S\ {\isasymin}\ sets\ {\isacharparenleft}{\kern0pt}F\ {\isacharparenleft}{\kern0pt}ss\ i\ j{\isacharparenright}{\kern0pt}{\isacharparenright}{\kern0pt}{\isachardoublequoteclose}\ \isakeyword{and}\ ss{\isacharunderscore}{\kern0pt}less{\isacharcolon}{\kern0pt}\ {\isachardoublequoteopen}ss\ i\ j\ {\isacharless}{\kern0pt}\ ts\ i\ j{\isachardoublequoteclose}\ \isakeyword{and}\ I{\isacharunderscore}{\kern0pt}eq{\isacharcolon}{\kern0pt}\ {\isachardoublequoteopen}I\ {\isacharequal}{\kern0pt}\ {\isacharparenleft}{\kern0pt}{\isasymUnion}i\ {\isacharcolon}{\kern0pt}{\isacharcolon}{\kern0pt}\ nat{\isachardot}{\kern0pt}\ {\isasymInter}\ j\ {\isacharcolon}{\kern0pt}{\isacharcolon}{\kern0pt}\ nat{\isachardot}{\kern0pt}\ {\isacharbraceleft}{\kern0pt}ss\ i\ j{\isacharless}{\kern0pt}{\isachardot}{\kern0pt}{\isachardot}{\kern0pt}ts\ i\ j{\isacharbraceright}{\kern0pt}{\isacharparenright}{\kern0pt}{\isachardoublequoteclose}\ \isakeyword{for}\ i\ j\ \isacommand{using}\isamarkupfalse%
\ assms{\isacharparenleft}{\kern0pt}{\isadigit{2}}{\isacharparenright}{\kern0pt}{\isacharbrackleft}{\kern0pt}OF\ False{\isacharbrackright}{\kern0pt}\ \isacommand{by}\isamarkupfalse%
\ auto\isanewline
\ \ \ \ \isacommand{have}\isamarkupfalse%
\ {\isacharasterisk}{\kern0pt}{\isacharasterisk}{\kern0pt}{\isacharcolon}{\kern0pt}\ {\isachardoublequoteopen}I\ {\isasymtimes}\ S\ {\isasymin}\ {\isasymSS}{\isachardoublequoteclose}\ \isacommand{unfolding}\isamarkupfalse%
\ {\isasymSS}{\isacharunderscore}{\kern0pt}def\ \isacommand{by}\isamarkupfalse%
\ {\isacharparenleft}{\kern0pt}simp\ add{\isacharcolon}{\kern0pt}\ I{\isacharunderscore}{\kern0pt}eq{\isacharcomma}{\kern0pt}\ metis\ S{\isacharunderscore}{\kern0pt}in{\isacharunderscore}{\kern0pt}sets{\isacharunderscore}{\kern0pt}F\ ss{\isacharunderscore}{\kern0pt}less{\isacharparenright}{\kern0pt}\isanewline
\ \ \ \ \isacommand{have}\isamarkupfalse%
\ {\isachardoublequoteopen}I\ {\isasymtimes}\ S\ {\isasymin}\ sigma{\isacharunderscore}{\kern0pt}sets\ {\isacharparenleft}{\kern0pt}UNIV\ {\isasymtimes}\ space\ M{\isacharparenright}{\kern0pt}\ {\isacharparenleft}{\kern0pt}{\isacharbraceleft}{\kern0pt}{\isacharbraceleft}{\kern0pt}s{\isacharless}{\kern0pt}{\isachardot}{\kern0pt}{\isachardot}{\kern0pt}t{\isacharbraceright}{\kern0pt}\ {\isasymtimes}\ A\ {\isacharbar}{\kern0pt}A\ s\ t{\isachardot}{\kern0pt}\ A\ {\isasymin}\ sets\ {\isacharparenleft}{\kern0pt}F\ s{\isacharparenright}{\kern0pt}\ {\isasymand}\ s\ {\isacharless}{\kern0pt}\ t{\isacharbraceright}{\kern0pt}{\isacharparenright}{\kern0pt}{\isachardoublequoteclose}\isanewline
\ \ \ \ \isacommand{proof}\isamarkupfalse%
\ {\isacharparenleft}{\kern0pt}intro\ subsetD{\isacharbrackleft}{\kern0pt}OF\ {\isacharunderscore}{\kern0pt}\ {\isacharasterisk}{\kern0pt}{\isacharasterisk}{\kern0pt}{\isacharbrackright}{\kern0pt}\ sigma{\isacharunderscore}{\kern0pt}sets{\isacharunderscore}{\kern0pt}mono{\isacharcomma}{\kern0pt}\ clarsimp\ simp\ add{\isacharcolon}{\kern0pt}\ {\isasymSS}{\isacharunderscore}{\kern0pt}def{\isacharcomma}{\kern0pt}\ goal{\isacharunderscore}{\kern0pt}cases{\isacharparenright}{\kern0pt}\isanewline
\ \ \ \ \ \ \isacommand{case}\isamarkupfalse%
\ {\isacharparenleft}{\kern0pt}{\isadigit{1}}\ A\ ss\ ts{\isacharparenright}{\kern0pt}\isanewline
\ \ \ \ \ \ \isacommand{hence}\isamarkupfalse%
\ {\isacharasterisk}{\kern0pt}{\isacharcolon}{\kern0pt}\ {\isachardoublequoteopen}{\isacharparenleft}{\kern0pt}{\isasymUnion}i{\isachardot}{\kern0pt}\ {\isasymInter}j{\isachardot}{\kern0pt}\ {\isacharbraceleft}{\kern0pt}ss\ i\ j{\isacharless}{\kern0pt}{\isachardot}{\kern0pt}{\isachardot}{\kern0pt}ts\ i\ j{\isacharbraceright}{\kern0pt}{\isacharparenright}{\kern0pt}\ {\isasymtimes}\ A\ {\isacharequal}{\kern0pt}\ {\isacharparenleft}{\kern0pt}{\isasymUnion}i{\isachardot}{\kern0pt}\ {\isasymInter}j{\isachardot}{\kern0pt}\ {\isacharbraceleft}{\kern0pt}ss\ i\ j{\isacharless}{\kern0pt}{\isachardot}{\kern0pt}{\isachardot}{\kern0pt}ts\ i\ j{\isacharbraceright}{\kern0pt}\ {\isasymtimes}\ A{\isacharparenright}{\kern0pt}{\isachardoublequoteclose}\ \isacommand{by}\isamarkupfalse%
\ auto\isanewline
\ \ \ \ \ \ \isacommand{thus}\isamarkupfalse%
\ {\isacharquery}{\kern0pt}case\ \isacommand{using}\isamarkupfalse%
\ space{\isacharunderscore}{\kern0pt}F\ sets{\isachardot}{\kern0pt}sets{\isacharunderscore}{\kern0pt}into{\isacharunderscore}{\kern0pt}space\ {\isadigit{1}}\ \isacommand{by}\isamarkupfalse%
\ {\isacharparenleft}{\kern0pt}fastforce\ simp\ add{\isacharcolon}{\kern0pt}\ {\isacharasterisk}{\kern0pt}\ intro{\isacharbang}{\kern0pt}{\isacharcolon}{\kern0pt}\ sigma{\isacharunderscore}{\kern0pt}sets{\isachardot}{\kern0pt}Union\ sigma{\isacharunderscore}{\kern0pt}sets{\isacharunderscore}{\kern0pt}Inter{\isacharparenright}{\kern0pt}\isanewline
\ \ \ \ \isacommand{qed}\isamarkupfalse%
\isanewline
\ \ \ \ \isacommand{thus}\isamarkupfalse%
\ {\isacharquery}{\kern0pt}thesis\ \isacommand{using}\isamarkupfalse%
\ {\isacharasterisk}{\kern0pt}\ \isacommand{by}\isamarkupfalse%
\ {\isacharparenleft}{\kern0pt}simp{\isacharcomma}{\kern0pt}\ meson\ sigma{\isacharunderscore}{\kern0pt}sets{\isacharunderscore}{\kern0pt}mono{\isacharprime}{\kern0pt}{\isacharprime}{\kern0pt}\ sigma{\isacharunderscore}{\kern0pt}sets{\isacharunderscore}{\kern0pt}top\ subsetD\ sup{\isacharunderscore}{\kern0pt}ge{\isadigit{1}}{\isacharparenright}{\kern0pt}\isanewline
\ \ \isacommand{qed}\isamarkupfalse%
\isanewline
\isacommand{qed}\isamarkupfalse%
%
\endisatagproof
{\isafoldproof}%
%
\isadelimproof
\isanewline
%
\endisadelimproof
\isanewline
\isacommand{definition}\isamarkupfalse%
\ predictable\ {\isacharcolon}{\kern0pt}{\isacharcolon}{\kern0pt}\ {\isachardoublequoteopen}{\isacharparenleft}{\kern0pt}{\isacharprime}{\kern0pt}t\ {\isasymRightarrow}\ {\isacharprime}{\kern0pt}a\ {\isasymRightarrow}\ {\isacharprime}{\kern0pt}b\ {\isacharcolon}{\kern0pt}{\isacharcolon}{\kern0pt}\ {\isacharbraceleft}{\kern0pt}second{\isacharunderscore}{\kern0pt}countable{\isacharunderscore}{\kern0pt}topology{\isacharcomma}{\kern0pt}banach{\isacharbraceright}{\kern0pt}{\isacharparenright}{\kern0pt}\ {\isasymRightarrow}\ bool{\isachardoublequoteclose}\ \isakeyword{where}\isanewline
\ \ {\isachardoublequoteopen}predictable\ X\ {\isacharequal}{\kern0pt}\ {\isacharparenleft}{\kern0pt}case{\isacharunderscore}{\kern0pt}prod\ X\ {\isasymin}\ borel{\isacharunderscore}{\kern0pt}measurable\ {\isacharparenleft}{\kern0pt}predictable{\isacharunderscore}{\kern0pt}sigma{\isacharparenright}{\kern0pt}{\isacharparenright}{\kern0pt}{\isachardoublequoteclose}\isanewline
\isanewline
\isacommand{lemmas}\isamarkupfalse%
\ predictableD\ {\isacharequal}{\kern0pt}\ measurable{\isacharunderscore}{\kern0pt}sets{\isacharbrackleft}{\kern0pt}OF\ predictable{\isacharunderscore}{\kern0pt}def{\isacharbrackleft}{\kern0pt}THEN\ iffD{\isadigit{1}}{\isacharbrackright}{\kern0pt}{\isacharcomma}{\kern0pt}\ unfolded\ space{\isacharunderscore}{\kern0pt}predictable{\isacharunderscore}{\kern0pt}sigma{\isacharbrackright}{\kern0pt}\isanewline
\isanewline
\isacommand{lemma}\isamarkupfalse%
\ {\isacharparenleft}{\kern0pt}\isakeyword{in}\ nat{\isacharunderscore}{\kern0pt}filtered{\isacharunderscore}{\kern0pt}sigma{\isacharunderscore}{\kern0pt}finite{\isacharunderscore}{\kern0pt}measure{\isacharparenright}{\kern0pt}\ predictable{\isacharunderscore}{\kern0pt}sets{\isacharunderscore}{\kern0pt}in{\isacharunderscore}{\kern0pt}F{\isacharcolon}{\kern0pt}\isanewline
\ \ \isakeyword{assumes}\ {\isachardoublequoteopen}{\isacharparenleft}{\kern0pt}{\isasymUnion}i{\isachardot}{\kern0pt}\ {\isacharbraceleft}{\kern0pt}i{\isacharbraceright}{\kern0pt}\ {\isasymtimes}\ A\ i{\isacharparenright}{\kern0pt}\ {\isasymin}\ predictable{\isacharunderscore}{\kern0pt}sigma{\isachardoublequoteclose}\isanewline
\ \ \isakeyword{shows}\ {\isachardoublequoteopen}A\ {\isacharparenleft}{\kern0pt}Suc\ i{\isacharparenright}{\kern0pt}\ {\isasymin}\ F\ i{\isachardoublequoteclose}\ \isanewline
\ \ \ \ \ \ \ \ {\isachardoublequoteopen}A\ {\isadigit{0}}\ {\isasymin}\ F\ {\isadigit{0}}{\isachardoublequoteclose}\isanewline
%
\isadelimproof
\ \ %
\endisadelimproof
%
\isatagproof
\isacommand{using}\isamarkupfalse%
\ assms\ \isacommand{unfolding}\isamarkupfalse%
\ sets{\isacharunderscore}{\kern0pt}predictable{\isacharunderscore}{\kern0pt}sigma\isanewline
\isacommand{proof}\isamarkupfalse%
\ {\isacharparenleft}{\kern0pt}induction\ {\isachardoublequoteopen}{\isacharparenleft}{\kern0pt}{\isasymUnion}i{\isachardot}{\kern0pt}\ {\isacharbraceleft}{\kern0pt}i{\isacharbraceright}{\kern0pt}\ {\isasymtimes}\ A\ i{\isacharparenright}{\kern0pt}{\isachardoublequoteclose}\ arbitrary{\isacharcolon}{\kern0pt}\ A{\isacharparenright}{\kern0pt}\isanewline
\ \ \isacommand{case}\isamarkupfalse%
\ Basic\isanewline
\ \ \isacommand{{\isacharbraceleft}{\kern0pt}}\isamarkupfalse%
\isanewline
\ \ \ \ \isacommand{assume}\isamarkupfalse%
\ {\isachardoublequoteopen}{\isasymexists}S{\isachardot}{\kern0pt}\ {\isacharparenleft}{\kern0pt}{\isasymUnion}i{\isachardot}{\kern0pt}\ {\isacharbraceleft}{\kern0pt}i{\isacharbraceright}{\kern0pt}\ {\isasymtimes}\ A\ i{\isacharparenright}{\kern0pt}\ {\isacharequal}{\kern0pt}\ {\isacharbraceleft}{\kern0pt}bot{\isacharbraceright}{\kern0pt}\ {\isasymtimes}\ S{\isachardoublequoteclose}\isanewline
\ \ \ \ \isacommand{then}\isamarkupfalse%
\ \isacommand{obtain}\isamarkupfalse%
\ S\ \isakeyword{where}\ S{\isacharcolon}{\kern0pt}\ {\isachardoublequoteopen}{\isacharparenleft}{\kern0pt}{\isasymUnion}i{\isachardot}{\kern0pt}\ {\isacharbraceleft}{\kern0pt}i{\isacharbraceright}{\kern0pt}\ {\isasymtimes}\ A\ i{\isacharparenright}{\kern0pt}\ {\isacharequal}{\kern0pt}\ {\isacharbraceleft}{\kern0pt}bot{\isacharbraceright}{\kern0pt}\ {\isasymtimes}\ S{\isachardoublequoteclose}\ \isacommand{by}\isamarkupfalse%
\ blast\isanewline
\ \ \ \ \isacommand{hence}\isamarkupfalse%
\ {\isachardoublequoteopen}S\ {\isasymin}\ F\ {\isadigit{0}}{\isachardoublequoteclose}\ \isacommand{using}\isamarkupfalse%
\ Basic\ \isacommand{by}\isamarkupfalse%
\ {\isacharparenleft}{\kern0pt}fastforce\ simp\ add{\isacharcolon}{\kern0pt}\ times{\isacharunderscore}{\kern0pt}eq{\isacharunderscore}{\kern0pt}iff\ bot{\isacharunderscore}{\kern0pt}nat{\isacharunderscore}{\kern0pt}def{\isacharparenright}{\kern0pt}\isanewline
\ \ \ \ \isacommand{moreover}\isamarkupfalse%
\ \isacommand{have}\isamarkupfalse%
\ {\isachardoublequoteopen}A\ i\ {\isacharequal}{\kern0pt}\ {\isacharbraceleft}{\kern0pt}{\isacharbraceright}{\kern0pt}{\isachardoublequoteclose}\ \isakeyword{if}\ {\isachardoublequoteopen}i\ {\isasymnoteq}\ bot{\isachardoublequoteclose}\ \isakeyword{for}\ i\ \isacommand{using}\isamarkupfalse%
\ that\ S\ \isacommand{by}\isamarkupfalse%
\ blast\isanewline
\ \ \ \ \isacommand{moreover}\isamarkupfalse%
\ \isacommand{have}\isamarkupfalse%
\ {\isachardoublequoteopen}A\ bot\ {\isacharequal}{\kern0pt}\ S{\isachardoublequoteclose}\ \isacommand{using}\isamarkupfalse%
\ S\ \isacommand{by}\isamarkupfalse%
\ blast\isanewline
\ \ \ \ \isacommand{ultimately}\isamarkupfalse%
\ \isacommand{have}\isamarkupfalse%
\ {\isachardoublequoteopen}A\ {\isacharparenleft}{\kern0pt}Suc\ i{\isacharparenright}{\kern0pt}\ {\isasymin}\ F\ i{\isachardoublequoteclose}\ {\isachardoublequoteopen}A\ {\isadigit{0}}\ {\isasymin}\ F\ {\isadigit{0}}{\isachardoublequoteclose}\ \isakeyword{for}\ i\ \isacommand{unfolding}\isamarkupfalse%
\ bot{\isacharunderscore}{\kern0pt}nat{\isacharunderscore}{\kern0pt}def\ \isacommand{by}\isamarkupfalse%
\ {\isacharparenleft}{\kern0pt}auto\ simp\ add{\isacharcolon}{\kern0pt}\ bot{\isacharunderscore}{\kern0pt}nat{\isacharunderscore}{\kern0pt}def{\isacharparenright}{\kern0pt}\isanewline
\ \ \isacommand{{\isacharbraceright}{\kern0pt}}\isamarkupfalse%
\isanewline
\ \ \isacommand{note}\isamarkupfalse%
\ {\isacharasterisk}{\kern0pt}\ {\isacharequal}{\kern0pt}\ this\isanewline
\ \ \isacommand{{\isacharbraceleft}{\kern0pt}}\isamarkupfalse%
\isanewline
\ \ \ \ \isacommand{assume}\isamarkupfalse%
\ {\isachardoublequoteopen}{\isasymnexists}S{\isachardot}{\kern0pt}\ {\isacharparenleft}{\kern0pt}{\isasymUnion}i{\isachardot}{\kern0pt}\ {\isacharbraceleft}{\kern0pt}i{\isacharbraceright}{\kern0pt}\ {\isasymtimes}\ A\ i{\isacharparenright}{\kern0pt}\ {\isacharequal}{\kern0pt}\ {\isacharbraceleft}{\kern0pt}bot{\isacharbraceright}{\kern0pt}\ {\isasymtimes}\ S{\isachardoublequoteclose}\isanewline
\ \ \ \ \isacommand{then}\isamarkupfalse%
\ \isacommand{obtain}\isamarkupfalse%
\ s\ t\ B\ \isakeyword{where}\ B{\isacharcolon}{\kern0pt}\ {\isachardoublequoteopen}{\isacharparenleft}{\kern0pt}{\isasymUnion}i{\isachardot}{\kern0pt}\ {\isacharbraceleft}{\kern0pt}i{\isacharbraceright}{\kern0pt}\ {\isasymtimes}\ A\ i{\isacharparenright}{\kern0pt}\ {\isacharequal}{\kern0pt}\ {\isacharbraceleft}{\kern0pt}s{\isacharless}{\kern0pt}{\isachardot}{\kern0pt}{\isachardot}{\kern0pt}t{\isacharbraceright}{\kern0pt}\ {\isasymtimes}\ B{\isachardoublequoteclose}\ {\isachardoublequoteopen}B\ {\isasymin}\ sets\ {\isacharparenleft}{\kern0pt}F\ s{\isacharparenright}{\kern0pt}{\isachardoublequoteclose}\ {\isachardoublequoteopen}s\ {\isacharless}{\kern0pt}\ t{\isachardoublequoteclose}\ \isacommand{using}\isamarkupfalse%
\ Basic\ \isacommand{by}\isamarkupfalse%
\ auto\isanewline
\ \ \ \ \isacommand{hence}\isamarkupfalse%
\ {\isachardoublequoteopen}A\ i\ {\isacharequal}{\kern0pt}\ B{\isachardoublequoteclose}\ \isakeyword{if}\ {\isachardoublequoteopen}i\ {\isasymin}\ {\isacharbraceleft}{\kern0pt}s{\isacharless}{\kern0pt}{\isachardot}{\kern0pt}{\isachardot}{\kern0pt}t{\isacharbraceright}{\kern0pt}{\isachardoublequoteclose}\ \isakeyword{for}\ i\ \isacommand{using}\isamarkupfalse%
\ that\ \isacommand{by}\isamarkupfalse%
\ fast\isanewline
\ \ \ \ \isacommand{moreover}\isamarkupfalse%
\ \isacommand{have}\isamarkupfalse%
\ {\isachardoublequoteopen}A\ i\ {\isacharequal}{\kern0pt}\ {\isacharbraceleft}{\kern0pt}{\isacharbraceright}{\kern0pt}{\isachardoublequoteclose}\ \isakeyword{if}\ {\isachardoublequoteopen}i\ {\isasymnotin}\ {\isacharbraceleft}{\kern0pt}s{\isacharless}{\kern0pt}{\isachardot}{\kern0pt}{\isachardot}{\kern0pt}t{\isacharbraceright}{\kern0pt}{\isachardoublequoteclose}\ \isakeyword{for}\ i\ \isacommand{using}\isamarkupfalse%
\ B\ that\ \isacommand{by}\isamarkupfalse%
\ fastforce\isanewline
\ \ \ \ \isacommand{ultimately}\isamarkupfalse%
\ \isacommand{have}\isamarkupfalse%
\ {\isachardoublequoteopen}A\ {\isacharparenleft}{\kern0pt}Suc\ i{\isacharparenright}{\kern0pt}\ {\isasymin}\ F\ i{\isachardoublequoteclose}\ {\isachardoublequoteopen}A\ {\isadigit{0}}\ {\isasymin}\ F\ {\isadigit{0}}{\isachardoublequoteclose}\ \isakeyword{for}\ i\ \isacommand{unfolding}\isamarkupfalse%
\ bot{\isacharunderscore}{\kern0pt}nat{\isacharunderscore}{\kern0pt}def\ \isacommand{using}\isamarkupfalse%
\ B\ sets{\isacharunderscore}{\kern0pt}F{\isacharunderscore}{\kern0pt}mono\ \isacommand{by}\isamarkupfalse%
\ {\isacharparenleft}{\kern0pt}auto\ simp\ add{\isacharcolon}{\kern0pt}\ bot{\isacharunderscore}{\kern0pt}nat{\isacharunderscore}{\kern0pt}def{\isacharparenright}{\kern0pt}\ {\isacharparenleft}{\kern0pt}metis\ less{\isacharunderscore}{\kern0pt}Suc{\isacharunderscore}{\kern0pt}eq{\isacharunderscore}{\kern0pt}le\ sets{\isachardot}{\kern0pt}empty{\isacharunderscore}{\kern0pt}sets\ subset{\isacharunderscore}{\kern0pt}eq{\isacharparenright}{\kern0pt}\isanewline
\ \ \isacommand{{\isacharbraceright}{\kern0pt}}\isamarkupfalse%
\isanewline
\ \ \isacommand{note}\isamarkupfalse%
\ {\isacharasterisk}{\kern0pt}{\isacharasterisk}{\kern0pt}\ {\isacharequal}{\kern0pt}\ this\isanewline
\ \ \isacommand{show}\isamarkupfalse%
\ {\isachardoublequoteopen}A\ {\isacharparenleft}{\kern0pt}Suc\ i{\isacharparenright}{\kern0pt}\ {\isasymin}\ sets\ {\isacharparenleft}{\kern0pt}F\ i{\isacharparenright}{\kern0pt}{\isachardoublequoteclose}\ {\isachardoublequoteopen}A\ {\isadigit{0}}\ {\isasymin}\ sets\ {\isacharparenleft}{\kern0pt}F\ {\isadigit{0}}{\isacharparenright}{\kern0pt}{\isachardoublequoteclose}\ \isacommand{using}\isamarkupfalse%
\ {\isacharasterisk}{\kern0pt}{\isacharparenleft}{\kern0pt}{\isadigit{1}}{\isacharparenright}{\kern0pt}{\isacharbrackleft}{\kern0pt}of\ i{\isacharbrackright}{\kern0pt}\ {\isacharasterisk}{\kern0pt}{\isacharparenleft}{\kern0pt}{\isadigit{2}}{\isacharparenright}{\kern0pt}\ {\isacharasterisk}{\kern0pt}{\isacharasterisk}{\kern0pt}{\isacharparenleft}{\kern0pt}{\isadigit{1}}{\isacharparenright}{\kern0pt}{\isacharbrackleft}{\kern0pt}of\ i{\isacharbrackright}{\kern0pt}\ {\isacharasterisk}{\kern0pt}{\isacharasterisk}{\kern0pt}{\isacharparenleft}{\kern0pt}{\isadigit{2}}{\isacharparenright}{\kern0pt}\ \isacommand{by}\isamarkupfalse%
\ auto\ blast{\isacharplus}{\kern0pt}\ \isanewline
\isacommand{next}\isamarkupfalse%
\isanewline
\ \ \isacommand{case}\isamarkupfalse%
\ Empty\isanewline
\ \ \isacommand{{\isacharbraceleft}{\kern0pt}}\isamarkupfalse%
\isanewline
\ \ \ \ \isacommand{case}\isamarkupfalse%
\ {\isadigit{1}}\isanewline
\ \ \ \ \isacommand{then}\isamarkupfalse%
\ \isacommand{show}\isamarkupfalse%
\ {\isacharquery}{\kern0pt}case\ \isacommand{using}\isamarkupfalse%
\ Empty\ \isacommand{by}\isamarkupfalse%
\ simp\isanewline
\ \ \isacommand{next}\isamarkupfalse%
\isanewline
\ \ \ \ \isacommand{case}\isamarkupfalse%
\ {\isadigit{2}}\isanewline
\ \ \ \ \isacommand{then}\isamarkupfalse%
\ \isacommand{show}\isamarkupfalse%
\ {\isacharquery}{\kern0pt}case\ \isacommand{using}\isamarkupfalse%
\ Empty\ \isacommand{by}\isamarkupfalse%
\ simp\isanewline
\ \ \isacommand{{\isacharbraceright}{\kern0pt}}\isamarkupfalse%
\isanewline
\isacommand{next}\isamarkupfalse%
\isanewline
\ \ \isacommand{case}\isamarkupfalse%
\ {\isacharparenleft}{\kern0pt}Compl\ a{\isacharparenright}{\kern0pt}\isanewline
\ \ \isacommand{have}\isamarkupfalse%
\ a{\isacharunderscore}{\kern0pt}in{\isacharcolon}{\kern0pt}\ {\isachardoublequoteopen}a\ {\isasymsubseteq}\ UNIV\ {\isasymtimes}\ space\ M{\isachardoublequoteclose}\ \isacommand{using}\isamarkupfalse%
\ Compl{\isacharparenleft}{\kern0pt}{\isadigit{1}}{\isacharparenright}{\kern0pt}\ sets{\isachardot}{\kern0pt}sets{\isacharunderscore}{\kern0pt}into{\isacharunderscore}{\kern0pt}space\ sets{\isacharunderscore}{\kern0pt}predictable{\isacharunderscore}{\kern0pt}sigma\ space{\isacharunderscore}{\kern0pt}predictable{\isacharunderscore}{\kern0pt}sigma\ \isacommand{by}\isamarkupfalse%
\ metis\isanewline
\ \ \isacommand{hence}\isamarkupfalse%
\ A{\isacharunderscore}{\kern0pt}in{\isacharcolon}{\kern0pt}\ {\isachardoublequoteopen}A\ i\ {\isasymsubseteq}\ space\ M{\isachardoublequoteclose}\ \isakeyword{for}\ i\ \isacommand{using}\isamarkupfalse%
\ Compl{\isacharparenleft}{\kern0pt}{\isadigit{4}}{\isacharparenright}{\kern0pt}\ \isacommand{by}\isamarkupfalse%
\ blast\isanewline
\ \ \isacommand{have}\isamarkupfalse%
\ a{\isacharcolon}{\kern0pt}\ {\isachardoublequoteopen}a\ {\isacharequal}{\kern0pt}\ UNIV\ {\isasymtimes}\ space\ M\ {\isacharminus}{\kern0pt}\ {\isacharparenleft}{\kern0pt}{\isasymUnion}i{\isachardot}{\kern0pt}\ {\isacharbraceleft}{\kern0pt}i{\isacharbraceright}{\kern0pt}\ {\isasymtimes}\ A\ i{\isacharparenright}{\kern0pt}{\isachardoublequoteclose}\ \isacommand{using}\isamarkupfalse%
\ a{\isacharunderscore}{\kern0pt}in\ Compl{\isacharparenleft}{\kern0pt}{\isadigit{4}}{\isacharparenright}{\kern0pt}\ \isacommand{by}\isamarkupfalse%
\ blast\isanewline
\ \ \isacommand{also}\isamarkupfalse%
\ \isacommand{have}\isamarkupfalse%
\ {\isachardoublequoteopen}{\isachardot}{\kern0pt}{\isachardot}{\kern0pt}{\isachardot}{\kern0pt}\ {\isacharequal}{\kern0pt}\ {\isacharparenleft}{\kern0pt}{\isasymUnion}j{\isachardot}{\kern0pt}\ {\isacharbraceleft}{\kern0pt}j{\isacharbraceright}{\kern0pt}\ {\isasymtimes}\ {\isacharparenleft}{\kern0pt}space\ M\ {\isacharminus}{\kern0pt}\ A\ j{\isacharparenright}{\kern0pt}{\isacharparenright}{\kern0pt}{\isachardoublequoteclose}\ \isacommand{by}\isamarkupfalse%
\ blast\isanewline
\ \ \isacommand{finally}\isamarkupfalse%
\ \isacommand{have}\isamarkupfalse%
\ {\isacharasterisk}{\kern0pt}{\isacharcolon}{\kern0pt}\ {\isachardoublequoteopen}{\isacharparenleft}{\kern0pt}space\ M\ {\isacharminus}{\kern0pt}\ A\ {\isacharparenleft}{\kern0pt}Suc\ i{\isacharparenright}{\kern0pt}{\isacharparenright}{\kern0pt}\ {\isasymin}\ F\ i{\isachardoublequoteclose}\ {\isachardoublequoteopen}{\isacharparenleft}{\kern0pt}space\ M\ {\isacharminus}{\kern0pt}\ A\ {\isadigit{0}}{\isacharparenright}{\kern0pt}\ {\isasymin}\ F\ {\isadigit{0}}{\isachardoublequoteclose}\ \isacommand{using}\isamarkupfalse%
\ Compl{\isacharparenleft}{\kern0pt}{\isadigit{2}}{\isacharcomma}{\kern0pt}{\isadigit{3}}{\isacharparenright}{\kern0pt}\ \isacommand{by}\isamarkupfalse%
\ auto\isanewline
\ \ \isacommand{{\isacharbraceleft}{\kern0pt}}\isamarkupfalse%
\isanewline
\ \ \ \ \isacommand{case}\isamarkupfalse%
\ {\isadigit{1}}\isanewline
\ \ \ \ \isacommand{then}\isamarkupfalse%
\ \isacommand{show}\isamarkupfalse%
\ {\isacharquery}{\kern0pt}case\ \isacommand{using}\isamarkupfalse%
\ {\isacharasterisk}{\kern0pt}\ A{\isacharunderscore}{\kern0pt}in\ \isacommand{by}\isamarkupfalse%
\ {\isacharparenleft}{\kern0pt}metis\ double{\isacharunderscore}{\kern0pt}diff\ sets{\isachardot}{\kern0pt}compl{\isacharunderscore}{\kern0pt}sets\ space{\isacharunderscore}{\kern0pt}F\ subset{\isacharunderscore}{\kern0pt}refl{\isacharparenright}{\kern0pt}\isanewline
\ \ \isacommand{next}\isamarkupfalse%
\isanewline
\ \ \ \ \isacommand{case}\isamarkupfalse%
\ {\isadigit{2}}\isanewline
\ \ \ \ \isacommand{then}\isamarkupfalse%
\ \isacommand{show}\isamarkupfalse%
\ {\isacharquery}{\kern0pt}case\ \isacommand{using}\isamarkupfalse%
\ {\isacharasterisk}{\kern0pt}\ A{\isacharunderscore}{\kern0pt}in\ \isacommand{by}\isamarkupfalse%
\ {\isacharparenleft}{\kern0pt}metis\ double{\isacharunderscore}{\kern0pt}diff\ sets{\isachardot}{\kern0pt}compl{\isacharunderscore}{\kern0pt}sets\ space{\isacharunderscore}{\kern0pt}F\ subset{\isacharunderscore}{\kern0pt}refl{\isacharparenright}{\kern0pt}\isanewline
\ \ \isacommand{{\isacharbraceright}{\kern0pt}}\isamarkupfalse%
\isanewline
\isacommand{next}\isamarkupfalse%
\isanewline
\ \ \isacommand{case}\isamarkupfalse%
\ {\isacharparenleft}{\kern0pt}Union\ a{\isacharparenright}{\kern0pt}\isanewline
\ \ \isacommand{have}\isamarkupfalse%
\ a{\isacharunderscore}{\kern0pt}in{\isacharcolon}{\kern0pt}\ {\isachardoublequoteopen}a\ i\ {\isasymsubseteq}\ UNIV\ {\isasymtimes}\ space\ M{\isachardoublequoteclose}\ \isakeyword{for}\ i\ \isacommand{using}\isamarkupfalse%
\ Union{\isacharparenleft}{\kern0pt}{\isadigit{1}}{\isacharparenright}{\kern0pt}\ sets{\isachardot}{\kern0pt}sets{\isacharunderscore}{\kern0pt}into{\isacharunderscore}{\kern0pt}space\ sets{\isacharunderscore}{\kern0pt}predictable{\isacharunderscore}{\kern0pt}sigma\ space{\isacharunderscore}{\kern0pt}predictable{\isacharunderscore}{\kern0pt}sigma\ \isacommand{by}\isamarkupfalse%
\ metis\isanewline
\ \ \isacommand{hence}\isamarkupfalse%
\ A{\isacharunderscore}{\kern0pt}in{\isacharcolon}{\kern0pt}\ {\isachardoublequoteopen}A\ i\ {\isasymsubseteq}\ space\ M{\isachardoublequoteclose}\ \isakeyword{for}\ i\ \isacommand{using}\isamarkupfalse%
\ Union{\isacharparenleft}{\kern0pt}{\isadigit{4}}{\isacharparenright}{\kern0pt}\ \isacommand{by}\isamarkupfalse%
\ blast\isanewline
\ \ \isacommand{hence}\isamarkupfalse%
\ a{\isacharunderscore}{\kern0pt}i{\isacharcolon}{\kern0pt}\ {\isachardoublequoteopen}a\ i\ {\isacharequal}{\kern0pt}\ {\isacharparenleft}{\kern0pt}{\isasymUnion}j{\isachardot}{\kern0pt}\ {\isacharbraceleft}{\kern0pt}j{\isacharbraceright}{\kern0pt}\ {\isasymtimes}\ {\isacharparenleft}{\kern0pt}snd\ {\isacharbackquote}{\kern0pt}\ {\isacharparenleft}{\kern0pt}a\ i\ {\isasyminter}\ {\isacharparenleft}{\kern0pt}{\isacharbraceleft}{\kern0pt}j{\isacharbraceright}{\kern0pt}\ {\isasymtimes}\ space\ M{\isacharparenright}{\kern0pt}{\isacharparenright}{\kern0pt}{\isacharparenright}{\kern0pt}{\isacharparenright}{\kern0pt}{\isachardoublequoteclose}\ \isakeyword{for}\ i\ \isacommand{by}\isamarkupfalse%
\ auto\ {\isacharparenleft}{\kern0pt}smt\ {\isacharparenleft}{\kern0pt}verit{\isacharcomma}{\kern0pt}\ del{\isacharunderscore}{\kern0pt}insts{\isacharparenright}{\kern0pt}\ IntI\ Union{\isachardot}{\kern0pt}hyps{\isacharparenleft}{\kern0pt}{\isadigit{1}}{\isacharparenright}{\kern0pt}\ image{\isacharunderscore}{\kern0pt}iff\ in{\isacharunderscore}{\kern0pt}mono\ insertCI\ mem{\isacharunderscore}{\kern0pt}Sigma{\isacharunderscore}{\kern0pt}iff\ sets{\isachardot}{\kern0pt}sets{\isacharunderscore}{\kern0pt}into{\isacharunderscore}{\kern0pt}space\ sets{\isacharunderscore}{\kern0pt}predictable{\isacharunderscore}{\kern0pt}sigma\ snd{\isacharunderscore}{\kern0pt}conv\ space{\isacharunderscore}{\kern0pt}predictable{\isacharunderscore}{\kern0pt}sigma{\isacharparenright}{\kern0pt}\isanewline
\ \ \isacommand{have}\isamarkupfalse%
\ A{\isacharunderscore}{\kern0pt}i{\isacharcolon}{\kern0pt}\ {\isachardoublequoteopen}A\ i\ {\isacharequal}{\kern0pt}\ snd\ {\isacharbackquote}{\kern0pt}\ {\isacharparenleft}{\kern0pt}{\isasymUnion}\ {\isacharparenleft}{\kern0pt}range\ a{\isacharparenright}{\kern0pt}\ {\isasyminter}\ {\isacharparenleft}{\kern0pt}{\isacharbraceleft}{\kern0pt}i{\isacharbraceright}{\kern0pt}\ {\isasymtimes}\ space\ M{\isacharparenright}{\kern0pt}{\isacharparenright}{\kern0pt}{\isachardoublequoteclose}\ \isakeyword{for}\ i\ \isacommand{unfolding}\isamarkupfalse%
\ Union{\isacharparenleft}{\kern0pt}{\isadigit{4}}{\isacharparenright}{\kern0pt}\ \isacommand{using}\isamarkupfalse%
\ A{\isacharunderscore}{\kern0pt}in\ \isacommand{by}\isamarkupfalse%
\ force\ \isanewline
\ \ \isacommand{have}\isamarkupfalse%
\ {\isacharasterisk}{\kern0pt}{\isacharcolon}{\kern0pt}\ {\isachardoublequoteopen}snd\ {\isacharbackquote}{\kern0pt}\ {\isacharparenleft}{\kern0pt}a\ j\ {\isasyminter}\ {\isacharparenleft}{\kern0pt}{\isacharbraceleft}{\kern0pt}Suc\ i{\isacharbraceright}{\kern0pt}\ {\isasymtimes}\ space\ M{\isacharparenright}{\kern0pt}{\isacharparenright}{\kern0pt}\ {\isasymin}\ F\ i{\isachardoublequoteclose}\ {\isachardoublequoteopen}snd\ {\isacharbackquote}{\kern0pt}\ {\isacharparenleft}{\kern0pt}a\ j\ {\isasyminter}\ {\isacharparenleft}{\kern0pt}{\isacharbraceleft}{\kern0pt}{\isadigit{0}}{\isacharbraceright}{\kern0pt}\ {\isasymtimes}\ space\ M{\isacharparenright}{\kern0pt}{\isacharparenright}{\kern0pt}\ {\isasymin}\ F\ {\isadigit{0}}{\isachardoublequoteclose}\ \isakeyword{for}\ j\ \isacommand{using}\isamarkupfalse%
\ Union{\isacharparenleft}{\kern0pt}{\isadigit{2}}{\isacharcomma}{\kern0pt}{\isadigit{3}}{\isacharparenright}{\kern0pt}{\isacharbrackleft}{\kern0pt}OF\ a{\isacharunderscore}{\kern0pt}i{\isacharbrackright}{\kern0pt}\ \isacommand{by}\isamarkupfalse%
\ auto\isanewline
\ \ \isacommand{{\isacharbraceleft}{\kern0pt}}\isamarkupfalse%
\isanewline
\ \ \ \ \isacommand{case}\isamarkupfalse%
\ {\isadigit{1}}\isanewline
\ \ \ \ \isacommand{have}\isamarkupfalse%
\ {\isachardoublequoteopen}{\isacharparenleft}{\kern0pt}{\isasymUnion}j{\isachardot}{\kern0pt}\ snd\ {\isacharbackquote}{\kern0pt}\ {\isacharparenleft}{\kern0pt}a\ j\ {\isasyminter}\ {\isacharparenleft}{\kern0pt}{\isacharbraceleft}{\kern0pt}Suc\ i{\isacharbraceright}{\kern0pt}\ {\isasymtimes}\ space\ M{\isacharparenright}{\kern0pt}{\isacharparenright}{\kern0pt}{\isacharparenright}{\kern0pt}\ {\isasymin}\ F\ i{\isachardoublequoteclose}\ \isacommand{using}\isamarkupfalse%
\ {\isacharasterisk}{\kern0pt}\ \isacommand{by}\isamarkupfalse%
\ fast\isanewline
\ \ \ \ \isacommand{moreover}\isamarkupfalse%
\ \isacommand{have}\isamarkupfalse%
\ {\isachardoublequoteopen}{\isacharparenleft}{\kern0pt}{\isasymUnion}j{\isachardot}{\kern0pt}\ snd\ {\isacharbackquote}{\kern0pt}\ {\isacharparenleft}{\kern0pt}a\ j\ {\isasyminter}\ {\isacharparenleft}{\kern0pt}{\isacharbraceleft}{\kern0pt}Suc\ i{\isacharbraceright}{\kern0pt}\ {\isasymtimes}\ space\ M{\isacharparenright}{\kern0pt}{\isacharparenright}{\kern0pt}{\isacharparenright}{\kern0pt}\ {\isacharequal}{\kern0pt}\ snd\ {\isacharbackquote}{\kern0pt}\ {\isacharparenleft}{\kern0pt}{\isasymUnion}\ {\isacharparenleft}{\kern0pt}range\ a{\isacharparenright}{\kern0pt}\ {\isasyminter}\ {\isacharparenleft}{\kern0pt}{\isacharbraceleft}{\kern0pt}Suc\ i{\isacharbraceright}{\kern0pt}\ {\isasymtimes}\ space\ M{\isacharparenright}{\kern0pt}{\isacharparenright}{\kern0pt}{\isachardoublequoteclose}\ \isacommand{by}\isamarkupfalse%
\ fast\isanewline
\ \ \ \ \isacommand{ultimately}\isamarkupfalse%
\ \isacommand{show}\isamarkupfalse%
\ {\isacharquery}{\kern0pt}case\ \isacommand{using}\isamarkupfalse%
\ A{\isacharunderscore}{\kern0pt}i\ \isacommand{by}\isamarkupfalse%
\ metis\isanewline
\ \ \isacommand{next}\isamarkupfalse%
\isanewline
\ \ \ \ \isacommand{case}\isamarkupfalse%
\ {\isadigit{2}}\isanewline
\ \ \ \ \isacommand{have}\isamarkupfalse%
\ {\isachardoublequoteopen}{\isacharparenleft}{\kern0pt}{\isasymUnion}j{\isachardot}{\kern0pt}\ snd\ {\isacharbackquote}{\kern0pt}\ {\isacharparenleft}{\kern0pt}a\ j\ {\isasyminter}\ {\isacharparenleft}{\kern0pt}{\isacharbraceleft}{\kern0pt}{\isadigit{0}}{\isacharbraceright}{\kern0pt}\ {\isasymtimes}\ space\ M{\isacharparenright}{\kern0pt}{\isacharparenright}{\kern0pt}{\isacharparenright}{\kern0pt}\ {\isasymin}\ F\ {\isadigit{0}}{\isachardoublequoteclose}\ \isacommand{using}\isamarkupfalse%
\ {\isacharasterisk}{\kern0pt}\ \isacommand{by}\isamarkupfalse%
\ fast\isanewline
\ \ \ \ \isacommand{moreover}\isamarkupfalse%
\ \isacommand{have}\isamarkupfalse%
\ {\isachardoublequoteopen}{\isacharparenleft}{\kern0pt}{\isasymUnion}j{\isachardot}{\kern0pt}\ snd\ {\isacharbackquote}{\kern0pt}\ {\isacharparenleft}{\kern0pt}a\ j\ {\isasyminter}\ {\isacharparenleft}{\kern0pt}{\isacharbraceleft}{\kern0pt}{\isadigit{0}}{\isacharbraceright}{\kern0pt}\ {\isasymtimes}\ space\ M{\isacharparenright}{\kern0pt}{\isacharparenright}{\kern0pt}{\isacharparenright}{\kern0pt}\ {\isacharequal}{\kern0pt}\ snd\ {\isacharbackquote}{\kern0pt}\ {\isacharparenleft}{\kern0pt}{\isasymUnion}\ {\isacharparenleft}{\kern0pt}range\ a{\isacharparenright}{\kern0pt}\ {\isasyminter}\ {\isacharparenleft}{\kern0pt}{\isacharbraceleft}{\kern0pt}{\isadigit{0}}{\isacharbraceright}{\kern0pt}\ {\isasymtimes}\ space\ M{\isacharparenright}{\kern0pt}{\isacharparenright}{\kern0pt}{\isachardoublequoteclose}\ \isacommand{by}\isamarkupfalse%
\ fast\isanewline
\ \ \ \ \isacommand{ultimately}\isamarkupfalse%
\ \isacommand{show}\isamarkupfalse%
\ {\isacharquery}{\kern0pt}case\ \isacommand{using}\isamarkupfalse%
\ A{\isacharunderscore}{\kern0pt}i\ \isacommand{by}\isamarkupfalse%
\ metis\isanewline
\ \ \isacommand{{\isacharbraceright}{\kern0pt}}\isamarkupfalse%
\isanewline
\isacommand{qed}\isamarkupfalse%
%
\endisatagproof
{\isafoldproof}%
%
\isadelimproof
\isanewline
%
\endisadelimproof
\isanewline
\isacommand{lemma}\isamarkupfalse%
\ {\isacharparenleft}{\kern0pt}\isakeyword{in}\ nat{\isacharunderscore}{\kern0pt}filtered{\isacharunderscore}{\kern0pt}sigma{\isacharunderscore}{\kern0pt}finite{\isacharunderscore}{\kern0pt}measure{\isacharparenright}{\kern0pt}\ predictable{\isacharunderscore}{\kern0pt}discrete{\isacharunderscore}{\kern0pt}time{\isacharunderscore}{\kern0pt}process{\isacharunderscore}{\kern0pt}measurable{\isacharcolon}{\kern0pt}\isanewline
\ \ \isakeyword{assumes}\ {\isachardoublequoteopen}predictable\ X{\isachardoublequoteclose}\isanewline
\ \ \isakeyword{shows}\ {\isachardoublequoteopen}X\ i\ {\isasymin}\ borel{\isacharunderscore}{\kern0pt}measurable\ {\isacharparenleft}{\kern0pt}F\ {\isacharparenleft}{\kern0pt}i\ {\isacharminus}{\kern0pt}\ {\isadigit{1}}{\isacharparenright}{\kern0pt}{\isacharparenright}{\kern0pt}{\isachardoublequoteclose}\isanewline
%
\isadelimproof
%
\endisadelimproof
%
\isatagproof
\isacommand{proof}\isamarkupfalse%
\ {\isacharparenleft}{\kern0pt}cases\ i{\isacharparenright}{\kern0pt}\isanewline
\ \ \isacommand{case}\isamarkupfalse%
\ {\isadigit{0}}\isanewline
\ \ \isacommand{{\isacharbraceleft}{\kern0pt}}\isamarkupfalse%
\isanewline
\ \ \ \ \isacommand{fix}\isamarkupfalse%
\ S\ {\isacharcolon}{\kern0pt}{\isacharcolon}{\kern0pt}\ {\isachardoublequoteopen}{\isacharprime}{\kern0pt}b\ set{\isachardoublequoteclose}\ \isacommand{assume}\isamarkupfalse%
\ open{\isacharunderscore}{\kern0pt}S{\isacharcolon}{\kern0pt}\ {\isachardoublequoteopen}open\ S{\isachardoublequoteclose}\isanewline
\ \ \ \ \isacommand{hence}\isamarkupfalse%
\ {\isachardoublequoteopen}{\isacharbraceleft}{\kern0pt}{\isadigit{0}}{\isacharbraceright}{\kern0pt}\ {\isasymtimes}\ space\ M\ {\isasymin}\ predictable{\isacharunderscore}{\kern0pt}sigma{\isachardoublequoteclose}\ \isacommand{by}\isamarkupfalse%
\ {\isacharparenleft}{\kern0pt}intro\ in{\isacharunderscore}{\kern0pt}predictable{\isacharunderscore}{\kern0pt}sigmaI{\isacharbrackleft}{\kern0pt}of\ {\isachardoublequoteopen}{\isacharbraceleft}{\kern0pt}{\isadigit{0}}{\isacharbraceright}{\kern0pt}{\isachardoublequoteclose}{\isacharbrackright}{\kern0pt}{\isacharparenright}{\kern0pt}\ {\isacharparenleft}{\kern0pt}auto\ simp\ add{\isacharcolon}{\kern0pt}\ space{\isacharunderscore}{\kern0pt}F{\isacharbrackleft}{\kern0pt}symmetric{\isacharcomma}{\kern0pt}\ of\ bot{\isacharbrackright}{\kern0pt}{\isacharparenright}{\kern0pt}\isanewline
\ \ \ \ \isacommand{moreover}\isamarkupfalse%
\ \isacommand{have}\isamarkupfalse%
\ {\isachardoublequoteopen}case{\isacharunderscore}{\kern0pt}prod\ X\ {\isacharminus}{\kern0pt}{\isacharbackquote}{\kern0pt}\ S\ {\isasyminter}\ {\isacharparenleft}{\kern0pt}UNIV\ {\isasymtimes}\ space\ M{\isacharparenright}{\kern0pt}\ {\isasymin}\ predictable{\isacharunderscore}{\kern0pt}sigma{\isachardoublequoteclose}\ \isacommand{using}\isamarkupfalse%
\ open{\isacharunderscore}{\kern0pt}S\ \isacommand{by}\isamarkupfalse%
\ {\isacharparenleft}{\kern0pt}intro\ predictableD{\isacharbrackleft}{\kern0pt}OF\ assms{\isacharbrackright}{\kern0pt}{\isacharcomma}{\kern0pt}\ simp\ add{\isacharcolon}{\kern0pt}\ borel{\isacharunderscore}{\kern0pt}open{\isacharparenright}{\kern0pt}\ \ \isanewline
\ \ \ \ \isacommand{ultimately}\isamarkupfalse%
\ \isacommand{have}\isamarkupfalse%
\ {\isachardoublequoteopen}case{\isacharunderscore}{\kern0pt}prod\ X\ {\isacharminus}{\kern0pt}{\isacharbackquote}{\kern0pt}\ S\ {\isasyminter}\ {\isacharparenleft}{\kern0pt}{\isacharbraceleft}{\kern0pt}{\isadigit{0}}{\isacharbraceright}{\kern0pt}\ {\isasymtimes}\ space\ M{\isacharparenright}{\kern0pt}\ {\isasymin}\ predictable{\isacharunderscore}{\kern0pt}sigma{\isachardoublequoteclose}\ \isacommand{unfolding}\isamarkupfalse%
\ sets{\isacharunderscore}{\kern0pt}predictable{\isacharunderscore}{\kern0pt}sigma\ \isacommand{using}\isamarkupfalse%
\ space{\isacharunderscore}{\kern0pt}F\ sets{\isachardot}{\kern0pt}sets{\isacharunderscore}{\kern0pt}into{\isacharunderscore}{\kern0pt}space\isanewline
\ \ \ \ \ \ \isacommand{by}\isamarkupfalse%
\ {\isacharparenleft}{\kern0pt}subst\ Times{\isacharunderscore}{\kern0pt}Int{\isacharunderscore}{\kern0pt}distrib{\isadigit{1}}{\isacharbrackleft}{\kern0pt}of\ {\isachardoublequoteopen}{\isacharbraceleft}{\kern0pt}{\isadigit{0}}{\isacharbraceright}{\kern0pt}{\isachardoublequoteclose}\ UNIV\ {\isachardoublequoteopen}space\ M{\isachardoublequoteclose}{\isacharcomma}{\kern0pt}\ simplified{\isacharbrackright}{\kern0pt}{\isacharcomma}{\kern0pt}\ subst\ inf{\isachardot}{\kern0pt}commute{\isacharbrackleft}{\kern0pt}of\ {\isachardoublequoteopen}{\isacharunderscore}{\kern0pt}\ {\isasymtimes}\ {\isacharunderscore}{\kern0pt}{\isachardoublequoteclose}{\isacharbrackright}{\kern0pt}{\isacharcomma}{\kern0pt}\ subst\ Int{\isacharunderscore}{\kern0pt}assoc{\isacharbrackleft}{\kern0pt}symmetric{\isacharbrackright}{\kern0pt}{\isacharcomma}{\kern0pt}\ subst\ Int{\isacharunderscore}{\kern0pt}range{\isacharunderscore}{\kern0pt}binary{\isacharparenright}{\kern0pt}\ \isanewline
\ \ \ \ \ \ \ \ \ {\isacharparenleft}{\kern0pt}intro\ sigma{\isacharunderscore}{\kern0pt}sets{\isacharunderscore}{\kern0pt}Inter\ binary{\isacharunderscore}{\kern0pt}in{\isacharunderscore}{\kern0pt}sigma{\isacharunderscore}{\kern0pt}sets{\isacharcomma}{\kern0pt}\ fast{\isacharparenright}{\kern0pt}{\isacharplus}{\kern0pt}\isanewline
\ \ \ \ \isacommand{moreover}\isamarkupfalse%
\ \isacommand{have}\isamarkupfalse%
\ {\isachardoublequoteopen}case{\isacharunderscore}{\kern0pt}prod\ X\ {\isacharminus}{\kern0pt}{\isacharbackquote}{\kern0pt}\ S\ {\isasyminter}\ {\isacharparenleft}{\kern0pt}{\isacharbraceleft}{\kern0pt}{\isadigit{0}}{\isacharbraceright}{\kern0pt}\ {\isasymtimes}\ space\ M{\isacharparenright}{\kern0pt}\ {\isacharequal}{\kern0pt}\ {\isacharbraceleft}{\kern0pt}{\isadigit{0}}{\isacharbraceright}{\kern0pt}\ {\isasymtimes}\ {\isacharparenleft}{\kern0pt}X\ {\isadigit{0}}\ {\isacharminus}{\kern0pt}{\isacharbackquote}{\kern0pt}\ S\ {\isasyminter}\ space\ M{\isacharparenright}{\kern0pt}{\isachardoublequoteclose}\ \isacommand{by}\isamarkupfalse%
\ {\isacharparenleft}{\kern0pt}auto\ simp\ add{\isacharcolon}{\kern0pt}\ le{\isacharunderscore}{\kern0pt}Suc{\isacharunderscore}{\kern0pt}eq{\isacharparenright}{\kern0pt}\isanewline
\ \ \ \ \isacommand{moreover}\isamarkupfalse%
\ \isacommand{have}\isamarkupfalse%
\ {\isachardoublequoteopen}{\isachardot}{\kern0pt}{\isachardot}{\kern0pt}{\isachardot}{\kern0pt}\ {\isacharequal}{\kern0pt}\ {\isacharparenleft}{\kern0pt}{\isasymUnion}i{\isachardot}{\kern0pt}\ {\isacharbraceleft}{\kern0pt}i{\isacharbraceright}{\kern0pt}\ {\isasymtimes}\ {\isacharparenleft}{\kern0pt}if\ i\ {\isacharequal}{\kern0pt}\ {\isadigit{0}}\ then\ X\ {\isadigit{0}}\ {\isacharminus}{\kern0pt}{\isacharbackquote}{\kern0pt}\ S\ {\isasyminter}\ space\ M\ else\ {\isacharbraceleft}{\kern0pt}{\isacharbraceright}{\kern0pt}{\isacharparenright}{\kern0pt}{\isacharparenright}{\kern0pt}{\isachardoublequoteclose}\ \isacommand{by}\isamarkupfalse%
\ {\isacharparenleft}{\kern0pt}auto\ split{\isacharcolon}{\kern0pt}\ if{\isacharunderscore}{\kern0pt}splits{\isacharparenright}{\kern0pt}\isanewline
\ \ \ \ \isacommand{ultimately}\isamarkupfalse%
\ \isacommand{have}\isamarkupfalse%
\ {\isachardoublequoteopen}{\isacharparenleft}{\kern0pt}{\isasymUnion}i{\isachardot}{\kern0pt}\ {\isacharbraceleft}{\kern0pt}i{\isacharbraceright}{\kern0pt}\ {\isasymtimes}\ {\isacharparenleft}{\kern0pt}if\ i\ {\isacharequal}{\kern0pt}\ {\isadigit{0}}\ then\ X\ {\isadigit{0}}\ {\isacharminus}{\kern0pt}{\isacharbackquote}{\kern0pt}\ S\ {\isasyminter}\ space\ M\ else\ {\isacharbraceleft}{\kern0pt}{\isacharbraceright}{\kern0pt}{\isacharparenright}{\kern0pt}{\isacharparenright}{\kern0pt}\ {\isasymin}\ predictable{\isacharunderscore}{\kern0pt}sigma{\isachardoublequoteclose}\ \isacommand{by}\isamarkupfalse%
\ argo\isanewline
\ \ \ \ \isacommand{then}\isamarkupfalse%
\ \isacommand{have}\isamarkupfalse%
\ {\isachardoublequoteopen}X\ {\isadigit{0}}\ {\isacharminus}{\kern0pt}{\isacharbackquote}{\kern0pt}\ S\ {\isasyminter}\ space\ M\ {\isasymin}\ sets\ {\isacharparenleft}{\kern0pt}F\ {\isadigit{0}}{\isacharparenright}{\kern0pt}{\isachardoublequoteclose}\ \isacommand{using}\isamarkupfalse%
\ predictable{\isacharunderscore}{\kern0pt}sets{\isacharunderscore}{\kern0pt}in{\isacharunderscore}{\kern0pt}F{\isacharbrackleft}{\kern0pt}of\ {\isachardoublequoteopen}{\isasymlambda}i{\isachardot}{\kern0pt}\ if\ i\ {\isacharequal}{\kern0pt}\ {\isadigit{0}}\ then\ X\ {\isadigit{0}}\ {\isacharminus}{\kern0pt}{\isacharbackquote}{\kern0pt}\ S\ {\isasyminter}\ space\ M\ else\ {\isacharbraceleft}{\kern0pt}{\isacharbraceright}{\kern0pt}{\isachardoublequoteclose}{\isacharbrackright}{\kern0pt}\ \isacommand{by}\isamarkupfalse%
\ presburger\isanewline
\ \ \isacommand{{\isacharbraceright}{\kern0pt}}\isamarkupfalse%
\isanewline
\ \ \isacommand{hence}\isamarkupfalse%
\ {\isachardoublequoteopen}X\ {\isadigit{0}}\ {\isasymin}\ borel{\isacharunderscore}{\kern0pt}measurable\ {\isacharparenleft}{\kern0pt}F\ {\isadigit{0}}{\isacharparenright}{\kern0pt}{\isachardoublequoteclose}\ \isacommand{by}\isamarkupfalse%
\ {\isacharparenleft}{\kern0pt}fastforce\ simp\ add{\isacharcolon}{\kern0pt}\ bot{\isacharunderscore}{\kern0pt}nat{\isacharunderscore}{\kern0pt}def\ space{\isacharunderscore}{\kern0pt}F\ intro{\isacharbang}{\kern0pt}{\isacharcolon}{\kern0pt}\ borel{\isacharunderscore}{\kern0pt}measurableI{\isacharparenright}{\kern0pt}\isanewline
\ \ \isacommand{thus}\isamarkupfalse%
\ {\isacharquery}{\kern0pt}thesis\ \isacommand{using}\isamarkupfalse%
\ {\isadigit{0}}\ \isacommand{by}\isamarkupfalse%
\ force\isanewline
\isacommand{next}\isamarkupfalse%
\isanewline
\ \ \isacommand{case}\isamarkupfalse%
\ {\isacharparenleft}{\kern0pt}Suc\ i{\isacharparenright}{\kern0pt}\isanewline
\ \ \isacommand{{\isacharbraceleft}{\kern0pt}}\isamarkupfalse%
\isanewline
\ \ \ \ \isacommand{fix}\isamarkupfalse%
\ S\ {\isacharcolon}{\kern0pt}{\isacharcolon}{\kern0pt}\ {\isachardoublequoteopen}{\isacharprime}{\kern0pt}b\ set{\isachardoublequoteclose}\ \isacommand{assume}\isamarkupfalse%
\ open{\isacharunderscore}{\kern0pt}S{\isacharcolon}{\kern0pt}\ {\isachardoublequoteopen}open\ S{\isachardoublequoteclose}\isanewline
\ \ \ \ \isacommand{hence}\isamarkupfalse%
\ {\isachardoublequoteopen}{\isacharbraceleft}{\kern0pt}Suc\ i{\isacharbraceright}{\kern0pt}\ {\isasymtimes}\ space\ M\ {\isasymin}\ predictable{\isacharunderscore}{\kern0pt}sigma{\isachardoublequoteclose}\ \isacommand{by}\isamarkupfalse%
\ {\isacharparenleft}{\kern0pt}intro\ in{\isacharunderscore}{\kern0pt}predictable{\isacharunderscore}{\kern0pt}sigmaI{\isacharbrackleft}{\kern0pt}of\ {\isachardoublequoteopen}{\isacharbraceleft}{\kern0pt}Suc\ i{\isacharbraceright}{\kern0pt}{\isachardoublequoteclose}\ {\isacharunderscore}{\kern0pt}\ {\isachardoublequoteopen}{\isasymlambda}{\isacharunderscore}{\kern0pt}\ {\isacharunderscore}{\kern0pt}{\isachardot}{\kern0pt}\ i{\isachardoublequoteclose}\ {\isachardoublequoteopen}{\isasymlambda}{\isacharunderscore}{\kern0pt}\ {\isacharunderscore}{\kern0pt}{\isachardot}{\kern0pt}\ Suc\ i{\isachardoublequoteclose}{\isacharbrackright}{\kern0pt}{\isacharparenright}{\kern0pt}\ {\isacharparenleft}{\kern0pt}force\ simp\ add{\isacharcolon}{\kern0pt}\ space{\isacharunderscore}{\kern0pt}F{\isacharbrackleft}{\kern0pt}symmetric{\isacharcomma}{\kern0pt}\ of\ bot{\isacharbrackright}{\kern0pt}{\isacharcomma}{\kern0pt}\ fastforce\ simp\ add{\isacharcolon}{\kern0pt}\ space{\isacharunderscore}{\kern0pt}F{\isacharbrackleft}{\kern0pt}symmetric{\isacharcomma}{\kern0pt}\ of\ i{\isacharbrackright}{\kern0pt}{\isacharparenright}{\kern0pt}\isanewline
\ \ \ \ \isacommand{moreover}\isamarkupfalse%
\ \isacommand{have}\isamarkupfalse%
\ {\isachardoublequoteopen}case{\isacharunderscore}{\kern0pt}prod\ X\ {\isacharminus}{\kern0pt}{\isacharbackquote}{\kern0pt}\ S\ {\isasyminter}\ {\isacharparenleft}{\kern0pt}UNIV\ {\isasymtimes}\ space\ M{\isacharparenright}{\kern0pt}\ {\isasymin}\ predictable{\isacharunderscore}{\kern0pt}sigma{\isachardoublequoteclose}\ \isacommand{using}\isamarkupfalse%
\ open{\isacharunderscore}{\kern0pt}S\ \isacommand{by}\isamarkupfalse%
\ {\isacharparenleft}{\kern0pt}intro\ predictableD{\isacharbrackleft}{\kern0pt}OF\ assms{\isacharbrackright}{\kern0pt}{\isacharcomma}{\kern0pt}\ simp\ add{\isacharcolon}{\kern0pt}\ borel{\isacharunderscore}{\kern0pt}open{\isacharparenright}{\kern0pt}\isanewline
\ \ \ \ \isacommand{ultimately}\isamarkupfalse%
\ \isacommand{have}\isamarkupfalse%
\ {\isachardoublequoteopen}case{\isacharunderscore}{\kern0pt}prod\ X\ {\isacharminus}{\kern0pt}{\isacharbackquote}{\kern0pt}\ S\ {\isasyminter}\ {\isacharparenleft}{\kern0pt}{\isacharbraceleft}{\kern0pt}Suc\ i{\isacharbraceright}{\kern0pt}\ {\isasymtimes}\ space\ M{\isacharparenright}{\kern0pt}\ {\isasymin}\ predictable{\isacharunderscore}{\kern0pt}sigma{\isachardoublequoteclose}\ \isacommand{unfolding}\isamarkupfalse%
\ sets{\isacharunderscore}{\kern0pt}predictable{\isacharunderscore}{\kern0pt}sigma\ \isacommand{using}\isamarkupfalse%
\ space{\isacharunderscore}{\kern0pt}F\ sets{\isachardot}{\kern0pt}sets{\isacharunderscore}{\kern0pt}into{\isacharunderscore}{\kern0pt}space\isanewline
\ \ \ \ \ \ \isacommand{by}\isamarkupfalse%
\ {\isacharparenleft}{\kern0pt}subst\ Times{\isacharunderscore}{\kern0pt}Int{\isacharunderscore}{\kern0pt}distrib{\isadigit{1}}{\isacharbrackleft}{\kern0pt}of\ {\isachardoublequoteopen}{\isacharbraceleft}{\kern0pt}Suc\ i{\isacharbraceright}{\kern0pt}{\isachardoublequoteclose}\ UNIV\ {\isachardoublequoteopen}space\ M{\isachardoublequoteclose}{\isacharcomma}{\kern0pt}\ simplified{\isacharbrackright}{\kern0pt}{\isacharcomma}{\kern0pt}\ subst\ inf{\isachardot}{\kern0pt}commute{\isacharbrackleft}{\kern0pt}of\ {\isachardoublequoteopen}{\isacharunderscore}{\kern0pt}\ {\isasymtimes}\ {\isacharunderscore}{\kern0pt}{\isachardoublequoteclose}{\isacharbrackright}{\kern0pt}{\isacharcomma}{\kern0pt}\ subst\ Int{\isacharunderscore}{\kern0pt}assoc{\isacharbrackleft}{\kern0pt}symmetric{\isacharbrackright}{\kern0pt}{\isacharcomma}{\kern0pt}\ subst\ Int{\isacharunderscore}{\kern0pt}range{\isacharunderscore}{\kern0pt}binary{\isacharparenright}{\kern0pt}\ \isanewline
\ \ \ \ \ \ \ \ \ {\isacharparenleft}{\kern0pt}intro\ sigma{\isacharunderscore}{\kern0pt}sets{\isacharunderscore}{\kern0pt}Inter\ binary{\isacharunderscore}{\kern0pt}in{\isacharunderscore}{\kern0pt}sigma{\isacharunderscore}{\kern0pt}sets{\isacharcomma}{\kern0pt}\ fast{\isacharparenright}{\kern0pt}{\isacharplus}{\kern0pt}\isanewline
\ \ \ \ \isacommand{moreover}\isamarkupfalse%
\ \isacommand{have}\isamarkupfalse%
\ {\isachardoublequoteopen}case{\isacharunderscore}{\kern0pt}prod\ X\ {\isacharminus}{\kern0pt}{\isacharbackquote}{\kern0pt}\ S\ {\isasyminter}\ {\isacharparenleft}{\kern0pt}{\isacharbraceleft}{\kern0pt}Suc\ i{\isacharbraceright}{\kern0pt}\ {\isasymtimes}\ space\ M{\isacharparenright}{\kern0pt}\ {\isacharequal}{\kern0pt}\ {\isacharbraceleft}{\kern0pt}Suc\ i{\isacharbraceright}{\kern0pt}\ {\isasymtimes}\ {\isacharparenleft}{\kern0pt}X\ {\isacharparenleft}{\kern0pt}Suc\ i{\isacharparenright}{\kern0pt}\ {\isacharminus}{\kern0pt}{\isacharbackquote}{\kern0pt}\ S\ {\isasyminter}\ space\ M{\isacharparenright}{\kern0pt}{\isachardoublequoteclose}\ \isacommand{by}\isamarkupfalse%
\ {\isacharparenleft}{\kern0pt}auto\ simp\ add{\isacharcolon}{\kern0pt}\ le{\isacharunderscore}{\kern0pt}Suc{\isacharunderscore}{\kern0pt}eq{\isacharparenright}{\kern0pt}\isanewline
\ \ \ \ \isacommand{moreover}\isamarkupfalse%
\ \isacommand{have}\isamarkupfalse%
\ {\isachardoublequoteopen}{\isachardot}{\kern0pt}{\isachardot}{\kern0pt}{\isachardot}{\kern0pt}\ {\isacharequal}{\kern0pt}\ {\isacharparenleft}{\kern0pt}{\isasymUnion}j{\isachardot}{\kern0pt}\ {\isacharbraceleft}{\kern0pt}j{\isacharbraceright}{\kern0pt}\ {\isasymtimes}\ {\isacharparenleft}{\kern0pt}if\ j\ {\isacharequal}{\kern0pt}\ Suc\ i\ then\ {\isacharparenleft}{\kern0pt}X\ {\isacharparenleft}{\kern0pt}Suc\ i{\isacharparenright}{\kern0pt}\ {\isacharminus}{\kern0pt}{\isacharbackquote}{\kern0pt}\ S\ {\isasyminter}\ space\ M{\isacharparenright}{\kern0pt}\ else\ {\isacharbraceleft}{\kern0pt}{\isacharbraceright}{\kern0pt}{\isacharparenright}{\kern0pt}{\isacharparenright}{\kern0pt}{\isachardoublequoteclose}\ \isacommand{by}\isamarkupfalse%
\ {\isacharparenleft}{\kern0pt}auto\ split{\isacharcolon}{\kern0pt}\ if{\isacharunderscore}{\kern0pt}splits{\isacharparenright}{\kern0pt}\isanewline
\ \ \ \ \isacommand{ultimately}\isamarkupfalse%
\ \isacommand{have}\isamarkupfalse%
\ {\isachardoublequoteopen}{\isacharparenleft}{\kern0pt}{\isasymUnion}j{\isachardot}{\kern0pt}\ {\isacharbraceleft}{\kern0pt}j{\isacharbraceright}{\kern0pt}\ {\isasymtimes}\ {\isacharparenleft}{\kern0pt}if\ j\ {\isacharequal}{\kern0pt}\ Suc\ i\ then\ {\isacharparenleft}{\kern0pt}X\ {\isacharparenleft}{\kern0pt}Suc\ i{\isacharparenright}{\kern0pt}\ {\isacharminus}{\kern0pt}{\isacharbackquote}{\kern0pt}\ S\ {\isasyminter}\ space\ M{\isacharparenright}{\kern0pt}\ else\ {\isacharbraceleft}{\kern0pt}{\isacharbraceright}{\kern0pt}{\isacharparenright}{\kern0pt}{\isacharparenright}{\kern0pt}\ {\isasymin}\ predictable{\isacharunderscore}{\kern0pt}sigma{\isachardoublequoteclose}\ \isacommand{by}\isamarkupfalse%
\ argo\isanewline
\ \ \ \ \isacommand{then}\isamarkupfalse%
\ \isacommand{have}\isamarkupfalse%
\ {\isachardoublequoteopen}X\ {\isacharparenleft}{\kern0pt}Suc\ i{\isacharparenright}{\kern0pt}\ {\isacharminus}{\kern0pt}{\isacharbackquote}{\kern0pt}\ S\ {\isasyminter}\ space\ M\ {\isasymin}\ sets\ {\isacharparenleft}{\kern0pt}F\ i{\isacharparenright}{\kern0pt}{\isachardoublequoteclose}\ \isacommand{using}\isamarkupfalse%
\ predictable{\isacharunderscore}{\kern0pt}sets{\isacharunderscore}{\kern0pt}in{\isacharunderscore}{\kern0pt}F{\isacharbrackleft}{\kern0pt}of\ {\isachardoublequoteopen}{\isasymlambda}j{\isachardot}{\kern0pt}\ if\ j\ {\isacharequal}{\kern0pt}\ Suc\ i\ then\ {\isacharparenleft}{\kern0pt}X\ {\isacharparenleft}{\kern0pt}Suc\ i{\isacharparenright}{\kern0pt}\ {\isacharminus}{\kern0pt}{\isacharbackquote}{\kern0pt}\ S\ {\isasyminter}\ space\ M{\isacharparenright}{\kern0pt}\ else\ {\isacharbraceleft}{\kern0pt}{\isacharbraceright}{\kern0pt}{\isachardoublequoteclose}{\isacharbrackright}{\kern0pt}\ \isacommand{by}\isamarkupfalse%
\ presburger\isanewline
\ \ \isacommand{{\isacharbraceright}{\kern0pt}}\isamarkupfalse%
\isanewline
\ \ \isacommand{hence}\isamarkupfalse%
\ {\isachardoublequoteopen}X\ {\isacharparenleft}{\kern0pt}Suc\ i{\isacharparenright}{\kern0pt}\ {\isasymin}\ borel{\isacharunderscore}{\kern0pt}measurable\ {\isacharparenleft}{\kern0pt}F\ i{\isacharparenright}{\kern0pt}{\isachardoublequoteclose}\ \isacommand{by}\isamarkupfalse%
\ {\isacharparenleft}{\kern0pt}fastforce\ simp\ add{\isacharcolon}{\kern0pt}\ space{\isacharunderscore}{\kern0pt}F\ intro{\isacharbang}{\kern0pt}{\isacharcolon}{\kern0pt}\ borel{\isacharunderscore}{\kern0pt}measurableI{\isacharparenright}{\kern0pt}\isanewline
\ \ \isacommand{then}\isamarkupfalse%
\ \isacommand{show}\isamarkupfalse%
\ {\isacharquery}{\kern0pt}thesis\ \isacommand{using}\isamarkupfalse%
\ Suc\ \isacommand{by}\isamarkupfalse%
\ force\isanewline
\isacommand{qed}\isamarkupfalse%
%
\endisatagproof
{\isafoldproof}%
%
\isadelimproof
\isanewline
%
\endisadelimproof
\isanewline
\isacommand{end}\isamarkupfalse%
\isanewline
\isanewline
\isanewline
%
\isadelimtheory
\isanewline
%
\endisadelimtheory
%
\isatagtheory
\isacommand{end}\isamarkupfalse%
%
\endisatagtheory
{\isafoldtheory}%
%
\isadelimtheory
%
\endisadelimtheory
%
\end{isabellebody}%
\endinput
%:%file=Stochastic_Process.tex%:%
%:%10=1%:%
%:%11=1%:%
%:%12=2%:%
%:%13=3%:%
%:%27=5%:%
%:%37=7%:%
%:%38=7%:%
%:%39=8%:%
%:%40=9%:%
%:%41=10%:%
%:%42=11%:%
%:%43=12%:%
%:%44=12%:%
%:%45=13%:%
%:%46=13%:%
%:%47=14%:%
%:%48=15%:%
%:%49=15%:%
%:%50=16%:%
%:%51=17%:%
%:%54=18%:%
%:%58=18%:%
%:%59=18%:%
%:%64=18%:%
%:%67=19%:%
%:%68=20%:%
%:%69=20%:%
%:%71=20%:%
%:%75=20%:%
%:%76=20%:%
%:%83=20%:%
%:%84=21%:%
%:%85=22%:%
%:%86=22%:%
%:%87=23%:%
%:%88=24%:%
%:%91=25%:%
%:%95=25%:%
%:%96=25%:%
%:%101=25%:%
%:%104=26%:%
%:%105=27%:%
%:%106=27%:%
%:%107=28%:%
%:%108=29%:%
%:%111=30%:%
%:%115=30%:%
%:%116=30%:%
%:%121=30%:%
%:%124=31%:%
%:%125=32%:%
%:%126=32%:%
%:%128=32%:%
%:%132=32%:%
%:%133=32%:%
%:%140=32%:%
%:%141=33%:%
%:%142=34%:%
%:%143=34%:%
%:%144=35%:%
%:%145=36%:%
%:%148=37%:%
%:%152=37%:%
%:%153=37%:%
%:%158=37%:%
%:%161=38%:%
%:%162=39%:%
%:%163=39%:%
%:%164=40%:%
%:%165=41%:%
%:%168=42%:%
%:%172=42%:%
%:%173=42%:%
%:%178=42%:%
%:%181=43%:%
%:%182=44%:%
%:%183=44%:%
%:%185=44%:%
%:%189=44%:%
%:%190=44%:%
%:%191=44%:%
%:%198=44%:%
%:%199=45%:%
%:%200=46%:%
%:%208=48%:%
%:%218=50%:%
%:%219=50%:%
%:%220=51%:%
%:%221=52%:%
%:%222=53%:%
%:%223=54%:%
%:%224=54%:%
%:%225=55%:%
%:%226=56%:%
%:%229=57%:%
%:%233=57%:%
%:%234=57%:%
%:%235=57%:%
%:%240=57%:%
%:%243=58%:%
%:%244=59%:%
%:%245=59%:%
%:%246=60%:%
%:%247=61%:%
%:%250=62%:%
%:%254=62%:%
%:%255=62%:%
%:%260=62%:%
%:%263=63%:%
%:%264=64%:%
%:%265=64%:%
%:%267=64%:%
%:%271=64%:%
%:%272=64%:%
%:%279=64%:%
%:%280=65%:%
%:%281=66%:%
%:%282=66%:%
%:%283=67%:%
%:%284=68%:%
%:%291=69%:%
%:%292=69%:%
%:%293=70%:%
%:%294=70%:%
%:%295=70%:%
%:%296=71%:%
%:%297=71%:%
%:%298=71%:%
%:%299=72%:%
%:%305=72%:%
%:%308=73%:%
%:%309=74%:%
%:%310=74%:%
%:%311=75%:%
%:%312=76%:%
%:%315=77%:%
%:%319=77%:%
%:%320=77%:%
%:%321=77%:%
%:%326=77%:%
%:%329=78%:%
%:%330=79%:%
%:%331=79%:%
%:%333=79%:%
%:%337=79%:%
%:%338=79%:%
%:%345=79%:%
%:%346=80%:%
%:%347=81%:%
%:%348=81%:%
%:%349=82%:%
%:%350=83%:%
%:%357=84%:%
%:%358=84%:%
%:%359=85%:%
%:%360=85%:%
%:%361=85%:%
%:%362=86%:%
%:%363=86%:%
%:%364=86%:%
%:%365=87%:%
%:%371=87%:%
%:%374=88%:%
%:%375=89%:%
%:%376=89%:%
%:%377=90%:%
%:%378=91%:%
%:%385=92%:%
%:%386=92%:%
%:%387=93%:%
%:%388=93%:%
%:%389=93%:%
%:%390=94%:%
%:%391=94%:%
%:%392=94%:%
%:%393=95%:%
%:%399=95%:%
%:%402=96%:%
%:%403=97%:%
%:%404=97%:%
%:%406=97%:%
%:%410=97%:%
%:%411=97%:%
%:%412=97%:%
%:%419=97%:%
%:%420=98%:%
%:%421=99%:%
%:%422=99%:%
%:%423=100%:%
%:%424=101%:%
%:%425=101%:%
%:%432=103%:%
%:%442=105%:%
%:%443=105%:%
%:%444=106%:%
%:%445=106%:%
%:%446=107%:%
%:%447=107%:%
%:%448=108%:%
%:%449=109%:%
%:%450=109%:%
%:%452=109%:%
%:%456=109%:%
%:%457=109%:%
%:%464=109%:%
%:%465=110%:%
%:%466=110%:%
%:%468=110%:%
%:%472=110%:%
%:%473=110%:%
%:%480=110%:%
%:%481=111%:%
%:%482=111%:%
%:%484=111%:%
%:%488=111%:%
%:%489=111%:%
%:%496=111%:%
%:%497=112%:%
%:%498=113%:%
%:%499=113%:%
%:%500=114%:%
%:%501=115%:%
%:%502=116%:%
%:%503=116%:%
%:%504=117%:%
%:%505=118%:%
%:%506=119%:%
%:%507=119%:%
%:%509=119%:%
%:%513=119%:%
%:%514=119%:%
%:%515=119%:%
%:%522=119%:%
%:%523=120%:%
%:%524=121%:%
%:%525=121%:%
%:%528=122%:%
%:%532=122%:%
%:%533=122%:%
%:%534=123%:%
%:%535=123%:%
%:%536=124%:%
%:%537=124%:%
%:%542=124%:%
%:%545=125%:%
%:%546=126%:%
%:%547=126%:%
%:%548=127%:%
%:%549=128%:%
%:%556=129%:%
%:%557=129%:%
%:%558=130%:%
%:%559=130%:%
%:%560=131%:%
%:%561=131%:%
%:%562=131%:%
%:%563=132%:%
%:%564=132%:%
%:%565=133%:%
%:%566=133%:%
%:%567=134%:%
%:%568=134%:%
%:%569=135%:%
%:%570=135%:%
%:%571=135%:%
%:%572=135%:%
%:%573=136%:%
%:%574=136%:%
%:%575=136%:%
%:%576=136%:%
%:%577=136%:%
%:%578=137%:%
%:%579=137%:%
%:%580=138%:%
%:%581=138%:%
%:%582=139%:%
%:%583=139%:%
%:%584=140%:%
%:%585=140%:%
%:%586=140%:%
%:%587=140%:%
%:%588=141%:%
%:%589=141%:%
%:%590=141%:%
%:%591=141%:%
%:%592=142%:%
%:%593=142%:%
%:%594=143%:%
%:%595=143%:%
%:%596=144%:%
%:%597=144%:%
%:%598=145%:%
%:%599=145%:%
%:%600=145%:%
%:%601=146%:%
%:%602=146%:%
%:%603=146%:%
%:%604=146%:%
%:%605=147%:%
%:%606=147%:%
%:%607=148%:%
%:%608=148%:%
%:%609=148%:%
%:%610=148%:%
%:%611=149%:%
%:%612=149%:%
%:%613=150%:%
%:%619=150%:%
%:%622=151%:%
%:%623=152%:%
%:%624=152%:%
%:%625=153%:%
%:%626=154%:%
%:%627=155%:%
%:%628=155%:%
%:%629=156%:%
%:%630=157%:%
%:%631=157%:%
%:%632=158%:%
%:%633=159%:%
%:%634=160%:%
%:%637=161%:%
%:%641=161%:%
%:%642=161%:%
%:%643=161%:%
%:%644=162%:%
%:%645=162%:%
%:%646=163%:%
%:%647=163%:%
%:%648=164%:%
%:%649=164%:%
%:%650=165%:%
%:%651=165%:%
%:%652=166%:%
%:%653=166%:%
%:%654=166%:%
%:%655=166%:%
%:%656=167%:%
%:%657=167%:%
%:%658=167%:%
%:%659=167%:%
%:%660=168%:%
%:%661=168%:%
%:%662=168%:%
%:%663=168%:%
%:%664=168%:%
%:%665=169%:%
%:%666=169%:%
%:%667=169%:%
%:%668=169%:%
%:%669=169%:%
%:%670=170%:%
%:%671=170%:%
%:%672=170%:%
%:%673=170%:%
%:%674=170%:%
%:%675=171%:%
%:%676=171%:%
%:%677=172%:%
%:%678=172%:%
%:%679=173%:%
%:%680=173%:%
%:%681=174%:%
%:%682=174%:%
%:%683=175%:%
%:%684=175%:%
%:%685=175%:%
%:%686=175%:%
%:%687=175%:%
%:%688=176%:%
%:%689=176%:%
%:%690=176%:%
%:%691=176%:%
%:%692=177%:%
%:%693=177%:%
%:%694=177%:%
%:%695=177%:%
%:%696=177%:%
%:%697=178%:%
%:%698=178%:%
%:%699=178%:%
%:%700=178%:%
%:%701=178%:%
%:%702=178%:%
%:%703=179%:%
%:%704=179%:%
%:%705=180%:%
%:%706=180%:%
%:%707=181%:%
%:%708=181%:%
%:%709=181%:%
%:%710=181%:%
%:%711=182%:%
%:%712=182%:%
%:%713=183%:%
%:%714=183%:%
%:%715=184%:%
%:%716=184%:%
%:%717=185%:%
%:%718=185%:%
%:%719=186%:%
%:%720=186%:%
%:%721=186%:%
%:%722=186%:%
%:%723=186%:%
%:%724=187%:%
%:%725=187%:%
%:%726=188%:%
%:%727=188%:%
%:%728=189%:%
%:%729=189%:%
%:%730=189%:%
%:%731=189%:%
%:%732=189%:%
%:%733=190%:%
%:%734=190%:%
%:%735=191%:%
%:%736=191%:%
%:%737=192%:%
%:%738=192%:%
%:%739=193%:%
%:%740=193%:%
%:%741=193%:%
%:%742=193%:%
%:%743=194%:%
%:%744=194%:%
%:%745=194%:%
%:%746=194%:%
%:%747=195%:%
%:%748=195%:%
%:%749=195%:%
%:%750=195%:%
%:%751=196%:%
%:%752=196%:%
%:%753=196%:%
%:%754=196%:%
%:%755=197%:%
%:%756=197%:%
%:%757=197%:%
%:%758=197%:%
%:%759=197%:%
%:%760=198%:%
%:%761=198%:%
%:%762=199%:%
%:%763=199%:%
%:%764=200%:%
%:%765=200%:%
%:%766=200%:%
%:%767=200%:%
%:%768=200%:%
%:%769=201%:%
%:%770=201%:%
%:%771=202%:%
%:%772=202%:%
%:%773=203%:%
%:%774=203%:%
%:%775=203%:%
%:%776=203%:%
%:%777=203%:%
%:%778=204%:%
%:%779=204%:%
%:%780=205%:%
%:%781=205%:%
%:%782=206%:%
%:%783=206%:%
%:%784=207%:%
%:%785=207%:%
%:%786=207%:%
%:%787=207%:%
%:%788=208%:%
%:%789=208%:%
%:%790=208%:%
%:%791=208%:%
%:%792=209%:%
%:%793=209%:%
%:%794=209%:%
%:%795=210%:%
%:%796=210%:%
%:%797=210%:%
%:%798=210%:%
%:%799=210%:%
%:%800=211%:%
%:%801=211%:%
%:%802=211%:%
%:%803=211%:%
%:%804=212%:%
%:%805=212%:%
%:%806=213%:%
%:%807=213%:%
%:%808=214%:%
%:%809=214%:%
%:%810=214%:%
%:%811=214%:%
%:%812=215%:%
%:%813=215%:%
%:%814=215%:%
%:%815=215%:%
%:%816=216%:%
%:%817=216%:%
%:%818=216%:%
%:%819=216%:%
%:%820=216%:%
%:%821=217%:%
%:%822=217%:%
%:%823=218%:%
%:%824=218%:%
%:%825=219%:%
%:%826=219%:%
%:%827=219%:%
%:%828=219%:%
%:%829=220%:%
%:%830=220%:%
%:%831=220%:%
%:%832=220%:%
%:%833=221%:%
%:%834=221%:%
%:%835=221%:%
%:%836=221%:%
%:%837=221%:%
%:%838=222%:%
%:%839=222%:%
%:%840=223%:%
%:%846=223%:%
%:%849=224%:%
%:%850=225%:%
%:%851=225%:%
%:%852=226%:%
%:%853=227%:%
%:%860=228%:%
%:%861=228%:%
%:%862=229%:%
%:%863=229%:%
%:%864=230%:%
%:%865=230%:%
%:%866=231%:%
%:%867=231%:%
%:%868=231%:%
%:%869=232%:%
%:%870=232%:%
%:%871=232%:%
%:%872=233%:%
%:%873=233%:%
%:%874=233%:%
%:%875=233%:%
%:%876=233%:%
%:%877=234%:%
%:%878=234%:%
%:%879=234%:%
%:%880=234%:%
%:%881=234%:%
%:%882=235%:%
%:%883=235%:%
%:%884=236%:%
%:%885=237%:%
%:%886=237%:%
%:%887=237%:%
%:%888=237%:%
%:%889=238%:%
%:%890=238%:%
%:%891=238%:%
%:%892=238%:%
%:%893=239%:%
%:%894=239%:%
%:%895=239%:%
%:%896=239%:%
%:%897=240%:%
%:%898=240%:%
%:%899=240%:%
%:%900=240%:%
%:%901=240%:%
%:%902=241%:%
%:%903=241%:%
%:%904=242%:%
%:%905=242%:%
%:%906=242%:%
%:%907=243%:%
%:%908=243%:%
%:%909=243%:%
%:%910=243%:%
%:%911=244%:%
%:%912=244%:%
%:%913=245%:%
%:%914=245%:%
%:%915=246%:%
%:%916=246%:%
%:%917=247%:%
%:%918=247%:%
%:%919=247%:%
%:%920=248%:%
%:%921=248%:%
%:%922=248%:%
%:%923=249%:%
%:%924=249%:%
%:%925=249%:%
%:%926=249%:%
%:%927=249%:%
%:%928=250%:%
%:%929=250%:%
%:%930=250%:%
%:%931=250%:%
%:%932=250%:%
%:%933=251%:%
%:%934=251%:%
%:%935=252%:%
%:%936=253%:%
%:%937=253%:%
%:%938=253%:%
%:%939=253%:%
%:%940=254%:%
%:%941=254%:%
%:%942=254%:%
%:%943=254%:%
%:%944=255%:%
%:%945=255%:%
%:%946=255%:%
%:%947=255%:%
%:%948=256%:%
%:%949=256%:%
%:%950=256%:%
%:%951=256%:%
%:%952=256%:%
%:%953=257%:%
%:%954=257%:%
%:%955=258%:%
%:%956=258%:%
%:%957=258%:%
%:%958=259%:%
%:%959=259%:%
%:%960=259%:%
%:%961=259%:%
%:%962=259%:%
%:%963=260%:%
%:%969=260%:%
%:%972=261%:%
%:%973=262%:%
%:%974=262%:%
%:%975=263%:%
%:%976=264%:%
%:%979=265%:%
%:%984=266%:%

%
\begin{isabellebody}%
\setisabellecontext{Martingale}%
%
\isadelimtheory
%
\endisadelimtheory
%
\isatagtheory
\isacommand{theory}\isamarkupfalse%
\ Martingale\ \ \ \ \ \ \ \ \ \ \ \ \ \ \ \ \ \isanewline
\ \ \isakeyword{imports}\ Stochastic{\isacharunderscore}{\kern0pt}Process\ Banach{\isacharunderscore}{\kern0pt}Conditional{\isacharunderscore}{\kern0pt}Expectation\ {\isachardoublequoteopen}HOL{\isacharminus}{\kern0pt}Probability{\isachardot}{\kern0pt}Probability{\isachardoublequoteclose}\isanewline
\isakeyword{begin}%
\endisatagtheory
{\isafoldtheory}%
%
\isadelimtheory
%
\endisadelimtheory
%
\isadelimdocument
%
\endisadelimdocument
%
\isatagdocument
%
\isamarkupsubsection{Martingale%
}
\isamarkuptrue%
%
\endisatagdocument
{\isafolddocument}%
%
\isadelimdocument
%
\endisadelimdocument
\isacommand{locale}\isamarkupfalse%
\ martingale\ {\isacharequal}{\kern0pt}\ adapted{\isacharunderscore}{\kern0pt}process\ {\isacharplus}{\kern0pt}\isanewline
\ \ \isakeyword{assumes}\ integrable{\isacharcolon}{\kern0pt}\ {\isachardoublequoteopen}{\isasymAnd}i{\isachardot}{\kern0pt}\ integrable\ M\ {\isacharparenleft}{\kern0pt}X\ i{\isacharparenright}{\kern0pt}{\isachardoublequoteclose}\isanewline
\ \ \ \ \ \ \isakeyword{and}\ martingale{\isacharunderscore}{\kern0pt}property{\isacharcolon}{\kern0pt}\ {\isachardoublequoteopen}{\isasymAnd}i\ j{\isachardot}{\kern0pt}\ i\ {\isasymle}\ j\ {\isasymLongrightarrow}\ AE\ {\isasymxi}\ in\ M{\isachardot}{\kern0pt}\ X\ i\ {\isasymxi}\ {\isacharequal}{\kern0pt}\ cond{\isacharunderscore}{\kern0pt}exp\ M\ {\isacharparenleft}{\kern0pt}F\ i{\isacharparenright}{\kern0pt}\ {\isacharparenleft}{\kern0pt}X\ j{\isacharparenright}{\kern0pt}\ {\isasymxi}{\isachardoublequoteclose}\isanewline
\isanewline
\isacommand{lemma}\isamarkupfalse%
\ {\isacharparenleft}{\kern0pt}\isakeyword{in}\ filtered{\isacharunderscore}{\kern0pt}sigma{\isacharunderscore}{\kern0pt}finite{\isacharunderscore}{\kern0pt}measure{\isacharparenright}{\kern0pt}\ martingale{\isacharunderscore}{\kern0pt}const{\isacharbrackleft}{\kern0pt}intro{\isacharbrackright}{\kern0pt}{\isacharcolon}{\kern0pt}\ \ \isanewline
\ \ \isakeyword{assumes}\ {\isachardoublequoteopen}integrable\ M\ f{\isachardoublequoteclose}\ {\isachardoublequoteopen}f\ {\isasymin}\ borel{\isacharunderscore}{\kern0pt}measurable\ {\isacharparenleft}{\kern0pt}F\ {\isasymbottom}{\isacharparenright}{\kern0pt}{\isachardoublequoteclose}\isanewline
\ \ \isakeyword{shows}\ {\isachardoublequoteopen}martingale\ M\ F\ {\isacharparenleft}{\kern0pt}{\isasymlambda}{\isacharunderscore}{\kern0pt}{\isachardot}{\kern0pt}\ f{\isacharparenright}{\kern0pt}{\isachardoublequoteclose}\isanewline
%
\isadelimproof
\ \ %
\endisadelimproof
%
\isatagproof
\isacommand{using}\isamarkupfalse%
\ assms\ cond{\isacharunderscore}{\kern0pt}exp{\isacharunderscore}{\kern0pt}F{\isacharunderscore}{\kern0pt}meas{\isacharbrackleft}{\kern0pt}OF\ assms{\isacharparenleft}{\kern0pt}{\isadigit{1}}{\isacharparenright}{\kern0pt}{\isacharcomma}{\kern0pt}\ THEN\ AE{\isacharunderscore}{\kern0pt}symmetric{\isacharbrackright}{\kern0pt}\isanewline
\ \ \isacommand{by}\isamarkupfalse%
\ {\isacharparenleft}{\kern0pt}unfold{\isacharunderscore}{\kern0pt}locales{\isacharparenright}{\kern0pt}\isanewline
\ \ \ \ \ {\isacharparenleft}{\kern0pt}simp\ add{\isacharcolon}{\kern0pt}\ borel{\isacharunderscore}{\kern0pt}measurable{\isacharunderscore}{\kern0pt}integrable{\isacharcomma}{\kern0pt}\isanewline
\ \ \ \ \ \ metis\ bot{\isachardot}{\kern0pt}extremum\ measurable{\isacharunderscore}{\kern0pt}from{\isacharunderscore}{\kern0pt}subalg\ sets{\isacharunderscore}{\kern0pt}F{\isacharunderscore}{\kern0pt}mono\ space{\isacharunderscore}{\kern0pt}F\ subalgebra{\isacharunderscore}{\kern0pt}def{\isacharcomma}{\kern0pt}\ blast{\isacharcomma}{\kern0pt}\isanewline
\ \ \ \ \ \ metis\ {\isacharparenleft}{\kern0pt}mono{\isacharunderscore}{\kern0pt}tags{\isacharcomma}{\kern0pt}\ lifting{\isacharparenright}{\kern0pt}\ borel{\isacharunderscore}{\kern0pt}measurable{\isacharunderscore}{\kern0pt}subalgebra\ bot{\isacharunderscore}{\kern0pt}least\ filtration{\isachardot}{\kern0pt}sets{\isacharunderscore}{\kern0pt}F{\isacharunderscore}{\kern0pt}mono\ filtration{\isacharunderscore}{\kern0pt}axioms\ space{\isacharunderscore}{\kern0pt}F{\isacharparenright}{\kern0pt}%
\endisatagproof
{\isafoldproof}%
%
\isadelimproof
\ \isanewline
%
\endisadelimproof
\isanewline
\isacommand{lemma}\isamarkupfalse%
\ {\isacharparenleft}{\kern0pt}\isakeyword{in}\ filtered{\isacharunderscore}{\kern0pt}sigma{\isacharunderscore}{\kern0pt}finite{\isacharunderscore}{\kern0pt}measure{\isacharparenright}{\kern0pt}\ martingale{\isacharunderscore}{\kern0pt}cond{\isacharunderscore}{\kern0pt}exp{\isacharbrackleft}{\kern0pt}intro{\isacharbrackright}{\kern0pt}{\isacharcolon}{\kern0pt}\ \ \isanewline
\ \ \isakeyword{assumes}\ {\isachardoublequoteopen}integrable\ M\ f{\isachardoublequoteclose}\isanewline
\ \ \isakeyword{shows}\ {\isachardoublequoteopen}martingale\ M\ F\ {\isacharparenleft}{\kern0pt}{\isasymlambda}i{\isachardot}{\kern0pt}\ cond{\isacharunderscore}{\kern0pt}exp\ M\ {\isacharparenleft}{\kern0pt}F\ i{\isacharparenright}{\kern0pt}\ f{\isacharparenright}{\kern0pt}{\isachardoublequoteclose}\isanewline
%
\isadelimproof
\ \ %
\endisadelimproof
%
\isatagproof
\isacommand{by}\isamarkupfalse%
\ {\isacharparenleft}{\kern0pt}unfold{\isacharunderscore}{\kern0pt}locales{\isacharcomma}{\kern0pt}\isanewline
\ \ \ \ \ \ auto\ simp\ add{\isacharcolon}{\kern0pt}\ subalgebra\ borel{\isacharunderscore}{\kern0pt}measurable{\isacharunderscore}{\kern0pt}cond{\isacharunderscore}{\kern0pt}exp\ borel{\isacharunderscore}{\kern0pt}measurable{\isacharunderscore}{\kern0pt}cond{\isacharunderscore}{\kern0pt}exp{\isacharprime}{\kern0pt}\ intro{\isacharbang}{\kern0pt}{\isacharcolon}{\kern0pt}\ cond{\isacharunderscore}{\kern0pt}exp{\isacharunderscore}{\kern0pt}nested{\isacharunderscore}{\kern0pt}subalg{\isacharbrackleft}{\kern0pt}OF\ assms{\isacharbrackright}{\kern0pt}{\isacharcomma}{\kern0pt}\isanewline
\ \ \ \ \ \ simp\ add{\isacharcolon}{\kern0pt}\ sets{\isacharunderscore}{\kern0pt}F{\isacharunderscore}{\kern0pt}mono\ space{\isacharunderscore}{\kern0pt}F\ subalgebra{\isacharunderscore}{\kern0pt}def{\isacharparenright}{\kern0pt}%
\endisatagproof
{\isafoldproof}%
%
\isadelimproof
%
\endisadelimproof
%
\isadelimdocument
%
\endisadelimdocument
%
\isatagdocument
%
\isamarkupsubsection{Submartingale%
}
\isamarkuptrue%
%
\endisatagdocument
{\isafolddocument}%
%
\isadelimdocument
%
\endisadelimdocument
\isacommand{locale}\isamarkupfalse%
\ submartingale\ {\isacharequal}{\kern0pt}\ adapted{\isacharunderscore}{\kern0pt}process{\isacharunderscore}{\kern0pt}order\ {\isacharplus}{\kern0pt}\isanewline
\ \ \isakeyword{assumes}\ integrable{\isacharcolon}{\kern0pt}\ {\isachardoublequoteopen}{\isasymAnd}i{\isachardot}{\kern0pt}\ integrable\ M\ {\isacharparenleft}{\kern0pt}X\ i{\isacharparenright}{\kern0pt}{\isachardoublequoteclose}\isanewline
\ \ \ \ \ \ \isakeyword{and}\ submartingale{\isacharunderscore}{\kern0pt}property{\isacharcolon}{\kern0pt}\ {\isachardoublequoteopen}{\isasymAnd}i\ j{\isachardot}{\kern0pt}\ i\ {\isasymle}\ j\ {\isasymLongrightarrow}\ AE\ {\isasymxi}\ in\ M{\isachardot}{\kern0pt}\ X\ i\ {\isasymxi}\ {\isasymle}\ cond{\isacharunderscore}{\kern0pt}exp\ M\ {\isacharparenleft}{\kern0pt}F\ i{\isacharparenright}{\kern0pt}\ {\isacharparenleft}{\kern0pt}X\ j{\isacharparenright}{\kern0pt}\ {\isasymxi}{\isachardoublequoteclose}%
\isadelimdocument
%
\endisadelimdocument
%
\isatagdocument
%
\isamarkupsubsection{Supermartingale%
}
\isamarkuptrue%
%
\endisatagdocument
{\isafolddocument}%
%
\isadelimdocument
%
\endisadelimdocument
\isacommand{locale}\isamarkupfalse%
\ supermartingale\ {\isacharequal}{\kern0pt}\ adapted{\isacharunderscore}{\kern0pt}process{\isacharunderscore}{\kern0pt}order\ {\isacharplus}{\kern0pt}\isanewline
\ \ \isakeyword{assumes}\ integrable{\isacharcolon}{\kern0pt}\ {\isachardoublequoteopen}{\isasymAnd}i{\isachardot}{\kern0pt}\ integrable\ M\ {\isacharparenleft}{\kern0pt}X\ i{\isacharparenright}{\kern0pt}{\isachardoublequoteclose}\isanewline
\ \ \ \ \ \ \isakeyword{and}\ supermartingale{\isacharunderscore}{\kern0pt}property{\isacharcolon}{\kern0pt}\ {\isachardoublequoteopen}{\isasymAnd}i\ j{\isachardot}{\kern0pt}\ i\ {\isasymle}\ j\ {\isasymLongrightarrow}\ AE\ {\isasymxi}\ in\ M{\isachardot}{\kern0pt}\ X\ i\ {\isasymxi}\ {\isasymge}\ cond{\isacharunderscore}{\kern0pt}exp\ M\ {\isacharparenleft}{\kern0pt}F\ i{\isacharparenright}{\kern0pt}\ {\isacharparenleft}{\kern0pt}X\ j{\isacharparenright}{\kern0pt}\ {\isasymxi}{\isachardoublequoteclose}%
\isadelimdocument
%
\endisadelimdocument
%
\isatagdocument
%
\isamarkupsubsection{Martingale Stuff%
}
\isamarkuptrue%
%
\endisatagdocument
{\isafolddocument}%
%
\isadelimdocument
%
\endisadelimdocument
\isacommand{locale}\isamarkupfalse%
\ martingale{\isacharunderscore}{\kern0pt}order\ {\isacharequal}{\kern0pt}\ martingale\ M\ F\ X\ \isakeyword{for}\ M\ F\ \isakeyword{and}\ X\ {\isacharcolon}{\kern0pt}{\isacharcolon}{\kern0pt}\ {\isachardoublequoteopen}{\isacharunderscore}{\kern0pt}\ {\isasymRightarrow}\ {\isacharunderscore}{\kern0pt}\ {\isasymRightarrow}\ {\isacharunderscore}{\kern0pt}\ {\isacharcolon}{\kern0pt}{\isacharcolon}{\kern0pt}\ {\isacharbraceleft}{\kern0pt}linorder{\isacharunderscore}{\kern0pt}topology{\isacharcomma}{\kern0pt}\ ordered{\isacharunderscore}{\kern0pt}real{\isacharunderscore}{\kern0pt}vector{\isacharbraceright}{\kern0pt}{\isachardoublequoteclose}\isanewline
\isakeyword{begin}\isanewline
\isanewline
\isacommand{lemma}\isamarkupfalse%
\ is{\isacharunderscore}{\kern0pt}submartingale{\isacharcolon}{\kern0pt}\ {\isachardoublequoteopen}submartingale\ M\ F\ X{\isachardoublequoteclose}%
\isadelimproof
\ %
\endisadelimproof
%
\isatagproof
\isacommand{using}\isamarkupfalse%
\ martingale{\isacharunderscore}{\kern0pt}property\ \isacommand{by}\isamarkupfalse%
\ {\isacharparenleft}{\kern0pt}unfold{\isacharunderscore}{\kern0pt}locales{\isacharparenright}{\kern0pt}\ {\isacharparenleft}{\kern0pt}force\ simp\ add{\isacharcolon}{\kern0pt}\ integrable{\isacharparenright}{\kern0pt}{\isacharplus}{\kern0pt}%
\endisatagproof
{\isafoldproof}%
%
\isadelimproof
%
\endisadelimproof
\isanewline
\isanewline
\isacommand{lemma}\isamarkupfalse%
\ is{\isacharunderscore}{\kern0pt}supermartingale{\isacharcolon}{\kern0pt}\ {\isachardoublequoteopen}supermartingale\ M\ F\ X{\isachardoublequoteclose}%
\isadelimproof
\ %
\endisadelimproof
%
\isatagproof
\isacommand{using}\isamarkupfalse%
\ martingale{\isacharunderscore}{\kern0pt}property\ \isacommand{by}\isamarkupfalse%
\ {\isacharparenleft}{\kern0pt}unfold{\isacharunderscore}{\kern0pt}locales{\isacharparenright}{\kern0pt}\ {\isacharparenleft}{\kern0pt}force\ simp\ add{\isacharcolon}{\kern0pt}\ integrable{\isacharparenright}{\kern0pt}{\isacharplus}{\kern0pt}%
\endisatagproof
{\isafoldproof}%
%
\isadelimproof
%
\endisadelimproof
\isanewline
\isanewline
\isacommand{end}\isamarkupfalse%
\isanewline
\isanewline
\isacommand{sublocale}\isamarkupfalse%
\ martingale{\isacharunderscore}{\kern0pt}order\ {\isasymsubseteq}\ martingale{\isacharunderscore}{\kern0pt}is{\isacharunderscore}{\kern0pt}submartingale{\isacharcolon}{\kern0pt}\ submartingale%
\isadelimproof
\ %
\endisadelimproof
%
\isatagproof
\isacommand{by}\isamarkupfalse%
\ {\isacharparenleft}{\kern0pt}rule\ is{\isacharunderscore}{\kern0pt}submartingale{\isacharparenright}{\kern0pt}%
\endisatagproof
{\isafoldproof}%
%
\isadelimproof
%
\endisadelimproof
\isanewline
\isanewline
\isacommand{sublocale}\isamarkupfalse%
\ martingale{\isacharunderscore}{\kern0pt}order\ {\isasymsubseteq}\ martingale{\isacharunderscore}{\kern0pt}is{\isacharunderscore}{\kern0pt}supermartingale{\isacharcolon}{\kern0pt}\ supermartingale%
\isadelimproof
\ %
\endisadelimproof
%
\isatagproof
\isacommand{by}\isamarkupfalse%
\ {\isacharparenleft}{\kern0pt}rule\ is{\isacharunderscore}{\kern0pt}supermartingale{\isacharparenright}{\kern0pt}%
\endisatagproof
{\isafoldproof}%
%
\isadelimproof
%
\endisadelimproof
\isanewline
\isanewline
\isacommand{locale}\isamarkupfalse%
\ submartingale{\isacharunderscore}{\kern0pt}lattice\ {\isacharequal}{\kern0pt}\ submartingale\ M\ F\ X\ \isakeyword{for}\ M\ F\ \isakeyword{and}\ X\ {\isacharcolon}{\kern0pt}{\isacharcolon}{\kern0pt}\ {\isachardoublequoteopen}{\isacharunderscore}{\kern0pt}\ {\isasymRightarrow}\ {\isacharunderscore}{\kern0pt}\ {\isasymRightarrow}\ {\isacharunderscore}{\kern0pt}\ {\isacharcolon}{\kern0pt}{\isacharcolon}{\kern0pt}\ {\isacharbraceleft}{\kern0pt}linorder{\isacharunderscore}{\kern0pt}topology{\isacharcomma}{\kern0pt}\ lattice{\isacharcomma}{\kern0pt}\ ordered{\isacharunderscore}{\kern0pt}real{\isacharunderscore}{\kern0pt}vector{\isacharbraceright}{\kern0pt}{\isachardoublequoteclose}\isanewline
\isanewline
\isacommand{locale}\isamarkupfalse%
\ supermartingale{\isacharunderscore}{\kern0pt}lattice\ {\isacharequal}{\kern0pt}\ supermartingale\ M\ F\ X\ \isakeyword{for}\ M\ F\ \isakeyword{and}\ X\ {\isacharcolon}{\kern0pt}{\isacharcolon}{\kern0pt}\ {\isachardoublequoteopen}{\isacharunderscore}{\kern0pt}\ {\isasymRightarrow}\ {\isacharunderscore}{\kern0pt}\ {\isasymRightarrow}\ {\isacharunderscore}{\kern0pt}\ {\isacharcolon}{\kern0pt}{\isacharcolon}{\kern0pt}\ {\isacharbraceleft}{\kern0pt}linorder{\isacharunderscore}{\kern0pt}topology{\isacharcomma}{\kern0pt}\ lattice{\isacharcomma}{\kern0pt}\ ordered{\isacharunderscore}{\kern0pt}real{\isacharunderscore}{\kern0pt}vector{\isacharbraceright}{\kern0pt}{\isachardoublequoteclose}\isanewline
\isanewline
\isacommand{locale}\isamarkupfalse%
\ martingale{\isacharunderscore}{\kern0pt}lattice\ {\isacharequal}{\kern0pt}\ martingale\ M\ F\ X\ \isakeyword{for}\ M\ F\ \isakeyword{and}\ X\ {\isacharcolon}{\kern0pt}{\isacharcolon}{\kern0pt}\ {\isachardoublequoteopen}{\isacharunderscore}{\kern0pt}\ {\isasymRightarrow}\ {\isacharunderscore}{\kern0pt}\ {\isasymRightarrow}\ {\isacharunderscore}{\kern0pt}\ {\isacharcolon}{\kern0pt}{\isacharcolon}{\kern0pt}\ {\isacharbraceleft}{\kern0pt}linorder{\isacharunderscore}{\kern0pt}topology{\isacharcomma}{\kern0pt}\ lattice{\isacharcomma}{\kern0pt}\ ordered{\isacharunderscore}{\kern0pt}real{\isacharunderscore}{\kern0pt}vector{\isacharbraceright}{\kern0pt}{\isachardoublequoteclose}\isanewline
\isakeyword{begin}\isanewline
\isanewline
\isacommand{lemma}\isamarkupfalse%
\ is{\isacharunderscore}{\kern0pt}submartingale{\isacharcolon}{\kern0pt}\ {\isachardoublequoteopen}submartingale{\isacharunderscore}{\kern0pt}lattice\ M\ F\ X{\isachardoublequoteclose}%
\isadelimproof
\ %
\endisadelimproof
%
\isatagproof
\isacommand{using}\isamarkupfalse%
\ martingale{\isacharunderscore}{\kern0pt}property\ \isacommand{by}\isamarkupfalse%
\ {\isacharparenleft}{\kern0pt}unfold{\isacharunderscore}{\kern0pt}locales{\isacharparenright}{\kern0pt}\ {\isacharparenleft}{\kern0pt}force\ simp\ add{\isacharcolon}{\kern0pt}\ integrable{\isacharparenright}{\kern0pt}{\isacharplus}{\kern0pt}%
\endisatagproof
{\isafoldproof}%
%
\isadelimproof
%
\endisadelimproof
\isanewline
\isanewline
\isacommand{lemma}\isamarkupfalse%
\ is{\isacharunderscore}{\kern0pt}supermartingale{\isacharcolon}{\kern0pt}\ {\isachardoublequoteopen}supermartingale{\isacharunderscore}{\kern0pt}lattice\ M\ F\ X{\isachardoublequoteclose}%
\isadelimproof
\ %
\endisadelimproof
%
\isatagproof
\isacommand{using}\isamarkupfalse%
\ martingale{\isacharunderscore}{\kern0pt}property\ \isacommand{by}\isamarkupfalse%
\ {\isacharparenleft}{\kern0pt}unfold{\isacharunderscore}{\kern0pt}locales{\isacharparenright}{\kern0pt}\ {\isacharparenleft}{\kern0pt}force\ simp\ add{\isacharcolon}{\kern0pt}\ integrable{\isacharparenright}{\kern0pt}{\isacharplus}{\kern0pt}%
\endisatagproof
{\isafoldproof}%
%
\isadelimproof
%
\endisadelimproof
\isanewline
\isanewline
\isacommand{end}\isamarkupfalse%
\isanewline
\isanewline
\isacommand{sublocale}\isamarkupfalse%
\ martingale{\isacharunderscore}{\kern0pt}lattice\ {\isasymsubseteq}\ martingale{\isacharunderscore}{\kern0pt}is{\isacharunderscore}{\kern0pt}submartingale{\isacharcolon}{\kern0pt}\ submartingale{\isacharunderscore}{\kern0pt}lattice%
\isadelimproof
\ %
\endisadelimproof
%
\isatagproof
\isacommand{by}\isamarkupfalse%
\ {\isacharparenleft}{\kern0pt}rule\ is{\isacharunderscore}{\kern0pt}submartingale{\isacharparenright}{\kern0pt}%
\endisatagproof
{\isafoldproof}%
%
\isadelimproof
%
\endisadelimproof
\isanewline
\isanewline
\isacommand{sublocale}\isamarkupfalse%
\ martingale{\isacharunderscore}{\kern0pt}lattice\ {\isasymsubseteq}\ martingale{\isacharunderscore}{\kern0pt}is{\isacharunderscore}{\kern0pt}supermartingale{\isacharcolon}{\kern0pt}\ supermartingale{\isacharunderscore}{\kern0pt}lattice%
\isadelimproof
\ %
\endisadelimproof
%
\isatagproof
\isacommand{by}\isamarkupfalse%
\ {\isacharparenleft}{\kern0pt}rule\ is{\isacharunderscore}{\kern0pt}supermartingale{\isacharparenright}{\kern0pt}%
\endisatagproof
{\isafoldproof}%
%
\isadelimproof
%
\endisadelimproof
\isanewline
\isanewline
\isacommand{context}\isamarkupfalse%
\ martingale\isanewline
\isakeyword{begin}\isanewline
\isanewline
\isacommand{lemma}\isamarkupfalse%
\ set{\isacharunderscore}{\kern0pt}integral{\isacharunderscore}{\kern0pt}eq{\isacharcolon}{\kern0pt}\isanewline
\ \ \isakeyword{assumes}\ {\isachardoublequoteopen}A\ {\isasymin}\ F\ i{\isachardoublequoteclose}\ {\isachardoublequoteopen}i\ {\isasymle}\ j{\isachardoublequoteclose}\isanewline
\ \ \isakeyword{shows}\ {\isachardoublequoteopen}set{\isacharunderscore}{\kern0pt}lebesgue{\isacharunderscore}{\kern0pt}integral\ M\ A\ {\isacharparenleft}{\kern0pt}X\ i{\isacharparenright}{\kern0pt}\ {\isacharequal}{\kern0pt}\ set{\isacharunderscore}{\kern0pt}lebesgue{\isacharunderscore}{\kern0pt}integral\ M\ A\ {\isacharparenleft}{\kern0pt}X\ j{\isacharparenright}{\kern0pt}{\isachardoublequoteclose}\isanewline
%
\isadelimproof
%
\endisadelimproof
%
\isatagproof
\isacommand{proof}\isamarkupfalse%
\ {\isacharminus}{\kern0pt}\isanewline
\ \ \isacommand{have}\isamarkupfalse%
\ {\isachardoublequoteopen}{\isasymintegral}x\ {\isasymin}\ A{\isachardot}{\kern0pt}\ X\ i\ x\ {\isasympartial}M\ {\isacharequal}{\kern0pt}\ {\isasymintegral}x\ {\isasymin}\ A{\isachardot}{\kern0pt}\ cond{\isacharunderscore}{\kern0pt}exp\ M\ {\isacharparenleft}{\kern0pt}F\ i{\isacharparenright}{\kern0pt}\ {\isacharparenleft}{\kern0pt}X\ j{\isacharparenright}{\kern0pt}\ x\ {\isasympartial}M{\isachardoublequoteclose}\ \isacommand{using}\isamarkupfalse%
\ martingale{\isacharunderscore}{\kern0pt}property{\isacharbrackleft}{\kern0pt}OF\ assms{\isacharparenleft}{\kern0pt}{\isadigit{2}}{\isacharparenright}{\kern0pt}{\isacharbrackright}{\kern0pt}\ borel{\isacharunderscore}{\kern0pt}measurable{\isacharunderscore}{\kern0pt}cond{\isacharunderscore}{\kern0pt}exp{\isacharprime}{\kern0pt}\ assms{\isacharparenleft}{\kern0pt}{\isadigit{1}}{\isacharparenright}{\kern0pt}\ subalgebra\ subalgebra{\isacharunderscore}{\kern0pt}def\ \isacommand{by}\isamarkupfalse%
\ {\isacharparenleft}{\kern0pt}intro\ set{\isacharunderscore}{\kern0pt}lebesgue{\isacharunderscore}{\kern0pt}integral{\isacharunderscore}{\kern0pt}cong{\isacharunderscore}{\kern0pt}AE{\isacharbrackleft}{\kern0pt}OF\ {\isacharunderscore}{\kern0pt}\ random{\isacharunderscore}{\kern0pt}variable{\isacharbrackright}{\kern0pt}{\isacharparenright}{\kern0pt}\ fastforce{\isacharplus}{\kern0pt}\isanewline
\ \ \isacommand{also}\isamarkupfalse%
\ \isacommand{have}\isamarkupfalse%
\ {\isachardoublequoteopen}{\isachardot}{\kern0pt}{\isachardot}{\kern0pt}{\isachardot}{\kern0pt}\ {\isacharequal}{\kern0pt}\ {\isasymintegral}x\ {\isasymin}\ A{\isachardot}{\kern0pt}\ X\ j\ x\ {\isasympartial}M{\isachardoublequoteclose}\ \isacommand{using}\isamarkupfalse%
\ assms{\isacharparenleft}{\kern0pt}{\isadigit{1}}{\isacharparenright}{\kern0pt}\ \isacommand{by}\isamarkupfalse%
\ {\isacharparenleft}{\kern0pt}auto\ simp{\isacharcolon}{\kern0pt}\ integrable\ intro{\isacharcolon}{\kern0pt}\ cond{\isacharunderscore}{\kern0pt}exp{\isacharunderscore}{\kern0pt}set{\isacharunderscore}{\kern0pt}integral{\isacharbrackleft}{\kern0pt}symmetric{\isacharbrackright}{\kern0pt}{\isacharparenright}{\kern0pt}\isanewline
\ \ \isacommand{finally}\isamarkupfalse%
\ \isacommand{show}\isamarkupfalse%
\ {\isacharquery}{\kern0pt}thesis\ \isacommand{{\isachardot}{\kern0pt}}\isamarkupfalse%
\isanewline
\isacommand{qed}\isamarkupfalse%
%
\endisatagproof
{\isafoldproof}%
%
\isadelimproof
\isanewline
%
\endisadelimproof
\isanewline
\isacommand{lemma}\isamarkupfalse%
\ scaleR{\isacharunderscore}{\kern0pt}const{\isacharbrackleft}{\kern0pt}intro{\isacharbrackright}{\kern0pt}{\isacharcolon}{\kern0pt}\isanewline
\ \ \isakeyword{shows}\ {\isachardoublequoteopen}martingale\ M\ F\ {\isacharparenleft}{\kern0pt}{\isasymlambda}i\ x{\isachardot}{\kern0pt}\ c\ {\isacharasterisk}{\kern0pt}\isactrlsub R\ X\ i\ x{\isacharparenright}{\kern0pt}{\isachardoublequoteclose}\isanewline
%
\isadelimproof
%
\endisadelimproof
%
\isatagproof
\isacommand{proof}\isamarkupfalse%
\ {\isacharminus}{\kern0pt}\isanewline
\ \ \isacommand{{\isacharbraceleft}{\kern0pt}}\isamarkupfalse%
\isanewline
\ \ \ \ \isacommand{fix}\isamarkupfalse%
\ i\ j\ {\isacharcolon}{\kern0pt}{\isacharcolon}{\kern0pt}\ {\isacharprime}{\kern0pt}b\ \isacommand{assume}\isamarkupfalse%
\ {\isachardoublequoteopen}i\ {\isasymle}\ j{\isachardoublequoteclose}\isanewline
\ \ \ \ \isacommand{hence}\isamarkupfalse%
\ {\isachardoublequoteopen}AE\ x\ in\ M{\isachardot}{\kern0pt}\ c\ {\isacharasterisk}{\kern0pt}\isactrlsub R\ X\ i\ x\ {\isacharequal}{\kern0pt}\ cond{\isacharunderscore}{\kern0pt}exp\ M\ {\isacharparenleft}{\kern0pt}F\ i{\isacharparenright}{\kern0pt}\ {\isacharparenleft}{\kern0pt}{\isasymlambda}x{\isachardot}{\kern0pt}\ c\ {\isacharasterisk}{\kern0pt}\isactrlsub R\ X\ j\ x{\isacharparenright}{\kern0pt}\ x{\isachardoublequoteclose}\ \isanewline
\ \ \ \ \ \ \isacommand{using}\isamarkupfalse%
\ cond{\isacharunderscore}{\kern0pt}exp{\isacharunderscore}{\kern0pt}scaleR{\isacharunderscore}{\kern0pt}right{\isacharbrackleft}{\kern0pt}OF\ integrable{\isacharcomma}{\kern0pt}\ of\ i\ c{\isacharcomma}{\kern0pt}\ THEN\ AE{\isacharunderscore}{\kern0pt}symmetric{\isacharbrackright}{\kern0pt}\ martingale{\isacharunderscore}{\kern0pt}property\ \isacommand{by}\isamarkupfalse%
\ force\isanewline
\ \ \isacommand{{\isacharbraceright}{\kern0pt}}\isamarkupfalse%
\isanewline
\ \ \isacommand{thus}\isamarkupfalse%
\ {\isacharquery}{\kern0pt}thesis\ \isacommand{by}\isamarkupfalse%
\ {\isacharparenleft}{\kern0pt}unfold{\isacharunderscore}{\kern0pt}locales{\isacharparenright}{\kern0pt}\ {\isacharparenleft}{\kern0pt}auto\ simp\ add{\isacharcolon}{\kern0pt}\ borel{\isacharunderscore}{\kern0pt}measurable{\isacharunderscore}{\kern0pt}const{\isacharunderscore}{\kern0pt}scaleR\ adapted\ random{\isacharunderscore}{\kern0pt}variable\ integrable{\isacharparenright}{\kern0pt}\isanewline
\isacommand{qed}\isamarkupfalse%
%
\endisatagproof
{\isafoldproof}%
%
\isadelimproof
\isanewline
%
\endisadelimproof
\isanewline
\isacommand{lemma}\isamarkupfalse%
\ uminus{\isacharbrackleft}{\kern0pt}intro{\isacharbrackright}{\kern0pt}{\isacharcolon}{\kern0pt}\isanewline
\ \ \isakeyword{shows}\ {\isachardoublequoteopen}martingale\ M\ F\ {\isacharparenleft}{\kern0pt}{\isacharminus}{\kern0pt}\ X{\isacharparenright}{\kern0pt}{\isachardoublequoteclose}\ \isanewline
%
\isadelimproof
\ \ %
\endisadelimproof
%
\isatagproof
\isacommand{using}\isamarkupfalse%
\ scaleR{\isacharunderscore}{\kern0pt}const{\isacharbrackleft}{\kern0pt}of\ {\isachardoublequoteopen}{\isacharminus}{\kern0pt}{\isadigit{1}}{\isachardoublequoteclose}{\isacharbrackright}{\kern0pt}\ \isacommand{by}\isamarkupfalse%
\ {\isacharparenleft}{\kern0pt}force\ intro{\isacharcolon}{\kern0pt}\ back{\isacharunderscore}{\kern0pt}subst{\isacharbrackleft}{\kern0pt}of\ {\isachardoublequoteopen}martingale\ M\ F{\isachardoublequoteclose}{\isacharbrackright}{\kern0pt}{\isacharparenright}{\kern0pt}%
\endisatagproof
{\isafoldproof}%
%
\isadelimproof
\isanewline
%
\endisadelimproof
\isanewline
\isacommand{lemma}\isamarkupfalse%
\ add{\isacharbrackleft}{\kern0pt}intro{\isacharbrackright}{\kern0pt}{\isacharcolon}{\kern0pt}\isanewline
\ \ \isakeyword{assumes}\ {\isachardoublequoteopen}martingale\ M\ F\ Y{\isachardoublequoteclose}\isanewline
\ \ \isakeyword{shows}\ {\isachardoublequoteopen}martingale\ M\ F\ {\isacharparenleft}{\kern0pt}{\isasymlambda}i\ {\isasymxi}{\isachardot}{\kern0pt}\ X\ i\ {\isasymxi}\ {\isacharplus}{\kern0pt}\ Y\ i\ {\isasymxi}{\isacharparenright}{\kern0pt}{\isachardoublequoteclose}\isanewline
%
\isadelimproof
%
\endisadelimproof
%
\isatagproof
\isacommand{proof}\isamarkupfalse%
\ {\isacharminus}{\kern0pt}\isanewline
\ \ \isacommand{interpret}\isamarkupfalse%
\ Y{\isacharcolon}{\kern0pt}\ martingale\ M\ F\ Y\ \isacommand{by}\isamarkupfalse%
\ {\isacharparenleft}{\kern0pt}rule\ assms{\isacharparenright}{\kern0pt}\isanewline
\ \ \isacommand{{\isacharbraceleft}{\kern0pt}}\isamarkupfalse%
\isanewline
\ \ \ \ \isacommand{fix}\isamarkupfalse%
\ i\ j\ {\isacharcolon}{\kern0pt}{\isacharcolon}{\kern0pt}\ {\isacharprime}{\kern0pt}b\ \isacommand{assume}\isamarkupfalse%
\ asm{\isacharcolon}{\kern0pt}\ {\isachardoublequoteopen}i\ {\isasymle}\ j{\isachardoublequoteclose}\isanewline
\ \ \ \ \isacommand{have}\isamarkupfalse%
\ {\isachardoublequoteopen}AE\ {\isasymxi}\ in\ M{\isachardot}{\kern0pt}\ X\ i\ {\isasymxi}\ {\isacharplus}{\kern0pt}\ Y\ i\ {\isasymxi}\ {\isacharequal}{\kern0pt}\ cond{\isacharunderscore}{\kern0pt}exp\ M\ {\isacharparenleft}{\kern0pt}F\ i{\isacharparenright}{\kern0pt}\ {\isacharparenleft}{\kern0pt}{\isasymlambda}x{\isachardot}{\kern0pt}\ X\ j\ x\ {\isacharplus}{\kern0pt}\ Y\ j\ x{\isacharparenright}{\kern0pt}\ {\isasymxi}{\isachardoublequoteclose}\ \isanewline
\ \ \ \ \ \ \isacommand{using}\isamarkupfalse%
\ cond{\isacharunderscore}{\kern0pt}exp{\isacharunderscore}{\kern0pt}add{\isacharbrackleft}{\kern0pt}OF\ integrable\ martingale{\isachardot}{\kern0pt}integrable{\isacharbrackleft}{\kern0pt}OF\ assms{\isacharbrackright}{\kern0pt}{\isacharcomma}{\kern0pt}\ of\ i\ j\ j{\isacharcomma}{\kern0pt}\ THEN\ AE{\isacharunderscore}{\kern0pt}symmetric{\isacharbrackright}{\kern0pt}\ \isanewline
\ \ \ \ \ \ \ \ \ \ \ \ martingale{\isacharunderscore}{\kern0pt}property{\isacharbrackleft}{\kern0pt}OF\ asm{\isacharbrackright}{\kern0pt}\ martingale{\isachardot}{\kern0pt}martingale{\isacharunderscore}{\kern0pt}property{\isacharbrackleft}{\kern0pt}OF\ assms\ asm{\isacharbrackright}{\kern0pt}\ \isacommand{by}\isamarkupfalse%
\ force\isanewline
\ \ \isacommand{{\isacharbraceright}{\kern0pt}}\isamarkupfalse%
\isanewline
\ \ \isacommand{thus}\isamarkupfalse%
\ {\isacharquery}{\kern0pt}thesis\ \isacommand{using}\isamarkupfalse%
\ assms\isanewline
\ \ \isacommand{by}\isamarkupfalse%
\ {\isacharparenleft}{\kern0pt}unfold{\isacharunderscore}{\kern0pt}locales{\isacharparenright}{\kern0pt}\ {\isacharparenleft}{\kern0pt}auto\ simp\ add{\isacharcolon}{\kern0pt}\ borel{\isacharunderscore}{\kern0pt}measurable{\isacharunderscore}{\kern0pt}add\ random{\isacharunderscore}{\kern0pt}variable\ adapted\ integrable\ Y{\isachardot}{\kern0pt}adapted\ Y{\isachardot}{\kern0pt}random{\isacharunderscore}{\kern0pt}variable\ martingale{\isachardot}{\kern0pt}integrable{\isacharparenright}{\kern0pt}\isanewline
\isacommand{qed}\isamarkupfalse%
%
\endisatagproof
{\isafoldproof}%
%
\isadelimproof
\isanewline
%
\endisadelimproof
\isanewline
\isacommand{lemma}\isamarkupfalse%
\ diff{\isacharbrackleft}{\kern0pt}intro{\isacharbrackright}{\kern0pt}{\isacharcolon}{\kern0pt}\isanewline
\ \ \isakeyword{assumes}\ {\isachardoublequoteopen}martingale\ M\ F\ Y{\isachardoublequoteclose}\isanewline
\ \ \isakeyword{shows}\ {\isachardoublequoteopen}martingale\ M\ F\ {\isacharparenleft}{\kern0pt}{\isasymlambda}i\ x{\isachardot}{\kern0pt}\ X\ i\ x\ {\isacharminus}{\kern0pt}\ Y\ i\ x{\isacharparenright}{\kern0pt}{\isachardoublequoteclose}\isanewline
%
\isadelimproof
%
\endisadelimproof
%
\isatagproof
\isacommand{proof}\isamarkupfalse%
\ {\isacharminus}{\kern0pt}\isanewline
\ \ \isacommand{interpret}\isamarkupfalse%
\ Y{\isacharcolon}{\kern0pt}\ martingale\ M\ F\ Y\ \isacommand{by}\isamarkupfalse%
\ {\isacharparenleft}{\kern0pt}rule\ assms{\isacharparenright}{\kern0pt}\isanewline
\ \ \isacommand{{\isacharbraceleft}{\kern0pt}}\isamarkupfalse%
\isanewline
\ \ \ \ \isacommand{fix}\isamarkupfalse%
\ i\ j\ {\isacharcolon}{\kern0pt}{\isacharcolon}{\kern0pt}\ {\isacharprime}{\kern0pt}b\ \isacommand{assume}\isamarkupfalse%
\ asm{\isacharcolon}{\kern0pt}\ {\isachardoublequoteopen}i\ {\isasymle}\ j{\isachardoublequoteclose}\isanewline
\ \ \ \ \isacommand{have}\isamarkupfalse%
\ {\isachardoublequoteopen}AE\ {\isasymxi}\ in\ M{\isachardot}{\kern0pt}\ X\ i\ {\isasymxi}\ {\isacharminus}{\kern0pt}\ Y\ i\ {\isasymxi}\ {\isacharequal}{\kern0pt}\ cond{\isacharunderscore}{\kern0pt}exp\ M\ {\isacharparenleft}{\kern0pt}F\ i{\isacharparenright}{\kern0pt}\ {\isacharparenleft}{\kern0pt}{\isasymlambda}x{\isachardot}{\kern0pt}\ X\ j\ x\ {\isacharminus}{\kern0pt}\ Y\ j\ x{\isacharparenright}{\kern0pt}\ {\isasymxi}{\isachardoublequoteclose}\ \isanewline
\ \ \ \ \ \ \isacommand{using}\isamarkupfalse%
\ cond{\isacharunderscore}{\kern0pt}exp{\isacharunderscore}{\kern0pt}diff{\isacharbrackleft}{\kern0pt}OF\ integrable\ martingale{\isachardot}{\kern0pt}integrable{\isacharbrackleft}{\kern0pt}OF\ assms{\isacharbrackright}{\kern0pt}{\isacharcomma}{\kern0pt}\ of\ i\ j\ j{\isacharcomma}{\kern0pt}\ THEN\ AE{\isacharunderscore}{\kern0pt}symmetric{\isacharcomma}{\kern0pt}\ unfolded\ fun{\isacharunderscore}{\kern0pt}diff{\isacharunderscore}{\kern0pt}def{\isacharbrackright}{\kern0pt}\ \isanewline
\ \ \ \ \ \ \ \ \ \ \ \ martingale{\isacharunderscore}{\kern0pt}property{\isacharbrackleft}{\kern0pt}OF\ asm{\isacharbrackright}{\kern0pt}\ martingale{\isachardot}{\kern0pt}martingale{\isacharunderscore}{\kern0pt}property{\isacharbrackleft}{\kern0pt}OF\ assms\ asm{\isacharbrackright}{\kern0pt}\ \isacommand{by}\isamarkupfalse%
\ fastforce\isanewline
\ \ \isacommand{{\isacharbraceright}{\kern0pt}}\isamarkupfalse%
\isanewline
\ \ \isacommand{thus}\isamarkupfalse%
\ {\isacharquery}{\kern0pt}thesis\ \isacommand{using}\isamarkupfalse%
\ assms\ \isacommand{by}\isamarkupfalse%
\ {\isacharparenleft}{\kern0pt}unfold{\isacharunderscore}{\kern0pt}locales{\isacharparenright}{\kern0pt}\ {\isacharparenleft}{\kern0pt}auto\ simp\ add{\isacharcolon}{\kern0pt}\ borel{\isacharunderscore}{\kern0pt}measurable{\isacharunderscore}{\kern0pt}diff\ random{\isacharunderscore}{\kern0pt}variable\ adapted\ integrable\ Y{\isachardot}{\kern0pt}random{\isacharunderscore}{\kern0pt}variable\ Y{\isachardot}{\kern0pt}adapted\ martingale{\isachardot}{\kern0pt}integrable{\isacharparenright}{\kern0pt}\ \ \isanewline
\isacommand{qed}\isamarkupfalse%
%
\endisatagproof
{\isafoldproof}%
%
\isadelimproof
\isanewline
%
\endisadelimproof
\isanewline
\isacommand{end}\isamarkupfalse%
\isanewline
\isanewline
\isacommand{lemma}\isamarkupfalse%
\ {\isacharparenleft}{\kern0pt}\isakeyword{in}\ adapted{\isacharunderscore}{\kern0pt}process{\isacharparenright}{\kern0pt}\ martingale{\isacharunderscore}{\kern0pt}of{\isacharunderscore}{\kern0pt}set{\isacharunderscore}{\kern0pt}integral{\isacharunderscore}{\kern0pt}eq{\isacharcolon}{\kern0pt}\isanewline
\ \ \isakeyword{assumes}\ integrable{\isacharcolon}{\kern0pt}\ {\isachardoublequoteopen}{\isasymAnd}i{\isachardot}{\kern0pt}\ integrable\ M\ {\isacharparenleft}{\kern0pt}X\ i{\isacharparenright}{\kern0pt}{\isachardoublequoteclose}\isanewline
\ \ \ \ \ \ \isakeyword{and}\ {\isachardoublequoteopen}{\isasymAnd}A\ i\ j{\isachardot}{\kern0pt}\ i\ {\isasymle}\ j\ {\isasymLongrightarrow}\ A\ {\isasymin}\ F\ i\ {\isasymLongrightarrow}\ set{\isacharunderscore}{\kern0pt}lebesgue{\isacharunderscore}{\kern0pt}integral\ M\ A\ {\isacharparenleft}{\kern0pt}X\ i{\isacharparenright}{\kern0pt}\ {\isacharequal}{\kern0pt}\ set{\isacharunderscore}{\kern0pt}lebesgue{\isacharunderscore}{\kern0pt}integral\ M\ A\ {\isacharparenleft}{\kern0pt}X\ j{\isacharparenright}{\kern0pt}{\isachardoublequoteclose}\ \isanewline
\ \ \ \ \isakeyword{shows}\ {\isachardoublequoteopen}martingale\ M\ F\ X{\isachardoublequoteclose}\isanewline
%
\isadelimproof
%
\endisadelimproof
%
\isatagproof
\isacommand{proof}\isamarkupfalse%
\ {\isacharparenleft}{\kern0pt}unfold{\isacharunderscore}{\kern0pt}locales{\isacharparenright}{\kern0pt}\isanewline
\ \ \isacommand{fix}\isamarkupfalse%
\ i\ j\ {\isacharcolon}{\kern0pt}{\isacharcolon}{\kern0pt}\ {\isacharprime}{\kern0pt}t\ \isacommand{assume}\isamarkupfalse%
\ asm{\isacharcolon}{\kern0pt}\ {\isachardoublequoteopen}i\ {\isasymle}\ j{\isachardoublequoteclose}\isanewline
\ \ \isacommand{interpret}\isamarkupfalse%
\ sigma{\isacharunderscore}{\kern0pt}finite{\isacharunderscore}{\kern0pt}measure\ {\isachardoublequoteopen}restr{\isacharunderscore}{\kern0pt}to{\isacharunderscore}{\kern0pt}subalg\ M\ {\isacharparenleft}{\kern0pt}F\ i{\isacharparenright}{\kern0pt}{\isachardoublequoteclose}\ \isacommand{by}\isamarkupfalse%
\ {\isacharparenleft}{\kern0pt}simp\ add{\isacharcolon}{\kern0pt}\ sigma{\isacharunderscore}{\kern0pt}fin{\isacharunderscore}{\kern0pt}subalg{\isacharparenright}{\kern0pt}\isanewline
\ \ \isacommand{{\isacharbraceleft}{\kern0pt}}\isamarkupfalse%
\isanewline
\ \ \ \ \isacommand{fix}\isamarkupfalse%
\ A\ \isacommand{assume}\isamarkupfalse%
\ {\isachardoublequoteopen}A\ {\isasymin}\ restr{\isacharunderscore}{\kern0pt}to{\isacharunderscore}{\kern0pt}subalg\ M\ {\isacharparenleft}{\kern0pt}F\ i{\isacharparenright}{\kern0pt}{\isachardoublequoteclose}\isanewline
\ \ \ \ \isacommand{hence}\isamarkupfalse%
\ {\isacharasterisk}{\kern0pt}{\isacharcolon}{\kern0pt}\ {\isachardoublequoteopen}A\ {\isasymin}\ F\ i{\isachardoublequoteclose}\ \isacommand{using}\isamarkupfalse%
\ sets{\isacharunderscore}{\kern0pt}restr{\isacharunderscore}{\kern0pt}to{\isacharunderscore}{\kern0pt}subalg\ subalgebra\ \isacommand{by}\isamarkupfalse%
\ blast\isanewline
\ \ \ \ \isacommand{have}\isamarkupfalse%
\ {\isachardoublequoteopen}set{\isacharunderscore}{\kern0pt}lebesgue{\isacharunderscore}{\kern0pt}integral\ {\isacharparenleft}{\kern0pt}restr{\isacharunderscore}{\kern0pt}to{\isacharunderscore}{\kern0pt}subalg\ M\ {\isacharparenleft}{\kern0pt}F\ i{\isacharparenright}{\kern0pt}{\isacharparenright}{\kern0pt}\ A\ {\isacharparenleft}{\kern0pt}X\ i{\isacharparenright}{\kern0pt}\ {\isacharequal}{\kern0pt}\ set{\isacharunderscore}{\kern0pt}lebesgue{\isacharunderscore}{\kern0pt}integral\ M\ A\ {\isacharparenleft}{\kern0pt}X\ i{\isacharparenright}{\kern0pt}{\isachardoublequoteclose}\ \isacommand{using}\isamarkupfalse%
\ {\isacharasterisk}{\kern0pt}\ subalg\ \isacommand{by}\isamarkupfalse%
\ {\isacharparenleft}{\kern0pt}auto\ simp{\isacharcolon}{\kern0pt}\ set{\isacharunderscore}{\kern0pt}lebesgue{\isacharunderscore}{\kern0pt}integral{\isacharunderscore}{\kern0pt}def\ intro{\isacharcolon}{\kern0pt}\ integral{\isacharunderscore}{\kern0pt}subalgebra{\isadigit{2}}\ borel{\isacharunderscore}{\kern0pt}measurable{\isacharunderscore}{\kern0pt}scaleR\ adapted\ borel{\isacharunderscore}{\kern0pt}measurable{\isacharunderscore}{\kern0pt}indicator{\isacharparenright}{\kern0pt}\ \isanewline
\ \ \ \ \isacommand{also}\isamarkupfalse%
\ \isacommand{have}\isamarkupfalse%
\ {\isachardoublequoteopen}{\isachardot}{\kern0pt}{\isachardot}{\kern0pt}{\isachardot}{\kern0pt}\ {\isacharequal}{\kern0pt}\ set{\isacharunderscore}{\kern0pt}lebesgue{\isacharunderscore}{\kern0pt}integral\ M\ A\ {\isacharparenleft}{\kern0pt}cond{\isacharunderscore}{\kern0pt}exp\ M\ {\isacharparenleft}{\kern0pt}F\ i{\isacharparenright}{\kern0pt}\ {\isacharparenleft}{\kern0pt}X\ j{\isacharparenright}{\kern0pt}{\isacharparenright}{\kern0pt}{\isachardoublequoteclose}\ \isacommand{using}\isamarkupfalse%
\ {\isacharasterisk}{\kern0pt}\ assms{\isacharparenleft}{\kern0pt}{\isadigit{2}}{\isacharparenright}{\kern0pt}{\isacharbrackleft}{\kern0pt}OF\ asm{\isacharbrackright}{\kern0pt}\ cond{\isacharunderscore}{\kern0pt}exp{\isacharunderscore}{\kern0pt}set{\isacharunderscore}{\kern0pt}integral{\isacharbrackleft}{\kern0pt}OF\ integrable{\isacharbrackright}{\kern0pt}\ \isacommand{by}\isamarkupfalse%
\ auto\isanewline
\ \ \ \ \isacommand{finally}\isamarkupfalse%
\ \isacommand{have}\isamarkupfalse%
\ {\isachardoublequoteopen}set{\isacharunderscore}{\kern0pt}lebesgue{\isacharunderscore}{\kern0pt}integral\ {\isacharparenleft}{\kern0pt}restr{\isacharunderscore}{\kern0pt}to{\isacharunderscore}{\kern0pt}subalg\ M\ {\isacharparenleft}{\kern0pt}F\ i{\isacharparenright}{\kern0pt}{\isacharparenright}{\kern0pt}\ A\ {\isacharparenleft}{\kern0pt}X\ i{\isacharparenright}{\kern0pt}\ {\isacharequal}{\kern0pt}\ set{\isacharunderscore}{\kern0pt}lebesgue{\isacharunderscore}{\kern0pt}integral\ {\isacharparenleft}{\kern0pt}restr{\isacharunderscore}{\kern0pt}to{\isacharunderscore}{\kern0pt}subalg\ M\ {\isacharparenleft}{\kern0pt}F\ i{\isacharparenright}{\kern0pt}{\isacharparenright}{\kern0pt}\ A\ {\isacharparenleft}{\kern0pt}cond{\isacharunderscore}{\kern0pt}exp\ M\ {\isacharparenleft}{\kern0pt}F\ i{\isacharparenright}{\kern0pt}\ {\isacharparenleft}{\kern0pt}X\ j{\isacharparenright}{\kern0pt}{\isacharparenright}{\kern0pt}{\isachardoublequoteclose}\ \isacommand{using}\isamarkupfalse%
\ {\isacharasterisk}{\kern0pt}\ subalg\ \isacommand{by}\isamarkupfalse%
\ {\isacharparenleft}{\kern0pt}auto\ simp{\isacharcolon}{\kern0pt}\ set{\isacharunderscore}{\kern0pt}lebesgue{\isacharunderscore}{\kern0pt}integral{\isacharunderscore}{\kern0pt}def\ intro{\isacharbang}{\kern0pt}{\isacharcolon}{\kern0pt}\ integral{\isacharunderscore}{\kern0pt}subalgebra{\isadigit{2}}{\isacharbrackleft}{\kern0pt}symmetric{\isacharbrackright}{\kern0pt}\ borel{\isacharunderscore}{\kern0pt}measurable{\isacharunderscore}{\kern0pt}scaleR\ borel{\isacharunderscore}{\kern0pt}measurable{\isacharunderscore}{\kern0pt}cond{\isacharunderscore}{\kern0pt}exp\ borel{\isacharunderscore}{\kern0pt}measurable{\isacharunderscore}{\kern0pt}indicator{\isacharparenright}{\kern0pt}\isanewline
\ \ \isacommand{{\isacharbraceright}{\kern0pt}}\isamarkupfalse%
\isanewline
\ \ \isacommand{hence}\isamarkupfalse%
\ {\isachardoublequoteopen}AE\ {\isasymxi}\ in\ restr{\isacharunderscore}{\kern0pt}to{\isacharunderscore}{\kern0pt}subalg\ M\ {\isacharparenleft}{\kern0pt}F\ i{\isacharparenright}{\kern0pt}{\isachardot}{\kern0pt}\ X\ i\ {\isasymxi}\ {\isacharequal}{\kern0pt}\ cond{\isacharunderscore}{\kern0pt}exp\ M\ {\isacharparenleft}{\kern0pt}F\ i{\isacharparenright}{\kern0pt}\ {\isacharparenleft}{\kern0pt}X\ j{\isacharparenright}{\kern0pt}\ {\isasymxi}{\isachardoublequoteclose}\ \isacommand{by}\isamarkupfalse%
\ {\isacharparenleft}{\kern0pt}intro\ density{\isacharunderscore}{\kern0pt}unique{\isacharcomma}{\kern0pt}\ auto\ intro{\isacharcolon}{\kern0pt}\ integrable{\isacharunderscore}{\kern0pt}in{\isacharunderscore}{\kern0pt}subalg\ subalg\ borel{\isacharunderscore}{\kern0pt}measurable{\isacharunderscore}{\kern0pt}cond{\isacharunderscore}{\kern0pt}exp\ integrable{\isacharparenright}{\kern0pt}\isanewline
\ \ \isacommand{thus}\isamarkupfalse%
\ {\isachardoublequoteopen}AE\ {\isasymxi}\ in\ M{\isachardot}{\kern0pt}\ X\ i\ {\isasymxi}\ {\isacharequal}{\kern0pt}\ cond{\isacharunderscore}{\kern0pt}exp\ M\ {\isacharparenleft}{\kern0pt}F\ i{\isacharparenright}{\kern0pt}\ {\isacharparenleft}{\kern0pt}X\ j{\isacharparenright}{\kern0pt}\ {\isasymxi}{\isachardoublequoteclose}\ \isacommand{using}\isamarkupfalse%
\ AE{\isacharunderscore}{\kern0pt}restr{\isacharunderscore}{\kern0pt}to{\isacharunderscore}{\kern0pt}subalg{\isacharbrackleft}{\kern0pt}OF\ subalg{\isacharbrackright}{\kern0pt}\ \isacommand{by}\isamarkupfalse%
\ blast\isanewline
\isacommand{qed}\isamarkupfalse%
\ {\isacharparenleft}{\kern0pt}simp\ add{\isacharcolon}{\kern0pt}\ integrable{\isacharparenright}{\kern0pt}%
\endisatagproof
{\isafoldproof}%
%
\isadelimproof
\isanewline
%
\endisadelimproof
\ \ \isanewline
\isacommand{lemma}\isamarkupfalse%
\ martingale{\isacharunderscore}{\kern0pt}orderI{\isacharcolon}{\kern0pt}\isanewline
\ \ \isakeyword{assumes}\ {\isachardoublequoteopen}submartingale\ M\ F\ X{\isachardoublequoteclose}\ {\isachardoublequoteopen}supermartingale\ M\ F\ X{\isachardoublequoteclose}\isanewline
\ \ \isakeyword{shows}\ {\isachardoublequoteopen}martingale{\isacharunderscore}{\kern0pt}order\ M\ F\ X{\isachardoublequoteclose}\ \isanewline
%
\isadelimproof
%
\endisadelimproof
%
\isatagproof
\isacommand{proof}\isamarkupfalse%
\ {\isacharminus}{\kern0pt}\isanewline
\ \ \isacommand{interpret}\isamarkupfalse%
\ submartingale\ M\ F\ X\ \isacommand{by}\isamarkupfalse%
\ {\isacharparenleft}{\kern0pt}rule\ assms{\isacharparenright}{\kern0pt}\isanewline
\ \ \isacommand{interpret}\isamarkupfalse%
\ supermartingale\ M\ F\ X\ \isacommand{by}\isamarkupfalse%
\ {\isacharparenleft}{\kern0pt}rule\ assms{\isacharparenright}{\kern0pt}\isanewline
\ \ \isacommand{show}\isamarkupfalse%
\ {\isacharquery}{\kern0pt}thesis\ \isacommand{using}\isamarkupfalse%
\ integrable\ submartingale{\isacharunderscore}{\kern0pt}property\ supermartingale{\isacharunderscore}{\kern0pt}property\ \isacommand{by}\isamarkupfalse%
\ {\isacharparenleft}{\kern0pt}unfold{\isacharunderscore}{\kern0pt}locales{\isacharparenright}{\kern0pt}\ {\isacharparenleft}{\kern0pt}fast\ intro{\isacharcolon}{\kern0pt}\ antisym{\isacharparenright}{\kern0pt}{\isacharplus}{\kern0pt}\isanewline
\isacommand{qed}\isamarkupfalse%
%
\endisatagproof
{\isafoldproof}%
%
\isadelimproof
\isanewline
%
\endisadelimproof
\isanewline
\isacommand{lemma}\isamarkupfalse%
\ martingale{\isacharunderscore}{\kern0pt}iff{\isacharcolon}{\kern0pt}\ {\isachardoublequoteopen}martingale\ M\ F\ X\ {\isasymlongleftrightarrow}\ submartingale\ M\ F\ X\ {\isasymand}\ supermartingale\ M\ F\ X{\isachardoublequoteclose}\isanewline
%
\isadelimproof
\ \ %
\endisadelimproof
%
\isatagproof
\isacommand{using}\isamarkupfalse%
\ martingale{\isacharunderscore}{\kern0pt}orderI\ martingale{\isacharunderscore}{\kern0pt}order{\isachardot}{\kern0pt}is{\isacharunderscore}{\kern0pt}submartingale\ martingale{\isacharunderscore}{\kern0pt}order{\isachardot}{\kern0pt}is{\isacharunderscore}{\kern0pt}supermartingale\ martingale{\isacharunderscore}{\kern0pt}order{\isacharunderscore}{\kern0pt}def\ \isacommand{by}\isamarkupfalse%
\ blast%
\endisatagproof
{\isafoldproof}%
%
\isadelimproof
%
\endisadelimproof
%
\isadelimdocument
%
\endisadelimdocument
%
\isatagdocument
%
\isamarkupsubsection{Submartingale Stuff%
}
\isamarkuptrue%
%
\endisatagdocument
{\isafolddocument}%
%
\isadelimdocument
%
\endisadelimdocument
\isacommand{context}\isamarkupfalse%
\ submartingale\isanewline
\isakeyword{begin}\isanewline
\isanewline
\isacommand{lemma}\isamarkupfalse%
\ set{\isacharunderscore}{\kern0pt}integral{\isacharunderscore}{\kern0pt}le{\isacharcolon}{\kern0pt}\isanewline
\ \ \isakeyword{assumes}\ {\isachardoublequoteopen}A\ {\isasymin}\ F\ i{\isachardoublequoteclose}\ {\isachardoublequoteopen}i\ {\isasymle}\ j{\isachardoublequoteclose}\isanewline
\ \ \isakeyword{shows}\ {\isachardoublequoteopen}set{\isacharunderscore}{\kern0pt}lebesgue{\isacharunderscore}{\kern0pt}integral\ M\ A\ {\isacharparenleft}{\kern0pt}X\ i{\isacharparenright}{\kern0pt}\ {\isasymle}\ set{\isacharunderscore}{\kern0pt}lebesgue{\isacharunderscore}{\kern0pt}integral\ M\ A\ {\isacharparenleft}{\kern0pt}X\ j{\isacharparenright}{\kern0pt}{\isachardoublequoteclose}\ \isanewline
%
\isadelimproof
\ \ %
\endisadelimproof
%
\isatagproof
\isacommand{unfolding}\isamarkupfalse%
\ cond{\isacharunderscore}{\kern0pt}exp{\isacharunderscore}{\kern0pt}set{\isacharunderscore}{\kern0pt}integral{\isacharbrackleft}{\kern0pt}OF\ integrable\ assms{\isacharparenleft}{\kern0pt}{\isadigit{1}}{\isacharparenright}{\kern0pt}{\isacharcomma}{\kern0pt}\ of\ j{\isacharbrackright}{\kern0pt}\ \ \isanewline
\ \ \isacommand{using}\isamarkupfalse%
\ submartingale{\isacharunderscore}{\kern0pt}property{\isacharbrackleft}{\kern0pt}OF\ assms{\isacharparenleft}{\kern0pt}{\isadigit{2}}{\isacharparenright}{\kern0pt}{\isacharbrackright}{\kern0pt}\isanewline
\ \ \isacommand{by}\isamarkupfalse%
\ {\isacharparenleft}{\kern0pt}simp\ only{\isacharcolon}{\kern0pt}\ set{\isacharunderscore}{\kern0pt}lebesgue{\isacharunderscore}{\kern0pt}integral{\isacharunderscore}{\kern0pt}def{\isacharcomma}{\kern0pt}\ intro\ integral{\isacharunderscore}{\kern0pt}mono{\isacharunderscore}{\kern0pt}AE{\isacharunderscore}{\kern0pt}banach{\isacharcomma}{\kern0pt}\ metis\ assms{\isacharparenleft}{\kern0pt}{\isadigit{1}}{\isacharparenright}{\kern0pt}\ in{\isacharunderscore}{\kern0pt}mono\ integrable\ integrable{\isacharunderscore}{\kern0pt}mult{\isacharunderscore}{\kern0pt}indicator\ subalgebra\ subalgebra{\isacharunderscore}{\kern0pt}def{\isacharcomma}{\kern0pt}\ metis\ assms{\isacharparenleft}{\kern0pt}{\isadigit{1}}{\isacharparenright}{\kern0pt}\ in{\isacharunderscore}{\kern0pt}mono\ integrable{\isacharunderscore}{\kern0pt}mult{\isacharunderscore}{\kern0pt}indicator\ subalgebra\ subalgebra{\isacharunderscore}{\kern0pt}def\ integrable{\isacharunderscore}{\kern0pt}cond{\isacharunderscore}{\kern0pt}exp{\isacharparenright}{\kern0pt}\ \isanewline
\ \ \ \ \ {\isacharparenleft}{\kern0pt}auto\ intro{\isacharcolon}{\kern0pt}\ scaleR{\isacharunderscore}{\kern0pt}left{\isacharunderscore}{\kern0pt}mono{\isacharparenright}{\kern0pt}%
\endisatagproof
{\isafoldproof}%
%
\isadelimproof
\isanewline
%
\endisadelimproof
\isanewline
\isacommand{lemma}\isamarkupfalse%
\ cond{\isacharunderscore}{\kern0pt}exp{\isacharunderscore}{\kern0pt}diff{\isacharunderscore}{\kern0pt}nonneg{\isacharcolon}{\kern0pt}\ \isanewline
\ \ \isakeyword{assumes}\ {\isachardoublequoteopen}i\ {\isasymle}\ j{\isachardoublequoteclose}\isanewline
\ \ \isakeyword{shows}\ {\isachardoublequoteopen}AE\ x\ in\ M{\isachardot}{\kern0pt}\ {\isadigit{0}}\ {\isasymle}\ cond{\isacharunderscore}{\kern0pt}exp\ M\ {\isacharparenleft}{\kern0pt}F\ i{\isacharparenright}{\kern0pt}\ {\isacharparenleft}{\kern0pt}{\isasymlambda}{\isasymxi}{\isachardot}{\kern0pt}\ X\ j\ {\isasymxi}\ {\isacharminus}{\kern0pt}\ X\ i\ {\isasymxi}{\isacharparenright}{\kern0pt}\ x{\isachardoublequoteclose}\isanewline
%
\isadelimproof
\ \ %
\endisadelimproof
%
\isatagproof
\isacommand{using}\isamarkupfalse%
\ submartingale{\isacharunderscore}{\kern0pt}property{\isacharbrackleft}{\kern0pt}OF\ assms{\isacharbrackright}{\kern0pt}\ cond{\isacharunderscore}{\kern0pt}exp{\isacharunderscore}{\kern0pt}diff{\isacharbrackleft}{\kern0pt}OF\ integrable{\isacharparenleft}{\kern0pt}{\isadigit{1}}{\isacharcomma}{\kern0pt}{\isadigit{1}}{\isacharparenright}{\kern0pt}{\isacharcomma}{\kern0pt}\ of\ i\ j\ i{\isacharbrackright}{\kern0pt}\ cond{\isacharunderscore}{\kern0pt}exp{\isacharunderscore}{\kern0pt}F{\isacharunderscore}{\kern0pt}meas{\isacharbrackleft}{\kern0pt}OF\ integrable\ adapted{\isacharcomma}{\kern0pt}\ of\ i{\isacharbrackright}{\kern0pt}\ \isacommand{by}\isamarkupfalse%
\ fastforce%
\endisatagproof
{\isafoldproof}%
%
\isadelimproof
\isanewline
%
\endisadelimproof
\isanewline
\isacommand{lemma}\isamarkupfalse%
\ add{\isacharbrackleft}{\kern0pt}intro{\isacharbrackright}{\kern0pt}{\isacharcolon}{\kern0pt}\isanewline
\ \ \isakeyword{assumes}\ {\isachardoublequoteopen}submartingale\ M\ F\ Y{\isachardoublequoteclose}\isanewline
\ \ \isakeyword{shows}\ {\isachardoublequoteopen}submartingale\ M\ F\ {\isacharparenleft}{\kern0pt}{\isasymlambda}i\ {\isasymxi}{\isachardot}{\kern0pt}\ X\ i\ {\isasymxi}\ {\isacharplus}{\kern0pt}\ Y\ i\ {\isasymxi}{\isacharparenright}{\kern0pt}{\isachardoublequoteclose}\isanewline
%
\isadelimproof
%
\endisadelimproof
%
\isatagproof
\isacommand{proof}\isamarkupfalse%
\ {\isacharminus}{\kern0pt}\isanewline
\ \ \isacommand{interpret}\isamarkupfalse%
\ Y{\isacharcolon}{\kern0pt}\ submartingale\ M\ F\ Y\ \isacommand{by}\isamarkupfalse%
\ {\isacharparenleft}{\kern0pt}rule\ assms{\isacharparenright}{\kern0pt}\isanewline
\ \ \isacommand{{\isacharbraceleft}{\kern0pt}}\isamarkupfalse%
\isanewline
\ \ \ \ \isacommand{fix}\isamarkupfalse%
\ i\ j\ {\isacharcolon}{\kern0pt}{\isacharcolon}{\kern0pt}\ {\isacharprime}{\kern0pt}b\ \isacommand{assume}\isamarkupfalse%
\ asm{\isacharcolon}{\kern0pt}\ {\isachardoublequoteopen}i\ {\isasymle}\ j{\isachardoublequoteclose}\isanewline
\ \ \ \ \isacommand{have}\isamarkupfalse%
\ {\isachardoublequoteopen}AE\ {\isasymxi}\ in\ M{\isachardot}{\kern0pt}\ X\ i\ {\isasymxi}\ {\isacharplus}{\kern0pt}\ Y\ i\ {\isasymxi}\ {\isasymle}\ cond{\isacharunderscore}{\kern0pt}exp\ M\ {\isacharparenleft}{\kern0pt}F\ i{\isacharparenright}{\kern0pt}\ {\isacharparenleft}{\kern0pt}{\isasymlambda}x{\isachardot}{\kern0pt}\ X\ j\ x\ {\isacharplus}{\kern0pt}\ Y\ j\ x{\isacharparenright}{\kern0pt}\ {\isasymxi}{\isachardoublequoteclose}\ \isanewline
\ \ \ \ \ \ \isacommand{using}\isamarkupfalse%
\ cond{\isacharunderscore}{\kern0pt}exp{\isacharunderscore}{\kern0pt}add{\isacharbrackleft}{\kern0pt}OF\ integrable\ submartingale{\isachardot}{\kern0pt}integrable{\isacharbrackleft}{\kern0pt}OF\ assms{\isacharbrackright}{\kern0pt}{\isacharcomma}{\kern0pt}\ of\ i\ j\ j{\isacharbrackright}{\kern0pt}\ \isanewline
\ \ \ \ \ \ \ \ \ \ \ \ submartingale{\isacharunderscore}{\kern0pt}property{\isacharbrackleft}{\kern0pt}OF\ asm{\isacharbrackright}{\kern0pt}\ submartingale{\isachardot}{\kern0pt}submartingale{\isacharunderscore}{\kern0pt}property{\isacharbrackleft}{\kern0pt}OF\ assms\ asm{\isacharbrackright}{\kern0pt}\ add{\isacharunderscore}{\kern0pt}mono{\isacharbrackleft}{\kern0pt}of\ {\isachardoublequoteopen}X\ i\ {\isacharunderscore}{\kern0pt}{\isachardoublequoteclose}\ {\isacharunderscore}{\kern0pt}\ {\isachardoublequoteopen}Y\ i\ {\isacharunderscore}{\kern0pt}{\isachardoublequoteclose}{\isacharbrackright}{\kern0pt}\ \isacommand{by}\isamarkupfalse%
\ force\isanewline
\ \ \isacommand{{\isacharbraceright}{\kern0pt}}\isamarkupfalse%
\isanewline
\ \ \isacommand{thus}\isamarkupfalse%
\ {\isacharquery}{\kern0pt}thesis\ \isacommand{using}\isamarkupfalse%
\ assms\ \isacommand{by}\isamarkupfalse%
\ {\isacharparenleft}{\kern0pt}unfold{\isacharunderscore}{\kern0pt}locales{\isacharparenright}{\kern0pt}\ {\isacharparenleft}{\kern0pt}auto\ simp\ add{\isacharcolon}{\kern0pt}\ borel{\isacharunderscore}{\kern0pt}measurable{\isacharunderscore}{\kern0pt}add\ random{\isacharunderscore}{\kern0pt}variable\ adapted\ integrable\ Y{\isachardot}{\kern0pt}random{\isacharunderscore}{\kern0pt}variable\ Y{\isachardot}{\kern0pt}adapted\ submartingale{\isachardot}{\kern0pt}integrable{\isacharparenright}{\kern0pt}\ \ \isanewline
\isacommand{qed}\isamarkupfalse%
%
\endisatagproof
{\isafoldproof}%
%
\isadelimproof
\isanewline
%
\endisadelimproof
\isanewline
\isacommand{lemma}\isamarkupfalse%
\ diff{\isacharbrackleft}{\kern0pt}intro{\isacharbrackright}{\kern0pt}{\isacharcolon}{\kern0pt}\isanewline
\ \ \isakeyword{assumes}\ {\isachardoublequoteopen}supermartingale\ M\ F\ Y{\isachardoublequoteclose}\isanewline
\ \ \isakeyword{shows}\ {\isachardoublequoteopen}submartingale\ M\ F\ {\isacharparenleft}{\kern0pt}{\isasymlambda}i\ {\isasymxi}{\isachardot}{\kern0pt}\ X\ i\ {\isasymxi}\ {\isacharminus}{\kern0pt}\ Y\ i\ {\isasymxi}{\isacharparenright}{\kern0pt}{\isachardoublequoteclose}\isanewline
%
\isadelimproof
%
\endisadelimproof
%
\isatagproof
\isacommand{proof}\isamarkupfalse%
\ {\isacharminus}{\kern0pt}\isanewline
\ \ \isacommand{interpret}\isamarkupfalse%
\ Y{\isacharcolon}{\kern0pt}\ supermartingale\ M\ F\ Y\ \isacommand{by}\isamarkupfalse%
\ {\isacharparenleft}{\kern0pt}rule\ assms{\isacharparenright}{\kern0pt}\isanewline
\ \ \isacommand{{\isacharbraceleft}{\kern0pt}}\isamarkupfalse%
\isanewline
\ \ \ \ \isacommand{fix}\isamarkupfalse%
\ i\ j\ {\isacharcolon}{\kern0pt}{\isacharcolon}{\kern0pt}\ {\isacharprime}{\kern0pt}b\ \isacommand{assume}\isamarkupfalse%
\ asm{\isacharcolon}{\kern0pt}\ {\isachardoublequoteopen}i\ {\isasymle}\ j{\isachardoublequoteclose}\isanewline
\ \ \ \ \isacommand{have}\isamarkupfalse%
\ {\isachardoublequoteopen}AE\ {\isasymxi}\ in\ M{\isachardot}{\kern0pt}\ X\ i\ {\isasymxi}\ {\isacharminus}{\kern0pt}\ Y\ i\ {\isasymxi}\ {\isasymle}\ cond{\isacharunderscore}{\kern0pt}exp\ M\ {\isacharparenleft}{\kern0pt}F\ i{\isacharparenright}{\kern0pt}\ {\isacharparenleft}{\kern0pt}{\isasymlambda}x{\isachardot}{\kern0pt}\ X\ j\ x\ {\isacharminus}{\kern0pt}\ Y\ j\ x{\isacharparenright}{\kern0pt}\ {\isasymxi}{\isachardoublequoteclose}\ \isanewline
\ \ \ \ \ \ \isacommand{using}\isamarkupfalse%
\ cond{\isacharunderscore}{\kern0pt}exp{\isacharunderscore}{\kern0pt}diff{\isacharbrackleft}{\kern0pt}OF\ integrable\ supermartingale{\isachardot}{\kern0pt}integrable{\isacharbrackleft}{\kern0pt}OF\ assms{\isacharbrackright}{\kern0pt}{\isacharcomma}{\kern0pt}\ of\ i\ j\ j{\isacharcomma}{\kern0pt}\ unfolded\ fun{\isacharunderscore}{\kern0pt}diff{\isacharunderscore}{\kern0pt}def{\isacharbrackright}{\kern0pt}\ \isanewline
\ \ \ \ \ \ \ \ \ \ \ \ submartingale{\isacharunderscore}{\kern0pt}property{\isacharbrackleft}{\kern0pt}OF\ asm{\isacharbrackright}{\kern0pt}\ supermartingale{\isachardot}{\kern0pt}supermartingale{\isacharunderscore}{\kern0pt}property{\isacharbrackleft}{\kern0pt}OF\ assms\ asm{\isacharbrackright}{\kern0pt}\ diff{\isacharunderscore}{\kern0pt}mono{\isacharbrackleft}{\kern0pt}of\ {\isachardoublequoteopen}X\ i\ {\isacharunderscore}{\kern0pt}{\isachardoublequoteclose}\ {\isacharunderscore}{\kern0pt}\ {\isacharunderscore}{\kern0pt}\ {\isachardoublequoteopen}Y\ i\ {\isacharunderscore}{\kern0pt}{\isachardoublequoteclose}{\isacharbrackright}{\kern0pt}\ \isacommand{by}\isamarkupfalse%
\ force\isanewline
\ \ \isacommand{{\isacharbraceright}{\kern0pt}}\isamarkupfalse%
\isanewline
\ \ \isacommand{thus}\isamarkupfalse%
\ {\isacharquery}{\kern0pt}thesis\ \isacommand{using}\isamarkupfalse%
\ assms\ \isacommand{by}\isamarkupfalse%
\ {\isacharparenleft}{\kern0pt}unfold{\isacharunderscore}{\kern0pt}locales{\isacharparenright}{\kern0pt}\ {\isacharparenleft}{\kern0pt}auto\ simp\ add{\isacharcolon}{\kern0pt}\ borel{\isacharunderscore}{\kern0pt}measurable{\isacharunderscore}{\kern0pt}diff\ random{\isacharunderscore}{\kern0pt}variable\ adapted\ integrable\ Y{\isachardot}{\kern0pt}random{\isacharunderscore}{\kern0pt}variable\ Y{\isachardot}{\kern0pt}adapted\ supermartingale{\isachardot}{\kern0pt}integrable{\isacharparenright}{\kern0pt}\ \ \isanewline
\isacommand{qed}\isamarkupfalse%
%
\endisatagproof
{\isafoldproof}%
%
\isadelimproof
\isanewline
%
\endisadelimproof
\isanewline
\isacommand{lemma}\isamarkupfalse%
\ scaleR{\isacharunderscore}{\kern0pt}nonneg{\isacharcolon}{\kern0pt}\ \isanewline
\ \ \isakeyword{assumes}\ {\isachardoublequoteopen}c\ {\isasymge}\ {\isadigit{0}}{\isachardoublequoteclose}\isanewline
\ \ \isakeyword{shows}\ {\isachardoublequoteopen}submartingale\ M\ F\ {\isacharparenleft}{\kern0pt}{\isasymlambda}i\ {\isasymxi}{\isachardot}{\kern0pt}\ c\ {\isacharasterisk}{\kern0pt}\isactrlsub R\ X\ i\ {\isasymxi}{\isacharparenright}{\kern0pt}{\isachardoublequoteclose}\isanewline
%
\isadelimproof
%
\endisadelimproof
%
\isatagproof
\isacommand{proof}\isamarkupfalse%
\isanewline
\ \ \isacommand{{\isacharbraceleft}{\kern0pt}}\isamarkupfalse%
\isanewline
\ \ \ \ \isacommand{fix}\isamarkupfalse%
\ i\ j\ {\isacharcolon}{\kern0pt}{\isacharcolon}{\kern0pt}\ {\isacharprime}{\kern0pt}b\ \isacommand{assume}\isamarkupfalse%
\ asm{\isacharcolon}{\kern0pt}\ {\isachardoublequoteopen}i\ {\isasymle}\ j{\isachardoublequoteclose}\isanewline
\ \ \ \ \isacommand{show}\isamarkupfalse%
\ {\isachardoublequoteopen}AE\ {\isasymxi}\ in\ M{\isachardot}{\kern0pt}\ c\ {\isacharasterisk}{\kern0pt}\isactrlsub R\ X\ i\ {\isasymxi}\ {\isasymle}\ cond{\isacharunderscore}{\kern0pt}exp\ M\ {\isacharparenleft}{\kern0pt}F\ i{\isacharparenright}{\kern0pt}\ {\isacharparenleft}{\kern0pt}{\isasymlambda}{\isasymxi}{\isachardot}{\kern0pt}\ c\ {\isacharasterisk}{\kern0pt}\isactrlsub R\ X\ j\ {\isasymxi}{\isacharparenright}{\kern0pt}\ {\isasymxi}{\isachardoublequoteclose}\ \ \isanewline
\ \ \ \ \ \ \isacommand{using}\isamarkupfalse%
\ cond{\isacharunderscore}{\kern0pt}exp{\isacharunderscore}{\kern0pt}scaleR{\isacharunderscore}{\kern0pt}right{\isacharbrackleft}{\kern0pt}OF\ integrable{\isacharcomma}{\kern0pt}\ of\ i\ {\isachardoublequoteopen}c{\isachardoublequoteclose}\ j{\isacharbrackright}{\kern0pt}\ submartingale{\isacharunderscore}{\kern0pt}property{\isacharbrackleft}{\kern0pt}OF\ asm{\isacharbrackright}{\kern0pt}\ \isacommand{by}\isamarkupfalse%
\ {\isacharparenleft}{\kern0pt}auto\ intro{\isacharbang}{\kern0pt}{\isacharcolon}{\kern0pt}\ scaleR{\isacharunderscore}{\kern0pt}left{\isacharunderscore}{\kern0pt}mono{\isacharbrackleft}{\kern0pt}OF\ {\isacharunderscore}{\kern0pt}\ assms{\isacharbrackright}{\kern0pt}{\isacharparenright}{\kern0pt}\isanewline
\ \ \isacommand{{\isacharbraceright}{\kern0pt}}\isamarkupfalse%
\isanewline
\isacommand{qed}\isamarkupfalse%
\ {\isacharparenleft}{\kern0pt}auto\ simp\ add{\isacharcolon}{\kern0pt}\ borel{\isacharunderscore}{\kern0pt}measurable{\isacharunderscore}{\kern0pt}integrable\ borel{\isacharunderscore}{\kern0pt}measurable{\isacharunderscore}{\kern0pt}scaleR\ integrable\ random{\isacharunderscore}{\kern0pt}variable\ adapted\ borel{\isacharunderscore}{\kern0pt}measurable{\isacharunderscore}{\kern0pt}const{\isacharunderscore}{\kern0pt}scaleR{\isacharparenright}{\kern0pt}%
\endisatagproof
{\isafoldproof}%
%
\isadelimproof
\isanewline
%
\endisadelimproof
\isanewline
\isacommand{lemma}\isamarkupfalse%
\ scaleR{\isacharunderscore}{\kern0pt}nonpos{\isacharcolon}{\kern0pt}\ \isanewline
\ \ \isakeyword{assumes}\ {\isachardoublequoteopen}c\ {\isasymle}\ {\isadigit{0}}{\isachardoublequoteclose}\isanewline
\ \ \isakeyword{shows}\ {\isachardoublequoteopen}supermartingale\ M\ F\ {\isacharparenleft}{\kern0pt}{\isasymlambda}i\ {\isasymxi}{\isachardot}{\kern0pt}\ c\ {\isacharasterisk}{\kern0pt}\isactrlsub R\ X\ i\ {\isasymxi}{\isacharparenright}{\kern0pt}{\isachardoublequoteclose}\isanewline
%
\isadelimproof
%
\endisadelimproof
%
\isatagproof
\isacommand{proof}\isamarkupfalse%
\isanewline
\ \ \isacommand{{\isacharbraceleft}{\kern0pt}}\isamarkupfalse%
\isanewline
\ \ \ \ \isacommand{fix}\isamarkupfalse%
\ i\ j\ {\isacharcolon}{\kern0pt}{\isacharcolon}{\kern0pt}\ {\isacharprime}{\kern0pt}b\ \isacommand{assume}\isamarkupfalse%
\ asm{\isacharcolon}{\kern0pt}\ {\isachardoublequoteopen}i\ {\isasymle}\ j{\isachardoublequoteclose}\isanewline
\ \ \ \ \isacommand{show}\isamarkupfalse%
\ {\isachardoublequoteopen}AE\ {\isasymxi}\ in\ M{\isachardot}{\kern0pt}\ c\ {\isacharasterisk}{\kern0pt}\isactrlsub R\ X\ i\ {\isasymxi}\ {\isasymge}\ cond{\isacharunderscore}{\kern0pt}exp\ M\ {\isacharparenleft}{\kern0pt}F\ i{\isacharparenright}{\kern0pt}\ {\isacharparenleft}{\kern0pt}{\isasymlambda}{\isasymxi}{\isachardot}{\kern0pt}\ c\ {\isacharasterisk}{\kern0pt}\isactrlsub R\ X\ j\ {\isasymxi}{\isacharparenright}{\kern0pt}\ {\isasymxi}{\isachardoublequoteclose}\ \isanewline
\ \ \ \ \ \ \isacommand{using}\isamarkupfalse%
\ cond{\isacharunderscore}{\kern0pt}exp{\isacharunderscore}{\kern0pt}scaleR{\isacharunderscore}{\kern0pt}right{\isacharbrackleft}{\kern0pt}OF\ integrable{\isacharcomma}{\kern0pt}\ of\ i\ {\isachardoublequoteopen}c{\isachardoublequoteclose}\ j{\isacharbrackright}{\kern0pt}\ submartingale{\isacharunderscore}{\kern0pt}property{\isacharbrackleft}{\kern0pt}OF\ asm{\isacharbrackright}{\kern0pt}\ \isacommand{by}\isamarkupfalse%
\ {\isacharparenleft}{\kern0pt}auto\ intro{\isacharbang}{\kern0pt}{\isacharcolon}{\kern0pt}\ scaleR{\isacharunderscore}{\kern0pt}left{\isacharunderscore}{\kern0pt}mono{\isacharunderscore}{\kern0pt}neg{\isacharbrackleft}{\kern0pt}OF\ {\isacharunderscore}{\kern0pt}\ assms{\isacharbrackright}{\kern0pt}{\isacharparenright}{\kern0pt}\isanewline
\ \ \isacommand{{\isacharbraceright}{\kern0pt}}\isamarkupfalse%
\isanewline
\isacommand{qed}\isamarkupfalse%
\ {\isacharparenleft}{\kern0pt}auto\ simp\ add{\isacharcolon}{\kern0pt}\ borel{\isacharunderscore}{\kern0pt}measurable{\isacharunderscore}{\kern0pt}integrable\ borel{\isacharunderscore}{\kern0pt}measurable{\isacharunderscore}{\kern0pt}scaleR\ integrable\ random{\isacharunderscore}{\kern0pt}variable\ adapted\ borel{\isacharunderscore}{\kern0pt}measurable{\isacharunderscore}{\kern0pt}const{\isacharunderscore}{\kern0pt}scaleR{\isacharparenright}{\kern0pt}%
\endisatagproof
{\isafoldproof}%
%
\isadelimproof
\isanewline
%
\endisadelimproof
\isanewline
\isacommand{lemma}\isamarkupfalse%
\ uminus{\isacharbrackleft}{\kern0pt}intro{\isacharbrackright}{\kern0pt}{\isacharcolon}{\kern0pt}\isanewline
\ \ \isakeyword{shows}\ {\isachardoublequoteopen}supermartingale\ M\ F\ {\isacharparenleft}{\kern0pt}{\isacharminus}{\kern0pt}\ X{\isacharparenright}{\kern0pt}{\isachardoublequoteclose}\isanewline
%
\isadelimproof
\ \ %
\endisadelimproof
%
\isatagproof
\isacommand{unfolding}\isamarkupfalse%
\ fun{\isacharunderscore}{\kern0pt}Compl{\isacharunderscore}{\kern0pt}def\ \isacommand{using}\isamarkupfalse%
\ scaleR{\isacharunderscore}{\kern0pt}nonpos{\isacharbrackleft}{\kern0pt}of\ {\isachardoublequoteopen}{\isacharminus}{\kern0pt}{\isadigit{1}}{\isachardoublequoteclose}{\isacharbrackright}{\kern0pt}\ \isacommand{by}\isamarkupfalse%
\ simp%
\endisatagproof
{\isafoldproof}%
%
\isadelimproof
\isanewline
%
\endisadelimproof
\isanewline
\isacommand{lemma}\isamarkupfalse%
\ max{\isacharcolon}{\kern0pt}\isanewline
\ \ \isakeyword{assumes}\ {\isachardoublequoteopen}submartingale\ M\ F\ Y{\isachardoublequoteclose}\isanewline
\ \ \isakeyword{shows}\ {\isachardoublequoteopen}submartingale\ M\ F\ {\isacharparenleft}{\kern0pt}{\isasymlambda}i\ {\isasymxi}{\isachardot}{\kern0pt}\ max\ {\isacharparenleft}{\kern0pt}X\ i\ {\isasymxi}{\isacharparenright}{\kern0pt}\ {\isacharparenleft}{\kern0pt}Y\ i\ {\isasymxi}{\isacharparenright}{\kern0pt}{\isacharparenright}{\kern0pt}{\isachardoublequoteclose}\isanewline
%
\isadelimproof
%
\endisadelimproof
%
\isatagproof
\isacommand{proof}\isamarkupfalse%
\ {\isacharparenleft}{\kern0pt}unfold{\isacharunderscore}{\kern0pt}locales{\isacharparenright}{\kern0pt}\isanewline
\ \ \isacommand{interpret}\isamarkupfalse%
\ Y{\isacharcolon}{\kern0pt}\ submartingale\ M\ F\ Y\ \isacommand{by}\isamarkupfalse%
\ {\isacharparenleft}{\kern0pt}rule\ assms{\isacharparenright}{\kern0pt}\isanewline
\ \ \isacommand{{\isacharbraceleft}{\kern0pt}}\isamarkupfalse%
\isanewline
\ \ \ \ \isacommand{fix}\isamarkupfalse%
\ i\ j\ {\isacharcolon}{\kern0pt}{\isacharcolon}{\kern0pt}\ {\isacharprime}{\kern0pt}b\ \isacommand{assume}\isamarkupfalse%
\ asm{\isacharcolon}{\kern0pt}\ {\isachardoublequoteopen}i\ {\isasymle}\ j{\isachardoublequoteclose}\isanewline
\ \ \ \ \isacommand{have}\isamarkupfalse%
\ {\isachardoublequoteopen}AE\ {\isasymxi}\ in\ M{\isachardot}{\kern0pt}\ max\ {\isacharparenleft}{\kern0pt}X\ i\ {\isasymxi}{\isacharparenright}{\kern0pt}\ {\isacharparenleft}{\kern0pt}Y\ i\ {\isasymxi}{\isacharparenright}{\kern0pt}\ {\isasymle}\ max\ {\isacharparenleft}{\kern0pt}cond{\isacharunderscore}{\kern0pt}exp\ M\ {\isacharparenleft}{\kern0pt}F\ i{\isacharparenright}{\kern0pt}\ {\isacharparenleft}{\kern0pt}X\ j{\isacharparenright}{\kern0pt}\ {\isasymxi}{\isacharparenright}{\kern0pt}\ {\isacharparenleft}{\kern0pt}cond{\isacharunderscore}{\kern0pt}exp\ M\ {\isacharparenleft}{\kern0pt}F\ i{\isacharparenright}{\kern0pt}\ {\isacharparenleft}{\kern0pt}Y\ j{\isacharparenright}{\kern0pt}\ {\isasymxi}{\isacharparenright}{\kern0pt}{\isachardoublequoteclose}\ \isacommand{using}\isamarkupfalse%
\ submartingale{\isacharunderscore}{\kern0pt}property\ Y{\isachardot}{\kern0pt}submartingale{\isacharunderscore}{\kern0pt}property\ asm\ \isacommand{unfolding}\isamarkupfalse%
\ max{\isacharunderscore}{\kern0pt}def\ \isacommand{by}\isamarkupfalse%
\ fastforce\isanewline
\ \ \ \ \isacommand{thus}\isamarkupfalse%
\ {\isachardoublequoteopen}AE\ {\isasymxi}\ in\ M{\isachardot}{\kern0pt}\ max\ {\isacharparenleft}{\kern0pt}X\ i\ {\isasymxi}{\isacharparenright}{\kern0pt}\ {\isacharparenleft}{\kern0pt}Y\ i\ {\isasymxi}{\isacharparenright}{\kern0pt}\ {\isasymle}\ cond{\isacharunderscore}{\kern0pt}exp\ M\ {\isacharparenleft}{\kern0pt}F\ i{\isacharparenright}{\kern0pt}\ {\isacharparenleft}{\kern0pt}{\isasymlambda}{\isasymxi}{\isachardot}{\kern0pt}\ max\ {\isacharparenleft}{\kern0pt}X\ j\ {\isasymxi}{\isacharparenright}{\kern0pt}\ {\isacharparenleft}{\kern0pt}Y\ j\ {\isasymxi}{\isacharparenright}{\kern0pt}{\isacharparenright}{\kern0pt}\ {\isasymxi}{\isachardoublequoteclose}\ \isacommand{using}\isamarkupfalse%
\ cond{\isacharunderscore}{\kern0pt}exp{\isacharunderscore}{\kern0pt}max{\isacharbrackleft}{\kern0pt}OF\ integrable\ Y{\isachardot}{\kern0pt}integrable{\isacharcomma}{\kern0pt}\ of\ i\ j\ j{\isacharbrackright}{\kern0pt}\ order{\isachardot}{\kern0pt}trans\ \isacommand{by}\isamarkupfalse%
\ fast\isanewline
\ \ \isacommand{{\isacharbraceright}{\kern0pt}}\isamarkupfalse%
\isanewline
\ \ \isacommand{show}\isamarkupfalse%
\ {\isachardoublequoteopen}{\isasymAnd}i{\isachardot}{\kern0pt}\ {\isacharparenleft}{\kern0pt}{\isasymlambda}{\isasymxi}{\isachardot}{\kern0pt}\ max\ {\isacharparenleft}{\kern0pt}X\ i\ {\isasymxi}{\isacharparenright}{\kern0pt}\ {\isacharparenleft}{\kern0pt}Y\ i\ {\isasymxi}{\isacharparenright}{\kern0pt}{\isacharparenright}{\kern0pt}\ {\isasymin}\ borel{\isacharunderscore}{\kern0pt}measurable\ M{\isachardoublequoteclose}\ {\isachardoublequoteopen}{\isasymAnd}i{\isachardot}{\kern0pt}\ {\isacharparenleft}{\kern0pt}{\isasymlambda}{\isasymxi}{\isachardot}{\kern0pt}\ max\ {\isacharparenleft}{\kern0pt}X\ i\ {\isasymxi}{\isacharparenright}{\kern0pt}\ {\isacharparenleft}{\kern0pt}Y\ i\ {\isasymxi}{\isacharparenright}{\kern0pt}{\isacharparenright}{\kern0pt}\ {\isasymin}\ borel{\isacharunderscore}{\kern0pt}measurable\ {\isacharparenleft}{\kern0pt}F\ i{\isacharparenright}{\kern0pt}{\isachardoublequoteclose}\ {\isachardoublequoteopen}{\isasymAnd}i{\isachardot}{\kern0pt}\ integrable\ M\ {\isacharparenleft}{\kern0pt}{\isasymlambda}{\isasymxi}{\isachardot}{\kern0pt}\ max\ {\isacharparenleft}{\kern0pt}X\ i\ {\isasymxi}{\isacharparenright}{\kern0pt}\ {\isacharparenleft}{\kern0pt}Y\ i\ {\isasymxi}{\isacharparenright}{\kern0pt}{\isacharparenright}{\kern0pt}{\isachardoublequoteclose}\ \isacommand{by}\isamarkupfalse%
\ {\isacharparenleft}{\kern0pt}force\ intro{\isacharcolon}{\kern0pt}\ Y{\isachardot}{\kern0pt}integrable\ integrable\ assms{\isacharparenright}{\kern0pt}{\isacharplus}{\kern0pt}\isanewline
\isacommand{qed}\isamarkupfalse%
%
\endisatagproof
{\isafoldproof}%
%
\isadelimproof
\isanewline
%
\endisadelimproof
\isanewline
\isacommand{lemma}\isamarkupfalse%
\ max{\isacharunderscore}{\kern0pt}{\isadigit{0}}{\isacharcolon}{\kern0pt}\isanewline
\ \ \isakeyword{shows}\ {\isachardoublequoteopen}submartingale\ M\ F\ {\isacharparenleft}{\kern0pt}{\isasymlambda}i\ {\isasymxi}{\isachardot}{\kern0pt}\ max\ {\isadigit{0}}\ {\isacharparenleft}{\kern0pt}X\ i\ {\isasymxi}{\isacharparenright}{\kern0pt}{\isacharparenright}{\kern0pt}{\isachardoublequoteclose}\isanewline
%
\isadelimproof
%
\endisadelimproof
%
\isatagproof
\isacommand{proof}\isamarkupfalse%
\ {\isacharminus}{\kern0pt}\isanewline
\ \ \isacommand{interpret}\isamarkupfalse%
\ zero{\isacharcolon}{\kern0pt}\ submartingale\ M\ F\ {\isachardoublequoteopen}{\isasymlambda}{\isacharunderscore}{\kern0pt}\ {\isacharunderscore}{\kern0pt}{\isachardot}{\kern0pt}\ {\isadigit{0}}{\isachardoublequoteclose}\ \isacommand{by}\isamarkupfalse%
\ {\isacharparenleft}{\kern0pt}intro\ martingale{\isacharunderscore}{\kern0pt}order{\isachardot}{\kern0pt}is{\isacharunderscore}{\kern0pt}submartingale{\isacharcomma}{\kern0pt}\ unfold{\isacharunderscore}{\kern0pt}locales{\isacharcomma}{\kern0pt}\ auto{\isacharparenright}{\kern0pt}\isanewline
\ \ \isacommand{show}\isamarkupfalse%
\ {\isacharquery}{\kern0pt}thesis\ \isacommand{by}\isamarkupfalse%
\ {\isacharparenleft}{\kern0pt}intro\ zero{\isachardot}{\kern0pt}max\ submartingale{\isacharunderscore}{\kern0pt}axioms{\isacharparenright}{\kern0pt}\isanewline
\isacommand{qed}\isamarkupfalse%
%
\endisatagproof
{\isafoldproof}%
%
\isadelimproof
\isanewline
%
\endisadelimproof
\isanewline
\isacommand{end}\isamarkupfalse%
\isanewline
\isanewline
\isacommand{lemma}\isamarkupfalse%
\ {\isacharparenleft}{\kern0pt}\isakeyword{in}\ submartingale{\isacharunderscore}{\kern0pt}lattice{\isacharparenright}{\kern0pt}\ sup{\isacharcolon}{\kern0pt}\isanewline
\ \ \isakeyword{assumes}\ {\isachardoublequoteopen}submartingale{\isacharunderscore}{\kern0pt}lattice\ M\ F\ Y{\isachardoublequoteclose}\isanewline
\ \ \isakeyword{shows}\ {\isachardoublequoteopen}submartingale{\isacharunderscore}{\kern0pt}lattice\ M\ F\ {\isacharparenleft}{\kern0pt}{\isasymlambda}i\ {\isasymxi}{\isachardot}{\kern0pt}\ sup\ {\isacharparenleft}{\kern0pt}X\ i\ {\isasymxi}{\isacharparenright}{\kern0pt}\ {\isacharparenleft}{\kern0pt}Y\ i\ {\isasymxi}{\isacharparenright}{\kern0pt}{\isacharparenright}{\kern0pt}{\isachardoublequoteclose}\isanewline
%
\isadelimproof
\ \ %
\endisadelimproof
%
\isatagproof
\isacommand{using}\isamarkupfalse%
\ submartingale{\isacharunderscore}{\kern0pt}lattice{\isachardot}{\kern0pt}intro\ submartingale{\isachardot}{\kern0pt}max{\isacharbrackleft}{\kern0pt}OF\ submartingale{\isacharunderscore}{\kern0pt}axioms\ assms{\isacharbrackleft}{\kern0pt}THEN\ submartingale{\isacharunderscore}{\kern0pt}lattice{\isachardot}{\kern0pt}axioms{\isacharbrackright}{\kern0pt}{\isacharbrackright}{\kern0pt}\ \isacommand{unfolding}\isamarkupfalse%
\ sup{\isacharunderscore}{\kern0pt}max{\isacharbrackleft}{\kern0pt}symmetric{\isacharbrackright}{\kern0pt}\ \isacommand{{\isachardot}{\kern0pt}}\isamarkupfalse%
%
\endisatagproof
{\isafoldproof}%
%
\isadelimproof
\isanewline
%
\endisadelimproof
\isanewline
\isacommand{lemma}\isamarkupfalse%
\ {\isacharparenleft}{\kern0pt}\isakeyword{in}\ adapted{\isacharunderscore}{\kern0pt}process{\isacharunderscore}{\kern0pt}order{\isacharparenright}{\kern0pt}\ submartingale{\isacharunderscore}{\kern0pt}of{\isacharunderscore}{\kern0pt}cond{\isacharunderscore}{\kern0pt}exp{\isacharunderscore}{\kern0pt}diff{\isacharunderscore}{\kern0pt}nonneg{\isacharcolon}{\kern0pt}\isanewline
\ \ \isakeyword{assumes}\ integrable{\isacharcolon}{\kern0pt}\ {\isachardoublequoteopen}{\isasymAnd}i{\isachardot}{\kern0pt}\ integrable\ M\ {\isacharparenleft}{\kern0pt}X\ i{\isacharparenright}{\kern0pt}{\isachardoublequoteclose}\ \isanewline
\ \ \ \ \ \ \isakeyword{and}\ diff{\isacharunderscore}{\kern0pt}nonneg{\isacharcolon}{\kern0pt}\ {\isachardoublequoteopen}{\isasymAnd}i\ j{\isachardot}{\kern0pt}\ i\ {\isasymle}\ j\ {\isasymLongrightarrow}\ AE\ x\ in\ M{\isachardot}{\kern0pt}\ {\isadigit{0}}\ {\isasymle}\ cond{\isacharunderscore}{\kern0pt}exp\ M\ {\isacharparenleft}{\kern0pt}F\ i{\isacharparenright}{\kern0pt}\ {\isacharparenleft}{\kern0pt}{\isasymlambda}{\isasymxi}{\isachardot}{\kern0pt}\ X\ j\ {\isasymxi}\ {\isacharminus}{\kern0pt}\ X\ i\ {\isasymxi}{\isacharparenright}{\kern0pt}\ x{\isachardoublequoteclose}\isanewline
\ \ \ \ \isakeyword{shows}\ {\isachardoublequoteopen}submartingale\ M\ F\ X{\isachardoublequoteclose}\isanewline
%
\isadelimproof
%
\endisadelimproof
%
\isatagproof
\isacommand{proof}\isamarkupfalse%
\ {\isacharparenleft}{\kern0pt}unfold{\isacharunderscore}{\kern0pt}locales{\isacharparenright}{\kern0pt}\isanewline
\ \ \isacommand{{\isacharbraceleft}{\kern0pt}}\isamarkupfalse%
\isanewline
\ \ \ \ \isacommand{fix}\isamarkupfalse%
\ i\ j\ {\isacharcolon}{\kern0pt}{\isacharcolon}{\kern0pt}\ {\isacharprime}{\kern0pt}t\ \isacommand{assume}\isamarkupfalse%
\ asm{\isacharcolon}{\kern0pt}\ {\isachardoublequoteopen}i\ {\isasymle}\ j{\isachardoublequoteclose}\isanewline
\ \ \ \ \isacommand{show}\isamarkupfalse%
\ {\isachardoublequoteopen}AE\ {\isasymxi}\ in\ M{\isachardot}{\kern0pt}\ X\ i\ {\isasymxi}\ {\isasymle}\ cond{\isacharunderscore}{\kern0pt}exp\ M\ {\isacharparenleft}{\kern0pt}F\ i{\isacharparenright}{\kern0pt}\ {\isacharparenleft}{\kern0pt}X\ j{\isacharparenright}{\kern0pt}\ {\isasymxi}{\isachardoublequoteclose}\ \isanewline
\ \ \ \ \ \ \isacommand{using}\isamarkupfalse%
\ diff{\isacharunderscore}{\kern0pt}nonneg{\isacharbrackleft}{\kern0pt}OF\ asm{\isacharbrackright}{\kern0pt}\ cond{\isacharunderscore}{\kern0pt}exp{\isacharunderscore}{\kern0pt}diff{\isacharbrackleft}{\kern0pt}OF\ integrable{\isacharparenleft}{\kern0pt}{\isadigit{1}}{\isacharcomma}{\kern0pt}{\isadigit{1}}{\isacharparenright}{\kern0pt}{\isacharcomma}{\kern0pt}\ of\ i\ j\ i{\isacharbrackright}{\kern0pt}\ cond{\isacharunderscore}{\kern0pt}exp{\isacharunderscore}{\kern0pt}F{\isacharunderscore}{\kern0pt}meas{\isacharbrackleft}{\kern0pt}OF\ integrable\ adapted{\isacharcomma}{\kern0pt}\ of\ i{\isacharbrackright}{\kern0pt}\ \isacommand{by}\isamarkupfalse%
\ fastforce\isanewline
\ \ \isacommand{{\isacharbraceright}{\kern0pt}}\isamarkupfalse%
\isanewline
\isacommand{qed}\isamarkupfalse%
\ {\isacharparenleft}{\kern0pt}intro\ integrable{\isacharparenright}{\kern0pt}%
\endisatagproof
{\isafoldproof}%
%
\isadelimproof
\isanewline
%
\endisadelimproof
\isanewline
\isacommand{lemma}\isamarkupfalse%
\ {\isacharparenleft}{\kern0pt}\isakeyword{in}\ adapted{\isacharunderscore}{\kern0pt}process{\isacharunderscore}{\kern0pt}order{\isacharparenright}{\kern0pt}\ submartingale{\isacharunderscore}{\kern0pt}of{\isacharunderscore}{\kern0pt}set{\isacharunderscore}{\kern0pt}integral{\isacharunderscore}{\kern0pt}le{\isacharcolon}{\kern0pt}\isanewline
\ \ \isakeyword{assumes}\ integrable{\isacharcolon}{\kern0pt}\ {\isachardoublequoteopen}{\isasymAnd}i{\isachardot}{\kern0pt}\ integrable\ M\ {\isacharparenleft}{\kern0pt}X\ i{\isacharparenright}{\kern0pt}{\isachardoublequoteclose}\isanewline
\ \ \ \ \ \ \isakeyword{and}\ {\isachardoublequoteopen}{\isasymAnd}A\ i\ j{\isachardot}{\kern0pt}\ i\ {\isasymle}\ j\ {\isasymLongrightarrow}\ A\ {\isasymin}\ F\ i\ {\isasymLongrightarrow}\ set{\isacharunderscore}{\kern0pt}lebesgue{\isacharunderscore}{\kern0pt}integral\ M\ A\ {\isacharparenleft}{\kern0pt}X\ i{\isacharparenright}{\kern0pt}\ {\isasymle}\ set{\isacharunderscore}{\kern0pt}lebesgue{\isacharunderscore}{\kern0pt}integral\ M\ A\ {\isacharparenleft}{\kern0pt}X\ j{\isacharparenright}{\kern0pt}{\isachardoublequoteclose}\isanewline
\ \ \ \ \isakeyword{shows}\ {\isachardoublequoteopen}submartingale\ M\ F\ X{\isachardoublequoteclose}\isanewline
%
\isadelimproof
%
\endisadelimproof
%
\isatagproof
\isacommand{proof}\isamarkupfalse%
\ {\isacharparenleft}{\kern0pt}unfold{\isacharunderscore}{\kern0pt}locales{\isacharparenright}{\kern0pt}\isanewline
\ \ \isacommand{{\isacharbraceleft}{\kern0pt}}\isamarkupfalse%
\isanewline
\ \ \ \ \isacommand{fix}\isamarkupfalse%
\ i\ j\ {\isacharcolon}{\kern0pt}{\isacharcolon}{\kern0pt}\ {\isacharprime}{\kern0pt}t\ \isacommand{assume}\isamarkupfalse%
\ asm{\isacharcolon}{\kern0pt}\ {\isachardoublequoteopen}i\ {\isasymle}\ j{\isachardoublequoteclose}\isanewline
\ \ \ \ \isacommand{interpret}\isamarkupfalse%
\ sigma{\isacharunderscore}{\kern0pt}finite{\isacharunderscore}{\kern0pt}measure\ {\isachardoublequoteopen}restr{\isacharunderscore}{\kern0pt}to{\isacharunderscore}{\kern0pt}subalg\ M\ {\isacharparenleft}{\kern0pt}F\ i{\isacharparenright}{\kern0pt}{\isachardoublequoteclose}\ \isacommand{by}\isamarkupfalse%
\ {\isacharparenleft}{\kern0pt}simp\ add{\isacharcolon}{\kern0pt}\ sigma{\isacharunderscore}{\kern0pt}fin{\isacharunderscore}{\kern0pt}subalg{\isacharparenright}{\kern0pt}\isanewline
\ \ \ \ \isacommand{{\isacharbraceleft}{\kern0pt}}\isamarkupfalse%
\isanewline
\ \ \ \ \ \ \isacommand{fix}\isamarkupfalse%
\ A\ \isacommand{assume}\isamarkupfalse%
\ {\isachardoublequoteopen}A\ {\isasymin}\ restr{\isacharunderscore}{\kern0pt}to{\isacharunderscore}{\kern0pt}subalg\ M\ {\isacharparenleft}{\kern0pt}F\ i{\isacharparenright}{\kern0pt}{\isachardoublequoteclose}\isanewline
\ \ \ \ \ \ \isacommand{hence}\isamarkupfalse%
\ {\isacharasterisk}{\kern0pt}{\isacharcolon}{\kern0pt}\ {\isachardoublequoteopen}A\ {\isasymin}\ F\ i{\isachardoublequoteclose}\ \isacommand{using}\isamarkupfalse%
\ sets{\isacharunderscore}{\kern0pt}restr{\isacharunderscore}{\kern0pt}to{\isacharunderscore}{\kern0pt}subalg\ subalgebra\ \isacommand{by}\isamarkupfalse%
\ blast\isanewline
\ \ \ \ \ \ \isacommand{have}\isamarkupfalse%
\ {\isachardoublequoteopen}set{\isacharunderscore}{\kern0pt}lebesgue{\isacharunderscore}{\kern0pt}integral\ {\isacharparenleft}{\kern0pt}restr{\isacharunderscore}{\kern0pt}to{\isacharunderscore}{\kern0pt}subalg\ M\ {\isacharparenleft}{\kern0pt}F\ i{\isacharparenright}{\kern0pt}{\isacharparenright}{\kern0pt}\ A\ {\isacharparenleft}{\kern0pt}X\ i{\isacharparenright}{\kern0pt}\ {\isacharequal}{\kern0pt}\ set{\isacharunderscore}{\kern0pt}lebesgue{\isacharunderscore}{\kern0pt}integral\ M\ A\ {\isacharparenleft}{\kern0pt}X\ i{\isacharparenright}{\kern0pt}{\isachardoublequoteclose}\ \isacommand{using}\isamarkupfalse%
\ {\isacharasterisk}{\kern0pt}\ subalg\ \isacommand{by}\isamarkupfalse%
\ {\isacharparenleft}{\kern0pt}auto\ simp{\isacharcolon}{\kern0pt}\ set{\isacharunderscore}{\kern0pt}lebesgue{\isacharunderscore}{\kern0pt}integral{\isacharunderscore}{\kern0pt}def\ intro{\isacharcolon}{\kern0pt}\ integral{\isacharunderscore}{\kern0pt}subalgebra{\isadigit{2}}\ borel{\isacharunderscore}{\kern0pt}measurable{\isacharunderscore}{\kern0pt}scaleR\ adapted\ borel{\isacharunderscore}{\kern0pt}measurable{\isacharunderscore}{\kern0pt}indicator{\isacharparenright}{\kern0pt}\ \isanewline
\ \ \ \ \ \ \isacommand{also}\isamarkupfalse%
\ \isacommand{have}\isamarkupfalse%
\ {\isachardoublequoteopen}{\isachardot}{\kern0pt}{\isachardot}{\kern0pt}{\isachardot}{\kern0pt}\ {\isasymle}\ set{\isacharunderscore}{\kern0pt}lebesgue{\isacharunderscore}{\kern0pt}integral\ M\ A\ {\isacharparenleft}{\kern0pt}cond{\isacharunderscore}{\kern0pt}exp\ M\ {\isacharparenleft}{\kern0pt}F\ i{\isacharparenright}{\kern0pt}\ {\isacharparenleft}{\kern0pt}X\ j{\isacharparenright}{\kern0pt}{\isacharparenright}{\kern0pt}{\isachardoublequoteclose}\ \isacommand{using}\isamarkupfalse%
\ {\isacharasterisk}{\kern0pt}\ assms{\isacharparenleft}{\kern0pt}{\isadigit{2}}{\isacharparenright}{\kern0pt}{\isacharbrackleft}{\kern0pt}OF\ asm{\isacharbrackright}{\kern0pt}\ cond{\isacharunderscore}{\kern0pt}exp{\isacharunderscore}{\kern0pt}set{\isacharunderscore}{\kern0pt}integral{\isacharbrackleft}{\kern0pt}OF\ integrable{\isacharbrackright}{\kern0pt}\ \isacommand{by}\isamarkupfalse%
\ auto\isanewline
\ \ \ \ \ \ \isacommand{also}\isamarkupfalse%
\ \isacommand{have}\isamarkupfalse%
\ {\isachardoublequoteopen}{\isachardot}{\kern0pt}{\isachardot}{\kern0pt}{\isachardot}{\kern0pt}\ {\isacharequal}{\kern0pt}\ set{\isacharunderscore}{\kern0pt}lebesgue{\isacharunderscore}{\kern0pt}integral\ {\isacharparenleft}{\kern0pt}restr{\isacharunderscore}{\kern0pt}to{\isacharunderscore}{\kern0pt}subalg\ M\ {\isacharparenleft}{\kern0pt}F\ i{\isacharparenright}{\kern0pt}{\isacharparenright}{\kern0pt}\ A\ {\isacharparenleft}{\kern0pt}cond{\isacharunderscore}{\kern0pt}exp\ M\ {\isacharparenleft}{\kern0pt}F\ i{\isacharparenright}{\kern0pt}\ {\isacharparenleft}{\kern0pt}X\ j{\isacharparenright}{\kern0pt}{\isacharparenright}{\kern0pt}{\isachardoublequoteclose}\ \isacommand{using}\isamarkupfalse%
\ {\isacharasterisk}{\kern0pt}\ subalg\ \isacommand{by}\isamarkupfalse%
\ {\isacharparenleft}{\kern0pt}auto\ simp{\isacharcolon}{\kern0pt}\ set{\isacharunderscore}{\kern0pt}lebesgue{\isacharunderscore}{\kern0pt}integral{\isacharunderscore}{\kern0pt}def\ intro{\isacharbang}{\kern0pt}{\isacharcolon}{\kern0pt}\ integral{\isacharunderscore}{\kern0pt}subalgebra{\isadigit{2}}{\isacharbrackleft}{\kern0pt}symmetric{\isacharbrackright}{\kern0pt}\ borel{\isacharunderscore}{\kern0pt}measurable{\isacharunderscore}{\kern0pt}scaleR\ borel{\isacharunderscore}{\kern0pt}measurable{\isacharunderscore}{\kern0pt}cond{\isacharunderscore}{\kern0pt}exp\ borel{\isacharunderscore}{\kern0pt}measurable{\isacharunderscore}{\kern0pt}indicator{\isacharparenright}{\kern0pt}\isanewline
\ \ \ \ \ \ \isacommand{finally}\isamarkupfalse%
\ \isacommand{have}\isamarkupfalse%
\ {\isachardoublequoteopen}{\isadigit{0}}\ {\isasymle}\ set{\isacharunderscore}{\kern0pt}lebesgue{\isacharunderscore}{\kern0pt}integral\ {\isacharparenleft}{\kern0pt}restr{\isacharunderscore}{\kern0pt}to{\isacharunderscore}{\kern0pt}subalg\ M\ {\isacharparenleft}{\kern0pt}F\ i{\isacharparenright}{\kern0pt}{\isacharparenright}{\kern0pt}\ A\ {\isacharparenleft}{\kern0pt}{\isasymlambda}{\isasymxi}{\isachardot}{\kern0pt}\ cond{\isacharunderscore}{\kern0pt}exp\ M\ {\isacharparenleft}{\kern0pt}F\ i{\isacharparenright}{\kern0pt}\ {\isacharparenleft}{\kern0pt}X\ j{\isacharparenright}{\kern0pt}\ {\isasymxi}\ {\isacharminus}{\kern0pt}\ X\ i\ {\isasymxi}{\isacharparenright}{\kern0pt}{\isachardoublequoteclose}\ \isacommand{using}\isamarkupfalse%
\ {\isacharasterisk}{\kern0pt}\ subalg\ \isacommand{by}\isamarkupfalse%
\ {\isacharparenleft}{\kern0pt}subst\ set{\isacharunderscore}{\kern0pt}integral{\isacharunderscore}{\kern0pt}diff{\isacharcomma}{\kern0pt}\ auto\ simp\ add{\isacharcolon}{\kern0pt}\ set{\isacharunderscore}{\kern0pt}integrable{\isacharunderscore}{\kern0pt}def\ sets{\isacharunderscore}{\kern0pt}restr{\isacharunderscore}{\kern0pt}to{\isacharunderscore}{\kern0pt}subalg\ intro{\isacharbang}{\kern0pt}{\isacharcolon}{\kern0pt}\ integrable\ adapted\ integrable{\isacharunderscore}{\kern0pt}in{\isacharunderscore}{\kern0pt}subalg\ borel{\isacharunderscore}{\kern0pt}measurable{\isacharunderscore}{\kern0pt}scaleR\ borel{\isacharunderscore}{\kern0pt}measurable{\isacharunderscore}{\kern0pt}indicator\ borel{\isacharunderscore}{\kern0pt}measurable{\isacharunderscore}{\kern0pt}cond{\isacharunderscore}{\kern0pt}exp\ integrable{\isacharunderscore}{\kern0pt}mult{\isacharunderscore}{\kern0pt}indicator{\isacharparenright}{\kern0pt}\isanewline
\ \ \ \ \isacommand{{\isacharbraceright}{\kern0pt}}\isamarkupfalse%
\isanewline
\ \ \ \ \isacommand{hence}\isamarkupfalse%
\ {\isachardoublequoteopen}AE\ {\isasymxi}\ in\ restr{\isacharunderscore}{\kern0pt}to{\isacharunderscore}{\kern0pt}subalg\ M\ {\isacharparenleft}{\kern0pt}F\ i{\isacharparenright}{\kern0pt}{\isachardot}{\kern0pt}\ {\isadigit{0}}\ {\isasymle}\ cond{\isacharunderscore}{\kern0pt}exp\ M\ {\isacharparenleft}{\kern0pt}F\ i{\isacharparenright}{\kern0pt}\ {\isacharparenleft}{\kern0pt}X\ j{\isacharparenright}{\kern0pt}\ {\isasymxi}\ {\isacharminus}{\kern0pt}\ X\ i\ {\isasymxi}{\isachardoublequoteclose}\ \isacommand{by}\isamarkupfalse%
\ {\isacharparenleft}{\kern0pt}intro\ density{\isacharunderscore}{\kern0pt}nonneg\ integrable{\isacharunderscore}{\kern0pt}in{\isacharunderscore}{\kern0pt}subalg\ subalg\ borel{\isacharunderscore}{\kern0pt}measurable{\isacharunderscore}{\kern0pt}diff\ borel{\isacharunderscore}{\kern0pt}measurable{\isacharunderscore}{\kern0pt}cond{\isacharunderscore}{\kern0pt}exp\ adapted\ Bochner{\isacharunderscore}{\kern0pt}Integration{\isachardot}{\kern0pt}integrable{\isacharunderscore}{\kern0pt}diff\ integrable{\isacharunderscore}{\kern0pt}cond{\isacharunderscore}{\kern0pt}exp\ integrable{\isacharparenright}{\kern0pt}\isanewline
\ \ \ \ \isacommand{thus}\isamarkupfalse%
\ {\isachardoublequoteopen}AE\ {\isasymxi}\ in\ M{\isachardot}{\kern0pt}\ X\ i\ {\isasymxi}\ {\isasymle}\ cond{\isacharunderscore}{\kern0pt}exp\ M\ {\isacharparenleft}{\kern0pt}F\ i{\isacharparenright}{\kern0pt}\ {\isacharparenleft}{\kern0pt}X\ j{\isacharparenright}{\kern0pt}\ {\isasymxi}{\isachardoublequoteclose}\ \isacommand{using}\isamarkupfalse%
\ AE{\isacharunderscore}{\kern0pt}restr{\isacharunderscore}{\kern0pt}to{\isacharunderscore}{\kern0pt}subalg{\isacharbrackleft}{\kern0pt}OF\ subalg{\isacharbrackright}{\kern0pt}\ \isacommand{by}\isamarkupfalse%
\ simp\isanewline
\ \ \ \ \isacommand{{\isacharbraceright}{\kern0pt}}\isamarkupfalse%
\isanewline
\isacommand{qed}\isamarkupfalse%
\ {\isacharparenleft}{\kern0pt}intro\ integrable{\isacharparenright}{\kern0pt}%
\endisatagproof
{\isafoldproof}%
%
\isadelimproof
%
\endisadelimproof
%
\isadelimdocument
%
\endisadelimdocument
%
\isatagdocument
%
\isamarkupsubsection{Supermartingale Stuff%
}
\isamarkuptrue%
%
\endisatagdocument
{\isafolddocument}%
%
\isadelimdocument
%
\endisadelimdocument
\isacommand{context}\isamarkupfalse%
\ supermartingale\isanewline
\isakeyword{begin}\isanewline
\isanewline
\isacommand{lemma}\isamarkupfalse%
\ set{\isacharunderscore}{\kern0pt}integral{\isacharunderscore}{\kern0pt}ge{\isacharcolon}{\kern0pt}\isanewline
\ \ \isakeyword{assumes}\ {\isachardoublequoteopen}A\ {\isasymin}\ F\ i{\isachardoublequoteclose}\ {\isachardoublequoteopen}i\ {\isasymle}\ j{\isachardoublequoteclose}\isanewline
\ \ \isakeyword{shows}\ {\isachardoublequoteopen}set{\isacharunderscore}{\kern0pt}lebesgue{\isacharunderscore}{\kern0pt}integral\ M\ A\ {\isacharparenleft}{\kern0pt}X\ i{\isacharparenright}{\kern0pt}\ {\isasymge}\ set{\isacharunderscore}{\kern0pt}lebesgue{\isacharunderscore}{\kern0pt}integral\ M\ A\ {\isacharparenleft}{\kern0pt}X\ j{\isacharparenright}{\kern0pt}{\isachardoublequoteclose}\isanewline
%
\isadelimproof
\ \ %
\endisadelimproof
%
\isatagproof
\isacommand{unfolding}\isamarkupfalse%
\ cond{\isacharunderscore}{\kern0pt}exp{\isacharunderscore}{\kern0pt}set{\isacharunderscore}{\kern0pt}integral{\isacharbrackleft}{\kern0pt}OF\ integrable\ assms{\isacharparenleft}{\kern0pt}{\isadigit{1}}{\isacharparenright}{\kern0pt}{\isacharcomma}{\kern0pt}\ of\ j{\isacharbrackright}{\kern0pt}\isanewline
\ \ \isacommand{using}\isamarkupfalse%
\ supermartingale{\isacharunderscore}{\kern0pt}property{\isacharbrackleft}{\kern0pt}OF\ assms{\isacharparenleft}{\kern0pt}{\isadigit{2}}{\isacharparenright}{\kern0pt}{\isacharbrackright}{\kern0pt}\ \isanewline
\ \ \isacommand{by}\isamarkupfalse%
\ {\isacharparenleft}{\kern0pt}simp\ only{\isacharcolon}{\kern0pt}\ set{\isacharunderscore}{\kern0pt}lebesgue{\isacharunderscore}{\kern0pt}integral{\isacharunderscore}{\kern0pt}def{\isacharcomma}{\kern0pt}\ intro\ integral{\isacharunderscore}{\kern0pt}mono{\isacharunderscore}{\kern0pt}AE{\isacharunderscore}{\kern0pt}banach{\isacharcomma}{\kern0pt}\ metis\ assms{\isacharparenleft}{\kern0pt}{\isadigit{1}}{\isacharparenright}{\kern0pt}\ in{\isacharunderscore}{\kern0pt}mono\ integrable{\isacharunderscore}{\kern0pt}mult{\isacharunderscore}{\kern0pt}indicator\ subalgebra\ subalgebra{\isacharunderscore}{\kern0pt}def\ integrable{\isacharunderscore}{\kern0pt}cond{\isacharunderscore}{\kern0pt}exp{\isacharcomma}{\kern0pt}\ metis\ assms{\isacharparenleft}{\kern0pt}{\isadigit{1}}{\isacharparenright}{\kern0pt}\ in{\isacharunderscore}{\kern0pt}mono\ integrable\ integrable{\isacharunderscore}{\kern0pt}mult{\isacharunderscore}{\kern0pt}indicator\ subalgebra\ subalgebra{\isacharunderscore}{\kern0pt}def{\isacharparenright}{\kern0pt}\isanewline
\ \ \ \ \ {\isacharparenleft}{\kern0pt}auto\ intro{\isacharcolon}{\kern0pt}\ scaleR{\isacharunderscore}{\kern0pt}left{\isacharunderscore}{\kern0pt}mono{\isacharparenright}{\kern0pt}%
\endisatagproof
{\isafoldproof}%
%
\isadelimproof
\isanewline
%
\endisadelimproof
\isanewline
\isacommand{lemma}\isamarkupfalse%
\ cond{\isacharunderscore}{\kern0pt}exp{\isacharunderscore}{\kern0pt}diff{\isacharunderscore}{\kern0pt}nonneg{\isacharcolon}{\kern0pt}\isanewline
\ \ \isakeyword{assumes}\ {\isachardoublequoteopen}i\ {\isasymle}\ j{\isachardoublequoteclose}\isanewline
\ \ \isakeyword{shows}\ {\isachardoublequoteopen}AE\ x\ in\ M{\isachardot}{\kern0pt}\ {\isadigit{0}}\ {\isasymle}\ cond{\isacharunderscore}{\kern0pt}exp\ M\ {\isacharparenleft}{\kern0pt}F\ i{\isacharparenright}{\kern0pt}\ {\isacharparenleft}{\kern0pt}{\isasymlambda}{\isasymxi}{\isachardot}{\kern0pt}\ X\ i\ {\isasymxi}\ {\isacharminus}{\kern0pt}\ X\ j\ {\isasymxi}{\isacharparenright}{\kern0pt}\ x{\isachardoublequoteclose}\isanewline
%
\isadelimproof
\ \ %
\endisadelimproof
%
\isatagproof
\isacommand{using}\isamarkupfalse%
\ supermartingale{\isacharunderscore}{\kern0pt}property{\isacharbrackleft}{\kern0pt}OF\ assms{\isacharbrackright}{\kern0pt}\ cond{\isacharunderscore}{\kern0pt}exp{\isacharunderscore}{\kern0pt}diff{\isacharbrackleft}{\kern0pt}OF\ integrable{\isacharparenleft}{\kern0pt}{\isadigit{1}}{\isacharcomma}{\kern0pt}{\isadigit{1}}{\isacharparenright}{\kern0pt}{\isacharcomma}{\kern0pt}\ of\ i\ i\ j{\isacharbrackright}{\kern0pt}\ cond{\isacharunderscore}{\kern0pt}exp{\isacharunderscore}{\kern0pt}F{\isacharunderscore}{\kern0pt}meas{\isacharbrackleft}{\kern0pt}OF\ integrable\ adapted{\isacharcomma}{\kern0pt}\ of\ i{\isacharbrackright}{\kern0pt}\ \isacommand{by}\isamarkupfalse%
\ fastforce%
\endisatagproof
{\isafoldproof}%
%
\isadelimproof
\isanewline
%
\endisadelimproof
\isanewline
\isacommand{lemma}\isamarkupfalse%
\ add{\isacharbrackleft}{\kern0pt}intro{\isacharbrackright}{\kern0pt}{\isacharcolon}{\kern0pt}\isanewline
\ \ \isakeyword{assumes}\ {\isachardoublequoteopen}supermartingale\ M\ F\ Y{\isachardoublequoteclose}\isanewline
\ \ \isakeyword{shows}\ {\isachardoublequoteopen}supermartingale\ M\ F\ {\isacharparenleft}{\kern0pt}{\isasymlambda}i\ {\isasymxi}{\isachardot}{\kern0pt}\ X\ i\ {\isasymxi}\ {\isacharplus}{\kern0pt}\ Y\ i\ {\isasymxi}{\isacharparenright}{\kern0pt}{\isachardoublequoteclose}\isanewline
%
\isadelimproof
%
\endisadelimproof
%
\isatagproof
\isacommand{proof}\isamarkupfalse%
\ {\isacharminus}{\kern0pt}\isanewline
\ \ \isacommand{interpret}\isamarkupfalse%
\ Y{\isacharcolon}{\kern0pt}\ supermartingale\ M\ F\ Y\ \isacommand{by}\isamarkupfalse%
\ {\isacharparenleft}{\kern0pt}rule\ assms{\isacharparenright}{\kern0pt}\isanewline
\ \ \isacommand{{\isacharbraceleft}{\kern0pt}}\isamarkupfalse%
\isanewline
\ \ \ \ \isacommand{fix}\isamarkupfalse%
\ i\ j\ {\isacharcolon}{\kern0pt}{\isacharcolon}{\kern0pt}\ {\isacharprime}{\kern0pt}b\ \isacommand{assume}\isamarkupfalse%
\ asm{\isacharcolon}{\kern0pt}\ {\isachardoublequoteopen}i\ {\isasymle}\ j{\isachardoublequoteclose}\isanewline
\ \ \ \ \isacommand{have}\isamarkupfalse%
\ {\isachardoublequoteopen}AE\ {\isasymxi}\ in\ M{\isachardot}{\kern0pt}\ X\ i\ {\isasymxi}\ {\isacharplus}{\kern0pt}\ Y\ i\ {\isasymxi}\ {\isasymge}\ cond{\isacharunderscore}{\kern0pt}exp\ M\ {\isacharparenleft}{\kern0pt}F\ i{\isacharparenright}{\kern0pt}\ {\isacharparenleft}{\kern0pt}{\isasymlambda}x{\isachardot}{\kern0pt}\ X\ j\ x\ {\isacharplus}{\kern0pt}\ Y\ j\ x{\isacharparenright}{\kern0pt}\ {\isasymxi}{\isachardoublequoteclose}\ \isanewline
\ \ \ \ \ \ \isacommand{using}\isamarkupfalse%
\ cond{\isacharunderscore}{\kern0pt}exp{\isacharunderscore}{\kern0pt}add{\isacharbrackleft}{\kern0pt}OF\ integrable\ supermartingale{\isachardot}{\kern0pt}integrable{\isacharbrackleft}{\kern0pt}OF\ assms{\isacharbrackright}{\kern0pt}{\isacharcomma}{\kern0pt}\ of\ i\ j\ j{\isacharbrackright}{\kern0pt}\ \isanewline
\ \ \ \ \ \ \ \ \ \ \ \ supermartingale{\isacharunderscore}{\kern0pt}property{\isacharbrackleft}{\kern0pt}OF\ asm{\isacharbrackright}{\kern0pt}\ supermartingale{\isachardot}{\kern0pt}supermartingale{\isacharunderscore}{\kern0pt}property{\isacharbrackleft}{\kern0pt}OF\ assms\ asm{\isacharbrackright}{\kern0pt}\ add{\isacharunderscore}{\kern0pt}mono{\isacharbrackleft}{\kern0pt}of\ {\isacharunderscore}{\kern0pt}\ {\isachardoublequoteopen}X\ i\ {\isacharunderscore}{\kern0pt}{\isachardoublequoteclose}\ {\isacharunderscore}{\kern0pt}\ {\isachardoublequoteopen}Y\ i\ {\isacharunderscore}{\kern0pt}{\isachardoublequoteclose}{\isacharbrackright}{\kern0pt}\ \isacommand{by}\isamarkupfalse%
\ force\isanewline
\ \ \isacommand{{\isacharbraceright}{\kern0pt}}\isamarkupfalse%
\isanewline
\ \ \isacommand{thus}\isamarkupfalse%
\ {\isacharquery}{\kern0pt}thesis\ \isacommand{using}\isamarkupfalse%
\ assms\ \isacommand{by}\isamarkupfalse%
\ {\isacharparenleft}{\kern0pt}unfold{\isacharunderscore}{\kern0pt}locales{\isacharparenright}{\kern0pt}\ {\isacharparenleft}{\kern0pt}auto\ simp\ add{\isacharcolon}{\kern0pt}\ borel{\isacharunderscore}{\kern0pt}measurable{\isacharunderscore}{\kern0pt}add\ random{\isacharunderscore}{\kern0pt}variable\ adapted\ integrable\ Y{\isachardot}{\kern0pt}random{\isacharunderscore}{\kern0pt}variable\ Y{\isachardot}{\kern0pt}adapted\ supermartingale{\isachardot}{\kern0pt}integrable{\isacharparenright}{\kern0pt}\ \ \isanewline
\isacommand{qed}\isamarkupfalse%
%
\endisatagproof
{\isafoldproof}%
%
\isadelimproof
\isanewline
%
\endisadelimproof
\isanewline
\isacommand{lemma}\isamarkupfalse%
\ diff{\isacharbrackleft}{\kern0pt}intro{\isacharbrackright}{\kern0pt}{\isacharcolon}{\kern0pt}\isanewline
\ \ \isakeyword{assumes}\ {\isachardoublequoteopen}submartingale\ M\ F\ Y{\isachardoublequoteclose}\isanewline
\ \ \isakeyword{shows}\ {\isachardoublequoteopen}supermartingale\ M\ F\ {\isacharparenleft}{\kern0pt}{\isasymlambda}i\ {\isasymxi}{\isachardot}{\kern0pt}\ X\ i\ {\isasymxi}\ {\isacharminus}{\kern0pt}\ Y\ i\ {\isasymxi}{\isacharparenright}{\kern0pt}{\isachardoublequoteclose}\isanewline
%
\isadelimproof
%
\endisadelimproof
%
\isatagproof
\isacommand{proof}\isamarkupfalse%
\ {\isacharminus}{\kern0pt}\isanewline
\ \ \isacommand{interpret}\isamarkupfalse%
\ Y{\isacharcolon}{\kern0pt}\ submartingale\ M\ F\ Y\ \isacommand{by}\isamarkupfalse%
\ {\isacharparenleft}{\kern0pt}rule\ assms{\isacharparenright}{\kern0pt}\isanewline
\ \ \isacommand{{\isacharbraceleft}{\kern0pt}}\isamarkupfalse%
\isanewline
\ \ \ \ \isacommand{fix}\isamarkupfalse%
\ i\ j\ {\isacharcolon}{\kern0pt}{\isacharcolon}{\kern0pt}\ {\isacharprime}{\kern0pt}b\ \isacommand{assume}\isamarkupfalse%
\ asm{\isacharcolon}{\kern0pt}\ {\isachardoublequoteopen}i\ {\isasymle}\ j{\isachardoublequoteclose}\isanewline
\ \ \ \ \isacommand{have}\isamarkupfalse%
\ {\isachardoublequoteopen}AE\ {\isasymxi}\ in\ M{\isachardot}{\kern0pt}\ X\ i\ {\isasymxi}\ {\isacharminus}{\kern0pt}\ Y\ i\ {\isasymxi}\ {\isasymge}\ cond{\isacharunderscore}{\kern0pt}exp\ M\ {\isacharparenleft}{\kern0pt}F\ i{\isacharparenright}{\kern0pt}\ {\isacharparenleft}{\kern0pt}{\isasymlambda}x{\isachardot}{\kern0pt}\ X\ j\ x\ {\isacharminus}{\kern0pt}\ Y\ j\ x{\isacharparenright}{\kern0pt}\ {\isasymxi}{\isachardoublequoteclose}\ \isanewline
\ \ \ \ \ \ \isacommand{using}\isamarkupfalse%
\ cond{\isacharunderscore}{\kern0pt}exp{\isacharunderscore}{\kern0pt}diff{\isacharbrackleft}{\kern0pt}OF\ integrable\ submartingale{\isachardot}{\kern0pt}integrable{\isacharbrackleft}{\kern0pt}OF\ assms{\isacharbrackright}{\kern0pt}{\isacharcomma}{\kern0pt}\ of\ i\ j\ j{\isacharcomma}{\kern0pt}\ unfolded\ fun{\isacharunderscore}{\kern0pt}diff{\isacharunderscore}{\kern0pt}def{\isacharbrackright}{\kern0pt}\ \isanewline
\ \ \ \ \ \ \ \ \ \ \ \ supermartingale{\isacharunderscore}{\kern0pt}property{\isacharbrackleft}{\kern0pt}OF\ asm{\isacharbrackright}{\kern0pt}\ submartingale{\isachardot}{\kern0pt}submartingale{\isacharunderscore}{\kern0pt}property{\isacharbrackleft}{\kern0pt}OF\ assms\ asm{\isacharbrackright}{\kern0pt}\ diff{\isacharunderscore}{\kern0pt}mono{\isacharbrackleft}{\kern0pt}of\ {\isacharunderscore}{\kern0pt}\ {\isachardoublequoteopen}X\ i\ {\isacharunderscore}{\kern0pt}{\isachardoublequoteclose}\ {\isachardoublequoteopen}Y\ i\ {\isacharunderscore}{\kern0pt}{\isachardoublequoteclose}{\isacharbrackright}{\kern0pt}\ \isacommand{by}\isamarkupfalse%
\ force\isanewline
\ \ \isacommand{{\isacharbraceright}{\kern0pt}}\isamarkupfalse%
\isanewline
\ \ \isacommand{thus}\isamarkupfalse%
\ {\isacharquery}{\kern0pt}thesis\ \isacommand{using}\isamarkupfalse%
\ assms\ \isacommand{by}\isamarkupfalse%
\ {\isacharparenleft}{\kern0pt}unfold{\isacharunderscore}{\kern0pt}locales{\isacharparenright}{\kern0pt}\ {\isacharparenleft}{\kern0pt}auto\ simp\ add{\isacharcolon}{\kern0pt}\ borel{\isacharunderscore}{\kern0pt}measurable{\isacharunderscore}{\kern0pt}diff\ random{\isacharunderscore}{\kern0pt}variable\ adapted\ integrable\ Y{\isachardot}{\kern0pt}random{\isacharunderscore}{\kern0pt}variable\ Y{\isachardot}{\kern0pt}adapted\ submartingale{\isachardot}{\kern0pt}integrable{\isacharparenright}{\kern0pt}\ \ \isanewline
\isacommand{qed}\isamarkupfalse%
%
\endisatagproof
{\isafoldproof}%
%
\isadelimproof
\isanewline
%
\endisadelimproof
\isanewline
\isacommand{lemma}\isamarkupfalse%
\ scaleR{\isacharunderscore}{\kern0pt}nonneg{\isacharcolon}{\kern0pt}\ \isanewline
\ \ \isakeyword{assumes}\ {\isachardoublequoteopen}c\ {\isasymge}\ {\isadigit{0}}{\isachardoublequoteclose}\isanewline
\ \ \isakeyword{shows}\ {\isachardoublequoteopen}supermartingale\ M\ F\ {\isacharparenleft}{\kern0pt}{\isasymlambda}i\ {\isasymxi}{\isachardot}{\kern0pt}\ c\ {\isacharasterisk}{\kern0pt}\isactrlsub R\ X\ i\ {\isasymxi}{\isacharparenright}{\kern0pt}{\isachardoublequoteclose}\isanewline
%
\isadelimproof
%
\endisadelimproof
%
\isatagproof
\isacommand{proof}\isamarkupfalse%
\isanewline
\ \ \isacommand{{\isacharbraceleft}{\kern0pt}}\isamarkupfalse%
\isanewline
\ \ \ \ \isacommand{fix}\isamarkupfalse%
\ i\ j\ {\isacharcolon}{\kern0pt}{\isacharcolon}{\kern0pt}\ {\isacharprime}{\kern0pt}b\ \isacommand{assume}\isamarkupfalse%
\ asm{\isacharcolon}{\kern0pt}\ {\isachardoublequoteopen}i\ {\isasymle}\ j{\isachardoublequoteclose}\isanewline
\ \ \ \ \isacommand{show}\isamarkupfalse%
\ {\isachardoublequoteopen}AE\ {\isasymxi}\ in\ M{\isachardot}{\kern0pt}\ c\ {\isacharasterisk}{\kern0pt}\isactrlsub R\ X\ i\ {\isasymxi}\ {\isasymge}\ cond{\isacharunderscore}{\kern0pt}exp\ M\ {\isacharparenleft}{\kern0pt}F\ i{\isacharparenright}{\kern0pt}\ {\isacharparenleft}{\kern0pt}{\isasymlambda}{\isasymxi}{\isachardot}{\kern0pt}\ c\ {\isacharasterisk}{\kern0pt}\isactrlsub R\ X\ j\ {\isasymxi}{\isacharparenright}{\kern0pt}\ {\isasymxi}{\isachardoublequoteclose}\ \isanewline
\ \ \ \ \ \ \isacommand{using}\isamarkupfalse%
\ cond{\isacharunderscore}{\kern0pt}exp{\isacharunderscore}{\kern0pt}scaleR{\isacharunderscore}{\kern0pt}right{\isacharbrackleft}{\kern0pt}OF\ integrable{\isacharcomma}{\kern0pt}\ of\ i\ {\isachardoublequoteopen}c{\isachardoublequoteclose}\ j{\isacharbrackright}{\kern0pt}\ supermartingale{\isacharunderscore}{\kern0pt}property{\isacharbrackleft}{\kern0pt}OF\ asm{\isacharbrackright}{\kern0pt}\ \isanewline
\ \ \ \ \ \ \isacommand{by}\isamarkupfalse%
\ {\isacharparenleft}{\kern0pt}auto\ intro{\isacharbang}{\kern0pt}{\isacharcolon}{\kern0pt}\ scaleR{\isacharunderscore}{\kern0pt}left{\isacharunderscore}{\kern0pt}mono{\isacharbrackleft}{\kern0pt}OF\ {\isacharunderscore}{\kern0pt}\ assms{\isacharbrackright}{\kern0pt}{\isacharparenright}{\kern0pt}\isanewline
\ \ \isacommand{{\isacharbraceright}{\kern0pt}}\isamarkupfalse%
\isanewline
\isacommand{qed}\isamarkupfalse%
\ {\isacharparenleft}{\kern0pt}auto\ simp\ add{\isacharcolon}{\kern0pt}\ borel{\isacharunderscore}{\kern0pt}measurable{\isacharunderscore}{\kern0pt}integrable\ borel{\isacharunderscore}{\kern0pt}measurable{\isacharunderscore}{\kern0pt}scaleR\ integrable\ random{\isacharunderscore}{\kern0pt}variable\ adapted\ borel{\isacharunderscore}{\kern0pt}measurable{\isacharunderscore}{\kern0pt}const{\isacharunderscore}{\kern0pt}scaleR{\isacharparenright}{\kern0pt}%
\endisatagproof
{\isafoldproof}%
%
\isadelimproof
\isanewline
%
\endisadelimproof
\isanewline
\isacommand{lemma}\isamarkupfalse%
\ scaleR{\isacharunderscore}{\kern0pt}nonpos{\isacharcolon}{\kern0pt}\ \isanewline
\ \ \isakeyword{assumes}\ {\isachardoublequoteopen}c\ {\isasymle}\ {\isadigit{0}}{\isachardoublequoteclose}\isanewline
\ \ \isakeyword{shows}\ {\isachardoublequoteopen}submartingale\ M\ F\ {\isacharparenleft}{\kern0pt}{\isasymlambda}i\ {\isasymxi}{\isachardot}{\kern0pt}\ c\ {\isacharasterisk}{\kern0pt}\isactrlsub R\ X\ i\ {\isasymxi}{\isacharparenright}{\kern0pt}{\isachardoublequoteclose}\isanewline
%
\isadelimproof
%
\endisadelimproof
%
\isatagproof
\isacommand{proof}\isamarkupfalse%
\isanewline
\ \ \isacommand{{\isacharbraceleft}{\kern0pt}}\isamarkupfalse%
\isanewline
\ \ \ \ \isacommand{fix}\isamarkupfalse%
\ i\ j\ {\isacharcolon}{\kern0pt}{\isacharcolon}{\kern0pt}\ {\isacharprime}{\kern0pt}b\ \isacommand{assume}\isamarkupfalse%
\ asm{\isacharcolon}{\kern0pt}\ {\isachardoublequoteopen}i\ {\isasymle}\ j{\isachardoublequoteclose}\isanewline
\ \ \ \ \isacommand{show}\isamarkupfalse%
\ {\isachardoublequoteopen}AE\ {\isasymxi}\ in\ M{\isachardot}{\kern0pt}\ c\ {\isacharasterisk}{\kern0pt}\isactrlsub R\ X\ i\ {\isasymxi}\ {\isasymle}\ cond{\isacharunderscore}{\kern0pt}exp\ M\ {\isacharparenleft}{\kern0pt}F\ i{\isacharparenright}{\kern0pt}\ {\isacharparenleft}{\kern0pt}{\isasymlambda}{\isasymxi}{\isachardot}{\kern0pt}\ c\ {\isacharasterisk}{\kern0pt}\isactrlsub R\ X\ j\ {\isasymxi}{\isacharparenright}{\kern0pt}\ {\isasymxi}{\isachardoublequoteclose}\ \isanewline
\ \ \ \ \ \ \isacommand{using}\isamarkupfalse%
\ cond{\isacharunderscore}{\kern0pt}exp{\isacharunderscore}{\kern0pt}scaleR{\isacharunderscore}{\kern0pt}right{\isacharbrackleft}{\kern0pt}OF\ integrable{\isacharcomma}{\kern0pt}\ of\ i\ {\isachardoublequoteopen}c{\isachardoublequoteclose}\ j{\isacharbrackright}{\kern0pt}\ supermartingale{\isacharunderscore}{\kern0pt}property{\isacharbrackleft}{\kern0pt}OF\ asm{\isacharbrackright}{\kern0pt}\ \isanewline
\ \ \ \ \ \ \isacommand{by}\isamarkupfalse%
\ {\isacharparenleft}{\kern0pt}auto\ intro{\isacharbang}{\kern0pt}{\isacharcolon}{\kern0pt}\ scaleR{\isacharunderscore}{\kern0pt}left{\isacharunderscore}{\kern0pt}mono{\isacharunderscore}{\kern0pt}neg{\isacharbrackleft}{\kern0pt}OF\ {\isacharunderscore}{\kern0pt}\ assms{\isacharbrackright}{\kern0pt}{\isacharparenright}{\kern0pt}\isanewline
\ \ \isacommand{{\isacharbraceright}{\kern0pt}}\isamarkupfalse%
\isanewline
\isacommand{qed}\isamarkupfalse%
\ {\isacharparenleft}{\kern0pt}auto\ simp\ add{\isacharcolon}{\kern0pt}\ borel{\isacharunderscore}{\kern0pt}measurable{\isacharunderscore}{\kern0pt}integrable\ borel{\isacharunderscore}{\kern0pt}measurable{\isacharunderscore}{\kern0pt}scaleR\ integrable\ random{\isacharunderscore}{\kern0pt}variable\ adapted\ borel{\isacharunderscore}{\kern0pt}measurable{\isacharunderscore}{\kern0pt}const{\isacharunderscore}{\kern0pt}scaleR{\isacharparenright}{\kern0pt}%
\endisatagproof
{\isafoldproof}%
%
\isadelimproof
\isanewline
%
\endisadelimproof
\isanewline
\isacommand{lemma}\isamarkupfalse%
\ uminus{\isacharbrackleft}{\kern0pt}intro{\isacharbrackright}{\kern0pt}{\isacharcolon}{\kern0pt}\isanewline
\ \ \isakeyword{shows}\ {\isachardoublequoteopen}submartingale\ M\ F\ {\isacharparenleft}{\kern0pt}{\isacharminus}{\kern0pt}\ X{\isacharparenright}{\kern0pt}{\isachardoublequoteclose}\isanewline
%
\isadelimproof
\ \ %
\endisadelimproof
%
\isatagproof
\isacommand{unfolding}\isamarkupfalse%
\ fun{\isacharunderscore}{\kern0pt}Compl{\isacharunderscore}{\kern0pt}def\ \isacommand{using}\isamarkupfalse%
\ scaleR{\isacharunderscore}{\kern0pt}nonpos{\isacharbrackleft}{\kern0pt}of\ {\isachardoublequoteopen}{\isacharminus}{\kern0pt}{\isadigit{1}}{\isachardoublequoteclose}{\isacharbrackright}{\kern0pt}\ \isacommand{by}\isamarkupfalse%
\ simp%
\endisatagproof
{\isafoldproof}%
%
\isadelimproof
\isanewline
%
\endisadelimproof
\isanewline
\isacommand{lemma}\isamarkupfalse%
\ min{\isacharcolon}{\kern0pt}\isanewline
\ \ \isakeyword{assumes}\ {\isachardoublequoteopen}supermartingale\ M\ F\ Y{\isachardoublequoteclose}\isanewline
\ \ \isakeyword{shows}\ {\isachardoublequoteopen}supermartingale\ M\ F\ {\isacharparenleft}{\kern0pt}{\isasymlambda}i\ {\isasymxi}{\isachardot}{\kern0pt}\ min\ {\isacharparenleft}{\kern0pt}X\ i\ {\isasymxi}{\isacharparenright}{\kern0pt}\ {\isacharparenleft}{\kern0pt}Y\ i\ {\isasymxi}{\isacharparenright}{\kern0pt}{\isacharparenright}{\kern0pt}{\isachardoublequoteclose}\isanewline
%
\isadelimproof
%
\endisadelimproof
%
\isatagproof
\isacommand{proof}\isamarkupfalse%
\ {\isacharparenleft}{\kern0pt}unfold{\isacharunderscore}{\kern0pt}locales{\isacharparenright}{\kern0pt}\isanewline
\ \ \isacommand{interpret}\isamarkupfalse%
\ Y{\isacharcolon}{\kern0pt}\ supermartingale\ M\ F\ Y\ \isacommand{by}\isamarkupfalse%
\ {\isacharparenleft}{\kern0pt}rule\ assms{\isacharparenright}{\kern0pt}\isanewline
\ \ \isacommand{{\isacharbraceleft}{\kern0pt}}\isamarkupfalse%
\isanewline
\ \ \ \ \isacommand{fix}\isamarkupfalse%
\ i\ j\ {\isacharcolon}{\kern0pt}{\isacharcolon}{\kern0pt}\ {\isacharprime}{\kern0pt}b\ \isacommand{assume}\isamarkupfalse%
\ asm{\isacharcolon}{\kern0pt}\ {\isachardoublequoteopen}i\ {\isasymle}\ j{\isachardoublequoteclose}\isanewline
\ \ \ \ \isacommand{have}\isamarkupfalse%
\ {\isachardoublequoteopen}AE\ {\isasymxi}\ in\ M{\isachardot}{\kern0pt}\ min\ {\isacharparenleft}{\kern0pt}X\ i\ {\isasymxi}{\isacharparenright}{\kern0pt}\ {\isacharparenleft}{\kern0pt}Y\ i\ {\isasymxi}{\isacharparenright}{\kern0pt}\ {\isasymge}\ min\ {\isacharparenleft}{\kern0pt}cond{\isacharunderscore}{\kern0pt}exp\ M\ {\isacharparenleft}{\kern0pt}F\ i{\isacharparenright}{\kern0pt}\ {\isacharparenleft}{\kern0pt}X\ j{\isacharparenright}{\kern0pt}\ {\isasymxi}{\isacharparenright}{\kern0pt}\ {\isacharparenleft}{\kern0pt}cond{\isacharunderscore}{\kern0pt}exp\ M\ {\isacharparenleft}{\kern0pt}F\ i{\isacharparenright}{\kern0pt}\ {\isacharparenleft}{\kern0pt}Y\ j{\isacharparenright}{\kern0pt}\ {\isasymxi}{\isacharparenright}{\kern0pt}{\isachardoublequoteclose}\ \isacommand{using}\isamarkupfalse%
\ supermartingale{\isacharunderscore}{\kern0pt}property\ Y{\isachardot}{\kern0pt}supermartingale{\isacharunderscore}{\kern0pt}property\ asm\ \isacommand{unfolding}\isamarkupfalse%
\ min{\isacharunderscore}{\kern0pt}def\ \isacommand{by}\isamarkupfalse%
\ fastforce\isanewline
\ \ \ \ \isacommand{thus}\isamarkupfalse%
\ {\isachardoublequoteopen}AE\ {\isasymxi}\ in\ M{\isachardot}{\kern0pt}\ min\ {\isacharparenleft}{\kern0pt}X\ i\ {\isasymxi}{\isacharparenright}{\kern0pt}\ {\isacharparenleft}{\kern0pt}Y\ i\ {\isasymxi}{\isacharparenright}{\kern0pt}\ {\isasymge}\ cond{\isacharunderscore}{\kern0pt}exp\ M\ {\isacharparenleft}{\kern0pt}F\ i{\isacharparenright}{\kern0pt}\ {\isacharparenleft}{\kern0pt}{\isasymlambda}{\isasymxi}{\isachardot}{\kern0pt}\ min\ {\isacharparenleft}{\kern0pt}X\ j\ {\isasymxi}{\isacharparenright}{\kern0pt}\ {\isacharparenleft}{\kern0pt}Y\ j\ {\isasymxi}{\isacharparenright}{\kern0pt}{\isacharparenright}{\kern0pt}\ {\isasymxi}{\isachardoublequoteclose}\ \isacommand{using}\isamarkupfalse%
\ cond{\isacharunderscore}{\kern0pt}exp{\isacharunderscore}{\kern0pt}min{\isacharbrackleft}{\kern0pt}OF\ integrable\ Y{\isachardot}{\kern0pt}integrable{\isacharcomma}{\kern0pt}\ of\ i\ j\ j{\isacharbrackright}{\kern0pt}\ order{\isachardot}{\kern0pt}trans\ \isacommand{by}\isamarkupfalse%
\ fast\isanewline
\ \ \isacommand{{\isacharbraceright}{\kern0pt}}\isamarkupfalse%
\isanewline
\ \ \isacommand{show}\isamarkupfalse%
\ {\isachardoublequoteopen}{\isasymAnd}i{\isachardot}{\kern0pt}\ {\isacharparenleft}{\kern0pt}{\isasymlambda}{\isasymxi}{\isachardot}{\kern0pt}\ min\ {\isacharparenleft}{\kern0pt}X\ i\ {\isasymxi}{\isacharparenright}{\kern0pt}\ {\isacharparenleft}{\kern0pt}Y\ i\ {\isasymxi}{\isacharparenright}{\kern0pt}{\isacharparenright}{\kern0pt}\ {\isasymin}\ borel{\isacharunderscore}{\kern0pt}measurable\ M{\isachardoublequoteclose}\ {\isachardoublequoteopen}{\isasymAnd}i{\isachardot}{\kern0pt}\ {\isacharparenleft}{\kern0pt}{\isasymlambda}{\isasymxi}{\isachardot}{\kern0pt}\ min\ {\isacharparenleft}{\kern0pt}X\ i\ {\isasymxi}{\isacharparenright}{\kern0pt}\ {\isacharparenleft}{\kern0pt}Y\ i\ {\isasymxi}{\isacharparenright}{\kern0pt}{\isacharparenright}{\kern0pt}\ {\isasymin}\ borel{\isacharunderscore}{\kern0pt}measurable\ {\isacharparenleft}{\kern0pt}F\ i{\isacharparenright}{\kern0pt}{\isachardoublequoteclose}\ {\isachardoublequoteopen}{\isasymAnd}i{\isachardot}{\kern0pt}\ integrable\ M\ {\isacharparenleft}{\kern0pt}{\isasymlambda}{\isasymxi}{\isachardot}{\kern0pt}\ min\ {\isacharparenleft}{\kern0pt}X\ i\ {\isasymxi}{\isacharparenright}{\kern0pt}\ {\isacharparenleft}{\kern0pt}Y\ i\ {\isasymxi}{\isacharparenright}{\kern0pt}{\isacharparenright}{\kern0pt}{\isachardoublequoteclose}\ \isacommand{by}\isamarkupfalse%
\ {\isacharparenleft}{\kern0pt}force\ intro{\isacharcolon}{\kern0pt}\ Y{\isachardot}{\kern0pt}integrable\ integrable\ assms{\isacharparenright}{\kern0pt}{\isacharplus}{\kern0pt}\isanewline
\isacommand{qed}\isamarkupfalse%
%
\endisatagproof
{\isafoldproof}%
%
\isadelimproof
\isanewline
%
\endisadelimproof
\isanewline
\isacommand{lemma}\isamarkupfalse%
\ min{\isacharunderscore}{\kern0pt}{\isadigit{0}}{\isacharcolon}{\kern0pt}\isanewline
\ \ \isakeyword{shows}\ {\isachardoublequoteopen}supermartingale\ M\ F\ {\isacharparenleft}{\kern0pt}{\isasymlambda}i\ {\isasymxi}{\isachardot}{\kern0pt}\ min\ {\isadigit{0}}\ {\isacharparenleft}{\kern0pt}X\ i\ {\isasymxi}{\isacharparenright}{\kern0pt}{\isacharparenright}{\kern0pt}{\isachardoublequoteclose}\isanewline
%
\isadelimproof
%
\endisadelimproof
%
\isatagproof
\isacommand{proof}\isamarkupfalse%
\ {\isacharminus}{\kern0pt}\isanewline
\ \ \isacommand{interpret}\isamarkupfalse%
\ zero{\isacharcolon}{\kern0pt}\ supermartingale\ M\ F\ {\isachardoublequoteopen}{\isasymlambda}{\isacharunderscore}{\kern0pt}\ {\isacharunderscore}{\kern0pt}{\isachardot}{\kern0pt}\ {\isadigit{0}}{\isachardoublequoteclose}\ \isacommand{by}\isamarkupfalse%
\ {\isacharparenleft}{\kern0pt}intro\ martingale{\isacharunderscore}{\kern0pt}order{\isachardot}{\kern0pt}is{\isacharunderscore}{\kern0pt}supermartingale{\isacharcomma}{\kern0pt}\ unfold{\isacharunderscore}{\kern0pt}locales{\isacharcomma}{\kern0pt}\ auto{\isacharparenright}{\kern0pt}\isanewline
\ \ \isacommand{show}\isamarkupfalse%
\ {\isacharquery}{\kern0pt}thesis\ \isacommand{by}\isamarkupfalse%
\ {\isacharparenleft}{\kern0pt}intro\ zero{\isachardot}{\kern0pt}min\ supermartingale{\isacharunderscore}{\kern0pt}axioms{\isacharparenright}{\kern0pt}\isanewline
\isacommand{qed}\isamarkupfalse%
%
\endisatagproof
{\isafoldproof}%
%
\isadelimproof
\isanewline
%
\endisadelimproof
\isanewline
\isacommand{end}\isamarkupfalse%
\isanewline
\isanewline
\isacommand{lemma}\isamarkupfalse%
\ {\isacharparenleft}{\kern0pt}\isakeyword{in}\ supermartingale{\isacharunderscore}{\kern0pt}lattice{\isacharparenright}{\kern0pt}\ inf{\isacharcolon}{\kern0pt}\isanewline
\ \ \isakeyword{assumes}\ {\isachardoublequoteopen}supermartingale{\isacharunderscore}{\kern0pt}lattice\ M\ F\ Y{\isachardoublequoteclose}\isanewline
\ \ \isakeyword{shows}\ {\isachardoublequoteopen}supermartingale{\isacharunderscore}{\kern0pt}lattice\ M\ F\ {\isacharparenleft}{\kern0pt}{\isasymlambda}i\ {\isasymxi}{\isachardot}{\kern0pt}\ inf\ {\isacharparenleft}{\kern0pt}X\ i\ {\isasymxi}{\isacharparenright}{\kern0pt}\ {\isacharparenleft}{\kern0pt}Y\ i\ {\isasymxi}{\isacharparenright}{\kern0pt}{\isacharparenright}{\kern0pt}{\isachardoublequoteclose}\isanewline
%
\isadelimproof
\ \ %
\endisadelimproof
%
\isatagproof
\isacommand{using}\isamarkupfalse%
\ supermartingale{\isacharunderscore}{\kern0pt}lattice{\isachardot}{\kern0pt}intro\ supermartingale{\isachardot}{\kern0pt}min{\isacharbrackleft}{\kern0pt}OF\ supermartingale{\isacharunderscore}{\kern0pt}axioms\ assms{\isacharbrackleft}{\kern0pt}THEN\ supermartingale{\isacharunderscore}{\kern0pt}lattice{\isachardot}{\kern0pt}axioms{\isacharbrackright}{\kern0pt}{\isacharbrackright}{\kern0pt}\ \isacommand{unfolding}\isamarkupfalse%
\ inf{\isacharunderscore}{\kern0pt}min{\isacharbrackleft}{\kern0pt}symmetric{\isacharbrackright}{\kern0pt}\ \isacommand{{\isachardot}{\kern0pt}}\isamarkupfalse%
%
\endisatagproof
{\isafoldproof}%
%
\isadelimproof
\isanewline
%
\endisadelimproof
\isanewline
\isacommand{lemma}\isamarkupfalse%
\ {\isacharparenleft}{\kern0pt}\isakeyword{in}\ adapted{\isacharunderscore}{\kern0pt}process{\isacharunderscore}{\kern0pt}order{\isacharparenright}{\kern0pt}\ supermartingale{\isacharunderscore}{\kern0pt}of{\isacharunderscore}{\kern0pt}cond{\isacharunderscore}{\kern0pt}exp{\isacharunderscore}{\kern0pt}diff{\isacharunderscore}{\kern0pt}nonneg{\isacharcolon}{\kern0pt}\ \isanewline
\ \ \isakeyword{assumes}\ integrable{\isacharcolon}{\kern0pt}\ {\isachardoublequoteopen}{\isasymAnd}i{\isachardot}{\kern0pt}\ integrable\ M\ {\isacharparenleft}{\kern0pt}X\ i{\isacharparenright}{\kern0pt}{\isachardoublequoteclose}\ \isanewline
\ \ \ \ \ \ \isakeyword{and}\ diff{\isacharunderscore}{\kern0pt}nonneg{\isacharcolon}{\kern0pt}\ {\isachardoublequoteopen}{\isasymAnd}i\ j{\isachardot}{\kern0pt}\ i\ {\isasymle}\ j\ {\isasymLongrightarrow}\ AE\ x\ in\ M{\isachardot}{\kern0pt}\ {\isadigit{0}}\ {\isasymle}\ cond{\isacharunderscore}{\kern0pt}exp\ M\ {\isacharparenleft}{\kern0pt}F\ i{\isacharparenright}{\kern0pt}\ {\isacharparenleft}{\kern0pt}{\isasymlambda}{\isasymxi}{\isachardot}{\kern0pt}\ X\ i\ {\isasymxi}\ {\isacharminus}{\kern0pt}\ X\ j\ {\isasymxi}{\isacharparenright}{\kern0pt}\ x{\isachardoublequoteclose}\isanewline
\ \ \ \ \isakeyword{shows}\ {\isachardoublequoteopen}supermartingale\ M\ F\ X{\isachardoublequoteclose}\isanewline
%
\isadelimproof
%
\endisadelimproof
%
\isatagproof
\isacommand{proof}\isamarkupfalse%
\ \isanewline
\ \ \isacommand{{\isacharbraceleft}{\kern0pt}}\isamarkupfalse%
\isanewline
\ \ \ \ \isacommand{fix}\isamarkupfalse%
\ i\ j\ {\isacharcolon}{\kern0pt}{\isacharcolon}{\kern0pt}\ {\isacharprime}{\kern0pt}t\ \isacommand{assume}\isamarkupfalse%
\ asm{\isacharcolon}{\kern0pt}\ {\isachardoublequoteopen}i\ {\isasymle}\ j{\isachardoublequoteclose}\isanewline
\ \ \ \ \isacommand{show}\isamarkupfalse%
\ {\isachardoublequoteopen}AE\ {\isasymxi}\ in\ M{\isachardot}{\kern0pt}\ X\ i\ {\isasymxi}\ {\isasymge}\ cond{\isacharunderscore}{\kern0pt}exp\ M\ {\isacharparenleft}{\kern0pt}F\ i{\isacharparenright}{\kern0pt}\ {\isacharparenleft}{\kern0pt}X\ j{\isacharparenright}{\kern0pt}\ {\isasymxi}{\isachardoublequoteclose}\ \isanewline
\ \ \ \ \ \ \isacommand{using}\isamarkupfalse%
\ diff{\isacharunderscore}{\kern0pt}nonneg{\isacharbrackleft}{\kern0pt}OF\ asm{\isacharbrackright}{\kern0pt}\ cond{\isacharunderscore}{\kern0pt}exp{\isacharunderscore}{\kern0pt}diff{\isacharbrackleft}{\kern0pt}OF\ integrable{\isacharparenleft}{\kern0pt}{\isadigit{1}}{\isacharcomma}{\kern0pt}{\isadigit{1}}{\isacharparenright}{\kern0pt}{\isacharcomma}{\kern0pt}\ of\ i\ i\ j{\isacharbrackright}{\kern0pt}\ cond{\isacharunderscore}{\kern0pt}exp{\isacharunderscore}{\kern0pt}F{\isacharunderscore}{\kern0pt}meas{\isacharbrackleft}{\kern0pt}OF\ integrable\ adapted{\isacharcomma}{\kern0pt}\ of\ i{\isacharbrackright}{\kern0pt}\ \isacommand{by}\isamarkupfalse%
\ fastforce\isanewline
\ \ \isacommand{{\isacharbraceright}{\kern0pt}}\isamarkupfalse%
\isanewline
\isacommand{qed}\isamarkupfalse%
\ {\isacharparenleft}{\kern0pt}intro\ integrable{\isacharparenright}{\kern0pt}%
\endisatagproof
{\isafoldproof}%
%
\isadelimproof
\isanewline
%
\endisadelimproof
\isanewline
\isacommand{lemma}\isamarkupfalse%
\ {\isacharparenleft}{\kern0pt}\isakeyword{in}\ adapted{\isacharunderscore}{\kern0pt}process{\isacharunderscore}{\kern0pt}order{\isacharparenright}{\kern0pt}\ supermartingale{\isacharunderscore}{\kern0pt}of{\isacharunderscore}{\kern0pt}set{\isacharunderscore}{\kern0pt}integral{\isacharunderscore}{\kern0pt}ge{\isacharcolon}{\kern0pt}\isanewline
\ \ \isakeyword{assumes}\ integrable{\isacharcolon}{\kern0pt}\ {\isachardoublequoteopen}{\isasymAnd}i{\isachardot}{\kern0pt}\ integrable\ M\ {\isacharparenleft}{\kern0pt}X\ i{\isacharparenright}{\kern0pt}{\isachardoublequoteclose}\ \isanewline
\ \ \ \ \ \ \isakeyword{and}\ {\isachardoublequoteopen}{\isasymAnd}A\ i\ j{\isachardot}{\kern0pt}\ i\ {\isasymle}\ j\ {\isasymLongrightarrow}\ A\ {\isasymin}\ F\ i\ {\isasymLongrightarrow}\ set{\isacharunderscore}{\kern0pt}lebesgue{\isacharunderscore}{\kern0pt}integral\ M\ A\ {\isacharparenleft}{\kern0pt}X\ j{\isacharparenright}{\kern0pt}\ {\isasymle}\ set{\isacharunderscore}{\kern0pt}lebesgue{\isacharunderscore}{\kern0pt}integral\ M\ A\ {\isacharparenleft}{\kern0pt}X\ i{\isacharparenright}{\kern0pt}{\isachardoublequoteclose}\ \isanewline
\ \ \ \ \isakeyword{shows}\ {\isachardoublequoteopen}supermartingale\ M\ F\ X{\isachardoublequoteclose}\isanewline
%
\isadelimproof
%
\endisadelimproof
%
\isatagproof
\isacommand{proof}\isamarkupfalse%
\ {\isacharminus}{\kern0pt}\isanewline
\ \ \isacommand{interpret}\isamarkupfalse%
\ uminus{\isacharunderscore}{\kern0pt}X{\isacharcolon}{\kern0pt}\ adapted{\isacharunderscore}{\kern0pt}process{\isacharunderscore}{\kern0pt}order\ M\ F\ {\isachardoublequoteopen}{\isacharminus}{\kern0pt}X{\isachardoublequoteclose}\ \isacommand{by}\isamarkupfalse%
\ {\isacharparenleft}{\kern0pt}intro\ adapted{\isacharunderscore}{\kern0pt}process{\isacharunderscore}{\kern0pt}order{\isachardot}{\kern0pt}intro\ uminus{\isacharparenright}{\kern0pt}\isanewline
\ \ \isacommand{note}\isamarkupfalse%
\ {\isacharasterisk}{\kern0pt}\ {\isacharequal}{\kern0pt}\ set{\isacharunderscore}{\kern0pt}integral{\isacharunderscore}{\kern0pt}uminus{\isacharbrackleft}{\kern0pt}unfolded\ set{\isacharunderscore}{\kern0pt}integrable{\isacharunderscore}{\kern0pt}def{\isacharcomma}{\kern0pt}\ OF\ integrable{\isacharunderscore}{\kern0pt}mult{\isacharunderscore}{\kern0pt}indicator{\isacharbrackleft}{\kern0pt}OF\ {\isacharunderscore}{\kern0pt}\ integrable{\isacharbrackright}{\kern0pt}{\isacharbrackright}{\kern0pt}\isanewline
\ \ \isacommand{have}\isamarkupfalse%
\ {\isachardoublequoteopen}supermartingale\ M\ F\ {\isacharparenleft}{\kern0pt}{\isacharminus}{\kern0pt}{\isacharparenleft}{\kern0pt}{\isacharminus}{\kern0pt}\ X{\isacharparenright}{\kern0pt}{\isacharparenright}{\kern0pt}{\isachardoublequoteclose}\ \isacommand{using}\isamarkupfalse%
\ ord{\isacharunderscore}{\kern0pt}eq{\isacharunderscore}{\kern0pt}le{\isacharunderscore}{\kern0pt}trans{\isacharbrackleft}{\kern0pt}OF\ {\isacharasterisk}{\kern0pt}\ ord{\isacharunderscore}{\kern0pt}le{\isacharunderscore}{\kern0pt}eq{\isacharunderscore}{\kern0pt}trans{\isacharbrackleft}{\kern0pt}OF\ le{\isacharunderscore}{\kern0pt}imp{\isacharunderscore}{\kern0pt}neg{\isacharunderscore}{\kern0pt}le{\isacharbrackleft}{\kern0pt}OF\ assms{\isacharparenleft}{\kern0pt}{\isadigit{2}}{\isacharparenright}{\kern0pt}{\isacharbrackright}{\kern0pt}\ {\isacharasterisk}{\kern0pt}{\isacharbrackleft}{\kern0pt}symmetric{\isacharbrackright}{\kern0pt}{\isacharbrackright}{\kern0pt}{\isacharbrackright}{\kern0pt}\ subalg\isanewline
\ \ \ \ \isacommand{by}\isamarkupfalse%
\ {\isacharparenleft}{\kern0pt}intro\ submartingale{\isachardot}{\kern0pt}uminus\ uminus{\isacharunderscore}{\kern0pt}X{\isachardot}{\kern0pt}submartingale{\isacharunderscore}{\kern0pt}of{\isacharunderscore}{\kern0pt}set{\isacharunderscore}{\kern0pt}integral{\isacharunderscore}{\kern0pt}le{\isacharparenright}{\kern0pt}\ {\isacharparenleft}{\kern0pt}auto\ simp\ add{\isacharcolon}{\kern0pt}\ subalgebra{\isacharunderscore}{\kern0pt}def\ integrable\ fun{\isacharunderscore}{\kern0pt}Compl{\isacharunderscore}{\kern0pt}def{\isacharcomma}{\kern0pt}\ blast{\isacharparenright}{\kern0pt}\isanewline
\ \ \isacommand{thus}\isamarkupfalse%
\ {\isacharquery}{\kern0pt}thesis\ \isacommand{unfolding}\isamarkupfalse%
\ fun{\isacharunderscore}{\kern0pt}Compl{\isacharunderscore}{\kern0pt}def\ \isacommand{by}\isamarkupfalse%
\ simp\isanewline
\isacommand{qed}\isamarkupfalse%
%
\endisatagproof
{\isafoldproof}%
%
\isadelimproof
%
\endisadelimproof
%
\isadelimdocument
%
\endisadelimdocument
%
\isatagdocument
%
\isamarkupsection{Discrete Time Martingales%
}
\isamarkuptrue%
%
\endisatagdocument
{\isafolddocument}%
%
\isadelimdocument
%
\endisadelimdocument
\isacommand{locale}\isamarkupfalse%
\ discrete{\isacharunderscore}{\kern0pt}time{\isacharunderscore}{\kern0pt}martingale\ {\isacharequal}{\kern0pt}\ martingale\ M\ F\ X\ \isakeyword{for}\ M\ F\ \isakeyword{and}\ X\ {\isacharcolon}{\kern0pt}{\isacharcolon}{\kern0pt}\ {\isachardoublequoteopen}nat\ {\isasymRightarrow}\ {\isacharunderscore}{\kern0pt}\ {\isasymRightarrow}\ {\isacharunderscore}{\kern0pt}{\isachardoublequoteclose}\isanewline
\isacommand{locale}\isamarkupfalse%
\ discrete{\isacharunderscore}{\kern0pt}time{\isacharunderscore}{\kern0pt}submartingale\ {\isacharequal}{\kern0pt}\ submartingale\ M\ F\ X\ \isakeyword{for}\ M\ F\ \isakeyword{and}\ X\ {\isacharcolon}{\kern0pt}{\isacharcolon}{\kern0pt}\ {\isachardoublequoteopen}nat\ {\isasymRightarrow}\ {\isacharunderscore}{\kern0pt}\ {\isasymRightarrow}\ {\isacharunderscore}{\kern0pt}{\isachardoublequoteclose}\isanewline
\isacommand{locale}\isamarkupfalse%
\ discrete{\isacharunderscore}{\kern0pt}time{\isacharunderscore}{\kern0pt}supermartingale\ {\isacharequal}{\kern0pt}\ supermartingale\ M\ F\ X\ \isakeyword{for}\ M\ F\ \isakeyword{and}\ X\ {\isacharcolon}{\kern0pt}{\isacharcolon}{\kern0pt}\ {\isachardoublequoteopen}nat\ {\isasymRightarrow}\ {\isacharunderscore}{\kern0pt}\ {\isasymRightarrow}\ {\isacharunderscore}{\kern0pt}{\isachardoublequoteclose}\isanewline
\isanewline
\isacommand{sublocale}\isamarkupfalse%
\ discrete{\isacharunderscore}{\kern0pt}time{\isacharunderscore}{\kern0pt}martingale\ {\isasymsubseteq}\ discrete{\isacharunderscore}{\kern0pt}time{\isacharunderscore}{\kern0pt}adapted{\isacharunderscore}{\kern0pt}process%
\isadelimproof
\ %
\endisadelimproof
%
\isatagproof
\isacommand{by}\isamarkupfalse%
\ {\isacharparenleft}{\kern0pt}unfold{\isacharunderscore}{\kern0pt}locales{\isacharparenright}{\kern0pt}%
\endisatagproof
{\isafoldproof}%
%
\isadelimproof
%
\endisadelimproof
\isanewline
\isacommand{sublocale}\isamarkupfalse%
\ discrete{\isacharunderscore}{\kern0pt}time{\isacharunderscore}{\kern0pt}submartingale\ {\isasymsubseteq}\ discrete{\isacharunderscore}{\kern0pt}time{\isacharunderscore}{\kern0pt}adapted{\isacharunderscore}{\kern0pt}process%
\isadelimproof
\ %
\endisadelimproof
%
\isatagproof
\isacommand{by}\isamarkupfalse%
\ {\isacharparenleft}{\kern0pt}unfold{\isacharunderscore}{\kern0pt}locales{\isacharparenright}{\kern0pt}%
\endisatagproof
{\isafoldproof}%
%
\isadelimproof
%
\endisadelimproof
\isanewline
\isacommand{sublocale}\isamarkupfalse%
\ discrete{\isacharunderscore}{\kern0pt}time{\isacharunderscore}{\kern0pt}supermartingale\ {\isasymsubseteq}\ discrete{\isacharunderscore}{\kern0pt}time{\isacharunderscore}{\kern0pt}adapted{\isacharunderscore}{\kern0pt}process%
\isadelimproof
\ %
\endisadelimproof
%
\isatagproof
\isacommand{by}\isamarkupfalse%
\ {\isacharparenleft}{\kern0pt}unfold{\isacharunderscore}{\kern0pt}locales{\isacharparenright}{\kern0pt}%
\endisatagproof
{\isafoldproof}%
%
\isadelimproof
%
\endisadelimproof
%
\isadelimdocument
%
\endisadelimdocument
%
\isatagdocument
%
\isamarkupsection{Discrete Time Martingales%
}
\isamarkuptrue%
%
\endisatagdocument
{\isafolddocument}%
%
\isadelimdocument
%
\endisadelimdocument
\isacommand{lemma}\isamarkupfalse%
\ {\isacharparenleft}{\kern0pt}\isakeyword{in}\ discrete{\isacharunderscore}{\kern0pt}time{\isacharunderscore}{\kern0pt}martingale{\isacharparenright}{\kern0pt}\ predictable{\isacharunderscore}{\kern0pt}eq{\isacharunderscore}{\kern0pt}bot{\isacharcolon}{\kern0pt}\isanewline
\ \ \isakeyword{assumes}\ {\isachardoublequoteopen}predictable\ X{\isachardoublequoteclose}\isanewline
\ \ \isakeyword{shows}\ {\isachardoublequoteopen}AE\ {\isasymxi}\ in\ M{\isachardot}{\kern0pt}\ X\ i\ {\isasymxi}\ {\isacharequal}{\kern0pt}\ X\ {\isasymbottom}\ {\isasymxi}{\isachardoublequoteclose}\isanewline
%
\isadelimproof
%
\endisadelimproof
%
\isatagproof
\isacommand{proof}\isamarkupfalse%
\ {\isacharparenleft}{\kern0pt}induction\ i{\isacharparenright}{\kern0pt}\isanewline
\ \ \isacommand{case}\isamarkupfalse%
\ {\isadigit{0}}\isanewline
\ \ \isacommand{then}\isamarkupfalse%
\ \isacommand{show}\isamarkupfalse%
\ {\isacharquery}{\kern0pt}case\ \isacommand{by}\isamarkupfalse%
\ {\isacharparenleft}{\kern0pt}simp\ add{\isacharcolon}{\kern0pt}\ bot{\isacharunderscore}{\kern0pt}nat{\isacharunderscore}{\kern0pt}def{\isacharparenright}{\kern0pt}\isanewline
\isacommand{next}\isamarkupfalse%
\isanewline
\ \ \isacommand{case}\isamarkupfalse%
\ {\isacharparenleft}{\kern0pt}Suc\ i{\isacharparenright}{\kern0pt}\isanewline
\ \ \isacommand{thus}\isamarkupfalse%
\ {\isacharquery}{\kern0pt}case\ \isacommand{using}\isamarkupfalse%
\ predictable{\isacharunderscore}{\kern0pt}discrete{\isacharunderscore}{\kern0pt}time{\isacharunderscore}{\kern0pt}process{\isacharunderscore}{\kern0pt}measurable{\isacharbrackleft}{\kern0pt}OF\ assms{\isacharcomma}{\kern0pt}\ of\ {\isachardoublequoteopen}Suc\ i{\isachardoublequoteclose}{\isacharbrackright}{\kern0pt}\ \isanewline
\ \ \ \ \ \ \ \ \ \ \ \ \ \ \ \ \ \ \ martingale{\isacharunderscore}{\kern0pt}property{\isacharbrackleft}{\kern0pt}OF\ le{\isacharunderscore}{\kern0pt}SucI{\isacharcomma}{\kern0pt}\ of\ i{\isacharbrackright}{\kern0pt}\isanewline
\ \ \ \ \ \ \ \ \ \ \ \ \ \ \ \ \ \ \ cond{\isacharunderscore}{\kern0pt}exp{\isacharunderscore}{\kern0pt}F{\isacharunderscore}{\kern0pt}meas{\isacharbrackleft}{\kern0pt}OF\ integrable{\isacharcomma}{\kern0pt}\ of\ {\isachardoublequoteopen}Suc\ i{\isachardoublequoteclose}\ i{\isacharbrackright}{\kern0pt}\ Suc\ \isacommand{by}\isamarkupfalse%
\ fastforce\isanewline
\isacommand{qed}\isamarkupfalse%
%
\endisatagproof
{\isafoldproof}%
%
\isadelimproof
\isanewline
%
\endisadelimproof
\isanewline
\isacommand{lemma}\isamarkupfalse%
\ {\isacharparenleft}{\kern0pt}\isakeyword{in}\ discrete{\isacharunderscore}{\kern0pt}time{\isacharunderscore}{\kern0pt}adapted{\isacharunderscore}{\kern0pt}process{\isacharparenright}{\kern0pt}\ martingale{\isacharunderscore}{\kern0pt}of{\isacharunderscore}{\kern0pt}set{\isacharunderscore}{\kern0pt}integral{\isacharunderscore}{\kern0pt}eq{\isacharunderscore}{\kern0pt}Suc{\isacharcolon}{\kern0pt}\isanewline
\ \ \isakeyword{assumes}\ integrable{\isacharcolon}{\kern0pt}\ {\isachardoublequoteopen}{\isasymAnd}i{\isachardot}{\kern0pt}\ integrable\ M\ {\isacharparenleft}{\kern0pt}X\ i{\isacharparenright}{\kern0pt}{\isachardoublequoteclose}\isanewline
\ \ \ \ \ \ \isakeyword{and}\ {\isachardoublequoteopen}{\isasymAnd}A\ i{\isachardot}{\kern0pt}\ A\ {\isasymin}\ F\ i\ {\isasymLongrightarrow}\ set{\isacharunderscore}{\kern0pt}lebesgue{\isacharunderscore}{\kern0pt}integral\ M\ A\ {\isacharparenleft}{\kern0pt}X\ i{\isacharparenright}{\kern0pt}\ {\isacharequal}{\kern0pt}\ set{\isacharunderscore}{\kern0pt}lebesgue{\isacharunderscore}{\kern0pt}integral\ M\ A\ {\isacharparenleft}{\kern0pt}X\ {\isacharparenleft}{\kern0pt}Suc\ i{\isacharparenright}{\kern0pt}{\isacharparenright}{\kern0pt}{\isachardoublequoteclose}\ \isanewline
\ \ \ \ \isakeyword{shows}\ {\isachardoublequoteopen}discrete{\isacharunderscore}{\kern0pt}time{\isacharunderscore}{\kern0pt}martingale\ M\ F\ X{\isachardoublequoteclose}\isanewline
%
\isadelimproof
%
\endisadelimproof
%
\isatagproof
\isacommand{proof}\isamarkupfalse%
\ {\isacharparenleft}{\kern0pt}intro\ discrete{\isacharunderscore}{\kern0pt}time{\isacharunderscore}{\kern0pt}martingale{\isachardot}{\kern0pt}intro\ martingale{\isacharunderscore}{\kern0pt}of{\isacharunderscore}{\kern0pt}set{\isacharunderscore}{\kern0pt}integral{\isacharunderscore}{\kern0pt}eq{\isacharparenright}{\kern0pt}\isanewline
\ \ \isacommand{fix}\isamarkupfalse%
\ i\ j\ A\ \isacommand{assume}\isamarkupfalse%
\ asm{\isacharcolon}{\kern0pt}\ {\isachardoublequoteopen}i\ {\isasymle}\ j{\isachardoublequoteclose}\ {\isachardoublequoteopen}A\ {\isasymin}\ sets\ {\isacharparenleft}{\kern0pt}F\ i{\isacharparenright}{\kern0pt}{\isachardoublequoteclose}\isanewline
\ \ \isacommand{show}\isamarkupfalse%
\ {\isachardoublequoteopen}set{\isacharunderscore}{\kern0pt}lebesgue{\isacharunderscore}{\kern0pt}integral\ M\ A\ {\isacharparenleft}{\kern0pt}X\ i{\isacharparenright}{\kern0pt}\ {\isacharequal}{\kern0pt}\ set{\isacharunderscore}{\kern0pt}lebesgue{\isacharunderscore}{\kern0pt}integral\ M\ A\ {\isacharparenleft}{\kern0pt}X\ j{\isacharparenright}{\kern0pt}{\isachardoublequoteclose}\ \isacommand{using}\isamarkupfalse%
\ asm\isanewline
\ \ \isacommand{proof}\isamarkupfalse%
\ {\isacharparenleft}{\kern0pt}induction\ {\isachardoublequoteopen}j\ {\isacharminus}{\kern0pt}\ i{\isachardoublequoteclose}\ arbitrary{\isacharcolon}{\kern0pt}\ i\ j{\isacharparenright}{\kern0pt}\isanewline
\ \ \ \ \isacommand{case}\isamarkupfalse%
\ {\isadigit{0}}\isanewline
\ \ \ \ \isacommand{then}\isamarkupfalse%
\ \isacommand{show}\isamarkupfalse%
\ {\isacharquery}{\kern0pt}case\ \isacommand{using}\isamarkupfalse%
\ asm\ \isacommand{by}\isamarkupfalse%
\ simp\isanewline
\ \ \isacommand{next}\isamarkupfalse%
\isanewline
\ \ \ \ \isacommand{case}\isamarkupfalse%
\ {\isacharparenleft}{\kern0pt}Suc\ n{\isacharparenright}{\kern0pt}\isanewline
\ \ \ \ \isacommand{hence}\isamarkupfalse%
\ {\isacharasterisk}{\kern0pt}{\isacharcolon}{\kern0pt}\ {\isachardoublequoteopen}n\ {\isacharequal}{\kern0pt}\ j\ {\isacharminus}{\kern0pt}\ Suc\ i{\isachardoublequoteclose}\ \isacommand{by}\isamarkupfalse%
\ linarith\isanewline
\ \ \ \ \isacommand{have}\isamarkupfalse%
\ {\isachardoublequoteopen}Suc\ i\ {\isasymle}\ j{\isachardoublequoteclose}\ \isacommand{using}\isamarkupfalse%
\ Suc{\isacharparenleft}{\kern0pt}{\isadigit{2}}{\isacharcomma}{\kern0pt}{\isadigit{3}}{\isacharparenright}{\kern0pt}\ \isacommand{by}\isamarkupfalse%
\ linarith\isanewline
\ \ \ \ \isacommand{thus}\isamarkupfalse%
\ {\isacharquery}{\kern0pt}case\ \isacommand{using}\isamarkupfalse%
\ sets{\isacharunderscore}{\kern0pt}F{\isacharunderscore}{\kern0pt}mono{\isacharbrackleft}{\kern0pt}OF\ le{\isacharunderscore}{\kern0pt}SucI{\isacharbrackright}{\kern0pt}\ Suc{\isacharparenleft}{\kern0pt}{\isadigit{4}}{\isacharparenright}{\kern0pt}\ Suc{\isacharparenleft}{\kern0pt}{\isadigit{1}}{\isacharparenright}{\kern0pt}{\isacharbrackleft}{\kern0pt}OF\ {\isacharasterisk}{\kern0pt}{\isacharbrackright}{\kern0pt}\ \isacommand{by}\isamarkupfalse%
\ {\isacharparenleft}{\kern0pt}auto\ intro{\isacharcolon}{\kern0pt}\ assms{\isacharparenleft}{\kern0pt}{\isadigit{2}}{\isacharparenright}{\kern0pt}{\isacharbrackleft}{\kern0pt}THEN\ trans{\isacharbrackright}{\kern0pt}{\isacharparenright}{\kern0pt}\isanewline
\ \ \isacommand{qed}\isamarkupfalse%
\isanewline
\isacommand{qed}\isamarkupfalse%
\ {\isacharparenleft}{\kern0pt}simp\ add{\isacharcolon}{\kern0pt}\ integrable{\isacharparenright}{\kern0pt}%
\endisatagproof
{\isafoldproof}%
%
\isadelimproof
\isanewline
%
\endisadelimproof
\isanewline
\isacommand{lemma}\isamarkupfalse%
\ {\isacharparenleft}{\kern0pt}\isakeyword{in}\ discrete{\isacharunderscore}{\kern0pt}time{\isacharunderscore}{\kern0pt}adapted{\isacharunderscore}{\kern0pt}process{\isacharparenright}{\kern0pt}\ martingale{\isacharunderscore}{\kern0pt}nat{\isacharcolon}{\kern0pt}\isanewline
\ \ \isakeyword{assumes}\ integrable{\isacharcolon}{\kern0pt}\ {\isachardoublequoteopen}{\isasymAnd}i{\isachardot}{\kern0pt}\ integrable\ M\ {\isacharparenleft}{\kern0pt}X\ i{\isacharparenright}{\kern0pt}{\isachardoublequoteclose}\ \isanewline
\ \ \ \ \ \ \isakeyword{and}\ {\isachardoublequoteopen}{\isasymAnd}i{\isachardot}{\kern0pt}\ AE\ {\isasymxi}\ in\ M{\isachardot}{\kern0pt}\ X\ i\ {\isasymxi}\ {\isacharequal}{\kern0pt}\ cond{\isacharunderscore}{\kern0pt}exp\ M\ {\isacharparenleft}{\kern0pt}F\ i{\isacharparenright}{\kern0pt}\ {\isacharparenleft}{\kern0pt}X\ {\isacharparenleft}{\kern0pt}Suc\ i{\isacharparenright}{\kern0pt}{\isacharparenright}{\kern0pt}\ {\isasymxi}{\isachardoublequoteclose}\ \isanewline
\ \ \ \ \isakeyword{shows}\ {\isachardoublequoteopen}discrete{\isacharunderscore}{\kern0pt}time{\isacharunderscore}{\kern0pt}martingale\ M\ F\ X{\isachardoublequoteclose}\isanewline
%
\isadelimproof
%
\endisadelimproof
%
\isatagproof
\isacommand{proof}\isamarkupfalse%
\ {\isacharparenleft}{\kern0pt}unfold{\isacharunderscore}{\kern0pt}locales{\isacharparenright}{\kern0pt}\isanewline
\ \ \isacommand{fix}\isamarkupfalse%
\ i\ j\ {\isacharcolon}{\kern0pt}{\isacharcolon}{\kern0pt}\ nat\ \isacommand{assume}\isamarkupfalse%
\ asm{\isacharcolon}{\kern0pt}\ {\isachardoublequoteopen}i\ {\isasymle}\ j{\isachardoublequoteclose}\isanewline
\ \ \isacommand{show}\isamarkupfalse%
\ {\isachardoublequoteopen}AE\ {\isasymxi}\ in\ M{\isachardot}{\kern0pt}\ X\ i\ {\isasymxi}\ {\isacharequal}{\kern0pt}\ cond{\isacharunderscore}{\kern0pt}exp\ M\ {\isacharparenleft}{\kern0pt}F\ i{\isacharparenright}{\kern0pt}\ {\isacharparenleft}{\kern0pt}X\ j{\isacharparenright}{\kern0pt}\ {\isasymxi}{\isachardoublequoteclose}\ \isacommand{using}\isamarkupfalse%
\ asm\isanewline
\ \ \isacommand{proof}\isamarkupfalse%
\ {\isacharparenleft}{\kern0pt}induction\ {\isachardoublequoteopen}j\ {\isacharminus}{\kern0pt}\ i{\isachardoublequoteclose}\ arbitrary{\isacharcolon}{\kern0pt}\ i\ j{\isacharparenright}{\kern0pt}\isanewline
\ \ \ \ \isacommand{case}\isamarkupfalse%
\ {\isadigit{0}}\isanewline
\ \ \ \ \isacommand{hence}\isamarkupfalse%
\ {\isachardoublequoteopen}j\ {\isacharequal}{\kern0pt}\ i{\isachardoublequoteclose}\ \isacommand{by}\isamarkupfalse%
\ simp\isanewline
\ \ \ \ \isacommand{thus}\isamarkupfalse%
\ {\isacharquery}{\kern0pt}case\ \isacommand{using}\isamarkupfalse%
\ cond{\isacharunderscore}{\kern0pt}exp{\isacharunderscore}{\kern0pt}F{\isacharunderscore}{\kern0pt}meas{\isacharbrackleft}{\kern0pt}OF\ integrable\ adapted{\isacharcomma}{\kern0pt}\ THEN\ AE{\isacharunderscore}{\kern0pt}symmetric{\isacharbrackright}{\kern0pt}\ \isacommand{by}\isamarkupfalse%
\ presburger\isanewline
\ \ \isacommand{next}\isamarkupfalse%
\isanewline
\ \ \ \ \isacommand{case}\isamarkupfalse%
\ {\isacharparenleft}{\kern0pt}Suc\ n{\isacharparenright}{\kern0pt}\isanewline
\ \ \ \ \isacommand{have}\isamarkupfalse%
\ j{\isacharcolon}{\kern0pt}\ {\isachardoublequoteopen}j\ {\isacharequal}{\kern0pt}\ Suc\ {\isacharparenleft}{\kern0pt}n\ {\isacharplus}{\kern0pt}\ i{\isacharparenright}{\kern0pt}{\isachardoublequoteclose}\ \isacommand{using}\isamarkupfalse%
\ Suc\ \isacommand{by}\isamarkupfalse%
\ linarith\isanewline
\ \ \ \ \isacommand{have}\isamarkupfalse%
\ n{\isacharcolon}{\kern0pt}\ {\isachardoublequoteopen}n\ {\isacharequal}{\kern0pt}\ n\ {\isacharplus}{\kern0pt}\ i\ {\isacharminus}{\kern0pt}\ i{\isachardoublequoteclose}\ \isacommand{using}\isamarkupfalse%
\ Suc\ \isacommand{by}\isamarkupfalse%
\ linarith\isanewline
\ \ \ \ \isacommand{have}\isamarkupfalse%
\ {\isacharasterisk}{\kern0pt}{\isacharcolon}{\kern0pt}\ {\isachardoublequoteopen}AE\ {\isasymxi}\ in\ M{\isachardot}{\kern0pt}\ cond{\isacharunderscore}{\kern0pt}exp\ M\ {\isacharparenleft}{\kern0pt}F\ {\isacharparenleft}{\kern0pt}n\ {\isacharplus}{\kern0pt}\ i{\isacharparenright}{\kern0pt}{\isacharparenright}{\kern0pt}\ {\isacharparenleft}{\kern0pt}X\ j{\isacharparenright}{\kern0pt}\ {\isasymxi}\ {\isacharequal}{\kern0pt}\ X\ {\isacharparenleft}{\kern0pt}n\ {\isacharplus}{\kern0pt}\ i{\isacharparenright}{\kern0pt}\ {\isasymxi}{\isachardoublequoteclose}\ \isacommand{unfolding}\isamarkupfalse%
\ j\ \isacommand{using}\isamarkupfalse%
\ assms{\isacharparenleft}{\kern0pt}{\isadigit{2}}{\isacharparenright}{\kern0pt}{\isacharbrackleft}{\kern0pt}THEN\ AE{\isacharunderscore}{\kern0pt}symmetric{\isacharbrackright}{\kern0pt}\ \isacommand{by}\isamarkupfalse%
\ blast\isanewline
\ \ \ \ \isacommand{have}\isamarkupfalse%
\ {\isachardoublequoteopen}AE\ {\isasymxi}\ in\ M{\isachardot}{\kern0pt}\ cond{\isacharunderscore}{\kern0pt}exp\ M\ {\isacharparenleft}{\kern0pt}F\ i{\isacharparenright}{\kern0pt}\ {\isacharparenleft}{\kern0pt}X\ j{\isacharparenright}{\kern0pt}\ {\isasymxi}\ {\isacharequal}{\kern0pt}\ cond{\isacharunderscore}{\kern0pt}exp\ M\ {\isacharparenleft}{\kern0pt}F\ i{\isacharparenright}{\kern0pt}\ {\isacharparenleft}{\kern0pt}cond{\isacharunderscore}{\kern0pt}exp\ M\ {\isacharparenleft}{\kern0pt}F\ {\isacharparenleft}{\kern0pt}n\ {\isacharplus}{\kern0pt}\ i{\isacharparenright}{\kern0pt}{\isacharparenright}{\kern0pt}\ {\isacharparenleft}{\kern0pt}X\ j{\isacharparenright}{\kern0pt}{\isacharparenright}{\kern0pt}\ {\isasymxi}{\isachardoublequoteclose}\ \isacommand{by}\isamarkupfalse%
\ {\isacharparenleft}{\kern0pt}intro\ cond{\isacharunderscore}{\kern0pt}exp{\isacharunderscore}{\kern0pt}nested{\isacharunderscore}{\kern0pt}subalg\ integrable\ subalg{\isacharcomma}{\kern0pt}\ simp\ add{\isacharcolon}{\kern0pt}\ subalgebra{\isacharunderscore}{\kern0pt}def\ space{\isacharunderscore}{\kern0pt}F\ sets{\isacharunderscore}{\kern0pt}F{\isacharunderscore}{\kern0pt}mono{\isacharparenright}{\kern0pt}\isanewline
\ \ \ \ \isacommand{hence}\isamarkupfalse%
\ {\isachardoublequoteopen}AE\ {\isasymxi}\ in\ M{\isachardot}{\kern0pt}\ cond{\isacharunderscore}{\kern0pt}exp\ M\ {\isacharparenleft}{\kern0pt}F\ i{\isacharparenright}{\kern0pt}\ {\isacharparenleft}{\kern0pt}X\ j{\isacharparenright}{\kern0pt}\ {\isasymxi}\ {\isacharequal}{\kern0pt}\ cond{\isacharunderscore}{\kern0pt}exp\ M\ {\isacharparenleft}{\kern0pt}F\ i{\isacharparenright}{\kern0pt}\ {\isacharparenleft}{\kern0pt}X\ {\isacharparenleft}{\kern0pt}n\ {\isacharplus}{\kern0pt}\ i{\isacharparenright}{\kern0pt}{\isacharparenright}{\kern0pt}\ {\isasymxi}{\isachardoublequoteclose}\ \isacommand{using}\isamarkupfalse%
\ cond{\isacharunderscore}{\kern0pt}exp{\isacharunderscore}{\kern0pt}cong{\isacharunderscore}{\kern0pt}AE{\isacharbrackleft}{\kern0pt}OF\ integrable{\isacharunderscore}{\kern0pt}cond{\isacharunderscore}{\kern0pt}exp\ integrable\ {\isacharasterisk}{\kern0pt}{\isacharbrackright}{\kern0pt}\ \isacommand{by}\isamarkupfalse%
\ force\isanewline
\ \ \ \ \isacommand{thus}\isamarkupfalse%
\ {\isacharquery}{\kern0pt}case\ \isacommand{using}\isamarkupfalse%
\ Suc{\isacharparenleft}{\kern0pt}{\isadigit{1}}{\isacharparenright}{\kern0pt}{\isacharbrackleft}{\kern0pt}OF\ n{\isacharbrackright}{\kern0pt}\ \isacommand{by}\isamarkupfalse%
\ fastforce\isanewline
\ \ \isacommand{qed}\isamarkupfalse%
\isanewline
\isacommand{qed}\isamarkupfalse%
\ {\isacharparenleft}{\kern0pt}simp\ add{\isacharcolon}{\kern0pt}\ integrable{\isacharparenright}{\kern0pt}%
\endisatagproof
{\isafoldproof}%
%
\isadelimproof
\isanewline
%
\endisadelimproof
\isanewline
\isacommand{lemma}\isamarkupfalse%
\ {\isacharparenleft}{\kern0pt}\isakeyword{in}\ discrete{\isacharunderscore}{\kern0pt}time{\isacharunderscore}{\kern0pt}adapted{\isacharunderscore}{\kern0pt}process{\isacharparenright}{\kern0pt}\ martingale{\isacharunderscore}{\kern0pt}of{\isacharunderscore}{\kern0pt}cond{\isacharunderscore}{\kern0pt}exp{\isacharunderscore}{\kern0pt}diff{\isacharunderscore}{\kern0pt}Suc{\isacharunderscore}{\kern0pt}eq{\isacharunderscore}{\kern0pt}{\isadigit{0}}{\isacharcolon}{\kern0pt}\isanewline
\ \ \isakeyword{assumes}\ integrable{\isacharcolon}{\kern0pt}\ {\isachardoublequoteopen}{\isasymAnd}i{\isachardot}{\kern0pt}\ integrable\ M\ {\isacharparenleft}{\kern0pt}X\ i{\isacharparenright}{\kern0pt}{\isachardoublequoteclose}\ \isanewline
\ \ \ \ \ \ \isakeyword{and}\ {\isachardoublequoteopen}{\isasymAnd}i{\isachardot}{\kern0pt}\ AE\ {\isasymxi}\ in\ M{\isachardot}{\kern0pt}\ {\isadigit{0}}\ {\isacharequal}{\kern0pt}\ cond{\isacharunderscore}{\kern0pt}exp\ M\ {\isacharparenleft}{\kern0pt}F\ i{\isacharparenright}{\kern0pt}\ {\isacharparenleft}{\kern0pt}{\isasymlambda}{\isasymxi}{\isachardot}{\kern0pt}\ X\ {\isacharparenleft}{\kern0pt}Suc\ i{\isacharparenright}{\kern0pt}\ {\isasymxi}\ {\isacharminus}{\kern0pt}\ X\ i\ {\isasymxi}{\isacharparenright}{\kern0pt}\ {\isasymxi}{\isachardoublequoteclose}\ \isanewline
\ \ \ \ \isakeyword{shows}\ {\isachardoublequoteopen}discrete{\isacharunderscore}{\kern0pt}time{\isacharunderscore}{\kern0pt}martingale\ M\ F\ X{\isachardoublequoteclose}\isanewline
%
\isadelimproof
%
\endisadelimproof
%
\isatagproof
\isacommand{proof}\isamarkupfalse%
\ {\isacharparenleft}{\kern0pt}intro\ martingale{\isacharunderscore}{\kern0pt}nat\ integrable{\isacharparenright}{\kern0pt}\ \isanewline
\ \ \isacommand{fix}\isamarkupfalse%
\ i\ \isanewline
\ \ \isacommand{show}\isamarkupfalse%
\ {\isachardoublequoteopen}AE\ {\isasymxi}\ in\ M{\isachardot}{\kern0pt}\ X\ i\ {\isasymxi}\ {\isacharequal}{\kern0pt}\ cond{\isacharunderscore}{\kern0pt}exp\ M\ {\isacharparenleft}{\kern0pt}F\ i{\isacharparenright}{\kern0pt}\ {\isacharparenleft}{\kern0pt}X\ {\isacharparenleft}{\kern0pt}Suc\ i{\isacharparenright}{\kern0pt}{\isacharparenright}{\kern0pt}\ {\isasymxi}{\isachardoublequoteclose}\ \isacommand{using}\isamarkupfalse%
\ cond{\isacharunderscore}{\kern0pt}exp{\isacharunderscore}{\kern0pt}diff{\isacharbrackleft}{\kern0pt}OF\ integrable{\isacharparenleft}{\kern0pt}{\isadigit{1}}{\isacharcomma}{\kern0pt}{\isadigit{1}}{\isacharparenright}{\kern0pt}{\isacharcomma}{\kern0pt}\ of\ i\ {\isachardoublequoteopen}Suc\ i{\isachardoublequoteclose}\ i{\isacharbrackright}{\kern0pt}\ cond{\isacharunderscore}{\kern0pt}exp{\isacharunderscore}{\kern0pt}F{\isacharunderscore}{\kern0pt}meas{\isacharbrackleft}{\kern0pt}OF\ integrable\ adapted{\isacharcomma}{\kern0pt}\ of\ i{\isacharbrackright}{\kern0pt}\ assms{\isacharparenleft}{\kern0pt}{\isadigit{2}}{\isacharparenright}{\kern0pt}{\isacharbrackleft}{\kern0pt}of\ i{\isacharbrackright}{\kern0pt}\ \isacommand{by}\isamarkupfalse%
\ fastforce\isanewline
\isacommand{qed}\isamarkupfalse%
%
\endisatagproof
{\isafoldproof}%
%
\isadelimproof
%
\endisadelimproof
%
\isadelimdocument
%
\endisadelimdocument
%
\isatagdocument
%
\isamarkupsection{Discrete Time Submartingales%
}
\isamarkuptrue%
%
\endisatagdocument
{\isafolddocument}%
%
\isadelimdocument
%
\endisadelimdocument
\isacommand{lemma}\isamarkupfalse%
\ {\isacharparenleft}{\kern0pt}\isakeyword{in}\ discrete{\isacharunderscore}{\kern0pt}time{\isacharunderscore}{\kern0pt}submartingale{\isacharparenright}{\kern0pt}\ predictable{\isacharunderscore}{\kern0pt}ge{\isacharunderscore}{\kern0pt}bot{\isacharcolon}{\kern0pt}\isanewline
\ \ \isakeyword{assumes}\ {\isachardoublequoteopen}predictable\ X{\isachardoublequoteclose}\isanewline
\ \ \isakeyword{shows}\ {\isachardoublequoteopen}AE\ {\isasymxi}\ in\ M{\isachardot}{\kern0pt}\ X\ i\ {\isasymxi}\ {\isasymge}\ X\ {\isasymbottom}\ {\isasymxi}{\isachardoublequoteclose}\isanewline
%
\isadelimproof
%
\endisadelimproof
%
\isatagproof
\isacommand{proof}\isamarkupfalse%
\ {\isacharparenleft}{\kern0pt}induction\ i{\isacharparenright}{\kern0pt}\isanewline
\ \ \isacommand{case}\isamarkupfalse%
\ {\isadigit{0}}\isanewline
\ \ \isacommand{then}\isamarkupfalse%
\ \isacommand{show}\isamarkupfalse%
\ {\isacharquery}{\kern0pt}case\ \isacommand{by}\isamarkupfalse%
\ {\isacharparenleft}{\kern0pt}simp\ add{\isacharcolon}{\kern0pt}\ bot{\isacharunderscore}{\kern0pt}nat{\isacharunderscore}{\kern0pt}def{\isacharparenright}{\kern0pt}\isanewline
\isacommand{next}\isamarkupfalse%
\isanewline
\ \ \isacommand{case}\isamarkupfalse%
\ {\isacharparenleft}{\kern0pt}Suc\ i{\isacharparenright}{\kern0pt}\isanewline
\ \ \isacommand{thus}\isamarkupfalse%
\ {\isacharquery}{\kern0pt}case\ \isacommand{using}\isamarkupfalse%
\ predictable{\isacharunderscore}{\kern0pt}discrete{\isacharunderscore}{\kern0pt}time{\isacharunderscore}{\kern0pt}process{\isacharunderscore}{\kern0pt}measurable{\isacharbrackleft}{\kern0pt}OF\ assms{\isacharcomma}{\kern0pt}\ of\ {\isachardoublequoteopen}Suc\ i{\isachardoublequoteclose}{\isacharbrackright}{\kern0pt}\ \isanewline
\ \ \ \ \ \ \ \ \ \ \ \ \ \ \ \ \ \ \ submartingale{\isacharunderscore}{\kern0pt}property{\isacharbrackleft}{\kern0pt}OF\ le{\isacharunderscore}{\kern0pt}SucI{\isacharcomma}{\kern0pt}\ of\ i{\isacharbrackright}{\kern0pt}\isanewline
\ \ \ \ \ \ \ \ \ \ \ \ \ \ \ \ \ \ \ cond{\isacharunderscore}{\kern0pt}exp{\isacharunderscore}{\kern0pt}F{\isacharunderscore}{\kern0pt}meas{\isacharbrackleft}{\kern0pt}OF\ integrable{\isacharcomma}{\kern0pt}\ of\ {\isachardoublequoteopen}Suc\ i{\isachardoublequoteclose}\ i{\isacharbrackright}{\kern0pt}\ Suc\ \isacommand{by}\isamarkupfalse%
\ fastforce\isanewline
\isacommand{qed}\isamarkupfalse%
%
\endisatagproof
{\isafoldproof}%
%
\isadelimproof
\isanewline
%
\endisadelimproof
\isanewline
\isacommand{lemma}\isamarkupfalse%
\ {\isacharparenleft}{\kern0pt}\isakeyword{in}\ discrete{\isacharunderscore}{\kern0pt}time{\isacharunderscore}{\kern0pt}adapted{\isacharunderscore}{\kern0pt}process{\isacharunderscore}{\kern0pt}order{\isacharparenright}{\kern0pt}\ submartingale{\isacharunderscore}{\kern0pt}of{\isacharunderscore}{\kern0pt}set{\isacharunderscore}{\kern0pt}integral{\isacharunderscore}{\kern0pt}le{\isacharunderscore}{\kern0pt}Suc{\isacharcolon}{\kern0pt}\isanewline
\ \ \isakeyword{assumes}\ integrable{\isacharcolon}{\kern0pt}\ {\isachardoublequoteopen}{\isasymAnd}i{\isachardot}{\kern0pt}\ integrable\ M\ {\isacharparenleft}{\kern0pt}X\ i{\isacharparenright}{\kern0pt}{\isachardoublequoteclose}\ \isanewline
\ \ \ \ \ \ \isakeyword{and}\ {\isachardoublequoteopen}{\isasymAnd}A\ i{\isachardot}{\kern0pt}\ A\ {\isasymin}\ F\ i\ {\isasymLongrightarrow}\ set{\isacharunderscore}{\kern0pt}lebesgue{\isacharunderscore}{\kern0pt}integral\ M\ A\ {\isacharparenleft}{\kern0pt}X\ i{\isacharparenright}{\kern0pt}\ {\isasymle}\ set{\isacharunderscore}{\kern0pt}lebesgue{\isacharunderscore}{\kern0pt}integral\ M\ A\ {\isacharparenleft}{\kern0pt}X\ {\isacharparenleft}{\kern0pt}Suc\ i{\isacharparenright}{\kern0pt}{\isacharparenright}{\kern0pt}{\isachardoublequoteclose}\ \isanewline
\ \ \ \ \isakeyword{shows}\ {\isachardoublequoteopen}discrete{\isacharunderscore}{\kern0pt}time{\isacharunderscore}{\kern0pt}submartingale\ M\ F\ X{\isachardoublequoteclose}\isanewline
%
\isadelimproof
%
\endisadelimproof
%
\isatagproof
\isacommand{proof}\isamarkupfalse%
\ {\isacharparenleft}{\kern0pt}intro\ discrete{\isacharunderscore}{\kern0pt}time{\isacharunderscore}{\kern0pt}submartingale{\isachardot}{\kern0pt}intro\ submartingale{\isacharunderscore}{\kern0pt}of{\isacharunderscore}{\kern0pt}set{\isacharunderscore}{\kern0pt}integral{\isacharunderscore}{\kern0pt}le{\isacharparenright}{\kern0pt}\isanewline
\ \ \isacommand{fix}\isamarkupfalse%
\ i\ j\ A\ \isacommand{assume}\isamarkupfalse%
\ asm{\isacharcolon}{\kern0pt}\ {\isachardoublequoteopen}i\ {\isasymle}\ j{\isachardoublequoteclose}\ {\isachardoublequoteopen}A\ {\isasymin}\ sets\ {\isacharparenleft}{\kern0pt}F\ i{\isacharparenright}{\kern0pt}{\isachardoublequoteclose}\isanewline
\ \ \isacommand{show}\isamarkupfalse%
\ {\isachardoublequoteopen}set{\isacharunderscore}{\kern0pt}lebesgue{\isacharunderscore}{\kern0pt}integral\ M\ A\ {\isacharparenleft}{\kern0pt}X\ i{\isacharparenright}{\kern0pt}\ {\isasymle}\ set{\isacharunderscore}{\kern0pt}lebesgue{\isacharunderscore}{\kern0pt}integral\ M\ A\ {\isacharparenleft}{\kern0pt}X\ j{\isacharparenright}{\kern0pt}{\isachardoublequoteclose}\ \isacommand{using}\isamarkupfalse%
\ asm\isanewline
\ \ \isacommand{proof}\isamarkupfalse%
\ {\isacharparenleft}{\kern0pt}induction\ {\isachardoublequoteopen}j\ {\isacharminus}{\kern0pt}\ i{\isachardoublequoteclose}\ arbitrary{\isacharcolon}{\kern0pt}\ i\ j{\isacharparenright}{\kern0pt}\isanewline
\ \ \ \ \isacommand{case}\isamarkupfalse%
\ {\isadigit{0}}\isanewline
\ \ \ \ \isacommand{then}\isamarkupfalse%
\ \isacommand{show}\isamarkupfalse%
\ {\isacharquery}{\kern0pt}case\ \isacommand{using}\isamarkupfalse%
\ asm\ \isacommand{by}\isamarkupfalse%
\ simp\isanewline
\ \ \isacommand{next}\isamarkupfalse%
\isanewline
\ \ \ \ \isacommand{case}\isamarkupfalse%
\ {\isacharparenleft}{\kern0pt}Suc\ n{\isacharparenright}{\kern0pt}\isanewline
\ \ \ \ \isacommand{hence}\isamarkupfalse%
\ {\isacharasterisk}{\kern0pt}{\isacharcolon}{\kern0pt}\ {\isachardoublequoteopen}n\ {\isacharequal}{\kern0pt}\ j\ {\isacharminus}{\kern0pt}\ Suc\ i{\isachardoublequoteclose}\ \isacommand{by}\isamarkupfalse%
\ linarith\isanewline
\ \ \ \ \isacommand{have}\isamarkupfalse%
\ {\isachardoublequoteopen}Suc\ i\ {\isasymle}\ j{\isachardoublequoteclose}\ \isacommand{using}\isamarkupfalse%
\ Suc{\isacharparenleft}{\kern0pt}{\isadigit{2}}{\isacharcomma}{\kern0pt}{\isadigit{3}}{\isacharparenright}{\kern0pt}\ \isacommand{by}\isamarkupfalse%
\ linarith\isanewline
\ \ \ \ \isacommand{thus}\isamarkupfalse%
\ {\isacharquery}{\kern0pt}case\ \isacommand{using}\isamarkupfalse%
\ sets{\isacharunderscore}{\kern0pt}F{\isacharunderscore}{\kern0pt}mono{\isacharbrackleft}{\kern0pt}OF\ le{\isacharunderscore}{\kern0pt}SucI{\isacharbrackright}{\kern0pt}\ Suc{\isacharparenleft}{\kern0pt}{\isadigit{4}}{\isacharparenright}{\kern0pt}\ Suc{\isacharparenleft}{\kern0pt}{\isadigit{1}}{\isacharparenright}{\kern0pt}{\isacharbrackleft}{\kern0pt}OF\ {\isacharasterisk}{\kern0pt}{\isacharbrackright}{\kern0pt}\ \isacommand{by}\isamarkupfalse%
\ {\isacharparenleft}{\kern0pt}auto\ intro{\isacharcolon}{\kern0pt}\ assms{\isacharparenleft}{\kern0pt}{\isadigit{2}}{\isacharparenright}{\kern0pt}{\isacharbrackleft}{\kern0pt}THEN\ order{\isacharunderscore}{\kern0pt}trans{\isacharbrackright}{\kern0pt}{\isacharparenright}{\kern0pt}\isanewline
\ \ \isacommand{qed}\isamarkupfalse%
\isanewline
\isacommand{qed}\isamarkupfalse%
\ {\isacharparenleft}{\kern0pt}simp\ add{\isacharcolon}{\kern0pt}\ integrable{\isacharparenright}{\kern0pt}%
\endisatagproof
{\isafoldproof}%
%
\isadelimproof
\isanewline
%
\endisadelimproof
\isanewline
\isacommand{lemma}\isamarkupfalse%
\ {\isacharparenleft}{\kern0pt}\isakeyword{in}\ discrete{\isacharunderscore}{\kern0pt}time{\isacharunderscore}{\kern0pt}adapted{\isacharunderscore}{\kern0pt}process{\isacharunderscore}{\kern0pt}order{\isacharparenright}{\kern0pt}\ submartingale{\isacharunderscore}{\kern0pt}nat{\isacharcolon}{\kern0pt}\isanewline
\ \ \isakeyword{assumes}\ integrable{\isacharcolon}{\kern0pt}\ {\isachardoublequoteopen}{\isasymAnd}i{\isachardot}{\kern0pt}\ integrable\ M\ {\isacharparenleft}{\kern0pt}X\ i{\isacharparenright}{\kern0pt}{\isachardoublequoteclose}\ \isanewline
\ \ \ \ \ \ \isakeyword{and}\ {\isachardoublequoteopen}{\isasymAnd}i{\isachardot}{\kern0pt}\ AE\ {\isasymxi}\ in\ M{\isachardot}{\kern0pt}\ X\ i\ {\isasymxi}\ {\isasymle}\ cond{\isacharunderscore}{\kern0pt}exp\ M\ {\isacharparenleft}{\kern0pt}F\ i{\isacharparenright}{\kern0pt}\ {\isacharparenleft}{\kern0pt}X\ {\isacharparenleft}{\kern0pt}Suc\ i{\isacharparenright}{\kern0pt}{\isacharparenright}{\kern0pt}\ {\isasymxi}{\isachardoublequoteclose}\ \isanewline
\ \ \ \ \isakeyword{shows}\ {\isachardoublequoteopen}discrete{\isacharunderscore}{\kern0pt}time{\isacharunderscore}{\kern0pt}submartingale\ M\ F\ X{\isachardoublequoteclose}\isanewline
%
\isadelimproof
\ \ %
\endisadelimproof
%
\isatagproof
\isacommand{using}\isamarkupfalse%
\ subalg\ integrable\ assms{\isacharparenleft}{\kern0pt}{\isadigit{2}}{\isacharparenright}{\kern0pt}\isanewline
\ \ \isacommand{by}\isamarkupfalse%
\ {\isacharparenleft}{\kern0pt}intro\ submartingale{\isacharunderscore}{\kern0pt}of{\isacharunderscore}{\kern0pt}set{\isacharunderscore}{\kern0pt}integral{\isacharunderscore}{\kern0pt}le{\isacharunderscore}{\kern0pt}Suc\ ord{\isacharunderscore}{\kern0pt}le{\isacharunderscore}{\kern0pt}eq{\isacharunderscore}{\kern0pt}trans{\isacharbrackleft}{\kern0pt}OF\ set{\isacharunderscore}{\kern0pt}integral{\isacharunderscore}{\kern0pt}mono{\isacharunderscore}{\kern0pt}AE{\isacharunderscore}{\kern0pt}banach\ cond{\isacharunderscore}{\kern0pt}exp{\isacharunderscore}{\kern0pt}set{\isacharunderscore}{\kern0pt}integral{\isacharbrackleft}{\kern0pt}symmetric{\isacharbrackright}{\kern0pt}{\isacharbrackright}{\kern0pt}{\isacharcomma}{\kern0pt}\ simp{\isacharparenright}{\kern0pt}\isanewline
\ \ \ \ \ \ \ \ \ {\isacharparenleft}{\kern0pt}meson\ in{\isacharunderscore}{\kern0pt}mono\ integrable{\isacharunderscore}{\kern0pt}mult{\isacharunderscore}{\kern0pt}indicator\ set{\isacharunderscore}{\kern0pt}integrable{\isacharunderscore}{\kern0pt}def\ subalgebra{\isacharunderscore}{\kern0pt}def{\isacharcomma}{\kern0pt}\isanewline
\ \ \ \ \ \ \ \ \ \ meson\ integrable{\isacharunderscore}{\kern0pt}cond{\isacharunderscore}{\kern0pt}exp\ in{\isacharunderscore}{\kern0pt}mono\ integrable{\isacharunderscore}{\kern0pt}mult{\isacharunderscore}{\kern0pt}indicator\ set{\isacharunderscore}{\kern0pt}integrable{\isacharunderscore}{\kern0pt}def\ subalgebra{\isacharunderscore}{\kern0pt}def{\isacharcomma}{\kern0pt}\isanewline
\ \ \ \ \ \ \ \ \ \ auto\ simp\ add{\isacharcolon}{\kern0pt}\ subalgebra{\isacharunderscore}{\kern0pt}def{\isacharcomma}{\kern0pt}\ metis\ {\isacharparenleft}{\kern0pt}mono{\isacharunderscore}{\kern0pt}tags{\isacharcomma}{\kern0pt}\ lifting{\isacharparenright}{\kern0pt}\ AE{\isacharunderscore}{\kern0pt}I{\isadigit{2}}\ AE{\isacharunderscore}{\kern0pt}mp{\isacharparenright}{\kern0pt}%
\endisatagproof
{\isafoldproof}%
%
\isadelimproof
\isanewline
%
\endisadelimproof
\isanewline
\isacommand{lemma}\isamarkupfalse%
\ {\isacharparenleft}{\kern0pt}\isakeyword{in}\ discrete{\isacharunderscore}{\kern0pt}time{\isacharunderscore}{\kern0pt}adapted{\isacharunderscore}{\kern0pt}process{\isacharunderscore}{\kern0pt}order{\isacharparenright}{\kern0pt}\ submartingale{\isacharunderscore}{\kern0pt}of{\isacharunderscore}{\kern0pt}cond{\isacharunderscore}{\kern0pt}exp{\isacharunderscore}{\kern0pt}diff{\isacharunderscore}{\kern0pt}Suc{\isacharunderscore}{\kern0pt}nonneg{\isacharcolon}{\kern0pt}\isanewline
\ \ \isakeyword{assumes}\ integrable{\isacharcolon}{\kern0pt}\ {\isachardoublequoteopen}{\isasymAnd}i{\isachardot}{\kern0pt}\ integrable\ M\ {\isacharparenleft}{\kern0pt}X\ i{\isacharparenright}{\kern0pt}{\isachardoublequoteclose}\ \isanewline
\ \ \ \ \ \ \isakeyword{and}\ {\isachardoublequoteopen}{\isasymAnd}i{\isachardot}{\kern0pt}\ AE\ {\isasymxi}\ in\ M{\isachardot}{\kern0pt}\ {\isadigit{0}}\ {\isasymle}\ cond{\isacharunderscore}{\kern0pt}exp\ M\ {\isacharparenleft}{\kern0pt}F\ i{\isacharparenright}{\kern0pt}\ {\isacharparenleft}{\kern0pt}{\isasymlambda}{\isasymxi}{\isachardot}{\kern0pt}\ X\ {\isacharparenleft}{\kern0pt}Suc\ i{\isacharparenright}{\kern0pt}\ {\isasymxi}\ {\isacharminus}{\kern0pt}\ X\ i\ {\isasymxi}{\isacharparenright}{\kern0pt}\ {\isasymxi}{\isachardoublequoteclose}\ \isanewline
\ \ \ \ \isakeyword{shows}\ {\isachardoublequoteopen}discrete{\isacharunderscore}{\kern0pt}time{\isacharunderscore}{\kern0pt}submartingale\ M\ F\ X{\isachardoublequoteclose}\isanewline
%
\isadelimproof
%
\endisadelimproof
%
\isatagproof
\isacommand{proof}\isamarkupfalse%
\ {\isacharparenleft}{\kern0pt}intro\ submartingale{\isacharunderscore}{\kern0pt}nat\ integrable{\isacharparenright}{\kern0pt}\ \isanewline
\ \ \isacommand{fix}\isamarkupfalse%
\ i\ \isanewline
\ \ \isacommand{show}\isamarkupfalse%
\ {\isachardoublequoteopen}AE\ {\isasymxi}\ in\ M{\isachardot}{\kern0pt}\ X\ i\ {\isasymxi}\ {\isasymle}\ cond{\isacharunderscore}{\kern0pt}exp\ M\ {\isacharparenleft}{\kern0pt}F\ i{\isacharparenright}{\kern0pt}\ {\isacharparenleft}{\kern0pt}X\ {\isacharparenleft}{\kern0pt}Suc\ i{\isacharparenright}{\kern0pt}{\isacharparenright}{\kern0pt}\ {\isasymxi}{\isachardoublequoteclose}\ \isacommand{using}\isamarkupfalse%
\ cond{\isacharunderscore}{\kern0pt}exp{\isacharunderscore}{\kern0pt}diff{\isacharbrackleft}{\kern0pt}OF\ integrable{\isacharparenleft}{\kern0pt}{\isadigit{1}}{\isacharcomma}{\kern0pt}{\isadigit{1}}{\isacharparenright}{\kern0pt}{\isacharcomma}{\kern0pt}\ of\ i\ {\isachardoublequoteopen}Suc\ i{\isachardoublequoteclose}\ i{\isacharbrackright}{\kern0pt}\ cond{\isacharunderscore}{\kern0pt}exp{\isacharunderscore}{\kern0pt}F{\isacharunderscore}{\kern0pt}meas{\isacharbrackleft}{\kern0pt}OF\ integrable\ adapted{\isacharcomma}{\kern0pt}\ of\ i{\isacharbrackright}{\kern0pt}\ assms{\isacharparenleft}{\kern0pt}{\isadigit{2}}{\isacharparenright}{\kern0pt}{\isacharbrackleft}{\kern0pt}of\ i{\isacharbrackright}{\kern0pt}\ \isacommand{by}\isamarkupfalse%
\ fastforce\isanewline
\isacommand{qed}\isamarkupfalse%
%
\endisatagproof
{\isafoldproof}%
%
\isadelimproof
%
\endisadelimproof
%
\isadelimdocument
%
\endisadelimdocument
%
\isatagdocument
%
\isamarkupsection{Discrete Time Supermartingales%
}
\isamarkuptrue%
%
\endisatagdocument
{\isafolddocument}%
%
\isadelimdocument
%
\endisadelimdocument
\isacommand{lemma}\isamarkupfalse%
\ {\isacharparenleft}{\kern0pt}\isakeyword{in}\ discrete{\isacharunderscore}{\kern0pt}time{\isacharunderscore}{\kern0pt}supermartingale{\isacharparenright}{\kern0pt}\ predictable{\isacharunderscore}{\kern0pt}le{\isacharunderscore}{\kern0pt}bot{\isacharcolon}{\kern0pt}\isanewline
\ \ \isakeyword{assumes}\ {\isachardoublequoteopen}predictable\ X{\isachardoublequoteclose}\isanewline
\ \ \isakeyword{shows}\ {\isachardoublequoteopen}AE\ {\isasymxi}\ in\ M{\isachardot}{\kern0pt}\ X\ i\ {\isasymxi}\ {\isasymle}\ X\ {\isasymbottom}\ {\isasymxi}{\isachardoublequoteclose}\isanewline
%
\isadelimproof
%
\endisadelimproof
%
\isatagproof
\isacommand{proof}\isamarkupfalse%
\ {\isacharparenleft}{\kern0pt}induction\ i{\isacharparenright}{\kern0pt}\isanewline
\ \ \isacommand{case}\isamarkupfalse%
\ {\isadigit{0}}\isanewline
\ \ \isacommand{then}\isamarkupfalse%
\ \isacommand{show}\isamarkupfalse%
\ {\isacharquery}{\kern0pt}case\ \isacommand{by}\isamarkupfalse%
\ {\isacharparenleft}{\kern0pt}simp\ add{\isacharcolon}{\kern0pt}\ bot{\isacharunderscore}{\kern0pt}nat{\isacharunderscore}{\kern0pt}def{\isacharparenright}{\kern0pt}\isanewline
\isacommand{next}\isamarkupfalse%
\isanewline
\ \ \isacommand{case}\isamarkupfalse%
\ {\isacharparenleft}{\kern0pt}Suc\ i{\isacharparenright}{\kern0pt}\isanewline
\ \ \isacommand{thus}\isamarkupfalse%
\ {\isacharquery}{\kern0pt}case\ \isacommand{using}\isamarkupfalse%
\ predictable{\isacharunderscore}{\kern0pt}discrete{\isacharunderscore}{\kern0pt}time{\isacharunderscore}{\kern0pt}process{\isacharunderscore}{\kern0pt}measurable{\isacharbrackleft}{\kern0pt}OF\ assms{\isacharcomma}{\kern0pt}\ of\ {\isachardoublequoteopen}Suc\ i{\isachardoublequoteclose}{\isacharbrackright}{\kern0pt}\ \isanewline
\ \ \ \ \ \ \ \ \ \ \ \ \ \ \ \ \ \ \ supermartingale{\isacharunderscore}{\kern0pt}property{\isacharbrackleft}{\kern0pt}OF\ le{\isacharunderscore}{\kern0pt}SucI{\isacharcomma}{\kern0pt}\ of\ i{\isacharbrackright}{\kern0pt}\isanewline
\ \ \ \ \ \ \ \ \ \ \ \ \ \ \ \ \ \ \ cond{\isacharunderscore}{\kern0pt}exp{\isacharunderscore}{\kern0pt}F{\isacharunderscore}{\kern0pt}meas{\isacharbrackleft}{\kern0pt}OF\ integrable{\isacharcomma}{\kern0pt}\ of\ {\isachardoublequoteopen}Suc\ i{\isachardoublequoteclose}\ i{\isacharbrackright}{\kern0pt}\ Suc\ \isacommand{by}\isamarkupfalse%
\ fastforce\isanewline
\isacommand{qed}\isamarkupfalse%
%
\endisatagproof
{\isafoldproof}%
%
\isadelimproof
\isanewline
%
\endisadelimproof
\isanewline
\isacommand{lemma}\isamarkupfalse%
\ {\isacharparenleft}{\kern0pt}\isakeyword{in}\ discrete{\isacharunderscore}{\kern0pt}time{\isacharunderscore}{\kern0pt}adapted{\isacharunderscore}{\kern0pt}process{\isacharunderscore}{\kern0pt}order{\isacharparenright}{\kern0pt}\ supermartingale{\isacharunderscore}{\kern0pt}of{\isacharunderscore}{\kern0pt}set{\isacharunderscore}{\kern0pt}integral{\isacharunderscore}{\kern0pt}ge{\isacharunderscore}{\kern0pt}Suc{\isacharcolon}{\kern0pt}\isanewline
\ \ \isakeyword{assumes}\ integrable{\isacharcolon}{\kern0pt}\ {\isachardoublequoteopen}{\isasymAnd}i{\isachardot}{\kern0pt}\ integrable\ M\ {\isacharparenleft}{\kern0pt}X\ i{\isacharparenright}{\kern0pt}{\isachardoublequoteclose}\ \isanewline
\ \ \ \ \ \ \isakeyword{and}\ {\isachardoublequoteopen}{\isasymAnd}A\ i{\isachardot}{\kern0pt}\ A\ {\isasymin}\ F\ i\ {\isasymLongrightarrow}\ set{\isacharunderscore}{\kern0pt}lebesgue{\isacharunderscore}{\kern0pt}integral\ M\ A\ {\isacharparenleft}{\kern0pt}X\ {\isacharparenleft}{\kern0pt}Suc\ i{\isacharparenright}{\kern0pt}{\isacharparenright}{\kern0pt}\ {\isasymle}\ set{\isacharunderscore}{\kern0pt}lebesgue{\isacharunderscore}{\kern0pt}integral\ M\ A\ {\isacharparenleft}{\kern0pt}X\ i{\isacharparenright}{\kern0pt}{\isachardoublequoteclose}\ \isanewline
\ \ \ \ \isakeyword{shows}\ {\isachardoublequoteopen}discrete{\isacharunderscore}{\kern0pt}time{\isacharunderscore}{\kern0pt}supermartingale\ M\ F\ X{\isachardoublequoteclose}\isanewline
%
\isadelimproof
%
\endisadelimproof
%
\isatagproof
\isacommand{proof}\isamarkupfalse%
\ {\isacharminus}{\kern0pt}\isanewline
\ \ \isacommand{interpret}\isamarkupfalse%
\ uminus{\isacharunderscore}{\kern0pt}X{\isacharcolon}{\kern0pt}\ discrete{\isacharunderscore}{\kern0pt}time{\isacharunderscore}{\kern0pt}adapted{\isacharunderscore}{\kern0pt}process{\isacharunderscore}{\kern0pt}order\ M\ F\ {\isachardoublequoteopen}{\isacharminus}{\kern0pt}X{\isachardoublequoteclose}\ \isacommand{by}\isamarkupfalse%
\ {\isacharparenleft}{\kern0pt}intro\ discrete{\isacharunderscore}{\kern0pt}time{\isacharunderscore}{\kern0pt}adapted{\isacharunderscore}{\kern0pt}process{\isacharunderscore}{\kern0pt}order{\isachardot}{\kern0pt}intro\ adapted{\isacharunderscore}{\kern0pt}process{\isacharunderscore}{\kern0pt}order{\isachardot}{\kern0pt}intro\ uminus{\isacharparenright}{\kern0pt}\isanewline
\ \ \isacommand{note}\isamarkupfalse%
\ {\isacharasterisk}{\kern0pt}\ {\isacharequal}{\kern0pt}\ set{\isacharunderscore}{\kern0pt}integral{\isacharunderscore}{\kern0pt}uminus{\isacharbrackleft}{\kern0pt}unfolded\ set{\isacharunderscore}{\kern0pt}integrable{\isacharunderscore}{\kern0pt}def{\isacharcomma}{\kern0pt}\ OF\ integrable{\isacharunderscore}{\kern0pt}mult{\isacharunderscore}{\kern0pt}indicator{\isacharbrackleft}{\kern0pt}OF\ {\isacharunderscore}{\kern0pt}\ integrable{\isacharbrackright}{\kern0pt}{\isacharbrackright}{\kern0pt}\isanewline
\ \ \isacommand{have}\isamarkupfalse%
\ {\isachardoublequoteopen}discrete{\isacharunderscore}{\kern0pt}time{\isacharunderscore}{\kern0pt}supermartingale\ M\ F\ {\isacharparenleft}{\kern0pt}{\isacharminus}{\kern0pt}{\isacharparenleft}{\kern0pt}{\isacharminus}{\kern0pt}\ X{\isacharparenright}{\kern0pt}{\isacharparenright}{\kern0pt}{\isachardoublequoteclose}\ \isacommand{using}\isamarkupfalse%
\ ord{\isacharunderscore}{\kern0pt}eq{\isacharunderscore}{\kern0pt}le{\isacharunderscore}{\kern0pt}trans{\isacharbrackleft}{\kern0pt}OF\ {\isacharasterisk}{\kern0pt}\ ord{\isacharunderscore}{\kern0pt}le{\isacharunderscore}{\kern0pt}eq{\isacharunderscore}{\kern0pt}trans{\isacharbrackleft}{\kern0pt}OF\ le{\isacharunderscore}{\kern0pt}imp{\isacharunderscore}{\kern0pt}neg{\isacharunderscore}{\kern0pt}le{\isacharbrackleft}{\kern0pt}OF\ assms{\isacharparenleft}{\kern0pt}{\isadigit{2}}{\isacharparenright}{\kern0pt}{\isacharbrackright}{\kern0pt}\ {\isacharasterisk}{\kern0pt}{\isacharbrackleft}{\kern0pt}symmetric{\isacharbrackright}{\kern0pt}{\isacharbrackright}{\kern0pt}{\isacharbrackright}{\kern0pt}\ subalg\isanewline
\ \ \ \ \isacommand{by}\isamarkupfalse%
\ {\isacharparenleft}{\kern0pt}intro\ discrete{\isacharunderscore}{\kern0pt}time{\isacharunderscore}{\kern0pt}supermartingale{\isachardot}{\kern0pt}intro\ submartingale{\isachardot}{\kern0pt}uminus\ discrete{\isacharunderscore}{\kern0pt}time{\isacharunderscore}{\kern0pt}submartingale{\isachardot}{\kern0pt}axioms\ uminus{\isacharunderscore}{\kern0pt}X{\isachardot}{\kern0pt}submartingale{\isacharunderscore}{\kern0pt}of{\isacharunderscore}{\kern0pt}set{\isacharunderscore}{\kern0pt}integral{\isacharunderscore}{\kern0pt}le{\isacharunderscore}{\kern0pt}Suc{\isacharparenright}{\kern0pt}\ {\isacharparenleft}{\kern0pt}auto\ simp\ add{\isacharcolon}{\kern0pt}\ subalgebra{\isacharunderscore}{\kern0pt}def\ integrable\ fun{\isacharunderscore}{\kern0pt}Compl{\isacharunderscore}{\kern0pt}def{\isacharcomma}{\kern0pt}\ blast{\isacharparenright}{\kern0pt}\isanewline
\ \ \isacommand{thus}\isamarkupfalse%
\ {\isacharquery}{\kern0pt}thesis\ \isacommand{unfolding}\isamarkupfalse%
\ fun{\isacharunderscore}{\kern0pt}Compl{\isacharunderscore}{\kern0pt}def\ \isacommand{by}\isamarkupfalse%
\ simp\isanewline
\isacommand{qed}\isamarkupfalse%
%
\endisatagproof
{\isafoldproof}%
%
\isadelimproof
\isanewline
%
\endisadelimproof
\isanewline
\isacommand{lemma}\isamarkupfalse%
\ {\isacharparenleft}{\kern0pt}\isakeyword{in}\ discrete{\isacharunderscore}{\kern0pt}time{\isacharunderscore}{\kern0pt}adapted{\isacharunderscore}{\kern0pt}process{\isacharunderscore}{\kern0pt}order{\isacharparenright}{\kern0pt}\ supermartingale{\isacharunderscore}{\kern0pt}nat{\isacharcolon}{\kern0pt}\isanewline
\ \ \isakeyword{assumes}\ integrable{\isacharcolon}{\kern0pt}\ {\isachardoublequoteopen}{\isasymAnd}i{\isachardot}{\kern0pt}\ integrable\ M\ {\isacharparenleft}{\kern0pt}X\ i{\isacharparenright}{\kern0pt}{\isachardoublequoteclose}\ \isanewline
\ \ \ \ \ \ \isakeyword{and}\ {\isachardoublequoteopen}{\isasymAnd}i{\isachardot}{\kern0pt}\ AE\ {\isasymxi}\ in\ M{\isachardot}{\kern0pt}\ X\ i\ {\isasymxi}\ {\isasymge}\ cond{\isacharunderscore}{\kern0pt}exp\ M\ {\isacharparenleft}{\kern0pt}F\ i{\isacharparenright}{\kern0pt}\ {\isacharparenleft}{\kern0pt}X\ {\isacharparenleft}{\kern0pt}Suc\ i{\isacharparenright}{\kern0pt}{\isacharparenright}{\kern0pt}\ {\isasymxi}{\isachardoublequoteclose}\ \isanewline
\ \ \ \ \isakeyword{shows}\ {\isachardoublequoteopen}discrete{\isacharunderscore}{\kern0pt}time{\isacharunderscore}{\kern0pt}supermartingale\ M\ F\ X{\isachardoublequoteclose}\isanewline
%
\isadelimproof
%
\endisadelimproof
%
\isatagproof
\isacommand{proof}\isamarkupfalse%
\ {\isacharminus}{\kern0pt}\isanewline
\ \ \isacommand{interpret}\isamarkupfalse%
\ uminus{\isacharunderscore}{\kern0pt}X{\isacharcolon}{\kern0pt}\ discrete{\isacharunderscore}{\kern0pt}time{\isacharunderscore}{\kern0pt}adapted{\isacharunderscore}{\kern0pt}process{\isacharunderscore}{\kern0pt}order\ M\ F\ {\isachardoublequoteopen}{\isacharminus}{\kern0pt}X{\isachardoublequoteclose}\ \isacommand{by}\isamarkupfalse%
\ {\isacharparenleft}{\kern0pt}intro\ discrete{\isacharunderscore}{\kern0pt}time{\isacharunderscore}{\kern0pt}adapted{\isacharunderscore}{\kern0pt}process{\isacharunderscore}{\kern0pt}order{\isachardot}{\kern0pt}intro\ adapted{\isacharunderscore}{\kern0pt}process{\isacharunderscore}{\kern0pt}order{\isachardot}{\kern0pt}intro\ uminus{\isacharparenright}{\kern0pt}\isanewline
\ \ \isacommand{have}\isamarkupfalse%
\ {\isachardoublequoteopen}AE\ {\isasymxi}\ in\ M{\isachardot}{\kern0pt}\ {\isacharminus}{\kern0pt}\ X\ i\ {\isasymxi}\ {\isasymle}\ cond{\isacharunderscore}{\kern0pt}exp\ M\ {\isacharparenleft}{\kern0pt}F\ i{\isacharparenright}{\kern0pt}\ {\isacharparenleft}{\kern0pt}{\isasymlambda}x{\isachardot}{\kern0pt}\ {\isacharminus}{\kern0pt}\ X\ {\isacharparenleft}{\kern0pt}Suc\ i{\isacharparenright}{\kern0pt}\ x{\isacharparenright}{\kern0pt}\ {\isasymxi}{\isachardoublequoteclose}\ \isakeyword{for}\ i\ \isacommand{using}\isamarkupfalse%
\ assms{\isacharparenleft}{\kern0pt}{\isadigit{2}}{\isacharparenright}{\kern0pt}\ cond{\isacharunderscore}{\kern0pt}exp{\isacharunderscore}{\kern0pt}uminus{\isacharbrackleft}{\kern0pt}OF\ integrable{\isacharcomma}{\kern0pt}\ of\ i\ {\isachardoublequoteopen}Suc\ i{\isachardoublequoteclose}{\isacharbrackright}{\kern0pt}\ \isacommand{by}\isamarkupfalse%
\ force\isanewline
\ \ \isacommand{hence}\isamarkupfalse%
\ {\isachardoublequoteopen}discrete{\isacharunderscore}{\kern0pt}time{\isacharunderscore}{\kern0pt}supermartingale\ M\ F\ {\isacharparenleft}{\kern0pt}{\isacharminus}{\kern0pt}{\isacharparenleft}{\kern0pt}{\isacharminus}{\kern0pt}\ X{\isacharparenright}{\kern0pt}{\isacharparenright}{\kern0pt}{\isachardoublequoteclose}\ \isacommand{by}\isamarkupfalse%
\ {\isacharparenleft}{\kern0pt}intro\ discrete{\isacharunderscore}{\kern0pt}time{\isacharunderscore}{\kern0pt}supermartingale{\isachardot}{\kern0pt}intro\ submartingale{\isachardot}{\kern0pt}uminus\ discrete{\isacharunderscore}{\kern0pt}time{\isacharunderscore}{\kern0pt}submartingale{\isachardot}{\kern0pt}axioms\ uminus{\isacharunderscore}{\kern0pt}X{\isachardot}{\kern0pt}submartingale{\isacharunderscore}{\kern0pt}nat{\isacharparenright}{\kern0pt}\ {\isacharparenleft}{\kern0pt}simp\ only{\isacharcolon}{\kern0pt}\ fun{\isacharunderscore}{\kern0pt}Compl{\isacharunderscore}{\kern0pt}def{\isacharcomma}{\kern0pt}\ intro\ integrable{\isacharunderscore}{\kern0pt}minus\ integrable{\isacharcomma}{\kern0pt}\ auto\ simp\ add{\isacharcolon}{\kern0pt}\ fun{\isacharunderscore}{\kern0pt}Compl{\isacharunderscore}{\kern0pt}def{\isacharparenright}{\kern0pt}\isanewline
\ \ \isacommand{thus}\isamarkupfalse%
\ {\isacharquery}{\kern0pt}thesis\ \isacommand{unfolding}\isamarkupfalse%
\ fun{\isacharunderscore}{\kern0pt}Compl{\isacharunderscore}{\kern0pt}def\ \isacommand{by}\isamarkupfalse%
\ simp\isanewline
\isacommand{qed}\isamarkupfalse%
%
\endisatagproof
{\isafoldproof}%
%
\isadelimproof
\isanewline
%
\endisadelimproof
\isanewline
\isacommand{lemma}\isamarkupfalse%
\ {\isacharparenleft}{\kern0pt}\isakeyword{in}\ discrete{\isacharunderscore}{\kern0pt}time{\isacharunderscore}{\kern0pt}adapted{\isacharunderscore}{\kern0pt}process{\isacharunderscore}{\kern0pt}order{\isacharparenright}{\kern0pt}\ supermartingale{\isacharunderscore}{\kern0pt}of{\isacharunderscore}{\kern0pt}cond{\isacharunderscore}{\kern0pt}exp{\isacharunderscore}{\kern0pt}diff{\isacharunderscore}{\kern0pt}Suc{\isacharunderscore}{\kern0pt}nonneg{\isacharcolon}{\kern0pt}\isanewline
\ \ \isakeyword{assumes}\ integrable{\isacharcolon}{\kern0pt}\ {\isachardoublequoteopen}{\isasymAnd}i{\isachardot}{\kern0pt}\ integrable\ M\ {\isacharparenleft}{\kern0pt}X\ i{\isacharparenright}{\kern0pt}{\isachardoublequoteclose}\ \isanewline
\ \ \ \ \ \ \isakeyword{and}\ {\isachardoublequoteopen}{\isasymAnd}i{\isachardot}{\kern0pt}\ AE\ {\isasymxi}\ in\ M{\isachardot}{\kern0pt}\ {\isadigit{0}}\ {\isasymle}\ cond{\isacharunderscore}{\kern0pt}exp\ M\ {\isacharparenleft}{\kern0pt}F\ i{\isacharparenright}{\kern0pt}\ {\isacharparenleft}{\kern0pt}{\isasymlambda}{\isasymxi}{\isachardot}{\kern0pt}\ X\ i\ {\isasymxi}\ {\isacharminus}{\kern0pt}\ X\ {\isacharparenleft}{\kern0pt}Suc\ i{\isacharparenright}{\kern0pt}\ {\isasymxi}{\isacharparenright}{\kern0pt}\ {\isasymxi}{\isachardoublequoteclose}\ \isanewline
\ \ \ \ \isakeyword{shows}\ {\isachardoublequoteopen}discrete{\isacharunderscore}{\kern0pt}time{\isacharunderscore}{\kern0pt}supermartingale\ M\ F\ X{\isachardoublequoteclose}\isanewline
%
\isadelimproof
%
\endisadelimproof
%
\isatagproof
\isacommand{proof}\isamarkupfalse%
\ {\isacharparenleft}{\kern0pt}intro\ supermartingale{\isacharunderscore}{\kern0pt}nat\ integrable{\isacharparenright}{\kern0pt}\ \isanewline
\ \ \isacommand{fix}\isamarkupfalse%
\ i\ \isanewline
\ \ \isacommand{show}\isamarkupfalse%
\ {\isachardoublequoteopen}AE\ {\isasymxi}\ in\ M{\isachardot}{\kern0pt}\ X\ i\ {\isasymxi}\ {\isasymge}\ cond{\isacharunderscore}{\kern0pt}exp\ M\ {\isacharparenleft}{\kern0pt}F\ i{\isacharparenright}{\kern0pt}\ {\isacharparenleft}{\kern0pt}X\ {\isacharparenleft}{\kern0pt}Suc\ i{\isacharparenright}{\kern0pt}{\isacharparenright}{\kern0pt}\ {\isasymxi}{\isachardoublequoteclose}\ \isacommand{using}\isamarkupfalse%
\ cond{\isacharunderscore}{\kern0pt}exp{\isacharunderscore}{\kern0pt}diff{\isacharbrackleft}{\kern0pt}OF\ integrable{\isacharparenleft}{\kern0pt}{\isadigit{1}}{\isacharcomma}{\kern0pt}{\isadigit{1}}{\isacharparenright}{\kern0pt}{\isacharcomma}{\kern0pt}\ of\ i\ i\ {\isachardoublequoteopen}Suc\ i{\isachardoublequoteclose}{\isacharbrackright}{\kern0pt}\ cond{\isacharunderscore}{\kern0pt}exp{\isacharunderscore}{\kern0pt}F{\isacharunderscore}{\kern0pt}meas{\isacharbrackleft}{\kern0pt}OF\ integrable\ adapted{\isacharcomma}{\kern0pt}\ of\ i{\isacharbrackright}{\kern0pt}\ assms{\isacharparenleft}{\kern0pt}{\isadigit{2}}{\isacharparenright}{\kern0pt}{\isacharbrackleft}{\kern0pt}of\ i{\isacharbrackright}{\kern0pt}\ \isacommand{by}\isamarkupfalse%
\ fastforce\isanewline
\isacommand{qed}\isamarkupfalse%
%
\endisatagproof
{\isafoldproof}%
%
\isadelimproof
\isanewline
%
\endisadelimproof
%
\isadelimtheory
\isanewline
%
\endisadelimtheory
%
\isatagtheory
\isacommand{end}\isamarkupfalse%
%
\endisatagtheory
{\isafoldtheory}%
%
\isadelimtheory
%
\endisadelimtheory
%
\end{isabellebody}%
\endinput
%:%file=Martingale.tex%:%
%:%10=1%:%
%:%11=1%:%
%:%12=2%:%
%:%13=3%:%
%:%27=5%:%
%:%37=7%:%
%:%38=7%:%
%:%39=8%:%
%:%40=9%:%
%:%41=10%:%
%:%42=11%:%
%:%43=11%:%
%:%44=12%:%
%:%45=13%:%
%:%48=14%:%
%:%52=14%:%
%:%53=14%:%
%:%54=15%:%
%:%55=15%:%
%:%56=16%:%
%:%57=17%:%
%:%58=18%:%
%:%63=18%:%
%:%66=19%:%
%:%67=20%:%
%:%68=20%:%
%:%69=21%:%
%:%70=22%:%
%:%73=23%:%
%:%77=23%:%
%:%78=23%:%
%:%79=24%:%
%:%80=25%:%
%:%94=27%:%
%:%104=29%:%
%:%105=29%:%
%:%106=30%:%
%:%107=31%:%
%:%114=33%:%
%:%124=35%:%
%:%125=35%:%
%:%126=36%:%
%:%127=37%:%
%:%134=39%:%
%:%144=43%:%
%:%145=43%:%
%:%146=44%:%
%:%147=45%:%
%:%148=46%:%
%:%149=46%:%
%:%151=46%:%
%:%155=46%:%
%:%156=46%:%
%:%157=46%:%
%:%164=46%:%
%:%165=47%:%
%:%166=48%:%
%:%167=48%:%
%:%169=48%:%
%:%173=48%:%
%:%174=48%:%
%:%175=48%:%
%:%182=48%:%
%:%183=49%:%
%:%184=50%:%
%:%185=50%:%
%:%186=51%:%
%:%187=52%:%
%:%188=52%:%
%:%190=52%:%
%:%194=52%:%
%:%195=52%:%
%:%202=52%:%
%:%203=53%:%
%:%204=54%:%
%:%205=54%:%
%:%207=54%:%
%:%211=54%:%
%:%212=54%:%
%:%219=54%:%
%:%220=55%:%
%:%221=56%:%
%:%222=56%:%
%:%223=57%:%
%:%224=58%:%
%:%225=58%:%
%:%226=59%:%
%:%227=60%:%
%:%228=60%:%
%:%229=61%:%
%:%230=62%:%
%:%231=63%:%
%:%232=63%:%
%:%234=63%:%
%:%238=63%:%
%:%239=63%:%
%:%240=63%:%
%:%247=63%:%
%:%248=64%:%
%:%249=65%:%
%:%250=65%:%
%:%252=65%:%
%:%256=65%:%
%:%257=65%:%
%:%258=65%:%
%:%265=65%:%
%:%266=66%:%
%:%267=67%:%
%:%268=67%:%
%:%269=68%:%
%:%270=69%:%
%:%271=69%:%
%:%273=69%:%
%:%277=69%:%
%:%278=69%:%
%:%285=69%:%
%:%286=70%:%
%:%287=71%:%
%:%288=71%:%
%:%290=71%:%
%:%294=71%:%
%:%295=71%:%
%:%302=71%:%
%:%303=72%:%
%:%304=73%:%
%:%305=73%:%
%:%306=74%:%
%:%307=75%:%
%:%308=76%:%
%:%309=76%:%
%:%310=77%:%
%:%311=78%:%
%:%318=79%:%
%:%319=79%:%
%:%320=80%:%
%:%321=80%:%
%:%322=80%:%
%:%323=80%:%
%:%324=81%:%
%:%325=81%:%
%:%326=81%:%
%:%327=81%:%
%:%328=81%:%
%:%329=82%:%
%:%330=82%:%
%:%331=82%:%
%:%332=82%:%
%:%333=83%:%
%:%339=83%:%
%:%342=84%:%
%:%343=85%:%
%:%344=85%:%
%:%345=86%:%
%:%352=87%:%
%:%353=87%:%
%:%354=88%:%
%:%355=88%:%
%:%356=89%:%
%:%357=89%:%
%:%358=89%:%
%:%359=90%:%
%:%360=90%:%
%:%361=91%:%
%:%362=91%:%
%:%363=91%:%
%:%364=92%:%
%:%365=92%:%
%:%366=93%:%
%:%367=93%:%
%:%368=93%:%
%:%369=94%:%
%:%375=94%:%
%:%378=95%:%
%:%379=96%:%
%:%380=96%:%
%:%381=97%:%
%:%384=98%:%
%:%388=98%:%
%:%389=98%:%
%:%390=98%:%
%:%395=98%:%
%:%398=99%:%
%:%399=100%:%
%:%400=100%:%
%:%401=101%:%
%:%402=102%:%
%:%409=103%:%
%:%410=103%:%
%:%411=104%:%
%:%412=104%:%
%:%413=104%:%
%:%414=105%:%
%:%415=105%:%
%:%416=106%:%
%:%417=106%:%
%:%418=106%:%
%:%419=107%:%
%:%420=107%:%
%:%421=108%:%
%:%422=108%:%
%:%423=109%:%
%:%424=109%:%
%:%425=110%:%
%:%426=110%:%
%:%427=111%:%
%:%428=111%:%
%:%429=111%:%
%:%430=112%:%
%:%431=112%:%
%:%432=113%:%
%:%438=113%:%
%:%441=114%:%
%:%442=115%:%
%:%443=115%:%
%:%444=116%:%
%:%445=117%:%
%:%452=118%:%
%:%453=118%:%
%:%454=119%:%
%:%455=119%:%
%:%456=119%:%
%:%457=120%:%
%:%458=120%:%
%:%459=121%:%
%:%460=121%:%
%:%461=121%:%
%:%462=122%:%
%:%463=122%:%
%:%464=123%:%
%:%465=123%:%
%:%466=124%:%
%:%467=124%:%
%:%468=125%:%
%:%469=125%:%
%:%470=126%:%
%:%471=126%:%
%:%472=126%:%
%:%473=126%:%
%:%474=127%:%
%:%480=127%:%
%:%483=128%:%
%:%484=129%:%
%:%485=129%:%
%:%486=130%:%
%:%487=131%:%
%:%488=131%:%
%:%489=132%:%
%:%490=133%:%
%:%491=134%:%
%:%498=135%:%
%:%499=135%:%
%:%500=136%:%
%:%501=136%:%
%:%502=136%:%
%:%503=137%:%
%:%504=137%:%
%:%505=137%:%
%:%506=138%:%
%:%507=138%:%
%:%508=139%:%
%:%509=139%:%
%:%510=139%:%
%:%511=140%:%
%:%512=140%:%
%:%513=140%:%
%:%514=140%:%
%:%515=141%:%
%:%516=141%:%
%:%517=141%:%
%:%518=141%:%
%:%519=142%:%
%:%520=142%:%
%:%521=142%:%
%:%522=142%:%
%:%523=142%:%
%:%524=143%:%
%:%525=143%:%
%:%526=143%:%
%:%527=143%:%
%:%528=143%:%
%:%529=144%:%
%:%530=144%:%
%:%531=145%:%
%:%532=145%:%
%:%533=145%:%
%:%534=146%:%
%:%535=146%:%
%:%536=146%:%
%:%537=146%:%
%:%538=147%:%
%:%539=147%:%
%:%544=147%:%
%:%547=148%:%
%:%548=149%:%
%:%549=149%:%
%:%550=150%:%
%:%551=151%:%
%:%558=152%:%
%:%559=152%:%
%:%560=153%:%
%:%561=153%:%
%:%562=153%:%
%:%563=154%:%
%:%564=154%:%
%:%565=154%:%
%:%566=155%:%
%:%567=155%:%
%:%568=155%:%
%:%569=155%:%
%:%570=156%:%
%:%576=156%:%
%:%579=157%:%
%:%580=158%:%
%:%581=158%:%
%:%584=159%:%
%:%588=159%:%
%:%589=159%:%
%:%590=159%:%
%:%604=161%:%
%:%614=163%:%
%:%615=163%:%
%:%616=164%:%
%:%617=165%:%
%:%618=166%:%
%:%619=166%:%
%:%620=167%:%
%:%621=168%:%
%:%624=169%:%
%:%628=169%:%
%:%629=169%:%
%:%630=170%:%
%:%631=170%:%
%:%632=171%:%
%:%633=171%:%
%:%634=172%:%
%:%639=172%:%
%:%642=173%:%
%:%643=174%:%
%:%644=174%:%
%:%645=175%:%
%:%646=176%:%
%:%649=177%:%
%:%653=177%:%
%:%654=177%:%
%:%655=177%:%
%:%660=177%:%
%:%663=178%:%
%:%664=179%:%
%:%665=179%:%
%:%666=180%:%
%:%667=181%:%
%:%674=182%:%
%:%675=182%:%
%:%676=183%:%
%:%677=183%:%
%:%678=183%:%
%:%679=184%:%
%:%680=184%:%
%:%681=185%:%
%:%682=185%:%
%:%683=185%:%
%:%684=186%:%
%:%685=186%:%
%:%686=187%:%
%:%687=187%:%
%:%688=188%:%
%:%689=188%:%
%:%690=189%:%
%:%691=189%:%
%:%692=190%:%
%:%693=190%:%
%:%694=190%:%
%:%695=190%:%
%:%696=191%:%
%:%702=191%:%
%:%705=192%:%
%:%706=193%:%
%:%707=193%:%
%:%708=194%:%
%:%709=195%:%
%:%716=196%:%
%:%717=196%:%
%:%718=197%:%
%:%719=197%:%
%:%720=197%:%
%:%721=198%:%
%:%722=198%:%
%:%723=199%:%
%:%724=199%:%
%:%725=199%:%
%:%726=200%:%
%:%727=200%:%
%:%728=201%:%
%:%729=201%:%
%:%730=202%:%
%:%731=202%:%
%:%732=203%:%
%:%733=203%:%
%:%734=204%:%
%:%735=204%:%
%:%736=204%:%
%:%737=204%:%
%:%738=205%:%
%:%744=205%:%
%:%747=206%:%
%:%748=207%:%
%:%749=207%:%
%:%750=208%:%
%:%751=209%:%
%:%758=210%:%
%:%759=210%:%
%:%760=211%:%
%:%761=211%:%
%:%762=212%:%
%:%763=212%:%
%:%764=212%:%
%:%765=213%:%
%:%766=213%:%
%:%767=214%:%
%:%768=214%:%
%:%769=214%:%
%:%770=215%:%
%:%771=215%:%
%:%772=216%:%
%:%773=216%:%
%:%778=216%:%
%:%781=217%:%
%:%782=218%:%
%:%783=218%:%
%:%784=219%:%
%:%785=220%:%
%:%792=221%:%
%:%793=221%:%
%:%794=222%:%
%:%795=222%:%
%:%796=223%:%
%:%797=223%:%
%:%798=223%:%
%:%799=224%:%
%:%800=224%:%
%:%801=225%:%
%:%802=225%:%
%:%803=225%:%
%:%804=226%:%
%:%805=226%:%
%:%806=227%:%
%:%807=227%:%
%:%812=227%:%
%:%815=228%:%
%:%816=229%:%
%:%817=229%:%
%:%818=230%:%
%:%821=231%:%
%:%825=231%:%
%:%826=231%:%
%:%827=231%:%
%:%828=231%:%
%:%833=231%:%
%:%836=232%:%
%:%837=233%:%
%:%838=233%:%
%:%839=234%:%
%:%840=235%:%
%:%847=236%:%
%:%848=236%:%
%:%849=237%:%
%:%850=237%:%
%:%851=237%:%
%:%852=238%:%
%:%853=238%:%
%:%854=239%:%
%:%855=239%:%
%:%856=239%:%
%:%857=240%:%
%:%858=240%:%
%:%859=240%:%
%:%860=240%:%
%:%861=240%:%
%:%862=241%:%
%:%863=241%:%
%:%864=241%:%
%:%865=241%:%
%:%866=242%:%
%:%867=242%:%
%:%868=243%:%
%:%869=243%:%
%:%870=243%:%
%:%871=244%:%
%:%877=244%:%
%:%880=245%:%
%:%881=246%:%
%:%882=246%:%
%:%883=247%:%
%:%890=248%:%
%:%891=248%:%
%:%892=249%:%
%:%893=249%:%
%:%894=249%:%
%:%895=250%:%
%:%896=250%:%
%:%897=250%:%
%:%898=251%:%
%:%904=251%:%
%:%907=252%:%
%:%908=253%:%
%:%909=253%:%
%:%910=254%:%
%:%911=255%:%
%:%912=255%:%
%:%913=256%:%
%:%914=257%:%
%:%917=258%:%
%:%921=258%:%
%:%922=258%:%
%:%923=258%:%
%:%929=258%:%
%:%932=259%:%
%:%933=260%:%
%:%934=260%:%
%:%935=261%:%
%:%936=262%:%
%:%937=263%:%
%:%944=264%:%
%:%945=264%:%
%:%946=265%:%
%:%947=265%:%
%:%948=266%:%
%:%949=266%:%
%:%950=266%:%
%:%951=267%:%
%:%952=267%:%
%:%953=268%:%
%:%954=268%:%
%:%955=268%:%
%:%956=269%:%
%:%957=269%:%
%:%958=270%:%
%:%959=270%:%
%:%964=270%:%
%:%967=271%:%
%:%968=272%:%
%:%969=272%:%
%:%970=273%:%
%:%971=274%:%
%:%972=275%:%
%:%979=276%:%
%:%980=276%:%
%:%981=277%:%
%:%982=277%:%
%:%983=278%:%
%:%984=278%:%
%:%985=278%:%
%:%986=279%:%
%:%987=279%:%
%:%988=279%:%
%:%989=280%:%
%:%990=280%:%
%:%991=281%:%
%:%992=281%:%
%:%993=281%:%
%:%994=282%:%
%:%995=282%:%
%:%996=282%:%
%:%997=282%:%
%:%998=283%:%
%:%999=283%:%
%:%1000=283%:%
%:%1001=283%:%
%:%1002=284%:%
%:%1003=284%:%
%:%1004=284%:%
%:%1005=284%:%
%:%1006=284%:%
%:%1007=285%:%
%:%1008=285%:%
%:%1009=285%:%
%:%1010=285%:%
%:%1011=285%:%
%:%1012=286%:%
%:%1013=286%:%
%:%1014=286%:%
%:%1015=286%:%
%:%1016=286%:%
%:%1017=287%:%
%:%1018=287%:%
%:%1019=288%:%
%:%1020=288%:%
%:%1021=288%:%
%:%1022=289%:%
%:%1023=289%:%
%:%1024=289%:%
%:%1025=289%:%
%:%1026=290%:%
%:%1027=290%:%
%:%1028=291%:%
%:%1029=291%:%
%:%1043=293%:%
%:%1053=295%:%
%:%1054=295%:%
%:%1055=296%:%
%:%1056=297%:%
%:%1057=298%:%
%:%1058=298%:%
%:%1059=299%:%
%:%1060=300%:%
%:%1063=301%:%
%:%1067=301%:%
%:%1068=301%:%
%:%1069=302%:%
%:%1070=302%:%
%:%1071=303%:%
%:%1072=303%:%
%:%1073=304%:%
%:%1078=304%:%
%:%1081=305%:%
%:%1082=306%:%
%:%1083=306%:%
%:%1084=307%:%
%:%1085=308%:%
%:%1088=309%:%
%:%1092=309%:%
%:%1093=309%:%
%:%1094=309%:%
%:%1099=309%:%
%:%1102=310%:%
%:%1103=311%:%
%:%1104=311%:%
%:%1105=312%:%
%:%1106=313%:%
%:%1113=314%:%
%:%1114=314%:%
%:%1115=315%:%
%:%1116=315%:%
%:%1117=315%:%
%:%1118=316%:%
%:%1119=316%:%
%:%1120=317%:%
%:%1121=317%:%
%:%1122=317%:%
%:%1123=318%:%
%:%1124=318%:%
%:%1125=319%:%
%:%1126=319%:%
%:%1127=320%:%
%:%1128=320%:%
%:%1129=321%:%
%:%1130=321%:%
%:%1131=322%:%
%:%1132=322%:%
%:%1133=322%:%
%:%1134=322%:%
%:%1135=323%:%
%:%1141=323%:%
%:%1144=324%:%
%:%1145=325%:%
%:%1146=325%:%
%:%1147=326%:%
%:%1148=327%:%
%:%1155=328%:%
%:%1156=328%:%
%:%1157=329%:%
%:%1158=329%:%
%:%1159=329%:%
%:%1160=330%:%
%:%1161=330%:%
%:%1162=331%:%
%:%1163=331%:%
%:%1164=331%:%
%:%1165=332%:%
%:%1166=332%:%
%:%1167=333%:%
%:%1168=333%:%
%:%1169=334%:%
%:%1170=334%:%
%:%1171=335%:%
%:%1172=335%:%
%:%1173=336%:%
%:%1174=336%:%
%:%1175=336%:%
%:%1176=336%:%
%:%1177=337%:%
%:%1183=337%:%
%:%1186=338%:%
%:%1187=339%:%
%:%1188=339%:%
%:%1189=340%:%
%:%1190=341%:%
%:%1197=342%:%
%:%1198=342%:%
%:%1199=343%:%
%:%1200=343%:%
%:%1201=344%:%
%:%1202=344%:%
%:%1203=344%:%
%:%1204=345%:%
%:%1205=345%:%
%:%1206=346%:%
%:%1207=346%:%
%:%1208=347%:%
%:%1209=347%:%
%:%1210=348%:%
%:%1211=348%:%
%:%1212=349%:%
%:%1213=349%:%
%:%1218=349%:%
%:%1221=350%:%
%:%1222=351%:%
%:%1223=351%:%
%:%1224=352%:%
%:%1225=353%:%
%:%1232=354%:%
%:%1233=354%:%
%:%1234=355%:%
%:%1235=355%:%
%:%1236=356%:%
%:%1237=356%:%
%:%1238=356%:%
%:%1239=357%:%
%:%1240=357%:%
%:%1241=358%:%
%:%1242=358%:%
%:%1243=359%:%
%:%1244=359%:%
%:%1245=360%:%
%:%1246=360%:%
%:%1247=361%:%
%:%1248=361%:%
%:%1253=361%:%
%:%1256=362%:%
%:%1257=363%:%
%:%1258=363%:%
%:%1259=364%:%
%:%1262=365%:%
%:%1266=365%:%
%:%1267=365%:%
%:%1268=365%:%
%:%1269=365%:%
%:%1274=365%:%
%:%1277=366%:%
%:%1278=367%:%
%:%1279=367%:%
%:%1280=368%:%
%:%1281=369%:%
%:%1288=370%:%
%:%1289=370%:%
%:%1290=371%:%
%:%1291=371%:%
%:%1292=371%:%
%:%1293=372%:%
%:%1294=372%:%
%:%1295=373%:%
%:%1296=373%:%
%:%1297=373%:%
%:%1298=374%:%
%:%1299=374%:%
%:%1300=374%:%
%:%1301=374%:%
%:%1302=374%:%
%:%1303=375%:%
%:%1304=375%:%
%:%1305=375%:%
%:%1306=375%:%
%:%1307=376%:%
%:%1308=376%:%
%:%1309=377%:%
%:%1310=377%:%
%:%1311=377%:%
%:%1312=378%:%
%:%1318=378%:%
%:%1321=379%:%
%:%1322=380%:%
%:%1323=380%:%
%:%1324=381%:%
%:%1331=382%:%
%:%1332=382%:%
%:%1333=383%:%
%:%1334=383%:%
%:%1335=383%:%
%:%1336=384%:%
%:%1337=384%:%
%:%1338=384%:%
%:%1339=385%:%
%:%1345=385%:%
%:%1348=386%:%
%:%1349=387%:%
%:%1350=387%:%
%:%1351=388%:%
%:%1352=389%:%
%:%1353=389%:%
%:%1354=390%:%
%:%1355=391%:%
%:%1358=392%:%
%:%1362=392%:%
%:%1363=392%:%
%:%1364=392%:%
%:%1370=392%:%
%:%1373=393%:%
%:%1374=394%:%
%:%1375=394%:%
%:%1376=395%:%
%:%1377=396%:%
%:%1378=397%:%
%:%1385=398%:%
%:%1386=398%:%
%:%1387=399%:%
%:%1388=399%:%
%:%1389=400%:%
%:%1390=400%:%
%:%1391=400%:%
%:%1392=401%:%
%:%1393=401%:%
%:%1394=402%:%
%:%1395=402%:%
%:%1396=402%:%
%:%1397=403%:%
%:%1398=403%:%
%:%1399=404%:%
%:%1400=404%:%
%:%1405=404%:%
%:%1408=405%:%
%:%1409=406%:%
%:%1410=406%:%
%:%1411=407%:%
%:%1412=408%:%
%:%1413=409%:%
%:%1420=410%:%
%:%1421=410%:%
%:%1422=411%:%
%:%1423=411%:%
%:%1424=411%:%
%:%1425=412%:%
%:%1426=412%:%
%:%1427=413%:%
%:%1428=413%:%
%:%1429=413%:%
%:%1430=414%:%
%:%1431=414%:%
%:%1432=415%:%
%:%1433=415%:%
%:%1434=415%:%
%:%1435=415%:%
%:%1436=416%:%
%:%1451=418%:%
%:%1461=420%:%
%:%1462=420%:%
%:%1463=421%:%
%:%1464=421%:%
%:%1465=422%:%
%:%1466=422%:%
%:%1467=423%:%
%:%1468=424%:%
%:%1469=424%:%
%:%1471=424%:%
%:%1475=424%:%
%:%1476=424%:%
%:%1483=424%:%
%:%1484=425%:%
%:%1485=425%:%
%:%1487=425%:%
%:%1491=425%:%
%:%1492=425%:%
%:%1499=425%:%
%:%1500=426%:%
%:%1501=426%:%
%:%1503=426%:%
%:%1507=426%:%
%:%1508=426%:%
%:%1522=428%:%
%:%1532=430%:%
%:%1533=430%:%
%:%1534=431%:%
%:%1535=432%:%
%:%1542=433%:%
%:%1543=433%:%
%:%1544=434%:%
%:%1545=434%:%
%:%1546=435%:%
%:%1547=435%:%
%:%1548=435%:%
%:%1549=435%:%
%:%1550=436%:%
%:%1551=436%:%
%:%1552=437%:%
%:%1553=437%:%
%:%1554=438%:%
%:%1555=438%:%
%:%1556=438%:%
%:%1557=439%:%
%:%1558=440%:%
%:%1559=440%:%
%:%1560=441%:%
%:%1566=441%:%
%:%1569=442%:%
%:%1570=443%:%
%:%1571=443%:%
%:%1572=444%:%
%:%1573=445%:%
%:%1574=446%:%
%:%1581=447%:%
%:%1582=447%:%
%:%1583=448%:%
%:%1584=448%:%
%:%1585=448%:%
%:%1586=449%:%
%:%1587=449%:%
%:%1588=449%:%
%:%1589=450%:%
%:%1590=450%:%
%:%1591=451%:%
%:%1592=451%:%
%:%1593=452%:%
%:%1594=452%:%
%:%1595=452%:%
%:%1596=452%:%
%:%1597=452%:%
%:%1598=453%:%
%:%1599=453%:%
%:%1600=454%:%
%:%1601=454%:%
%:%1602=455%:%
%:%1603=455%:%
%:%1604=455%:%
%:%1605=456%:%
%:%1606=456%:%
%:%1607=456%:%
%:%1608=456%:%
%:%1609=457%:%
%:%1610=457%:%
%:%1611=457%:%
%:%1612=457%:%
%:%1613=458%:%
%:%1614=458%:%
%:%1615=459%:%
%:%1616=459%:%
%:%1621=459%:%
%:%1624=460%:%
%:%1625=461%:%
%:%1626=461%:%
%:%1627=462%:%
%:%1628=463%:%
%:%1629=464%:%
%:%1636=465%:%
%:%1637=465%:%
%:%1638=466%:%
%:%1639=466%:%
%:%1640=466%:%
%:%1641=467%:%
%:%1642=467%:%
%:%1643=467%:%
%:%1644=468%:%
%:%1645=468%:%
%:%1646=469%:%
%:%1647=469%:%
%:%1648=470%:%
%:%1649=470%:%
%:%1650=470%:%
%:%1651=471%:%
%:%1652=471%:%
%:%1653=471%:%
%:%1654=471%:%
%:%1655=472%:%
%:%1656=472%:%
%:%1657=473%:%
%:%1658=473%:%
%:%1659=474%:%
%:%1660=474%:%
%:%1661=474%:%
%:%1662=474%:%
%:%1663=475%:%
%:%1664=475%:%
%:%1665=475%:%
%:%1666=475%:%
%:%1667=476%:%
%:%1668=476%:%
%:%1669=476%:%
%:%1670=476%:%
%:%1671=476%:%
%:%1672=477%:%
%:%1673=477%:%
%:%1674=477%:%
%:%1675=478%:%
%:%1676=478%:%
%:%1677=478%:%
%:%1678=478%:%
%:%1679=479%:%
%:%1680=479%:%
%:%1681=479%:%
%:%1682=479%:%
%:%1683=480%:%
%:%1684=480%:%
%:%1685=481%:%
%:%1686=481%:%
%:%1691=481%:%
%:%1694=482%:%
%:%1695=483%:%
%:%1696=483%:%
%:%1697=484%:%
%:%1698=485%:%
%:%1699=486%:%
%:%1706=487%:%
%:%1707=487%:%
%:%1708=488%:%
%:%1709=488%:%
%:%1710=489%:%
%:%1711=489%:%
%:%1712=489%:%
%:%1713=489%:%
%:%1714=490%:%
%:%1729=492%:%
%:%1739=494%:%
%:%1740=494%:%
%:%1741=495%:%
%:%1742=496%:%
%:%1749=497%:%
%:%1750=497%:%
%:%1751=498%:%
%:%1752=498%:%
%:%1753=499%:%
%:%1754=499%:%
%:%1755=499%:%
%:%1756=499%:%
%:%1757=500%:%
%:%1758=500%:%
%:%1759=501%:%
%:%1760=501%:%
%:%1761=502%:%
%:%1762=502%:%
%:%1763=502%:%
%:%1764=503%:%
%:%1765=504%:%
%:%1766=504%:%
%:%1767=505%:%
%:%1773=505%:%
%:%1776=506%:%
%:%1777=507%:%
%:%1778=507%:%
%:%1779=508%:%
%:%1780=509%:%
%:%1781=510%:%
%:%1788=511%:%
%:%1789=511%:%
%:%1790=512%:%
%:%1791=512%:%
%:%1792=512%:%
%:%1793=513%:%
%:%1794=513%:%
%:%1795=513%:%
%:%1796=514%:%
%:%1797=514%:%
%:%1798=515%:%
%:%1799=515%:%
%:%1800=516%:%
%:%1801=516%:%
%:%1802=516%:%
%:%1803=516%:%
%:%1804=516%:%
%:%1805=517%:%
%:%1806=517%:%
%:%1807=518%:%
%:%1808=518%:%
%:%1809=519%:%
%:%1810=519%:%
%:%1811=519%:%
%:%1812=520%:%
%:%1813=520%:%
%:%1814=520%:%
%:%1815=520%:%
%:%1816=521%:%
%:%1817=521%:%
%:%1818=521%:%
%:%1819=521%:%
%:%1820=522%:%
%:%1821=522%:%
%:%1822=523%:%
%:%1823=523%:%
%:%1828=523%:%
%:%1831=524%:%
%:%1832=525%:%
%:%1833=525%:%
%:%1834=526%:%
%:%1835=527%:%
%:%1836=528%:%
%:%1839=529%:%
%:%1843=529%:%
%:%1844=529%:%
%:%1845=530%:%
%:%1846=530%:%
%:%1847=531%:%
%:%1848=532%:%
%:%1849=533%:%
%:%1854=533%:%
%:%1857=534%:%
%:%1858=535%:%
%:%1859=535%:%
%:%1860=536%:%
%:%1861=537%:%
%:%1862=538%:%
%:%1869=539%:%
%:%1870=539%:%
%:%1871=540%:%
%:%1872=540%:%
%:%1873=541%:%
%:%1874=541%:%
%:%1875=541%:%
%:%1876=541%:%
%:%1877=542%:%
%:%1892=544%:%
%:%1902=546%:%
%:%1903=546%:%
%:%1904=547%:%
%:%1905=548%:%
%:%1912=549%:%
%:%1913=549%:%
%:%1914=550%:%
%:%1915=550%:%
%:%1916=551%:%
%:%1917=551%:%
%:%1918=551%:%
%:%1919=551%:%
%:%1920=552%:%
%:%1921=552%:%
%:%1922=553%:%
%:%1923=553%:%
%:%1924=554%:%
%:%1925=554%:%
%:%1926=554%:%
%:%1927=555%:%
%:%1928=556%:%
%:%1929=556%:%
%:%1930=557%:%
%:%1936=557%:%
%:%1939=558%:%
%:%1940=559%:%
%:%1941=559%:%
%:%1942=560%:%
%:%1943=561%:%
%:%1944=562%:%
%:%1951=563%:%
%:%1952=563%:%
%:%1953=564%:%
%:%1954=564%:%
%:%1955=564%:%
%:%1956=565%:%
%:%1957=565%:%
%:%1958=566%:%
%:%1959=566%:%
%:%1960=566%:%
%:%1961=567%:%
%:%1962=567%:%
%:%1963=568%:%
%:%1964=568%:%
%:%1965=568%:%
%:%1966=568%:%
%:%1967=569%:%
%:%1973=569%:%
%:%1976=570%:%
%:%1977=571%:%
%:%1978=571%:%
%:%1979=572%:%
%:%1980=573%:%
%:%1981=574%:%
%:%1988=575%:%
%:%1989=575%:%
%:%1990=576%:%
%:%1991=576%:%
%:%1992=576%:%
%:%1993=577%:%
%:%1994=577%:%
%:%1995=577%:%
%:%1996=577%:%
%:%1997=578%:%
%:%1998=578%:%
%:%1999=578%:%
%:%2000=579%:%
%:%2001=579%:%
%:%2002=579%:%
%:%2003=579%:%
%:%2004=580%:%
%:%2010=580%:%
%:%2013=581%:%
%:%2014=582%:%
%:%2015=582%:%
%:%2016=583%:%
%:%2017=584%:%
%:%2018=585%:%
%:%2025=586%:%
%:%2026=586%:%
%:%2027=587%:%
%:%2028=587%:%
%:%2029=588%:%
%:%2030=588%:%
%:%2031=588%:%
%:%2032=588%:%
%:%2033=589%:%
%:%2039=589%:%
%:%2044=590%:%
%:%2049=591%:%



\bibliographystyle{abbrv}
\bibliography{root}

\end{document}
