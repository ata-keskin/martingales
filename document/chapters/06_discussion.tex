% !TeX root = ../main.tex
% Add the above to each chapter to make compiling the PDF easier in some editors.

\chapter{Discussion}\label{chapter:discussion}

\section{Formalization Approach}

A convex function of a martingale is a submartingale, by Jensen's inequality. For example, the square of the gambler's fortune in the fair coin game is a submartingale (which also follows from the fact that Xn2 - n is a martingale). Similarly, a concave function of a martingale is a supermartingale. Stopped Processes... Wiener Processes

\section{Comparison with Existing Formalizations}

The following tables provide a list of the entries in the \textsf{mathlib} formalization of martingales, all of which have counterparts in our formalization.

{\small
\begin{longtable}{p{.5\textwidth} p{.5\textwidth}}
	\hline
	\textsf{Lean} & \textsf{Isabelle} \\ \hline
	\texttt{martingale} & \texttt{martingale (locale)}  \\
	\texttt{martingale.adapted} & \texttt{adapted\_process.adapted}  \\
	\texttt{martingale.add} & \texttt{martingale.add}  \\
	\texttt{martingale.condexp\_ae\_eq} & \texttt{martingale.martingale\_property}  \\
	\texttt{martingale.eq\_zero\_of\_predictable} & \texttt{martingale.predictable\_eq\_zero}  \\
	\texttt{martingale.integrable} & \texttt{martingale.integrable}  \\
	\texttt{martingale.neg} & \texttt{martingale.uminus}  \\
	\texttt{martingale.set\_integral\_eq} & \texttt{martingale.set\_integral\_eq}  \\
	\texttt{martingale.smul} & \texttt{martingale.scaleR}  \\
	\texttt{martingale.strongly\_measurable} & \texttt{stochastic\_process.random\_variable}  \\
	\texttt{martingale.sub} & \texttt{martingale.diff}  \\
	\texttt{martingale.submartingale} & \textsf{via sublocale relation}  \\
	\texttt{martingale.supermartingale} & \textsf{via sublocale relation}  \\
	\texttt{martingale\_condexp} & \texttt{sigma\_finite\_filtered\_measure.martingale\_cond\_exp}  \\
	\texttt{martingale\_const} & \texttt{finite\_filtered\_measure.martingale\_const}  \\
	\texttt{martingale\_const\_fun} & \texttt{sigma\_finite\_filtered\_measure.martingale\_const}  \\
	\texttt{martingale\_iff} & \texttt{martingale\_iff}  \\
	\texttt{martingale\_nat} & \texttt{nat\_sigma\_finite\_adapted\_process.martingale\_nat}  \\
	\texttt{martingale\_of\_condexp\_sub\_eq\_zero\_nat} & \texttt{nat\_sigma\_finite\_adapted\_process.martingale\_of \_cond\_exp\_diff\_Suc\_eq\_zero}  \\
	\texttt{martingale\_of\_set\_integral\_eq\_succ} & \texttt{nat\_sigma\_finite\_adapted\_process.martingale\_of \_set\_integral\_eq\_Suc}  \\
	\texttt{martingale\_zero} & \texttt{sigma\_finite\_filtered\_measure.martingale\_zero} \\
	\caption[Lookup Table for Martingale Lemmas and Definitions]{Lookup table for martingale lemmas and definitions}\label{tab:martingale_theories}
\end{longtable}
\begin{longtable}{p{.5\textwidth} p{.5\textwidth}}
	\hline
	\textsf{Lean} & \textsf{Isabelle} \\ \hline
	\texttt{submartingale} & \texttt{submartingale (locale)}  \\
	\texttt{submartingale.adapted} & \texttt{adapted\_process.adapted}  \\
	\texttt{submartingale.add} & \texttt{submartingale.add}  \\
	\texttt{submartingale.add\_martingale} & \texttt{submartingale.add}  \\
	\texttt{submartingale.ae\_le\_condexp} & \texttt{submartingale\_property}  \\
	\texttt{submartingale.condexp\_sub\_nonneg} & \texttt{submartingale.cond\_exp\_diff\_nonneg}  \\
	\texttt{submartingale.integrable} & \texttt{submartingale.integrable}  \\
	\texttt{submartingale.neg} & \texttt{submartingale.uminus}  \\
	\texttt{submartingale.pos} & \texttt{submartingale.max\_0}  \\
	\texttt{submartingale.set\_integral\_le} & \texttt{submartingale\_linorder.set\_integral\_le}  \\
	\texttt{submartingale.smul\_nonneg} & \texttt{submartingale.scaleR\_nonneg}  \\
	\texttt{submartingale.smul\_nonpos} & \texttt{submartingale.scaleR\_nonpos}  \\
	\texttt{submartingale.strongly\_measurable} & \texttt{stochastic\_process.random\_variable}  \\
	\texttt{submartingale.sub\_martingale} & \texttt{submartingale.diff}  \\
	\texttt{submartingale.sub\_supermartingale} & \texttt{submartingale.diff}  \\
	\texttt{submartingale.sum\_mul\_sub} & \texttt{nat\_submartingale.partial\_sum\_scaleR}  \\
	\texttt{submartingale.sum\_mul\_sub'} & \texttt{nat\_submartingale.partial\_sum\_scaleR'}  \\
	\texttt{submartingale.sup} & \texttt{submartingale\_linorder.max}  \\
	\texttt{submartingale.zero\_le\_of\_predictable} & \texttt{nat\_submartingale.predictable\_ge\_bot}  \\
	\texttt{submartingale\_nat} & \texttt{nat\_sigma\_finite\_adapted\_process\_linorder .submartingale\_nat}  \\
	\texttt{submartingale\_of\_condexp\_sub\_nonneg} & \texttt{sigma\_finite\_adapted\_process\_order .submartingale\_of\_cond\_exp\_diff\_nonneg}  \\
	\texttt{submartingale\_of\_condexp\_sub\_nonneg\_nat} & \texttt{nat\_sigma\_finite\_adapted\_process\_linorder .submartingale\_of\_cond\_exp\_diff\_Suc\_nonneg}  \\
	\texttt{submartingale\_of\_set\_integral\_le} & \texttt{sigma\_finite\_adapted\_process\_linorder .submartingale\_of\_set\_integral\_le}  \\
	\texttt{submartingale\_of\_set\_integral\_le\_succ} & \texttt{nat\_sigma\_finite\_adapted\_process\_linorder .submartingale\_of\_set\_integral\_le\_Suc} \\
	\caption[Lookup Table for Submartingale Lemmas and Definitions]{Lookup table for submartingale lemmas and definitions}\label{tab:submartingale_theories}
\end{longtable}
\begin{longtable}{p{.5\textwidth} p{.5\textwidth}}
	\hline
	\textsf{Lean} & \textsf{Isabelle} \\ \hline
	\texttt{supermartingale} & \texttt{supermartingale (locale)}  \\
	\texttt{supermartingale.adapted} & \texttt{adapted\_process.adapted}  \\
	\texttt{supermartingale.add} & \texttt{supermartingale.add}  \\
	\texttt{supermartingale.add\_martingale} & \texttt{supermartingale.add}  \\
	\texttt{supermartingale.condexp\_ae\_le} & \texttt{supermartingale\_property}  \\
	\texttt{supermartingale.integrable} & \texttt{supermartingale.integrable}  \\
	\texttt{supermartingale.le\_zero\_of\_predictable} & \texttt{supermartingale.predictable\_le\_zero}  \\
	\texttt{supermartingale.neg} & \texttt{supermartingale.uminus}  \\
	\texttt{supermartingale.set\_integral\_le} & \texttt{supermartingale\_linorder.set\_integral\_ge}  \\
	\texttt{supermartingale.smul\_nonneg} & \texttt{supermartingale.scaleR\_nonneg}  \\
	\texttt{supermartingale.smul\_nonpos} & \texttt{supermartingale.scaleR\_nonpos}  \\
	\texttt{supermartingale.strongly\_measurable} & \texttt{stochastic\_process.random\_variable}  \\
	\texttt{supermartingale.sub\_martingale} & \texttt{supermartingale.diff}  \\
	\texttt{supermartingale.sub\_submartingale} & \texttt{supermartingale.diff}  \\
	\texttt{supermartingale\_nat} & \texttt{nat\_sigma\_finite\_adapted\_process\_linorder .supermartingale\_nat}  \\
	\texttt{supermartingale\_of\_condexp\_sub\_nonneg\_nat} & \texttt{nat\_sigma\_finite\_adapted\_process\_linorder .supermartingale\_of\_cond\_exp\_diff\_Suc\_nonneg}  \\
	\texttt{supermartingale\_of\_set\_integral\_succ\_le} & \texttt{nat\_sigma\_finite\_adapted\_process\_linorder .supermartingale\_of\_set\_integral\_le\_Suc} \\
	\caption[Lookup Table for Supermartingale Lemmas and Definitions]{Lookup table for supermartingale lemmas and definitions}\label{tab:supermartingale_theories}
\end{longtable}
}
\section{Challenges and Limitations}

\section{Future Research}

\paragraph{Semimartingales}

\paragraph{Doob's Martingale Convergence}

\paragraph{Fundamental Theorem of Arbitrage} 

The fundamental theorem of asset pricing relates the concept of a fair market price for a financial asset to the notion of a risk-neutral measure. 

It provides the necessary and sufficient conditions for a market to be arbitrage-free. 

In this framework, the prices of financial assets can be treated as martingales, ensuring that there is no arbitrage opportunity.
