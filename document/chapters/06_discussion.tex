% !TeX root = ../main.tex
% Add the above to each chapter to make compiling the PDF easier in some editors.

\chapter{Discussion}\label{chapter:discussion}

In this chapter, we discuss the decisions we took throughout the formalization process. Furthermore, we compare our work with the existing formalization of martingales in Lean, and outline directions for future research.

While constructing the conditional expectation operator, we decided to first create a predicate \texttt{has\_cond\_exp} which characterizes the conditional expectation, then define the actual operator using the Hilbert's choice function. This has a couple of advantages. First and foremost, it allows us to state many of the properties of the conditional expectation operator without first needing to show that it actually exists. From a mathematical perspective, this doesn't change anything in a major way. However in practice, it makes the formalization much easier. Instinctively, we could've first defined the conditional expectation operator for simple functions, then extended it to general functions, and finally shown that the definitions coincide for simple functions. Unfortunately, this wouldn't be so straightforward. The conditional expectation is unique up to a $\mu$-null set, i.e. it is an element of $L^1(E)$. For simple functions it is easy to choose a canonical representative. For arbitrary integrable functions, this would be quite cumbersome. Additionally, it would not bring any advantages, since it is already only unique as an element of $L^1(E)$. 

To actually show that the conditional expectation of a function $f \in L^1(E)$ exists we have come up with the following approach. 

We first constructed the conditional expectation explicitly for simple functions. Then we showed that the conditional expectation is a contraction on simple functions, i.e. $\lVert \mathbb{E}(s \vert F)(x) \rVert \le \mathbb{E}(\lVert s(x) \rVert \vert F)$ for $\mu$-almost all $x \in \Omega$ with $s : \Omega \rightarrow E$ simple and integrable. Using this, we were able to show that the conditional expectation of a convergent sequence of simple functions is again convergent. Finally, we showed that this limit exhibits the properties of a conditional expectation. This approach has the benefit of being straightforward and easy to implement, since we could make use of the existing formalization for real-valued functions. However, there are other ways to construct the conditional expectation. The following alternative method is described in detail in \cite{Hytoenen_2016}.

One first shows that the conditional expectation exists for functions in $f \in L^2(\mathbb{R})$. Then one uses the fact that functions in $L^1(\mathbb{R})$ can be approximated by functions in $L^1(\mathbb{R}) \cap L^\infty(\mathbb{R})$ to obtain a conditional expectation operator $L^1(\mathbb{R})\rightarrow L^1(\mathbb{R})$. Then, one shows that this operator is bounded and positive. Hence, it can be extended to a bounded operator $L^1(E)\rightarrow L^1(E)$, which retains the properties of a conditional expectation. In more detail, one argues as follows. 

Let $F \subset \Sigma$, be a sub-$\sigma$-algebra. First, one shows that the subspace 
\[
	L^2(\mathbb{R}; F) := \{f \in L^2(\mathbb{R}) \;\vert\; f \; \textrm{is} \; F\textrm{-measurable}\}
\] is a closed and convex subset of $L^2(\mathbb{R})$. Using the Hilbert projection theorem, we obtain a projection $P : L^2(\mathbb{R}) \rightarrow L^2(\mathbb{R}; F)$. We then verify that the projected function $Pf$ satisfies the properties of a conditional expectation using the fact that projections are self-adjoint. 

Next, we show that the conditional expectation is contractive with respect to the $L^1$-norm. We know that $L^1(\mathbb{R}) \cap L^\infty(\mathbb{R}) \subseteq L^2(\mathbb{R})$ and $L^1(\mathbb{R}) \cap L^\infty(\mathbb{R})$ is dense in $L^1(\mathbb{R})$. Therefore, we can extend the conditional expectation operator, which is currently only defined for functions in $L^2(\mathbb{R})$ to a contraction $L^1(\mathbb{R}) \rightarrow L^1(\mathbb{R}; F)$. Then, it is straightforward to verify that this operator is indeed the conditional expectation. Finally, one shows that this operator is positive and extends it to a bounded operator $L^1(E)\rightarrow L^1(E; F)$ which still has the properties of a conditional expectation.

This is an elegant way of showing that the conditional expectation exists as a bounded operator on $L^1(E)$. Furthermore, one can easily extend this definition to functions in $L^p(E)$. Our approach also uses similar arguments; it is based on showing that the conditional expectation is a contraction on the dense subset of $L^1(E)$ generated by integrable simple functions. 

Another reason why we did not employ this approach is because the only formalization of the Hilbert projection theorem in Isabelle/HOL is for complex vector spaces. Therefore we decided to take a simpler approach and construct the conditional expectation using mostly measure theoretical arguments. This also makes the proofs more accesible in our opinion.

One of the short-comings of our formalization is how lemmas concerning ordered Banach spaces are developed. In many stages, we require that the ordering on the Banach spaces be linear. Otherwise, it doesn't necessarily follow that the sets $[a,\infty) = \{x \in E\;\vert\; a \le x\}$ and $(\infty, a] = \{x \in E\;\vert\; x \le a\}$ are closed. This is cruicial in many stages of our formalization. 

With this in mind, there are weaker restrictions we can place on the ordering that allow us to obtain the same results. A Banach space $(E, \lVert \cdot \rVert)$ equipped with a lattice ordering is called a ``Banach lattice'', if for any $a, b\in E$ the following implication holds.
\[
	\lvert a \rvert \le \lvert b \rvert \implies \lVert a \rVert \le \lVert b \rVert
\]
where $\lvert x \rvert := x \vee -x$. Under this weaker assumption, the aforementioned sets are still closed and one can still show the monotonicity results concerning the conditional expectation. This is actually how the formalization is done on \textsf{mathlib}. Even though it would be great to introduce Banach lattices and show these results in this more general setting, we didn't have the time or the mental resources necessary to undertake this endeavour. This is a subject we might will explore in future projects.

Our main motivation for this project was to port the \textsf{mathlib} formalization on martingales to Isabelle/HOL. We have fully accomplished this goal. In the appendices section, we present a tabular overview of the results in the \textsf{mathlib} document \texttt{probability.mar\-tingale.basic} with their counterparts in our formalization. We have aimed to state all our results with at least the same level of generality as their \textsf{mathlib} counterparts. Furthermore, we were able to restate all results which contained the real numbers with an arbitrary linearly ordered Banach space instead. In the future, we aim to further weaken the assumption to only require Banach lattices.

As briefly stated at the end of the last chapter, we have primarily focused on laying the groundwork for formalizing martingales in arbitrary Banach spaces, with specific emphasis on extending the conditional expectation operator and generalizing specific concepts from Bochner integration. Building upon our formalization framework, the natural next step is to delve into the formalization of key martingale theorems and properties, such as the martingale convergence theorem, the optional stopping theorem, and the Doob decomposition, among others. 

Another direction for exploration is the further development of the theory of stochastic processes introduced in this work. This will lead us into more advanced territories, such as stochastic differential equations (SDEs), It\^o calculus, and the theory of semimartingales. It\^o calculus is used to rigorously define and solve SDEs. SDEs describe how quantities evolve over time when influenced by both deterministic trends and random fluctuations. This is particularly valuable in finance for modeling asset prices, interest rates, and other financial variables that exhibit inherent uncertainty. The ability to work with SDEs allows researchers to develop sophisticated pricing models for financial derivatives, assess risk accurately, and optimize investment strategies.

