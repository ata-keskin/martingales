% !TeX root = ../main.tex
% Add the above to each chapter to make compiling the PDF easier in some editors.

\chapter{Conclusion}\label{chapter:conclusion}

This thesis has been dedicated to the formalization of martingales in arbitrary Banach spaces, using the proof assistant Isabelle. Our central objective was to provide a rigorous foundation for this endeavor and expand upon existing formalizations. As we conclude our work, we reflect upon the contributions we have made.

A major achievement of our work is the extension of the conditional expectation operator from the familiar real-valued setting to the broader context of Banach spaces. This generalization allows for a more versatile and comprehensive approach to modeling stochastic processes, accommodating a wider range of applications. Furthermore, we have lifted many of the commonly used properties of the conditional expectation to this more general setting. We have introduced locale definitions to characterize various types of stochastic processes and filtered measure spaces, which was essential for the development of martingale theory in Banach spaces. We have introduced corresponding locales for discrete-time and continuous-time processes. Additionally, we have characterized predictable processes in the discrete-time setting, using proof methods from measure theory.

Finally, we have introduced suitable locales for martingales, demonstrated their basic properties, and presented alternative characterizations in the discrete-time setting. As we look ahead, we face several potential directions for future formalization endeavors, such as martingale convergence theorems, It\^o calculus and the theory of semimartingales. By providing a robust framework for martingales and related concepts in Banach spaces, we hope to facilitate further exploration and development in this field. Prior to our work, there was no development of martingales in a general Banach space setting within the \textsf{AFP}. With our contributions, this gap has been addressed, offering a solid starting point for further formalizations within the theory of stochastic processes.