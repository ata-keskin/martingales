% !TeX root = ../main.tex
% Add the above to each chapter to make compiling the PDF easier in some editors.

\chapter{Introduction}\label{chapter:introduction}

Martingales hold a central position in the theory of stochastic processes, making them a fundamental concept for the working mathematician. They provide a powerful way to study and analyze random phenomena, offering a formal framework for understanding the behavior of random variables over time. In various real-world scenarios, we encounter systems that evolve randomly over time. Representing such systems as martingales, we are able to investigate whether these systems remain bounded or converge to certain values in the long run.

In finance and economics, martingales are an invaluable tool for modeling asset prices and option pricing. They provide insights into risk assessment, portfolio management, and the efficient market hypothesis, which postulates that asset prices fully reflect all available information. Moreover, martingales play a crucial role in the analysis of fair gambling games and betting strategies. 

Martingales are also closely related to several important probability limit theorems. These theorems, such as the strong law of large numbers and the central limit theorem, formalize the asymptotic behavior of sample means and sums of random variables. They have profound implications in statistics, allowing us to draw conclusions about large datasets and make predictions based on limited information.

In addition to their relevance in mathematics, martingales find applications in various interdisciplinary fields. Their ability to model randomness and analyze dynamic systems makes them useful in physics, biology, and computer science, among others.

In the scope of this thesis, we present a formalization of martingales in arbitrary Banach spaces using Isabelle/HOL. The background and related work section examines existing formalizations in two prominent formal proof repositories, \textsf{mathlib} (which uses the Lean theorem prover) and the \textsf{\ac{AFP}} (which uses Isabelle). Additionally, we conduct a short review of literature on conditional expectation and martingales in Banach spaces, laying a solid foundation for our research.

The current formalization of conditional expectation in the Isabelle library is limited to real-valued functions. To overcome this limitation, we extend the construction of conditional expectation to general Banach spaces, employing an approach similar to the one in \cite{Hytoenen_2016}. We justify our approach, by comparing it to two alternative constructions of the conditional expectation.

Subsequently, we define stochastic processes and introduce the concepts of adapted, progressively measurable and predictable processes using suitable locale definitions. Most importantly, we provide a generalization for the already present locale \lstinline{filtration} by introducing the locales \lstinline{filtered_measure} and \lstinline{filtered_sigma_finite_measure}. These locales serve to formalize the concept of a filtered measure space. The latter also serves to generalize the locale \lstinline{sigma_finite_subalgebra} which is necessary for the development of the theory of martingales. 

Moving forward, we rigorously define martingales, submartingales, and supermartingales, presenting their first consequences and corollaries. Discrete and continuous time martingales are also covered in the formalization, benefiting from the complex and powerful locale system of Isabelle.

Our formalization fully encompasses the introductory \textsf{mathlib} theory on martingales and offers more generalization.

The thesis further contributes by generalizing concepts in Bochner integration, extending their application to general Banach spaces. Induction schemes for simple, integrable, and Borel measurable functions on Banach spaces are introduced, accommodating scenarios with or without a real vector ordering. These amendments expand the applicability of Bochner integration techniques.

The thesis concludes with reflections on the formalization approach, encountered challenges, and suggests future research directions.