% !TeX root = ../main.tex
% Add the above to each chapter to make compiling the PDF easier in some editors.

\chapter{Introduction}\label{chapter:introduction}

Martingales hold a central position in the theory of stochastic processes, making them a fundamental concept for the working mathematician. They provide a powerful way to study and analyze random phenomena, offering a mathematical framework for understanding their behavior. 

In various real-world scenarios, we encounter systems that evolve randomly over time, which can be modeled using martingales. In finance and economics, martingales are an invaluable tool for modeling asset prices \cite{fama1965} and option pricing \cite{Musiela_Rutkowski_2005}. They provide insight into risk assessment, portfolio management, and the efficient market hypothesis, which postulates that asset prices fully reflect all available information \cite{yaes1989}. 

Martingales are also closely related to several important limit theorems in probability theory. These theorems, such as the strong law of large numbers and the central limit theorem, formalize the asymptotic behavior of sample means and sums of random variables. They have profound implications in statistics, allowing us to draw conclusions about large datasets and make predictions based on limited information. Martingale theory allows us to investigate whether these systems remain bounded or converge to certain values in the long run.

In addition to their relevance in mathematics, martingales find applications in various interdisciplinary fields. Their ability to model randomness and analyze dynamic systems makes them useful in physics \cite{roldan2023}, biology, and computer science \cite{mitzenmacher_upfal_2005}, among others.

In the scope of this thesis, we present a formalization of martingales in arbitrary Banach spaces using Isabelle/HOL \cite{Keskin_A_Formalization_of_2023}. We start our discourse by examining existing formalizations in two prominent formal proof repositories, the Lean Mathematical Library (\textsf{mathlib}) and the Archive of Formal Proofs (\textsf{AFP}). Afterwards, we go over some of the basic concepts concerning the theory of integration in Banach spaces, laying a solid foundation for our research.

The current formalization of conditional expectation in the Isabelle library is limited to real-valued functions. To overcome this limitation, we extend the construction of conditional expectation to general Banach spaces, employing an approach similar to the one described in \cite{Hytoenen_2016}. We use measure theoretic arguments to construct the conditional expectation using simple functions and limiting arguments. We compare our construction\footnote{\texttt{Martingale.Conditional\_Expectation\_Banach}} with the approach in \cite{Hytoenen_2016} in Chap. \ref{chapter:discussion}.

Subsequently, we define stochastic processes and introduce the concepts of adapted, progressively measurable and predictable processes using suitable locale definitions\footnote{\texttt{Martingale.Stochastic\_Process}}. We pay special attention to predictable processes in discrete-time, providing a characterization using adapted processes. Moving forward, we rigorously define martingales, submartingales, and supermartingales, presenting their first consequences and corollaries\footnote{\texttt{Martingale.Martingale}}. Discrete-time martingales are given special attention in the formalization. In every step of our formalization, we make extensive use of the powerful locale system of Isabelle.
Our formalization fully encompasses the introductory \textsf{mathlib} theory \texttt{probability.mar\-tingale.basic} on martingales \cite{Degenne_Ying_2022}, even offering more generalization at certain stages.

Our thesis further contributes by generalizing concepts in Bochner integration by extending their application from the real numbers to arbitrary Banach spaces equipped with a second-countable topology. Induction schemes for integrable simple functions on Banach spaces are introduced, accommodating various scenarios with or without a real vector ordering\footnote{\texttt{Martingale.Bochner\_Integration\_Addendum}}. These amendments expand the applicability of Bochner integration techniques.
We conclude our thesis with reflections on the formalization approach and suggestions for future research directions.