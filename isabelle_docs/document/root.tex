\documentclass[11pt,a4paper]{article}
\usepackage[T1]{fontenc}
\usepackage{amsmath}
\usepackage{amssymb}
\usepackage{isabelle,isabellesym}

% further packages required for unusual symbols (see also
% isabellesym.sty), use only when needed

%\usepackage{amssymb}
  %for \<leadsto>, \<box>, \<diamond>, \<sqsupset>, \<mho>, \<Join>,
  %\<lhd>, \<lesssim>, \<greatersim>, \<lessapprox>, \<greaterapprox>,
  %\<triangleq>, \<yen>, \<lozenge>

%\usepackage{eurosym}
  %for \<euro>

%\usepackage[only,bigsqcap,bigparallel,fatsemi,interleave,sslash]{stmaryrd}
  %for \<Sqinter>, \<Parallel>, \<Zsemi>, \<Parallel>, \<sslash>

%\usepackage{eufrak}
  %for \<AA> ... \<ZZ>, \<aa> ... \<zz> (also included in amssymb)

%\usepackage{textcomp}
  %for \<onequarter>, \<onehalf>, \<threequarters>, \<degree>, \<cent>,
  %\<currency>

% this should be the last package used
\usepackage{pdfsetup}

% urls in roman style, theory text in math-similar italics
\urlstyle{rm}
\isabellestyle{it}

% for uniform font size
%\renewcommand{\isastyle}{\isastyleminor}

\begin{document}

\title{On the Formalization of Martingales}
\author{Ata Keskin}
\maketitle
\begin{abstract}
In the scope of this project, we present a formalization of martingales in arbitrary Banach spaces using Isabelle/HOL \cite{Keskin_A_Formalization_of_2023}.

The current formalization of conditional expectation in the Isabelle library is limited to real-valued functions. To overcome this limitation, we extend the construction of conditional expectation to general Banach spaces, employing an approach similar to the one described in \cite{Hytoenen_2016}. We use measure theoretic arguments to construct the conditional expectation using suitable limits of simple functions.

Subsequently, we define stochastic processes and introduce the concepts of adapted, progressively measurable and predictable processes using suitable locale definitions\footnote{\texttt{Martingale.Stochastic\_Process} in \cite{Keskin_A_Formalization_of_2023}}. We show the relation
\[
	\text{adapted} \supseteq \text{progressive} \supseteq \text{predictable}
\]
Furthermore, we show that progressive measurability and adaptedness are equivalent when the indexing set is discrete.
We pay special attention to predictable processes in discrete-time, showing that $(X_n)_{n \in \mathbb{N}}$ is predictable if and only if $(X_{n + 1})_{n \in \mathbb{N}}$ is adapted.

Moving forward, we rigorously define martingales, submartingales, and supermartingales, presenting their first consequences and corollaries\footnote{\texttt{Martingale.Martingale} in \cite{Keskin_A_Formalization_of_2023}}. Discrete-time martingales are given special attention in the formalization. In every step of our formalization, we make extensive use of the powerful locale system of Isabelle.

The formalization further contributes by generalizing concepts in Bochner integration by extending their application from the real numbers to arbitrary Banach spaces equipped with a second-countable topology. Induction schemes for integrable simple functions on Banach spaces are introduced, accommodating various scenarios with or without a real vector ordering\footnote{\texttt{Martingale.Bochner\_Integration\_Addendum} in \cite{Keskin_A_Formalization_of_2023}}. Specifically, we formalize a powerful result called the ``Averaging Theorem''\cite{Lang_1993} which allows us to show that densities are unique in Banach spaces.

In depth information on the formalization and the proofs of the individual theorems can be found in \cite{keskin2023formalization}.
\end{abstract}
\tableofcontents
\pagebreak

% sane default for proof documents
\parindent 0pt\parskip 0.5ex

% generated text of all theories
%
\begin{isabellebody}%
\setisabellecontext{Measure{\isacharunderscore}{\kern0pt}Space{\isacharunderscore}{\kern0pt}Addendum}%
%
\isadelimtheory
%
\endisadelimtheory
%
\isatagtheory
\isacommand{theory}\isamarkupfalse%
\ Measure{\isacharunderscore}{\kern0pt}Space{\isacharunderscore}{\kern0pt}Addendum\isanewline
\ \ \isakeyword{imports}\ {\isachardoublequoteopen}HOL{\isacharminus}{\kern0pt}Analysis{\isachardot}{\kern0pt}Measure{\isacharunderscore}{\kern0pt}Space{\isachardoublequoteclose}\isanewline
\isakeyword{begin}%
\endisatagtheory
{\isafoldtheory}%
%
\isadelimtheory
%
\endisadelimtheory
%
\isadelimdocument
%
\endisadelimdocument
%
\isatagdocument
%
\isamarkupsubsection{Sigma Algebra Generated by a Family of Functions%
}
\isamarkuptrue%
%
\endisatagdocument
{\isafolddocument}%
%
\isadelimdocument
%
\endisadelimdocument
\isacommand{definition}\isamarkupfalse%
\ sigma{\isacharunderscore}{\kern0pt}gen\ {\isacharcolon}{\kern0pt}{\isacharcolon}{\kern0pt}\ {\isachardoublequoteopen}{\isacharprime}{\kern0pt}a\ set\ {\isasymRightarrow}\ {\isacharprime}{\kern0pt}b\ measure\ {\isasymRightarrow}\ {\isacharparenleft}{\kern0pt}{\isacharprime}{\kern0pt}a\ {\isasymRightarrow}\ {\isacharprime}{\kern0pt}b{\isacharparenright}{\kern0pt}\ set\ {\isasymRightarrow}\ {\isacharprime}{\kern0pt}a\ measure{\isachardoublequoteclose}\ \isakeyword{where}\isanewline
\ \ {\isachardoublequoteopen}sigma{\isacharunderscore}{\kern0pt}gen\ {\isasymOmega}\ N\ S\ {\isasymequiv}\ sigma\ {\isasymOmega}\ {\isacharparenleft}{\kern0pt}{\isasymUnion}f\ {\isasymin}\ S{\isachardot}{\kern0pt}\ {\isacharbraceleft}{\kern0pt}f\ {\isacharminus}{\kern0pt}{\isacharbackquote}{\kern0pt}\ A\ {\isasyminter}\ {\isasymOmega}\ {\isacharbar}{\kern0pt}\ A{\isachardot}{\kern0pt}\ A\ {\isasymin}\ N{\isacharbraceright}{\kern0pt}{\isacharparenright}{\kern0pt}{\isachardoublequoteclose}\isanewline
\isanewline
\isacommand{lemma}\isamarkupfalse%
\ {\isacharbrackleft}{\kern0pt}simp{\isacharbrackright}{\kern0pt}{\isacharcolon}{\kern0pt}\isanewline
\ \ \isakeyword{shows}\ sets{\isacharunderscore}{\kern0pt}sigma{\isacharunderscore}{\kern0pt}gen{\isacharcolon}{\kern0pt}\ {\isachardoublequoteopen}sets\ {\isacharparenleft}{\kern0pt}sigma{\isacharunderscore}{\kern0pt}gen\ {\isasymOmega}\ N\ S{\isacharparenright}{\kern0pt}\ {\isacharequal}{\kern0pt}\ sigma{\isacharunderscore}{\kern0pt}sets\ {\isasymOmega}\ {\isacharparenleft}{\kern0pt}{\isasymUnion}f\ {\isasymin}\ S{\isachardot}{\kern0pt}\ {\isacharbraceleft}{\kern0pt}f\ {\isacharminus}{\kern0pt}{\isacharbackquote}{\kern0pt}\ A\ {\isasyminter}\ {\isasymOmega}\ {\isacharbar}{\kern0pt}\ A{\isachardot}{\kern0pt}\ A\ {\isasymin}\ N{\isacharbraceright}{\kern0pt}{\isacharparenright}{\kern0pt}{\isachardoublequoteclose}\ \isanewline
\ \ \ \ \isakeyword{and}\ space{\isacharunderscore}{\kern0pt}sigma{\isacharunderscore}{\kern0pt}gen{\isacharcolon}{\kern0pt}\ {\isachardoublequoteopen}space\ {\isacharparenleft}{\kern0pt}sigma{\isacharunderscore}{\kern0pt}gen\ {\isasymOmega}\ N\ S{\isacharparenright}{\kern0pt}\ {\isacharequal}{\kern0pt}\ {\isasymOmega}{\isachardoublequoteclose}\isanewline
%
\isadelimproof
\ \ %
\endisadelimproof
%
\isatagproof
\isacommand{by}\isamarkupfalse%
\ {\isacharparenleft}{\kern0pt}auto\ simp\ add{\isacharcolon}{\kern0pt}\ sigma{\isacharunderscore}{\kern0pt}gen{\isacharunderscore}{\kern0pt}def\ sets{\isacharunderscore}{\kern0pt}measure{\isacharunderscore}{\kern0pt}of{\isacharunderscore}{\kern0pt}conv\ space{\isacharunderscore}{\kern0pt}measure{\isacharunderscore}{\kern0pt}of{\isacharunderscore}{\kern0pt}conv{\isacharparenright}{\kern0pt}%
\endisatagproof
{\isafoldproof}%
%
\isadelimproof
\isanewline
%
\endisadelimproof
\isanewline
\isacommand{lemma}\isamarkupfalse%
\ measurable{\isacharunderscore}{\kern0pt}sigma{\isacharunderscore}{\kern0pt}gen{\isacharcolon}{\kern0pt}\isanewline
\ \ \isakeyword{assumes}\ {\isachardoublequoteopen}f\ {\isasymin}\ S{\isachardoublequoteclose}\ {\isachardoublequoteopen}f\ {\isasymin}\ {\isasymOmega}\ {\isasymrightarrow}\ space\ N{\isachardoublequoteclose}\isanewline
\ \ \isakeyword{shows}\ {\isachardoublequoteopen}f\ {\isasymin}\ sigma{\isacharunderscore}{\kern0pt}gen\ {\isasymOmega}\ N\ S\ {\isasymrightarrow}\isactrlsub M\ N{\isachardoublequoteclose}\isanewline
%
\isadelimproof
\ \ %
\endisadelimproof
%
\isatagproof
\isacommand{using}\isamarkupfalse%
\ assms\ \isacommand{by}\isamarkupfalse%
\ {\isacharparenleft}{\kern0pt}intro\ measurableI{\isacharcomma}{\kern0pt}\ auto{\isacharparenright}{\kern0pt}%
\endisatagproof
{\isafoldproof}%
%
\isadelimproof
\isanewline
%
\endisadelimproof
\isanewline
\isacommand{lemma}\isamarkupfalse%
\ measurable{\isacharunderscore}{\kern0pt}sigma{\isacharunderscore}{\kern0pt}gen{\isacharunderscore}{\kern0pt}singleton{\isacharcolon}{\kern0pt}\isanewline
\ \ \isakeyword{assumes}\ {\isachardoublequoteopen}f\ {\isasymin}\ {\isasymOmega}\ {\isasymrightarrow}\ space\ N{\isachardoublequoteclose}\isanewline
\ \ \isakeyword{shows}\ {\isachardoublequoteopen}f\ {\isasymin}\ sigma{\isacharunderscore}{\kern0pt}gen\ {\isasymOmega}\ N\ {\isacharbraceleft}{\kern0pt}f{\isacharbraceright}{\kern0pt}{\isasymrightarrow}\isactrlsub M\ N{\isachardoublequoteclose}\isanewline
%
\isadelimproof
\ \ %
\endisadelimproof
%
\isatagproof
\isacommand{using}\isamarkupfalse%
\ assms\ measurable{\isacharunderscore}{\kern0pt}sigma{\isacharunderscore}{\kern0pt}gen\ \isacommand{by}\isamarkupfalse%
\ blast%
\endisatagproof
{\isafoldproof}%
%
\isadelimproof
\isanewline
%
\endisadelimproof
\isanewline
\isacommand{lemma}\isamarkupfalse%
\ measurable{\isacharunderscore}{\kern0pt}iff{\isacharunderscore}{\kern0pt}contains{\isacharunderscore}{\kern0pt}sigma{\isacharunderscore}{\kern0pt}gen{\isacharcolon}{\kern0pt}\isanewline
\ \ \isakeyword{shows}\ {\isachardoublequoteopen}{\isacharparenleft}{\kern0pt}f\ {\isasymin}\ M\ {\isasymrightarrow}\isactrlsub M\ N{\isacharparenright}{\kern0pt}\ {\isasymlongleftrightarrow}\ f\ {\isasymin}\ space\ M\ {\isasymrightarrow}\ space\ N\ {\isasymand}\ sigma{\isacharunderscore}{\kern0pt}gen\ {\isacharparenleft}{\kern0pt}space\ M{\isacharparenright}{\kern0pt}\ N\ {\isacharbraceleft}{\kern0pt}f{\isacharbraceright}{\kern0pt}\ {\isasymsubseteq}\ M{\isachardoublequoteclose}\isanewline
%
\isadelimproof
%
\endisadelimproof
%
\isatagproof
\isacommand{proof}\isamarkupfalse%
\ {\isacharparenleft}{\kern0pt}standard{\isacharcomma}{\kern0pt}\ goal{\isacharunderscore}{\kern0pt}cases{\isacharparenright}{\kern0pt}\isanewline
\ \ \isacommand{case}\isamarkupfalse%
\ {\isadigit{1}}\isanewline
\ \ \isacommand{hence}\isamarkupfalse%
\ {\isachardoublequoteopen}f\ {\isasymin}\ space\ M\ {\isasymrightarrow}\ space\ N{\isachardoublequoteclose}\ \isacommand{using}\isamarkupfalse%
\ measurable{\isacharunderscore}{\kern0pt}space\ \isacommand{by}\isamarkupfalse%
\ fast\isanewline
\ \ \isacommand{thus}\isamarkupfalse%
\ {\isacharquery}{\kern0pt}case\ \isacommand{unfolding}\isamarkupfalse%
\ sets{\isacharunderscore}{\kern0pt}sigma{\isacharunderscore}{\kern0pt}gen\ \isacommand{by}\isamarkupfalse%
\ {\isacharparenleft}{\kern0pt}simp{\isacharcomma}{\kern0pt}\ intro\ sigma{\isacharunderscore}{\kern0pt}algebra{\isachardot}{\kern0pt}sigma{\isacharunderscore}{\kern0pt}sets{\isacharunderscore}{\kern0pt}subset{\isacharcomma}{\kern0pt}\ {\isacharparenleft}{\kern0pt}blast\ intro{\isacharcolon}{\kern0pt}\ sets{\isachardot}{\kern0pt}sigma{\isacharunderscore}{\kern0pt}algebra{\isacharunderscore}{\kern0pt}axioms\ measurable{\isacharunderscore}{\kern0pt}sets{\isacharbrackleft}{\kern0pt}OF\ {\isadigit{1}}{\isacharbrackright}{\kern0pt}{\isacharparenright}{\kern0pt}{\isacharplus}{\kern0pt}{\isacharparenright}{\kern0pt}\ \isanewline
\isacommand{next}\isamarkupfalse%
\isanewline
\ \ \isacommand{case}\isamarkupfalse%
\ {\isadigit{2}}\isanewline
\ \ \isacommand{thus}\isamarkupfalse%
\ {\isacharquery}{\kern0pt}case\ \isacommand{using}\isamarkupfalse%
\ measurable{\isacharunderscore}{\kern0pt}mono{\isacharbrackleft}{\kern0pt}OF\ {\isacharunderscore}{\kern0pt}\ refl\ {\isacharunderscore}{\kern0pt}\ space{\isacharunderscore}{\kern0pt}sigma{\isacharunderscore}{\kern0pt}gen{\isacharcomma}{\kern0pt}\ of\ N\ M{\isacharbrackright}{\kern0pt}\ measurable{\isacharunderscore}{\kern0pt}sigma{\isacharunderscore}{\kern0pt}gen{\isacharunderscore}{\kern0pt}singleton\ \isacommand{by}\isamarkupfalse%
\ fast\isanewline
\isacommand{qed}\isamarkupfalse%
%
\endisatagproof
{\isafoldproof}%
%
\isadelimproof
\isanewline
%
\endisadelimproof
\isanewline
\isacommand{lemma}\isamarkupfalse%
\ measurable{\isacharunderscore}{\kern0pt}iff{\isacharunderscore}{\kern0pt}contains{\isacharunderscore}{\kern0pt}sigma{\isacharunderscore}{\kern0pt}gen{\isacharprime}{\kern0pt}{\isacharcolon}{\kern0pt}\isanewline
\ \ \isakeyword{shows}\ {\isachardoublequoteopen}{\isacharparenleft}{\kern0pt}S\ {\isasymsubseteq}\ M\ {\isasymrightarrow}\isactrlsub M\ N{\isacharparenright}{\kern0pt}\ {\isasymlongleftrightarrow}\ S\ {\isasymsubseteq}\ space\ M\ {\isasymrightarrow}\ space\ N\ {\isasymand}\ sigma{\isacharunderscore}{\kern0pt}gen\ {\isacharparenleft}{\kern0pt}space\ M{\isacharparenright}{\kern0pt}\ N\ S\ {\isasymsubseteq}\ M{\isachardoublequoteclose}\isanewline
%
\isadelimproof
%
\endisadelimproof
%
\isatagproof
\isacommand{proof}\isamarkupfalse%
\ {\isacharparenleft}{\kern0pt}standard{\isacharcomma}{\kern0pt}\ goal{\isacharunderscore}{\kern0pt}cases{\isacharparenright}{\kern0pt}\isanewline
\ \ \isacommand{case}\isamarkupfalse%
\ {\isadigit{1}}\isanewline
\ \ \isacommand{hence}\isamarkupfalse%
\ subset{\isacharcolon}{\kern0pt}\ {\isachardoublequoteopen}S\ {\isasymsubseteq}\ space\ M\ {\isasymrightarrow}\ space\ N{\isachardoublequoteclose}\ \isacommand{using}\isamarkupfalse%
\ measurable{\isacharunderscore}{\kern0pt}space\ \isacommand{by}\isamarkupfalse%
\ fast\isanewline
\ \ \isacommand{have}\isamarkupfalse%
\ {\isachardoublequoteopen}{\isacharbraceleft}{\kern0pt}f\ {\isacharminus}{\kern0pt}{\isacharbackquote}{\kern0pt}\ A\ {\isasyminter}\ space\ M\ {\isacharbar}{\kern0pt}A{\isachardot}{\kern0pt}\ A\ {\isasymin}\ N{\isacharbraceright}{\kern0pt}\ {\isasymsubseteq}\ M{\isachardoublequoteclose}\ \isakeyword{if}\ {\isachardoublequoteopen}f\ {\isasymin}\ S{\isachardoublequoteclose}\ \isakeyword{for}\ f\ \isacommand{using}\isamarkupfalse%
\ measurable{\isacharunderscore}{\kern0pt}iff{\isacharunderscore}{\kern0pt}contains{\isacharunderscore}{\kern0pt}sigma{\isacharunderscore}{\kern0pt}gen{\isacharbrackleft}{\kern0pt}unfolded\ sets{\isacharunderscore}{\kern0pt}sigma{\isacharunderscore}{\kern0pt}gen{\isacharcomma}{\kern0pt}\ of\ f{\isacharbrackright}{\kern0pt}\ {\isadigit{1}}\ subset\ that\ \isacommand{by}\isamarkupfalse%
\ blast\isanewline
\ \ \isacommand{then}\isamarkupfalse%
\ \isacommand{show}\isamarkupfalse%
\ {\isacharquery}{\kern0pt}case\ \isacommand{unfolding}\isamarkupfalse%
\ sets{\isacharunderscore}{\kern0pt}sigma{\isacharunderscore}{\kern0pt}gen\ \isacommand{using}\isamarkupfalse%
\ sets{\isachardot}{\kern0pt}sigma{\isacharunderscore}{\kern0pt}algebra{\isacharunderscore}{\kern0pt}axioms\ \isacommand{by}\isamarkupfalse%
\ {\isacharparenleft}{\kern0pt}simp\ add{\isacharcolon}{\kern0pt}\ subset{\isacharcomma}{\kern0pt}\ intro\ sigma{\isacharunderscore}{\kern0pt}algebra{\isachardot}{\kern0pt}sigma{\isacharunderscore}{\kern0pt}sets{\isacharunderscore}{\kern0pt}subset{\isacharcomma}{\kern0pt}\ blast{\isacharplus}{\kern0pt}{\isacharparenright}{\kern0pt}\isanewline
\isacommand{next}\isamarkupfalse%
\isanewline
\ \ \isacommand{case}\isamarkupfalse%
\ {\isadigit{2}}\isanewline
\ \ \isacommand{hence}\isamarkupfalse%
\ subset{\isacharcolon}{\kern0pt}\ {\isachardoublequoteopen}S\ {\isasymsubseteq}\ space\ M\ {\isasymrightarrow}\ space\ N{\isachardoublequoteclose}\ \isacommand{by}\isamarkupfalse%
\ simp\isanewline
\ \ \isacommand{show}\isamarkupfalse%
\ {\isacharquery}{\kern0pt}case\isanewline
\ \ \isacommand{proof}\isamarkupfalse%
\ {\isacharparenleft}{\kern0pt}standard{\isacharcomma}{\kern0pt}\ goal{\isacharunderscore}{\kern0pt}cases{\isacharparenright}{\kern0pt}\isanewline
\ \ \ \ \isacommand{case}\isamarkupfalse%
\ {\isacharparenleft}{\kern0pt}{\isadigit{1}}\ x{\isacharparenright}{\kern0pt}\isanewline
\ \ \ \ \isacommand{have}\isamarkupfalse%
\ {\isachardoublequoteopen}sigma{\isacharunderscore}{\kern0pt}gen\ {\isacharparenleft}{\kern0pt}space\ M{\isacharparenright}{\kern0pt}\ N\ {\isacharbraceleft}{\kern0pt}x{\isacharbraceright}{\kern0pt}\ {\isasymsubseteq}\ M{\isachardoublequoteclose}\ \isacommand{by}\isamarkupfalse%
\ {\isacharparenleft}{\kern0pt}metis\ {\isacharparenleft}{\kern0pt}no{\isacharunderscore}{\kern0pt}types{\isacharcomma}{\kern0pt}\ lifting{\isacharparenright}{\kern0pt}\ {\isadigit{1}}\ {\isadigit{2}}\ sets{\isacharunderscore}{\kern0pt}sigma{\isacharunderscore}{\kern0pt}gen\ SUP{\isacharunderscore}{\kern0pt}le{\isacharunderscore}{\kern0pt}iff\ sigma{\isacharunderscore}{\kern0pt}sets{\isacharunderscore}{\kern0pt}le{\isacharunderscore}{\kern0pt}sets{\isacharunderscore}{\kern0pt}iff\ singletonD{\isacharparenright}{\kern0pt}\isanewline
\ \ \ \ \isacommand{thus}\isamarkupfalse%
\ {\isacharquery}{\kern0pt}case\ \isacommand{using}\isamarkupfalse%
\ measurable{\isacharunderscore}{\kern0pt}iff{\isacharunderscore}{\kern0pt}contains{\isacharunderscore}{\kern0pt}sigma{\isacharunderscore}{\kern0pt}gen\ subset{\isacharbrackleft}{\kern0pt}THEN\ subsetD{\isacharcomma}{\kern0pt}\ OF\ {\isadigit{1}}{\isacharbrackright}{\kern0pt}\ \isacommand{by}\isamarkupfalse%
\ fast\ \isanewline
\ \ \isacommand{qed}\isamarkupfalse%
\isanewline
\isacommand{qed}\isamarkupfalse%
%
\endisatagproof
{\isafoldproof}%
%
\isadelimproof
\isanewline
%
\endisadelimproof
%
\isadelimtheory
\isanewline
%
\endisadelimtheory
%
\isatagtheory
\isacommand{end}\isamarkupfalse%
%
\endisatagtheory
{\isafoldtheory}%
%
\isadelimtheory
%
\endisadelimtheory
%
\end{isabellebody}%
\endinput
%:%file=Measure_Space_Addendum.tex%:%
%:%10=1%:%
%:%11=1%:%
%:%12=2%:%
%:%13=3%:%
%:%27=5%:%
%:%37=7%:%
%:%38=7%:%
%:%39=8%:%
%:%40=9%:%
%:%41=10%:%
%:%42=10%:%
%:%43=11%:%
%:%44=12%:%
%:%47=13%:%
%:%51=13%:%
%:%52=13%:%
%:%57=13%:%
%:%60=14%:%
%:%61=15%:%
%:%62=15%:%
%:%63=16%:%
%:%64=17%:%
%:%67=18%:%
%:%71=18%:%
%:%72=18%:%
%:%73=18%:%
%:%78=18%:%
%:%81=19%:%
%:%82=20%:%
%:%83=20%:%
%:%84=21%:%
%:%85=22%:%
%:%88=23%:%
%:%92=23%:%
%:%93=23%:%
%:%94=23%:%
%:%99=23%:%
%:%102=24%:%
%:%103=25%:%
%:%104=25%:%
%:%105=26%:%
%:%112=27%:%
%:%113=27%:%
%:%114=28%:%
%:%115=28%:%
%:%116=29%:%
%:%117=29%:%
%:%118=29%:%
%:%119=29%:%
%:%120=30%:%
%:%121=30%:%
%:%122=30%:%
%:%123=30%:%
%:%124=31%:%
%:%125=31%:%
%:%126=32%:%
%:%127=32%:%
%:%128=33%:%
%:%129=33%:%
%:%130=33%:%
%:%131=33%:%
%:%132=34%:%
%:%138=34%:%
%:%141=35%:%
%:%142=36%:%
%:%143=36%:%
%:%144=37%:%
%:%151=38%:%
%:%152=38%:%
%:%153=39%:%
%:%154=39%:%
%:%155=40%:%
%:%156=40%:%
%:%157=40%:%
%:%158=40%:%
%:%159=41%:%
%:%160=41%:%
%:%161=41%:%
%:%162=41%:%
%:%163=42%:%
%:%164=42%:%
%:%165=42%:%
%:%166=42%:%
%:%167=42%:%
%:%168=42%:%
%:%169=43%:%
%:%170=43%:%
%:%171=44%:%
%:%172=44%:%
%:%173=45%:%
%:%174=45%:%
%:%175=45%:%
%:%176=46%:%
%:%177=46%:%
%:%178=47%:%
%:%179=47%:%
%:%180=48%:%
%:%181=48%:%
%:%182=49%:%
%:%183=49%:%
%:%184=49%:%
%:%185=50%:%
%:%186=50%:%
%:%187=50%:%
%:%188=50%:%
%:%189=51%:%
%:%190=51%:%
%:%191=52%:%
%:%197=52%:%
%:%202=53%:%
%:%207=54%:%

%
\begin{isabellebody}%
\setisabellecontext{Elementary{\isacharunderscore}{\kern0pt}Metric{\isacharunderscore}{\kern0pt}Spaces{\isacharunderscore}{\kern0pt}Addendum}%
%
\isadelimtheory
%
\endisadelimtheory
%
\isatagtheory
\isacommand{theory}\isamarkupfalse%
\ Elementary{\isacharunderscore}{\kern0pt}Metric{\isacharunderscore}{\kern0pt}Spaces{\isacharunderscore}{\kern0pt}Addendum\isanewline
\ \ \isakeyword{imports}\ {\isachardoublequoteopen}HOL{\isacharminus}{\kern0pt}Analysis{\isachardot}{\kern0pt}Elementary{\isacharunderscore}{\kern0pt}Metric{\isacharunderscore}{\kern0pt}Spaces{\isachardoublequoteclose}\isanewline
\isakeyword{begin}%
\endisatagtheory
{\isafoldtheory}%
%
\isadelimtheory
%
\endisadelimtheory
%
\isadelimdocument
%
\endisadelimdocument
%
\isatagdocument
%
\isamarkupsection{Diameter Lemma%
}
\isamarkuptrue%
%
\endisatagdocument
{\isafolddocument}%
%
\isadelimdocument
%
\endisadelimdocument
\isacommand{lemma}\isamarkupfalse%
\ diameter{\isacharunderscore}{\kern0pt}comp{\isacharunderscore}{\kern0pt}strict{\isacharunderscore}{\kern0pt}mono{\isacharcolon}{\kern0pt}\isanewline
\ \ \isakeyword{fixes}\ s\ {\isacharcolon}{\kern0pt}{\isacharcolon}{\kern0pt}\ {\isachardoublequoteopen}nat\ {\isasymRightarrow}\ {\isacharprime}{\kern0pt}a\ {\isacharcolon}{\kern0pt}{\isacharcolon}{\kern0pt}\ metric{\isacharunderscore}{\kern0pt}space{\isachardoublequoteclose}\isanewline
\ \ \isakeyword{assumes}\ {\isachardoublequoteopen}strict{\isacharunderscore}{\kern0pt}mono\ r{\isachardoublequoteclose}\ {\isachardoublequoteopen}bounded\ {\isacharbraceleft}{\kern0pt}s\ i\ {\isacharbar}{\kern0pt}i{\isachardot}{\kern0pt}\ r\ n\ {\isasymle}\ i{\isacharbraceright}{\kern0pt}{\isachardoublequoteclose}\isanewline
\ \ \isakeyword{shows}\ {\isachardoublequoteopen}diameter\ {\isacharbraceleft}{\kern0pt}s\ {\isacharparenleft}{\kern0pt}r\ i{\isacharparenright}{\kern0pt}\ {\isacharbar}{\kern0pt}\ i{\isachardot}{\kern0pt}\ n\ {\isasymle}\ i{\isacharbraceright}{\kern0pt}\ {\isasymle}\ diameter\ {\isacharbraceleft}{\kern0pt}s\ i\ {\isacharbar}{\kern0pt}\ i{\isachardot}{\kern0pt}\ r\ n\ {\isasymle}\ i{\isacharbraceright}{\kern0pt}{\isachardoublequoteclose}\isanewline
%
\isadelimproof
%
\endisadelimproof
%
\isatagproof
\isacommand{proof}\isamarkupfalse%
\ {\isacharparenleft}{\kern0pt}rule\ diameter{\isacharunderscore}{\kern0pt}subset{\isacharparenright}{\kern0pt}\isanewline
\ \ \isacommand{show}\isamarkupfalse%
\ {\isachardoublequoteopen}{\isacharbraceleft}{\kern0pt}s\ {\isacharparenleft}{\kern0pt}r\ i{\isacharparenright}{\kern0pt}\ {\isacharbar}{\kern0pt}\ i{\isachardot}{\kern0pt}\ n\ {\isasymle}\ i{\isacharbraceright}{\kern0pt}\ {\isasymsubseteq}\ {\isacharbraceleft}{\kern0pt}s\ i\ {\isacharbar}{\kern0pt}\ i{\isachardot}{\kern0pt}\ r\ n\ {\isasymle}\ i{\isacharbraceright}{\kern0pt}{\isachardoublequoteclose}\ \isacommand{using}\isamarkupfalse%
\ assms{\isacharparenleft}{\kern0pt}{\isadigit{1}}{\isacharparenright}{\kern0pt}\ monotoneD\ strict{\isacharunderscore}{\kern0pt}mono{\isacharunderscore}{\kern0pt}mono\ \isacommand{by}\isamarkupfalse%
\ fastforce\isanewline
\isacommand{qed}\isamarkupfalse%
\ {\isacharparenleft}{\kern0pt}intro\ assms{\isacharparenleft}{\kern0pt}{\isadigit{2}}{\isacharparenright}{\kern0pt}{\isacharparenright}{\kern0pt}%
\endisatagproof
{\isafoldproof}%
%
\isadelimproof
\isanewline
%
\endisadelimproof
\isanewline
\isacommand{lemma}\isamarkupfalse%
\ diameter{\isacharunderscore}{\kern0pt}bounded{\isacharunderscore}{\kern0pt}bound{\isacharprime}{\kern0pt}{\isacharcolon}{\kern0pt}\isanewline
\ \ \isakeyword{fixes}\ S\ {\isacharcolon}{\kern0pt}{\isacharcolon}{\kern0pt}\ {\isachardoublequoteopen}{\isacharprime}{\kern0pt}a\ {\isacharcolon}{\kern0pt}{\isacharcolon}{\kern0pt}\ metric{\isacharunderscore}{\kern0pt}space\ set{\isachardoublequoteclose}\isanewline
\ \ \isakeyword{assumes}\ S{\isacharcolon}{\kern0pt}\ {\isachardoublequoteopen}bdd{\isacharunderscore}{\kern0pt}above\ {\isacharparenleft}{\kern0pt}case{\isacharunderscore}{\kern0pt}prod\ dist\ {\isacharbackquote}{\kern0pt}\ {\isacharparenleft}{\kern0pt}S{\isasymtimes}S{\isacharparenright}{\kern0pt}{\isacharparenright}{\kern0pt}{\isachardoublequoteclose}\ {\isachardoublequoteopen}x\ {\isasymin}\ S{\isachardoublequoteclose}\ {\isachardoublequoteopen}y\ {\isasymin}\ S{\isachardoublequoteclose}\isanewline
\ \ \isakeyword{shows}\ {\isachardoublequoteopen}dist\ x\ y\ {\isasymle}\ diameter\ S{\isachardoublequoteclose}\isanewline
%
\isadelimproof
%
\endisadelimproof
%
\isatagproof
\isacommand{proof}\isamarkupfalse%
\ {\isacharminus}{\kern0pt}\isanewline
\ \ \isacommand{have}\isamarkupfalse%
\ {\isachardoublequoteopen}{\isacharparenleft}{\kern0pt}x{\isacharcomma}{\kern0pt}y{\isacharparenright}{\kern0pt}\ {\isasymin}\ S{\isasymtimes}S{\isachardoublequoteclose}\ \isacommand{using}\isamarkupfalse%
\ S\ \isacommand{by}\isamarkupfalse%
\ auto\isanewline
\ \ \isacommand{then}\isamarkupfalse%
\ \isacommand{have}\isamarkupfalse%
\ {\isachardoublequoteopen}dist\ x\ y\ {\isasymle}\ {\isacharparenleft}{\kern0pt}SUP\ {\isacharparenleft}{\kern0pt}x{\isacharcomma}{\kern0pt}y{\isacharparenright}{\kern0pt}{\isasymin}S{\isasymtimes}S{\isachardot}{\kern0pt}\ dist\ x\ y{\isacharparenright}{\kern0pt}{\isachardoublequoteclose}\ \isacommand{by}\isamarkupfalse%
\ {\isacharparenleft}{\kern0pt}rule\ cSUP{\isacharunderscore}{\kern0pt}upper{\isadigit{2}}{\isacharbrackleft}{\kern0pt}OF\ assms{\isacharparenleft}{\kern0pt}{\isadigit{1}}{\isacharparenright}{\kern0pt}{\isacharbrackright}{\kern0pt}{\isacharparenright}{\kern0pt}\ simp\isanewline
\ \ \isacommand{with}\isamarkupfalse%
\ {\isacartoucheopen}x\ {\isasymin}\ S{\isacartoucheclose}\ \isacommand{show}\isamarkupfalse%
\ {\isacharquery}{\kern0pt}thesis\ \isacommand{by}\isamarkupfalse%
\ {\isacharparenleft}{\kern0pt}auto\ simp{\isacharcolon}{\kern0pt}\ diameter{\isacharunderscore}{\kern0pt}def{\isacharparenright}{\kern0pt}\isanewline
\isacommand{qed}\isamarkupfalse%
%
\endisatagproof
{\isafoldproof}%
%
\isadelimproof
\isanewline
%
\endisadelimproof
\isanewline
\isacommand{lemma}\isamarkupfalse%
\ bounded{\isacharunderscore}{\kern0pt}imp{\isacharunderscore}{\kern0pt}dist{\isacharunderscore}{\kern0pt}bounded{\isacharcolon}{\kern0pt}\isanewline
\ \ \isakeyword{assumes}\ {\isachardoublequoteopen}bounded\ {\isacharparenleft}{\kern0pt}range\ s{\isacharparenright}{\kern0pt}{\isachardoublequoteclose}\isanewline
\ \ \isakeyword{shows}\ {\isachardoublequoteopen}bounded\ {\isacharparenleft}{\kern0pt}{\isacharparenleft}{\kern0pt}{\isasymlambda}{\isacharparenleft}{\kern0pt}i{\isacharcomma}{\kern0pt}\ j{\isacharparenright}{\kern0pt}{\isachardot}{\kern0pt}\ dist\ {\isacharparenleft}{\kern0pt}s\ i{\isacharparenright}{\kern0pt}\ {\isacharparenleft}{\kern0pt}s\ j{\isacharparenright}{\kern0pt}{\isacharparenright}{\kern0pt}\ {\isacharbackquote}{\kern0pt}\ {\isacharparenleft}{\kern0pt}{\isacharbraceleft}{\kern0pt}n{\isachardot}{\kern0pt}{\isachardot}{\kern0pt}{\isacharbraceright}{\kern0pt}\ {\isasymtimes}\ {\isacharbraceleft}{\kern0pt}n{\isachardot}{\kern0pt}{\isachardot}{\kern0pt}{\isacharbraceright}{\kern0pt}{\isacharparenright}{\kern0pt}{\isacharparenright}{\kern0pt}{\isachardoublequoteclose}\isanewline
%
\isadelimproof
\ \ %
\endisadelimproof
%
\isatagproof
\isacommand{using}\isamarkupfalse%
\ bounded{\isacharunderscore}{\kern0pt}dist{\isacharunderscore}{\kern0pt}comp{\isacharbrackleft}{\kern0pt}OF\ bounded{\isacharunderscore}{\kern0pt}fst\ bounded{\isacharunderscore}{\kern0pt}snd{\isacharcomma}{\kern0pt}\ OF\ bounded{\isacharunderscore}{\kern0pt}Times{\isacharparenleft}{\kern0pt}{\isadigit{1}}{\isacharcomma}{\kern0pt}{\isadigit{1}}{\isacharparenright}{\kern0pt}{\isacharbrackleft}{\kern0pt}OF\ assms{\isacharparenleft}{\kern0pt}{\isadigit{1}}{\isacharcomma}{\kern0pt}{\isadigit{1}}{\isacharparenright}{\kern0pt}{\isacharbrackright}{\kern0pt}{\isacharbrackright}{\kern0pt}\ \isacommand{by}\isamarkupfalse%
\ {\isacharparenleft}{\kern0pt}rule\ bounded{\isacharunderscore}{\kern0pt}subset{\isacharcomma}{\kern0pt}\ force{\isacharparenright}{\kern0pt}%
\endisatagproof
{\isafoldproof}%
%
\isadelimproof
\ \isanewline
%
\endisadelimproof
\isanewline
\isacommand{lemma}\isamarkupfalse%
\ cauchy{\isacharunderscore}{\kern0pt}iff{\isacharunderscore}{\kern0pt}diameter{\isacharunderscore}{\kern0pt}tends{\isacharunderscore}{\kern0pt}to{\isacharunderscore}{\kern0pt}zero{\isacharunderscore}{\kern0pt}and{\isacharunderscore}{\kern0pt}bounded{\isacharcolon}{\kern0pt}\isanewline
\ \ \isakeyword{fixes}\ s\ {\isacharcolon}{\kern0pt}{\isacharcolon}{\kern0pt}\ {\isachardoublequoteopen}nat\ {\isasymRightarrow}\ {\isacharprime}{\kern0pt}a\ {\isacharcolon}{\kern0pt}{\isacharcolon}{\kern0pt}\ metric{\isacharunderscore}{\kern0pt}space{\isachardoublequoteclose}\isanewline
\ \ \isakeyword{shows}\ {\isachardoublequoteopen}Cauchy\ s\ {\isasymlongleftrightarrow}\ {\isacharparenleft}{\kern0pt}{\isacharparenleft}{\kern0pt}{\isasymlambda}n{\isachardot}{\kern0pt}\ diameter\ {\isacharbraceleft}{\kern0pt}s\ i\ {\isacharbar}{\kern0pt}\ i{\isachardot}{\kern0pt}\ i\ {\isasymge}\ n{\isacharbraceright}{\kern0pt}{\isacharparenright}{\kern0pt}\ {\isasymlonglonglongrightarrow}\ {\isadigit{0}}\ {\isasymand}\ bounded\ {\isacharparenleft}{\kern0pt}range\ s{\isacharparenright}{\kern0pt}{\isacharparenright}{\kern0pt}{\isachardoublequoteclose}\isanewline
%
\isadelimproof
%
\endisadelimproof
%
\isatagproof
\isacommand{proof}\isamarkupfalse%
\ {\isacharminus}{\kern0pt}\isanewline
\ \ \isacommand{have}\isamarkupfalse%
\ {\isachardoublequoteopen}{\isacharbraceleft}{\kern0pt}s\ i\ {\isacharbar}{\kern0pt}i{\isachardot}{\kern0pt}\ N\ {\isasymle}\ i{\isacharbraceright}{\kern0pt}\ {\isasymnoteq}\ {\isacharbraceleft}{\kern0pt}{\isacharbraceright}{\kern0pt}{\isachardoublequoteclose}\ \isakeyword{for}\ N\ \isacommand{by}\isamarkupfalse%
\ blast\isanewline
\ \ \isacommand{hence}\isamarkupfalse%
\ diameter{\isacharunderscore}{\kern0pt}SUP{\isacharcolon}{\kern0pt}\ {\isachardoublequoteopen}diameter\ {\isacharbraceleft}{\kern0pt}s\ i\ {\isacharbar}{\kern0pt}i{\isachardot}{\kern0pt}\ N\ {\isasymle}\ i{\isacharbraceright}{\kern0pt}\ {\isacharequal}{\kern0pt}\ {\isacharparenleft}{\kern0pt}SUP\ {\isacharparenleft}{\kern0pt}i{\isacharcomma}{\kern0pt}\ j{\isacharparenright}{\kern0pt}\ {\isasymin}\ {\isacharbraceleft}{\kern0pt}N{\isachardot}{\kern0pt}{\isachardot}{\kern0pt}{\isacharbraceright}{\kern0pt}\ {\isasymtimes}\ {\isacharbraceleft}{\kern0pt}N{\isachardot}{\kern0pt}{\isachardot}{\kern0pt}{\isacharbraceright}{\kern0pt}{\isachardot}{\kern0pt}\ dist\ {\isacharparenleft}{\kern0pt}s\ i{\isacharparenright}{\kern0pt}\ {\isacharparenleft}{\kern0pt}s\ j{\isacharparenright}{\kern0pt}{\isacharparenright}{\kern0pt}{\isachardoublequoteclose}\ \isakeyword{for}\ N\ \isacommand{unfolding}\isamarkupfalse%
\ diameter{\isacharunderscore}{\kern0pt}def\ \isacommand{by}\isamarkupfalse%
\ {\isacharparenleft}{\kern0pt}auto\ intro{\isacharbang}{\kern0pt}{\isacharcolon}{\kern0pt}\ arg{\isacharunderscore}{\kern0pt}cong{\isacharbrackleft}{\kern0pt}of\ {\isacharunderscore}{\kern0pt}\ {\isacharunderscore}{\kern0pt}\ Sup{\isacharbrackright}{\kern0pt}{\isacharparenright}{\kern0pt}\isanewline
\ \ \isacommand{show}\isamarkupfalse%
\ {\isacharquery}{\kern0pt}thesis\ \isanewline
\ \ \isacommand{proof}\isamarkupfalse%
\ {\isacharparenleft}{\kern0pt}{\isacharparenleft}{\kern0pt}intro\ iffI{\isacharparenright}{\kern0pt}\ {\isacharsemicolon}{\kern0pt}\ clarsimp{\isacharparenright}{\kern0pt}\isanewline
\ \ \ \ \isacommand{assume}\isamarkupfalse%
\ asm{\isacharcolon}{\kern0pt}\ {\isachardoublequoteopen}Cauchy\ s{\isachardoublequoteclose}\isanewline
\ \ \ \ \isacommand{have}\isamarkupfalse%
\ {\isachardoublequoteopen}{\isasymexists}N{\isachardot}{\kern0pt}\ {\isasymforall}n{\isasymge}N{\isachardot}{\kern0pt}\ norm\ {\isacharparenleft}{\kern0pt}diameter\ {\isacharbraceleft}{\kern0pt}s\ i\ {\isacharbar}{\kern0pt}i{\isachardot}{\kern0pt}\ n\ {\isasymle}\ i{\isacharbraceright}{\kern0pt}{\isacharparenright}{\kern0pt}\ {\isacharless}{\kern0pt}\ e{\isachardoublequoteclose}\ \isakeyword{if}\ e{\isacharunderscore}{\kern0pt}pos{\isacharcolon}{\kern0pt}\ {\isachardoublequoteopen}e\ {\isachargreater}{\kern0pt}\ {\isadigit{0}}{\isachardoublequoteclose}\ \isakeyword{for}\ e\isanewline
\ \ \ \ \isacommand{proof}\isamarkupfalse%
\ {\isacharminus}{\kern0pt}\isanewline
\ \ \ \ \ \ \isacommand{obtain}\isamarkupfalse%
\ N\ \isakeyword{where}\ dist{\isacharunderscore}{\kern0pt}less{\isacharcolon}{\kern0pt}\ {\isachardoublequoteopen}dist\ {\isacharparenleft}{\kern0pt}s\ n{\isacharparenright}{\kern0pt}\ {\isacharparenleft}{\kern0pt}s\ m{\isacharparenright}{\kern0pt}\ {\isacharless}{\kern0pt}\ {\isacharparenleft}{\kern0pt}{\isadigit{1}}{\isacharslash}{\kern0pt}{\isadigit{2}}{\isacharparenright}{\kern0pt}\ {\isacharasterisk}{\kern0pt}\ e{\isachardoublequoteclose}\ \isakeyword{if}\ {\isachardoublequoteopen}n\ {\isasymge}\ N{\isachardoublequoteclose}\ {\isachardoublequoteopen}m\ {\isasymge}\ N{\isachardoublequoteclose}\ \isakeyword{for}\ n\ m\ \isacommand{using}\isamarkupfalse%
\ asm\ e{\isacharunderscore}{\kern0pt}pos\ \isacommand{by}\isamarkupfalse%
\ {\isacharparenleft}{\kern0pt}metis\ Cauchy{\isacharunderscore}{\kern0pt}def\ mult{\isacharunderscore}{\kern0pt}pos{\isacharunderscore}{\kern0pt}pos\ zero{\isacharunderscore}{\kern0pt}less{\isacharunderscore}{\kern0pt}divide{\isacharunderscore}{\kern0pt}iff\ zero{\isacharunderscore}{\kern0pt}less{\isacharunderscore}{\kern0pt}numeral\ zero{\isacharunderscore}{\kern0pt}less{\isacharunderscore}{\kern0pt}one{\isacharparenright}{\kern0pt}\isanewline
\ \ \ \ \ \ \isacommand{{\isacharbraceleft}{\kern0pt}}\isamarkupfalse%
\isanewline
\ \ \ \ \ \ \ \ \isacommand{fix}\isamarkupfalse%
\ r\ \isacommand{assume}\isamarkupfalse%
\ {\isachardoublequoteopen}r\ {\isasymge}\ N{\isachardoublequoteclose}\isanewline
\ \ \ \ \ \ \ \ \isacommand{hence}\isamarkupfalse%
\ {\isachardoublequoteopen}dist\ {\isacharparenleft}{\kern0pt}s\ n{\isacharparenright}{\kern0pt}\ {\isacharparenleft}{\kern0pt}s\ m{\isacharparenright}{\kern0pt}\ {\isacharless}{\kern0pt}\ {\isacharparenleft}{\kern0pt}{\isadigit{1}}{\isacharslash}{\kern0pt}{\isadigit{2}}{\isacharparenright}{\kern0pt}\ {\isacharasterisk}{\kern0pt}\ e{\isachardoublequoteclose}\ \isakeyword{if}\ {\isachardoublequoteopen}n\ {\isasymge}\ r{\isachardoublequoteclose}\ {\isachardoublequoteopen}m\ {\isasymge}\ r{\isachardoublequoteclose}\ \isakeyword{for}\ n\ m\ \isacommand{using}\isamarkupfalse%
\ dist{\isacharunderscore}{\kern0pt}less\ that\ \isacommand{by}\isamarkupfalse%
\ simp\isanewline
\ \ \ \ \ \ \ \ \isacommand{hence}\isamarkupfalse%
\ {\isachardoublequoteopen}{\isacharparenleft}{\kern0pt}SUP\ {\isacharparenleft}{\kern0pt}i{\isacharcomma}{\kern0pt}\ j{\isacharparenright}{\kern0pt}\ {\isasymin}\ {\isacharbraceleft}{\kern0pt}r{\isachardot}{\kern0pt}{\isachardot}{\kern0pt}{\isacharbraceright}{\kern0pt}\ {\isasymtimes}\ {\isacharbraceleft}{\kern0pt}r{\isachardot}{\kern0pt}{\isachardot}{\kern0pt}{\isacharbraceright}{\kern0pt}{\isachardot}{\kern0pt}\ dist\ {\isacharparenleft}{\kern0pt}s\ i{\isacharparenright}{\kern0pt}\ {\isacharparenleft}{\kern0pt}s\ j{\isacharparenright}{\kern0pt}{\isacharparenright}{\kern0pt}\ {\isasymle}\ {\isacharparenleft}{\kern0pt}{\isadigit{1}}{\isacharslash}{\kern0pt}{\isadigit{2}}{\isacharparenright}{\kern0pt}\ {\isacharasterisk}{\kern0pt}\ e{\isachardoublequoteclose}\ \isacommand{by}\isamarkupfalse%
\ {\isacharparenleft}{\kern0pt}intro\ cSup{\isacharunderscore}{\kern0pt}least{\isacharparenright}{\kern0pt}\ fastforce{\isacharplus}{\kern0pt}\isanewline
\ \ \ \ \ \ \ \ \isacommand{also}\isamarkupfalse%
\ \isacommand{have}\isamarkupfalse%
\ {\isachardoublequoteopen}{\isachardot}{\kern0pt}{\isachardot}{\kern0pt}{\isachardot}{\kern0pt}\ {\isacharless}{\kern0pt}\ e{\isachardoublequoteclose}\ \isacommand{using}\isamarkupfalse%
\ e{\isacharunderscore}{\kern0pt}pos\ \isacommand{by}\isamarkupfalse%
\ simp\isanewline
\ \ \ \ \ \ \ \ \isacommand{finally}\isamarkupfalse%
\ \isacommand{have}\isamarkupfalse%
\ {\isachardoublequoteopen}diameter\ {\isacharbraceleft}{\kern0pt}s\ i\ {\isacharbar}{\kern0pt}i{\isachardot}{\kern0pt}\ r\ {\isasymle}\ i{\isacharbraceright}{\kern0pt}\ {\isacharless}{\kern0pt}\ e{\isachardoublequoteclose}\ \isacommand{using}\isamarkupfalse%
\ diameter{\isacharunderscore}{\kern0pt}SUP\ \isacommand{by}\isamarkupfalse%
\ presburger\isanewline
\ \ \ \ \ \ \isacommand{{\isacharbraceright}{\kern0pt}}\isamarkupfalse%
\isanewline
\ \ \ \ \ \ \isacommand{moreover}\isamarkupfalse%
\ \isacommand{have}\isamarkupfalse%
\ {\isachardoublequoteopen}diameter\ {\isacharbraceleft}{\kern0pt}s\ i\ {\isacharbar}{\kern0pt}i{\isachardot}{\kern0pt}\ r\ {\isasymle}\ i{\isacharbraceright}{\kern0pt}\ {\isasymge}\ {\isadigit{0}}{\isachardoublequoteclose}\ \isakeyword{for}\ r\ \isacommand{unfolding}\isamarkupfalse%
\ diameter{\isacharunderscore}{\kern0pt}SUP\ \isacommand{using}\isamarkupfalse%
\ bounded{\isacharunderscore}{\kern0pt}imp{\isacharunderscore}{\kern0pt}dist{\isacharunderscore}{\kern0pt}bounded{\isacharbrackleft}{\kern0pt}OF\ cauchy{\isacharunderscore}{\kern0pt}imp{\isacharunderscore}{\kern0pt}bounded{\isacharcomma}{\kern0pt}\ THEN\ bounded{\isacharunderscore}{\kern0pt}imp{\isacharunderscore}{\kern0pt}bdd{\isacharunderscore}{\kern0pt}above{\isacharcomma}{\kern0pt}\ OF\ asm{\isacharbrackright}{\kern0pt}\ \isacommand{by}\isamarkupfalse%
\ {\isacharparenleft}{\kern0pt}intro\ cSup{\isacharunderscore}{\kern0pt}upper{\isadigit{2}}{\isacharcomma}{\kern0pt}\ auto{\isacharparenright}{\kern0pt}\isanewline
\ \ \ \ \ \ \isacommand{ultimately}\isamarkupfalse%
\ \isacommand{show}\isamarkupfalse%
\ {\isacharquery}{\kern0pt}thesis\ \isacommand{by}\isamarkupfalse%
\ auto\isanewline
\ \ \ \ \isacommand{qed}\isamarkupfalse%
\ \ \ \ \ \ \ \ \ \ \ \ \ \ \ \ \ \isanewline
\ \ \ \ \isacommand{thus}\isamarkupfalse%
\ {\isachardoublequoteopen}{\isacharparenleft}{\kern0pt}{\isasymlambda}n{\isachardot}{\kern0pt}\ diameter\ {\isacharbraceleft}{\kern0pt}s\ i\ {\isacharbar}{\kern0pt}i{\isachardot}{\kern0pt}\ n\ {\isasymle}\ i{\isacharbraceright}{\kern0pt}{\isacharparenright}{\kern0pt}\ {\isasymlonglonglongrightarrow}\ {\isadigit{0}}\ {\isasymand}\ bounded\ {\isacharparenleft}{\kern0pt}range\ s{\isacharparenright}{\kern0pt}{\isachardoublequoteclose}\ \isacommand{using}\isamarkupfalse%
\ cauchy{\isacharunderscore}{\kern0pt}imp{\isacharunderscore}{\kern0pt}bounded{\isacharbrackleft}{\kern0pt}OF\ asm{\isacharbrackright}{\kern0pt}\ \isacommand{by}\isamarkupfalse%
\ {\isacharparenleft}{\kern0pt}simp\ add{\isacharcolon}{\kern0pt}\ LIMSEQ{\isacharunderscore}{\kern0pt}iff{\isacharparenright}{\kern0pt}\isanewline
\ \ \isacommand{next}\isamarkupfalse%
\isanewline
\ \ \ \ \isacommand{assume}\isamarkupfalse%
\ asm{\isacharcolon}{\kern0pt}\ {\isachardoublequoteopen}{\isacharparenleft}{\kern0pt}{\isasymlambda}n{\isachardot}{\kern0pt}\ diameter\ {\isacharbraceleft}{\kern0pt}s\ i\ {\isacharbar}{\kern0pt}i{\isachardot}{\kern0pt}\ n\ {\isasymle}\ i{\isacharbraceright}{\kern0pt}{\isacharparenright}{\kern0pt}\ {\isasymlonglonglongrightarrow}\ {\isadigit{0}}{\isachardoublequoteclose}\ {\isachardoublequoteopen}bounded\ {\isacharparenleft}{\kern0pt}range\ s{\isacharparenright}{\kern0pt}{\isachardoublequoteclose}\isanewline
\ \ \ \ \isacommand{have}\isamarkupfalse%
\ {\isachardoublequoteopen}{\isasymexists}N{\isachardot}{\kern0pt}\ {\isasymforall}n{\isasymge}N{\isachardot}{\kern0pt}\ {\isasymforall}m{\isasymge}N{\isachardot}{\kern0pt}\ dist\ {\isacharparenleft}{\kern0pt}s\ n{\isacharparenright}{\kern0pt}\ {\isacharparenleft}{\kern0pt}s\ m{\isacharparenright}{\kern0pt}\ {\isacharless}{\kern0pt}\ e{\isachardoublequoteclose}\ \isakeyword{if}\ e{\isacharunderscore}{\kern0pt}pos{\isacharcolon}{\kern0pt}\ {\isachardoublequoteopen}e\ {\isachargreater}{\kern0pt}\ {\isadigit{0}}{\isachardoublequoteclose}\ \isakeyword{for}\ e\isanewline
\ \ \ \ \isacommand{proof}\isamarkupfalse%
\ {\isacharminus}{\kern0pt}\isanewline
\ \ \ \ \ \ \isacommand{obtain}\isamarkupfalse%
\ N\ \isakeyword{where}\ diam{\isacharunderscore}{\kern0pt}less{\isacharcolon}{\kern0pt}\ {\isachardoublequoteopen}diameter\ {\isacharbraceleft}{\kern0pt}s\ i\ {\isacharbar}{\kern0pt}i{\isachardot}{\kern0pt}\ r\ {\isasymle}\ i{\isacharbraceright}{\kern0pt}\ {\isacharless}{\kern0pt}\ e{\isachardoublequoteclose}\ \isakeyword{if}\ {\isachardoublequoteopen}r\ {\isasymge}\ N{\isachardoublequoteclose}\ \isakeyword{for}\ r\ \isacommand{using}\isamarkupfalse%
\ LIMSEQ{\isacharunderscore}{\kern0pt}D\ asm{\isacharparenleft}{\kern0pt}{\isadigit{1}}{\isacharparenright}{\kern0pt}\ e{\isacharunderscore}{\kern0pt}pos\ \isacommand{by}\isamarkupfalse%
\ fastforce\isanewline
\ \ \ \ \ \ \isacommand{{\isacharbraceleft}{\kern0pt}}\isamarkupfalse%
\isanewline
\ \ \ \ \ \ \ \ \isacommand{fix}\isamarkupfalse%
\ n\ m\ \isacommand{assume}\isamarkupfalse%
\ {\isachardoublequoteopen}n\ {\isasymge}\ N{\isachardoublequoteclose}\ {\isachardoublequoteopen}m\ {\isasymge}\ N{\isachardoublequoteclose}\isanewline
\ \ \ \ \ \ \ \ \isacommand{hence}\isamarkupfalse%
\ {\isachardoublequoteopen}dist\ {\isacharparenleft}{\kern0pt}s\ n{\isacharparenright}{\kern0pt}\ {\isacharparenleft}{\kern0pt}s\ m{\isacharparenright}{\kern0pt}\ {\isacharless}{\kern0pt}\ e{\isachardoublequoteclose}\ \isacommand{using}\isamarkupfalse%
\ cSUP{\isacharunderscore}{\kern0pt}lessD{\isacharbrackleft}{\kern0pt}OF\ bounded{\isacharunderscore}{\kern0pt}imp{\isacharunderscore}{\kern0pt}dist{\isacharunderscore}{\kern0pt}bounded{\isacharbrackleft}{\kern0pt}THEN\ bounded{\isacharunderscore}{\kern0pt}imp{\isacharunderscore}{\kern0pt}bdd{\isacharunderscore}{\kern0pt}above{\isacharbrackright}{\kern0pt}{\isacharcomma}{\kern0pt}\ OF\ asm{\isacharparenleft}{\kern0pt}{\isadigit{2}}{\isacharparenright}{\kern0pt}\ diam{\isacharunderscore}{\kern0pt}less{\isacharbrackleft}{\kern0pt}unfolded\ diameter{\isacharunderscore}{\kern0pt}SUP{\isacharbrackright}{\kern0pt}{\isacharbrackright}{\kern0pt}\ \isacommand{by}\isamarkupfalse%
\ auto\isanewline
\ \ \ \ \ \ \isacommand{{\isacharbraceright}{\kern0pt}}\isamarkupfalse%
\isanewline
\ \ \ \ \ \ \isacommand{thus}\isamarkupfalse%
\ {\isacharquery}{\kern0pt}thesis\ \isacommand{by}\isamarkupfalse%
\ blast\isanewline
\ \ \ \ \isacommand{qed}\isamarkupfalse%
\isanewline
\ \ \ \ \isacommand{then}\isamarkupfalse%
\ \isacommand{show}\isamarkupfalse%
\ {\isachardoublequoteopen}Cauchy\ s{\isachardoublequoteclose}\ \isacommand{by}\isamarkupfalse%
\ {\isacharparenleft}{\kern0pt}simp\ add{\isacharcolon}{\kern0pt}\ Cauchy{\isacharunderscore}{\kern0pt}def{\isacharparenright}{\kern0pt}\isanewline
\ \ \isacommand{qed}\isamarkupfalse%
\ \ \ \ \ \ \ \ \ \ \ \ \isanewline
\isacommand{qed}\isamarkupfalse%
%
\endisatagproof
{\isafoldproof}%
%
\isadelimproof
\isanewline
%
\endisadelimproof
\isanewline
%
\isadelimtheory
\ \ \isanewline
%
\endisadelimtheory
%
\isatagtheory
\isacommand{end}\isamarkupfalse%
%
\endisatagtheory
{\isafoldtheory}%
%
\isadelimtheory
%
\endisadelimtheory
%
\end{isabellebody}%
\endinput
%:%file=Elementary_Metric_Spaces_Addendum.tex%:%
%:%10=1%:%
%:%11=1%:%
%:%12=2%:%
%:%13=3%:%
%:%27=5%:%
%:%37=7%:%
%:%38=7%:%
%:%39=8%:%
%:%40=9%:%
%:%41=10%:%
%:%48=11%:%
%:%49=11%:%
%:%50=12%:%
%:%51=12%:%
%:%52=12%:%
%:%53=12%:%
%:%54=13%:%
%:%55=13%:%
%:%60=13%:%
%:%63=14%:%
%:%64=15%:%
%:%65=15%:%
%:%66=16%:%
%:%67=17%:%
%:%68=18%:%
%:%75=19%:%
%:%76=19%:%
%:%77=20%:%
%:%78=20%:%
%:%79=20%:%
%:%80=20%:%
%:%81=21%:%
%:%82=21%:%
%:%83=21%:%
%:%84=21%:%
%:%85=22%:%
%:%86=22%:%
%:%87=22%:%
%:%88=22%:%
%:%89=23%:%
%:%95=23%:%
%:%98=24%:%
%:%99=25%:%
%:%100=25%:%
%:%101=26%:%
%:%102=27%:%
%:%105=28%:%
%:%109=28%:%
%:%110=28%:%
%:%111=28%:%
%:%116=28%:%
%:%119=29%:%
%:%120=30%:%
%:%121=30%:%
%:%122=31%:%
%:%123=32%:%
%:%130=33%:%
%:%131=33%:%
%:%132=34%:%
%:%133=34%:%
%:%134=34%:%
%:%135=35%:%
%:%136=35%:%
%:%137=35%:%
%:%138=35%:%
%:%139=36%:%
%:%140=36%:%
%:%141=37%:%
%:%142=37%:%
%:%143=38%:%
%:%144=38%:%
%:%145=39%:%
%:%146=39%:%
%:%147=40%:%
%:%148=40%:%
%:%149=41%:%
%:%150=41%:%
%:%151=41%:%
%:%152=41%:%
%:%153=42%:%
%:%154=42%:%
%:%155=43%:%
%:%156=43%:%
%:%157=43%:%
%:%158=44%:%
%:%159=44%:%
%:%160=44%:%
%:%161=44%:%
%:%162=45%:%
%:%163=45%:%
%:%164=45%:%
%:%165=46%:%
%:%166=46%:%
%:%167=46%:%
%:%168=46%:%
%:%169=46%:%
%:%170=47%:%
%:%171=47%:%
%:%172=47%:%
%:%173=47%:%
%:%174=47%:%
%:%175=48%:%
%:%176=48%:%
%:%177=49%:%
%:%178=49%:%
%:%179=49%:%
%:%180=49%:%
%:%181=49%:%
%:%182=49%:%
%:%183=50%:%
%:%184=50%:%
%:%185=50%:%
%:%186=50%:%
%:%187=51%:%
%:%188=51%:%
%:%189=52%:%
%:%190=52%:%
%:%191=52%:%
%:%192=52%:%
%:%193=53%:%
%:%194=53%:%
%:%195=54%:%
%:%196=54%:%
%:%197=55%:%
%:%198=55%:%
%:%199=56%:%
%:%200=56%:%
%:%201=57%:%
%:%202=57%:%
%:%203=57%:%
%:%204=57%:%
%:%205=58%:%
%:%206=58%:%
%:%207=59%:%
%:%208=59%:%
%:%209=59%:%
%:%210=60%:%
%:%211=60%:%
%:%212=60%:%
%:%213=60%:%
%:%214=61%:%
%:%215=61%:%
%:%216=62%:%
%:%217=62%:%
%:%218=62%:%
%:%219=63%:%
%:%220=63%:%
%:%221=64%:%
%:%222=64%:%
%:%223=64%:%
%:%224=64%:%
%:%225=65%:%
%:%226=65%:%
%:%227=66%:%
%:%233=66%:%
%:%236=67%:%
%:%239=68%:%
%:%244=69%:%

%
\begin{isabellebody}%
\setisabellecontext{Bochner{\isacharunderscore}{\kern0pt}Integration{\isacharunderscore}{\kern0pt}Addendum}%
%
\isadelimtheory
%
\endisadelimtheory
%
\isatagtheory
\isacommand{theory}\isamarkupfalse%
\ Bochner{\isacharunderscore}{\kern0pt}Integration{\isacharunderscore}{\kern0pt}Addendum\isanewline
\ \ \isakeyword{imports}\ {\isachardoublequoteopen}HOL{\isacharminus}{\kern0pt}Analysis{\isachardot}{\kern0pt}Bochner{\isacharunderscore}{\kern0pt}Integration{\isachardoublequoteclose}\isanewline
\isakeyword{begin}%
\endisatagtheory
{\isafoldtheory}%
%
\isadelimtheory
%
\endisadelimtheory
%
\isadelimdocument
%
\endisadelimdocument
%
\isatagdocument
%
\isamarkupsubsection{Simple Functions%
}
\isamarkuptrue%
%
\endisatagdocument
{\isafolddocument}%
%
\isadelimdocument
%
\endisadelimdocument
\isacommand{lemma}\isamarkupfalse%
\ integrable{\isacharunderscore}{\kern0pt}implies{\isacharunderscore}{\kern0pt}simple{\isacharunderscore}{\kern0pt}function{\isacharunderscore}{\kern0pt}sequence{\isacharcolon}{\kern0pt}\isanewline
\ \ \isakeyword{fixes}\ f\ {\isacharcolon}{\kern0pt}{\isacharcolon}{\kern0pt}\ {\isachardoublequoteopen}{\isacharprime}{\kern0pt}a\ {\isasymRightarrow}\ {\isacharprime}{\kern0pt}b{\isacharcolon}{\kern0pt}{\isacharcolon}{\kern0pt}{\isacharbraceleft}{\kern0pt}banach{\isacharcomma}{\kern0pt}\ second{\isacharunderscore}{\kern0pt}countable{\isacharunderscore}{\kern0pt}topology{\isacharbraceright}{\kern0pt}{\isachardoublequoteclose}\isanewline
\ \ \isakeyword{assumes}\ {\isachardoublequoteopen}integrable\ M\ f{\isachardoublequoteclose}\isanewline
\ \ \isakeyword{obtains}\ s\ \isakeyword{where}\ {\isachardoublequoteopen}{\isasymAnd}i{\isachardot}{\kern0pt}\ simple{\isacharunderscore}{\kern0pt}function\ M\ {\isacharparenleft}{\kern0pt}s\ i{\isacharparenright}{\kern0pt}{\isachardoublequoteclose}\isanewline
\ \ \ \ \ \ \isakeyword{and}\ {\isachardoublequoteopen}{\isasymAnd}i{\isachardot}{\kern0pt}\ emeasure\ M\ {\isacharbraceleft}{\kern0pt}y\ {\isasymin}\ space\ M{\isachardot}{\kern0pt}\ s\ i\ y\ {\isasymnoteq}\ {\isadigit{0}}{\isacharbraceright}{\kern0pt}\ {\isasymnoteq}\ {\isasyminfinity}{\isachardoublequoteclose}\isanewline
\ \ \ \ \ \ \isakeyword{and}\ {\isachardoublequoteopen}{\isasymAnd}x{\isachardot}{\kern0pt}\ x\ {\isasymin}\ space\ M\ {\isasymLongrightarrow}\ {\isacharparenleft}{\kern0pt}{\isasymlambda}i{\isachardot}{\kern0pt}\ s\ i\ x{\isacharparenright}{\kern0pt}\ {\isasymlonglonglongrightarrow}\ f\ x{\isachardoublequoteclose}\isanewline
\ \ \ \ \ \ \isakeyword{and}\ {\isachardoublequoteopen}{\isasymAnd}x\ i{\isachardot}{\kern0pt}\ x\ {\isasymin}\ space\ M\ {\isasymLongrightarrow}\ norm\ {\isacharparenleft}{\kern0pt}s\ i\ x{\isacharparenright}{\kern0pt}\ {\isasymle}\ {\isadigit{2}}\ {\isacharasterisk}{\kern0pt}\ norm\ {\isacharparenleft}{\kern0pt}f\ x{\isacharparenright}{\kern0pt}{\isachardoublequoteclose}\isanewline
%
\isadelimproof
%
\endisadelimproof
%
\isatagproof
\isacommand{proof}\isamarkupfalse%
{\isacharminus}{\kern0pt}\isanewline
\ \ \isacommand{have}\isamarkupfalse%
\ f{\isacharcolon}{\kern0pt}\ {\isachardoublequoteopen}f\ {\isasymin}\ borel{\isacharunderscore}{\kern0pt}measurable\ M{\isachardoublequoteclose}\ {\isachardoublequoteopen}{\isacharparenleft}{\kern0pt}{\isasymintegral}\isactrlsup {\isacharplus}{\kern0pt}x{\isachardot}{\kern0pt}\ norm\ {\isacharparenleft}{\kern0pt}f\ x{\isacharparenright}{\kern0pt}\ {\isasympartial}M{\isacharparenright}{\kern0pt}\ {\isacharless}{\kern0pt}\ {\isasyminfinity}{\isachardoublequoteclose}\ \isacommand{using}\isamarkupfalse%
\ assms\ \isacommand{unfolding}\isamarkupfalse%
\ integrable{\isacharunderscore}{\kern0pt}iff{\isacharunderscore}{\kern0pt}bounded\ \isacommand{by}\isamarkupfalse%
\ auto\isanewline
\ \ \isacommand{obtain}\isamarkupfalse%
\ s\ \isakeyword{where}\ s{\isacharcolon}{\kern0pt}\ {\isachardoublequoteopen}{\isasymAnd}i{\isachardot}{\kern0pt}\ simple{\isacharunderscore}{\kern0pt}function\ M\ {\isacharparenleft}{\kern0pt}s\ i{\isacharparenright}{\kern0pt}{\isachardoublequoteclose}\ {\isachardoublequoteopen}{\isasymAnd}x{\isachardot}{\kern0pt}\ x\ {\isasymin}\ space\ M\ {\isasymLongrightarrow}\ {\isacharparenleft}{\kern0pt}{\isasymlambda}i{\isachardot}{\kern0pt}\ s\ i\ x{\isacharparenright}{\kern0pt}\ {\isasymlonglonglongrightarrow}\ f\ x{\isachardoublequoteclose}\ {\isachardoublequoteopen}{\isasymAnd}i\ x{\isachardot}{\kern0pt}\ x\ {\isasymin}\ space\ M\ {\isasymLongrightarrow}\ norm\ {\isacharparenleft}{\kern0pt}s\ i\ x{\isacharparenright}{\kern0pt}\ {\isasymle}\ {\isadigit{2}}\ {\isacharasterisk}{\kern0pt}\ norm\ {\isacharparenleft}{\kern0pt}f\ x{\isacharparenright}{\kern0pt}{\isachardoublequoteclose}\ \isacommand{using}\isamarkupfalse%
\ borel{\isacharunderscore}{\kern0pt}measurable{\isacharunderscore}{\kern0pt}implies{\isacharunderscore}{\kern0pt}sequence{\isacharunderscore}{\kern0pt}metric{\isacharbrackleft}{\kern0pt}OF\ f{\isacharparenleft}{\kern0pt}{\isadigit{1}}{\isacharparenright}{\kern0pt}{\isacharbrackright}{\kern0pt}\ \isacommand{unfolding}\isamarkupfalse%
\ norm{\isacharunderscore}{\kern0pt}conv{\isacharunderscore}{\kern0pt}dist\ \isacommand{by}\isamarkupfalse%
\ metis\isanewline
\ \ \isacommand{{\isacharbraceleft}{\kern0pt}}\isamarkupfalse%
\isanewline
\ \ \ \ \isacommand{fix}\isamarkupfalse%
\ i\isanewline
\ \ \ \ \isacommand{have}\isamarkupfalse%
\ {\isachardoublequoteopen}{\isacharparenleft}{\kern0pt}{\isasymintegral}\isactrlsup {\isacharplus}{\kern0pt}x{\isachardot}{\kern0pt}\ norm\ {\isacharparenleft}{\kern0pt}s\ i\ x{\isacharparenright}{\kern0pt}\ {\isasympartial}M{\isacharparenright}{\kern0pt}\ {\isasymle}\ {\isacharparenleft}{\kern0pt}{\isasymintegral}\isactrlsup {\isacharplus}{\kern0pt}x{\isachardot}{\kern0pt}\ ennreal\ {\isacharparenleft}{\kern0pt}{\isadigit{2}}\ {\isacharasterisk}{\kern0pt}\ norm\ {\isacharparenleft}{\kern0pt}f\ x{\isacharparenright}{\kern0pt}{\isacharparenright}{\kern0pt}\ {\isasympartial}M{\isacharparenright}{\kern0pt}{\isachardoublequoteclose}\ \isacommand{using}\isamarkupfalse%
\ s\ \isacommand{by}\isamarkupfalse%
\ {\isacharparenleft}{\kern0pt}intro\ nn{\isacharunderscore}{\kern0pt}integral{\isacharunderscore}{\kern0pt}mono{\isacharcomma}{\kern0pt}\ auto{\isacharparenright}{\kern0pt}\isanewline
\ \ \ \ \isacommand{also}\isamarkupfalse%
\ \isacommand{have}\isamarkupfalse%
\ {\isachardoublequoteopen}{\isasymdots}\ {\isacharless}{\kern0pt}\ {\isasyminfinity}{\isachardoublequoteclose}\ \isacommand{using}\isamarkupfalse%
\ f\ \isacommand{by}\isamarkupfalse%
\ {\isacharparenleft}{\kern0pt}simp\ add{\isacharcolon}{\kern0pt}\ nn{\isacharunderscore}{\kern0pt}integral{\isacharunderscore}{\kern0pt}cmult\ ennreal{\isacharunderscore}{\kern0pt}mult{\isacharunderscore}{\kern0pt}less{\isacharunderscore}{\kern0pt}top\ ennreal{\isacharunderscore}{\kern0pt}mult{\isacharparenright}{\kern0pt}\isanewline
\ \ \ \ \isacommand{finally}\isamarkupfalse%
\ \isacommand{have}\isamarkupfalse%
\ sbi{\isacharcolon}{\kern0pt}\ {\isachardoublequoteopen}Bochner{\isacharunderscore}{\kern0pt}Integration{\isachardot}{\kern0pt}simple{\isacharunderscore}{\kern0pt}bochner{\isacharunderscore}{\kern0pt}integrable\ M\ {\isacharparenleft}{\kern0pt}s\ i{\isacharparenright}{\kern0pt}{\isachardoublequoteclose}\ \isacommand{using}\isamarkupfalse%
\ s\ \isacommand{by}\isamarkupfalse%
\ {\isacharparenleft}{\kern0pt}intro\ simple{\isacharunderscore}{\kern0pt}bochner{\isacharunderscore}{\kern0pt}integrableI{\isacharunderscore}{\kern0pt}bounded{\isacharparenright}{\kern0pt}\ auto\isanewline
\ \ \ \ \isacommand{hence}\isamarkupfalse%
\ {\isachardoublequoteopen}emeasure\ M\ {\isacharbraceleft}{\kern0pt}y\ {\isasymin}\ space\ M{\isachardot}{\kern0pt}\ s\ i\ y\ {\isasymnoteq}\ {\isadigit{0}}{\isacharbraceright}{\kern0pt}\ {\isasymnoteq}\ {\isasyminfinity}{\isachardoublequoteclose}\ \isacommand{by}\isamarkupfalse%
\ {\isacharparenleft}{\kern0pt}auto\ intro{\isacharcolon}{\kern0pt}\ integrableI{\isacharunderscore}{\kern0pt}simple{\isacharunderscore}{\kern0pt}bochner{\isacharunderscore}{\kern0pt}integrable\ simple{\isacharunderscore}{\kern0pt}bochner{\isacharunderscore}{\kern0pt}integrable{\isachardot}{\kern0pt}cases{\isacharparenright}{\kern0pt}\isanewline
\ \ \isacommand{{\isacharbraceright}{\kern0pt}}\isamarkupfalse%
\isanewline
\ \ \isacommand{thus}\isamarkupfalse%
\ {\isacharquery}{\kern0pt}thesis\ \isacommand{using}\isamarkupfalse%
\ that\ s\ \isacommand{by}\isamarkupfalse%
\ blast\isanewline
\isacommand{qed}\isamarkupfalse%
%
\endisatagproof
{\isafoldproof}%
%
\isadelimproof
\isanewline
%
\endisadelimproof
\isanewline
\isacommand{lemma}\isamarkupfalse%
\ simple{\isacharunderscore}{\kern0pt}function{\isacharunderscore}{\kern0pt}indicator{\isacharunderscore}{\kern0pt}representation{\isacharcolon}{\kern0pt}\isanewline
\ \ \isakeyword{fixes}\ f\ {\isacharcolon}{\kern0pt}{\isacharcolon}{\kern0pt}{\isachardoublequoteopen}{\isacharprime}{\kern0pt}a\ {\isasymRightarrow}\ {\isacharprime}{\kern0pt}b\ {\isacharcolon}{\kern0pt}{\isacharcolon}{\kern0pt}\ {\isacharbraceleft}{\kern0pt}second{\isacharunderscore}{\kern0pt}countable{\isacharunderscore}{\kern0pt}topology{\isacharcomma}{\kern0pt}\ banach{\isacharbraceright}{\kern0pt}{\isachardoublequoteclose}\isanewline
\ \ \isakeyword{assumes}\ f{\isacharcolon}{\kern0pt}\ {\isachardoublequoteopen}simple{\isacharunderscore}{\kern0pt}function\ M\ f{\isachardoublequoteclose}\ \isakeyword{and}\ x{\isacharcolon}{\kern0pt}\ {\isachardoublequoteopen}x\ {\isasymin}\ space\ M{\isachardoublequoteclose}\isanewline
\ \ \isakeyword{shows}\ {\isachardoublequoteopen}f\ x\ {\isacharequal}{\kern0pt}\ {\isacharparenleft}{\kern0pt}{\isasymSum}y\ {\isasymin}\ f\ {\isacharbackquote}{\kern0pt}\ space\ M{\isachardot}{\kern0pt}\ indicator\ {\isacharparenleft}{\kern0pt}f\ {\isacharminus}{\kern0pt}{\isacharbackquote}{\kern0pt}\ {\isacharbraceleft}{\kern0pt}y{\isacharbraceright}{\kern0pt}\ {\isasyminter}\ space\ M{\isacharparenright}{\kern0pt}\ x\ {\isacharasterisk}{\kern0pt}\isactrlsub R\ y{\isacharparenright}{\kern0pt}{\isachardoublequoteclose}\isanewline
\ \ {\isacharparenleft}{\kern0pt}\isakeyword{is}\ {\isachardoublequoteopen}{\isacharquery}{\kern0pt}l\ {\isacharequal}{\kern0pt}\ {\isacharquery}{\kern0pt}r{\isachardoublequoteclose}{\isacharparenright}{\kern0pt}\isanewline
%
\isadelimproof
%
\endisadelimproof
%
\isatagproof
\isacommand{proof}\isamarkupfalse%
\ {\isacharminus}{\kern0pt}\isanewline
\ \ \isacommand{have}\isamarkupfalse%
\ {\isachardoublequoteopen}{\isacharquery}{\kern0pt}r\ {\isacharequal}{\kern0pt}\ {\isacharparenleft}{\kern0pt}{\isasymSum}y\ {\isasymin}\ f\ {\isacharbackquote}{\kern0pt}\ space\ M{\isachardot}{\kern0pt}\isanewline
\ \ \ \ {\isacharparenleft}{\kern0pt}if\ y\ {\isacharequal}{\kern0pt}\ f\ x\ then\ indicator\ {\isacharparenleft}{\kern0pt}f\ {\isacharminus}{\kern0pt}{\isacharbackquote}{\kern0pt}\ {\isacharbraceleft}{\kern0pt}y{\isacharbraceright}{\kern0pt}\ {\isasyminter}\ space\ M{\isacharparenright}{\kern0pt}\ x\ {\isacharasterisk}{\kern0pt}\isactrlsub R\ y\ else\ {\isadigit{0}}{\isacharparenright}{\kern0pt}{\isacharparenright}{\kern0pt}{\isachardoublequoteclose}\ \isacommand{by}\isamarkupfalse%
\ {\isacharparenleft}{\kern0pt}auto\ intro{\isacharbang}{\kern0pt}{\isacharcolon}{\kern0pt}\ sum{\isachardot}{\kern0pt}cong{\isacharparenright}{\kern0pt}\isanewline
\ \ \isacommand{also}\isamarkupfalse%
\ \isacommand{have}\isamarkupfalse%
\ {\isachardoublequoteopen}{\isachardot}{\kern0pt}{\isachardot}{\kern0pt}{\isachardot}{\kern0pt}\ {\isacharequal}{\kern0pt}\ \ indicator\ {\isacharparenleft}{\kern0pt}f\ {\isacharminus}{\kern0pt}{\isacharbackquote}{\kern0pt}\ {\isacharbraceleft}{\kern0pt}f\ x{\isacharbraceright}{\kern0pt}\ {\isasyminter}\ space\ M{\isacharparenright}{\kern0pt}\ x\ {\isacharasterisk}{\kern0pt}\isactrlsub R\ f\ x{\isachardoublequoteclose}\ \isacommand{using}\isamarkupfalse%
\ assms\ \isacommand{by}\isamarkupfalse%
\ {\isacharparenleft}{\kern0pt}auto\ dest{\isacharcolon}{\kern0pt}\ simple{\isacharunderscore}{\kern0pt}functionD{\isacharparenright}{\kern0pt}\isanewline
\ \ \isacommand{also}\isamarkupfalse%
\ \isacommand{have}\isamarkupfalse%
\ {\isachardoublequoteopen}{\isachardot}{\kern0pt}{\isachardot}{\kern0pt}{\isachardot}{\kern0pt}\ {\isacharequal}{\kern0pt}\ f\ x{\isachardoublequoteclose}\ \isacommand{using}\isamarkupfalse%
\ x\ \isacommand{by}\isamarkupfalse%
\ {\isacharparenleft}{\kern0pt}auto\ simp{\isacharcolon}{\kern0pt}\ indicator{\isacharunderscore}{\kern0pt}def{\isacharparenright}{\kern0pt}\isanewline
\ \ \isacommand{finally}\isamarkupfalse%
\ \isacommand{show}\isamarkupfalse%
\ {\isacharquery}{\kern0pt}thesis\ \isacommand{by}\isamarkupfalse%
\ auto\isanewline
\isacommand{qed}\isamarkupfalse%
%
\endisatagproof
{\isafoldproof}%
%
\isadelimproof
\isanewline
%
\endisadelimproof
\isanewline
\isacommand{lemma}\isamarkupfalse%
\ simple{\isacharunderscore}{\kern0pt}function{\isacharunderscore}{\kern0pt}indicator{\isacharunderscore}{\kern0pt}representation{\isacharunderscore}{\kern0pt}AE{\isacharcolon}{\kern0pt}\isanewline
\ \ \isakeyword{fixes}\ f\ {\isacharcolon}{\kern0pt}{\isacharcolon}{\kern0pt}{\isachardoublequoteopen}{\isacharprime}{\kern0pt}a\ {\isasymRightarrow}\ {\isacharprime}{\kern0pt}b\ {\isacharcolon}{\kern0pt}{\isacharcolon}{\kern0pt}\ {\isacharbraceleft}{\kern0pt}second{\isacharunderscore}{\kern0pt}countable{\isacharunderscore}{\kern0pt}topology{\isacharcomma}{\kern0pt}\ banach{\isacharbraceright}{\kern0pt}{\isachardoublequoteclose}\isanewline
\ \ \isakeyword{assumes}\ f{\isacharcolon}{\kern0pt}\ {\isachardoublequoteopen}simple{\isacharunderscore}{\kern0pt}function\ M\ f{\isachardoublequoteclose}\isanewline
\ \ \isakeyword{shows}\ {\isachardoublequoteopen}AE\ x\ in\ M{\isachardot}{\kern0pt}\ f\ x\ {\isacharequal}{\kern0pt}\ {\isacharparenleft}{\kern0pt}{\isasymSum}y\ {\isasymin}\ f\ {\isacharbackquote}{\kern0pt}\ space\ M{\isachardot}{\kern0pt}\ indicator\ {\isacharparenleft}{\kern0pt}f\ {\isacharminus}{\kern0pt}{\isacharbackquote}{\kern0pt}\ {\isacharbraceleft}{\kern0pt}y{\isacharbraceright}{\kern0pt}\ {\isasyminter}\ space\ M{\isacharparenright}{\kern0pt}\ x\ {\isacharasterisk}{\kern0pt}\isactrlsub R\ y{\isacharparenright}{\kern0pt}{\isachardoublequoteclose}\ \ \isanewline
%
\isadelimproof
\ \ %
\endisadelimproof
%
\isatagproof
\isacommand{by}\isamarkupfalse%
\ {\isacharparenleft}{\kern0pt}metis\ {\isacharparenleft}{\kern0pt}mono{\isacharunderscore}{\kern0pt}tags{\isacharcomma}{\kern0pt}\ lifting{\isacharparenright}{\kern0pt}\ AE{\isacharunderscore}{\kern0pt}I{\isadigit{2}}\ simple{\isacharunderscore}{\kern0pt}function{\isacharunderscore}{\kern0pt}indicator{\isacharunderscore}{\kern0pt}representation\ f{\isacharparenright}{\kern0pt}%
\endisatagproof
{\isafoldproof}%
%
\isadelimproof
\isanewline
%
\endisadelimproof
\isanewline
\isacommand{lemmas}\isamarkupfalse%
\ simple{\isacharunderscore}{\kern0pt}function{\isacharunderscore}{\kern0pt}scaleR{\isacharbrackleft}{\kern0pt}intro{\isacharbrackright}{\kern0pt}\ {\isacharequal}{\kern0pt}\ simple{\isacharunderscore}{\kern0pt}function{\isacharunderscore}{\kern0pt}compose{\isadigit{2}}{\isacharbrackleft}{\kern0pt}\isakeyword{where}\ h{\isacharequal}{\kern0pt}{\isachardoublequoteopen}{\isacharparenleft}{\kern0pt}{\isacharasterisk}{\kern0pt}\isactrlsub R{\isacharparenright}{\kern0pt}{\isachardoublequoteclose}{\isacharbrackright}{\kern0pt}\isanewline
\isacommand{lemmas}\isamarkupfalse%
\ integrable{\isacharunderscore}{\kern0pt}simple{\isacharunderscore}{\kern0pt}function\ {\isacharequal}{\kern0pt}\ simple{\isacharunderscore}{\kern0pt}bochner{\isacharunderscore}{\kern0pt}integrable{\isachardot}{\kern0pt}intros{\isacharbrackleft}{\kern0pt}THEN\ has{\isacharunderscore}{\kern0pt}bochner{\isacharunderscore}{\kern0pt}integral{\isacharunderscore}{\kern0pt}simple{\isacharunderscore}{\kern0pt}bochner{\isacharunderscore}{\kern0pt}integrable{\isacharcomma}{\kern0pt}\ THEN\ integrable{\isachardot}{\kern0pt}intros{\isacharbrackright}{\kern0pt}\ \isanewline
\isanewline
\isanewline
%
\isadelimimportant
\isanewline
%
\endisadelimimportant
%
\isatagimportant
\isacommand{lemma}\isamarkupfalse%
\ simple{\isacharunderscore}{\kern0pt}integrable{\isacharunderscore}{\kern0pt}function{\isacharunderscore}{\kern0pt}induct{\isacharbrackleft}{\kern0pt}consumes\ {\isadigit{2}}{\isacharcomma}{\kern0pt}\ case{\isacharunderscore}{\kern0pt}names\ cong\ indicator\ add{\isacharcomma}{\kern0pt}\ induct\ set{\isacharcolon}{\kern0pt}\ simple{\isacharunderscore}{\kern0pt}function{\isacharbrackright}{\kern0pt}{\isacharcolon}{\kern0pt}\isanewline
\ \ \isakeyword{fixes}\ f\ {\isacharcolon}{\kern0pt}{\isacharcolon}{\kern0pt}\ {\isachardoublequoteopen}{\isacharprime}{\kern0pt}a\ {\isasymRightarrow}\ {\isacharprime}{\kern0pt}b\ {\isacharcolon}{\kern0pt}{\isacharcolon}{\kern0pt}\ {\isacharbraceleft}{\kern0pt}second{\isacharunderscore}{\kern0pt}countable{\isacharunderscore}{\kern0pt}topology{\isacharcomma}{\kern0pt}\ banach{\isacharbraceright}{\kern0pt}{\isachardoublequoteclose}\isanewline
\ \ \isakeyword{assumes}\ f{\isacharcolon}{\kern0pt}\ {\isachardoublequoteopen}simple{\isacharunderscore}{\kern0pt}function\ M\ f{\isachardoublequoteclose}\ {\isachardoublequoteopen}emeasure\ M\ {\isacharbraceleft}{\kern0pt}y\ {\isasymin}\ space\ M{\isachardot}{\kern0pt}\ f\ y\ {\isasymnoteq}\ {\isadigit{0}}{\isacharbraceright}{\kern0pt}\ {\isasymnoteq}\ {\isasyminfinity}{\isachardoublequoteclose}\isanewline
\ \ \isakeyword{assumes}\ cong{\isacharcolon}{\kern0pt}\ {\isachardoublequoteopen}{\isasymAnd}f\ g{\isachardot}{\kern0pt}\ simple{\isacharunderscore}{\kern0pt}function\ M\ f\ {\isasymLongrightarrow}\ emeasure\ M\ {\isacharbraceleft}{\kern0pt}y\ {\isasymin}\ space\ M{\isachardot}{\kern0pt}\ f\ y\ {\isasymnoteq}\ {\isadigit{0}}{\isacharbraceright}{\kern0pt}\ {\isasymnoteq}\ {\isasyminfinity}\ \isanewline
\ \ \ \ \ \ \ \ \ \ \ \ \ \ \ \ \ \ \ \ {\isasymLongrightarrow}\ simple{\isacharunderscore}{\kern0pt}function\ M\ g\ {\isasymLongrightarrow}\ emeasure\ M\ {\isacharbraceleft}{\kern0pt}y\ {\isasymin}\ space\ M{\isachardot}{\kern0pt}\ g\ y\ {\isasymnoteq}\ {\isadigit{0}}{\isacharbraceright}{\kern0pt}\ {\isasymnoteq}\ {\isasyminfinity}\ \isanewline
\ \ \ \ \ \ \ \ \ \ \ \ \ \ \ \ \ \ \ \ {\isasymLongrightarrow}\ {\isacharparenleft}{\kern0pt}{\isasymAnd}x{\isachardot}{\kern0pt}\ x\ {\isasymin}\ space\ M\ {\isasymLongrightarrow}\ f\ x\ {\isacharequal}{\kern0pt}\ g\ x{\isacharparenright}{\kern0pt}\ {\isasymLongrightarrow}\ P\ f\ {\isasymLongrightarrow}\ P\ g{\isachardoublequoteclose}\isanewline
\ \ \isakeyword{assumes}\ indicator{\isacharcolon}{\kern0pt}\ {\isachardoublequoteopen}{\isasymAnd}A\ y{\isachardot}{\kern0pt}\ A\ {\isasymin}\ sets\ M\ {\isasymLongrightarrow}\ emeasure\ M\ A\ {\isacharless}{\kern0pt}\ {\isasyminfinity}\ {\isasymLongrightarrow}\ P\ {\isacharparenleft}{\kern0pt}{\isasymlambda}x{\isachardot}{\kern0pt}\ indicator\ A\ x\ {\isacharasterisk}{\kern0pt}\isactrlsub R\ y{\isacharparenright}{\kern0pt}{\isachardoublequoteclose}\isanewline
\ \ \isakeyword{assumes}\ add{\isacharcolon}{\kern0pt}\ {\isachardoublequoteopen}{\isasymAnd}f\ g{\isachardot}{\kern0pt}\ simple{\isacharunderscore}{\kern0pt}function\ M\ f\ {\isasymLongrightarrow}\ emeasure\ M\ {\isacharbraceleft}{\kern0pt}y\ {\isasymin}\ space\ M{\isachardot}{\kern0pt}\ f\ y\ {\isasymnoteq}\ {\isadigit{0}}{\isacharbraceright}{\kern0pt}\ {\isasymnoteq}\ {\isasyminfinity}\ {\isasymLongrightarrow}\ \isanewline
\ \ \ \ \ \ \ \ \ \ \ \ \ \ \ \ \ \ \ \ \ \ simple{\isacharunderscore}{\kern0pt}function\ M\ g\ {\isasymLongrightarrow}\ emeasure\ M\ {\isacharbraceleft}{\kern0pt}y\ {\isasymin}\ space\ M{\isachardot}{\kern0pt}\ g\ y\ {\isasymnoteq}\ {\isadigit{0}}{\isacharbraceright}{\kern0pt}\ {\isasymnoteq}\ {\isasyminfinity}\ {\isasymLongrightarrow}\ \isanewline
\ \ \ \ \ \ \ \ \ \ \ \ \ \ \ \ \ \ \ \ \ \ {\isacharparenleft}{\kern0pt}{\isasymAnd}z{\isachardot}{\kern0pt}\ z\ {\isasymin}\ space\ M\ {\isasymLongrightarrow}\ norm\ {\isacharparenleft}{\kern0pt}f\ z\ {\isacharplus}{\kern0pt}\ g\ z{\isacharparenright}{\kern0pt}\ {\isacharequal}{\kern0pt}\ norm\ {\isacharparenleft}{\kern0pt}f\ z{\isacharparenright}{\kern0pt}\ {\isacharplus}{\kern0pt}\ norm\ {\isacharparenleft}{\kern0pt}g\ z{\isacharparenright}{\kern0pt}{\isacharparenright}{\kern0pt}\ {\isasymLongrightarrow}\isanewline
\ \ \ \ \ \ \ \ \ \ \ \ \ \ \ \ \ \ \ \ \ \ P\ f\ {\isasymLongrightarrow}\ P\ g\ {\isasymLongrightarrow}\ P\ {\isacharparenleft}{\kern0pt}{\isasymlambda}x{\isachardot}{\kern0pt}\ f\ x\ {\isacharplus}{\kern0pt}\ g\ x{\isacharparenright}{\kern0pt}{\isachardoublequoteclose}\isanewline
\ \ \isakeyword{shows}\ {\isachardoublequoteopen}P\ f{\isachardoublequoteclose}%
\endisatagimportant
{\isafoldimportant}%
%
\isadelimimportant
\isanewline
%
\endisadelimimportant
%
\isadelimproof
%
\endisadelimproof
%
\isatagproof
\isacommand{proof}\isamarkupfalse%
{\isacharminus}{\kern0pt}\isanewline
\ \ \isacommand{let}\isamarkupfalse%
\ {\isacharquery}{\kern0pt}f\ {\isacharequal}{\kern0pt}\ {\isachardoublequoteopen}{\isasymlambda}x{\isachardot}{\kern0pt}\ {\isacharparenleft}{\kern0pt}{\isasymSum}y{\isasymin}f\ {\isacharbackquote}{\kern0pt}\ space\ M{\isachardot}{\kern0pt}\ indicat{\isacharunderscore}{\kern0pt}real\ {\isacharparenleft}{\kern0pt}f\ {\isacharminus}{\kern0pt}{\isacharbackquote}{\kern0pt}\ {\isacharbraceleft}{\kern0pt}y{\isacharbraceright}{\kern0pt}\ {\isasyminter}\ space\ M{\isacharparenright}{\kern0pt}\ x\ {\isacharasterisk}{\kern0pt}\isactrlsub R\ y{\isacharparenright}{\kern0pt}{\isachardoublequoteclose}\isanewline
\ \ \isacommand{have}\isamarkupfalse%
\ f{\isacharunderscore}{\kern0pt}ae{\isacharunderscore}{\kern0pt}eq{\isacharcolon}{\kern0pt}\ {\isachardoublequoteopen}f\ x\ {\isacharequal}{\kern0pt}\ {\isacharquery}{\kern0pt}f\ x{\isachardoublequoteclose}\ \isakeyword{if}\ {\isachardoublequoteopen}x\ {\isasymin}\ space\ M{\isachardoublequoteclose}\ \isakeyword{for}\ x\ \isacommand{using}\isamarkupfalse%
\ simple{\isacharunderscore}{\kern0pt}function{\isacharunderscore}{\kern0pt}indicator{\isacharunderscore}{\kern0pt}representation{\isacharbrackleft}{\kern0pt}OF\ f{\isacharparenleft}{\kern0pt}{\isadigit{1}}{\isacharparenright}{\kern0pt}\ that{\isacharbrackright}{\kern0pt}\ \isacommand{{\isachardot}{\kern0pt}}\isamarkupfalse%
\isanewline
\ \ \isacommand{moreover}\isamarkupfalse%
\ \isacommand{have}\isamarkupfalse%
\ {\isachardoublequoteopen}emeasure\ M\ {\isacharbraceleft}{\kern0pt}y\ {\isasymin}\ space\ M{\isachardot}{\kern0pt}\ {\isacharquery}{\kern0pt}f\ y\ {\isasymnoteq}\ {\isadigit{0}}{\isacharbraceright}{\kern0pt}\ {\isasymnoteq}\ {\isasyminfinity}{\isachardoublequoteclose}\ \isacommand{by}\isamarkupfalse%
\ {\isacharparenleft}{\kern0pt}metis\ {\isacharparenleft}{\kern0pt}no{\isacharunderscore}{\kern0pt}types{\isacharcomma}{\kern0pt}\ lifting{\isacharparenright}{\kern0pt}\ Collect{\isacharunderscore}{\kern0pt}cong\ calculation\ f{\isacharparenleft}{\kern0pt}{\isadigit{2}}{\isacharparenright}{\kern0pt}{\isacharparenright}{\kern0pt}\isanewline
\ \ \isacommand{moreover}\isamarkupfalse%
\ \isacommand{have}\isamarkupfalse%
\ {\isachardoublequoteopen}P\ {\isacharparenleft}{\kern0pt}{\isasymlambda}x{\isachardot}{\kern0pt}\ {\isasymSum}y{\isasymin}S{\isachardot}{\kern0pt}\ indicat{\isacharunderscore}{\kern0pt}real\ {\isacharparenleft}{\kern0pt}f\ {\isacharminus}{\kern0pt}{\isacharbackquote}{\kern0pt}\ {\isacharbraceleft}{\kern0pt}y{\isacharbraceright}{\kern0pt}\ {\isasyminter}\ space\ M{\isacharparenright}{\kern0pt}\ x\ {\isacharasterisk}{\kern0pt}\isactrlsub R\ y{\isacharparenright}{\kern0pt}{\isachardoublequoteclose}\isanewline
\ \ \ \ \ \ \ \ \ \ \ \ \ \ \ \ {\isachardoublequoteopen}simple{\isacharunderscore}{\kern0pt}function\ M\ {\isacharparenleft}{\kern0pt}{\isasymlambda}x{\isachardot}{\kern0pt}\ {\isasymSum}y{\isasymin}S{\isachardot}{\kern0pt}\ indicat{\isacharunderscore}{\kern0pt}real\ {\isacharparenleft}{\kern0pt}f\ {\isacharminus}{\kern0pt}{\isacharbackquote}{\kern0pt}\ {\isacharbraceleft}{\kern0pt}y{\isacharbraceright}{\kern0pt}\ {\isasyminter}\ space\ M{\isacharparenright}{\kern0pt}\ x\ {\isacharasterisk}{\kern0pt}\isactrlsub R\ y{\isacharparenright}{\kern0pt}{\isachardoublequoteclose}\isanewline
\ \ \ \ \ \ \ \ \ \ \ \ \ \ \ \ {\isachardoublequoteopen}emeasure\ M\ {\isacharbraceleft}{\kern0pt}y\ {\isasymin}\ space\ M{\isachardot}{\kern0pt}\ {\isacharparenleft}{\kern0pt}{\isasymSum}x{\isasymin}S{\isachardot}{\kern0pt}\ indicat{\isacharunderscore}{\kern0pt}real\ {\isacharparenleft}{\kern0pt}f\ {\isacharminus}{\kern0pt}{\isacharbackquote}{\kern0pt}\ {\isacharbraceleft}{\kern0pt}x{\isacharbraceright}{\kern0pt}\ {\isasyminter}\ space\ M{\isacharparenright}{\kern0pt}\ y\ {\isacharasterisk}{\kern0pt}\isactrlsub R\ x{\isacharparenright}{\kern0pt}\ {\isasymnoteq}\ {\isadigit{0}}{\isacharbraceright}{\kern0pt}\ {\isasymnoteq}\ {\isasyminfinity}{\isachardoublequoteclose}\isanewline
\ \ \ \ \ \ \ \ \ \ \ \ \ \ \ \ \isakeyword{if}\ {\isachardoublequoteopen}S\ {\isasymsubseteq}\ f\ {\isacharbackquote}{\kern0pt}\ space\ M{\isachardoublequoteclose}\ \isakeyword{for}\ S\ \isacommand{using}\isamarkupfalse%
\ simple{\isacharunderscore}{\kern0pt}functionD{\isacharparenleft}{\kern0pt}{\isadigit{1}}{\isacharparenright}{\kern0pt}{\isacharbrackleft}{\kern0pt}OF\ assms{\isacharparenleft}{\kern0pt}{\isadigit{1}}{\isacharparenright}{\kern0pt}{\isacharcomma}{\kern0pt}\ THEN\ rev{\isacharunderscore}{\kern0pt}finite{\isacharunderscore}{\kern0pt}subset{\isacharcomma}{\kern0pt}\ OF\ that{\isacharbrackright}{\kern0pt}\ that\ \isanewline
\ \ \isacommand{proof}\isamarkupfalse%
\ {\isacharparenleft}{\kern0pt}induction\ rule{\isacharcolon}{\kern0pt}\ finite{\isacharunderscore}{\kern0pt}induct{\isacharparenright}{\kern0pt}\isanewline
\ \ \ \ \isacommand{case}\isamarkupfalse%
\ empty\isanewline
\ \ \ \ \isacommand{{\isacharbraceleft}{\kern0pt}}\isamarkupfalse%
\isanewline
\ \ \ \ \ \ \isacommand{case}\isamarkupfalse%
\ {\isadigit{1}}\isanewline
\ \ \ \ \ \ \isacommand{then}\isamarkupfalse%
\ \isacommand{show}\isamarkupfalse%
\ {\isacharquery}{\kern0pt}case\ \isacommand{using}\isamarkupfalse%
\ indicator{\isacharbrackleft}{\kern0pt}of\ {\isachardoublequoteopen}{\isacharbraceleft}{\kern0pt}{\isacharbraceright}{\kern0pt}{\isachardoublequoteclose}{\isacharbrackright}{\kern0pt}\ \isacommand{by}\isamarkupfalse%
\ force\isanewline
\ \ \ \ \isacommand{next}\isamarkupfalse%
\isanewline
\ \ \ \ \ \ \isacommand{case}\isamarkupfalse%
\ {\isadigit{2}}\isanewline
\ \ \ \ \ \ \isacommand{then}\isamarkupfalse%
\ \isacommand{show}\isamarkupfalse%
\ {\isacharquery}{\kern0pt}case\ \isacommand{by}\isamarkupfalse%
\ force\ \isanewline
\ \ \ \ \isacommand{next}\isamarkupfalse%
\isanewline
\ \ \ \ \ \ \isacommand{case}\isamarkupfalse%
\ {\isadigit{3}}\isanewline
\ \ \ \ \ \ \isacommand{then}\isamarkupfalse%
\ \isacommand{show}\isamarkupfalse%
\ {\isacharquery}{\kern0pt}case\ \isacommand{by}\isamarkupfalse%
\ force\ \isanewline
\ \ \ \ \isacommand{{\isacharbraceright}{\kern0pt}}\isamarkupfalse%
\isanewline
\ \ \isacommand{next}\isamarkupfalse%
\isanewline
\ \ \ \ \isacommand{case}\isamarkupfalse%
\ {\isacharparenleft}{\kern0pt}insert\ x\ F{\isacharparenright}{\kern0pt}\isanewline
\ \ \ \ \isacommand{have}\isamarkupfalse%
\ {\isachardoublequoteopen}{\isacharparenleft}{\kern0pt}f\ {\isacharminus}{\kern0pt}{\isacharbackquote}{\kern0pt}\ {\isacharbraceleft}{\kern0pt}x{\isacharbraceright}{\kern0pt}\ {\isasyminter}\ space\ M{\isacharparenright}{\kern0pt}\ {\isasymsubseteq}\ {\isacharbraceleft}{\kern0pt}y\ {\isasymin}\ space\ M{\isachardot}{\kern0pt}\ f\ y\ {\isasymnoteq}\ {\isadigit{0}}{\isacharbraceright}{\kern0pt}{\isachardoublequoteclose}\ \isakeyword{if}\ {\isachardoublequoteopen}x\ {\isasymnoteq}\ {\isadigit{0}}{\isachardoublequoteclose}\ \isacommand{using}\isamarkupfalse%
\ that\ \isacommand{by}\isamarkupfalse%
\ blast\isanewline
\ \ \ \ \isacommand{moreover}\isamarkupfalse%
\ \isacommand{have}\isamarkupfalse%
\ {\isachardoublequoteopen}{\isacharbraceleft}{\kern0pt}y\ {\isasymin}\ space\ M{\isachardot}{\kern0pt}\ f\ y\ {\isasymnoteq}\ {\isadigit{0}}{\isacharbraceright}{\kern0pt}\ {\isacharequal}{\kern0pt}\ space\ M\ {\isacharminus}{\kern0pt}\ {\isacharparenleft}{\kern0pt}f\ {\isacharminus}{\kern0pt}{\isacharbackquote}{\kern0pt}\ {\isacharbraceleft}{\kern0pt}{\isadigit{0}}{\isacharbraceright}{\kern0pt}\ {\isasyminter}\ space\ M{\isacharparenright}{\kern0pt}{\isachardoublequoteclose}\ \isacommand{by}\isamarkupfalse%
\ blast\isanewline
\ \ \ \ \isacommand{moreover}\isamarkupfalse%
\ \isacommand{have}\isamarkupfalse%
\ {\isachardoublequoteopen}space\ M\ {\isacharminus}{\kern0pt}\ {\isacharparenleft}{\kern0pt}f\ {\isacharminus}{\kern0pt}{\isacharbackquote}{\kern0pt}\ {\isacharbraceleft}{\kern0pt}{\isadigit{0}}{\isacharbraceright}{\kern0pt}\ {\isasyminter}\ space\ M{\isacharparenright}{\kern0pt}\ {\isasymin}\ sets\ M{\isachardoublequoteclose}\ \isacommand{using}\isamarkupfalse%
\ simple{\isacharunderscore}{\kern0pt}functionD{\isacharparenleft}{\kern0pt}{\isadigit{2}}{\isacharparenright}{\kern0pt}{\isacharbrackleft}{\kern0pt}OF\ f{\isacharparenleft}{\kern0pt}{\isadigit{1}}{\isacharparenright}{\kern0pt}{\isacharbrackright}{\kern0pt}\ \isacommand{by}\isamarkupfalse%
\ blast\isanewline
\ \ \ \ \isacommand{ultimately}\isamarkupfalse%
\ \isacommand{have}\isamarkupfalse%
\ fin{\isacharunderscore}{\kern0pt}{\isadigit{0}}{\isacharcolon}{\kern0pt}\ {\isachardoublequoteopen}emeasure\ M\ {\isacharparenleft}{\kern0pt}f\ {\isacharminus}{\kern0pt}{\isacharbackquote}{\kern0pt}\ {\isacharbraceleft}{\kern0pt}x{\isacharbraceright}{\kern0pt}\ {\isasyminter}\ space\ M{\isacharparenright}{\kern0pt}\ {\isacharless}{\kern0pt}\ {\isasyminfinity}{\isachardoublequoteclose}\ \isakeyword{if}\ {\isachardoublequoteopen}x\ {\isasymnoteq}\ {\isadigit{0}}{\isachardoublequoteclose}\ \isacommand{using}\isamarkupfalse%
\ that\ \isacommand{by}\isamarkupfalse%
\ {\isacharparenleft}{\kern0pt}metis\ emeasure{\isacharunderscore}{\kern0pt}mono\ f{\isacharparenleft}{\kern0pt}{\isadigit{2}}{\isacharparenright}{\kern0pt}\ infinity{\isacharunderscore}{\kern0pt}ennreal{\isacharunderscore}{\kern0pt}def\ top{\isachardot}{\kern0pt}not{\isacharunderscore}{\kern0pt}eq{\isacharunderscore}{\kern0pt}extremum\ top{\isacharunderscore}{\kern0pt}unique{\isacharparenright}{\kern0pt}\isanewline
\ \ \ \ \isacommand{hence}\isamarkupfalse%
\ fin{\isacharunderscore}{\kern0pt}{\isadigit{1}}{\isacharcolon}{\kern0pt}\ {\isachardoublequoteopen}emeasure\ M\ {\isacharbraceleft}{\kern0pt}y\ {\isasymin}\ space\ M{\isachardot}{\kern0pt}\ indicat{\isacharunderscore}{\kern0pt}real\ {\isacharparenleft}{\kern0pt}f\ {\isacharminus}{\kern0pt}{\isacharbackquote}{\kern0pt}\ {\isacharbraceleft}{\kern0pt}x{\isacharbraceright}{\kern0pt}\ {\isasyminter}\ space\ M{\isacharparenright}{\kern0pt}\ y\ {\isacharasterisk}{\kern0pt}\isactrlsub R\ x\ {\isasymnoteq}\ {\isadigit{0}}{\isacharbraceright}{\kern0pt}\ {\isasymnoteq}\ {\isasyminfinity}{\isachardoublequoteclose}\ \isakeyword{if}\ {\isachardoublequoteopen}x\ {\isasymnoteq}\ {\isadigit{0}}{\isachardoublequoteclose}\ \isacommand{by}\isamarkupfalse%
\ {\isacharparenleft}{\kern0pt}metis\ {\isacharparenleft}{\kern0pt}mono{\isacharunderscore}{\kern0pt}tags{\isacharcomma}{\kern0pt}\ lifting{\isacharparenright}{\kern0pt}\ emeasure{\isacharunderscore}{\kern0pt}mono\ f{\isacharparenleft}{\kern0pt}{\isadigit{1}}{\isacharparenright}{\kern0pt}\ indicator{\isacharunderscore}{\kern0pt}simps{\isacharparenleft}{\kern0pt}{\isadigit{2}}{\isacharparenright}{\kern0pt}\ linorder{\isacharunderscore}{\kern0pt}not{\isacharunderscore}{\kern0pt}less\ mem{\isacharunderscore}{\kern0pt}Collect{\isacharunderscore}{\kern0pt}eq\ scaleR{\isacharunderscore}{\kern0pt}eq{\isacharunderscore}{\kern0pt}{\isadigit{0}}{\isacharunderscore}{\kern0pt}iff\ simple{\isacharunderscore}{\kern0pt}functionD{\isacharparenleft}{\kern0pt}{\isadigit{2}}{\isacharparenright}{\kern0pt}\ subsetI\ that{\isacharparenright}{\kern0pt}\isanewline
\isanewline
\ \ \ \ \isacommand{have}\isamarkupfalse%
\ {\isacharasterisk}{\kern0pt}{\isacharcolon}{\kern0pt}\ {\isachardoublequoteopen}{\isacharparenleft}{\kern0pt}{\isasymSum}y{\isasymin}insert\ x\ F{\isachardot}{\kern0pt}\ indicat{\isacharunderscore}{\kern0pt}real\ {\isacharparenleft}{\kern0pt}f\ {\isacharminus}{\kern0pt}{\isacharbackquote}{\kern0pt}\ {\isacharbraceleft}{\kern0pt}y{\isacharbraceright}{\kern0pt}\ {\isasyminter}\ space\ M{\isacharparenright}{\kern0pt}\ xa\ {\isacharasterisk}{\kern0pt}\isactrlsub R\ y{\isacharparenright}{\kern0pt}\ {\isacharequal}{\kern0pt}\ {\isacharparenleft}{\kern0pt}{\isasymSum}y{\isasymin}F{\isachardot}{\kern0pt}\ indicat{\isacharunderscore}{\kern0pt}real\ {\isacharparenleft}{\kern0pt}f\ {\isacharminus}{\kern0pt}{\isacharbackquote}{\kern0pt}\ {\isacharbraceleft}{\kern0pt}y{\isacharbraceright}{\kern0pt}\ {\isasyminter}\ space\ M{\isacharparenright}{\kern0pt}\ xa\ {\isacharasterisk}{\kern0pt}\isactrlsub R\ y{\isacharparenright}{\kern0pt}\ {\isacharplus}{\kern0pt}\ indicat{\isacharunderscore}{\kern0pt}real\ {\isacharparenleft}{\kern0pt}f\ {\isacharminus}{\kern0pt}{\isacharbackquote}{\kern0pt}\ {\isacharbraceleft}{\kern0pt}x{\isacharbraceright}{\kern0pt}\ {\isasyminter}\ space\ M{\isacharparenright}{\kern0pt}\ xa\ {\isacharasterisk}{\kern0pt}\isactrlsub R\ x{\isachardoublequoteclose}\ \isakeyword{for}\ xa\ \isacommand{by}\isamarkupfalse%
\ {\isacharparenleft}{\kern0pt}metis\ {\isacharparenleft}{\kern0pt}no{\isacharunderscore}{\kern0pt}types{\isacharcomma}{\kern0pt}\ lifting{\isacharparenright}{\kern0pt}\ Diff{\isacharunderscore}{\kern0pt}empty\ Diff{\isacharunderscore}{\kern0pt}insert{\isadigit{0}}\ add{\isachardot}{\kern0pt}commute\ insert{\isachardot}{\kern0pt}hyps{\isacharparenleft}{\kern0pt}{\isadigit{1}}{\isacharparenright}{\kern0pt}\ insert{\isachardot}{\kern0pt}hyps{\isacharparenleft}{\kern0pt}{\isadigit{2}}{\isacharparenright}{\kern0pt}\ sum{\isachardot}{\kern0pt}insert{\isacharunderscore}{\kern0pt}remove{\isacharparenright}{\kern0pt}\isanewline
\ \ \ \ \isacommand{have}\isamarkupfalse%
\ {\isacharasterisk}{\kern0pt}{\isacharasterisk}{\kern0pt}{\isacharcolon}{\kern0pt}\ {\isachardoublequoteopen}{\isacharbraceleft}{\kern0pt}y\ {\isasymin}\ space\ M{\isachardot}{\kern0pt}\ {\isacharparenleft}{\kern0pt}{\isasymSum}x{\isasymin}insert\ x\ F{\isachardot}{\kern0pt}\ indicat{\isacharunderscore}{\kern0pt}real\ {\isacharparenleft}{\kern0pt}f\ {\isacharminus}{\kern0pt}{\isacharbackquote}{\kern0pt}\ {\isacharbraceleft}{\kern0pt}x{\isacharbraceright}{\kern0pt}\ {\isasyminter}\ space\ M{\isacharparenright}{\kern0pt}\ y\ {\isacharasterisk}{\kern0pt}\isactrlsub R\ x{\isacharparenright}{\kern0pt}\ {\isasymnoteq}\ {\isadigit{0}}{\isacharbraceright}{\kern0pt}\ {\isasymsubseteq}\ {\isacharbraceleft}{\kern0pt}y\ {\isasymin}\ space\ M{\isachardot}{\kern0pt}\ {\isacharparenleft}{\kern0pt}{\isasymSum}x{\isasymin}F{\isachardot}{\kern0pt}\ indicat{\isacharunderscore}{\kern0pt}real\ {\isacharparenleft}{\kern0pt}f\ {\isacharminus}{\kern0pt}{\isacharbackquote}{\kern0pt}\ {\isacharbraceleft}{\kern0pt}x{\isacharbraceright}{\kern0pt}\ {\isasyminter}\ space\ M{\isacharparenright}{\kern0pt}\ y\ {\isacharasterisk}{\kern0pt}\isactrlsub R\ x{\isacharparenright}{\kern0pt}\ {\isasymnoteq}\ {\isadigit{0}}{\isacharbraceright}{\kern0pt}\ {\isasymunion}\ {\isacharbraceleft}{\kern0pt}y\ {\isasymin}\ space\ M{\isachardot}{\kern0pt}\ indicat{\isacharunderscore}{\kern0pt}real\ {\isacharparenleft}{\kern0pt}f\ {\isacharminus}{\kern0pt}{\isacharbackquote}{\kern0pt}\ {\isacharbraceleft}{\kern0pt}x{\isacharbraceright}{\kern0pt}\ {\isasyminter}\ space\ M{\isacharparenright}{\kern0pt}\ y\ {\isacharasterisk}{\kern0pt}\isactrlsub R\ x\ {\isasymnoteq}\ {\isadigit{0}}{\isacharbraceright}{\kern0pt}{\isachardoublequoteclose}\ \isacommand{unfolding}\isamarkupfalse%
\ {\isacharasterisk}{\kern0pt}\ \isacommand{by}\isamarkupfalse%
\ fastforce\ \ \ \ \isanewline
\ \ \ \ \isacommand{{\isacharbraceleft}{\kern0pt}}\isamarkupfalse%
\isanewline
\ \ \ \ \ \ \isacommand{case}\isamarkupfalse%
\ {\isadigit{1}}\isanewline
\ \ \ \ \ \ \isacommand{hence}\isamarkupfalse%
\ x{\isacharcolon}{\kern0pt}\ {\isachardoublequoteopen}x\ {\isasymin}\ f\ {\isacharbackquote}{\kern0pt}\ space\ M{\isachardoublequoteclose}\ \isakeyword{and}\ F{\isacharcolon}{\kern0pt}\ {\isachardoublequoteopen}F\ {\isasymsubseteq}\ f\ {\isacharbackquote}{\kern0pt}\ space\ M{\isachardoublequoteclose}\ \isacommand{by}\isamarkupfalse%
\ auto\isanewline
\ \ \ \ \ \ \isacommand{show}\isamarkupfalse%
\ {\isacharquery}{\kern0pt}case\ \isanewline
\ \ \ \ \ \ \isacommand{proof}\isamarkupfalse%
\ {\isacharparenleft}{\kern0pt}cases\ {\isachardoublequoteopen}x\ {\isacharequal}{\kern0pt}\ {\isadigit{0}}{\isachardoublequoteclose}{\isacharparenright}{\kern0pt}\isanewline
\ \ \ \ \ \ \ \ \isacommand{case}\isamarkupfalse%
\ True\isanewline
\ \ \ \ \ \ \ \ \isacommand{then}\isamarkupfalse%
\ \isacommand{show}\isamarkupfalse%
\ {\isacharquery}{\kern0pt}thesis\ \isacommand{unfolding}\isamarkupfalse%
\ {\isacharasterisk}{\kern0pt}\ \isacommand{using}\isamarkupfalse%
\ insert{\isacharparenleft}{\kern0pt}{\isadigit{3}}{\isacharparenright}{\kern0pt}{\isacharbrackleft}{\kern0pt}OF\ F{\isacharbrackright}{\kern0pt}\ \isacommand{by}\isamarkupfalse%
\ simp\isanewline
\ \ \ \ \ \ \isacommand{next}\isamarkupfalse%
\isanewline
\ \ \ \ \ \ \ \ \isacommand{case}\isamarkupfalse%
\ False\isanewline
\ \ \ \ \ \ \ \ \isacommand{have}\isamarkupfalse%
\ norm{\isacharunderscore}{\kern0pt}argument{\isacharcolon}{\kern0pt}\ {\isachardoublequoteopen}norm\ {\isacharparenleft}{\kern0pt}{\isacharparenleft}{\kern0pt}{\isasymSum}y{\isasymin}F{\isachardot}{\kern0pt}\ indicat{\isacharunderscore}{\kern0pt}real\ {\isacharparenleft}{\kern0pt}f\ {\isacharminus}{\kern0pt}{\isacharbackquote}{\kern0pt}\ {\isacharbraceleft}{\kern0pt}y{\isacharbraceright}{\kern0pt}\ {\isasyminter}\ space\ M{\isacharparenright}{\kern0pt}\ z\ {\isacharasterisk}{\kern0pt}\isactrlsub R\ y{\isacharparenright}{\kern0pt}\ {\isacharplus}{\kern0pt}\ indicat{\isacharunderscore}{\kern0pt}real\ {\isacharparenleft}{\kern0pt}f\ {\isacharminus}{\kern0pt}{\isacharbackquote}{\kern0pt}\ {\isacharbraceleft}{\kern0pt}x{\isacharbraceright}{\kern0pt}\ {\isasyminter}\ space\ M{\isacharparenright}{\kern0pt}\ z\ {\isacharasterisk}{\kern0pt}\isactrlsub R\ x{\isacharparenright}{\kern0pt}\ {\isacharequal}{\kern0pt}\ norm\ {\isacharparenleft}{\kern0pt}{\isasymSum}y{\isasymin}F{\isachardot}{\kern0pt}\ indicat{\isacharunderscore}{\kern0pt}real\ {\isacharparenleft}{\kern0pt}f\ {\isacharminus}{\kern0pt}{\isacharbackquote}{\kern0pt}\ {\isacharbraceleft}{\kern0pt}y{\isacharbraceright}{\kern0pt}\ {\isasyminter}\ space\ M{\isacharparenright}{\kern0pt}\ z\ {\isacharasterisk}{\kern0pt}\isactrlsub R\ y{\isacharparenright}{\kern0pt}\ {\isacharplus}{\kern0pt}\ norm\ {\isacharparenleft}{\kern0pt}indicat{\isacharunderscore}{\kern0pt}real\ {\isacharparenleft}{\kern0pt}f\ {\isacharminus}{\kern0pt}{\isacharbackquote}{\kern0pt}\ {\isacharbraceleft}{\kern0pt}x{\isacharbraceright}{\kern0pt}\ {\isasyminter}\ space\ M{\isacharparenright}{\kern0pt}\ z\ {\isacharasterisk}{\kern0pt}\isactrlsub R\ x{\isacharparenright}{\kern0pt}{\isachardoublequoteclose}\ \isakeyword{if}\ z{\isacharcolon}{\kern0pt}\ {\isachardoublequoteopen}z\ {\isasymin}\ space\ M{\isachardoublequoteclose}\ \isakeyword{for}\ z\isanewline
\ \ \ \ \ \ \ \ \isacommand{proof}\isamarkupfalse%
\ {\isacharparenleft}{\kern0pt}cases\ {\isachardoublequoteopen}f\ z\ {\isacharequal}{\kern0pt}\ x{\isachardoublequoteclose}{\isacharparenright}{\kern0pt}\isanewline
\ \ \ \ \ \ \ \ \ \ \isacommand{case}\isamarkupfalse%
\ True\isanewline
\ \ \ \ \ \ \ \ \ \ \isacommand{have}\isamarkupfalse%
\ {\isachardoublequoteopen}indicat{\isacharunderscore}{\kern0pt}real\ {\isacharparenleft}{\kern0pt}f\ {\isacharminus}{\kern0pt}{\isacharbackquote}{\kern0pt}\ {\isacharbraceleft}{\kern0pt}y{\isacharbraceright}{\kern0pt}\ {\isasyminter}\ space\ M{\isacharparenright}{\kern0pt}\ z\ {\isacharasterisk}{\kern0pt}\isactrlsub R\ y\ {\isacharequal}{\kern0pt}\ {\isadigit{0}}{\isachardoublequoteclose}\ \isakeyword{if}\ {\isachardoublequoteopen}y\ {\isasymin}\ F{\isachardoublequoteclose}\ \isakeyword{for}\ y\ \isacommand{using}\isamarkupfalse%
\ True\ insert{\isacharparenleft}{\kern0pt}{\isadigit{2}}{\isacharparenright}{\kern0pt}\ z\ that\ {\isadigit{1}}\ \isacommand{unfolding}\isamarkupfalse%
\ indicator{\isacharunderscore}{\kern0pt}def\ \isacommand{by}\isamarkupfalse%
\ force\isanewline
\ \ \ \ \ \ \ \ \ \ \isacommand{hence}\isamarkupfalse%
\ {\isachardoublequoteopen}{\isacharparenleft}{\kern0pt}{\isasymSum}y{\isasymin}F{\isachardot}{\kern0pt}\ indicat{\isacharunderscore}{\kern0pt}real\ {\isacharparenleft}{\kern0pt}f\ {\isacharminus}{\kern0pt}{\isacharbackquote}{\kern0pt}\ {\isacharbraceleft}{\kern0pt}y{\isacharbraceright}{\kern0pt}\ {\isasyminter}\ space\ M{\isacharparenright}{\kern0pt}\ z\ {\isacharasterisk}{\kern0pt}\isactrlsub R\ y{\isacharparenright}{\kern0pt}\ {\isacharequal}{\kern0pt}\ {\isadigit{0}}{\isachardoublequoteclose}\ \isacommand{by}\isamarkupfalse%
\ {\isacharparenleft}{\kern0pt}meson\ sum{\isachardot}{\kern0pt}neutral{\isacharparenright}{\kern0pt}\isanewline
\ \ \ \ \ \ \ \ \ \ \isacommand{then}\isamarkupfalse%
\ \isacommand{show}\isamarkupfalse%
\ {\isacharquery}{\kern0pt}thesis\ \isacommand{by}\isamarkupfalse%
\ force\isanewline
\ \ \ \ \ \ \ \ \isacommand{next}\isamarkupfalse%
\isanewline
\ \ \ \ \ \ \ \ \ \ \isacommand{case}\isamarkupfalse%
\ False\isanewline
\ \ \ \ \ \ \ \ \ \ \isacommand{then}\isamarkupfalse%
\ \isacommand{show}\isamarkupfalse%
\ {\isacharquery}{\kern0pt}thesis\ \isacommand{by}\isamarkupfalse%
\ force\isanewline
\ \ \ \ \ \ \ \ \isacommand{qed}\isamarkupfalse%
\isanewline
\ \ \ \ \ \ \ \ \isacommand{show}\isamarkupfalse%
\ {\isacharquery}{\kern0pt}thesis\ \isacommand{using}\isamarkupfalse%
\ False\ simple{\isacharunderscore}{\kern0pt}functionD{\isacharparenleft}{\kern0pt}{\isadigit{2}}{\isacharparenright}{\kern0pt}{\isacharbrackleft}{\kern0pt}OF\ f{\isacharparenleft}{\kern0pt}{\isadigit{1}}{\isacharparenright}{\kern0pt}{\isacharbrackright}{\kern0pt}\ insert{\isacharparenleft}{\kern0pt}{\isadigit{3}}{\isacharcomma}{\kern0pt}{\isadigit{5}}{\isacharparenright}{\kern0pt}{\isacharbrackleft}{\kern0pt}OF\ F{\isacharbrackright}{\kern0pt}\ simple{\isacharunderscore}{\kern0pt}function{\isacharunderscore}{\kern0pt}scaleR\ fin{\isacharunderscore}{\kern0pt}{\isadigit{0}}\ fin{\isacharunderscore}{\kern0pt}{\isadigit{1}}\ \isacommand{by}\isamarkupfalse%
\ {\isacharparenleft}{\kern0pt}subst\ {\isacharasterisk}{\kern0pt}{\isacharcomma}{\kern0pt}\ subst\ add{\isacharcomma}{\kern0pt}\ subst\ simple{\isacharunderscore}{\kern0pt}function{\isacharunderscore}{\kern0pt}sum{\isacharparenright}{\kern0pt}\ {\isacharparenleft}{\kern0pt}blast\ intro{\isacharcolon}{\kern0pt}\ norm{\isacharunderscore}{\kern0pt}argument\ indicator{\isacharparenright}{\kern0pt}{\isacharplus}{\kern0pt}\isanewline
\ \ \ \ \ \ \isacommand{qed}\isamarkupfalse%
\ \isanewline
\ \ \ \ \isacommand{next}\isamarkupfalse%
\isanewline
\ \ \ \ \ \ \isacommand{case}\isamarkupfalse%
\ {\isadigit{2}}\isanewline
\ \ \ \ \ \ \isacommand{hence}\isamarkupfalse%
\ x{\isacharcolon}{\kern0pt}\ {\isachardoublequoteopen}x\ {\isasymin}\ f\ {\isacharbackquote}{\kern0pt}\ space\ M{\isachardoublequoteclose}\ \isakeyword{and}\ F{\isacharcolon}{\kern0pt}\ {\isachardoublequoteopen}F\ {\isasymsubseteq}\ f\ {\isacharbackquote}{\kern0pt}\ space\ M{\isachardoublequoteclose}\ \isacommand{by}\isamarkupfalse%
\ auto\isanewline
\ \ \ \ \ \ \isacommand{show}\isamarkupfalse%
\ {\isacharquery}{\kern0pt}case\ \isanewline
\ \ \ \ \ \ \isacommand{proof}\isamarkupfalse%
\ {\isacharparenleft}{\kern0pt}cases\ {\isachardoublequoteopen}x\ {\isacharequal}{\kern0pt}\ {\isadigit{0}}{\isachardoublequoteclose}{\isacharparenright}{\kern0pt}\isanewline
\ \ \ \ \ \ \ \ \isacommand{case}\isamarkupfalse%
\ True\isanewline
\ \ \ \ \ \ \ \ \isacommand{then}\isamarkupfalse%
\ \isacommand{show}\isamarkupfalse%
\ {\isacharquery}{\kern0pt}thesis\ \isacommand{unfolding}\isamarkupfalse%
\ {\isacharasterisk}{\kern0pt}\ \isacommand{using}\isamarkupfalse%
\ insert{\isacharparenleft}{\kern0pt}{\isadigit{4}}{\isacharparenright}{\kern0pt}{\isacharbrackleft}{\kern0pt}OF\ F{\isacharbrackright}{\kern0pt}\ \isacommand{by}\isamarkupfalse%
\ simp\isanewline
\ \ \ \ \ \ \isacommand{next}\isamarkupfalse%
\isanewline
\ \ \ \ \ \ \ \ \isacommand{case}\isamarkupfalse%
\ False\isanewline
\ \ \ \ \ \ \ \ \isacommand{then}\isamarkupfalse%
\ \isacommand{show}\isamarkupfalse%
\ {\isacharquery}{\kern0pt}thesis\ \isacommand{unfolding}\isamarkupfalse%
\ {\isacharasterisk}{\kern0pt}\ \isacommand{using}\isamarkupfalse%
\ insert{\isacharparenleft}{\kern0pt}{\isadigit{4}}{\isacharparenright}{\kern0pt}{\isacharbrackleft}{\kern0pt}OF\ F{\isacharbrackright}{\kern0pt}\ simple{\isacharunderscore}{\kern0pt}functionD{\isacharparenleft}{\kern0pt}{\isadigit{2}}{\isacharparenright}{\kern0pt}{\isacharbrackleft}{\kern0pt}OF\ f{\isacharparenleft}{\kern0pt}{\isadigit{1}}{\isacharparenright}{\kern0pt}{\isacharbrackright}{\kern0pt}\ \isacommand{by}\isamarkupfalse%
\ fast\isanewline
\ \ \ \ \ \ \isacommand{qed}\isamarkupfalse%
\isanewline
\ \ \ \ \isacommand{next}\isamarkupfalse%
\isanewline
\ \ \ \ \ \ \isacommand{case}\isamarkupfalse%
\ {\isadigit{3}}\isanewline
\ \ \ \ \ \ \isacommand{hence}\isamarkupfalse%
\ x{\isacharcolon}{\kern0pt}\ {\isachardoublequoteopen}x\ {\isasymin}\ f\ {\isacharbackquote}{\kern0pt}\ space\ M{\isachardoublequoteclose}\ \isakeyword{and}\ F{\isacharcolon}{\kern0pt}\ {\isachardoublequoteopen}F\ {\isasymsubseteq}\ f\ {\isacharbackquote}{\kern0pt}\ space\ M{\isachardoublequoteclose}\ \isacommand{by}\isamarkupfalse%
\ auto\isanewline
\ \ \ \ \ \ \isacommand{show}\isamarkupfalse%
\ {\isacharquery}{\kern0pt}case\ \isanewline
\ \ \ \ \ \ \isacommand{proof}\isamarkupfalse%
\ {\isacharparenleft}{\kern0pt}cases\ {\isachardoublequoteopen}x\ {\isacharequal}{\kern0pt}\ {\isadigit{0}}{\isachardoublequoteclose}{\isacharparenright}{\kern0pt}\isanewline
\ \ \ \ \ \ \ \ \isacommand{case}\isamarkupfalse%
\ True\isanewline
\ \ \ \ \ \ \ \ \isacommand{then}\isamarkupfalse%
\ \isacommand{show}\isamarkupfalse%
\ {\isacharquery}{\kern0pt}thesis\ \isacommand{unfolding}\isamarkupfalse%
\ {\isacharasterisk}{\kern0pt}\ \isacommand{using}\isamarkupfalse%
\ insert{\isacharparenleft}{\kern0pt}{\isadigit{5}}{\isacharparenright}{\kern0pt}{\isacharbrackleft}{\kern0pt}OF\ F{\isacharbrackright}{\kern0pt}\ \isacommand{by}\isamarkupfalse%
\ simp\isanewline
\ \ \ \ \ \ \isacommand{next}\isamarkupfalse%
\isanewline
\ \ \ \ \ \ \ \ \isacommand{case}\isamarkupfalse%
\ False\isanewline
\ \ \ \ \ \ \ \ \isacommand{have}\isamarkupfalse%
\ {\isachardoublequoteopen}emeasure\ M\ {\isacharbraceleft}{\kern0pt}y\ {\isasymin}\ space\ M{\isachardot}{\kern0pt}\ {\isacharparenleft}{\kern0pt}{\isasymSum}x{\isasymin}insert\ x\ F{\isachardot}{\kern0pt}\ indicat{\isacharunderscore}{\kern0pt}real\ {\isacharparenleft}{\kern0pt}f\ {\isacharminus}{\kern0pt}{\isacharbackquote}{\kern0pt}\ {\isacharbraceleft}{\kern0pt}x{\isacharbraceright}{\kern0pt}\ {\isasyminter}\ space\ M{\isacharparenright}{\kern0pt}\ y\ {\isacharasterisk}{\kern0pt}\isactrlsub R\ x{\isacharparenright}{\kern0pt}\ {\isasymnoteq}\ {\isadigit{0}}{\isacharbraceright}{\kern0pt}\ {\isasymle}\ emeasure\ M\ {\isacharparenleft}{\kern0pt}{\isacharbraceleft}{\kern0pt}y\ {\isasymin}\ space\ M{\isachardot}{\kern0pt}\ {\isacharparenleft}{\kern0pt}{\isasymSum}x{\isasymin}F{\isachardot}{\kern0pt}\ indicat{\isacharunderscore}{\kern0pt}real\ {\isacharparenleft}{\kern0pt}f\ {\isacharminus}{\kern0pt}{\isacharbackquote}{\kern0pt}\ {\isacharbraceleft}{\kern0pt}x{\isacharbraceright}{\kern0pt}\ {\isasyminter}\ space\ M{\isacharparenright}{\kern0pt}\ y\ {\isacharasterisk}{\kern0pt}\isactrlsub R\ x{\isacharparenright}{\kern0pt}\ {\isasymnoteq}\ {\isadigit{0}}{\isacharbraceright}{\kern0pt}\ {\isasymunion}\ {\isacharbraceleft}{\kern0pt}y\ {\isasymin}\ space\ M{\isachardot}{\kern0pt}\ indicat{\isacharunderscore}{\kern0pt}real\ {\isacharparenleft}{\kern0pt}f\ {\isacharminus}{\kern0pt}{\isacharbackquote}{\kern0pt}\ {\isacharbraceleft}{\kern0pt}x{\isacharbraceright}{\kern0pt}\ {\isasyminter}\ space\ M{\isacharparenright}{\kern0pt}\ y\ {\isacharasterisk}{\kern0pt}\isactrlsub R\ x\ {\isasymnoteq}\ {\isadigit{0}}{\isacharbraceright}{\kern0pt}{\isacharparenright}{\kern0pt}{\isachardoublequoteclose}\isanewline
\ \ \ \ \ \ \ \ \ \ \isacommand{using}\isamarkupfalse%
\ {\isacharasterisk}{\kern0pt}{\isacharasterisk}{\kern0pt}\ simple{\isacharunderscore}{\kern0pt}functionD{\isacharparenleft}{\kern0pt}{\isadigit{2}}{\isacharparenright}{\kern0pt}{\isacharbrackleft}{\kern0pt}OF\ insert{\isacharparenleft}{\kern0pt}{\isadigit{4}}{\isacharparenright}{\kern0pt}{\isacharbrackleft}{\kern0pt}OF\ F{\isacharbrackright}{\kern0pt}{\isacharbrackright}{\kern0pt}\ simple{\isacharunderscore}{\kern0pt}functionD{\isacharparenleft}{\kern0pt}{\isadigit{2}}{\isacharparenright}{\kern0pt}{\isacharbrackleft}{\kern0pt}OF\ f{\isacharparenleft}{\kern0pt}{\isadigit{1}}{\isacharparenright}{\kern0pt}{\isacharbrackright}{\kern0pt}\ \isacommand{by}\isamarkupfalse%
\ {\isacharparenleft}{\kern0pt}intro\ emeasure{\isacharunderscore}{\kern0pt}mono{\isacharcomma}{\kern0pt}\ force{\isacharplus}{\kern0pt}{\isacharparenright}{\kern0pt}\isanewline
\ \ \ \ \ \ \ \ \isacommand{also}\isamarkupfalse%
\ \isacommand{have}\isamarkupfalse%
\ {\isachardoublequoteopen}{\isachardot}{\kern0pt}{\isachardot}{\kern0pt}{\isachardot}{\kern0pt}\ {\isasymle}\ emeasure\ M\ {\isacharbraceleft}{\kern0pt}y\ {\isasymin}\ space\ M{\isachardot}{\kern0pt}\ {\isacharparenleft}{\kern0pt}{\isasymSum}x{\isasymin}F{\isachardot}{\kern0pt}\ indicat{\isacharunderscore}{\kern0pt}real\ {\isacharparenleft}{\kern0pt}f\ {\isacharminus}{\kern0pt}{\isacharbackquote}{\kern0pt}\ {\isacharbraceleft}{\kern0pt}x{\isacharbraceright}{\kern0pt}\ {\isasyminter}\ space\ M{\isacharparenright}{\kern0pt}\ y\ {\isacharasterisk}{\kern0pt}\isactrlsub R\ x{\isacharparenright}{\kern0pt}\ {\isasymnoteq}\ {\isadigit{0}}{\isacharbraceright}{\kern0pt}\ {\isacharplus}{\kern0pt}\ emeasure\ M\ {\isacharbraceleft}{\kern0pt}y\ {\isasymin}\ space\ M{\isachardot}{\kern0pt}\ indicat{\isacharunderscore}{\kern0pt}real\ {\isacharparenleft}{\kern0pt}f\ {\isacharminus}{\kern0pt}{\isacharbackquote}{\kern0pt}\ {\isacharbraceleft}{\kern0pt}x{\isacharbraceright}{\kern0pt}\ {\isasyminter}\ space\ M{\isacharparenright}{\kern0pt}\ y\ {\isacharasterisk}{\kern0pt}\isactrlsub R\ x\ {\isasymnoteq}\ {\isadigit{0}}{\isacharbraceright}{\kern0pt}{\isachardoublequoteclose}\isanewline
\ \ \ \ \ \ \ \ \ \ \isacommand{using}\isamarkupfalse%
\ simple{\isacharunderscore}{\kern0pt}functionD{\isacharparenleft}{\kern0pt}{\isadigit{2}}{\isacharparenright}{\kern0pt}{\isacharbrackleft}{\kern0pt}OF\ insert{\isacharparenleft}{\kern0pt}{\isadigit{4}}{\isacharparenright}{\kern0pt}{\isacharbrackleft}{\kern0pt}OF\ F{\isacharbrackright}{\kern0pt}{\isacharbrackright}{\kern0pt}\ simple{\isacharunderscore}{\kern0pt}functionD{\isacharparenleft}{\kern0pt}{\isadigit{2}}{\isacharparenright}{\kern0pt}{\isacharbrackleft}{\kern0pt}OF\ f{\isacharparenleft}{\kern0pt}{\isadigit{1}}{\isacharparenright}{\kern0pt}{\isacharbrackright}{\kern0pt}\ \isacommand{by}\isamarkupfalse%
\ {\isacharparenleft}{\kern0pt}intro\ emeasure{\isacharunderscore}{\kern0pt}subadditive{\isacharcomma}{\kern0pt}\ force{\isacharplus}{\kern0pt}{\isacharparenright}{\kern0pt}\isanewline
\ \ \ \ \ \ \ \ \isacommand{also}\isamarkupfalse%
\ \isacommand{have}\isamarkupfalse%
\ {\isachardoublequoteopen}{\isachardot}{\kern0pt}{\isachardot}{\kern0pt}{\isachardot}{\kern0pt}\ {\isacharless}{\kern0pt}\ {\isasyminfinity}{\isachardoublequoteclose}\ \isacommand{using}\isamarkupfalse%
\ insert{\isacharparenleft}{\kern0pt}{\isadigit{5}}{\isacharparenright}{\kern0pt}{\isacharbrackleft}{\kern0pt}OF\ F{\isacharbrackright}{\kern0pt}\ fin{\isacharunderscore}{\kern0pt}{\isadigit{1}}{\isacharbrackleft}{\kern0pt}OF\ False{\isacharbrackright}{\kern0pt}\ \isacommand{by}\isamarkupfalse%
\ {\isacharparenleft}{\kern0pt}simp\ add{\isacharcolon}{\kern0pt}\ less{\isacharunderscore}{\kern0pt}top{\isacharparenright}{\kern0pt}\isanewline
\ \ \ \ \ \ \ \ \isacommand{finally}\isamarkupfalse%
\ \isacommand{show}\isamarkupfalse%
\ {\isacharquery}{\kern0pt}thesis\ \isacommand{by}\isamarkupfalse%
\ simp\isanewline
\ \ \ \ \ \ \isacommand{qed}\isamarkupfalse%
\isanewline
\ \ \ \ \isacommand{{\isacharbraceright}{\kern0pt}}\isamarkupfalse%
\isanewline
\ \ \isacommand{qed}\isamarkupfalse%
\isanewline
\ \ \isacommand{moreover}\isamarkupfalse%
\ \isacommand{have}\isamarkupfalse%
\ {\isachardoublequoteopen}simple{\isacharunderscore}{\kern0pt}function\ M\ {\isacharparenleft}{\kern0pt}{\isasymlambda}x{\isachardot}{\kern0pt}\ {\isasymSum}y{\isasymin}f\ {\isacharbackquote}{\kern0pt}\ space\ M{\isachardot}{\kern0pt}\ indicat{\isacharunderscore}{\kern0pt}real\ {\isacharparenleft}{\kern0pt}f\ {\isacharminus}{\kern0pt}{\isacharbackquote}{\kern0pt}\ {\isacharbraceleft}{\kern0pt}y{\isacharbraceright}{\kern0pt}\ {\isasyminter}\ space\ M{\isacharparenright}{\kern0pt}\ x\ {\isacharasterisk}{\kern0pt}\isactrlsub R\ y{\isacharparenright}{\kern0pt}{\isachardoublequoteclose}\ \isacommand{using}\isamarkupfalse%
\ calculation\ \isacommand{by}\isamarkupfalse%
\ blast\isanewline
\ \ \isacommand{moreover}\isamarkupfalse%
\ \isacommand{have}\isamarkupfalse%
\ {\isachardoublequoteopen}P\ {\isacharparenleft}{\kern0pt}{\isasymlambda}x{\isachardot}{\kern0pt}\ {\isasymSum}y{\isasymin}f\ {\isacharbackquote}{\kern0pt}\ space\ M{\isachardot}{\kern0pt}\ indicat{\isacharunderscore}{\kern0pt}real\ {\isacharparenleft}{\kern0pt}f\ {\isacharminus}{\kern0pt}{\isacharbackquote}{\kern0pt}\ {\isacharbraceleft}{\kern0pt}y{\isacharbraceright}{\kern0pt}\ {\isasyminter}\ space\ M{\isacharparenright}{\kern0pt}\ x\ {\isacharasterisk}{\kern0pt}\isactrlsub R\ y{\isacharparenright}{\kern0pt}{\isachardoublequoteclose}\ \isacommand{using}\isamarkupfalse%
\ calculation\ \isacommand{by}\isamarkupfalse%
\ blast\isanewline
\ \ \isacommand{ultimately}\isamarkupfalse%
\ \isacommand{show}\isamarkupfalse%
\ {\isacharquery}{\kern0pt}thesis\ \isacommand{by}\isamarkupfalse%
\ {\isacharparenleft}{\kern0pt}intro\ cong{\isacharbrackleft}{\kern0pt}OF\ {\isacharunderscore}{\kern0pt}\ {\isacharunderscore}{\kern0pt}\ f{\isacharparenleft}{\kern0pt}{\isadigit{1}}{\isacharcomma}{\kern0pt}{\isadigit{2}}{\isacharparenright}{\kern0pt}{\isacharbrackright}{\kern0pt}{\isacharcomma}{\kern0pt}\ blast{\isacharcomma}{\kern0pt}\ presburger{\isacharplus}{\kern0pt}{\isacharparenright}{\kern0pt}\ \isanewline
\isacommand{qed}\isamarkupfalse%
%
\endisatagproof
{\isafoldproof}%
%
\isadelimproof
\isanewline
%
\endisadelimproof
\isanewline
%
\isadelimimportant
\isanewline
%
\endisadelimimportant
%
\isatagimportant
\isacommand{lemma}\isamarkupfalse%
\ simple{\isacharunderscore}{\kern0pt}integrable{\isacharunderscore}{\kern0pt}function{\isacharunderscore}{\kern0pt}induct{\isacharunderscore}{\kern0pt}nn{\isacharbrackleft}{\kern0pt}consumes\ {\isadigit{3}}{\isacharcomma}{\kern0pt}\ case{\isacharunderscore}{\kern0pt}names\ cong\ indicator\ add{\isacharcomma}{\kern0pt}\ induct\ set{\isacharcolon}{\kern0pt}\ simple{\isacharunderscore}{\kern0pt}function{\isacharbrackright}{\kern0pt}{\isacharcolon}{\kern0pt}\isanewline
\ \ \isakeyword{fixes}\ f\ {\isacharcolon}{\kern0pt}{\isacharcolon}{\kern0pt}\ {\isachardoublequoteopen}{\isacharprime}{\kern0pt}a\ {\isasymRightarrow}\ {\isacharprime}{\kern0pt}b\ {\isacharcolon}{\kern0pt}{\isacharcolon}{\kern0pt}\ {\isacharbraceleft}{\kern0pt}second{\isacharunderscore}{\kern0pt}countable{\isacharunderscore}{\kern0pt}topology{\isacharcomma}{\kern0pt}\ banach{\isacharcomma}{\kern0pt}\ linorder{\isacharunderscore}{\kern0pt}topology{\isacharcomma}{\kern0pt}\ ordered{\isacharunderscore}{\kern0pt}real{\isacharunderscore}{\kern0pt}vector{\isacharbraceright}{\kern0pt}{\isachardoublequoteclose}\isanewline
\ \ \isakeyword{assumes}\ f{\isacharcolon}{\kern0pt}\ {\isachardoublequoteopen}simple{\isacharunderscore}{\kern0pt}function\ M\ f{\isachardoublequoteclose}\ {\isachardoublequoteopen}emeasure\ M\ {\isacharbraceleft}{\kern0pt}y\ {\isasymin}\ space\ M{\isachardot}{\kern0pt}\ f\ y\ {\isasymnoteq}\ {\isadigit{0}}{\isacharbraceright}{\kern0pt}\ {\isasymnoteq}\ {\isasyminfinity}{\isachardoublequoteclose}\ {\isachardoublequoteopen}{\isasymAnd}x{\isachardot}{\kern0pt}\ x\ {\isasymin}\ space\ M\ {\isasymlongrightarrow}\ f\ x\ {\isasymge}\ {\isadigit{0}}{\isachardoublequoteclose}\isanewline
\ \ \isakeyword{assumes}\ cong{\isacharcolon}{\kern0pt}\ {\isachardoublequoteopen}{\isasymAnd}f\ g{\isachardot}{\kern0pt}\ simple{\isacharunderscore}{\kern0pt}function\ M\ f\ {\isasymLongrightarrow}\ emeasure\ M\ {\isacharbraceleft}{\kern0pt}y\ {\isasymin}\ space\ M{\isachardot}{\kern0pt}\ f\ y\ {\isasymnoteq}\ {\isadigit{0}}{\isacharbraceright}{\kern0pt}\ {\isasymnoteq}\ {\isasyminfinity}\ {\isasymLongrightarrow}\ {\isacharparenleft}{\kern0pt}{\isasymAnd}x{\isachardot}{\kern0pt}\ x\ {\isasymin}\ space\ M\ {\isasymLongrightarrow}\ f\ x\ {\isasymge}\ {\isadigit{0}}{\isacharparenright}{\kern0pt}\ {\isasymLongrightarrow}\ simple{\isacharunderscore}{\kern0pt}function\ M\ g\ {\isasymLongrightarrow}\ emeasure\ M\ {\isacharbraceleft}{\kern0pt}y\ {\isasymin}\ space\ M{\isachardot}{\kern0pt}\ g\ y\ {\isasymnoteq}\ {\isadigit{0}}{\isacharbraceright}{\kern0pt}\ {\isasymnoteq}\ {\isasyminfinity}\ {\isasymLongrightarrow}\ {\isacharparenleft}{\kern0pt}{\isasymAnd}x{\isachardot}{\kern0pt}\ x\ {\isasymin}\ space\ M\ {\isasymLongrightarrow}\ g\ x\ {\isasymge}\ {\isadigit{0}}{\isacharparenright}{\kern0pt}\ {\isasymLongrightarrow}\ {\isacharparenleft}{\kern0pt}{\isasymAnd}x{\isachardot}{\kern0pt}\ x\ {\isasymin}\ space\ M\ {\isasymLongrightarrow}\ f\ x\ {\isacharequal}{\kern0pt}\ g\ x{\isacharparenright}{\kern0pt}\ {\isasymLongrightarrow}\ P\ f\ {\isasymLongrightarrow}\ P\ g{\isachardoublequoteclose}\isanewline
\ \ \isakeyword{assumes}\ indicator{\isacharcolon}{\kern0pt}\ {\isachardoublequoteopen}{\isasymAnd}A\ y{\isachardot}{\kern0pt}\ y\ {\isasymge}\ {\isadigit{0}}\ {\isasymLongrightarrow}\ A\ {\isasymin}\ sets\ M\ {\isasymLongrightarrow}\ emeasure\ M\ A\ {\isacharless}{\kern0pt}\ {\isasyminfinity}\ {\isasymLongrightarrow}\ P\ {\isacharparenleft}{\kern0pt}{\isasymlambda}x{\isachardot}{\kern0pt}\ indicator\ A\ x\ {\isacharasterisk}{\kern0pt}\isactrlsub R\ y{\isacharparenright}{\kern0pt}{\isachardoublequoteclose}\isanewline
\ \ \isakeyword{assumes}\ add{\isacharcolon}{\kern0pt}\ {\isachardoublequoteopen}{\isasymAnd}f\ g{\isachardot}{\kern0pt}\ {\isacharparenleft}{\kern0pt}{\isasymAnd}x{\isachardot}{\kern0pt}\ x\ {\isasymin}\ space\ M\ {\isasymLongrightarrow}\ f\ x\ {\isasymge}\ {\isadigit{0}}{\isacharparenright}{\kern0pt}\ {\isasymLongrightarrow}\ simple{\isacharunderscore}{\kern0pt}function\ M\ f\ {\isasymLongrightarrow}\ emeasure\ M\ {\isacharbraceleft}{\kern0pt}y\ {\isasymin}\ space\ M{\isachardot}{\kern0pt}\ f\ y\ {\isasymnoteq}\ {\isadigit{0}}{\isacharbraceright}{\kern0pt}\ {\isasymnoteq}\ {\isasyminfinity}\ {\isasymLongrightarrow}\ \isanewline
\ \ \ \ \ \ \ \ \ \ \ \ \ \ \ \ \ \ \ \ \ \ {\isacharparenleft}{\kern0pt}{\isasymAnd}x{\isachardot}{\kern0pt}\ x\ {\isasymin}\ space\ M\ {\isasymLongrightarrow}\ g\ x\ {\isasymge}\ {\isadigit{0}}{\isacharparenright}{\kern0pt}\ {\isasymLongrightarrow}\ simple{\isacharunderscore}{\kern0pt}function\ M\ g\ {\isasymLongrightarrow}\ emeasure\ M\ {\isacharbraceleft}{\kern0pt}y\ {\isasymin}\ space\ M{\isachardot}{\kern0pt}\ g\ y\ {\isasymnoteq}\ {\isadigit{0}}{\isacharbraceright}{\kern0pt}\ {\isasymnoteq}\ {\isasyminfinity}\ {\isasymLongrightarrow}\ \isanewline
\ \ \ \ \ \ \ \ \ \ \ \ \ \ \ \ \ \ \ \ \ \ {\isacharparenleft}{\kern0pt}{\isasymAnd}z{\isachardot}{\kern0pt}\ z\ {\isasymin}\ space\ M\ {\isasymLongrightarrow}\ norm\ {\isacharparenleft}{\kern0pt}f\ z\ {\isacharplus}{\kern0pt}\ g\ z{\isacharparenright}{\kern0pt}\ {\isacharequal}{\kern0pt}\ norm\ {\isacharparenleft}{\kern0pt}f\ z{\isacharparenright}{\kern0pt}\ {\isacharplus}{\kern0pt}\ norm\ {\isacharparenleft}{\kern0pt}g\ z{\isacharparenright}{\kern0pt}{\isacharparenright}{\kern0pt}\ {\isasymLongrightarrow}\isanewline
\ \ \ \ \ \ \ \ \ \ \ \ \ \ \ \ \ \ \ \ \ \ P\ f\ {\isasymLongrightarrow}\ P\ g\ {\isasymLongrightarrow}\ P\ {\isacharparenleft}{\kern0pt}{\isasymlambda}x{\isachardot}{\kern0pt}\ f\ x\ {\isacharplus}{\kern0pt}\ g\ x{\isacharparenright}{\kern0pt}{\isachardoublequoteclose}\isanewline
\ \ \isakeyword{shows}\ {\isachardoublequoteopen}P\ f{\isachardoublequoteclose}%
\endisatagimportant
{\isafoldimportant}%
%
\isadelimimportant
\isanewline
%
\endisadelimimportant
%
\isadelimproof
%
\endisadelimproof
%
\isatagproof
\isacommand{proof}\isamarkupfalse%
{\isacharminus}{\kern0pt}\isanewline
\ \ \isacommand{let}\isamarkupfalse%
\ {\isacharquery}{\kern0pt}f\ {\isacharequal}{\kern0pt}\ {\isachardoublequoteopen}{\isasymlambda}x{\isachardot}{\kern0pt}\ {\isacharparenleft}{\kern0pt}{\isasymSum}y{\isasymin}f\ {\isacharbackquote}{\kern0pt}\ space\ M{\isachardot}{\kern0pt}\ indicat{\isacharunderscore}{\kern0pt}real\ {\isacharparenleft}{\kern0pt}f\ {\isacharminus}{\kern0pt}{\isacharbackquote}{\kern0pt}\ {\isacharbraceleft}{\kern0pt}y{\isacharbraceright}{\kern0pt}\ {\isasyminter}\ space\ M{\isacharparenright}{\kern0pt}\ x\ {\isacharasterisk}{\kern0pt}\isactrlsub R\ y{\isacharparenright}{\kern0pt}{\isachardoublequoteclose}\isanewline
\ \ \isacommand{have}\isamarkupfalse%
\ f{\isacharunderscore}{\kern0pt}ae{\isacharunderscore}{\kern0pt}eq{\isacharcolon}{\kern0pt}\ {\isachardoublequoteopen}f\ x\ {\isacharequal}{\kern0pt}\ {\isacharquery}{\kern0pt}f\ x{\isachardoublequoteclose}\ \isakeyword{if}\ {\isachardoublequoteopen}x\ {\isasymin}\ space\ M{\isachardoublequoteclose}\ \isakeyword{for}\ x\ \isacommand{using}\isamarkupfalse%
\ simple{\isacharunderscore}{\kern0pt}function{\isacharunderscore}{\kern0pt}indicator{\isacharunderscore}{\kern0pt}representation{\isacharbrackleft}{\kern0pt}OF\ f{\isacharparenleft}{\kern0pt}{\isadigit{1}}{\isacharparenright}{\kern0pt}\ that{\isacharbrackright}{\kern0pt}\ \isacommand{{\isachardot}{\kern0pt}}\isamarkupfalse%
\isanewline
\ \ \isacommand{moreover}\isamarkupfalse%
\ \isacommand{have}\isamarkupfalse%
\ {\isachardoublequoteopen}emeasure\ M\ {\isacharbraceleft}{\kern0pt}y\ {\isasymin}\ space\ M{\isachardot}{\kern0pt}\ {\isacharquery}{\kern0pt}f\ y\ {\isasymnoteq}\ {\isadigit{0}}{\isacharbraceright}{\kern0pt}\ {\isasymnoteq}\ {\isasyminfinity}{\isachardoublequoteclose}\ \isacommand{by}\isamarkupfalse%
\ {\isacharparenleft}{\kern0pt}metis\ {\isacharparenleft}{\kern0pt}no{\isacharunderscore}{\kern0pt}types{\isacharcomma}{\kern0pt}\ lifting{\isacharparenright}{\kern0pt}\ Collect{\isacharunderscore}{\kern0pt}cong\ calculation\ f{\isacharparenleft}{\kern0pt}{\isadigit{2}}{\isacharparenright}{\kern0pt}{\isacharparenright}{\kern0pt}\isanewline
\ \ \isacommand{moreover}\isamarkupfalse%
\ \isacommand{have}\isamarkupfalse%
\ {\isachardoublequoteopen}P\ {\isacharparenleft}{\kern0pt}{\isasymlambda}x{\isachardot}{\kern0pt}\ {\isasymSum}y{\isasymin}S{\isachardot}{\kern0pt}\ indicat{\isacharunderscore}{\kern0pt}real\ {\isacharparenleft}{\kern0pt}f\ {\isacharminus}{\kern0pt}{\isacharbackquote}{\kern0pt}\ {\isacharbraceleft}{\kern0pt}y{\isacharbraceright}{\kern0pt}\ {\isasyminter}\ space\ M{\isacharparenright}{\kern0pt}\ x\ {\isacharasterisk}{\kern0pt}\isactrlsub R\ y{\isacharparenright}{\kern0pt}{\isachardoublequoteclose}\isanewline
\ \ \ \ \ \ \ \ \ \ \ \ \ \ \ \ {\isachardoublequoteopen}simple{\isacharunderscore}{\kern0pt}function\ M\ {\isacharparenleft}{\kern0pt}{\isasymlambda}x{\isachardot}{\kern0pt}\ {\isasymSum}y{\isasymin}S{\isachardot}{\kern0pt}\ indicat{\isacharunderscore}{\kern0pt}real\ {\isacharparenleft}{\kern0pt}f\ {\isacharminus}{\kern0pt}{\isacharbackquote}{\kern0pt}\ {\isacharbraceleft}{\kern0pt}y{\isacharbraceright}{\kern0pt}\ {\isasyminter}\ space\ M{\isacharparenright}{\kern0pt}\ x\ {\isacharasterisk}{\kern0pt}\isactrlsub R\ y{\isacharparenright}{\kern0pt}{\isachardoublequoteclose}\isanewline
\ \ \ \ \ \ \ \ \ \ \ \ \ \ \ \ {\isachardoublequoteopen}emeasure\ M\ {\isacharbraceleft}{\kern0pt}y\ {\isasymin}\ space\ M{\isachardot}{\kern0pt}\ {\isacharparenleft}{\kern0pt}{\isasymSum}x{\isasymin}S{\isachardot}{\kern0pt}\ indicat{\isacharunderscore}{\kern0pt}real\ {\isacharparenleft}{\kern0pt}f\ {\isacharminus}{\kern0pt}{\isacharbackquote}{\kern0pt}\ {\isacharbraceleft}{\kern0pt}x{\isacharbraceright}{\kern0pt}\ {\isasyminter}\ space\ M{\isacharparenright}{\kern0pt}\ y\ {\isacharasterisk}{\kern0pt}\isactrlsub R\ x{\isacharparenright}{\kern0pt}\ {\isasymnoteq}\ {\isadigit{0}}{\isacharbraceright}{\kern0pt}\ {\isasymnoteq}\ {\isasyminfinity}{\isachardoublequoteclose}\isanewline
\ \ \ \ \ \ \ \ \ \ \ \ \ \ \ \ {\isachardoublequoteopen}{\isasymAnd}x{\isachardot}{\kern0pt}\ x\ {\isasymin}\ space\ M\ {\isasymLongrightarrow}\ {\isadigit{0}}\ {\isasymle}\ {\isacharparenleft}{\kern0pt}{\isasymSum}y{\isasymin}S{\isachardot}{\kern0pt}\ indicat{\isacharunderscore}{\kern0pt}real\ {\isacharparenleft}{\kern0pt}f\ {\isacharminus}{\kern0pt}{\isacharbackquote}{\kern0pt}\ {\isacharbraceleft}{\kern0pt}y{\isacharbraceright}{\kern0pt}\ {\isasyminter}\ space\ M{\isacharparenright}{\kern0pt}\ x\ {\isacharasterisk}{\kern0pt}\isactrlsub R\ y{\isacharparenright}{\kern0pt}{\isachardoublequoteclose}\isanewline
\ \ \ \ \ \ \ \ \ \ \ \ \ \ \ \ \isakeyword{if}\ {\isachardoublequoteopen}S\ {\isasymsubseteq}\ f\ {\isacharbackquote}{\kern0pt}\ space\ M{\isachardoublequoteclose}\ \isakeyword{for}\ S\ \isacommand{using}\isamarkupfalse%
\ simple{\isacharunderscore}{\kern0pt}functionD{\isacharparenleft}{\kern0pt}{\isadigit{1}}{\isacharparenright}{\kern0pt}{\isacharbrackleft}{\kern0pt}OF\ assms{\isacharparenleft}{\kern0pt}{\isadigit{1}}{\isacharparenright}{\kern0pt}{\isacharcomma}{\kern0pt}\ THEN\ rev{\isacharunderscore}{\kern0pt}finite{\isacharunderscore}{\kern0pt}subset{\isacharcomma}{\kern0pt}\ OF\ that{\isacharbrackright}{\kern0pt}\ that\ \isanewline
\ \ \isacommand{proof}\isamarkupfalse%
\ {\isacharparenleft}{\kern0pt}induction\ rule{\isacharcolon}{\kern0pt}\ finite{\isacharunderscore}{\kern0pt}subset{\isacharunderscore}{\kern0pt}induct{\isacharprime}{\kern0pt}{\isacharparenright}{\kern0pt}\isanewline
\ \ \ \ \isacommand{case}\isamarkupfalse%
\ empty\isanewline
\ \ \ \ \isacommand{{\isacharbraceleft}{\kern0pt}}\isamarkupfalse%
\isanewline
\ \ \ \ \ \ \isacommand{case}\isamarkupfalse%
\ {\isadigit{1}}\isanewline
\ \ \ \ \ \ \isacommand{then}\isamarkupfalse%
\ \isacommand{show}\isamarkupfalse%
\ {\isacharquery}{\kern0pt}case\ \isacommand{using}\isamarkupfalse%
\ indicator{\isacharbrackleft}{\kern0pt}of\ {\isadigit{0}}\ {\isachardoublequoteopen}{\isacharbraceleft}{\kern0pt}{\isacharbraceright}{\kern0pt}{\isachardoublequoteclose}{\isacharbrackright}{\kern0pt}\ \isacommand{by}\isamarkupfalse%
\ force\isanewline
\ \ \ \ \isacommand{next}\isamarkupfalse%
\isanewline
\ \ \ \ \ \ \isacommand{case}\isamarkupfalse%
\ {\isadigit{2}}\isanewline
\ \ \ \ \ \ \isacommand{then}\isamarkupfalse%
\ \isacommand{show}\isamarkupfalse%
\ {\isacharquery}{\kern0pt}case\ \isacommand{by}\isamarkupfalse%
\ force\ \isanewline
\ \ \ \ \isacommand{next}\isamarkupfalse%
\isanewline
\ \ \ \ \ \ \isacommand{case}\isamarkupfalse%
\ {\isadigit{3}}\isanewline
\ \ \ \ \ \ \isacommand{then}\isamarkupfalse%
\ \isacommand{show}\isamarkupfalse%
\ {\isacharquery}{\kern0pt}case\ \isacommand{by}\isamarkupfalse%
\ force\ \isanewline
\ \ \ \ \isacommand{next}\isamarkupfalse%
\isanewline
\ \ \ \ \ \ \isacommand{case}\isamarkupfalse%
\ {\isadigit{4}}\isanewline
\ \ \ \ \ \ \isacommand{then}\isamarkupfalse%
\ \isacommand{show}\isamarkupfalse%
\ {\isacharquery}{\kern0pt}case\ \isacommand{by}\isamarkupfalse%
\ force\ \isanewline
\ \ \ \ \isacommand{{\isacharbraceright}{\kern0pt}}\isamarkupfalse%
\isanewline
\ \ \isacommand{next}\isamarkupfalse%
\isanewline
\ \ \ \ \isacommand{case}\isamarkupfalse%
\ {\isacharparenleft}{\kern0pt}insert\ x\ F{\isacharparenright}{\kern0pt}\isanewline
\ \ \ \ \isacommand{have}\isamarkupfalse%
\ {\isachardoublequoteopen}{\isacharparenleft}{\kern0pt}f\ {\isacharminus}{\kern0pt}{\isacharbackquote}{\kern0pt}\ {\isacharbraceleft}{\kern0pt}x{\isacharbraceright}{\kern0pt}\ {\isasyminter}\ space\ M{\isacharparenright}{\kern0pt}\ {\isasymsubseteq}\ {\isacharbraceleft}{\kern0pt}y\ {\isasymin}\ space\ M{\isachardot}{\kern0pt}\ f\ y\ {\isasymnoteq}\ {\isadigit{0}}{\isacharbraceright}{\kern0pt}{\isachardoublequoteclose}\ \isakeyword{if}\ {\isachardoublequoteopen}x\ {\isasymnoteq}\ {\isadigit{0}}{\isachardoublequoteclose}\ \isacommand{using}\isamarkupfalse%
\ that\ \isacommand{by}\isamarkupfalse%
\ blast\isanewline
\ \ \ \ \isacommand{moreover}\isamarkupfalse%
\ \isacommand{have}\isamarkupfalse%
\ {\isachardoublequoteopen}{\isacharbraceleft}{\kern0pt}y\ {\isasymin}\ space\ M{\isachardot}{\kern0pt}\ f\ y\ {\isasymnoteq}\ {\isadigit{0}}{\isacharbraceright}{\kern0pt}\ {\isacharequal}{\kern0pt}\ space\ M\ {\isacharminus}{\kern0pt}\ {\isacharparenleft}{\kern0pt}f\ {\isacharminus}{\kern0pt}{\isacharbackquote}{\kern0pt}\ {\isacharbraceleft}{\kern0pt}{\isadigit{0}}{\isacharbraceright}{\kern0pt}\ {\isasyminter}\ space\ M{\isacharparenright}{\kern0pt}{\isachardoublequoteclose}\ \isacommand{by}\isamarkupfalse%
\ blast\isanewline
\ \ \ \ \isacommand{moreover}\isamarkupfalse%
\ \isacommand{have}\isamarkupfalse%
\ {\isachardoublequoteopen}space\ M\ {\isacharminus}{\kern0pt}\ {\isacharparenleft}{\kern0pt}f\ {\isacharminus}{\kern0pt}{\isacharbackquote}{\kern0pt}\ {\isacharbraceleft}{\kern0pt}{\isadigit{0}}{\isacharbraceright}{\kern0pt}\ {\isasyminter}\ space\ M{\isacharparenright}{\kern0pt}\ {\isasymin}\ sets\ M{\isachardoublequoteclose}\ \isacommand{using}\isamarkupfalse%
\ simple{\isacharunderscore}{\kern0pt}functionD{\isacharparenleft}{\kern0pt}{\isadigit{2}}{\isacharparenright}{\kern0pt}{\isacharbrackleft}{\kern0pt}OF\ f{\isacharparenleft}{\kern0pt}{\isadigit{1}}{\isacharparenright}{\kern0pt}{\isacharbrackright}{\kern0pt}\ \isacommand{by}\isamarkupfalse%
\ blast\isanewline
\ \ \ \ \isacommand{ultimately}\isamarkupfalse%
\ \isacommand{have}\isamarkupfalse%
\ fin{\isacharunderscore}{\kern0pt}{\isadigit{0}}{\isacharcolon}{\kern0pt}\ {\isachardoublequoteopen}emeasure\ M\ {\isacharparenleft}{\kern0pt}f\ {\isacharminus}{\kern0pt}{\isacharbackquote}{\kern0pt}\ {\isacharbraceleft}{\kern0pt}x{\isacharbraceright}{\kern0pt}\ {\isasyminter}\ space\ M{\isacharparenright}{\kern0pt}\ {\isacharless}{\kern0pt}\ {\isasyminfinity}{\isachardoublequoteclose}\ \isakeyword{if}\ {\isachardoublequoteopen}x\ {\isasymnoteq}\ {\isadigit{0}}{\isachardoublequoteclose}\ \isacommand{using}\isamarkupfalse%
\ that\ \isacommand{by}\isamarkupfalse%
\ {\isacharparenleft}{\kern0pt}metis\ emeasure{\isacharunderscore}{\kern0pt}mono\ f{\isacharparenleft}{\kern0pt}{\isadigit{2}}{\isacharparenright}{\kern0pt}\ infinity{\isacharunderscore}{\kern0pt}ennreal{\isacharunderscore}{\kern0pt}def\ top{\isachardot}{\kern0pt}not{\isacharunderscore}{\kern0pt}eq{\isacharunderscore}{\kern0pt}extremum\ top{\isacharunderscore}{\kern0pt}unique{\isacharparenright}{\kern0pt}\isanewline
\ \ \ \ \isacommand{hence}\isamarkupfalse%
\ fin{\isacharunderscore}{\kern0pt}{\isadigit{1}}{\isacharcolon}{\kern0pt}\ {\isachardoublequoteopen}emeasure\ M\ {\isacharbraceleft}{\kern0pt}y\ {\isasymin}\ space\ M{\isachardot}{\kern0pt}\ indicat{\isacharunderscore}{\kern0pt}real\ {\isacharparenleft}{\kern0pt}f\ {\isacharminus}{\kern0pt}{\isacharbackquote}{\kern0pt}\ {\isacharbraceleft}{\kern0pt}x{\isacharbraceright}{\kern0pt}\ {\isasyminter}\ space\ M{\isacharparenright}{\kern0pt}\ y\ {\isacharasterisk}{\kern0pt}\isactrlsub R\ x\ {\isasymnoteq}\ {\isadigit{0}}{\isacharbraceright}{\kern0pt}\ {\isasymnoteq}\ {\isasyminfinity}{\isachardoublequoteclose}\ \isakeyword{if}\ {\isachardoublequoteopen}x\ {\isasymnoteq}\ {\isadigit{0}}{\isachardoublequoteclose}\ \isacommand{by}\isamarkupfalse%
\ {\isacharparenleft}{\kern0pt}metis\ {\isacharparenleft}{\kern0pt}mono{\isacharunderscore}{\kern0pt}tags{\isacharcomma}{\kern0pt}\ lifting{\isacharparenright}{\kern0pt}\ emeasure{\isacharunderscore}{\kern0pt}mono\ f{\isacharparenleft}{\kern0pt}{\isadigit{1}}{\isacharparenright}{\kern0pt}\ indicator{\isacharunderscore}{\kern0pt}simps{\isacharparenleft}{\kern0pt}{\isadigit{2}}{\isacharparenright}{\kern0pt}\ linorder{\isacharunderscore}{\kern0pt}not{\isacharunderscore}{\kern0pt}less\ mem{\isacharunderscore}{\kern0pt}Collect{\isacharunderscore}{\kern0pt}eq\ scaleR{\isacharunderscore}{\kern0pt}eq{\isacharunderscore}{\kern0pt}{\isadigit{0}}{\isacharunderscore}{\kern0pt}iff\ simple{\isacharunderscore}{\kern0pt}functionD{\isacharparenleft}{\kern0pt}{\isadigit{2}}{\isacharparenright}{\kern0pt}\ subsetI\ that{\isacharparenright}{\kern0pt}\isanewline
\isanewline
\ \ \ \ \isacommand{have}\isamarkupfalse%
\ nonneg{\isacharunderscore}{\kern0pt}x{\isacharcolon}{\kern0pt}\ {\isachardoublequoteopen}x\ {\isasymge}\ {\isadigit{0}}{\isachardoublequoteclose}\ \isacommand{using}\isamarkupfalse%
\ insert\ f\ \isacommand{by}\isamarkupfalse%
\ blast\isanewline
\ \ \ \ \isacommand{have}\isamarkupfalse%
\ {\isacharasterisk}{\kern0pt}{\isacharcolon}{\kern0pt}\ {\isachardoublequoteopen}{\isacharparenleft}{\kern0pt}{\isasymSum}y{\isasymin}insert\ x\ F{\isachardot}{\kern0pt}\ indicat{\isacharunderscore}{\kern0pt}real\ {\isacharparenleft}{\kern0pt}f\ {\isacharminus}{\kern0pt}{\isacharbackquote}{\kern0pt}\ {\isacharbraceleft}{\kern0pt}y{\isacharbraceright}{\kern0pt}\ {\isasyminter}\ space\ M{\isacharparenright}{\kern0pt}\ xa\ {\isacharasterisk}{\kern0pt}\isactrlsub R\ y{\isacharparenright}{\kern0pt}\ {\isacharequal}{\kern0pt}\ {\isacharparenleft}{\kern0pt}{\isasymSum}y{\isasymin}F{\isachardot}{\kern0pt}\ indicat{\isacharunderscore}{\kern0pt}real\ {\isacharparenleft}{\kern0pt}f\ {\isacharminus}{\kern0pt}{\isacharbackquote}{\kern0pt}\ {\isacharbraceleft}{\kern0pt}y{\isacharbraceright}{\kern0pt}\ {\isasyminter}\ space\ M{\isacharparenright}{\kern0pt}\ xa\ {\isacharasterisk}{\kern0pt}\isactrlsub R\ y{\isacharparenright}{\kern0pt}\ {\isacharplus}{\kern0pt}\ indicat{\isacharunderscore}{\kern0pt}real\ {\isacharparenleft}{\kern0pt}f\ {\isacharminus}{\kern0pt}{\isacharbackquote}{\kern0pt}\ {\isacharbraceleft}{\kern0pt}x{\isacharbraceright}{\kern0pt}\ {\isasyminter}\ space\ M{\isacharparenright}{\kern0pt}\ xa\ {\isacharasterisk}{\kern0pt}\isactrlsub R\ x{\isachardoublequoteclose}\ \isakeyword{for}\ xa\ \isacommand{by}\isamarkupfalse%
\ {\isacharparenleft}{\kern0pt}metis\ {\isacharparenleft}{\kern0pt}no{\isacharunderscore}{\kern0pt}types{\isacharcomma}{\kern0pt}\ lifting{\isacharparenright}{\kern0pt}\ add{\isachardot}{\kern0pt}commute\ insert{\isachardot}{\kern0pt}hyps{\isacharparenleft}{\kern0pt}{\isadigit{1}}{\isacharparenright}{\kern0pt}\ insert{\isachardot}{\kern0pt}hyps{\isacharparenleft}{\kern0pt}{\isadigit{4}}{\isacharparenright}{\kern0pt}\ sum{\isachardot}{\kern0pt}insert{\isacharparenright}{\kern0pt}\isanewline
\ \ \ \ \isacommand{have}\isamarkupfalse%
\ {\isacharasterisk}{\kern0pt}{\isacharasterisk}{\kern0pt}{\isacharcolon}{\kern0pt}\ {\isachardoublequoteopen}{\isacharbraceleft}{\kern0pt}y\ {\isasymin}\ space\ M{\isachardot}{\kern0pt}\ {\isacharparenleft}{\kern0pt}{\isasymSum}x{\isasymin}insert\ x\ F{\isachardot}{\kern0pt}\ indicat{\isacharunderscore}{\kern0pt}real\ {\isacharparenleft}{\kern0pt}f\ {\isacharminus}{\kern0pt}{\isacharbackquote}{\kern0pt}\ {\isacharbraceleft}{\kern0pt}x{\isacharbraceright}{\kern0pt}\ {\isasyminter}\ space\ M{\isacharparenright}{\kern0pt}\ y\ {\isacharasterisk}{\kern0pt}\isactrlsub R\ x{\isacharparenright}{\kern0pt}\ {\isasymnoteq}\ {\isadigit{0}}{\isacharbraceright}{\kern0pt}\ {\isasymsubseteq}\ {\isacharbraceleft}{\kern0pt}y\ {\isasymin}\ space\ M{\isachardot}{\kern0pt}\ {\isacharparenleft}{\kern0pt}{\isasymSum}x{\isasymin}F{\isachardot}{\kern0pt}\ indicat{\isacharunderscore}{\kern0pt}real\ {\isacharparenleft}{\kern0pt}f\ {\isacharminus}{\kern0pt}{\isacharbackquote}{\kern0pt}\ {\isacharbraceleft}{\kern0pt}x{\isacharbraceright}{\kern0pt}\ {\isasyminter}\ space\ M{\isacharparenright}{\kern0pt}\ y\ {\isacharasterisk}{\kern0pt}\isactrlsub R\ x{\isacharparenright}{\kern0pt}\ {\isasymnoteq}\ {\isadigit{0}}{\isacharbraceright}{\kern0pt}\ {\isasymunion}\ {\isacharbraceleft}{\kern0pt}y\ {\isasymin}\ space\ M{\isachardot}{\kern0pt}\ indicat{\isacharunderscore}{\kern0pt}real\ {\isacharparenleft}{\kern0pt}f\ {\isacharminus}{\kern0pt}{\isacharbackquote}{\kern0pt}\ {\isacharbraceleft}{\kern0pt}x{\isacharbraceright}{\kern0pt}\ {\isasyminter}\ space\ M{\isacharparenright}{\kern0pt}\ y\ {\isacharasterisk}{\kern0pt}\isactrlsub R\ x\ {\isasymnoteq}\ {\isadigit{0}}{\isacharbraceright}{\kern0pt}{\isachardoublequoteclose}\ \isacommand{unfolding}\isamarkupfalse%
\ {\isacharasterisk}{\kern0pt}\ \isacommand{by}\isamarkupfalse%
\ fastforce\ \ \ \ \isanewline
\ \ \ \ \isacommand{{\isacharbraceleft}{\kern0pt}}\isamarkupfalse%
\isanewline
\ \ \ \ \ \ \isacommand{case}\isamarkupfalse%
\ {\isadigit{1}}\isanewline
\ \ \ \ \ \ \isacommand{show}\isamarkupfalse%
\ {\isacharquery}{\kern0pt}case\ \isanewline
\ \ \ \ \ \ \isacommand{proof}\isamarkupfalse%
\ {\isacharparenleft}{\kern0pt}cases\ {\isachardoublequoteopen}x\ {\isacharequal}{\kern0pt}\ {\isadigit{0}}{\isachardoublequoteclose}{\isacharparenright}{\kern0pt}\isanewline
\ \ \ \ \ \ \ \ \isacommand{case}\isamarkupfalse%
\ True\isanewline
\ \ \ \ \ \ \ \ \isacommand{then}\isamarkupfalse%
\ \isacommand{show}\isamarkupfalse%
\ {\isacharquery}{\kern0pt}thesis\ \isacommand{unfolding}\isamarkupfalse%
\ {\isacharasterisk}{\kern0pt}\ \isacommand{using}\isamarkupfalse%
\ insert\ \isacommand{by}\isamarkupfalse%
\ simp\isanewline
\ \ \ \ \ \ \isacommand{next}\isamarkupfalse%
\isanewline
\ \ \ \ \ \ \ \ \isacommand{case}\isamarkupfalse%
\ False\isanewline
\ \ \ \ \ \ \ \ \isacommand{have}\isamarkupfalse%
\ norm{\isacharunderscore}{\kern0pt}argument{\isacharcolon}{\kern0pt}\ {\isachardoublequoteopen}norm\ {\isacharparenleft}{\kern0pt}{\isacharparenleft}{\kern0pt}{\isasymSum}y{\isasymin}F{\isachardot}{\kern0pt}\ indicat{\isacharunderscore}{\kern0pt}real\ {\isacharparenleft}{\kern0pt}f\ {\isacharminus}{\kern0pt}{\isacharbackquote}{\kern0pt}\ {\isacharbraceleft}{\kern0pt}y{\isacharbraceright}{\kern0pt}\ {\isasyminter}\ space\ M{\isacharparenright}{\kern0pt}\ z\ {\isacharasterisk}{\kern0pt}\isactrlsub R\ y{\isacharparenright}{\kern0pt}\ {\isacharplus}{\kern0pt}\ indicat{\isacharunderscore}{\kern0pt}real\ {\isacharparenleft}{\kern0pt}f\ {\isacharminus}{\kern0pt}{\isacharbackquote}{\kern0pt}\ {\isacharbraceleft}{\kern0pt}x{\isacharbraceright}{\kern0pt}\ {\isasyminter}\ space\ M{\isacharparenright}{\kern0pt}\ z\ {\isacharasterisk}{\kern0pt}\isactrlsub R\ x{\isacharparenright}{\kern0pt}\ {\isacharequal}{\kern0pt}\ norm\ {\isacharparenleft}{\kern0pt}{\isasymSum}y{\isasymin}F{\isachardot}{\kern0pt}\ indicat{\isacharunderscore}{\kern0pt}real\ {\isacharparenleft}{\kern0pt}f\ {\isacharminus}{\kern0pt}{\isacharbackquote}{\kern0pt}\ {\isacharbraceleft}{\kern0pt}y{\isacharbraceright}{\kern0pt}\ {\isasyminter}\ space\ M{\isacharparenright}{\kern0pt}\ z\ {\isacharasterisk}{\kern0pt}\isactrlsub R\ y{\isacharparenright}{\kern0pt}\ {\isacharplus}{\kern0pt}\ norm\ {\isacharparenleft}{\kern0pt}indicat{\isacharunderscore}{\kern0pt}real\ {\isacharparenleft}{\kern0pt}f\ {\isacharminus}{\kern0pt}{\isacharbackquote}{\kern0pt}\ {\isacharbraceleft}{\kern0pt}x{\isacharbraceright}{\kern0pt}\ {\isasyminter}\ space\ M{\isacharparenright}{\kern0pt}\ z\ {\isacharasterisk}{\kern0pt}\isactrlsub R\ x{\isacharparenright}{\kern0pt}{\isachardoublequoteclose}\ \isakeyword{if}\ z{\isacharcolon}{\kern0pt}\ {\isachardoublequoteopen}z\ {\isasymin}\ space\ M{\isachardoublequoteclose}\ \isakeyword{for}\ z\isanewline
\ \ \ \ \ \ \ \ \isacommand{proof}\isamarkupfalse%
\ {\isacharparenleft}{\kern0pt}cases\ {\isachardoublequoteopen}f\ z\ {\isacharequal}{\kern0pt}\ x{\isachardoublequoteclose}{\isacharparenright}{\kern0pt}\isanewline
\ \ \ \ \ \ \ \ \ \ \isacommand{case}\isamarkupfalse%
\ True\isanewline
\ \ \ \ \ \ \ \ \ \ \isacommand{have}\isamarkupfalse%
\ {\isachardoublequoteopen}indicat{\isacharunderscore}{\kern0pt}real\ {\isacharparenleft}{\kern0pt}f\ {\isacharminus}{\kern0pt}{\isacharbackquote}{\kern0pt}\ {\isacharbraceleft}{\kern0pt}y{\isacharbraceright}{\kern0pt}\ {\isasyminter}\ space\ M{\isacharparenright}{\kern0pt}\ z\ {\isacharasterisk}{\kern0pt}\isactrlsub R\ y\ {\isacharequal}{\kern0pt}\ {\isadigit{0}}{\isachardoublequoteclose}\ \isakeyword{if}\ {\isachardoublequoteopen}y\ {\isasymin}\ F{\isachardoublequoteclose}\ \isakeyword{for}\ y\ \isacommand{using}\isamarkupfalse%
\ True\ insert\ z\ that\ {\isadigit{1}}\ \isacommand{unfolding}\isamarkupfalse%
\ indicator{\isacharunderscore}{\kern0pt}def\ \isacommand{by}\isamarkupfalse%
\ force\isanewline
\ \ \ \ \ \ \ \ \ \ \isacommand{hence}\isamarkupfalse%
\ {\isachardoublequoteopen}{\isacharparenleft}{\kern0pt}{\isasymSum}y{\isasymin}F{\isachardot}{\kern0pt}\ indicat{\isacharunderscore}{\kern0pt}real\ {\isacharparenleft}{\kern0pt}f\ {\isacharminus}{\kern0pt}{\isacharbackquote}{\kern0pt}\ {\isacharbraceleft}{\kern0pt}y{\isacharbraceright}{\kern0pt}\ {\isasyminter}\ space\ M{\isacharparenright}{\kern0pt}\ z\ {\isacharasterisk}{\kern0pt}\isactrlsub R\ y{\isacharparenright}{\kern0pt}\ {\isacharequal}{\kern0pt}\ {\isadigit{0}}{\isachardoublequoteclose}\ \isacommand{by}\isamarkupfalse%
\ {\isacharparenleft}{\kern0pt}meson\ sum{\isachardot}{\kern0pt}neutral{\isacharparenright}{\kern0pt}\isanewline
\ \ \ \ \ \ \ \ \ \ \isacommand{thus}\isamarkupfalse%
\ {\isacharquery}{\kern0pt}thesis\ \isacommand{by}\isamarkupfalse%
\ force\isanewline
\ \ \ \ \ \ \ \ \isacommand{qed}\isamarkupfalse%
\ {\isacharparenleft}{\kern0pt}force{\isacharparenright}{\kern0pt}\isanewline
\ \ \ \ \ \ \ \ \isacommand{show}\isamarkupfalse%
\ {\isacharquery}{\kern0pt}thesis\ \isacommand{using}\isamarkupfalse%
\ False\ fin{\isacharunderscore}{\kern0pt}{\isadigit{0}}\ fin{\isacharunderscore}{\kern0pt}{\isadigit{1}}\ f\ norm{\isacharunderscore}{\kern0pt}argument\ \isacommand{by}\isamarkupfalse%
\ {\isacharparenleft}{\kern0pt}subst\ {\isacharasterisk}{\kern0pt}{\isacharcomma}{\kern0pt}\ subst\ add{\isacharcomma}{\kern0pt}\ presburger\ add{\isacharcolon}{\kern0pt}\ insert{\isacharcomma}{\kern0pt}\ intro\ insert{\isacharcomma}{\kern0pt}\ intro\ insert{\isacharcomma}{\kern0pt}\ fastforce\ simp\ add{\isacharcolon}{\kern0pt}\ indicator{\isacharunderscore}{\kern0pt}def\ intro{\isacharbang}{\kern0pt}{\isacharcolon}{\kern0pt}\ insert{\isacharparenleft}{\kern0pt}{\isadigit{2}}{\isacharparenright}{\kern0pt}\ f{\isacharparenleft}{\kern0pt}{\isadigit{3}}{\isacharparenright}{\kern0pt}{\isacharcomma}{\kern0pt}\ auto\ intro{\isacharbang}{\kern0pt}{\isacharcolon}{\kern0pt}\ indicator\ insert\ nonneg{\isacharunderscore}{\kern0pt}x{\isacharparenright}{\kern0pt}\isanewline
\ \ \ \ \ \ \isacommand{qed}\isamarkupfalse%
\ \isanewline
\ \ \ \ \isacommand{next}\isamarkupfalse%
\isanewline
\ \ \ \ \ \ \isacommand{case}\isamarkupfalse%
\ {\isadigit{2}}\isanewline
\ \ \ \ \ \ \isacommand{show}\isamarkupfalse%
\ {\isacharquery}{\kern0pt}case\ \isanewline
\ \ \ \ \ \ \isacommand{proof}\isamarkupfalse%
\ {\isacharparenleft}{\kern0pt}cases\ {\isachardoublequoteopen}x\ {\isacharequal}{\kern0pt}\ {\isadigit{0}}{\isachardoublequoteclose}{\isacharparenright}{\kern0pt}\isanewline
\ \ \ \ \ \ \ \ \isacommand{case}\isamarkupfalse%
\ True\isanewline
\ \ \ \ \ \ \ \ \isacommand{then}\isamarkupfalse%
\ \isacommand{show}\isamarkupfalse%
\ {\isacharquery}{\kern0pt}thesis\ \isacommand{unfolding}\isamarkupfalse%
\ {\isacharasterisk}{\kern0pt}\ \isacommand{using}\isamarkupfalse%
\ insert\ \isacommand{by}\isamarkupfalse%
\ simp\isanewline
\ \ \ \ \ \ \isacommand{next}\isamarkupfalse%
\isanewline
\ \ \ \ \ \ \ \ \isacommand{case}\isamarkupfalse%
\ False\isanewline
\ \ \ \ \ \ \ \ \isacommand{then}\isamarkupfalse%
\ \isacommand{show}\isamarkupfalse%
\ {\isacharquery}{\kern0pt}thesis\ \isacommand{unfolding}\isamarkupfalse%
\ {\isacharasterisk}{\kern0pt}\ \isacommand{using}\isamarkupfalse%
\ insert\ simple{\isacharunderscore}{\kern0pt}functionD{\isacharparenleft}{\kern0pt}{\isadigit{2}}{\isacharparenright}{\kern0pt}{\isacharbrackleft}{\kern0pt}OF\ f{\isacharparenleft}{\kern0pt}{\isadigit{1}}{\isacharparenright}{\kern0pt}{\isacharbrackright}{\kern0pt}\ \isacommand{by}\isamarkupfalse%
\ fast\isanewline
\ \ \ \ \ \ \isacommand{qed}\isamarkupfalse%
\isanewline
\ \ \ \ \isacommand{next}\isamarkupfalse%
\isanewline
\ \ \ \ \ \ \isacommand{case}\isamarkupfalse%
\ {\isadigit{3}}\isanewline
\ \ \ \ \ \ \isacommand{show}\isamarkupfalse%
\ {\isacharquery}{\kern0pt}case\ \isanewline
\ \ \ \ \ \ \isacommand{proof}\isamarkupfalse%
\ {\isacharparenleft}{\kern0pt}cases\ {\isachardoublequoteopen}x\ {\isacharequal}{\kern0pt}\ {\isadigit{0}}{\isachardoublequoteclose}{\isacharparenright}{\kern0pt}\isanewline
\ \ \ \ \ \ \ \ \isacommand{case}\isamarkupfalse%
\ True\isanewline
\ \ \ \ \ \ \ \ \isacommand{then}\isamarkupfalse%
\ \isacommand{show}\isamarkupfalse%
\ {\isacharquery}{\kern0pt}thesis\ \isacommand{unfolding}\isamarkupfalse%
\ {\isacharasterisk}{\kern0pt}\ \isacommand{using}\isamarkupfalse%
\ insert\ \isacommand{by}\isamarkupfalse%
\ simp\isanewline
\ \ \ \ \ \ \isacommand{next}\isamarkupfalse%
\isanewline
\ \ \ \ \ \ \ \ \isacommand{case}\isamarkupfalse%
\ False\isanewline
\ \ \ \ \ \ \ \ \isacommand{have}\isamarkupfalse%
\ {\isachardoublequoteopen}emeasure\ M\ {\isacharbraceleft}{\kern0pt}y\ {\isasymin}\ space\ M{\isachardot}{\kern0pt}\ {\isacharparenleft}{\kern0pt}{\isasymSum}x{\isasymin}insert\ x\ F{\isachardot}{\kern0pt}\ indicat{\isacharunderscore}{\kern0pt}real\ {\isacharparenleft}{\kern0pt}f\ {\isacharminus}{\kern0pt}{\isacharbackquote}{\kern0pt}\ {\isacharbraceleft}{\kern0pt}x{\isacharbraceright}{\kern0pt}\ {\isasyminter}\ space\ M{\isacharparenright}{\kern0pt}\ y\ {\isacharasterisk}{\kern0pt}\isactrlsub R\ x{\isacharparenright}{\kern0pt}\ {\isasymnoteq}\ {\isadigit{0}}{\isacharbraceright}{\kern0pt}\ {\isasymle}\ emeasure\ M\ {\isacharparenleft}{\kern0pt}{\isacharbraceleft}{\kern0pt}y\ {\isasymin}\ space\ M{\isachardot}{\kern0pt}\ {\isacharparenleft}{\kern0pt}{\isasymSum}x{\isasymin}F{\isachardot}{\kern0pt}\ indicat{\isacharunderscore}{\kern0pt}real\ {\isacharparenleft}{\kern0pt}f\ {\isacharminus}{\kern0pt}{\isacharbackquote}{\kern0pt}\ {\isacharbraceleft}{\kern0pt}x{\isacharbraceright}{\kern0pt}\ {\isasyminter}\ space\ M{\isacharparenright}{\kern0pt}\ y\ {\isacharasterisk}{\kern0pt}\isactrlsub R\ x{\isacharparenright}{\kern0pt}\ {\isasymnoteq}\ {\isadigit{0}}{\isacharbraceright}{\kern0pt}\ {\isasymunion}\ {\isacharbraceleft}{\kern0pt}y\ {\isasymin}\ space\ M{\isachardot}{\kern0pt}\ indicat{\isacharunderscore}{\kern0pt}real\ {\isacharparenleft}{\kern0pt}f\ {\isacharminus}{\kern0pt}{\isacharbackquote}{\kern0pt}\ {\isacharbraceleft}{\kern0pt}x{\isacharbraceright}{\kern0pt}\ {\isasyminter}\ space\ M{\isacharparenright}{\kern0pt}\ y\ {\isacharasterisk}{\kern0pt}\isactrlsub R\ x\ {\isasymnoteq}\ {\isadigit{0}}{\isacharbraceright}{\kern0pt}{\isacharparenright}{\kern0pt}{\isachardoublequoteclose}\isanewline
\ \ \ \ \ \ \ \ \ \ \isacommand{using}\isamarkupfalse%
\ {\isacharasterisk}{\kern0pt}{\isacharasterisk}{\kern0pt}\ simple{\isacharunderscore}{\kern0pt}functionD{\isacharparenleft}{\kern0pt}{\isadigit{2}}{\isacharparenright}{\kern0pt}{\isacharbrackleft}{\kern0pt}OF\ insert{\isacharparenleft}{\kern0pt}{\isadigit{6}}{\isacharparenright}{\kern0pt}{\isacharbrackright}{\kern0pt}\ simple{\isacharunderscore}{\kern0pt}functionD{\isacharparenleft}{\kern0pt}{\isadigit{2}}{\isacharparenright}{\kern0pt}{\isacharbrackleft}{\kern0pt}OF\ f{\isacharparenleft}{\kern0pt}{\isadigit{1}}{\isacharparenright}{\kern0pt}{\isacharbrackright}{\kern0pt}\ insert{\isachardot}{\kern0pt}IH{\isacharparenleft}{\kern0pt}{\isadigit{2}}{\isacharparenright}{\kern0pt}\ \isacommand{by}\isamarkupfalse%
\ {\isacharparenleft}{\kern0pt}intro\ emeasure{\isacharunderscore}{\kern0pt}mono{\isacharcomma}{\kern0pt}\ blast{\isacharcomma}{\kern0pt}\ simp{\isacharparenright}{\kern0pt}\ \isanewline
\ \ \ \ \ \ \ \ \isacommand{also}\isamarkupfalse%
\ \isacommand{have}\isamarkupfalse%
\ {\isachardoublequoteopen}{\isachardot}{\kern0pt}{\isachardot}{\kern0pt}{\isachardot}{\kern0pt}\ {\isasymle}\ emeasure\ M\ {\isacharbraceleft}{\kern0pt}y\ {\isasymin}\ space\ M{\isachardot}{\kern0pt}\ {\isacharparenleft}{\kern0pt}{\isasymSum}x{\isasymin}F{\isachardot}{\kern0pt}\ indicat{\isacharunderscore}{\kern0pt}real\ {\isacharparenleft}{\kern0pt}f\ {\isacharminus}{\kern0pt}{\isacharbackquote}{\kern0pt}\ {\isacharbraceleft}{\kern0pt}x{\isacharbraceright}{\kern0pt}\ {\isasyminter}\ space\ M{\isacharparenright}{\kern0pt}\ y\ {\isacharasterisk}{\kern0pt}\isactrlsub R\ x{\isacharparenright}{\kern0pt}\ {\isasymnoteq}\ {\isadigit{0}}{\isacharbraceright}{\kern0pt}\ {\isacharplus}{\kern0pt}\ emeasure\ M\ {\isacharbraceleft}{\kern0pt}y\ {\isasymin}\ space\ M{\isachardot}{\kern0pt}\ indicat{\isacharunderscore}{\kern0pt}real\ {\isacharparenleft}{\kern0pt}f\ {\isacharminus}{\kern0pt}{\isacharbackquote}{\kern0pt}\ {\isacharbraceleft}{\kern0pt}x{\isacharbraceright}{\kern0pt}\ {\isasyminter}\ space\ M{\isacharparenright}{\kern0pt}\ y\ {\isacharasterisk}{\kern0pt}\isactrlsub R\ x\ {\isasymnoteq}\ {\isadigit{0}}{\isacharbraceright}{\kern0pt}{\isachardoublequoteclose}\isanewline
\ \ \ \ \ \ \ \ \ \ \isacommand{using}\isamarkupfalse%
\ simple{\isacharunderscore}{\kern0pt}functionD{\isacharparenleft}{\kern0pt}{\isadigit{2}}{\isacharparenright}{\kern0pt}{\isacharbrackleft}{\kern0pt}OF\ insert{\isacharparenleft}{\kern0pt}{\isadigit{6}}{\isacharparenright}{\kern0pt}{\isacharbrackright}{\kern0pt}\ simple{\isacharunderscore}{\kern0pt}functionD{\isacharparenleft}{\kern0pt}{\isadigit{2}}{\isacharparenright}{\kern0pt}{\isacharbrackleft}{\kern0pt}OF\ f{\isacharparenleft}{\kern0pt}{\isadigit{1}}{\isacharparenright}{\kern0pt}{\isacharbrackright}{\kern0pt}\ \isacommand{by}\isamarkupfalse%
\ {\isacharparenleft}{\kern0pt}intro\ emeasure{\isacharunderscore}{\kern0pt}subadditive{\isacharcomma}{\kern0pt}\ force{\isacharplus}{\kern0pt}{\isacharparenright}{\kern0pt}\isanewline
\ \ \ \ \ \ \ \ \isacommand{also}\isamarkupfalse%
\ \isacommand{have}\isamarkupfalse%
\ {\isachardoublequoteopen}{\isachardot}{\kern0pt}{\isachardot}{\kern0pt}{\isachardot}{\kern0pt}\ {\isacharless}{\kern0pt}\ {\isasyminfinity}{\isachardoublequoteclose}\ \isacommand{using}\isamarkupfalse%
\ insert{\isacharparenleft}{\kern0pt}{\isadigit{7}}{\isacharparenright}{\kern0pt}\ fin{\isacharunderscore}{\kern0pt}{\isadigit{1}}{\isacharbrackleft}{\kern0pt}OF\ False{\isacharbrackright}{\kern0pt}\ \isacommand{by}\isamarkupfalse%
\ {\isacharparenleft}{\kern0pt}simp\ add{\isacharcolon}{\kern0pt}\ less{\isacharunderscore}{\kern0pt}top{\isacharparenright}{\kern0pt}\isanewline
\ \ \ \ \ \ \ \ \isacommand{finally}\isamarkupfalse%
\ \isacommand{show}\isamarkupfalse%
\ {\isacharquery}{\kern0pt}thesis\ \isacommand{by}\isamarkupfalse%
\ simp\isanewline
\ \ \ \ \ \ \isacommand{qed}\isamarkupfalse%
\isanewline
\ \ \ \ \isacommand{next}\isamarkupfalse%
\isanewline
\ \ \ \ \ \ \isacommand{case}\isamarkupfalse%
\ {\isacharparenleft}{\kern0pt}{\isadigit{4}}\ {\isasymxi}{\isacharparenright}{\kern0pt}\isanewline
\ \ \ \ \ \ \isacommand{thus}\isamarkupfalse%
\ {\isacharquery}{\kern0pt}case\ \isacommand{using}\isamarkupfalse%
\ insert\ nonneg{\isacharunderscore}{\kern0pt}x\ f{\isacharparenleft}{\kern0pt}{\isadigit{3}}{\isacharparenright}{\kern0pt}\ \isacommand{by}\isamarkupfalse%
\ {\isacharparenleft}{\kern0pt}auto\ simp\ add{\isacharcolon}{\kern0pt}\ scaleR{\isacharunderscore}{\kern0pt}nonneg{\isacharunderscore}{\kern0pt}nonneg\ intro{\isacharcolon}{\kern0pt}\ sum{\isacharunderscore}{\kern0pt}nonneg{\isacharparenright}{\kern0pt}\isanewline
\ \ \ \ \isacommand{{\isacharbraceright}{\kern0pt}}\isamarkupfalse%
\isanewline
\ \ \isacommand{qed}\isamarkupfalse%
\isanewline
\ \ \isacommand{moreover}\isamarkupfalse%
\ \isacommand{have}\isamarkupfalse%
\ {\isachardoublequoteopen}simple{\isacharunderscore}{\kern0pt}function\ M\ {\isacharparenleft}{\kern0pt}{\isasymlambda}x{\isachardot}{\kern0pt}\ {\isasymSum}y{\isasymin}f\ {\isacharbackquote}{\kern0pt}\ space\ M{\isachardot}{\kern0pt}\ indicat{\isacharunderscore}{\kern0pt}real\ {\isacharparenleft}{\kern0pt}f\ {\isacharminus}{\kern0pt}{\isacharbackquote}{\kern0pt}\ {\isacharbraceleft}{\kern0pt}y{\isacharbraceright}{\kern0pt}\ {\isasyminter}\ space\ M{\isacharparenright}{\kern0pt}\ x\ {\isacharasterisk}{\kern0pt}\isactrlsub R\ y{\isacharparenright}{\kern0pt}{\isachardoublequoteclose}\ \isacommand{using}\isamarkupfalse%
\ calculation\ \isacommand{by}\isamarkupfalse%
\ blast\isanewline
\ \ \isacommand{moreover}\isamarkupfalse%
\ \isacommand{have}\isamarkupfalse%
\ {\isachardoublequoteopen}P\ {\isacharparenleft}{\kern0pt}{\isasymlambda}x{\isachardot}{\kern0pt}\ {\isasymSum}y{\isasymin}f\ {\isacharbackquote}{\kern0pt}\ space\ M{\isachardot}{\kern0pt}\ indicat{\isacharunderscore}{\kern0pt}real\ {\isacharparenleft}{\kern0pt}f\ {\isacharminus}{\kern0pt}{\isacharbackquote}{\kern0pt}\ {\isacharbraceleft}{\kern0pt}y{\isacharbraceright}{\kern0pt}\ {\isasyminter}\ space\ M{\isacharparenright}{\kern0pt}\ x\ {\isacharasterisk}{\kern0pt}\isactrlsub R\ y{\isacharparenright}{\kern0pt}{\isachardoublequoteclose}\ \isacommand{using}\isamarkupfalse%
\ calculation\ \isacommand{by}\isamarkupfalse%
\ blast\isanewline
\ \ \isacommand{moreover}\isamarkupfalse%
\ \isacommand{have}\isamarkupfalse%
\ {\isachardoublequoteopen}{\isasymAnd}x{\isachardot}{\kern0pt}\ x\ {\isasymin}\ space\ M\ {\isasymLongrightarrow}\ {\isadigit{0}}\ {\isasymle}\ f\ x{\isachardoublequoteclose}\ \isacommand{using}\isamarkupfalse%
\ f{\isacharparenleft}{\kern0pt}{\isadigit{3}}{\isacharparenright}{\kern0pt}\ \isacommand{by}\isamarkupfalse%
\ simp\isanewline
\ \ \isacommand{ultimately}\isamarkupfalse%
\ \isacommand{show}\isamarkupfalse%
\ {\isacharquery}{\kern0pt}thesis\ \isacommand{by}\isamarkupfalse%
\ {\isacharparenleft}{\kern0pt}intro\ cong{\isacharbrackleft}{\kern0pt}OF\ {\isacharunderscore}{\kern0pt}\ {\isacharunderscore}{\kern0pt}\ {\isacharunderscore}{\kern0pt}\ f{\isacharparenleft}{\kern0pt}{\isadigit{1}}{\isacharcomma}{\kern0pt}{\isadigit{2}}{\isacharparenright}{\kern0pt}{\isacharbrackright}{\kern0pt}{\isacharcomma}{\kern0pt}\ blast{\isacharcomma}{\kern0pt}\ blast{\isacharcomma}{\kern0pt}\ fast{\isacharparenright}{\kern0pt}\ presburger{\isacharplus}{\kern0pt}\isanewline
\isacommand{qed}\isamarkupfalse%
%
\endisatagproof
{\isafoldproof}%
%
\isadelimproof
\isanewline
%
\endisadelimproof
\isanewline
\isacommand{lemma}\isamarkupfalse%
\ finite{\isacharunderscore}{\kern0pt}nn{\isacharunderscore}{\kern0pt}integral{\isacharunderscore}{\kern0pt}imp{\isacharunderscore}{\kern0pt}ae{\isacharunderscore}{\kern0pt}finite{\isacharcolon}{\kern0pt}\isanewline
\ \ \isakeyword{fixes}\ f\ {\isacharcolon}{\kern0pt}{\isacharcolon}{\kern0pt}\ {\isachardoublequoteopen}{\isacharprime}{\kern0pt}a\ {\isasymRightarrow}\ ennreal{\isachardoublequoteclose}\isanewline
\ \ \isakeyword{assumes}\ {\isachardoublequoteopen}f\ {\isasymin}\ borel{\isacharunderscore}{\kern0pt}measurable\ M{\isachardoublequoteclose}\ {\isachardoublequoteopen}{\isacharparenleft}{\kern0pt}{\isasymintegral}\isactrlsup {\isacharplus}{\kern0pt}x{\isachardot}{\kern0pt}\ f\ x\ {\isasympartial}M{\isacharparenright}{\kern0pt}\ {\isacharless}{\kern0pt}\ {\isasyminfinity}{\isachardoublequoteclose}\isanewline
\ \ \isakeyword{shows}\ {\isachardoublequoteopen}AE\ x\ in\ M{\isachardot}{\kern0pt}\ f\ x\ {\isacharless}{\kern0pt}\ {\isasyminfinity}{\isachardoublequoteclose}\isanewline
%
\isadelimproof
%
\endisadelimproof
%
\isatagproof
\isacommand{proof}\isamarkupfalse%
\ {\isacharparenleft}{\kern0pt}rule\ ccontr{\isacharcomma}{\kern0pt}\ goal{\isacharunderscore}{\kern0pt}cases{\isacharparenright}{\kern0pt}\isanewline
\ \ \isacommand{case}\isamarkupfalse%
\ {\isadigit{1}}\isanewline
\ \ \isacommand{let}\isamarkupfalse%
\ {\isacharquery}{\kern0pt}A\ {\isacharequal}{\kern0pt}\ {\isachardoublequoteopen}space\ M\ {\isasyminter}\ {\isacharbraceleft}{\kern0pt}x{\isachardot}{\kern0pt}\ f\ x\ {\isacharequal}{\kern0pt}\ {\isasyminfinity}{\isacharbraceright}{\kern0pt}{\isachardoublequoteclose}\isanewline
\ \ \isacommand{have}\isamarkupfalse%
\ {\isacharasterisk}{\kern0pt}{\isacharcolon}{\kern0pt}\ {\isachardoublequoteopen}emeasure\ M\ {\isacharquery}{\kern0pt}A\ {\isachargreater}{\kern0pt}\ {\isadigit{0}}{\isachardoublequoteclose}\ \isacommand{using}\isamarkupfalse%
\ {\isadigit{1}}\ assms{\isacharparenleft}{\kern0pt}{\isadigit{1}}{\isacharparenright}{\kern0pt}\ \isacommand{by}\isamarkupfalse%
\ {\isacharparenleft}{\kern0pt}metis\ {\isacharparenleft}{\kern0pt}mono{\isacharunderscore}{\kern0pt}tags{\isacharcomma}{\kern0pt}\ lifting{\isacharparenright}{\kern0pt}\ assms{\isacharparenleft}{\kern0pt}{\isadigit{2}}{\isacharparenright}{\kern0pt}\ eventually{\isacharunderscore}{\kern0pt}mono\ infinity{\isacharunderscore}{\kern0pt}ennreal{\isacharunderscore}{\kern0pt}def\ nn{\isacharunderscore}{\kern0pt}integral{\isacharunderscore}{\kern0pt}noteq{\isacharunderscore}{\kern0pt}infinite\ top{\isachardot}{\kern0pt}not{\isacharunderscore}{\kern0pt}eq{\isacharunderscore}{\kern0pt}extremum{\isacharparenright}{\kern0pt}\isanewline
\ \ \isacommand{have}\isamarkupfalse%
\ {\isachardoublequoteopen}{\isacharparenleft}{\kern0pt}{\isasymintegral}\isactrlsup {\isacharplus}{\kern0pt}x{\isachardot}{\kern0pt}\ f\ x\ {\isacharasterisk}{\kern0pt}\ indicator\ {\isacharquery}{\kern0pt}A\ x\ {\isasympartial}M{\isacharparenright}{\kern0pt}\ {\isacharequal}{\kern0pt}\ {\isacharparenleft}{\kern0pt}{\isasymintegral}\isactrlsup {\isacharplus}{\kern0pt}x{\isachardot}{\kern0pt}\ {\isasyminfinity}\ {\isacharasterisk}{\kern0pt}\ indicator\ {\isacharquery}{\kern0pt}A\ x\ {\isasympartial}M{\isacharparenright}{\kern0pt}{\isachardoublequoteclose}\ \isacommand{by}\isamarkupfalse%
\ {\isacharparenleft}{\kern0pt}metis\ {\isacharparenleft}{\kern0pt}mono{\isacharunderscore}{\kern0pt}tags{\isacharcomma}{\kern0pt}\ lifting{\isacharparenright}{\kern0pt}\ indicator{\isacharunderscore}{\kern0pt}inter{\isacharunderscore}{\kern0pt}arith\ indicator{\isacharunderscore}{\kern0pt}simps{\isacharparenleft}{\kern0pt}{\isadigit{2}}{\isacharparenright}{\kern0pt}\ mem{\isacharunderscore}{\kern0pt}Collect{\isacharunderscore}{\kern0pt}eq\ mult{\isacharunderscore}{\kern0pt}eq{\isacharunderscore}{\kern0pt}{\isadigit{0}}{\isacharunderscore}{\kern0pt}iff{\isacharparenright}{\kern0pt}\isanewline
\ \ \isacommand{also}\isamarkupfalse%
\ \isacommand{have}\isamarkupfalse%
\ {\isachardoublequoteopen}{\isachardot}{\kern0pt}{\isachardot}{\kern0pt}{\isachardot}{\kern0pt}\ {\isacharequal}{\kern0pt}\ {\isasyminfinity}\ {\isacharasterisk}{\kern0pt}\ emeasure\ M\ {\isacharquery}{\kern0pt}A{\isachardoublequoteclose}\ \isacommand{using}\isamarkupfalse%
\ assms{\isacharparenleft}{\kern0pt}{\isadigit{1}}{\isacharparenright}{\kern0pt}\ \isacommand{by}\isamarkupfalse%
\ {\isacharparenleft}{\kern0pt}intro\ nn{\isacharunderscore}{\kern0pt}integral{\isacharunderscore}{\kern0pt}cmult{\isacharunderscore}{\kern0pt}indicator{\isacharcomma}{\kern0pt}\ simp{\isacharparenright}{\kern0pt}\isanewline
\ \ \isacommand{also}\isamarkupfalse%
\ \isacommand{have}\isamarkupfalse%
\ {\isachardoublequoteopen}{\isachardot}{\kern0pt}{\isachardot}{\kern0pt}{\isachardot}{\kern0pt}\ {\isacharequal}{\kern0pt}\ {\isasyminfinity}{\isachardoublequoteclose}\ \isacommand{using}\isamarkupfalse%
\ {\isacharasterisk}{\kern0pt}\ \isacommand{by}\isamarkupfalse%
\ fastforce\isanewline
\ \ \isacommand{finally}\isamarkupfalse%
\ \isacommand{have}\isamarkupfalse%
\ {\isachardoublequoteopen}{\isacharparenleft}{\kern0pt}{\isasymintegral}\isactrlsup {\isacharplus}{\kern0pt}x{\isachardot}{\kern0pt}\ f\ x\ {\isacharasterisk}{\kern0pt}\ indicator\ {\isacharquery}{\kern0pt}A\ x\ {\isasympartial}M{\isacharparenright}{\kern0pt}\ {\isacharequal}{\kern0pt}\ {\isasyminfinity}{\isachardoublequoteclose}\ \isacommand{{\isachardot}{\kern0pt}}\isamarkupfalse%
\isanewline
\ \ \isacommand{moreover}\isamarkupfalse%
\ \isacommand{have}\isamarkupfalse%
\ {\isachardoublequoteopen}{\isacharparenleft}{\kern0pt}{\isasymintegral}\isactrlsup {\isacharplus}{\kern0pt}x{\isachardot}{\kern0pt}\ f\ x\ {\isacharasterisk}{\kern0pt}\ indicator\ {\isacharquery}{\kern0pt}A\ x\ {\isasympartial}M{\isacharparenright}{\kern0pt}\ {\isasymle}\ {\isacharparenleft}{\kern0pt}{\isasymintegral}\isactrlsup {\isacharplus}{\kern0pt}x{\isachardot}{\kern0pt}\ f\ x\ {\isasympartial}M{\isacharparenright}{\kern0pt}{\isachardoublequoteclose}\ \isacommand{by}\isamarkupfalse%
\ {\isacharparenleft}{\kern0pt}intro\ nn{\isacharunderscore}{\kern0pt}integral{\isacharunderscore}{\kern0pt}mono{\isacharcomma}{\kern0pt}\ simp\ add{\isacharcolon}{\kern0pt}\ indicator{\isacharunderscore}{\kern0pt}def{\isacharparenright}{\kern0pt}\isanewline
\ \ \isacommand{ultimately}\isamarkupfalse%
\ \isacommand{show}\isamarkupfalse%
\ {\isacharquery}{\kern0pt}case\ \isacommand{using}\isamarkupfalse%
\ assms{\isacharparenleft}{\kern0pt}{\isadigit{2}}{\isacharparenright}{\kern0pt}\ \isacommand{by}\isamarkupfalse%
\ simp\isanewline
\isacommand{qed}\isamarkupfalse%
%
\endisatagproof
{\isafoldproof}%
%
\isadelimproof
\isanewline
%
\endisadelimproof
\isanewline
\isacommand{lemma}\isamarkupfalse%
\ cauchy{\isacharunderscore}{\kern0pt}L{\isadigit{1}}{\isacharunderscore}{\kern0pt}AE{\isacharunderscore}{\kern0pt}cauchy{\isacharunderscore}{\kern0pt}subseq{\isacharcolon}{\kern0pt}\isanewline
\ \ \isakeyword{fixes}\ s\ {\isacharcolon}{\kern0pt}{\isacharcolon}{\kern0pt}\ {\isachardoublequoteopen}nat\ {\isasymRightarrow}\ {\isacharprime}{\kern0pt}a\ {\isasymRightarrow}\ {\isacharprime}{\kern0pt}b{\isacharcolon}{\kern0pt}{\isacharcolon}{\kern0pt}{\isacharbraceleft}{\kern0pt}banach{\isacharcomma}{\kern0pt}\ second{\isacharunderscore}{\kern0pt}countable{\isacharunderscore}{\kern0pt}topology{\isacharbraceright}{\kern0pt}{\isachardoublequoteclose}\isanewline
\ \ \isakeyword{assumes}\ {\isacharbrackleft}{\kern0pt}measurable{\isacharbrackright}{\kern0pt}{\isacharcolon}{\kern0pt}\ {\isachardoublequoteopen}{\isasymAnd}n{\isachardot}{\kern0pt}\ integrable\ M\ {\isacharparenleft}{\kern0pt}s\ n{\isacharparenright}{\kern0pt}{\isachardoublequoteclose}\isanewline
\ \ \ \ \ \ \isakeyword{and}\ {\isachardoublequoteopen}{\isasymAnd}e{\isachardot}{\kern0pt}\ e\ {\isachargreater}{\kern0pt}\ {\isadigit{0}}\ {\isasymLongrightarrow}\ {\isasymexists}N{\isachardot}{\kern0pt}\ {\isasymforall}i{\isasymge}N{\isachardot}{\kern0pt}\ {\isasymforall}j{\isasymge}N{\isachardot}{\kern0pt}\ LINT\ x{\isacharbar}{\kern0pt}M{\isachardot}{\kern0pt}\ dist\ {\isacharparenleft}{\kern0pt}s\ i\ x{\isacharparenright}{\kern0pt}\ {\isacharparenleft}{\kern0pt}s\ j\ x{\isacharparenright}{\kern0pt}\ {\isacharless}{\kern0pt}\ e{\isachardoublequoteclose}\isanewline
\ \ \isakeyword{obtains}\ r\ \isakeyword{where}\ {\isachardoublequoteopen}strict{\isacharunderscore}{\kern0pt}mono\ r{\isachardoublequoteclose}\ {\isachardoublequoteopen}AE\ x\ in\ M{\isachardot}{\kern0pt}\ Cauchy\ {\isacharparenleft}{\kern0pt}{\isasymlambda}i{\isachardot}{\kern0pt}\ s\ {\isacharparenleft}{\kern0pt}r\ i{\isacharparenright}{\kern0pt}\ x{\isacharparenright}{\kern0pt}{\isachardoublequoteclose}\isanewline
%
\isadelimproof
%
\endisadelimproof
%
\isatagproof
\isacommand{proof}\isamarkupfalse%
{\isacharminus}{\kern0pt}\isanewline
\ \ \isacommand{have}\isamarkupfalse%
\ {\isachardoublequoteopen}{\isasymexists}r{\isachardot}{\kern0pt}\ {\isasymforall}n{\isachardot}{\kern0pt}\ {\isacharparenleft}{\kern0pt}{\isasymforall}i{\isasymge}r\ n{\isachardot}{\kern0pt}\ {\isasymforall}j{\isasymge}\ r\ n{\isachardot}{\kern0pt}\ LINT\ x{\isacharbar}{\kern0pt}M{\isachardot}{\kern0pt}\ dist\ {\isacharparenleft}{\kern0pt}s\ i\ x{\isacharparenright}{\kern0pt}\ {\isacharparenleft}{\kern0pt}s\ j\ x{\isacharparenright}{\kern0pt}\ {\isacharless}{\kern0pt}\ {\isacharparenleft}{\kern0pt}{\isadigit{1}}\ {\isacharslash}{\kern0pt}\ {\isadigit{2}}{\isacharparenright}{\kern0pt}\ {\isacharcircum}{\kern0pt}\ n{\isacharparenright}{\kern0pt}\ {\isasymand}\ {\isacharparenleft}{\kern0pt}r\ {\isacharparenleft}{\kern0pt}Suc\ n{\isacharparenright}{\kern0pt}\ {\isachargreater}{\kern0pt}\ r\ n{\isacharparenright}{\kern0pt}{\isachardoublequoteclose}\isanewline
\ \ \isacommand{proof}\isamarkupfalse%
\ {\isacharparenleft}{\kern0pt}intro\ dependent{\isacharunderscore}{\kern0pt}nat{\isacharunderscore}{\kern0pt}choice{\isacharcomma}{\kern0pt}\ goal{\isacharunderscore}{\kern0pt}cases{\isacharparenright}{\kern0pt}\isanewline
\ \ \ \ \isacommand{case}\isamarkupfalse%
\ {\isadigit{1}}\isanewline
\ \ \ \ \isacommand{then}\isamarkupfalse%
\ \isacommand{show}\isamarkupfalse%
\ {\isacharquery}{\kern0pt}case\ \isacommand{using}\isamarkupfalse%
\ assms{\isacharparenleft}{\kern0pt}{\isadigit{2}}{\isacharparenright}{\kern0pt}\ \isacommand{by}\isamarkupfalse%
\ presburger\isanewline
\ \ \isacommand{next}\isamarkupfalse%
\isanewline
\ \ \ \ \isacommand{case}\isamarkupfalse%
\ {\isacharparenleft}{\kern0pt}{\isadigit{2}}\ x\ n{\isacharparenright}{\kern0pt}\isanewline
\ \ \ \ \isacommand{obtain}\isamarkupfalse%
\ N\ \isakeyword{where}\ {\isacharasterisk}{\kern0pt}{\isacharcolon}{\kern0pt}\ {\isachardoublequoteopen}LINT\ x{\isacharbar}{\kern0pt}M{\isachardot}{\kern0pt}\ dist\ {\isacharparenleft}{\kern0pt}s\ i\ x{\isacharparenright}{\kern0pt}\ {\isacharparenleft}{\kern0pt}s\ j\ x{\isacharparenright}{\kern0pt}\ {\isacharless}{\kern0pt}\ {\isacharparenleft}{\kern0pt}{\isadigit{1}}\ {\isacharslash}{\kern0pt}\ {\isadigit{2}}{\isacharparenright}{\kern0pt}\ {\isacharcircum}{\kern0pt}\ Suc\ n{\isachardoublequoteclose}\ \isakeyword{if}\ {\isachardoublequoteopen}i\ {\isasymge}\ N{\isachardoublequoteclose}\ {\isachardoublequoteopen}j\ {\isasymge}\ N{\isachardoublequoteclose}\ \isakeyword{for}\ i\ j\ \isacommand{using}\isamarkupfalse%
\ assms{\isacharparenleft}{\kern0pt}{\isadigit{2}}{\isacharparenright}{\kern0pt}{\isacharbrackleft}{\kern0pt}of\ {\isachardoublequoteopen}{\isacharparenleft}{\kern0pt}{\isadigit{1}}\ {\isacharslash}{\kern0pt}\ {\isadigit{2}}{\isacharparenright}{\kern0pt}\ {\isacharcircum}{\kern0pt}\ Suc\ n{\isachardoublequoteclose}{\isacharbrackright}{\kern0pt}\ \isacommand{by}\isamarkupfalse%
\ auto\isanewline
\ \ \ \ \isacommand{{\isacharbraceleft}{\kern0pt}}\isamarkupfalse%
\isanewline
\ \ \ \ \ \ \isacommand{fix}\isamarkupfalse%
\ i\ j\ \isacommand{assume}\isamarkupfalse%
\ {\isachardoublequoteopen}i\ {\isasymge}\ max\ N\ {\isacharparenleft}{\kern0pt}Suc\ x{\isacharparenright}{\kern0pt}{\isachardoublequoteclose}\ {\isachardoublequoteopen}j\ {\isasymge}\ max\ N\ {\isacharparenleft}{\kern0pt}Suc\ x{\isacharparenright}{\kern0pt}{\isachardoublequoteclose}\isanewline
\ \ \ \ \ \ \isacommand{hence}\isamarkupfalse%
\ {\isachardoublequoteopen}integral\isactrlsup L\ M\ {\isacharparenleft}{\kern0pt}{\isasymlambda}x{\isachardot}{\kern0pt}\ dist\ {\isacharparenleft}{\kern0pt}s\ i\ x{\isacharparenright}{\kern0pt}\ {\isacharparenleft}{\kern0pt}s\ j\ x{\isacharparenright}{\kern0pt}{\isacharparenright}{\kern0pt}\ {\isacharless}{\kern0pt}\ {\isacharparenleft}{\kern0pt}{\isadigit{1}}\ {\isacharslash}{\kern0pt}\ {\isadigit{2}}{\isacharparenright}{\kern0pt}\ {\isacharcircum}{\kern0pt}\ Suc\ n{\isachardoublequoteclose}\ \isacommand{using}\isamarkupfalse%
\ {\isacharasterisk}{\kern0pt}\ \isacommand{by}\isamarkupfalse%
\ fastforce\isanewline
\ \ \ \ \isacommand{{\isacharbraceright}{\kern0pt}}\isamarkupfalse%
\isanewline
\ \ \ \ \isacommand{then}\isamarkupfalse%
\ \isacommand{show}\isamarkupfalse%
\ {\isacharquery}{\kern0pt}case\ \isacommand{by}\isamarkupfalse%
\ fastforce\isanewline
\ \ \isacommand{qed}\isamarkupfalse%
\isanewline
\ \ \isacommand{then}\isamarkupfalse%
\ \isacommand{obtain}\isamarkupfalse%
\ r\ \isakeyword{where}\ strict{\isacharunderscore}{\kern0pt}mono{\isacharcolon}{\kern0pt}\ {\isachardoublequoteopen}strict{\isacharunderscore}{\kern0pt}mono\ r{\isachardoublequoteclose}\ \isakeyword{and}\ {\isachardoublequoteopen}{\isasymforall}i{\isasymge}r\ n{\isachardot}{\kern0pt}\ {\isasymforall}j{\isasymge}\ r\ n{\isachardot}{\kern0pt}\ LINT\ x{\isacharbar}{\kern0pt}M{\isachardot}{\kern0pt}\ dist\ {\isacharparenleft}{\kern0pt}s\ i\ x{\isacharparenright}{\kern0pt}\ {\isacharparenleft}{\kern0pt}s\ j\ x{\isacharparenright}{\kern0pt}\ {\isacharless}{\kern0pt}\ {\isacharparenleft}{\kern0pt}{\isadigit{1}}\ {\isacharslash}{\kern0pt}\ {\isadigit{2}}{\isacharparenright}{\kern0pt}\ {\isacharcircum}{\kern0pt}\ n{\isachardoublequoteclose}\ \isakeyword{for}\ n\ \isacommand{using}\isamarkupfalse%
\ strict{\isacharunderscore}{\kern0pt}mono{\isacharunderscore}{\kern0pt}Suc{\isacharunderscore}{\kern0pt}iff\ \isacommand{by}\isamarkupfalse%
\ blast\isanewline
\ \ \isacommand{hence}\isamarkupfalse%
\ r{\isacharunderscore}{\kern0pt}is{\isacharcolon}{\kern0pt}\ {\isachardoublequoteopen}LINT\ x{\isacharbar}{\kern0pt}M{\isachardot}{\kern0pt}\ dist\ {\isacharparenleft}{\kern0pt}s\ {\isacharparenleft}{\kern0pt}r\ {\isacharparenleft}{\kern0pt}Suc\ n{\isacharparenright}{\kern0pt}{\isacharparenright}{\kern0pt}\ x{\isacharparenright}{\kern0pt}\ {\isacharparenleft}{\kern0pt}s\ {\isacharparenleft}{\kern0pt}r\ n{\isacharparenright}{\kern0pt}\ x{\isacharparenright}{\kern0pt}\ {\isacharless}{\kern0pt}\ {\isacharparenleft}{\kern0pt}{\isadigit{1}}\ {\isacharslash}{\kern0pt}\ {\isadigit{2}}{\isacharparenright}{\kern0pt}\ {\isacharcircum}{\kern0pt}\ n{\isachardoublequoteclose}\ \isakeyword{for}\ n\ \isacommand{by}\isamarkupfalse%
\ {\isacharparenleft}{\kern0pt}simp\ add{\isacharcolon}{\kern0pt}\ strict{\isacharunderscore}{\kern0pt}mono{\isacharunderscore}{\kern0pt}leD{\isacharparenright}{\kern0pt}\isanewline
\isanewline
\ \ \isacommand{define}\isamarkupfalse%
\ g\ \isakeyword{where}\ {\isachardoublequoteopen}g\ {\isacharequal}{\kern0pt}\ {\isacharparenleft}{\kern0pt}{\isasymlambda}n\ x{\isachardot}{\kern0pt}\ {\isacharparenleft}{\kern0pt}{\isasymSum}i{\isasymle}n{\isachardot}{\kern0pt}\ ennreal\ {\isacharparenleft}{\kern0pt}dist\ {\isacharparenleft}{\kern0pt}s\ {\isacharparenleft}{\kern0pt}r\ {\isacharparenleft}{\kern0pt}Suc\ i{\isacharparenright}{\kern0pt}{\isacharparenright}{\kern0pt}\ x{\isacharparenright}{\kern0pt}\ {\isacharparenleft}{\kern0pt}s\ {\isacharparenleft}{\kern0pt}r\ i{\isacharparenright}{\kern0pt}\ x{\isacharparenright}{\kern0pt}{\isacharparenright}{\kern0pt}{\isacharparenright}{\kern0pt}{\isacharparenright}{\kern0pt}{\isachardoublequoteclose}\isanewline
\ \ \isacommand{define}\isamarkupfalse%
\ g{\isacharprime}{\kern0pt}\ \isakeyword{where}\ {\isachardoublequoteopen}g{\isacharprime}{\kern0pt}\ {\isacharequal}{\kern0pt}\ {\isacharparenleft}{\kern0pt}{\isasymlambda}x{\isachardot}{\kern0pt}\ {\isasymSum}i{\isachardot}{\kern0pt}\ ennreal\ {\isacharparenleft}{\kern0pt}dist\ {\isacharparenleft}{\kern0pt}s\ {\isacharparenleft}{\kern0pt}r\ {\isacharparenleft}{\kern0pt}Suc\ i{\isacharparenright}{\kern0pt}{\isacharparenright}{\kern0pt}\ x{\isacharparenright}{\kern0pt}\ {\isacharparenleft}{\kern0pt}s\ {\isacharparenleft}{\kern0pt}r\ i{\isacharparenright}{\kern0pt}\ x{\isacharparenright}{\kern0pt}{\isacharparenright}{\kern0pt}{\isacharparenright}{\kern0pt}{\isachardoublequoteclose}\isanewline
\isanewline
\ \ \isacommand{have}\isamarkupfalse%
\ integrable{\isacharunderscore}{\kern0pt}g{\isacharcolon}{\kern0pt}\ {\isachardoublequoteopen}{\isacharparenleft}{\kern0pt}{\isasymintegral}\isactrlsup {\isacharplus}{\kern0pt}\ x{\isachardot}{\kern0pt}\ g\ n\ x\ {\isasympartial}M{\isacharparenright}{\kern0pt}\ {\isacharless}{\kern0pt}\ {\isadigit{2}}{\isachardoublequoteclose}\ \isakeyword{for}\ n\isanewline
\ \ \isacommand{proof}\isamarkupfalse%
\ {\isacharminus}{\kern0pt}\isanewline
\ \ \ \ \isacommand{have}\isamarkupfalse%
\ {\isachardoublequoteopen}{\isacharparenleft}{\kern0pt}{\isasymintegral}\isactrlsup {\isacharplus}{\kern0pt}\ x{\isachardot}{\kern0pt}\ g\ n\ x\ {\isasympartial}M{\isacharparenright}{\kern0pt}\ {\isacharequal}{\kern0pt}\ {\isacharparenleft}{\kern0pt}{\isasymintegral}\isactrlsup {\isacharplus}{\kern0pt}\ x{\isachardot}{\kern0pt}\ {\isacharparenleft}{\kern0pt}{\isasymSum}i{\isasymle}n{\isachardot}{\kern0pt}\ ennreal\ {\isacharparenleft}{\kern0pt}dist\ {\isacharparenleft}{\kern0pt}s\ {\isacharparenleft}{\kern0pt}r\ {\isacharparenleft}{\kern0pt}Suc\ i{\isacharparenright}{\kern0pt}{\isacharparenright}{\kern0pt}\ x{\isacharparenright}{\kern0pt}\ {\isacharparenleft}{\kern0pt}s\ {\isacharparenleft}{\kern0pt}r\ i{\isacharparenright}{\kern0pt}\ x{\isacharparenright}{\kern0pt}{\isacharparenright}{\kern0pt}{\isacharparenright}{\kern0pt}\ {\isasympartial}M{\isacharparenright}{\kern0pt}{\isachardoublequoteclose}\ \isacommand{using}\isamarkupfalse%
\ g{\isacharunderscore}{\kern0pt}def\ \isacommand{by}\isamarkupfalse%
\ simp\isanewline
\ \ \ \ \isacommand{also}\isamarkupfalse%
\ \isacommand{have}\isamarkupfalse%
\ {\isachardoublequoteopen}{\isachardot}{\kern0pt}{\isachardot}{\kern0pt}{\isachardot}{\kern0pt}\ {\isacharequal}{\kern0pt}\ {\isacharparenleft}{\kern0pt}{\isasymSum}i{\isasymle}n{\isachardot}{\kern0pt}\ {\isacharparenleft}{\kern0pt}{\isasymintegral}\isactrlsup {\isacharplus}{\kern0pt}\ x{\isachardot}{\kern0pt}\ ennreal\ {\isacharparenleft}{\kern0pt}dist\ {\isacharparenleft}{\kern0pt}s\ {\isacharparenleft}{\kern0pt}r\ {\isacharparenleft}{\kern0pt}Suc\ i{\isacharparenright}{\kern0pt}{\isacharparenright}{\kern0pt}\ x{\isacharparenright}{\kern0pt}\ {\isacharparenleft}{\kern0pt}s\ {\isacharparenleft}{\kern0pt}r\ i{\isacharparenright}{\kern0pt}\ x{\isacharparenright}{\kern0pt}{\isacharparenright}{\kern0pt}\ {\isasympartial}M{\isacharparenright}{\kern0pt}{\isacharparenright}{\kern0pt}{\isachardoublequoteclose}\ \isacommand{by}\isamarkupfalse%
\ {\isacharparenleft}{\kern0pt}intro\ nn{\isacharunderscore}{\kern0pt}integral{\isacharunderscore}{\kern0pt}sum{\isacharcomma}{\kern0pt}\ simp{\isacharparenright}{\kern0pt}\isanewline
\ \ \ \ \isacommand{also}\isamarkupfalse%
\ \isacommand{have}\isamarkupfalse%
\ {\isachardoublequoteopen}{\isachardot}{\kern0pt}{\isachardot}{\kern0pt}{\isachardot}{\kern0pt}\ {\isacharequal}{\kern0pt}\ {\isacharparenleft}{\kern0pt}{\isasymSum}i{\isasymle}n{\isachardot}{\kern0pt}\ LINT\ x{\isacharbar}{\kern0pt}M{\isachardot}{\kern0pt}\ dist\ {\isacharparenleft}{\kern0pt}s\ {\isacharparenleft}{\kern0pt}r\ {\isacharparenleft}{\kern0pt}Suc\ i{\isacharparenright}{\kern0pt}{\isacharparenright}{\kern0pt}\ x{\isacharparenright}{\kern0pt}\ {\isacharparenleft}{\kern0pt}s\ {\isacharparenleft}{\kern0pt}r\ i{\isacharparenright}{\kern0pt}\ x{\isacharparenright}{\kern0pt}{\isacharparenright}{\kern0pt}{\isachardoublequoteclose}\ \isacommand{unfolding}\isamarkupfalse%
\ dist{\isacharunderscore}{\kern0pt}norm\ \isacommand{using}\isamarkupfalse%
\ assms{\isacharparenleft}{\kern0pt}{\isadigit{1}}{\isacharparenright}{\kern0pt}\ \isacommand{by}\isamarkupfalse%
\ {\isacharparenleft}{\kern0pt}subst\ nn{\isacharunderscore}{\kern0pt}integral{\isacharunderscore}{\kern0pt}eq{\isacharunderscore}{\kern0pt}integral{\isacharbrackleft}{\kern0pt}OF\ integrable{\isacharunderscore}{\kern0pt}norm{\isacharbrackright}{\kern0pt}{\isacharcomma}{\kern0pt}\ auto{\isacharparenright}{\kern0pt}\isanewline
\ \ \ \ \isacommand{also}\isamarkupfalse%
\ \isacommand{have}\isamarkupfalse%
\ {\isachardoublequoteopen}{\isachardot}{\kern0pt}{\isachardot}{\kern0pt}{\isachardot}{\kern0pt}\ {\isacharless}{\kern0pt}\ ennreal\ {\isacharparenleft}{\kern0pt}{\isasymSum}i{\isasymle}n{\isachardot}{\kern0pt}\ {\isacharparenleft}{\kern0pt}{\isadigit{1}}\ {\isacharslash}{\kern0pt}\ {\isadigit{2}}{\isacharparenright}{\kern0pt}\ {\isacharcircum}{\kern0pt}\ i{\isacharparenright}{\kern0pt}{\isachardoublequoteclose}\ \isacommand{by}\isamarkupfalse%
\ {\isacharparenleft}{\kern0pt}intro\ ennreal{\isacharunderscore}{\kern0pt}lessI{\isacharbrackleft}{\kern0pt}OF\ sum{\isacharunderscore}{\kern0pt}pos\ sum{\isacharunderscore}{\kern0pt}strict{\isacharunderscore}{\kern0pt}mono{\isacharbrackleft}{\kern0pt}OF\ finite{\isacharunderscore}{\kern0pt}atMost\ {\isacharunderscore}{\kern0pt}\ r{\isacharunderscore}{\kern0pt}is{\isacharbrackright}{\kern0pt}{\isacharbrackright}{\kern0pt}{\isacharcomma}{\kern0pt}\ auto{\isacharparenright}{\kern0pt}\isanewline
\ \ \ \ \isacommand{also}\isamarkupfalse%
\ \isacommand{have}\isamarkupfalse%
\ {\isachardoublequoteopen}{\isachardot}{\kern0pt}{\isachardot}{\kern0pt}{\isachardot}{\kern0pt}\ {\isasymle}\ ennreal\ {\isadigit{2}}{\isachardoublequoteclose}\ \isacommand{unfolding}\isamarkupfalse%
\ sum{\isacharunderscore}{\kern0pt}gp{\isadigit{0}}{\isacharbrackleft}{\kern0pt}of\ {\isachardoublequoteopen}{\isadigit{1}}\ {\isacharslash}{\kern0pt}\ {\isadigit{2}}{\isachardoublequoteclose}\ n{\isacharbrackright}{\kern0pt}\ \isacommand{by}\isamarkupfalse%
\ {\isacharparenleft}{\kern0pt}intro\ ennreal{\isacharunderscore}{\kern0pt}leI{\isacharcomma}{\kern0pt}\ auto{\isacharparenright}{\kern0pt}\isanewline
\ \ \ \ \isacommand{finally}\isamarkupfalse%
\ \isacommand{show}\isamarkupfalse%
\ {\isachardoublequoteopen}{\isacharparenleft}{\kern0pt}{\isasymintegral}\isactrlsup {\isacharplus}{\kern0pt}\ x{\isachardot}{\kern0pt}\ g\ n\ x\ {\isasympartial}M{\isacharparenright}{\kern0pt}\ {\isacharless}{\kern0pt}\ {\isadigit{2}}{\isachardoublequoteclose}\ \isacommand{by}\isamarkupfalse%
\ simp\isanewline
\ \ \isacommand{qed}\isamarkupfalse%
\isanewline
\isanewline
\ \ \isacommand{have}\isamarkupfalse%
\ integrable{\isacharunderscore}{\kern0pt}g{\isacharprime}{\kern0pt}{\isacharcolon}{\kern0pt}\ {\isachardoublequoteopen}{\isacharparenleft}{\kern0pt}{\isasymintegral}\isactrlsup {\isacharplus}{\kern0pt}\ x{\isachardot}{\kern0pt}\ g{\isacharprime}{\kern0pt}\ x\ {\isasympartial}M{\isacharparenright}{\kern0pt}\ {\isasymle}\ {\isadigit{2}}{\isachardoublequoteclose}\isanewline
\ \ \isacommand{proof}\isamarkupfalse%
\ {\isacharminus}{\kern0pt}\isanewline
\ \ \ \ \isacommand{have}\isamarkupfalse%
\ {\isachardoublequoteopen}incseq\ {\isacharparenleft}{\kern0pt}{\isasymlambda}n{\isachardot}{\kern0pt}\ g\ n\ x{\isacharparenright}{\kern0pt}{\isachardoublequoteclose}\ \isakeyword{for}\ x\ \isacommand{by}\isamarkupfalse%
\ {\isacharparenleft}{\kern0pt}intro\ incseq{\isacharunderscore}{\kern0pt}SucI{\isacharcomma}{\kern0pt}\ auto\ simp\ add{\isacharcolon}{\kern0pt}\ g{\isacharunderscore}{\kern0pt}def\ ennreal{\isacharunderscore}{\kern0pt}leI{\isacharparenright}{\kern0pt}\isanewline
\ \ \ \ \isacommand{hence}\isamarkupfalse%
\ {\isachardoublequoteopen}convergent\ {\isacharparenleft}{\kern0pt}{\isasymlambda}n{\isachardot}{\kern0pt}\ g\ n\ x{\isacharparenright}{\kern0pt}{\isachardoublequoteclose}\ \isakeyword{for}\ x\ \isacommand{unfolding}\isamarkupfalse%
\ convergent{\isacharunderscore}{\kern0pt}def\ \isacommand{using}\isamarkupfalse%
\ LIMSEQ{\isacharunderscore}{\kern0pt}incseq{\isacharunderscore}{\kern0pt}SUP\ \isacommand{by}\isamarkupfalse%
\ fast\isanewline
\ \ \ \ \isacommand{hence}\isamarkupfalse%
\ {\isachardoublequoteopen}{\isacharparenleft}{\kern0pt}{\isasymlambda}n{\isachardot}{\kern0pt}\ g\ n\ x{\isacharparenright}{\kern0pt}\ {\isasymlonglonglongrightarrow}\ g{\isacharprime}{\kern0pt}\ x{\isachardoublequoteclose}\ \isakeyword{for}\ x\ \isacommand{unfolding}\isamarkupfalse%
\ g{\isacharunderscore}{\kern0pt}def\ g{\isacharprime}{\kern0pt}{\isacharunderscore}{\kern0pt}def\ \isacommand{by}\isamarkupfalse%
\ {\isacharparenleft}{\kern0pt}intro\ summable{\isacharunderscore}{\kern0pt}iff{\isacharunderscore}{\kern0pt}convergent{\isacharprime}{\kern0pt}{\isacharbrackleft}{\kern0pt}THEN\ iffD{\isadigit{2}}{\isacharcomma}{\kern0pt}\ THEN\ summable{\isacharunderscore}{\kern0pt}LIMSEQ{\isacharprime}{\kern0pt}{\isacharbrackright}{\kern0pt}{\isacharcomma}{\kern0pt}\ blast{\isacharparenright}{\kern0pt}\isanewline
\ \ \ \ \isacommand{hence}\isamarkupfalse%
\ {\isachardoublequoteopen}{\isacharparenleft}{\kern0pt}{\isasymintegral}\isactrlsup {\isacharplus}{\kern0pt}\ x{\isachardot}{\kern0pt}\ g{\isacharprime}{\kern0pt}\ x\ {\isasympartial}M{\isacharparenright}{\kern0pt}\ {\isacharequal}{\kern0pt}\ {\isacharparenleft}{\kern0pt}{\isasymintegral}\isactrlsup {\isacharplus}{\kern0pt}\ x{\isachardot}{\kern0pt}\ liminf\ {\isacharparenleft}{\kern0pt}{\isasymlambda}n{\isachardot}{\kern0pt}\ g\ n\ x{\isacharparenright}{\kern0pt}\ {\isasympartial}M{\isacharparenright}{\kern0pt}{\isachardoublequoteclose}\ \isacommand{by}\isamarkupfalse%
\ {\isacharparenleft}{\kern0pt}metis\ lim{\isacharunderscore}{\kern0pt}imp{\isacharunderscore}{\kern0pt}Liminf\ trivial{\isacharunderscore}{\kern0pt}limit{\isacharunderscore}{\kern0pt}sequentially{\isacharparenright}{\kern0pt}\isanewline
\ \ \ \ \isacommand{also}\isamarkupfalse%
\ \isacommand{have}\isamarkupfalse%
\ {\isachardoublequoteopen}{\isachardot}{\kern0pt}{\isachardot}{\kern0pt}{\isachardot}{\kern0pt}\ {\isasymle}\ liminf\ {\isacharparenleft}{\kern0pt}{\isasymlambda}n{\isachardot}{\kern0pt}\ {\isasymintegral}\isactrlsup {\isacharplus}{\kern0pt}\ x{\isachardot}{\kern0pt}\ g\ n\ x\ {\isasympartial}M{\isacharparenright}{\kern0pt}{\isachardoublequoteclose}\ \isacommand{by}\isamarkupfalse%
\ {\isacharparenleft}{\kern0pt}intro\ nn{\isacharunderscore}{\kern0pt}integral{\isacharunderscore}{\kern0pt}liminf{\isacharcomma}{\kern0pt}\ simp\ add{\isacharcolon}{\kern0pt}\ g{\isacharunderscore}{\kern0pt}def{\isacharparenright}{\kern0pt}\isanewline
\ \ \ \ \isacommand{also}\isamarkupfalse%
\ \isacommand{have}\isamarkupfalse%
\ {\isachardoublequoteopen}{\isachardot}{\kern0pt}{\isachardot}{\kern0pt}{\isachardot}{\kern0pt}\ {\isasymle}\ liminf\ {\isacharparenleft}{\kern0pt}{\isasymlambda}n{\isachardot}{\kern0pt}\ {\isadigit{2}}{\isacharparenright}{\kern0pt}{\isachardoublequoteclose}\ \isacommand{using}\isamarkupfalse%
\ integrable{\isacharunderscore}{\kern0pt}g\ \isacommand{by}\isamarkupfalse%
\ {\isacharparenleft}{\kern0pt}intro\ Liminf{\isacharunderscore}{\kern0pt}mono{\isacharparenright}{\kern0pt}\ {\isacharparenleft}{\kern0pt}simp\ add{\isacharcolon}{\kern0pt}\ order{\isacharunderscore}{\kern0pt}le{\isacharunderscore}{\kern0pt}less{\isacharparenright}{\kern0pt}\isanewline
\ \ \ \ \isacommand{also}\isamarkupfalse%
\ \isacommand{have}\isamarkupfalse%
\ {\isachardoublequoteopen}{\isachardot}{\kern0pt}{\isachardot}{\kern0pt}{\isachardot}{\kern0pt}\ {\isacharequal}{\kern0pt}\ {\isadigit{2}}{\isachardoublequoteclose}\ \isacommand{using}\isamarkupfalse%
\ sequentially{\isacharunderscore}{\kern0pt}bot\ tendsto{\isacharunderscore}{\kern0pt}iff{\isacharunderscore}{\kern0pt}Liminf{\isacharunderscore}{\kern0pt}eq{\isacharunderscore}{\kern0pt}Limsup\ \isacommand{by}\isamarkupfalse%
\ blast\isanewline
\ \ \ \ \isacommand{finally}\isamarkupfalse%
\ \isacommand{show}\isamarkupfalse%
\ {\isacharquery}{\kern0pt}thesis\ \isacommand{{\isachardot}{\kern0pt}}\isamarkupfalse%
\isanewline
\ \ \isacommand{qed}\isamarkupfalse%
\isanewline
\ \ \isacommand{hence}\isamarkupfalse%
\ {\isachardoublequoteopen}AE\ x\ in\ M{\isachardot}{\kern0pt}\ g{\isacharprime}{\kern0pt}\ x\ {\isacharless}{\kern0pt}\ {\isasyminfinity}{\isachardoublequoteclose}\ \isacommand{by}\isamarkupfalse%
\ {\isacharparenleft}{\kern0pt}intro\ finite{\isacharunderscore}{\kern0pt}nn{\isacharunderscore}{\kern0pt}integral{\isacharunderscore}{\kern0pt}imp{\isacharunderscore}{\kern0pt}ae{\isacharunderscore}{\kern0pt}finite{\isacharparenright}{\kern0pt}\ {\isacharparenleft}{\kern0pt}auto\ simp\ add{\isacharcolon}{\kern0pt}\ order{\isacharunderscore}{\kern0pt}le{\isacharunderscore}{\kern0pt}less{\isacharunderscore}{\kern0pt}trans\ g{\isacharprime}{\kern0pt}{\isacharunderscore}{\kern0pt}def{\isacharparenright}{\kern0pt}\isanewline
\ \ \isacommand{moreover}\isamarkupfalse%
\ \isacommand{have}\isamarkupfalse%
\ {\isachardoublequoteopen}summable\ {\isacharparenleft}{\kern0pt}{\isasymlambda}i{\isachardot}{\kern0pt}\ dist\ {\isacharparenleft}{\kern0pt}s\ {\isacharparenleft}{\kern0pt}r\ {\isacharparenleft}{\kern0pt}Suc\ i{\isacharparenright}{\kern0pt}{\isacharparenright}{\kern0pt}\ x{\isacharparenright}{\kern0pt}\ {\isacharparenleft}{\kern0pt}s\ {\isacharparenleft}{\kern0pt}r\ i{\isacharparenright}{\kern0pt}\ x{\isacharparenright}{\kern0pt}{\isacharparenright}{\kern0pt}{\isachardoublequoteclose}\ \isakeyword{if}\ {\isachardoublequoteopen}g{\isacharprime}{\kern0pt}\ x\ {\isasymnoteq}\ {\isasyminfinity}{\isachardoublequoteclose}\ \isakeyword{for}\ x\ \isacommand{using}\isamarkupfalse%
\ that\ \isacommand{unfolding}\isamarkupfalse%
\ g{\isacharprime}{\kern0pt}{\isacharunderscore}{\kern0pt}def\ \isacommand{by}\isamarkupfalse%
\ {\isacharparenleft}{\kern0pt}intro\ summable{\isacharunderscore}{\kern0pt}suminf{\isacharunderscore}{\kern0pt}not{\isacharunderscore}{\kern0pt}top{\isacharcomma}{\kern0pt}\ intro\ zero{\isacharunderscore}{\kern0pt}le{\isacharunderscore}{\kern0pt}dist{\isacharcomma}{\kern0pt}\ fastforce{\isacharparenright}{\kern0pt}\ \isanewline
\ \ \isacommand{ultimately}\isamarkupfalse%
\ \isacommand{have}\isamarkupfalse%
\ ae{\isacharunderscore}{\kern0pt}summable{\isacharcolon}{\kern0pt}\ {\isachardoublequoteopen}AE\ x\ in\ M{\isachardot}{\kern0pt}\ summable\ {\isacharparenleft}{\kern0pt}{\isasymlambda}i{\isachardot}{\kern0pt}\ s\ {\isacharparenleft}{\kern0pt}r\ {\isacharparenleft}{\kern0pt}Suc\ i{\isacharparenright}{\kern0pt}{\isacharparenright}{\kern0pt}\ x\ {\isacharminus}{\kern0pt}\ s\ {\isacharparenleft}{\kern0pt}r\ i{\isacharparenright}{\kern0pt}\ x{\isacharparenright}{\kern0pt}{\isachardoublequoteclose}\ \isacommand{using}\isamarkupfalse%
\ summable{\isacharunderscore}{\kern0pt}norm{\isacharunderscore}{\kern0pt}cancel\ \isacommand{unfolding}\isamarkupfalse%
\ dist{\isacharunderscore}{\kern0pt}norm\ \isacommand{by}\isamarkupfalse%
\ force\isanewline
\isanewline
\ \ \isacommand{{\isacharbraceleft}{\kern0pt}}\isamarkupfalse%
\isanewline
\ \ \ \ \isacommand{fix}\isamarkupfalse%
\ x\ \isacommand{assume}\isamarkupfalse%
\ {\isachardoublequoteopen}summable\ {\isacharparenleft}{\kern0pt}{\isasymlambda}i{\isachardot}{\kern0pt}\ s\ {\isacharparenleft}{\kern0pt}r\ {\isacharparenleft}{\kern0pt}Suc\ i{\isacharparenright}{\kern0pt}{\isacharparenright}{\kern0pt}\ x\ {\isacharminus}{\kern0pt}\ s\ {\isacharparenleft}{\kern0pt}r\ i{\isacharparenright}{\kern0pt}\ x{\isacharparenright}{\kern0pt}{\isachardoublequoteclose}\isanewline
\ \ \ \ \isacommand{hence}\isamarkupfalse%
\ {\isachardoublequoteopen}{\isacharparenleft}{\kern0pt}{\isasymlambda}n{\isachardot}{\kern0pt}\ {\isasymSum}i{\isacharless}{\kern0pt}n{\isachardot}{\kern0pt}\ s\ {\isacharparenleft}{\kern0pt}r\ {\isacharparenleft}{\kern0pt}Suc\ i{\isacharparenright}{\kern0pt}{\isacharparenright}{\kern0pt}\ x\ {\isacharminus}{\kern0pt}\ s\ {\isacharparenleft}{\kern0pt}r\ i{\isacharparenright}{\kern0pt}\ x{\isacharparenright}{\kern0pt}\ {\isasymlonglonglongrightarrow}\ {\isacharparenleft}{\kern0pt}{\isasymSum}i{\isachardot}{\kern0pt}\ s\ {\isacharparenleft}{\kern0pt}r\ {\isacharparenleft}{\kern0pt}Suc\ i{\isacharparenright}{\kern0pt}{\isacharparenright}{\kern0pt}\ x\ {\isacharminus}{\kern0pt}\ s\ {\isacharparenleft}{\kern0pt}r\ i{\isacharparenright}{\kern0pt}\ x{\isacharparenright}{\kern0pt}{\isachardoublequoteclose}\ \isacommand{using}\isamarkupfalse%
\ summable{\isacharunderscore}{\kern0pt}LIMSEQ\ \isacommand{by}\isamarkupfalse%
\ blast\isanewline
\ \ \ \ \isacommand{moreover}\isamarkupfalse%
\ \isacommand{have}\isamarkupfalse%
\ {\isachardoublequoteopen}{\isacharparenleft}{\kern0pt}{\isasymlambda}n{\isachardot}{\kern0pt}\ {\isacharparenleft}{\kern0pt}{\isasymSum}i{\isacharless}{\kern0pt}n{\isachardot}{\kern0pt}\ s\ {\isacharparenleft}{\kern0pt}r\ {\isacharparenleft}{\kern0pt}Suc\ i{\isacharparenright}{\kern0pt}{\isacharparenright}{\kern0pt}\ x\ {\isacharminus}{\kern0pt}\ s\ {\isacharparenleft}{\kern0pt}r\ i{\isacharparenright}{\kern0pt}\ x{\isacharparenright}{\kern0pt}{\isacharparenright}{\kern0pt}\ {\isacharequal}{\kern0pt}\ {\isacharparenleft}{\kern0pt}{\isasymlambda}n{\isachardot}{\kern0pt}\ s\ {\isacharparenleft}{\kern0pt}r\ n{\isacharparenright}{\kern0pt}\ x\ {\isacharminus}{\kern0pt}\ s\ {\isacharparenleft}{\kern0pt}r\ {\isadigit{0}}{\isacharparenright}{\kern0pt}\ x{\isacharparenright}{\kern0pt}{\isachardoublequoteclose}\ \isacommand{using}\isamarkupfalse%
\ sum{\isacharunderscore}{\kern0pt}lessThan{\isacharunderscore}{\kern0pt}telescope\ \isacommand{by}\isamarkupfalse%
\ fastforce\isanewline
\ \ \ \ \isacommand{ultimately}\isamarkupfalse%
\ \isacommand{have}\isamarkupfalse%
\ {\isachardoublequoteopen}{\isacharparenleft}{\kern0pt}{\isasymlambda}n{\isachardot}{\kern0pt}\ s\ {\isacharparenleft}{\kern0pt}r\ n{\isacharparenright}{\kern0pt}\ x\ {\isacharminus}{\kern0pt}\ s\ {\isacharparenleft}{\kern0pt}r\ {\isadigit{0}}{\isacharparenright}{\kern0pt}\ x{\isacharparenright}{\kern0pt}\ {\isasymlonglonglongrightarrow}\ {\isacharparenleft}{\kern0pt}{\isasymSum}i{\isachardot}{\kern0pt}\ s\ {\isacharparenleft}{\kern0pt}r\ {\isacharparenleft}{\kern0pt}Suc\ i{\isacharparenright}{\kern0pt}{\isacharparenright}{\kern0pt}\ x\ {\isacharminus}{\kern0pt}\ s\ {\isacharparenleft}{\kern0pt}r\ i{\isacharparenright}{\kern0pt}\ x{\isacharparenright}{\kern0pt}{\isachardoublequoteclose}\ \isacommand{by}\isamarkupfalse%
\ argo\isanewline
\ \ \ \ \isacommand{hence}\isamarkupfalse%
\ {\isachardoublequoteopen}{\isacharparenleft}{\kern0pt}{\isasymlambda}n{\isachardot}{\kern0pt}\ s\ {\isacharparenleft}{\kern0pt}r\ n{\isacharparenright}{\kern0pt}\ x\ {\isacharminus}{\kern0pt}\ s\ {\isacharparenleft}{\kern0pt}r\ {\isadigit{0}}{\isacharparenright}{\kern0pt}\ x\ {\isacharplus}{\kern0pt}\ s\ {\isacharparenleft}{\kern0pt}r\ {\isadigit{0}}{\isacharparenright}{\kern0pt}\ x{\isacharparenright}{\kern0pt}\ {\isasymlonglonglongrightarrow}\ {\isacharparenleft}{\kern0pt}{\isasymSum}i{\isachardot}{\kern0pt}\ s\ {\isacharparenleft}{\kern0pt}r\ {\isacharparenleft}{\kern0pt}Suc\ i{\isacharparenright}{\kern0pt}{\isacharparenright}{\kern0pt}\ x\ {\isacharminus}{\kern0pt}\ s\ {\isacharparenleft}{\kern0pt}r\ i{\isacharparenright}{\kern0pt}\ x{\isacharparenright}{\kern0pt}\ {\isacharplus}{\kern0pt}\ s\ {\isacharparenleft}{\kern0pt}r\ {\isadigit{0}}{\isacharparenright}{\kern0pt}\ x{\isachardoublequoteclose}\ \isacommand{by}\isamarkupfalse%
\ {\isacharparenleft}{\kern0pt}intro\ isCont{\isacharunderscore}{\kern0pt}tendsto{\isacharunderscore}{\kern0pt}compose{\isacharbrackleft}{\kern0pt}of\ {\isacharunderscore}{\kern0pt}\ {\isachardoublequoteopen}{\isasymlambda}z{\isachardot}{\kern0pt}\ z\ {\isacharplus}{\kern0pt}\ s\ {\isacharparenleft}{\kern0pt}r\ {\isadigit{0}}{\isacharparenright}{\kern0pt}\ x{\isachardoublequoteclose}{\isacharbrackright}{\kern0pt}{\isacharcomma}{\kern0pt}\ auto{\isacharparenright}{\kern0pt}\isanewline
\ \ \ \ \isacommand{hence}\isamarkupfalse%
\ {\isachardoublequoteopen}Cauchy\ {\isacharparenleft}{\kern0pt}{\isasymlambda}n{\isachardot}{\kern0pt}\ s\ {\isacharparenleft}{\kern0pt}r\ n{\isacharparenright}{\kern0pt}\ x{\isacharparenright}{\kern0pt}{\isachardoublequoteclose}\ \isacommand{by}\isamarkupfalse%
\ {\isacharparenleft}{\kern0pt}simp\ add{\isacharcolon}{\kern0pt}\ LIMSEQ{\isacharunderscore}{\kern0pt}imp{\isacharunderscore}{\kern0pt}Cauchy{\isacharparenright}{\kern0pt}\isanewline
\ \ \isacommand{{\isacharbraceright}{\kern0pt}}\isamarkupfalse%
\isanewline
\isanewline
\ \ \isacommand{hence}\isamarkupfalse%
\ {\isachardoublequoteopen}AE\ x\ in\ M{\isachardot}{\kern0pt}\ Cauchy\ {\isacharparenleft}{\kern0pt}{\isasymlambda}i{\isachardot}{\kern0pt}\ s\ {\isacharparenleft}{\kern0pt}r\ i{\isacharparenright}{\kern0pt}\ x{\isacharparenright}{\kern0pt}{\isachardoublequoteclose}\ \isacommand{using}\isamarkupfalse%
\ ae{\isacharunderscore}{\kern0pt}summable\ \isacommand{by}\isamarkupfalse%
\ fast\isanewline
\ \ \isacommand{thus}\isamarkupfalse%
\ {\isacharquery}{\kern0pt}thesis\ \isacommand{by}\isamarkupfalse%
\ {\isacharparenleft}{\kern0pt}rule\ that{\isacharbrackleft}{\kern0pt}OF\ strict{\isacharunderscore}{\kern0pt}mono{\isacharparenleft}{\kern0pt}{\isadigit{1}}{\isacharparenright}{\kern0pt}{\isacharbrackright}{\kern0pt}{\isacharparenright}{\kern0pt}\isanewline
\isacommand{qed}\isamarkupfalse%
%
\endisatagproof
{\isafoldproof}%
%
\isadelimproof
\isanewline
%
\endisadelimproof
\isanewline
\isacommand{lemma}\isamarkupfalse%
\ integrable{\isacharunderscore}{\kern0pt}max{\isacharbrackleft}{\kern0pt}simp{\isacharcomma}{\kern0pt}\ intro{\isacharbrackright}{\kern0pt}{\isacharcolon}{\kern0pt}\isanewline
\ \ \isakeyword{fixes}\ f\ {\isacharcolon}{\kern0pt}{\isacharcolon}{\kern0pt}\ {\isachardoublequoteopen}{\isacharprime}{\kern0pt}a\ {\isasymRightarrow}\ {\isacharprime}{\kern0pt}b\ {\isacharcolon}{\kern0pt}{\isacharcolon}{\kern0pt}\ {\isacharbraceleft}{\kern0pt}second{\isacharunderscore}{\kern0pt}countable{\isacharunderscore}{\kern0pt}topology{\isacharcomma}{\kern0pt}\ banach{\isacharcomma}{\kern0pt}\ linorder{\isacharunderscore}{\kern0pt}topology{\isacharbraceright}{\kern0pt}{\isachardoublequoteclose}\isanewline
\ \ \isakeyword{assumes}\ fg{\isacharbrackleft}{\kern0pt}measurable{\isacharbrackright}{\kern0pt}{\isacharcolon}{\kern0pt}\ {\isachardoublequoteopen}integrable\ M\ f{\isachardoublequoteclose}\ {\isachardoublequoteopen}integrable\ M\ g{\isachardoublequoteclose}\isanewline
\ \ \isakeyword{shows}\ {\isachardoublequoteopen}integrable\ M\ {\isacharparenleft}{\kern0pt}{\isasymlambda}x{\isachardot}{\kern0pt}\ max\ {\isacharparenleft}{\kern0pt}f\ x{\isacharparenright}{\kern0pt}\ {\isacharparenleft}{\kern0pt}g\ x{\isacharparenright}{\kern0pt}{\isacharparenright}{\kern0pt}{\isachardoublequoteclose}\isanewline
%
\isadelimproof
%
\endisadelimproof
%
\isatagproof
\isacommand{proof}\isamarkupfalse%
\ {\isacharparenleft}{\kern0pt}rule\ Bochner{\isacharunderscore}{\kern0pt}Integration{\isachardot}{\kern0pt}integrable{\isacharunderscore}{\kern0pt}bound{\isacharparenright}{\kern0pt}\isanewline
\ \ \isacommand{{\isacharbraceleft}{\kern0pt}}\isamarkupfalse%
\isanewline
\ \ \ \ \isacommand{fix}\isamarkupfalse%
\ x\ y\ {\isacharcolon}{\kern0pt}{\isacharcolon}{\kern0pt}\ {\isacharprime}{\kern0pt}b\ \ \ \ \ \ \ \ \ \ \ \ \ \ \ \ \ \ \ \ \ \ \ \ \ \ \ \ \ \ \ \ \ \ \ \ \ \ \ \ \ \ \ \ \ \isanewline
\ \ \ \ \isacommand{have}\isamarkupfalse%
\ {\isachardoublequoteopen}norm\ {\isacharparenleft}{\kern0pt}max\ x\ y{\isacharparenright}{\kern0pt}\ {\isasymle}\ max\ {\isacharparenleft}{\kern0pt}norm\ x{\isacharparenright}{\kern0pt}\ {\isacharparenleft}{\kern0pt}norm\ y{\isacharparenright}{\kern0pt}{\isachardoublequoteclose}\ \isacommand{by}\isamarkupfalse%
\ linarith\isanewline
\ \ \ \ \isacommand{also}\isamarkupfalse%
\ \isacommand{have}\isamarkupfalse%
\ {\isachardoublequoteopen}{\isachardot}{\kern0pt}{\isachardot}{\kern0pt}{\isachardot}{\kern0pt}\ {\isasymle}\ norm\ x\ {\isacharplus}{\kern0pt}\ norm\ y{\isachardoublequoteclose}\ \isacommand{by}\isamarkupfalse%
\ simp\isanewline
\ \ \ \ \isacommand{finally}\isamarkupfalse%
\ \isacommand{have}\isamarkupfalse%
\ {\isachardoublequoteopen}norm\ {\isacharparenleft}{\kern0pt}max\ x\ y{\isacharparenright}{\kern0pt}\ {\isasymle}\ norm\ {\isacharparenleft}{\kern0pt}norm\ x\ {\isacharplus}{\kern0pt}\ norm\ y{\isacharparenright}{\kern0pt}{\isachardoublequoteclose}\ \isacommand{by}\isamarkupfalse%
\ simp\isanewline
\ \ \isacommand{{\isacharbraceright}{\kern0pt}}\isamarkupfalse%
\isanewline
\ \ \isacommand{thus}\isamarkupfalse%
\ {\isachardoublequoteopen}AE\ x\ in\ M{\isachardot}{\kern0pt}\ norm\ {\isacharparenleft}{\kern0pt}max\ {\isacharparenleft}{\kern0pt}f\ x{\isacharparenright}{\kern0pt}\ {\isacharparenleft}{\kern0pt}g\ x{\isacharparenright}{\kern0pt}{\isacharparenright}{\kern0pt}\ {\isasymle}\ norm\ {\isacharparenleft}{\kern0pt}norm\ {\isacharparenleft}{\kern0pt}f\ x{\isacharparenright}{\kern0pt}\ {\isacharplus}{\kern0pt}\ norm\ {\isacharparenleft}{\kern0pt}g\ x{\isacharparenright}{\kern0pt}{\isacharparenright}{\kern0pt}{\isachardoublequoteclose}\ \isacommand{by}\isamarkupfalse%
\ simp\isanewline
\isacommand{qed}\isamarkupfalse%
\ {\isacharparenleft}{\kern0pt}auto\ intro{\isacharcolon}{\kern0pt}\ Bochner{\isacharunderscore}{\kern0pt}Integration{\isachardot}{\kern0pt}integrable{\isacharunderscore}{\kern0pt}add{\isacharbrackleft}{\kern0pt}OF\ integrable{\isacharunderscore}{\kern0pt}norm{\isacharbrackleft}{\kern0pt}OF\ fg{\isacharparenleft}{\kern0pt}{\isadigit{1}}{\isacharparenright}{\kern0pt}{\isacharbrackright}{\kern0pt}\ integrable{\isacharunderscore}{\kern0pt}norm{\isacharbrackleft}{\kern0pt}OF\ fg{\isacharparenleft}{\kern0pt}{\isadigit{2}}{\isacharparenright}{\kern0pt}{\isacharbrackright}{\kern0pt}{\isacharbrackright}{\kern0pt}{\isacharparenright}{\kern0pt}%
\endisatagproof
{\isafoldproof}%
%
\isadelimproof
\isanewline
%
\endisadelimproof
\isanewline
\isacommand{lemma}\isamarkupfalse%
\ integrable{\isacharunderscore}{\kern0pt}min{\isacharbrackleft}{\kern0pt}simp{\isacharcomma}{\kern0pt}\ intro{\isacharbrackright}{\kern0pt}{\isacharcolon}{\kern0pt}\isanewline
\ \ \isakeyword{fixes}\ f\ {\isacharcolon}{\kern0pt}{\isacharcolon}{\kern0pt}\ {\isachardoublequoteopen}{\isacharprime}{\kern0pt}a\ {\isasymRightarrow}\ {\isacharprime}{\kern0pt}b\ {\isacharcolon}{\kern0pt}{\isacharcolon}{\kern0pt}\ {\isacharbraceleft}{\kern0pt}second{\isacharunderscore}{\kern0pt}countable{\isacharunderscore}{\kern0pt}topology{\isacharcomma}{\kern0pt}\ banach{\isacharcomma}{\kern0pt}\ linorder{\isacharunderscore}{\kern0pt}topology{\isacharbraceright}{\kern0pt}{\isachardoublequoteclose}\isanewline
\ \ \isakeyword{assumes}\ {\isacharbrackleft}{\kern0pt}measurable{\isacharbrackright}{\kern0pt}{\isacharcolon}{\kern0pt}\ {\isachardoublequoteopen}integrable\ M\ f{\isachardoublequoteclose}\ {\isachardoublequoteopen}integrable\ M\ g{\isachardoublequoteclose}\isanewline
\ \ \isakeyword{shows}\ {\isachardoublequoteopen}integrable\ M\ {\isacharparenleft}{\kern0pt}{\isasymlambda}x{\isachardot}{\kern0pt}\ min\ {\isacharparenleft}{\kern0pt}f\ x{\isacharparenright}{\kern0pt}\ {\isacharparenleft}{\kern0pt}g\ x{\isacharparenright}{\kern0pt}{\isacharparenright}{\kern0pt}{\isachardoublequoteclose}\isanewline
%
\isadelimproof
%
\endisadelimproof
%
\isatagproof
\isacommand{proof}\isamarkupfalse%
\ {\isacharminus}{\kern0pt}\isanewline
\ \ \isacommand{have}\isamarkupfalse%
\ {\isachardoublequoteopen}norm\ {\isacharparenleft}{\kern0pt}min\ {\isacharparenleft}{\kern0pt}f\ x{\isacharparenright}{\kern0pt}\ {\isacharparenleft}{\kern0pt}g\ x{\isacharparenright}{\kern0pt}{\isacharparenright}{\kern0pt}\ {\isasymle}\ norm\ {\isacharparenleft}{\kern0pt}f\ x{\isacharparenright}{\kern0pt}\ {\isasymor}\ norm\ {\isacharparenleft}{\kern0pt}min\ {\isacharparenleft}{\kern0pt}f\ x{\isacharparenright}{\kern0pt}\ {\isacharparenleft}{\kern0pt}g\ x{\isacharparenright}{\kern0pt}{\isacharparenright}{\kern0pt}\ {\isasymle}\ norm\ {\isacharparenleft}{\kern0pt}g\ x{\isacharparenright}{\kern0pt}{\isachardoublequoteclose}\ \isakeyword{for}\ x\ \isacommand{by}\isamarkupfalse%
\ linarith\isanewline
\ \ \isacommand{thus}\isamarkupfalse%
\ {\isacharquery}{\kern0pt}thesis\ \isacommand{by}\isamarkupfalse%
\ {\isacharparenleft}{\kern0pt}intro\ integrable{\isacharunderscore}{\kern0pt}bound{\isacharbrackleft}{\kern0pt}OF\ integrable{\isacharunderscore}{\kern0pt}max{\isacharbrackleft}{\kern0pt}OF\ integrable{\isacharunderscore}{\kern0pt}norm{\isacharparenleft}{\kern0pt}{\isadigit{1}}{\isacharcomma}{\kern0pt}{\isadigit{1}}{\isacharparenright}{\kern0pt}{\isacharcomma}{\kern0pt}\ OF\ assms{\isacharbrackright}{\kern0pt}{\isacharbrackright}{\kern0pt}{\isacharcomma}{\kern0pt}\ auto{\isacharparenright}{\kern0pt}\isanewline
\isacommand{qed}\isamarkupfalse%
%
\endisatagproof
{\isafoldproof}%
%
\isadelimproof
\isanewline
%
\endisadelimproof
\isanewline
\isanewline
\isanewline
\isacommand{lemma}\isamarkupfalse%
\ integral{\isacharunderscore}{\kern0pt}nonneg{\isacharunderscore}{\kern0pt}AE{\isacharunderscore}{\kern0pt}banach{\isacharcolon}{\kern0pt}\ \ \ \ \ \ \ \ \ \ \ \ \ \ \ \ \ \ \ \ \ \ \ \ \isanewline
\ \ \isakeyword{fixes}\ f\ {\isacharcolon}{\kern0pt}{\isacharcolon}{\kern0pt}\ {\isachardoublequoteopen}{\isacharprime}{\kern0pt}a\ {\isasymRightarrow}\ {\isacharprime}{\kern0pt}b\ {\isacharcolon}{\kern0pt}{\isacharcolon}{\kern0pt}\ {\isacharbraceleft}{\kern0pt}second{\isacharunderscore}{\kern0pt}countable{\isacharunderscore}{\kern0pt}topology{\isacharcomma}{\kern0pt}\ banach{\isacharcomma}{\kern0pt}\ linorder{\isacharunderscore}{\kern0pt}topology{\isacharcomma}{\kern0pt}\ ordered{\isacharunderscore}{\kern0pt}real{\isacharunderscore}{\kern0pt}vector{\isacharbraceright}{\kern0pt}{\isachardoublequoteclose}\isanewline
\ \ \isakeyword{assumes}\ {\isacharbrackleft}{\kern0pt}measurable{\isacharbrackright}{\kern0pt}{\isacharcolon}{\kern0pt}\ {\isachardoublequoteopen}f\ {\isasymin}\ borel{\isacharunderscore}{\kern0pt}measurable\ M{\isachardoublequoteclose}\ \isakeyword{and}\ nonneg{\isacharcolon}{\kern0pt}\ {\isachardoublequoteopen}AE\ x\ in\ M{\isachardot}{\kern0pt}\ {\isadigit{0}}\ {\isasymle}\ f\ x{\isachardoublequoteclose}\isanewline
\ \ \isakeyword{shows}\ {\isachardoublequoteopen}{\isadigit{0}}\ {\isasymle}\ integral\isactrlsup L\ M\ f{\isachardoublequoteclose}\isanewline
%
\isadelimproof
%
\endisadelimproof
%
\isatagproof
\isacommand{proof}\isamarkupfalse%
\ cases\isanewline
\ \ \isacommand{assume}\isamarkupfalse%
\ integrable{\isacharcolon}{\kern0pt}\ {\isachardoublequoteopen}integrable\ M\ f{\isachardoublequoteclose}\isanewline
\ \ \isacommand{hence}\isamarkupfalse%
\ max{\isacharcolon}{\kern0pt}\ {\isachardoublequoteopen}{\isacharparenleft}{\kern0pt}{\isasymlambda}x{\isachardot}{\kern0pt}\ max\ {\isadigit{0}}\ {\isacharparenleft}{\kern0pt}f\ x{\isacharparenright}{\kern0pt}{\isacharparenright}{\kern0pt}\ {\isasymin}\ borel{\isacharunderscore}{\kern0pt}measurable\ M{\isachardoublequoteclose}\ {\isachardoublequoteopen}{\isasymAnd}x{\isachardot}{\kern0pt}\ {\isadigit{0}}\ {\isasymle}\ max\ {\isadigit{0}}\ {\isacharparenleft}{\kern0pt}f\ x{\isacharparenright}{\kern0pt}{\isachardoublequoteclose}\ {\isachardoublequoteopen}integrable\ M\ {\isacharparenleft}{\kern0pt}{\isasymlambda}x{\isachardot}{\kern0pt}\ max\ {\isadigit{0}}\ {\isacharparenleft}{\kern0pt}f\ x{\isacharparenright}{\kern0pt}{\isacharparenright}{\kern0pt}{\isachardoublequoteclose}\ \isacommand{by}\isamarkupfalse%
\ auto\isanewline
\ \ \isacommand{hence}\isamarkupfalse%
\ {\isachardoublequoteopen}{\isadigit{0}}\ {\isasymle}\ integral\isactrlsup L\ M\ {\isacharparenleft}{\kern0pt}{\isasymlambda}x{\isachardot}{\kern0pt}\ max\ {\isadigit{0}}\ {\isacharparenleft}{\kern0pt}f\ x{\isacharparenright}{\kern0pt}{\isacharparenright}{\kern0pt}{\isachardoublequoteclose}\isanewline
\ \ \isacommand{proof}\isamarkupfalse%
\ {\isacharminus}{\kern0pt}\isanewline
\ \ \isacommand{obtain}\isamarkupfalse%
\ s\ \isakeyword{where}\ {\isacharasterisk}{\kern0pt}{\isacharcolon}{\kern0pt}\ {\isachardoublequoteopen}{\isasymAnd}i{\isachardot}{\kern0pt}\ simple{\isacharunderscore}{\kern0pt}function\ M\ {\isacharparenleft}{\kern0pt}s\ i{\isacharparenright}{\kern0pt}{\isachardoublequoteclose}\ \isanewline
\ \ \ \ \ \ \ \ \ \ \ \ \ \ \ \ \ \ \ \ {\isachardoublequoteopen}{\isasymAnd}i{\isachardot}{\kern0pt}\ emeasure\ M\ {\isacharbraceleft}{\kern0pt}y\ {\isasymin}\ space\ M{\isachardot}{\kern0pt}\ s\ i\ y\ {\isasymnoteq}\ {\isadigit{0}}{\isacharbraceright}{\kern0pt}\ {\isasymnoteq}\ {\isasyminfinity}{\isachardoublequoteclose}\ \isanewline
\ \ \ \ \ \ \ \ \ \ \ \ \ \ \ \ \ \ \ \ {\isachardoublequoteopen}{\isasymAnd}x{\isachardot}{\kern0pt}\ x\ {\isasymin}\ space\ M\ {\isasymLongrightarrow}\ {\isacharparenleft}{\kern0pt}{\isasymlambda}i{\isachardot}{\kern0pt}\ s\ i\ x{\isacharparenright}{\kern0pt}\ {\isasymlonglonglongrightarrow}\ f\ x{\isachardoublequoteclose}\ \isanewline
\ \ \ \ \ \ \ \ \ \ \ \ \ \ \ \ \ \ \ \ {\isachardoublequoteopen}{\isasymAnd}x\ i{\isachardot}{\kern0pt}\ x\ {\isasymin}\ space\ M\ {\isasymLongrightarrow}\ norm\ {\isacharparenleft}{\kern0pt}s\ i\ x{\isacharparenright}{\kern0pt}\ {\isasymle}\ {\isadigit{2}}\ {\isacharasterisk}{\kern0pt}\ norm\ {\isacharparenleft}{\kern0pt}f\ x{\isacharparenright}{\kern0pt}{\isachardoublequoteclose}\ \isacommand{using}\isamarkupfalse%
\ integrable{\isacharunderscore}{\kern0pt}implies{\isacharunderscore}{\kern0pt}simple{\isacharunderscore}{\kern0pt}function{\isacharunderscore}{\kern0pt}sequence{\isacharbrackleft}{\kern0pt}OF\ integrable{\isacharbrackright}{\kern0pt}\ \isacommand{by}\isamarkupfalse%
\ blast\isanewline
\ \ \ \ \isacommand{have}\isamarkupfalse%
\ simple{\isacharcolon}{\kern0pt}\ {\isachardoublequoteopen}{\isasymAnd}i{\isachardot}{\kern0pt}\ simple{\isacharunderscore}{\kern0pt}function\ M\ {\isacharparenleft}{\kern0pt}{\isasymlambda}x{\isachardot}{\kern0pt}\ max\ {\isadigit{0}}\ {\isacharparenleft}{\kern0pt}s\ i\ x{\isacharparenright}{\kern0pt}{\isacharparenright}{\kern0pt}{\isachardoublequoteclose}\ \isacommand{using}\isamarkupfalse%
\ {\isacharasterisk}{\kern0pt}\ \isacommand{by}\isamarkupfalse%
\ fast\isanewline
\ \ \ \ \isacommand{have}\isamarkupfalse%
\ {\isachardoublequoteopen}{\isasymAnd}i{\isachardot}{\kern0pt}\ {\isacharbraceleft}{\kern0pt}y\ {\isasymin}\ space\ M{\isachardot}{\kern0pt}\ max\ {\isadigit{0}}\ {\isacharparenleft}{\kern0pt}s\ i\ y{\isacharparenright}{\kern0pt}\ {\isasymnoteq}\ {\isadigit{0}}{\isacharbraceright}{\kern0pt}\ {\isasymsubseteq}\ {\isacharbraceleft}{\kern0pt}y\ {\isasymin}\ space\ M{\isachardot}{\kern0pt}\ s\ i\ y\ {\isasymnoteq}\ {\isadigit{0}}{\isacharbraceright}{\kern0pt}{\isachardoublequoteclose}\ \isacommand{unfolding}\isamarkupfalse%
\ max{\isacharunderscore}{\kern0pt}def\ \isacommand{by}\isamarkupfalse%
\ force\isanewline
\ \ \ \ \isacommand{moreover}\isamarkupfalse%
\ \isacommand{have}\isamarkupfalse%
\ {\isachardoublequoteopen}{\isasymAnd}i{\isachardot}{\kern0pt}\ {\isacharbraceleft}{\kern0pt}y\ {\isasymin}\ space\ M{\isachardot}{\kern0pt}\ s\ i\ y\ {\isasymnoteq}\ {\isadigit{0}}{\isacharbraceright}{\kern0pt}\ {\isasymin}\ sets\ M{\isachardoublequoteclose}\ \isacommand{using}\isamarkupfalse%
\ {\isacharasterisk}{\kern0pt}\ \isacommand{by}\isamarkupfalse%
\ measurable\isanewline
\ \ \ \ \isacommand{ultimately}\isamarkupfalse%
\ \isacommand{have}\isamarkupfalse%
\ {\isachardoublequoteopen}{\isasymAnd}i{\isachardot}{\kern0pt}\ emeasure\ M\ {\isacharbraceleft}{\kern0pt}y\ {\isasymin}\ space\ M{\isachardot}{\kern0pt}\ max\ {\isadigit{0}}\ {\isacharparenleft}{\kern0pt}s\ i\ y{\isacharparenright}{\kern0pt}\ {\isasymnoteq}\ {\isadigit{0}}{\isacharbraceright}{\kern0pt}\ {\isasymle}\ emeasure\ M\ {\isacharbraceleft}{\kern0pt}y\ {\isasymin}\ space\ M{\isachardot}{\kern0pt}\ s\ i\ y\ {\isasymnoteq}\ {\isadigit{0}}{\isacharbraceright}{\kern0pt}{\isachardoublequoteclose}\ \isacommand{by}\isamarkupfalse%
\ {\isacharparenleft}{\kern0pt}rule\ emeasure{\isacharunderscore}{\kern0pt}mono{\isacharparenright}{\kern0pt}\ \isanewline
\ \ \ \ \isacommand{hence}\isamarkupfalse%
\ {\isacharasterisk}{\kern0pt}{\isacharasterisk}{\kern0pt}{\isacharcolon}{\kern0pt}{\isachardoublequoteopen}{\isasymAnd}i{\isachardot}{\kern0pt}\ emeasure\ M\ {\isacharbraceleft}{\kern0pt}y\ {\isasymin}\ space\ M{\isachardot}{\kern0pt}\ max\ {\isadigit{0}}\ {\isacharparenleft}{\kern0pt}s\ i\ y{\isacharparenright}{\kern0pt}\ {\isasymnoteq}\ {\isadigit{0}}{\isacharbraceright}{\kern0pt}\ {\isasymnoteq}\ {\isasyminfinity}{\isachardoublequoteclose}\ \isacommand{using}\isamarkupfalse%
\ {\isacharasterisk}{\kern0pt}{\isacharparenleft}{\kern0pt}{\isadigit{2}}{\isacharparenright}{\kern0pt}\ \isacommand{by}\isamarkupfalse%
\ {\isacharparenleft}{\kern0pt}auto\ intro{\isacharcolon}{\kern0pt}\ order{\isachardot}{\kern0pt}strict{\isacharunderscore}{\kern0pt}trans{\isadigit{1}}\ simp\ add{\isacharcolon}{\kern0pt}\ \ top{\isachardot}{\kern0pt}not{\isacharunderscore}{\kern0pt}eq{\isacharunderscore}{\kern0pt}extremum{\isacharparenright}{\kern0pt}\isanewline
\ \ \ \ \isacommand{have}\isamarkupfalse%
\ {\isachardoublequoteopen}{\isasymAnd}x{\isachardot}{\kern0pt}\ x\ {\isasymin}\ space\ M\ {\isasymLongrightarrow}\ {\isacharparenleft}{\kern0pt}{\isasymlambda}i{\isachardot}{\kern0pt}\ max\ {\isadigit{0}}\ {\isacharparenleft}{\kern0pt}s\ i\ x{\isacharparenright}{\kern0pt}{\isacharparenright}{\kern0pt}\ {\isasymlonglonglongrightarrow}\ max\ {\isadigit{0}}\ {\isacharparenleft}{\kern0pt}f\ x{\isacharparenright}{\kern0pt}{\isachardoublequoteclose}\ \isacommand{using}\isamarkupfalse%
\ {\isacharasterisk}{\kern0pt}{\isacharparenleft}{\kern0pt}{\isadigit{3}}{\isacharparenright}{\kern0pt}\ tendsto{\isacharunderscore}{\kern0pt}max\ \isacommand{by}\isamarkupfalse%
\ blast\isanewline
\ \ \ \ \isacommand{moreover}\isamarkupfalse%
\ \isacommand{have}\isamarkupfalse%
\ {\isachardoublequoteopen}{\isasymAnd}x\ i{\isachardot}{\kern0pt}\ x\ {\isasymin}\ space\ M\ {\isasymLongrightarrow}\ norm\ {\isacharparenleft}{\kern0pt}max\ {\isadigit{0}}\ {\isacharparenleft}{\kern0pt}s\ i\ x{\isacharparenright}{\kern0pt}{\isacharparenright}{\kern0pt}\ {\isasymle}\ norm\ {\isacharparenleft}{\kern0pt}{\isadigit{2}}\ {\isacharasterisk}{\kern0pt}\isactrlsub R\ f\ x{\isacharparenright}{\kern0pt}{\isachardoublequoteclose}\ \isacommand{using}\isamarkupfalse%
\ {\isacharasterisk}{\kern0pt}{\isacharparenleft}{\kern0pt}{\isadigit{4}}{\isacharparenright}{\kern0pt}\ \isacommand{unfolding}\isamarkupfalse%
\ max{\isacharunderscore}{\kern0pt}def\ \isacommand{by}\isamarkupfalse%
\ auto\isanewline
\ \ \ \ \isacommand{ultimately}\isamarkupfalse%
\ \isacommand{have}\isamarkupfalse%
\ tendsto{\isacharcolon}{\kern0pt}\ {\isachardoublequoteopen}{\isacharparenleft}{\kern0pt}{\isasymlambda}i{\isachardot}{\kern0pt}\ integral\isactrlsup L\ M\ {\isacharparenleft}{\kern0pt}{\isasymlambda}x{\isachardot}{\kern0pt}\ max\ {\isadigit{0}}\ {\isacharparenleft}{\kern0pt}s\ i\ x{\isacharparenright}{\kern0pt}{\isacharparenright}{\kern0pt}{\isacharparenright}{\kern0pt}\ {\isasymlonglonglongrightarrow}\ integral\isactrlsup L\ M\ {\isacharparenleft}{\kern0pt}{\isasymlambda}x{\isachardot}{\kern0pt}\ max\ {\isadigit{0}}\ {\isacharparenleft}{\kern0pt}f\ x{\isacharparenright}{\kern0pt}{\isacharparenright}{\kern0pt}{\isachardoublequoteclose}\ \isanewline
\ \ \ \ \ \ \isacommand{using}\isamarkupfalse%
\ borel{\isacharunderscore}{\kern0pt}measurable{\isacharunderscore}{\kern0pt}simple{\isacharunderscore}{\kern0pt}function\ simple\ integrable\ \isacommand{by}\isamarkupfalse%
\ {\isacharparenleft}{\kern0pt}intro\ integral{\isacharunderscore}{\kern0pt}dominated{\isacharunderscore}{\kern0pt}convergence{\isacharbrackleft}{\kern0pt}OF\ max{\isacharparenleft}{\kern0pt}{\isadigit{1}}{\isacharparenright}{\kern0pt}\ {\isacharunderscore}{\kern0pt}\ integrable{\isacharunderscore}{\kern0pt}norm{\isacharbrackleft}{\kern0pt}OF\ integrable{\isacharunderscore}{\kern0pt}scaleR{\isacharunderscore}{\kern0pt}right{\isacharbrackright}{\kern0pt}{\isacharcomma}{\kern0pt}\ of\ {\isacharunderscore}{\kern0pt}\ {\isadigit{2}}\ f{\isacharbrackright}{\kern0pt}{\isacharcomma}{\kern0pt}\ blast{\isacharplus}{\kern0pt}{\isacharparenright}{\kern0pt}\isanewline
\ \ \ \ \isacommand{{\isacharbraceleft}{\kern0pt}}\isamarkupfalse%
\isanewline
\ \ \ \ \ \ \isacommand{fix}\isamarkupfalse%
\ h\ {\isacharcolon}{\kern0pt}{\isacharcolon}{\kern0pt}\ {\isachardoublequoteopen}{\isacharprime}{\kern0pt}a\ {\isasymRightarrow}\ {\isacharprime}{\kern0pt}b\ {\isacharcolon}{\kern0pt}{\isacharcolon}{\kern0pt}\ {\isacharbraceleft}{\kern0pt}second{\isacharunderscore}{\kern0pt}countable{\isacharunderscore}{\kern0pt}topology{\isacharcomma}{\kern0pt}\ banach{\isacharcomma}{\kern0pt}\ linorder{\isacharunderscore}{\kern0pt}topology{\isacharcomma}{\kern0pt}\ ordered{\isacharunderscore}{\kern0pt}real{\isacharunderscore}{\kern0pt}vector{\isacharbraceright}{\kern0pt}{\isachardoublequoteclose}\ \isanewline
\ \ \ \ \ \ \isacommand{assume}\isamarkupfalse%
\ {\isachardoublequoteopen}simple{\isacharunderscore}{\kern0pt}function\ M\ h{\isachardoublequoteclose}\ {\isachardoublequoteopen}emeasure\ M\ {\isacharbraceleft}{\kern0pt}y\ {\isasymin}\ space\ M{\isachardot}{\kern0pt}\ h\ y\ {\isasymnoteq}\ {\isadigit{0}}{\isacharbraceright}{\kern0pt}\ {\isasymnoteq}\ {\isasyminfinity}{\isachardoublequoteclose}\ {\isachardoublequoteopen}{\isasymAnd}x{\isachardot}{\kern0pt}\ x\ {\isasymin}\ space\ M\ {\isasymlongrightarrow}\ h\ x\ {\isasymge}\ {\isadigit{0}}{\isachardoublequoteclose}\isanewline
\ \ \ \ \ \ \isacommand{hence}\isamarkupfalse%
\ {\isacharasterisk}{\kern0pt}{\isacharcolon}{\kern0pt}\ {\isachardoublequoteopen}integral\isactrlsup L\ M\ h\ {\isasymge}\ {\isadigit{0}}{\isachardoublequoteclose}\isanewline
\ \ \ \ \ \ \isacommand{proof}\isamarkupfalse%
\ {\isacharparenleft}{\kern0pt}induct\ rule{\isacharcolon}{\kern0pt}\ simple{\isacharunderscore}{\kern0pt}integrable{\isacharunderscore}{\kern0pt}function{\isacharunderscore}{\kern0pt}induct{\isacharunderscore}{\kern0pt}nn{\isacharparenright}{\kern0pt}\isanewline
\ \ \ \ \ \ \ \ \isacommand{case}\isamarkupfalse%
\ {\isacharparenleft}{\kern0pt}cong\ f\ g{\isacharparenright}{\kern0pt}\ \ \ \ \ \ \ \ \ \ \ \ \ \ \ \ \ \ \ \isanewline
\ \ \ \ \ \ \ \ \isacommand{then}\isamarkupfalse%
\ \isacommand{show}\isamarkupfalse%
\ {\isacharquery}{\kern0pt}case\ \isacommand{using}\isamarkupfalse%
\ Bochner{\isacharunderscore}{\kern0pt}Integration{\isachardot}{\kern0pt}integral{\isacharunderscore}{\kern0pt}cong\ \isacommand{by}\isamarkupfalse%
\ force\isanewline
\ \ \ \ \ \ \isacommand{next}\isamarkupfalse%
\isanewline
\ \ \ \ \ \ \ \ \isacommand{case}\isamarkupfalse%
\ {\isacharparenleft}{\kern0pt}indicator\ A\ y{\isacharparenright}{\kern0pt}\isanewline
\ \ \ \ \ \ \ \ \isacommand{hence}\isamarkupfalse%
\ {\isachardoublequoteopen}A\ {\isasymnoteq}\ {\isacharbraceleft}{\kern0pt}{\isacharbraceright}{\kern0pt}\ {\isasymLongrightarrow}\ y\ {\isasymge}\ {\isadigit{0}}{\isachardoublequoteclose}\ \isacommand{using}\isamarkupfalse%
\ sets{\isachardot}{\kern0pt}sets{\isacharunderscore}{\kern0pt}into{\isacharunderscore}{\kern0pt}space\ \isacommand{by}\isamarkupfalse%
\ fastforce\isanewline
\ \ \ \ \ \ \ \ \isacommand{then}\isamarkupfalse%
\ \isacommand{show}\isamarkupfalse%
\ {\isacharquery}{\kern0pt}case\ \isacommand{using}\isamarkupfalse%
\ indicator\ \isacommand{by}\isamarkupfalse%
\ {\isacharparenleft}{\kern0pt}cases\ {\isachardoublequoteopen}A\ {\isacharequal}{\kern0pt}\ {\isacharbraceleft}{\kern0pt}{\isacharbraceright}{\kern0pt}{\isachardoublequoteclose}{\isacharcomma}{\kern0pt}\ auto\ simp\ add{\isacharcolon}{\kern0pt}\ scaleR{\isacharunderscore}{\kern0pt}nonneg{\isacharunderscore}{\kern0pt}nonneg{\isacharparenright}{\kern0pt}\isanewline
\ \ \ \ \ \ \isacommand{next}\isamarkupfalse%
\isanewline
\ \ \ \ \ \ \ \ \isacommand{case}\isamarkupfalse%
\ {\isacharparenleft}{\kern0pt}add\ f\ g{\isacharparenright}{\kern0pt}\isanewline
\ \ \ \ \ \ \ \ \isacommand{then}\isamarkupfalse%
\ \isacommand{show}\isamarkupfalse%
\ {\isacharquery}{\kern0pt}case\ \isacommand{by}\isamarkupfalse%
\ {\isacharparenleft}{\kern0pt}simp\ add{\isacharcolon}{\kern0pt}\ integrable{\isacharunderscore}{\kern0pt}simple{\isacharunderscore}{\kern0pt}function{\isacharparenright}{\kern0pt}\isanewline
\ \ \ \ \ \ \isacommand{qed}\isamarkupfalse%
\isanewline
\ \ \ \ \isacommand{{\isacharbraceright}{\kern0pt}}\isamarkupfalse%
\isanewline
\ \ \ \ \isacommand{thus}\isamarkupfalse%
\ {\isacharquery}{\kern0pt}thesis\ \isacommand{using}\isamarkupfalse%
\ LIMSEQ{\isacharunderscore}{\kern0pt}le{\isacharunderscore}{\kern0pt}const{\isacharbrackleft}{\kern0pt}OF\ tendsto{\isacharcomma}{\kern0pt}\ of\ {\isadigit{0}}{\isacharbrackright}{\kern0pt}\ {\isacharasterisk}{\kern0pt}{\isacharasterisk}{\kern0pt}\ simple\ \isacommand{by}\isamarkupfalse%
\ fastforce\isanewline
\ \ \isacommand{qed}\isamarkupfalse%
\isanewline
\ \ \isacommand{also}\isamarkupfalse%
\ \isacommand{have}\isamarkupfalse%
\ {\isachardoublequoteopen}{\isasymdots}\ {\isacharequal}{\kern0pt}\ integral\isactrlsup L\ M\ f{\isachardoublequoteclose}\ \isacommand{using}\isamarkupfalse%
\ nonneg\ \isacommand{by}\isamarkupfalse%
\ {\isacharparenleft}{\kern0pt}auto\ intro{\isacharcolon}{\kern0pt}\ integral{\isacharunderscore}{\kern0pt}cong{\isacharunderscore}{\kern0pt}AE{\isacharparenright}{\kern0pt}\isanewline
\ \ \isacommand{finally}\isamarkupfalse%
\ \isacommand{show}\isamarkupfalse%
\ {\isacharquery}{\kern0pt}thesis\ \isacommand{{\isachardot}{\kern0pt}}\isamarkupfalse%
\isanewline
\isacommand{qed}\isamarkupfalse%
\ {\isacharparenleft}{\kern0pt}simp\ add{\isacharcolon}{\kern0pt}\ not{\isacharunderscore}{\kern0pt}integrable{\isacharunderscore}{\kern0pt}integral{\isacharunderscore}{\kern0pt}eq{\isacharparenright}{\kern0pt}%
\endisatagproof
{\isafoldproof}%
%
\isadelimproof
\isanewline
%
\endisadelimproof
\isanewline
\isacommand{lemma}\isamarkupfalse%
\ integral{\isacharunderscore}{\kern0pt}mono{\isacharunderscore}{\kern0pt}AE{\isacharunderscore}{\kern0pt}banach{\isacharcolon}{\kern0pt}\isanewline
\ \ \isakeyword{fixes}\ f\ g\ {\isacharcolon}{\kern0pt}{\isacharcolon}{\kern0pt}\ {\isachardoublequoteopen}{\isacharprime}{\kern0pt}a\ {\isasymRightarrow}\ {\isacharprime}{\kern0pt}b\ {\isacharcolon}{\kern0pt}{\isacharcolon}{\kern0pt}\ {\isacharbraceleft}{\kern0pt}second{\isacharunderscore}{\kern0pt}countable{\isacharunderscore}{\kern0pt}topology{\isacharcomma}{\kern0pt}\ banach{\isacharcomma}{\kern0pt}\ linorder{\isacharunderscore}{\kern0pt}topology{\isacharcomma}{\kern0pt}\ ordered{\isacharunderscore}{\kern0pt}real{\isacharunderscore}{\kern0pt}vector{\isacharbraceright}{\kern0pt}{\isachardoublequoteclose}\isanewline
\ \ \isakeyword{assumes}\ {\isachardoublequoteopen}integrable\ M\ f{\isachardoublequoteclose}\ {\isachardoublequoteopen}integrable\ M\ g{\isachardoublequoteclose}\ {\isachardoublequoteopen}AE\ x\ in\ M{\isachardot}{\kern0pt}\ f\ x\ {\isasymle}\ g\ x{\isachardoublequoteclose}\isanewline
\ \ \isakeyword{shows}\ {\isachardoublequoteopen}integral\isactrlsup L\ M\ f\ {\isasymle}\ integral\isactrlsup L\ M\ g{\isachardoublequoteclose}\isanewline
%
\isadelimproof
\ \ %
\endisadelimproof
%
\isatagproof
\isacommand{using}\isamarkupfalse%
\ integral{\isacharunderscore}{\kern0pt}nonneg{\isacharunderscore}{\kern0pt}AE{\isacharunderscore}{\kern0pt}banach{\isacharbrackleft}{\kern0pt}of\ {\isachardoublequoteopen}{\isasymlambda}x{\isachardot}{\kern0pt}\ g\ x\ {\isacharminus}{\kern0pt}\ f\ x{\isachardoublequoteclose}{\isacharbrackright}{\kern0pt}\ assms\ Bochner{\isacharunderscore}{\kern0pt}Integration{\isachardot}{\kern0pt}integral{\isacharunderscore}{\kern0pt}diff{\isacharbrackleft}{\kern0pt}OF\ assms{\isacharparenleft}{\kern0pt}{\isadigit{1}}{\isacharcomma}{\kern0pt}{\isadigit{2}}{\isacharparenright}{\kern0pt}{\isacharbrackright}{\kern0pt}\ \isacommand{by}\isamarkupfalse%
\ force%
\endisatagproof
{\isafoldproof}%
%
\isadelimproof
\isanewline
%
\endisadelimproof
\isanewline
\isacommand{lemma}\isamarkupfalse%
\ integral{\isacharunderscore}{\kern0pt}mono{\isacharunderscore}{\kern0pt}banach{\isacharcolon}{\kern0pt}\isanewline
\ \ \isakeyword{fixes}\ f\ g\ {\isacharcolon}{\kern0pt}{\isacharcolon}{\kern0pt}\ {\isachardoublequoteopen}{\isacharprime}{\kern0pt}a\ {\isasymRightarrow}\ {\isacharprime}{\kern0pt}b\ {\isacharcolon}{\kern0pt}{\isacharcolon}{\kern0pt}\ {\isacharbraceleft}{\kern0pt}second{\isacharunderscore}{\kern0pt}countable{\isacharunderscore}{\kern0pt}topology{\isacharcomma}{\kern0pt}\ banach{\isacharcomma}{\kern0pt}\ linorder{\isacharunderscore}{\kern0pt}topology{\isacharcomma}{\kern0pt}\ ordered{\isacharunderscore}{\kern0pt}real{\isacharunderscore}{\kern0pt}vector{\isacharbraceright}{\kern0pt}{\isachardoublequoteclose}\isanewline
\ \ \isakeyword{assumes}\ {\isachardoublequoteopen}integrable\ M\ f{\isachardoublequoteclose}\ {\isachardoublequoteopen}integrable\ M\ g{\isachardoublequoteclose}\ {\isachardoublequoteopen}{\isasymAnd}x{\isachardot}{\kern0pt}\ x\ {\isasymin}\ space\ M\ {\isasymLongrightarrow}\ f\ x\ {\isasymle}\ g\ x{\isachardoublequoteclose}\isanewline
\ \ \isakeyword{shows}\ {\isachardoublequoteopen}integral\isactrlsup L\ M\ f\ {\isasymle}\ integral\isactrlsup L\ M\ g{\isachardoublequoteclose}\isanewline
%
\isadelimproof
\ \ %
\endisadelimproof
%
\isatagproof
\isacommand{using}\isamarkupfalse%
\ integral{\isacharunderscore}{\kern0pt}mono{\isacharunderscore}{\kern0pt}AE{\isacharunderscore}{\kern0pt}banach\ assms\ \isacommand{by}\isamarkupfalse%
\ blast%
\endisatagproof
{\isafoldproof}%
%
\isadelimproof
\isanewline
%
\endisadelimproof
%
\isadelimtheory
\isanewline
%
\endisadelimtheory
%
\isatagtheory
\isacommand{end}\isamarkupfalse%
%
\endisatagtheory
{\isafoldtheory}%
%
\isadelimtheory
%
\endisadelimtheory
%
\end{isabellebody}%
\endinput
%:%file=Bochner_Integration_Addendum.tex%:%
%:%10=1%:%
%:%11=1%:%
%:%12=2%:%
%:%13=3%:%
%:%27=5%:%
%:%37=7%:%
%:%38=7%:%
%:%39=8%:%
%:%40=9%:%
%:%41=10%:%
%:%42=11%:%
%:%43=12%:%
%:%44=13%:%
%:%51=14%:%
%:%52=14%:%
%:%53=15%:%
%:%54=15%:%
%:%55=15%:%
%:%56=15%:%
%:%57=15%:%
%:%58=16%:%
%:%59=16%:%
%:%60=16%:%
%:%61=16%:%
%:%62=16%:%
%:%63=17%:%
%:%64=17%:%
%:%65=18%:%
%:%66=18%:%
%:%67=19%:%
%:%68=19%:%
%:%69=19%:%
%:%70=19%:%
%:%71=20%:%
%:%72=20%:%
%:%73=20%:%
%:%74=20%:%
%:%75=20%:%
%:%76=21%:%
%:%77=21%:%
%:%78=21%:%
%:%79=21%:%
%:%80=21%:%
%:%81=22%:%
%:%82=22%:%
%:%83=22%:%
%:%84=23%:%
%:%85=23%:%
%:%86=24%:%
%:%87=24%:%
%:%88=24%:%
%:%89=24%:%
%:%90=25%:%
%:%96=25%:%
%:%99=26%:%
%:%100=27%:%
%:%101=27%:%
%:%102=28%:%
%:%103=29%:%
%:%104=30%:%
%:%105=31%:%
%:%112=32%:%
%:%113=32%:%
%:%114=33%:%
%:%115=33%:%
%:%116=34%:%
%:%117=34%:%
%:%118=35%:%
%:%119=35%:%
%:%120=35%:%
%:%121=35%:%
%:%122=35%:%
%:%123=36%:%
%:%124=36%:%
%:%125=36%:%
%:%126=36%:%
%:%127=36%:%
%:%128=37%:%
%:%129=37%:%
%:%130=37%:%
%:%131=37%:%
%:%132=38%:%
%:%138=38%:%
%:%141=39%:%
%:%142=40%:%
%:%143=40%:%
%:%144=41%:%
%:%145=42%:%
%:%146=43%:%
%:%149=44%:%
%:%153=44%:%
%:%154=44%:%
%:%159=44%:%
%:%162=45%:%
%:%163=46%:%
%:%164=46%:%
%:%165=47%:%
%:%166=47%:%
%:%167=48%:%
%:%168=49%:%
%:%171=50%:%
%:%176=51%:%
%:%177=51%:%
%:%178=52%:%
%:%179=53%:%
%:%180=54%:%
%:%182=56%:%
%:%183=57%:%
%:%184=58%:%
%:%187=61%:%
%:%188=62%:%
%:%193=62%:%
%:%202=63%:%
%:%203=63%:%
%:%204=64%:%
%:%205=64%:%
%:%206=65%:%
%:%207=65%:%
%:%208=65%:%
%:%209=65%:%
%:%210=66%:%
%:%211=66%:%
%:%212=66%:%
%:%213=66%:%
%:%214=67%:%
%:%215=67%:%
%:%216=67%:%
%:%217=68%:%
%:%218=69%:%
%:%219=70%:%
%:%220=70%:%
%:%221=71%:%
%:%222=71%:%
%:%223=72%:%
%:%224=72%:%
%:%225=73%:%
%:%226=73%:%
%:%227=74%:%
%:%228=74%:%
%:%229=75%:%
%:%230=75%:%
%:%231=75%:%
%:%232=75%:%
%:%233=75%:%
%:%234=76%:%
%:%235=76%:%
%:%236=77%:%
%:%237=77%:%
%:%238=78%:%
%:%239=78%:%
%:%240=78%:%
%:%241=78%:%
%:%242=79%:%
%:%243=79%:%
%:%244=80%:%
%:%245=80%:%
%:%246=81%:%
%:%247=81%:%
%:%248=81%:%
%:%249=81%:%
%:%250=82%:%
%:%251=82%:%
%:%252=83%:%
%:%253=83%:%
%:%254=84%:%
%:%255=84%:%
%:%256=85%:%
%:%257=85%:%
%:%258=85%:%
%:%259=85%:%
%:%260=86%:%
%:%261=86%:%
%:%262=86%:%
%:%263=86%:%
%:%264=87%:%
%:%265=87%:%
%:%266=87%:%
%:%267=87%:%
%:%268=87%:%
%:%269=88%:%
%:%270=88%:%
%:%271=88%:%
%:%272=88%:%
%:%273=88%:%
%:%274=89%:%
%:%275=89%:%
%:%276=89%:%
%:%277=90%:%
%:%278=91%:%
%:%279=91%:%
%:%280=91%:%
%:%281=92%:%
%:%282=92%:%
%:%283=92%:%
%:%284=92%:%
%:%285=93%:%
%:%286=93%:%
%:%287=94%:%
%:%288=94%:%
%:%289=95%:%
%:%290=95%:%
%:%291=95%:%
%:%292=96%:%
%:%293=96%:%
%:%294=97%:%
%:%295=97%:%
%:%296=98%:%
%:%297=98%:%
%:%298=99%:%
%:%299=99%:%
%:%300=99%:%
%:%301=99%:%
%:%302=99%:%
%:%303=99%:%
%:%304=100%:%
%:%305=100%:%
%:%306=101%:%
%:%307=101%:%
%:%308=102%:%
%:%309=102%:%
%:%310=103%:%
%:%311=103%:%
%:%312=104%:%
%:%313=104%:%
%:%314=105%:%
%:%315=105%:%
%:%316=105%:%
%:%317=105%:%
%:%318=105%:%
%:%319=106%:%
%:%320=106%:%
%:%321=106%:%
%:%322=107%:%
%:%323=107%:%
%:%324=107%:%
%:%325=107%:%
%:%326=108%:%
%:%327=108%:%
%:%328=109%:%
%:%329=109%:%
%:%330=110%:%
%:%331=110%:%
%:%332=110%:%
%:%333=110%:%
%:%334=111%:%
%:%335=111%:%
%:%336=112%:%
%:%337=112%:%
%:%338=112%:%
%:%339=112%:%
%:%340=113%:%
%:%341=113%:%
%:%342=114%:%
%:%343=114%:%
%:%344=115%:%
%:%345=115%:%
%:%346=116%:%
%:%347=116%:%
%:%348=116%:%
%:%349=117%:%
%:%350=117%:%
%:%351=118%:%
%:%352=118%:%
%:%353=119%:%
%:%354=119%:%
%:%355=120%:%
%:%356=120%:%
%:%357=120%:%
%:%358=120%:%
%:%359=120%:%
%:%360=120%:%
%:%361=121%:%
%:%362=121%:%
%:%363=122%:%
%:%364=122%:%
%:%365=123%:%
%:%366=123%:%
%:%367=123%:%
%:%368=123%:%
%:%369=123%:%
%:%370=123%:%
%:%371=124%:%
%:%372=124%:%
%:%373=125%:%
%:%374=125%:%
%:%375=126%:%
%:%376=126%:%
%:%377=127%:%
%:%378=127%:%
%:%379=127%:%
%:%380=128%:%
%:%381=128%:%
%:%382=129%:%
%:%383=129%:%
%:%384=130%:%
%:%385=130%:%
%:%386=131%:%
%:%387=131%:%
%:%388=131%:%
%:%389=131%:%
%:%390=131%:%
%:%391=131%:%
%:%392=132%:%
%:%393=132%:%
%:%394=133%:%
%:%395=133%:%
%:%396=134%:%
%:%397=134%:%
%:%398=135%:%
%:%399=135%:%
%:%400=135%:%
%:%401=136%:%
%:%402=136%:%
%:%403=136%:%
%:%404=137%:%
%:%405=137%:%
%:%406=137%:%
%:%407=138%:%
%:%408=138%:%
%:%409=138%:%
%:%410=138%:%
%:%411=138%:%
%:%412=139%:%
%:%413=139%:%
%:%414=139%:%
%:%415=139%:%
%:%416=140%:%
%:%417=140%:%
%:%418=141%:%
%:%419=141%:%
%:%420=142%:%
%:%421=142%:%
%:%422=143%:%
%:%423=143%:%
%:%424=143%:%
%:%425=143%:%
%:%426=143%:%
%:%427=144%:%
%:%428=144%:%
%:%429=144%:%
%:%430=144%:%
%:%431=144%:%
%:%432=145%:%
%:%433=145%:%
%:%434=145%:%
%:%435=145%:%
%:%436=146%:%
%:%442=146%:%
%:%445=147%:%
%:%448=148%:%
%:%453=149%:%
%:%454=149%:%
%:%455=150%:%
%:%456=151%:%
%:%457=152%:%
%:%458=153%:%
%:%459=154%:%
%:%462=157%:%
%:%463=158%:%
%:%468=158%:%
%:%477=159%:%
%:%478=159%:%
%:%479=160%:%
%:%480=160%:%
%:%481=161%:%
%:%482=161%:%
%:%483=161%:%
%:%484=161%:%
%:%485=162%:%
%:%486=162%:%
%:%487=162%:%
%:%488=162%:%
%:%489=163%:%
%:%490=163%:%
%:%491=163%:%
%:%492=164%:%
%:%493=165%:%
%:%494=166%:%
%:%495=167%:%
%:%496=167%:%
%:%497=168%:%
%:%498=168%:%
%:%499=169%:%
%:%500=169%:%
%:%501=170%:%
%:%502=170%:%
%:%503=171%:%
%:%504=171%:%
%:%505=172%:%
%:%506=172%:%
%:%507=172%:%
%:%508=172%:%
%:%509=172%:%
%:%510=173%:%
%:%511=173%:%
%:%512=174%:%
%:%513=174%:%
%:%514=175%:%
%:%515=175%:%
%:%516=175%:%
%:%517=175%:%
%:%518=176%:%
%:%519=176%:%
%:%520=177%:%
%:%521=177%:%
%:%522=178%:%
%:%523=178%:%
%:%524=178%:%
%:%525=178%:%
%:%526=179%:%
%:%527=179%:%
%:%528=180%:%
%:%529=180%:%
%:%530=181%:%
%:%531=181%:%
%:%532=181%:%
%:%533=181%:%
%:%534=182%:%
%:%535=182%:%
%:%536=183%:%
%:%537=183%:%
%:%538=184%:%
%:%539=184%:%
%:%540=185%:%
%:%541=185%:%
%:%542=185%:%
%:%543=185%:%
%:%544=186%:%
%:%545=186%:%
%:%546=186%:%
%:%547=186%:%
%:%548=187%:%
%:%549=187%:%
%:%550=187%:%
%:%551=187%:%
%:%552=187%:%
%:%553=188%:%
%:%554=188%:%
%:%555=188%:%
%:%556=188%:%
%:%557=188%:%
%:%558=189%:%
%:%559=189%:%
%:%560=189%:%
%:%561=190%:%
%:%562=191%:%
%:%563=191%:%
%:%564=191%:%
%:%565=191%:%
%:%566=192%:%
%:%567=192%:%
%:%568=192%:%
%:%569=193%:%
%:%570=193%:%
%:%571=193%:%
%:%572=193%:%
%:%573=194%:%
%:%574=194%:%
%:%575=195%:%
%:%576=195%:%
%:%577=196%:%
%:%578=196%:%
%:%579=197%:%
%:%580=197%:%
%:%581=198%:%
%:%582=198%:%
%:%583=199%:%
%:%584=199%:%
%:%585=199%:%
%:%586=199%:%
%:%587=199%:%
%:%588=199%:%
%:%589=200%:%
%:%590=200%:%
%:%591=201%:%
%:%592=201%:%
%:%593=202%:%
%:%594=202%:%
%:%595=203%:%
%:%596=203%:%
%:%597=204%:%
%:%598=204%:%
%:%599=205%:%
%:%600=205%:%
%:%601=205%:%
%:%602=205%:%
%:%603=205%:%
%:%604=206%:%
%:%605=206%:%
%:%606=206%:%
%:%607=207%:%
%:%608=207%:%
%:%609=207%:%
%:%610=208%:%
%:%611=208%:%
%:%612=209%:%
%:%613=209%:%
%:%614=209%:%
%:%615=209%:%
%:%616=210%:%
%:%617=210%:%
%:%618=211%:%
%:%619=211%:%
%:%620=212%:%
%:%621=212%:%
%:%622=213%:%
%:%623=213%:%
%:%624=214%:%
%:%625=214%:%
%:%626=215%:%
%:%627=215%:%
%:%628=216%:%
%:%629=216%:%
%:%630=216%:%
%:%631=216%:%
%:%632=216%:%
%:%633=216%:%
%:%634=217%:%
%:%635=217%:%
%:%636=218%:%
%:%637=218%:%
%:%638=219%:%
%:%639=219%:%
%:%640=219%:%
%:%641=219%:%
%:%642=219%:%
%:%643=219%:%
%:%644=220%:%
%:%645=220%:%
%:%646=221%:%
%:%647=221%:%
%:%648=222%:%
%:%649=222%:%
%:%650=223%:%
%:%651=223%:%
%:%652=224%:%
%:%653=224%:%
%:%654=225%:%
%:%655=225%:%
%:%656=226%:%
%:%657=226%:%
%:%658=226%:%
%:%659=226%:%
%:%660=226%:%
%:%661=226%:%
%:%662=227%:%
%:%663=227%:%
%:%664=228%:%
%:%665=228%:%
%:%666=229%:%
%:%667=229%:%
%:%668=230%:%
%:%669=230%:%
%:%670=230%:%
%:%671=231%:%
%:%672=231%:%
%:%673=231%:%
%:%674=232%:%
%:%675=232%:%
%:%676=232%:%
%:%677=233%:%
%:%678=233%:%
%:%679=233%:%
%:%680=233%:%
%:%681=233%:%
%:%682=234%:%
%:%683=234%:%
%:%684=234%:%
%:%685=234%:%
%:%686=235%:%
%:%687=235%:%
%:%688=236%:%
%:%689=236%:%
%:%690=237%:%
%:%691=237%:%
%:%692=238%:%
%:%693=238%:%
%:%694=238%:%
%:%695=238%:%
%:%696=239%:%
%:%697=239%:%
%:%698=240%:%
%:%699=240%:%
%:%700=241%:%
%:%701=241%:%
%:%702=241%:%
%:%703=241%:%
%:%704=241%:%
%:%705=242%:%
%:%706=242%:%
%:%707=242%:%
%:%708=242%:%
%:%709=242%:%
%:%710=243%:%
%:%711=243%:%
%:%712=243%:%
%:%713=243%:%
%:%714=243%:%
%:%715=244%:%
%:%716=244%:%
%:%717=244%:%
%:%718=244%:%
%:%719=245%:%
%:%725=245%:%
%:%728=246%:%
%:%729=247%:%
%:%730=247%:%
%:%731=248%:%
%:%732=249%:%
%:%733=250%:%
%:%740=251%:%
%:%741=251%:%
%:%742=252%:%
%:%743=252%:%
%:%744=253%:%
%:%745=253%:%
%:%746=254%:%
%:%747=254%:%
%:%748=254%:%
%:%749=254%:%
%:%750=255%:%
%:%751=255%:%
%:%752=255%:%
%:%753=256%:%
%:%754=256%:%
%:%755=256%:%
%:%756=256%:%
%:%757=256%:%
%:%758=257%:%
%:%759=257%:%
%:%760=257%:%
%:%761=257%:%
%:%762=257%:%
%:%763=258%:%
%:%764=258%:%
%:%765=258%:%
%:%766=258%:%
%:%767=259%:%
%:%768=259%:%
%:%769=259%:%
%:%770=259%:%
%:%771=260%:%
%:%772=260%:%
%:%773=260%:%
%:%774=260%:%
%:%775=260%:%
%:%776=261%:%
%:%782=261%:%
%:%785=262%:%
%:%786=263%:%
%:%787=263%:%
%:%788=264%:%
%:%789=265%:%
%:%790=266%:%
%:%791=267%:%
%:%798=268%:%
%:%799=268%:%
%:%800=269%:%
%:%801=269%:%
%:%802=270%:%
%:%803=270%:%
%:%804=271%:%
%:%805=271%:%
%:%806=272%:%
%:%807=272%:%
%:%808=272%:%
%:%809=272%:%
%:%810=272%:%
%:%811=273%:%
%:%812=273%:%
%:%813=274%:%
%:%814=274%:%
%:%815=275%:%
%:%816=275%:%
%:%817=275%:%
%:%818=275%:%
%:%819=276%:%
%:%820=276%:%
%:%821=277%:%
%:%822=277%:%
%:%823=277%:%
%:%824=278%:%
%:%825=278%:%
%:%826=278%:%
%:%827=278%:%
%:%828=279%:%
%:%829=279%:%
%:%830=280%:%
%:%831=280%:%
%:%832=280%:%
%:%833=280%:%
%:%834=281%:%
%:%835=281%:%
%:%836=282%:%
%:%837=282%:%
%:%838=282%:%
%:%839=282%:%
%:%840=282%:%
%:%841=283%:%
%:%842=283%:%
%:%843=283%:%
%:%844=284%:%
%:%845=285%:%
%:%846=285%:%
%:%847=286%:%
%:%848=286%:%
%:%849=287%:%
%:%850=288%:%
%:%851=288%:%
%:%852=289%:%
%:%853=289%:%
%:%854=290%:%
%:%855=290%:%
%:%856=290%:%
%:%857=290%:%
%:%858=291%:%
%:%859=291%:%
%:%860=291%:%
%:%861=291%:%
%:%862=292%:%
%:%863=292%:%
%:%864=292%:%
%:%865=292%:%
%:%866=292%:%
%:%867=292%:%
%:%868=293%:%
%:%869=293%:%
%:%870=293%:%
%:%871=293%:%
%:%872=294%:%
%:%873=294%:%
%:%874=294%:%
%:%875=294%:%
%:%876=294%:%
%:%877=295%:%
%:%878=295%:%
%:%879=295%:%
%:%880=295%:%
%:%881=296%:%
%:%882=296%:%
%:%883=297%:%
%:%884=298%:%
%:%885=298%:%
%:%886=299%:%
%:%887=299%:%
%:%888=300%:%
%:%889=300%:%
%:%890=300%:%
%:%891=301%:%
%:%892=301%:%
%:%893=301%:%
%:%894=301%:%
%:%895=301%:%
%:%896=302%:%
%:%897=302%:%
%:%898=302%:%
%:%899=302%:%
%:%900=303%:%
%:%901=303%:%
%:%902=303%:%
%:%903=304%:%
%:%904=304%:%
%:%905=304%:%
%:%906=304%:%
%:%907=305%:%
%:%908=305%:%
%:%909=305%:%
%:%910=305%:%
%:%911=305%:%
%:%912=306%:%
%:%913=306%:%
%:%914=306%:%
%:%915=306%:%
%:%916=306%:%
%:%917=307%:%
%:%918=307%:%
%:%919=307%:%
%:%920=307%:%
%:%921=308%:%
%:%922=308%:%
%:%923=309%:%
%:%924=309%:%
%:%925=309%:%
%:%926=310%:%
%:%927=310%:%
%:%928=310%:%
%:%929=310%:%
%:%930=310%:%
%:%931=310%:%
%:%932=311%:%
%:%933=311%:%
%:%934=311%:%
%:%935=311%:%
%:%936=311%:%
%:%937=311%:%
%:%938=312%:%
%:%939=313%:%
%:%940=313%:%
%:%941=314%:%
%:%942=314%:%
%:%943=314%:%
%:%944=315%:%
%:%945=315%:%
%:%946=315%:%
%:%947=315%:%
%:%948=316%:%
%:%949=316%:%
%:%950=316%:%
%:%951=316%:%
%:%952=316%:%
%:%953=317%:%
%:%954=317%:%
%:%955=317%:%
%:%956=317%:%
%:%957=318%:%
%:%958=318%:%
%:%959=318%:%
%:%960=319%:%
%:%961=319%:%
%:%962=319%:%
%:%963=320%:%
%:%964=320%:%
%:%965=321%:%
%:%966=322%:%
%:%967=322%:%
%:%968=322%:%
%:%969=322%:%
%:%970=323%:%
%:%971=323%:%
%:%972=323%:%
%:%973=324%:%
%:%979=324%:%
%:%982=325%:%
%:%983=326%:%
%:%984=326%:%
%:%985=327%:%
%:%986=328%:%
%:%987=329%:%
%:%994=330%:%
%:%995=330%:%
%:%996=331%:%
%:%997=331%:%
%:%998=332%:%
%:%999=332%:%
%:%1000=333%:%
%:%1001=333%:%
%:%1002=333%:%
%:%1003=334%:%
%:%1004=334%:%
%:%1005=334%:%
%:%1006=334%:%
%:%1007=335%:%
%:%1008=335%:%
%:%1009=335%:%
%:%1010=335%:%
%:%1011=336%:%
%:%1012=336%:%
%:%1013=337%:%
%:%1014=337%:%
%:%1015=337%:%
%:%1016=338%:%
%:%1017=338%:%
%:%1022=338%:%
%:%1025=339%:%
%:%1026=340%:%
%:%1027=340%:%
%:%1028=341%:%
%:%1029=342%:%
%:%1030=343%:%
%:%1037=344%:%
%:%1038=344%:%
%:%1039=345%:%
%:%1040=345%:%
%:%1041=345%:%
%:%1042=346%:%
%:%1043=346%:%
%:%1044=346%:%
%:%1045=347%:%
%:%1051=347%:%
%:%1054=348%:%
%:%1055=349%:%
%:%1056=350%:%
%:%1057=351%:%
%:%1058=351%:%
%:%1059=352%:%
%:%1060=353%:%
%:%1061=354%:%
%:%1068=355%:%
%:%1069=355%:%
%:%1070=356%:%
%:%1071=356%:%
%:%1072=357%:%
%:%1073=357%:%
%:%1074=357%:%
%:%1075=358%:%
%:%1076=358%:%
%:%1077=359%:%
%:%1078=359%:%
%:%1079=360%:%
%:%1080=360%:%
%:%1081=361%:%
%:%1082=362%:%
%:%1083=363%:%
%:%1084=363%:%
%:%1085=363%:%
%:%1086=364%:%
%:%1087=364%:%
%:%1088=364%:%
%:%1089=364%:%
%:%1090=365%:%
%:%1091=365%:%
%:%1092=365%:%
%:%1093=365%:%
%:%1094=366%:%
%:%1095=366%:%
%:%1096=366%:%
%:%1097=366%:%
%:%1098=366%:%
%:%1099=367%:%
%:%1100=367%:%
%:%1101=367%:%
%:%1102=367%:%
%:%1103=368%:%
%:%1104=368%:%
%:%1105=368%:%
%:%1106=368%:%
%:%1107=369%:%
%:%1108=369%:%
%:%1109=369%:%
%:%1110=369%:%
%:%1111=370%:%
%:%1112=370%:%
%:%1113=370%:%
%:%1114=370%:%
%:%1115=370%:%
%:%1116=370%:%
%:%1117=371%:%
%:%1118=371%:%
%:%1119=371%:%
%:%1120=372%:%
%:%1121=372%:%
%:%1122=372%:%
%:%1123=373%:%
%:%1124=373%:%
%:%1125=374%:%
%:%1126=374%:%
%:%1127=375%:%
%:%1128=375%:%
%:%1129=376%:%
%:%1130=376%:%
%:%1131=377%:%
%:%1132=377%:%
%:%1133=378%:%
%:%1134=378%:%
%:%1135=379%:%
%:%1136=379%:%
%:%1137=379%:%
%:%1138=379%:%
%:%1139=379%:%
%:%1140=380%:%
%:%1141=380%:%
%:%1142=381%:%
%:%1143=381%:%
%:%1144=382%:%
%:%1145=382%:%
%:%1146=382%:%
%:%1147=382%:%
%:%1148=383%:%
%:%1149=383%:%
%:%1150=383%:%
%:%1151=383%:%
%:%1152=383%:%
%:%1153=384%:%
%:%1154=384%:%
%:%1155=385%:%
%:%1156=385%:%
%:%1157=386%:%
%:%1158=386%:%
%:%1159=386%:%
%:%1160=386%:%
%:%1161=387%:%
%:%1162=387%:%
%:%1163=388%:%
%:%1164=388%:%
%:%1165=389%:%
%:%1166=389%:%
%:%1167=389%:%
%:%1168=389%:%
%:%1169=390%:%
%:%1170=390%:%
%:%1171=391%:%
%:%1172=391%:%
%:%1173=391%:%
%:%1174=391%:%
%:%1175=391%:%
%:%1176=392%:%
%:%1177=392%:%
%:%1178=392%:%
%:%1179=392%:%
%:%1180=393%:%
%:%1181=393%:%
%:%1186=393%:%
%:%1189=394%:%
%:%1190=395%:%
%:%1191=395%:%
%:%1192=396%:%
%:%1193=397%:%
%:%1194=398%:%
%:%1197=399%:%
%:%1201=399%:%
%:%1202=399%:%
%:%1203=399%:%
%:%1208=399%:%
%:%1211=400%:%
%:%1212=401%:%
%:%1213=401%:%
%:%1214=402%:%
%:%1215=403%:%
%:%1216=404%:%
%:%1219=405%:%
%:%1223=405%:%
%:%1224=405%:%
%:%1225=405%:%
%:%1230=405%:%
%:%1235=406%:%
%:%1240=407%:%

%
\begin{isabellebody}%
\setisabellecontext{Set{\isacharunderscore}{\kern0pt}Integral{\isacharunderscore}{\kern0pt}Addendum}%
%
\isadelimtheory
%
\endisadelimtheory
%
\isatagtheory
\isacommand{theory}\isamarkupfalse%
\ Set{\isacharunderscore}{\kern0pt}Integral{\isacharunderscore}{\kern0pt}Addendum\isanewline
\ \ \isakeyword{imports}\ {\isachardoublequoteopen}HOL{\isacharminus}{\kern0pt}Analysis{\isachardot}{\kern0pt}Set{\isacharunderscore}{\kern0pt}Integral{\isachardoublequoteclose}\ Bochner{\isacharunderscore}{\kern0pt}Integration{\isacharunderscore}{\kern0pt}Addendum\isanewline
\isakeyword{begin}%
\endisatagtheory
{\isafoldtheory}%
%
\isadelimtheory
%
\endisadelimtheory
%
\isadelimdocument
%
\endisadelimdocument
%
\isatagdocument
%
\isamarkupsection{Auxiliary Lemmas for Integrals on a Set%
}
\isamarkuptrue%
%
\endisatagdocument
{\isafolddocument}%
%
\isadelimdocument
%
\endisadelimdocument
\isacommand{lemma}\isamarkupfalse%
\ set{\isacharunderscore}{\kern0pt}integral{\isacharunderscore}{\kern0pt}scaleR{\isacharunderscore}{\kern0pt}left{\isacharcolon}{\kern0pt}\ \isanewline
\ \ \isakeyword{assumes}\ {\isachardoublequoteopen}A\ {\isasymin}\ sets\ M{\isachardoublequoteclose}\ {\isachardoublequoteopen}c\ {\isasymnoteq}\ {\isadigit{0}}\ {\isasymLongrightarrow}\ integrable\ M\ f{\isachardoublequoteclose}\isanewline
\ \ \isakeyword{shows}\ {\isachardoublequoteopen}LINT\ t{\isacharcolon}{\kern0pt}A{\isacharbar}{\kern0pt}M{\isachardot}{\kern0pt}\ f\ t\ {\isacharasterisk}{\kern0pt}\isactrlsub R\ c\ {\isacharequal}{\kern0pt}\ {\isacharparenleft}{\kern0pt}LINT\ t{\isacharcolon}{\kern0pt}A{\isacharbar}{\kern0pt}M{\isachardot}{\kern0pt}\ f\ t{\isacharparenright}{\kern0pt}\ {\isacharasterisk}{\kern0pt}\isactrlsub R\ c{\isachardoublequoteclose}\isanewline
%
\isadelimproof
\ \ %
\endisadelimproof
%
\isatagproof
\isacommand{unfolding}\isamarkupfalse%
\ set{\isacharunderscore}{\kern0pt}lebesgue{\isacharunderscore}{\kern0pt}integral{\isacharunderscore}{\kern0pt}def\ \isanewline
\ \ \isacommand{using}\isamarkupfalse%
\ integrable{\isacharunderscore}{\kern0pt}mult{\isacharunderscore}{\kern0pt}indicator{\isacharbrackleft}{\kern0pt}OF\ assms{\isacharbrackright}{\kern0pt}\isanewline
\ \ \isacommand{by}\isamarkupfalse%
\ {\isacharparenleft}{\kern0pt}subst\ integral{\isacharunderscore}{\kern0pt}scaleR{\isacharunderscore}{\kern0pt}left{\isacharbrackleft}{\kern0pt}symmetric{\isacharbrackright}{\kern0pt}{\isacharcomma}{\kern0pt}\ auto{\isacharparenright}{\kern0pt}%
\endisatagproof
{\isafoldproof}%
%
\isadelimproof
\isanewline
%
\endisadelimproof
\isanewline
\isacommand{lemma}\isamarkupfalse%
\ nn{\isacharunderscore}{\kern0pt}set{\isacharunderscore}{\kern0pt}integral{\isacharunderscore}{\kern0pt}eq{\isacharunderscore}{\kern0pt}set{\isacharunderscore}{\kern0pt}integral{\isacharcolon}{\kern0pt}\isanewline
\ \ \isakeyword{assumes}\ {\isacharbrackleft}{\kern0pt}measurable{\isacharbrackright}{\kern0pt}{\isacharcolon}{\kern0pt}{\isachardoublequoteopen}integrable\ M\ f{\isachardoublequoteclose}\isanewline
\ \ \ \ \ \ \isakeyword{and}\ {\isachardoublequoteopen}AE\ x\ {\isasymin}\ A\ in\ M{\isachardot}{\kern0pt}\ {\isadigit{0}}\ {\isasymle}\ f\ x{\isachardoublequoteclose}\ {\isachardoublequoteopen}A\ {\isasymin}\ sets\ M{\isachardoublequoteclose}\isanewline
\ \ \ \ \isakeyword{shows}\ {\isachardoublequoteopen}{\isacharparenleft}{\kern0pt}{\isasymintegral}\isactrlsup {\isacharplus}{\kern0pt}x{\isasymin}A{\isachardot}{\kern0pt}\ f\ x\ {\isasympartial}M{\isacharparenright}{\kern0pt}\ {\isacharequal}{\kern0pt}\ {\isacharparenleft}{\kern0pt}{\isasymintegral}\ x\ {\isasymin}\ A{\isachardot}{\kern0pt}\ f\ x\ {\isasympartial}M{\isacharparenright}{\kern0pt}{\isachardoublequoteclose}\isanewline
%
\isadelimproof
%
\endisadelimproof
%
\isatagproof
\isacommand{proof}\isamarkupfalse%
{\isacharminus}{\kern0pt}\isanewline
\ \ \isacommand{have}\isamarkupfalse%
\ {\isachardoublequoteopen}{\isacharparenleft}{\kern0pt}{\isasymintegral}\isactrlsup {\isacharplus}{\kern0pt}x{\isachardot}{\kern0pt}\ indicator\ A\ x\ {\isacharasterisk}{\kern0pt}\isactrlsub R\ f\ x\ {\isasympartial}M{\isacharparenright}{\kern0pt}\ {\isacharequal}{\kern0pt}\ {\isacharparenleft}{\kern0pt}{\isasymintegral}\ x\ {\isasymin}\ A{\isachardot}{\kern0pt}\ f\ x\ {\isasympartial}M{\isacharparenright}{\kern0pt}{\isachardoublequoteclose}\isanewline
\ \ \isacommand{unfolding}\isamarkupfalse%
\ set{\isacharunderscore}{\kern0pt}lebesgue{\isacharunderscore}{\kern0pt}integral{\isacharunderscore}{\kern0pt}def\ \isacommand{using}\isamarkupfalse%
\ assms{\isacharparenleft}{\kern0pt}{\isadigit{2}}{\isacharparenright}{\kern0pt}\ \isacommand{by}\isamarkupfalse%
\ {\isacharparenleft}{\kern0pt}intro\ nn{\isacharunderscore}{\kern0pt}integral{\isacharunderscore}{\kern0pt}eq{\isacharunderscore}{\kern0pt}integral{\isacharbrackleft}{\kern0pt}of\ {\isacharunderscore}{\kern0pt}\ {\isachardoublequoteopen}{\isasymlambda}x{\isachardot}{\kern0pt}\ indicat{\isacharunderscore}{\kern0pt}real\ A\ x\ {\isacharasterisk}{\kern0pt}\isactrlsub R\ f\ x{\isachardoublequoteclose}{\isacharbrackright}{\kern0pt}{\isacharcomma}{\kern0pt}\ blast\ intro{\isacharcolon}{\kern0pt}\ assms\ integrable{\isacharunderscore}{\kern0pt}mult{\isacharunderscore}{\kern0pt}indicator{\isacharcomma}{\kern0pt}\ fastforce{\isacharparenright}{\kern0pt}\isanewline
\ \ \isacommand{moreover}\isamarkupfalse%
\ \isacommand{have}\isamarkupfalse%
\ {\isachardoublequoteopen}{\isacharparenleft}{\kern0pt}{\isasymintegral}\isactrlsup {\isacharplus}{\kern0pt}x{\isachardot}{\kern0pt}\ indicator\ A\ x\ {\isacharasterisk}{\kern0pt}\isactrlsub R\ f\ x\ {\isasympartial}M{\isacharparenright}{\kern0pt}\ {\isacharequal}{\kern0pt}\ {\isacharparenleft}{\kern0pt}{\isasymintegral}\isactrlsup {\isacharplus}{\kern0pt}x{\isasymin}A{\isachardot}{\kern0pt}\ f\ x\ {\isasympartial}M{\isacharparenright}{\kern0pt}{\isachardoublequoteclose}\ \ \isacommand{by}\isamarkupfalse%
\ {\isacharparenleft}{\kern0pt}metis\ ennreal{\isacharunderscore}{\kern0pt}{\isadigit{0}}\ indicator{\isacharunderscore}{\kern0pt}simps{\isacharparenleft}{\kern0pt}{\isadigit{1}}{\isacharparenright}{\kern0pt}\ indicator{\isacharunderscore}{\kern0pt}simps{\isacharparenleft}{\kern0pt}{\isadigit{2}}{\isacharparenright}{\kern0pt}\ mult{\isachardot}{\kern0pt}commute\ mult{\isacharunderscore}{\kern0pt}{\isadigit{1}}\ mult{\isacharunderscore}{\kern0pt}zero{\isacharunderscore}{\kern0pt}left\ real{\isacharunderscore}{\kern0pt}scaleR{\isacharunderscore}{\kern0pt}def{\isacharparenright}{\kern0pt}\isanewline
\ \ \isacommand{ultimately}\isamarkupfalse%
\ \isacommand{show}\isamarkupfalse%
\ {\isacharquery}{\kern0pt}thesis\ \isacommand{by}\isamarkupfalse%
\ argo\isanewline
\isacommand{qed}\isamarkupfalse%
%
\endisatagproof
{\isafoldproof}%
%
\isadelimproof
\isanewline
%
\endisadelimproof
\isanewline
\isacommand{lemma}\isamarkupfalse%
\ set{\isacharunderscore}{\kern0pt}integral{\isacharunderscore}{\kern0pt}restrict{\isacharunderscore}{\kern0pt}space{\isacharcolon}{\kern0pt}\isanewline
\ \ \isakeyword{fixes}\ f\ {\isacharcolon}{\kern0pt}{\isacharcolon}{\kern0pt}\ {\isachardoublequoteopen}{\isacharprime}{\kern0pt}a\ {\isasymRightarrow}\ {\isacharprime}{\kern0pt}b{\isacharcolon}{\kern0pt}{\isacharcolon}{\kern0pt}{\isacharbraceleft}{\kern0pt}banach{\isacharcomma}{\kern0pt}\ second{\isacharunderscore}{\kern0pt}countable{\isacharunderscore}{\kern0pt}topology{\isacharbraceright}{\kern0pt}{\isachardoublequoteclose}\isanewline
\ \ \isakeyword{assumes}\ {\isachardoublequoteopen}{\isasymOmega}\ {\isasyminter}\ space\ M\ {\isasymin}\ sets\ M{\isachardoublequoteclose}\isanewline
\ \ \isakeyword{shows}\ {\isachardoublequoteopen}set{\isacharunderscore}{\kern0pt}lebesgue{\isacharunderscore}{\kern0pt}integral\ {\isacharparenleft}{\kern0pt}restrict{\isacharunderscore}{\kern0pt}space\ M\ {\isasymOmega}{\isacharparenright}{\kern0pt}\ A\ f\ {\isacharequal}{\kern0pt}\ set{\isacharunderscore}{\kern0pt}lebesgue{\isacharunderscore}{\kern0pt}integral\ M\ A\ {\isacharparenleft}{\kern0pt}{\isasymlambda}x{\isachardot}{\kern0pt}\ indicator\ {\isasymOmega}\ x\ {\isacharasterisk}{\kern0pt}\isactrlsub R\ f\ x{\isacharparenright}{\kern0pt}{\isachardoublequoteclose}\isanewline
%
\isadelimproof
\ \ %
\endisadelimproof
%
\isatagproof
\isacommand{unfolding}\isamarkupfalse%
\ set{\isacharunderscore}{\kern0pt}lebesgue{\isacharunderscore}{\kern0pt}integral{\isacharunderscore}{\kern0pt}def\ \isanewline
\ \ \isacommand{by}\isamarkupfalse%
\ {\isacharparenleft}{\kern0pt}subst\ integral{\isacharunderscore}{\kern0pt}restrict{\isacharunderscore}{\kern0pt}space{\isacharcomma}{\kern0pt}\ auto\ intro{\isacharbang}{\kern0pt}{\isacharcolon}{\kern0pt}\ integrable{\isacharunderscore}{\kern0pt}mult{\isacharunderscore}{\kern0pt}indicator\ assms\ simp{\isacharcolon}{\kern0pt}\ mult{\isachardot}{\kern0pt}commute{\isacharparenright}{\kern0pt}%
\endisatagproof
{\isafoldproof}%
%
\isadelimproof
\isanewline
%
\endisadelimproof
\isanewline
\isacommand{lemma}\isamarkupfalse%
\ set{\isacharunderscore}{\kern0pt}integral{\isacharunderscore}{\kern0pt}const{\isacharcolon}{\kern0pt}\isanewline
\ \ \isakeyword{fixes}\ c\ {\isacharcolon}{\kern0pt}{\isacharcolon}{\kern0pt}\ {\isachardoublequoteopen}{\isacharprime}{\kern0pt}b{\isacharcolon}{\kern0pt}{\isacharcolon}{\kern0pt}{\isacharbraceleft}{\kern0pt}banach{\isacharcomma}{\kern0pt}\ second{\isacharunderscore}{\kern0pt}countable{\isacharunderscore}{\kern0pt}topology{\isacharbraceright}{\kern0pt}{\isachardoublequoteclose}\isanewline
\ \ \isakeyword{assumes}\ {\isachardoublequoteopen}A\ {\isasymin}\ sets\ M{\isachardoublequoteclose}\ {\isachardoublequoteopen}emeasure\ M\ A\ {\isasymnoteq}\ {\isasyminfinity}{\isachardoublequoteclose}\isanewline
\ \ \isakeyword{shows}\ {\isachardoublequoteopen}set{\isacharunderscore}{\kern0pt}lebesgue{\isacharunderscore}{\kern0pt}integral\ M\ A\ {\isacharparenleft}{\kern0pt}{\isasymlambda}{\isacharunderscore}{\kern0pt}{\isachardot}{\kern0pt}\ c{\isacharparenright}{\kern0pt}\ {\isacharequal}{\kern0pt}\ measure\ M\ A\ {\isacharasterisk}{\kern0pt}\isactrlsub R\ c{\isachardoublequoteclose}\isanewline
%
\isadelimproof
\ \ %
\endisadelimproof
%
\isatagproof
\isacommand{unfolding}\isamarkupfalse%
\ set{\isacharunderscore}{\kern0pt}lebesgue{\isacharunderscore}{\kern0pt}integral{\isacharunderscore}{\kern0pt}def\ \isanewline
\ \ \isacommand{using}\isamarkupfalse%
\ assms\ \isacommand{by}\isamarkupfalse%
\ {\isacharparenleft}{\kern0pt}metis\ has{\isacharunderscore}{\kern0pt}bochner{\isacharunderscore}{\kern0pt}integral{\isacharunderscore}{\kern0pt}indicator\ has{\isacharunderscore}{\kern0pt}bochner{\isacharunderscore}{\kern0pt}integral{\isacharunderscore}{\kern0pt}integral{\isacharunderscore}{\kern0pt}eq\ infinity{\isacharunderscore}{\kern0pt}ennreal{\isacharunderscore}{\kern0pt}def\ less{\isacharunderscore}{\kern0pt}top{\isacharparenright}{\kern0pt}%
\endisatagproof
{\isafoldproof}%
%
\isadelimproof
\isanewline
%
\endisadelimproof
\isanewline
\isacommand{lemma}\isamarkupfalse%
\ set{\isacharunderscore}{\kern0pt}integral{\isacharunderscore}{\kern0pt}mono{\isacharunderscore}{\kern0pt}banach{\isacharcolon}{\kern0pt}\isanewline
\ \ \isakeyword{fixes}\ f\ g\ {\isacharcolon}{\kern0pt}{\isacharcolon}{\kern0pt}\ {\isachardoublequoteopen}{\isacharprime}{\kern0pt}a\ {\isasymRightarrow}\ {\isacharprime}{\kern0pt}b\ {\isacharcolon}{\kern0pt}{\isacharcolon}{\kern0pt}\ {\isacharbraceleft}{\kern0pt}second{\isacharunderscore}{\kern0pt}countable{\isacharunderscore}{\kern0pt}topology{\isacharcomma}{\kern0pt}\ banach{\isacharcomma}{\kern0pt}\ linorder{\isacharunderscore}{\kern0pt}topology{\isacharcomma}{\kern0pt}\ ordered{\isacharunderscore}{\kern0pt}real{\isacharunderscore}{\kern0pt}vector{\isacharbraceright}{\kern0pt}{\isachardoublequoteclose}\isanewline
\ \ \isakeyword{assumes}\ {\isachardoublequoteopen}set{\isacharunderscore}{\kern0pt}integrable\ M\ A\ f{\isachardoublequoteclose}\ {\isachardoublequoteopen}set{\isacharunderscore}{\kern0pt}integrable\ M\ A\ g{\isachardoublequoteclose}\isanewline
\ \ \ \ {\isachardoublequoteopen}{\isasymAnd}x{\isachardot}{\kern0pt}\ x\ {\isasymin}\ A\ {\isasymLongrightarrow}\ f\ x\ {\isasymle}\ g\ x{\isachardoublequoteclose}\isanewline
\ \ \isakeyword{shows}\ {\isachardoublequoteopen}{\isacharparenleft}{\kern0pt}LINT\ x{\isacharcolon}{\kern0pt}A{\isacharbar}{\kern0pt}M{\isachardot}{\kern0pt}\ f\ x{\isacharparenright}{\kern0pt}\ {\isasymle}\ {\isacharparenleft}{\kern0pt}LINT\ x{\isacharcolon}{\kern0pt}A{\isacharbar}{\kern0pt}M{\isachardot}{\kern0pt}\ g\ x{\isacharparenright}{\kern0pt}{\isachardoublequoteclose}\isanewline
%
\isadelimproof
\ \ %
\endisadelimproof
%
\isatagproof
\isacommand{using}\isamarkupfalse%
\ assms\ \isacommand{unfolding}\isamarkupfalse%
\ set{\isacharunderscore}{\kern0pt}integrable{\isacharunderscore}{\kern0pt}def\ set{\isacharunderscore}{\kern0pt}lebesgue{\isacharunderscore}{\kern0pt}integral{\isacharunderscore}{\kern0pt}def\isanewline
\ \ \isacommand{by}\isamarkupfalse%
\ {\isacharparenleft}{\kern0pt}auto\ intro{\isacharcolon}{\kern0pt}\ integral{\isacharunderscore}{\kern0pt}mono{\isacharunderscore}{\kern0pt}banach\ split{\isacharcolon}{\kern0pt}\ split{\isacharunderscore}{\kern0pt}indicator{\isacharparenright}{\kern0pt}%
\endisatagproof
{\isafoldproof}%
%
\isadelimproof
\isanewline
%
\endisadelimproof
\isanewline
\isacommand{lemma}\isamarkupfalse%
\ set{\isacharunderscore}{\kern0pt}integral{\isacharunderscore}{\kern0pt}mono{\isacharunderscore}{\kern0pt}AE{\isacharunderscore}{\kern0pt}banach{\isacharcolon}{\kern0pt}\isanewline
\ \ \isakeyword{fixes}\ f\ g\ {\isacharcolon}{\kern0pt}{\isacharcolon}{\kern0pt}\ {\isachardoublequoteopen}{\isacharprime}{\kern0pt}a\ {\isasymRightarrow}\ {\isacharprime}{\kern0pt}b\ {\isacharcolon}{\kern0pt}{\isacharcolon}{\kern0pt}\ {\isacharbraceleft}{\kern0pt}second{\isacharunderscore}{\kern0pt}countable{\isacharunderscore}{\kern0pt}topology{\isacharcomma}{\kern0pt}\ banach{\isacharcomma}{\kern0pt}\ linorder{\isacharunderscore}{\kern0pt}topology{\isacharcomma}{\kern0pt}\ ordered{\isacharunderscore}{\kern0pt}real{\isacharunderscore}{\kern0pt}vector{\isacharbraceright}{\kern0pt}{\isachardoublequoteclose}\isanewline
\ \ \isakeyword{assumes}\ {\isachardoublequoteopen}set{\isacharunderscore}{\kern0pt}integrable\ M\ A\ f{\isachardoublequoteclose}\ {\isachardoublequoteopen}set{\isacharunderscore}{\kern0pt}integrable\ M\ A\ g{\isachardoublequoteclose}\ {\isachardoublequoteopen}AE\ x{\isasymin}A\ in\ M{\isachardot}{\kern0pt}\ f\ x\ {\isasymle}\ g\ x{\isachardoublequoteclose}\isanewline
\ \ \isakeyword{shows}\ {\isachardoublequoteopen}set{\isacharunderscore}{\kern0pt}lebesgue{\isacharunderscore}{\kern0pt}integral\ M\ A\ f\ {\isasymle}\ set{\isacharunderscore}{\kern0pt}lebesgue{\isacharunderscore}{\kern0pt}integral\ M\ A\ g{\isachardoublequoteclose}%
\isadelimproof
\ %
\endisadelimproof
%
\isatagproof
\isacommand{using}\isamarkupfalse%
\ assms\ \isacommand{unfolding}\isamarkupfalse%
\ set{\isacharunderscore}{\kern0pt}lebesgue{\isacharunderscore}{\kern0pt}integral{\isacharunderscore}{\kern0pt}def\ \isacommand{by}\isamarkupfalse%
\ {\isacharparenleft}{\kern0pt}auto\ simp\ add{\isacharcolon}{\kern0pt}\ set{\isacharunderscore}{\kern0pt}integrable{\isacharunderscore}{\kern0pt}def\ intro{\isacharbang}{\kern0pt}{\isacharcolon}{\kern0pt}\ integral{\isacharunderscore}{\kern0pt}mono{\isacharunderscore}{\kern0pt}AE{\isacharunderscore}{\kern0pt}banach{\isacharbrackleft}{\kern0pt}of\ M\ {\isachardoublequoteopen}{\isasymlambda}x{\isachardot}{\kern0pt}\ indicator\ A\ x\ {\isacharasterisk}{\kern0pt}\isactrlsub R\ f\ x{\isachardoublequoteclose}\ {\isachardoublequoteopen}{\isasymlambda}x{\isachardot}{\kern0pt}\ indicator\ A\ x\ {\isacharasterisk}{\kern0pt}\isactrlsub R\ g\ x{\isachardoublequoteclose}{\isacharbrackright}{\kern0pt}{\isacharcomma}{\kern0pt}\ simp\ add{\isacharcolon}{\kern0pt}\ indicator{\isacharunderscore}{\kern0pt}def{\isacharparenright}{\kern0pt}%
\endisatagproof
{\isafoldproof}%
%
\isadelimproof
%
\endisadelimproof
\isanewline
%
\isadelimtheory
\isanewline
%
\endisadelimtheory
%
\isatagtheory
\isacommand{end}\isamarkupfalse%
%
\endisatagtheory
{\isafoldtheory}%
%
\isadelimtheory
%
\endisadelimtheory
%
\end{isabellebody}%
\endinput
%:%file=Set_Integral_Addendum.tex%:%
%:%10=1%:%
%:%11=1%:%
%:%12=2%:%
%:%13=3%:%
%:%27=5%:%
%:%37=7%:%
%:%38=7%:%
%:%39=8%:%
%:%40=9%:%
%:%43=10%:%
%:%47=10%:%
%:%48=10%:%
%:%49=11%:%
%:%50=11%:%
%:%51=12%:%
%:%52=12%:%
%:%57=12%:%
%:%60=13%:%
%:%61=14%:%
%:%62=14%:%
%:%63=15%:%
%:%64=16%:%
%:%65=17%:%
%:%72=18%:%
%:%73=18%:%
%:%74=19%:%
%:%75=19%:%
%:%76=20%:%
%:%77=20%:%
%:%78=20%:%
%:%79=20%:%
%:%80=21%:%
%:%81=21%:%
%:%82=21%:%
%:%83=21%:%
%:%84=22%:%
%:%85=22%:%
%:%86=22%:%
%:%87=22%:%
%:%88=23%:%
%:%94=23%:%
%:%97=24%:%
%:%98=25%:%
%:%99=25%:%
%:%100=26%:%
%:%101=27%:%
%:%102=28%:%
%:%105=29%:%
%:%109=29%:%
%:%110=29%:%
%:%111=30%:%
%:%112=30%:%
%:%117=30%:%
%:%120=31%:%
%:%121=32%:%
%:%122=32%:%
%:%123=33%:%
%:%124=34%:%
%:%125=35%:%
%:%128=36%:%
%:%132=36%:%
%:%133=36%:%
%:%134=37%:%
%:%135=37%:%
%:%136=37%:%
%:%141=37%:%
%:%144=38%:%
%:%145=39%:%
%:%146=39%:%
%:%147=40%:%
%:%148=41%:%
%:%149=42%:%
%:%150=43%:%
%:%153=44%:%
%:%157=44%:%
%:%158=44%:%
%:%159=44%:%
%:%160=45%:%
%:%161=45%:%
%:%166=45%:%
%:%169=46%:%
%:%170=47%:%
%:%171=47%:%
%:%172=48%:%
%:%173=49%:%
%:%174=50%:%
%:%176=50%:%
%:%180=50%:%
%:%181=50%:%
%:%182=50%:%
%:%183=50%:%
%:%190=50%:%
%:%193=51%:%
%:%198=52%:%

%
\begin{isabellebody}%
\setisabellecontext{Sigma{\isacharunderscore}{\kern0pt}Finite{\isacharunderscore}{\kern0pt}Measure{\isacharunderscore}{\kern0pt}Addendum}%
%
\isadelimtheory
%
\endisadelimtheory
%
\isatagtheory
\isacommand{theory}\isamarkupfalse%
\ Sigma{\isacharunderscore}{\kern0pt}Finite{\isacharunderscore}{\kern0pt}Measure{\isacharunderscore}{\kern0pt}Addendum\isanewline
\isakeyword{imports}\ Set{\isacharunderscore}{\kern0pt}Integral{\isacharunderscore}{\kern0pt}Addendum\isanewline
\isakeyword{begin}%
\endisatagtheory
{\isafoldtheory}%
%
\isadelimtheory
%
\endisadelimtheory
%
\isadelimdocument
%
\endisadelimdocument
%
\isatagdocument
%
\isamarkupsection{Averaging Theorem%
}
\isamarkuptrue%
%
\endisatagdocument
{\isafolddocument}%
%
\isadelimdocument
%
\endisadelimdocument
\isacommand{lemma}\isamarkupfalse%
\ balls{\isacharunderscore}{\kern0pt}countable{\isacharunderscore}{\kern0pt}basis{\isacharcolon}{\kern0pt}\isanewline
\ \ \isakeyword{obtains}\ D\ {\isacharcolon}{\kern0pt}{\isacharcolon}{\kern0pt}\ {\isachardoublequoteopen}{\isacharprime}{\kern0pt}a\ {\isacharcolon}{\kern0pt}{\isacharcolon}{\kern0pt}\ {\isacharbraceleft}{\kern0pt}metric{\isacharunderscore}{\kern0pt}space{\isacharcomma}{\kern0pt}\ second{\isacharunderscore}{\kern0pt}countable{\isacharunderscore}{\kern0pt}topology{\isacharbraceright}{\kern0pt}\ set{\isachardoublequoteclose}\ \isanewline
\ \ \isakeyword{where}\ {\isachardoublequoteopen}topological{\isacharunderscore}{\kern0pt}basis\ {\isacharparenleft}{\kern0pt}case{\isacharunderscore}{\kern0pt}prod\ ball\ {\isacharbackquote}{\kern0pt}\ {\isacharparenleft}{\kern0pt}D\ {\isasymtimes}\ {\isacharparenleft}{\kern0pt}{\isasymrat}\ {\isasyminter}\ {\isacharbraceleft}{\kern0pt}{\isadigit{0}}{\isacharless}{\kern0pt}{\isachardot}{\kern0pt}{\isachardot}{\kern0pt}{\isacharbraceright}{\kern0pt}{\isacharparenright}{\kern0pt}{\isacharparenright}{\kern0pt}{\isacharparenright}{\kern0pt}{\isachardoublequoteclose}\isanewline
\ \ \ \ \isakeyword{and}\ {\isachardoublequoteopen}countable\ D{\isachardoublequoteclose}\isanewline
\ \ \ \ \isakeyword{and}\ {\isachardoublequoteopen}D\ {\isasymnoteq}\ {\isacharbraceleft}{\kern0pt}{\isacharbraceright}{\kern0pt}{\isachardoublequoteclose}\ \ \ \ \isanewline
%
\isadelimproof
%
\endisadelimproof
%
\isatagproof
\isacommand{proof}\isamarkupfalse%
\ {\isacharminus}{\kern0pt}\isanewline
\ \ \isacommand{obtain}\isamarkupfalse%
\ D\ {\isacharcolon}{\kern0pt}{\isacharcolon}{\kern0pt}\ {\isachardoublequoteopen}{\isacharprime}{\kern0pt}a\ set{\isachardoublequoteclose}\ \isakeyword{where}\ dense{\isacharunderscore}{\kern0pt}subset{\isacharcolon}{\kern0pt}\ {\isachardoublequoteopen}countable\ D{\isachardoublequoteclose}\ {\isachardoublequoteopen}D\ {\isasymnoteq}\ {\isacharbraceleft}{\kern0pt}{\isacharbraceright}{\kern0pt}{\isachardoublequoteclose}\ {\isachardoublequoteopen}{\isasymlbrakk}open\ U{\isacharsemicolon}{\kern0pt}\ U\ {\isasymnoteq}\ {\isacharbraceleft}{\kern0pt}{\isacharbraceright}{\kern0pt}{\isasymrbrakk}\ {\isasymLongrightarrow}\ {\isasymexists}y\ {\isasymin}\ D{\isachardot}{\kern0pt}\ y\ {\isasymin}\ U{\isachardoublequoteclose}\ \isakeyword{for}\ U\ \isacommand{using}\isamarkupfalse%
\ countable{\isacharunderscore}{\kern0pt}dense{\isacharunderscore}{\kern0pt}exists\ \isacommand{by}\isamarkupfalse%
\ blast\isanewline
\ \ \isacommand{have}\isamarkupfalse%
\ {\isachardoublequoteopen}topological{\isacharunderscore}{\kern0pt}basis\ {\isacharparenleft}{\kern0pt}case{\isacharunderscore}{\kern0pt}prod\ ball\ {\isacharbackquote}{\kern0pt}\ {\isacharparenleft}{\kern0pt}D\ {\isasymtimes}\ {\isacharparenleft}{\kern0pt}{\isasymrat}\ {\isasyminter}\ {\isacharbraceleft}{\kern0pt}{\isadigit{0}}{\isacharless}{\kern0pt}{\isachardot}{\kern0pt}{\isachardot}{\kern0pt}{\isacharbraceright}{\kern0pt}{\isacharparenright}{\kern0pt}{\isacharparenright}{\kern0pt}{\isacharparenright}{\kern0pt}{\isachardoublequoteclose}\isanewline
\ \ \isacommand{proof}\isamarkupfalse%
\ {\isacharparenleft}{\kern0pt}intro\ topological{\isacharunderscore}{\kern0pt}basis{\isacharunderscore}{\kern0pt}iff{\isacharbrackleft}{\kern0pt}THEN\ iffD{\isadigit{2}}{\isacharbrackright}{\kern0pt}{\isacharcomma}{\kern0pt}\ fast{\isacharcomma}{\kern0pt}\ clarify{\isacharparenright}{\kern0pt}\isanewline
\ \ \ \ \isacommand{fix}\isamarkupfalse%
\ U\ \isakeyword{and}\ x\ {\isacharcolon}{\kern0pt}{\isacharcolon}{\kern0pt}\ {\isacharprime}{\kern0pt}a\ \isacommand{assume}\isamarkupfalse%
\ asm{\isacharcolon}{\kern0pt}\ {\isachardoublequoteopen}open\ U{\isachardoublequoteclose}\ {\isachardoublequoteopen}x\ {\isasymin}\ U{\isachardoublequoteclose}\isanewline
\ \ \ \ \isacommand{obtain}\isamarkupfalse%
\ e\ \isakeyword{where}\ e{\isacharcolon}{\kern0pt}\ {\isachardoublequoteopen}e\ {\isachargreater}{\kern0pt}\ {\isadigit{0}}{\isachardoublequoteclose}\ {\isachardoublequoteopen}ball\ x\ e\ {\isasymsubseteq}\ U{\isachardoublequoteclose}\ \isacommand{using}\isamarkupfalse%
\ asm\ openE\ \isacommand{by}\isamarkupfalse%
\ blast\isanewline
\ \ \ \ \isacommand{obtain}\isamarkupfalse%
\ y\ \isakeyword{where}\ y{\isacharcolon}{\kern0pt}\ {\isachardoublequoteopen}y\ {\isasymin}\ D{\isachardoublequoteclose}\ {\isachardoublequoteopen}y\ {\isasymin}\ ball\ x\ {\isacharparenleft}{\kern0pt}e\ {\isacharslash}{\kern0pt}\ {\isadigit{3}}{\isacharparenright}{\kern0pt}{\isachardoublequoteclose}\ \isacommand{using}\isamarkupfalse%
\ dense{\isacharunderscore}{\kern0pt}subset{\isacharparenleft}{\kern0pt}{\isadigit{3}}{\isacharparenright}{\kern0pt}{\isacharbrackleft}{\kern0pt}OF\ open{\isacharunderscore}{\kern0pt}ball{\isacharcomma}{\kern0pt}\ of\ x\ {\isachardoublequoteopen}e\ {\isacharslash}{\kern0pt}\ {\isadigit{3}}{\isachardoublequoteclose}{\isacharbrackright}{\kern0pt}\ centre{\isacharunderscore}{\kern0pt}in{\isacharunderscore}{\kern0pt}ball{\isacharbrackleft}{\kern0pt}THEN\ iffD{\isadigit{2}}{\isacharcomma}{\kern0pt}\ OF\ divide{\isacharunderscore}{\kern0pt}pos{\isacharunderscore}{\kern0pt}pos{\isacharbrackleft}{\kern0pt}OF\ e{\isacharparenleft}{\kern0pt}{\isadigit{1}}{\isacharparenright}{\kern0pt}{\isacharcomma}{\kern0pt}\ of\ {\isadigit{3}}{\isacharbrackright}{\kern0pt}{\isacharbrackright}{\kern0pt}\ \isacommand{by}\isamarkupfalse%
\ force\isanewline
\ \ \ \ \isacommand{obtain}\isamarkupfalse%
\ r\ \isakeyword{where}\ r{\isacharcolon}{\kern0pt}\ {\isachardoublequoteopen}r\ {\isasymin}\ {\isasymrat}\ {\isasyminter}\ {\isacharbraceleft}{\kern0pt}e{\isacharslash}{\kern0pt}{\isadigit{3}}{\isacharless}{\kern0pt}{\isachardot}{\kern0pt}{\isachardot}{\kern0pt}{\isacharless}{\kern0pt}e{\isacharslash}{\kern0pt}{\isadigit{2}}{\isacharbraceright}{\kern0pt}{\isachardoublequoteclose}\ \isacommand{unfolding}\isamarkupfalse%
\ Rats{\isacharunderscore}{\kern0pt}def\ \isacommand{using}\isamarkupfalse%
\ of{\isacharunderscore}{\kern0pt}rat{\isacharunderscore}{\kern0pt}dense{\isacharbrackleft}{\kern0pt}OF\ divide{\isacharunderscore}{\kern0pt}strict{\isacharunderscore}{\kern0pt}left{\isacharunderscore}{\kern0pt}mono{\isacharbrackleft}{\kern0pt}OF\ {\isacharunderscore}{\kern0pt}\ e{\isacharparenleft}{\kern0pt}{\isadigit{1}}{\isacharparenright}{\kern0pt}{\isacharbrackright}{\kern0pt}{\isacharcomma}{\kern0pt}\ of\ {\isadigit{2}}\ {\isadigit{3}}{\isacharbrackright}{\kern0pt}\ \isacommand{by}\isamarkupfalse%
\ auto\isanewline
\isanewline
\ \ \ \ \isacommand{have}\isamarkupfalse%
\ {\isacharasterisk}{\kern0pt}{\isacharcolon}{\kern0pt}\ {\isachardoublequoteopen}x\ {\isasymin}\ ball\ y\ r{\isachardoublequoteclose}\ \isacommand{using}\isamarkupfalse%
\ r\ y\ \isacommand{by}\isamarkupfalse%
\ {\isacharparenleft}{\kern0pt}simp\ add{\isacharcolon}{\kern0pt}\ dist{\isacharunderscore}{\kern0pt}commute{\isacharparenright}{\kern0pt}\isanewline
\ \ \ \ \isacommand{hence}\isamarkupfalse%
\ {\isachardoublequoteopen}ball\ y\ r\ {\isasymsubseteq}\ U{\isachardoublequoteclose}\ \isacommand{using}\isamarkupfalse%
\ r\ \isacommand{by}\isamarkupfalse%
\ {\isacharparenleft}{\kern0pt}intro\ order{\isacharunderscore}{\kern0pt}trans{\isacharbrackleft}{\kern0pt}OF\ {\isacharunderscore}{\kern0pt}\ e{\isacharparenleft}{\kern0pt}{\isadigit{2}}{\isacharparenright}{\kern0pt}{\isacharbrackright}{\kern0pt}{\isacharcomma}{\kern0pt}\ simp{\isacharcomma}{\kern0pt}\ metric{\isacharparenright}{\kern0pt}\isanewline
\ \ \ \ \isacommand{moreover}\isamarkupfalse%
\ \isacommand{have}\isamarkupfalse%
\ {\isachardoublequoteopen}ball\ y\ r\ {\isasymin}\ {\isacharparenleft}{\kern0pt}case{\isacharunderscore}{\kern0pt}prod\ ball\ {\isacharbackquote}{\kern0pt}\ {\isacharparenleft}{\kern0pt}D\ {\isasymtimes}\ {\isacharparenleft}{\kern0pt}{\isasymrat}\ {\isasyminter}\ {\isacharbraceleft}{\kern0pt}{\isadigit{0}}{\isacharless}{\kern0pt}{\isachardot}{\kern0pt}{\isachardot}{\kern0pt}{\isacharbraceright}{\kern0pt}{\isacharparenright}{\kern0pt}{\isacharparenright}{\kern0pt}{\isacharparenright}{\kern0pt}{\isachardoublequoteclose}\ \isacommand{using}\isamarkupfalse%
\ y{\isacharparenleft}{\kern0pt}{\isadigit{1}}{\isacharparenright}{\kern0pt}\ r\ \isacommand{by}\isamarkupfalse%
\ force\isanewline
\ \ \ \ \isacommand{ultimately}\isamarkupfalse%
\ \isacommand{show}\isamarkupfalse%
\ {\isachardoublequoteopen}{\isasymexists}B{\isacharprime}{\kern0pt}{\isasymin}{\isacharparenleft}{\kern0pt}case{\isacharunderscore}{\kern0pt}prod\ ball\ {\isacharbackquote}{\kern0pt}\ {\isacharparenleft}{\kern0pt}D\ {\isasymtimes}\ {\isacharparenleft}{\kern0pt}{\isasymrat}\ {\isasyminter}\ {\isacharbraceleft}{\kern0pt}{\isadigit{0}}{\isacharless}{\kern0pt}{\isachardot}{\kern0pt}{\isachardot}{\kern0pt}{\isacharbraceright}{\kern0pt}{\isacharparenright}{\kern0pt}{\isacharparenright}{\kern0pt}{\isacharparenright}{\kern0pt}{\isachardot}{\kern0pt}\ x\ {\isasymin}\ B{\isacharprime}{\kern0pt}\ {\isasymand}\ B{\isacharprime}{\kern0pt}\ {\isasymsubseteq}\ U{\isachardoublequoteclose}\ \isacommand{using}\isamarkupfalse%
\ {\isacharasterisk}{\kern0pt}\ \isacommand{by}\isamarkupfalse%
\ meson\isanewline
\ \ \isacommand{qed}\isamarkupfalse%
\isanewline
\ \ \isacommand{thus}\isamarkupfalse%
\ {\isacharquery}{\kern0pt}thesis\ \isacommand{using}\isamarkupfalse%
\ that\ dense{\isacharunderscore}{\kern0pt}subset\ \isacommand{by}\isamarkupfalse%
\ blast\isanewline
\isacommand{qed}\isamarkupfalse%
%
\endisatagproof
{\isafoldproof}%
%
\isadelimproof
\isanewline
%
\endisadelimproof
\isanewline
\isacommand{context}\isamarkupfalse%
\ sigma{\isacharunderscore}{\kern0pt}finite{\isacharunderscore}{\kern0pt}measure\isanewline
\isakeyword{begin}\ \ \ \ \ \ \ \ \ \isanewline
\isanewline
\isacommand{lemma}\isamarkupfalse%
\ sigma{\isacharunderscore}{\kern0pt}finite{\isacharunderscore}{\kern0pt}measure{\isacharunderscore}{\kern0pt}induct{\isacharbrackleft}{\kern0pt}case{\isacharunderscore}{\kern0pt}names\ finite{\isacharunderscore}{\kern0pt}measure{\isacharcomma}{\kern0pt}\ consumes\ {\isadigit{0}}{\isacharbrackright}{\kern0pt}{\isacharcolon}{\kern0pt}\isanewline
\ \ \isakeyword{assumes}\ {\isachardoublequoteopen}{\isasymAnd}{\isacharparenleft}{\kern0pt}N\ {\isacharcolon}{\kern0pt}{\isacharcolon}{\kern0pt}\ {\isacharprime}{\kern0pt}a\ measure{\isacharparenright}{\kern0pt}\ {\isasymOmega}{\isachardot}{\kern0pt}\ finite{\isacharunderscore}{\kern0pt}measure\ N\ \isanewline
\ \ \ \ \ \ \ \ \ \ \ \ \ \ \ \ \ \ \ \ \ \ \ \ \ \ \ \ \ \ {\isasymLongrightarrow}\ N\ {\isacharequal}{\kern0pt}\ restrict{\isacharunderscore}{\kern0pt}space\ M\ {\isasymOmega}\isanewline
\ \ \ \ \ \ \ \ \ \ \ \ \ \ \ \ \ \ \ \ \ \ \ \ \ \ \ \ \ \ {\isasymLongrightarrow}\ {\isasymOmega}\ {\isasymin}\ sets\ M\ \isanewline
\ \ \ \ \ \ \ \ \ \ \ \ \ \ \ \ \ \ \ \ \ \ \ \ \ \ \ \ \ \ {\isasymLongrightarrow}\ emeasure\ N\ {\isasymOmega}\ {\isasymnoteq}\ {\isasyminfinity}\ \isanewline
\ \ \ \ \ \ \ \ \ \ \ \ \ \ \ \ \ \ \ \ \ \ \ \ \ \ \ \ \ \ {\isasymLongrightarrow}\ emeasure\ N\ {\isasymOmega}\ {\isasymnoteq}\ {\isadigit{0}}\ \isanewline
\ \ \ \ \ \ \ \ \ \ \ \ \ \ \ \ \ \ \ \ \ \ \ \ \ \ \ \ \ \ {\isasymLongrightarrow}\ almost{\isacharunderscore}{\kern0pt}everywhere\ N\ Q{\isachardoublequoteclose}\isanewline
\ \ \ \ \ \ \isakeyword{and}\ {\isacharbrackleft}{\kern0pt}measurable{\isacharbrackright}{\kern0pt}{\isacharcolon}{\kern0pt}\ {\isachardoublequoteopen}Measurable{\isachardot}{\kern0pt}pred\ M\ Q{\isachardoublequoteclose}\isanewline
\ \ \isakeyword{shows}\ {\isachardoublequoteopen}almost{\isacharunderscore}{\kern0pt}everywhere\ M\ Q{\isachardoublequoteclose}\isanewline
%
\isadelimproof
%
\endisadelimproof
%
\isatagproof
\isacommand{proof}\isamarkupfalse%
\ {\isacharminus}{\kern0pt}\isanewline
\ \ \isacommand{have}\isamarkupfalse%
\ {\isacharasterisk}{\kern0pt}{\isacharcolon}{\kern0pt}\ {\isachardoublequoteopen}almost{\isacharunderscore}{\kern0pt}everywhere\ N\ Q{\isachardoublequoteclose}\ \isakeyword{if}\ {\isachardoublequoteopen}finite{\isacharunderscore}{\kern0pt}measure\ N{\isachardoublequoteclose}\ {\isachardoublequoteopen}N\ {\isacharequal}{\kern0pt}\ restrict{\isacharunderscore}{\kern0pt}space\ M\ {\isasymOmega}{\isachardoublequoteclose}\ {\isachardoublequoteopen}{\isasymOmega}\ {\isasymin}\ sets\ M{\isachardoublequoteclose}\ {\isachardoublequoteopen}emeasure\ N\ {\isasymOmega}\ {\isasymnoteq}\ {\isasyminfinity}{\isachardoublequoteclose}\ \isakeyword{for}\ N\ {\isasymOmega}\ \isacommand{using}\isamarkupfalse%
\ that\ \isacommand{by}\isamarkupfalse%
\ {\isacharparenleft}{\kern0pt}cases\ {\isachardoublequoteopen}emeasure\ N\ {\isasymOmega}\ {\isacharequal}{\kern0pt}\ {\isadigit{0}}{\isachardoublequoteclose}{\isacharcomma}{\kern0pt}\ auto\ intro{\isacharcolon}{\kern0pt}\ emeasure{\isacharunderscore}{\kern0pt}{\isadigit{0}}{\isacharunderscore}{\kern0pt}AE\ assms{\isacharparenleft}{\kern0pt}{\isadigit{1}}{\isacharparenright}{\kern0pt}{\isacharparenright}{\kern0pt}\isanewline
\isanewline
\ \ \isacommand{obtain}\isamarkupfalse%
\ A\ {\isacharcolon}{\kern0pt}{\isacharcolon}{\kern0pt}\ {\isachardoublequoteopen}nat\ {\isasymRightarrow}\ {\isacharprime}{\kern0pt}a\ set{\isachardoublequoteclose}\ \isakeyword{where}\ A{\isacharcolon}{\kern0pt}\ {\isachardoublequoteopen}range\ A\ {\isasymsubseteq}\ sets\ M{\isachardoublequoteclose}\ {\isachardoublequoteopen}{\isacharparenleft}{\kern0pt}{\isasymUnion}i{\isachardot}{\kern0pt}\ A\ i{\isacharparenright}{\kern0pt}\ {\isacharequal}{\kern0pt}\ space\ M{\isachardoublequoteclose}\ \isakeyword{and}\ emeasure{\isacharunderscore}{\kern0pt}finite{\isacharcolon}{\kern0pt}\ {\isachardoublequoteopen}emeasure\ M\ {\isacharparenleft}{\kern0pt}A\ i{\isacharparenright}{\kern0pt}\ {\isasymnoteq}\ {\isasyminfinity}{\isachardoublequoteclose}\ \isakeyword{for}\ i\ \isacommand{using}\isamarkupfalse%
\ sigma{\isacharunderscore}{\kern0pt}finite\ \isacommand{by}\isamarkupfalse%
\ metis\isanewline
\ \ \isacommand{note}\isamarkupfalse%
\ A{\isacharparenleft}{\kern0pt}{\isadigit{1}}{\isacharparenright}{\kern0pt}{\isacharbrackleft}{\kern0pt}measurable{\isacharbrackright}{\kern0pt}\isanewline
\ \ \isacommand{have}\isamarkupfalse%
\ space{\isacharunderscore}{\kern0pt}restr{\isacharcolon}{\kern0pt}\ {\isachardoublequoteopen}space\ {\isacharparenleft}{\kern0pt}restrict{\isacharunderscore}{\kern0pt}space\ M\ {\isacharparenleft}{\kern0pt}A\ i{\isacharparenright}{\kern0pt}{\isacharparenright}{\kern0pt}\ {\isacharequal}{\kern0pt}\ A\ i{\isachardoublequoteclose}\ \isakeyword{for}\ i\ \isacommand{unfolding}\isamarkupfalse%
\ space{\isacharunderscore}{\kern0pt}restrict{\isacharunderscore}{\kern0pt}space\ \isacommand{by}\isamarkupfalse%
\ simp\isanewline
\ \ \isacommand{{\isacharbraceleft}{\kern0pt}}\isamarkupfalse%
\isanewline
\ \ \ \ \isacommand{fix}\isamarkupfalse%
\ i\ \ \ \ \isanewline
\ \ \ \ \isacommand{have}\isamarkupfalse%
\ {\isacharasterisk}{\kern0pt}{\isacharcolon}{\kern0pt}\ {\isachardoublequoteopen}{\isacharbraceleft}{\kern0pt}x\ {\isasymin}\ A\ i\ {\isasyminter}\ space\ M{\isachardot}{\kern0pt}\ Q\ x{\isacharbraceright}{\kern0pt}\ {\isacharequal}{\kern0pt}\ {\isacharbraceleft}{\kern0pt}x\ {\isasymin}\ space\ M{\isachardot}{\kern0pt}\ Q\ x{\isacharbraceright}{\kern0pt}\ {\isasyminter}\ {\isacharparenleft}{\kern0pt}A\ i{\isacharparenright}{\kern0pt}{\isachardoublequoteclose}\ \isacommand{by}\isamarkupfalse%
\ fast\isanewline
\ \ \ \ \isacommand{have}\isamarkupfalse%
\ {\isachardoublequoteopen}Measurable{\isachardot}{\kern0pt}pred\ {\isacharparenleft}{\kern0pt}restrict{\isacharunderscore}{\kern0pt}space\ M\ {\isacharparenleft}{\kern0pt}A\ i{\isacharparenright}{\kern0pt}{\isacharparenright}{\kern0pt}\ Q{\isachardoublequoteclose}\ \isacommand{using}\isamarkupfalse%
\ A\ \isacommand{by}\isamarkupfalse%
\ {\isacharparenleft}{\kern0pt}intro\ measurableI{\isacharcomma}{\kern0pt}\ auto\ simp\ add{\isacharcolon}{\kern0pt}\ space{\isacharunderscore}{\kern0pt}restr\ intro{\isacharbang}{\kern0pt}{\isacharcolon}{\kern0pt}\ sets{\isacharunderscore}{\kern0pt}restrict{\isacharunderscore}{\kern0pt}space{\isacharunderscore}{\kern0pt}iff{\isacharbrackleft}{\kern0pt}THEN\ iffD{\isadigit{2}}{\isacharbrackright}{\kern0pt}{\isacharcomma}{\kern0pt}\ measurable{\isacharcomma}{\kern0pt}\ auto{\isacharparenright}{\kern0pt}\isanewline
\ \ \isacommand{{\isacharbraceright}{\kern0pt}}\isamarkupfalse%
\isanewline
\ \ \isacommand{note}\isamarkupfalse%
\ this{\isacharbrackleft}{\kern0pt}measurable{\isacharbrackright}{\kern0pt}\isanewline
\ \ \isacommand{{\isacharbraceleft}{\kern0pt}}\isamarkupfalse%
\isanewline
\ \ \ \ \isacommand{fix}\isamarkupfalse%
\ i\isanewline
\ \ \ \ \isacommand{have}\isamarkupfalse%
\ {\isachardoublequoteopen}finite{\isacharunderscore}{\kern0pt}measure\ {\isacharparenleft}{\kern0pt}restrict{\isacharunderscore}{\kern0pt}space\ M\ {\isacharparenleft}{\kern0pt}A\ i{\isacharparenright}{\kern0pt}{\isacharparenright}{\kern0pt}{\isachardoublequoteclose}\ \isacommand{using}\isamarkupfalse%
\ emeasure{\isacharunderscore}{\kern0pt}finite\ \isacommand{by}\isamarkupfalse%
\ {\isacharparenleft}{\kern0pt}intro\ finite{\isacharunderscore}{\kern0pt}measureI{\isacharcomma}{\kern0pt}\ subst\ space{\isacharunderscore}{\kern0pt}restr{\isacharcomma}{\kern0pt}\ subst\ emeasure{\isacharunderscore}{\kern0pt}restrict{\isacharunderscore}{\kern0pt}space{\isacharcomma}{\kern0pt}\ auto{\isacharparenright}{\kern0pt}\isanewline
\ \ \ \ \isacommand{hence}\isamarkupfalse%
\ {\isachardoublequoteopen}emeasure\ {\isacharparenleft}{\kern0pt}restrict{\isacharunderscore}{\kern0pt}space\ M\ {\isacharparenleft}{\kern0pt}A\ i{\isacharparenright}{\kern0pt}{\isacharparenright}{\kern0pt}\ {\isacharbraceleft}{\kern0pt}x\ {\isasymin}\ A\ i{\isachardot}{\kern0pt}\ {\isasymnot}Q\ x{\isacharbraceright}{\kern0pt}\ {\isacharequal}{\kern0pt}\ {\isadigit{0}}{\isachardoublequoteclose}\ \isacommand{using}\isamarkupfalse%
\ emeasure{\isacharunderscore}{\kern0pt}finite\ \isacommand{by}\isamarkupfalse%
\ {\isacharparenleft}{\kern0pt}intro\ AE{\isacharunderscore}{\kern0pt}iff{\isacharunderscore}{\kern0pt}measurable{\isacharbrackleft}{\kern0pt}THEN\ iffD{\isadigit{1}}{\isacharcomma}{\kern0pt}\ OF\ {\isacharunderscore}{\kern0pt}\ {\isacharunderscore}{\kern0pt}\ {\isacharasterisk}{\kern0pt}{\isacharbrackright}{\kern0pt}{\isacharcomma}{\kern0pt}\ measurable{\isacharcomma}{\kern0pt}\ subst\ space{\isacharunderscore}{\kern0pt}restr{\isacharbrackleft}{\kern0pt}symmetric{\isacharbrackright}{\kern0pt}{\isacharcomma}{\kern0pt}\ intro\ sets{\isachardot}{\kern0pt}top{\isacharcomma}{\kern0pt}\ auto\ simp\ add{\isacharcolon}{\kern0pt}\ emeasure{\isacharunderscore}{\kern0pt}restrict{\isacharunderscore}{\kern0pt}space{\isacharparenright}{\kern0pt}\isanewline
\ \ \ \ \isacommand{hence}\isamarkupfalse%
\ {\isachardoublequoteopen}emeasure\ M\ {\isacharbraceleft}{\kern0pt}x\ {\isasymin}\ A\ i{\isachardot}{\kern0pt}\ {\isasymnot}\ Q\ x{\isacharbraceright}{\kern0pt}\ {\isacharequal}{\kern0pt}\ {\isadigit{0}}{\isachardoublequoteclose}\ \isacommand{by}\isamarkupfalse%
\ {\isacharparenleft}{\kern0pt}subst\ emeasure{\isacharunderscore}{\kern0pt}restrict{\isacharunderscore}{\kern0pt}space{\isacharbrackleft}{\kern0pt}symmetric{\isacharbrackright}{\kern0pt}{\isacharcomma}{\kern0pt}\ auto{\isacharparenright}{\kern0pt}\isanewline
\ \ \isacommand{{\isacharbraceright}{\kern0pt}}\isamarkupfalse%
\isanewline
\ \ \isacommand{hence}\isamarkupfalse%
\ {\isachardoublequoteopen}emeasure\ M\ {\isacharparenleft}{\kern0pt}{\isasymUnion}i{\isachardot}{\kern0pt}\ {\isacharbraceleft}{\kern0pt}x\ {\isasymin}\ A\ i{\isachardot}{\kern0pt}\ {\isasymnot}\ Q\ x{\isacharbraceright}{\kern0pt}{\isacharparenright}{\kern0pt}\ {\isacharequal}{\kern0pt}\ {\isadigit{0}}{\isachardoublequoteclose}\ \isacommand{by}\isamarkupfalse%
\ {\isacharparenleft}{\kern0pt}intro\ emeasure{\isacharunderscore}{\kern0pt}UN{\isacharunderscore}{\kern0pt}eq{\isacharunderscore}{\kern0pt}{\isadigit{0}}{\isacharcomma}{\kern0pt}\ auto{\isacharparenright}{\kern0pt}\isanewline
\ \ \isacommand{moreover}\isamarkupfalse%
\ \isacommand{have}\isamarkupfalse%
\ {\isachardoublequoteopen}{\isacharparenleft}{\kern0pt}{\isasymUnion}i{\isachardot}{\kern0pt}\ {\isacharbraceleft}{\kern0pt}x\ {\isasymin}\ A\ i{\isachardot}{\kern0pt}\ {\isasymnot}\ Q\ x{\isacharbraceright}{\kern0pt}{\isacharparenright}{\kern0pt}\ {\isacharequal}{\kern0pt}\ {\isacharbraceleft}{\kern0pt}x\ {\isasymin}\ space\ M{\isachardot}{\kern0pt}\ {\isasymnot}\ Q\ x{\isacharbraceright}{\kern0pt}{\isachardoublequoteclose}\ \isacommand{using}\isamarkupfalse%
\ A\ \isacommand{by}\isamarkupfalse%
\ auto\isanewline
\ \ \isacommand{ultimately}\isamarkupfalse%
\ \isacommand{show}\isamarkupfalse%
\ {\isacharquery}{\kern0pt}thesis\ \isacommand{by}\isamarkupfalse%
\ {\isacharparenleft}{\kern0pt}intro\ AE{\isacharunderscore}{\kern0pt}iff{\isacharunderscore}{\kern0pt}measurable{\isacharbrackleft}{\kern0pt}THEN\ iffD{\isadigit{2}}{\isacharbrackright}{\kern0pt}{\isacharcomma}{\kern0pt}\ auto{\isacharparenright}{\kern0pt}\isanewline
\isacommand{qed}\isamarkupfalse%
%
\endisatagproof
{\isafoldproof}%
%
\isadelimproof
\isanewline
%
\endisadelimproof
\isanewline
\isanewline
\isacommand{lemma}\isamarkupfalse%
\ averaging{\isacharunderscore}{\kern0pt}theorem{\isacharcolon}{\kern0pt}\isanewline
\ \ \isakeyword{fixes}\ f{\isacharcolon}{\kern0pt}{\isacharcolon}{\kern0pt}{\isachardoublequoteopen}{\isacharunderscore}{\kern0pt}\ {\isasymRightarrow}\ {\isacharprime}{\kern0pt}b{\isacharcolon}{\kern0pt}{\isacharcolon}{\kern0pt}{\isacharbraceleft}{\kern0pt}second{\isacharunderscore}{\kern0pt}countable{\isacharunderscore}{\kern0pt}topology{\isacharcomma}{\kern0pt}\ banach{\isacharbraceright}{\kern0pt}{\isachardoublequoteclose}\isanewline
\ \ \isakeyword{assumes}\ {\isacharbrackleft}{\kern0pt}measurable{\isacharbrackright}{\kern0pt}{\isacharcolon}{\kern0pt}{\isachardoublequoteopen}integrable\ M\ f{\isachardoublequoteclose}\ \isanewline
\ \ \ \ \ \ \isakeyword{and}\ closed{\isacharcolon}{\kern0pt}\ {\isachardoublequoteopen}closed\ S{\isachardoublequoteclose}\isanewline
\ \ \ \ \ \ \isakeyword{and}\ {\isachardoublequoteopen}{\isasymAnd}A{\isachardot}{\kern0pt}\ A\ {\isasymin}\ sets\ M\ {\isasymLongrightarrow}\ measure\ M\ A\ {\isachargreater}{\kern0pt}\ {\isadigit{0}}\ {\isasymLongrightarrow}\ {\isacharparenleft}{\kern0pt}{\isadigit{1}}\ {\isacharslash}{\kern0pt}\ measure\ M\ A{\isacharparenright}{\kern0pt}\ {\isacharasterisk}{\kern0pt}\isactrlsub R\ set{\isacharunderscore}{\kern0pt}lebesgue{\isacharunderscore}{\kern0pt}integral\ M\ A\ f\ {\isasymin}\ S{\isachardoublequoteclose}\isanewline
\ \ \ \ \isakeyword{shows}\ {\isachardoublequoteopen}AE\ x\ in\ M{\isachardot}{\kern0pt}\ f\ x\ {\isasymin}\ S{\isachardoublequoteclose}\isanewline
%
\isadelimproof
%
\endisadelimproof
%
\isatagproof
\isacommand{proof}\isamarkupfalse%
\ {\isacharparenleft}{\kern0pt}induct\ rule{\isacharcolon}{\kern0pt}\ sigma{\isacharunderscore}{\kern0pt}finite{\isacharunderscore}{\kern0pt}measure{\isacharunderscore}{\kern0pt}induct{\isacharparenright}{\kern0pt}\isanewline
\ \ \isacommand{case}\isamarkupfalse%
\ {\isacharparenleft}{\kern0pt}finite{\isacharunderscore}{\kern0pt}measure\ N\ {\isasymOmega}{\isacharparenright}{\kern0pt}\isanewline
\isanewline
\ \ \isacommand{interpret}\isamarkupfalse%
\ finite{\isacharunderscore}{\kern0pt}measure\ N\ \isacommand{by}\isamarkupfalse%
\ {\isacharparenleft}{\kern0pt}rule\ finite{\isacharunderscore}{\kern0pt}measure{\isacharparenright}{\kern0pt}\isanewline
\ \ \isanewline
\ \ \isacommand{have}\isamarkupfalse%
\ integrable{\isacharbrackleft}{\kern0pt}measurable{\isacharbrackright}{\kern0pt}{\isacharcolon}{\kern0pt}\ {\isachardoublequoteopen}integrable\ N\ f{\isachardoublequoteclose}\ \isacommand{using}\isamarkupfalse%
\ assms\ finite{\isacharunderscore}{\kern0pt}measure\ \isacommand{by}\isamarkupfalse%
\ {\isacharparenleft}{\kern0pt}auto\ simp{\isacharcolon}{\kern0pt}\ integrable{\isacharunderscore}{\kern0pt}restrict{\isacharunderscore}{\kern0pt}space\ integrable{\isacharunderscore}{\kern0pt}mult{\isacharunderscore}{\kern0pt}indicator{\isacharparenright}{\kern0pt}\isanewline
\ \ \isacommand{have}\isamarkupfalse%
\ average{\isacharcolon}{\kern0pt}\ {\isachardoublequoteopen}{\isacharparenleft}{\kern0pt}{\isadigit{1}}\ {\isacharslash}{\kern0pt}\ Sigma{\isacharunderscore}{\kern0pt}Algebra{\isachardot}{\kern0pt}measure\ N\ A{\isacharparenright}{\kern0pt}\ {\isacharasterisk}{\kern0pt}\isactrlsub R\ set{\isacharunderscore}{\kern0pt}lebesgue{\isacharunderscore}{\kern0pt}integral\ N\ A\ f\ {\isasymin}\ S{\isachardoublequoteclose}\ \isakeyword{if}\ {\isachardoublequoteopen}A\ {\isasymin}\ sets\ N{\isachardoublequoteclose}\ {\isachardoublequoteopen}measure\ N\ A\ {\isachargreater}{\kern0pt}\ {\isadigit{0}}{\isachardoublequoteclose}\ \isakeyword{for}\ A\isanewline
\ \ \isacommand{proof}\isamarkupfalse%
\ {\isacharminus}{\kern0pt}\isanewline
\ \ \ \ \isacommand{have}\isamarkupfalse%
\ {\isacharasterisk}{\kern0pt}{\isacharcolon}{\kern0pt}\ {\isachardoublequoteopen}A\ {\isasymin}\ sets\ M{\isachardoublequoteclose}\ \isacommand{using}\isamarkupfalse%
\ that\ \isacommand{by}\isamarkupfalse%
\ {\isacharparenleft}{\kern0pt}simp\ add{\isacharcolon}{\kern0pt}\ sets{\isacharunderscore}{\kern0pt}restrict{\isacharunderscore}{\kern0pt}space{\isacharunderscore}{\kern0pt}iff\ finite{\isacharunderscore}{\kern0pt}measure{\isacharparenright}{\kern0pt}\isanewline
\ \ \ \ \isacommand{have}\isamarkupfalse%
\ {\isachardoublequoteopen}A\ {\isacharequal}{\kern0pt}\ A\ {\isasyminter}\ {\isasymOmega}{\isachardoublequoteclose}\ \isacommand{by}\isamarkupfalse%
\ {\isacharparenleft}{\kern0pt}metis\ finite{\isacharunderscore}{\kern0pt}measure{\isacharparenleft}{\kern0pt}{\isadigit{2}}{\isacharcomma}{\kern0pt}{\isadigit{3}}{\isacharparenright}{\kern0pt}\ inf{\isachardot}{\kern0pt}orderE\ sets{\isachardot}{\kern0pt}sets{\isacharunderscore}{\kern0pt}into{\isacharunderscore}{\kern0pt}space\ space{\isacharunderscore}{\kern0pt}restrict{\isacharunderscore}{\kern0pt}space\ that{\isacharparenleft}{\kern0pt}{\isadigit{1}}{\isacharparenright}{\kern0pt}{\isacharparenright}{\kern0pt}\isanewline
\ \ \ \ \isacommand{hence}\isamarkupfalse%
\ {\isachardoublequoteopen}set{\isacharunderscore}{\kern0pt}lebesgue{\isacharunderscore}{\kern0pt}integral\ N\ A\ f\ {\isacharequal}{\kern0pt}\ set{\isacharunderscore}{\kern0pt}lebesgue{\isacharunderscore}{\kern0pt}integral\ M\ A\ f{\isachardoublequoteclose}\ \isacommand{unfolding}\isamarkupfalse%
\ finite{\isacharunderscore}{\kern0pt}measure\ \isacommand{by}\isamarkupfalse%
\ {\isacharparenleft}{\kern0pt}subst\ set{\isacharunderscore}{\kern0pt}integral{\isacharunderscore}{\kern0pt}restrict{\isacharunderscore}{\kern0pt}space{\isacharcomma}{\kern0pt}\ auto\ simp\ add{\isacharcolon}{\kern0pt}\ finite{\isacharunderscore}{\kern0pt}measure\ set{\isacharunderscore}{\kern0pt}lebesgue{\isacharunderscore}{\kern0pt}integral{\isacharunderscore}{\kern0pt}def\ indicator{\isacharunderscore}{\kern0pt}inter{\isacharunderscore}{\kern0pt}arith{\isacharbrackleft}{\kern0pt}symmetric{\isacharbrackright}{\kern0pt}{\isacharparenright}{\kern0pt}\isanewline
\ \ \ \ \isacommand{moreover}\isamarkupfalse%
\ \isacommand{have}\isamarkupfalse%
\ {\isachardoublequoteopen}measure\ N\ A\ {\isacharequal}{\kern0pt}\ measure\ M\ A{\isachardoublequoteclose}\ \isacommand{using}\isamarkupfalse%
\ that\ \isacommand{by}\isamarkupfalse%
\ {\isacharparenleft}{\kern0pt}auto\ intro{\isacharbang}{\kern0pt}{\isacharcolon}{\kern0pt}\ measure{\isacharunderscore}{\kern0pt}restrict{\isacharunderscore}{\kern0pt}space\ simp\ add{\isacharcolon}{\kern0pt}\ finite{\isacharunderscore}{\kern0pt}measure\ sets{\isacharunderscore}{\kern0pt}restrict{\isacharunderscore}{\kern0pt}space{\isacharunderscore}{\kern0pt}iff{\isacharparenright}{\kern0pt}\isanewline
\ \ \ \ \isacommand{ultimately}\isamarkupfalse%
\ \isacommand{show}\isamarkupfalse%
\ {\isacharquery}{\kern0pt}thesis\ \isacommand{using}\isamarkupfalse%
\ that\ {\isacharasterisk}{\kern0pt}\ assms{\isacharparenleft}{\kern0pt}{\isadigit{3}}{\isacharparenright}{\kern0pt}\ \isacommand{by}\isamarkupfalse%
\ presburger\isanewline
\ \ \isacommand{qed}\isamarkupfalse%
\isanewline
\isanewline
\ \ \isacommand{obtain}\isamarkupfalse%
\ D\ {\isacharcolon}{\kern0pt}{\isacharcolon}{\kern0pt}\ {\isachardoublequoteopen}{\isacharprime}{\kern0pt}b\ set{\isachardoublequoteclose}\ \isakeyword{where}\ balls{\isacharunderscore}{\kern0pt}basis{\isacharcolon}{\kern0pt}\ {\isachardoublequoteopen}topological{\isacharunderscore}{\kern0pt}basis\ {\isacharparenleft}{\kern0pt}case{\isacharunderscore}{\kern0pt}prod\ ball\ {\isacharbackquote}{\kern0pt}\ {\isacharparenleft}{\kern0pt}D\ {\isasymtimes}\ {\isacharparenleft}{\kern0pt}{\isasymrat}\ {\isasyminter}\ {\isacharbraceleft}{\kern0pt}{\isadigit{0}}{\isacharless}{\kern0pt}{\isachardot}{\kern0pt}{\isachardot}{\kern0pt}{\isacharbraceright}{\kern0pt}{\isacharparenright}{\kern0pt}{\isacharparenright}{\kern0pt}{\isacharparenright}{\kern0pt}{\isachardoublequoteclose}\ \isakeyword{and}\ countable{\isacharunderscore}{\kern0pt}D{\isacharcolon}{\kern0pt}\ {\isachardoublequoteopen}countable\ D{\isachardoublequoteclose}\ \isacommand{using}\isamarkupfalse%
\ balls{\isacharunderscore}{\kern0pt}countable{\isacharunderscore}{\kern0pt}basis\ \isacommand{by}\isamarkupfalse%
\ blast\isanewline
\ \ \isacommand{have}\isamarkupfalse%
\ countable{\isacharunderscore}{\kern0pt}balls{\isacharcolon}{\kern0pt}\ {\isachardoublequoteopen}countable\ {\isacharparenleft}{\kern0pt}case{\isacharunderscore}{\kern0pt}prod\ ball\ {\isacharbackquote}{\kern0pt}\ {\isacharparenleft}{\kern0pt}D\ {\isasymtimes}\ {\isacharparenleft}{\kern0pt}{\isasymrat}\ {\isasyminter}\ {\isacharbraceleft}{\kern0pt}{\isadigit{0}}{\isacharless}{\kern0pt}{\isachardot}{\kern0pt}{\isachardot}{\kern0pt}{\isacharbraceright}{\kern0pt}{\isacharparenright}{\kern0pt}{\isacharparenright}{\kern0pt}{\isacharparenright}{\kern0pt}{\isachardoublequoteclose}\ \isacommand{using}\isamarkupfalse%
\ countable{\isacharunderscore}{\kern0pt}rat\ countable{\isacharunderscore}{\kern0pt}D\ \isacommand{by}\isamarkupfalse%
\ blast\isanewline
\isanewline
\ \ \isacommand{obtain}\isamarkupfalse%
\ B\ \isakeyword{where}\ B{\isacharunderscore}{\kern0pt}balls{\isacharcolon}{\kern0pt}\ {\isachardoublequoteopen}B\ {\isasymsubseteq}\ case{\isacharunderscore}{\kern0pt}prod\ ball\ {\isacharbackquote}{\kern0pt}\ {\isacharparenleft}{\kern0pt}D\ {\isasymtimes}\ {\isacharparenleft}{\kern0pt}{\isasymrat}\ {\isasyminter}\ {\isacharbraceleft}{\kern0pt}{\isadigit{0}}{\isacharless}{\kern0pt}{\isachardot}{\kern0pt}{\isachardot}{\kern0pt}{\isacharbraceright}{\kern0pt}{\isacharparenright}{\kern0pt}{\isacharparenright}{\kern0pt}{\isachardoublequoteclose}\ {\isachardoublequoteopen}{\isasymUnion}B\ {\isacharequal}{\kern0pt}\ {\isacharminus}{\kern0pt}S{\isachardoublequoteclose}\ \isacommand{using}\isamarkupfalse%
\ topological{\isacharunderscore}{\kern0pt}basis{\isacharbrackleft}{\kern0pt}THEN\ iffD{\isadigit{1}}{\isacharcomma}{\kern0pt}\ OF\ balls{\isacharunderscore}{\kern0pt}basis{\isacharbrackright}{\kern0pt}\ open{\isacharunderscore}{\kern0pt}Compl{\isacharbrackleft}{\kern0pt}OF\ assms{\isacharparenleft}{\kern0pt}{\isadigit{2}}{\isacharparenright}{\kern0pt}{\isacharbrackright}{\kern0pt}\ \isacommand{by}\isamarkupfalse%
\ meson\isanewline
\ \ \isacommand{hence}\isamarkupfalse%
\ countable{\isacharunderscore}{\kern0pt}B{\isacharcolon}{\kern0pt}\ {\isachardoublequoteopen}countable\ B{\isachardoublequoteclose}\ \isacommand{using}\isamarkupfalse%
\ countable{\isacharunderscore}{\kern0pt}balls\ countable{\isacharunderscore}{\kern0pt}subset\ \isacommand{by}\isamarkupfalse%
\ fast\isanewline
\isanewline
\ \ \isacommand{define}\isamarkupfalse%
\ b\ \isakeyword{where}\ {\isachardoublequoteopen}b\ {\isacharequal}{\kern0pt}\ from{\isacharunderscore}{\kern0pt}nat{\isacharunderscore}{\kern0pt}into\ {\isacharparenleft}{\kern0pt}B\ {\isasymunion}\ {\isacharbraceleft}{\kern0pt}{\isacharbraceleft}{\kern0pt}{\isacharbraceright}{\kern0pt}{\isacharbraceright}{\kern0pt}{\isacharparenright}{\kern0pt}{\isachardoublequoteclose}\isanewline
\ \ \isacommand{have}\isamarkupfalse%
\ {\isachardoublequoteopen}B\ {\isasymunion}\ {\isacharbraceleft}{\kern0pt}{\isacharbraceleft}{\kern0pt}{\isacharbraceright}{\kern0pt}{\isacharbraceright}{\kern0pt}\ {\isasymnoteq}\ {\isacharbraceleft}{\kern0pt}{\isacharbraceright}{\kern0pt}{\isachardoublequoteclose}\ \isacommand{by}\isamarkupfalse%
\ simp\isanewline
\ \ \isacommand{have}\isamarkupfalse%
\ range{\isacharunderscore}{\kern0pt}b{\isacharcolon}{\kern0pt}\ {\isachardoublequoteopen}range\ b\ {\isacharequal}{\kern0pt}\ B\ {\isasymunion}\ {\isacharbraceleft}{\kern0pt}{\isacharbraceleft}{\kern0pt}{\isacharbraceright}{\kern0pt}{\isacharbraceright}{\kern0pt}{\isachardoublequoteclose}\ \isacommand{using}\isamarkupfalse%
\ countable{\isacharunderscore}{\kern0pt}B\ \isacommand{by}\isamarkupfalse%
\ {\isacharparenleft}{\kern0pt}auto\ simp\ add{\isacharcolon}{\kern0pt}\ b{\isacharunderscore}{\kern0pt}def\ intro{\isacharbang}{\kern0pt}{\isacharcolon}{\kern0pt}\ range{\isacharunderscore}{\kern0pt}from{\isacharunderscore}{\kern0pt}nat{\isacharunderscore}{\kern0pt}into{\isacharparenright}{\kern0pt}\isanewline
\ \ \isacommand{have}\isamarkupfalse%
\ open{\isacharunderscore}{\kern0pt}b{\isacharcolon}{\kern0pt}\ {\isachardoublequoteopen}open\ {\isacharparenleft}{\kern0pt}b\ i{\isacharparenright}{\kern0pt}{\isachardoublequoteclose}\ \isakeyword{for}\ i\ \isacommand{unfolding}\isamarkupfalse%
\ b{\isacharunderscore}{\kern0pt}def\ \isacommand{using}\isamarkupfalse%
\ B{\isacharunderscore}{\kern0pt}balls\ open{\isacharunderscore}{\kern0pt}ball\ from{\isacharunderscore}{\kern0pt}nat{\isacharunderscore}{\kern0pt}into{\isacharbrackleft}{\kern0pt}of\ {\isachardoublequoteopen}B\ {\isasymunion}\ {\isacharbraceleft}{\kern0pt}{\isacharbraceleft}{\kern0pt}{\isacharbraceright}{\kern0pt}{\isacharbraceright}{\kern0pt}{\isachardoublequoteclose}\ i{\isacharbrackright}{\kern0pt}\ \isacommand{by}\isamarkupfalse%
\ force\isanewline
\ \ \isacommand{have}\isamarkupfalse%
\ Union{\isacharunderscore}{\kern0pt}range{\isacharunderscore}{\kern0pt}b{\isacharcolon}{\kern0pt}\ {\isachardoublequoteopen}{\isasymUnion}{\isacharparenleft}{\kern0pt}range\ b{\isacharparenright}{\kern0pt}\ {\isacharequal}{\kern0pt}\ {\isacharminus}{\kern0pt}S{\isachardoublequoteclose}\ \isacommand{using}\isamarkupfalse%
\ B{\isacharunderscore}{\kern0pt}balls\ range{\isacharunderscore}{\kern0pt}b\ \isacommand{by}\isamarkupfalse%
\ simp\isanewline
\isanewline
\ \ \isacommand{{\isacharbraceleft}{\kern0pt}}\isamarkupfalse%
\isanewline
\ \ \ \ \isacommand{fix}\isamarkupfalse%
\ v\ r\ \isacommand{assume}\isamarkupfalse%
\ ball{\isacharunderscore}{\kern0pt}in{\isacharunderscore}{\kern0pt}Compl{\isacharcolon}{\kern0pt}\ {\isachardoublequoteopen}ball\ v\ r\ {\isasymsubseteq}\ {\isacharminus}{\kern0pt}S{\isachardoublequoteclose}\isanewline
\ \ \ \ \isacommand{define}\isamarkupfalse%
\ A\ \isakeyword{where}\ {\isachardoublequoteopen}A\ {\isacharequal}{\kern0pt}\ f\ {\isacharminus}{\kern0pt}{\isacharbackquote}{\kern0pt}\ ball\ v\ r\ {\isasyminter}\ space\ N{\isachardoublequoteclose}\isanewline
\ \ \ \ \isacommand{have}\isamarkupfalse%
\ dist{\isacharunderscore}{\kern0pt}less{\isacharcolon}{\kern0pt}\ {\isachardoublequoteopen}dist\ {\isacharparenleft}{\kern0pt}f\ x{\isacharparenright}{\kern0pt}\ v\ {\isacharless}{\kern0pt}\ r{\isachardoublequoteclose}\ \isakeyword{if}\ {\isachardoublequoteopen}x\ {\isasymin}\ A{\isachardoublequoteclose}\ \isakeyword{for}\ x\ \isacommand{using}\isamarkupfalse%
\ that\ \isacommand{unfolding}\isamarkupfalse%
\ A{\isacharunderscore}{\kern0pt}def\ vimage{\isacharunderscore}{\kern0pt}def\ \isacommand{by}\isamarkupfalse%
\ {\isacharparenleft}{\kern0pt}simp\ add{\isacharcolon}{\kern0pt}\ dist{\isacharunderscore}{\kern0pt}commute{\isacharparenright}{\kern0pt}\isanewline
\ \ \ \ \isacommand{hence}\isamarkupfalse%
\ AE{\isacharunderscore}{\kern0pt}less{\isacharcolon}{\kern0pt}\ {\isachardoublequoteopen}AE\ x\ {\isasymin}\ A\ in\ N{\isachardot}{\kern0pt}\ norm\ {\isacharparenleft}{\kern0pt}f\ x\ {\isacharminus}{\kern0pt}\ v{\isacharparenright}{\kern0pt}\ {\isacharless}{\kern0pt}\ r{\isachardoublequoteclose}\ \isacommand{by}\isamarkupfalse%
\ {\isacharparenleft}{\kern0pt}auto\ simp\ add{\isacharcolon}{\kern0pt}\ dist{\isacharunderscore}{\kern0pt}norm{\isacharparenright}{\kern0pt}\isanewline
\ \ \ \ \isacommand{have}\isamarkupfalse%
\ {\isacharasterisk}{\kern0pt}{\isacharcolon}{\kern0pt}\ {\isachardoublequoteopen}A\ {\isasymin}\ sets\ N{\isachardoublequoteclose}\ \isacommand{unfolding}\isamarkupfalse%
\ A{\isacharunderscore}{\kern0pt}def\ \isacommand{by}\isamarkupfalse%
\ simp\isanewline
\ \ \ \ \isacommand{have}\isamarkupfalse%
\ {\isachardoublequoteopen}emeasure\ N\ A\ {\isacharequal}{\kern0pt}\ {\isadigit{0}}{\isachardoublequoteclose}\ \isanewline
\ \ \ \ \isacommand{proof}\isamarkupfalse%
\ {\isacharminus}{\kern0pt}\isanewline
\ \ \ \ \ \ \isacommand{{\isacharbraceleft}{\kern0pt}}\isamarkupfalse%
\isanewline
\ \ \ \ \ \ \ \ \isacommand{assume}\isamarkupfalse%
\ asm{\isacharcolon}{\kern0pt}\ {\isachardoublequoteopen}emeasure\ N\ A\ {\isachargreater}{\kern0pt}\ {\isadigit{0}}{\isachardoublequoteclose}\isanewline
\ \ \ \ \ \ \ \ \isacommand{hence}\isamarkupfalse%
\ measure{\isacharunderscore}{\kern0pt}pos{\isacharcolon}{\kern0pt}\ {\isachardoublequoteopen}measure\ N\ A\ {\isachargreater}{\kern0pt}\ {\isadigit{0}}{\isachardoublequoteclose}\ \isacommand{unfolding}\isamarkupfalse%
\ emeasure{\isacharunderscore}{\kern0pt}eq{\isacharunderscore}{\kern0pt}measure\ \isacommand{by}\isamarkupfalse%
\ simp\isanewline
\ \ \ \ \ \ \ \ \isacommand{hence}\isamarkupfalse%
\ {\isachardoublequoteopen}{\isacharparenleft}{\kern0pt}{\isadigit{1}}\ {\isacharslash}{\kern0pt}\ measure\ N\ A{\isacharparenright}{\kern0pt}\ {\isacharasterisk}{\kern0pt}\isactrlsub R\ set{\isacharunderscore}{\kern0pt}lebesgue{\isacharunderscore}{\kern0pt}integral\ N\ A\ f\ {\isacharminus}{\kern0pt}\ v\ {\isacharequal}{\kern0pt}\ {\isacharparenleft}{\kern0pt}{\isadigit{1}}\ {\isacharslash}{\kern0pt}\ measure\ N\ A{\isacharparenright}{\kern0pt}\ {\isacharasterisk}{\kern0pt}\isactrlsub R\ set{\isacharunderscore}{\kern0pt}lebesgue{\isacharunderscore}{\kern0pt}integral\ N\ A\ {\isacharparenleft}{\kern0pt}{\isasymlambda}x{\isachardot}{\kern0pt}\ f\ x\ {\isacharminus}{\kern0pt}\ v{\isacharparenright}{\kern0pt}{\isachardoublequoteclose}\ \isacommand{using}\isamarkupfalse%
\ integrable\ integrable{\isacharunderscore}{\kern0pt}const\ {\isacharasterisk}{\kern0pt}\ \isacommand{by}\isamarkupfalse%
\ {\isacharparenleft}{\kern0pt}subst\ set{\isacharunderscore}{\kern0pt}integral{\isacharunderscore}{\kern0pt}diff{\isacharparenleft}{\kern0pt}{\isadigit{2}}{\isacharparenright}{\kern0pt}{\isacharcomma}{\kern0pt}\ auto\ simp\ add{\isacharcolon}{\kern0pt}\ set{\isacharunderscore}{\kern0pt}integrable{\isacharunderscore}{\kern0pt}def\ set{\isacharunderscore}{\kern0pt}integral{\isacharunderscore}{\kern0pt}const{\isacharbrackleft}{\kern0pt}OF\ {\isacharasterisk}{\kern0pt}{\isacharbrackright}{\kern0pt}\ algebra{\isacharunderscore}{\kern0pt}simps\ intro{\isacharbang}{\kern0pt}{\isacharcolon}{\kern0pt}\ integrable{\isacharunderscore}{\kern0pt}mult{\isacharunderscore}{\kern0pt}indicator{\isacharparenright}{\kern0pt}\isanewline
\ \ \ \ \ \ \ \ \isacommand{moreover}\isamarkupfalse%
\ \isacommand{have}\isamarkupfalse%
\ {\isachardoublequoteopen}norm\ {\isacharparenleft}{\kern0pt}{\isasymintegral}x{\isasymin}A{\isachardot}{\kern0pt}\ {\isacharparenleft}{\kern0pt}f\ x\ {\isacharminus}{\kern0pt}\ v{\isacharparenright}{\kern0pt}{\isasympartial}N{\isacharparenright}{\kern0pt}\ {\isasymle}\ {\isacharparenleft}{\kern0pt}{\isasymintegral}x{\isasymin}A{\isachardot}{\kern0pt}\ norm\ {\isacharparenleft}{\kern0pt}f\ x\ {\isacharminus}{\kern0pt}\ v{\isacharparenright}{\kern0pt}{\isasympartial}N{\isacharparenright}{\kern0pt}{\isachardoublequoteclose}\ \isacommand{using}\isamarkupfalse%
\ {\isacharasterisk}{\kern0pt}\ \isacommand{by}\isamarkupfalse%
\ {\isacharparenleft}{\kern0pt}auto\ intro{\isacharbang}{\kern0pt}{\isacharcolon}{\kern0pt}\ integral{\isacharunderscore}{\kern0pt}norm{\isacharunderscore}{\kern0pt}bound{\isacharbrackleft}{\kern0pt}of\ N\ {\isachardoublequoteopen}{\isasymlambda}x{\isachardot}{\kern0pt}\ indicator\ A\ x\ {\isacharasterisk}{\kern0pt}\isactrlsub R\ {\isacharparenleft}{\kern0pt}f\ x\ {\isacharminus}{\kern0pt}\ v{\isacharparenright}{\kern0pt}{\isachardoublequoteclose}{\isacharcomma}{\kern0pt}\ THEN\ order{\isacharunderscore}{\kern0pt}trans{\isacharbrackright}{\kern0pt}\ integrable{\isacharunderscore}{\kern0pt}mult{\isacharunderscore}{\kern0pt}indicator\ integrable\ simp\ add{\isacharcolon}{\kern0pt}\ set{\isacharunderscore}{\kern0pt}lebesgue{\isacharunderscore}{\kern0pt}integral{\isacharunderscore}{\kern0pt}def{\isacharparenright}{\kern0pt}\isanewline
\ \ \ \ \ \ \ \ \isacommand{ultimately}\isamarkupfalse%
\ \isacommand{have}\isamarkupfalse%
\ {\isachardoublequoteopen}norm\ {\isacharparenleft}{\kern0pt}{\isacharparenleft}{\kern0pt}{\isadigit{1}}\ {\isacharslash}{\kern0pt}\ measure\ N\ A{\isacharparenright}{\kern0pt}\ {\isacharasterisk}{\kern0pt}\isactrlsub R\ set{\isacharunderscore}{\kern0pt}lebesgue{\isacharunderscore}{\kern0pt}integral\ N\ A\ f\ {\isacharminus}{\kern0pt}\ v{\isacharparenright}{\kern0pt}\ {\isasymle}\ \ set{\isacharunderscore}{\kern0pt}lebesgue{\isacharunderscore}{\kern0pt}integral\ N\ A\ {\isacharparenleft}{\kern0pt}{\isasymlambda}x{\isachardot}{\kern0pt}\ norm\ {\isacharparenleft}{\kern0pt}f\ x\ {\isacharminus}{\kern0pt}\ v{\isacharparenright}{\kern0pt}{\isacharparenright}{\kern0pt}\ {\isacharslash}{\kern0pt}\ measure\ N\ A{\isachardoublequoteclose}\ \isacommand{using}\isamarkupfalse%
\ asm\ \isacommand{by}\isamarkupfalse%
\ {\isacharparenleft}{\kern0pt}auto\ intro{\isacharcolon}{\kern0pt}\ divide{\isacharunderscore}{\kern0pt}right{\isacharunderscore}{\kern0pt}mono{\isacharparenright}{\kern0pt}\isanewline
\ \ \ \ \ \ \ \ \isacommand{also}\isamarkupfalse%
\ \isacommand{have}\isamarkupfalse%
\ {\isachardoublequoteopen}{\isachardot}{\kern0pt}{\isachardot}{\kern0pt}{\isachardot}{\kern0pt}\ {\isacharless}{\kern0pt}\ set{\isacharunderscore}{\kern0pt}lebesgue{\isacharunderscore}{\kern0pt}integral\ N\ A\ {\isacharparenleft}{\kern0pt}{\isasymlambda}x{\isachardot}{\kern0pt}\ r{\isacharparenright}{\kern0pt}\ {\isacharslash}{\kern0pt}\ measure\ N\ A{\isachardoublequoteclose}\ \isanewline
\ \ \ \ \ \ \ \ \ \ \isacommand{unfolding}\isamarkupfalse%
\ set{\isacharunderscore}{\kern0pt}lebesgue{\isacharunderscore}{\kern0pt}integral{\isacharunderscore}{\kern0pt}def\ \isanewline
\ \ \ \ \ \ \ \ \ \ \isacommand{using}\isamarkupfalse%
\ asm\ {\isacharasterisk}{\kern0pt}\ integrable\ integrable{\isacharunderscore}{\kern0pt}const\ AE{\isacharunderscore}{\kern0pt}less\ measure{\isacharunderscore}{\kern0pt}pos\isanewline
\ \ \ \ \ \ \ \ \ \ \isacommand{by}\isamarkupfalse%
\ {\isacharparenleft}{\kern0pt}intro\ divide{\isacharunderscore}{\kern0pt}strict{\isacharunderscore}{\kern0pt}right{\isacharunderscore}{\kern0pt}mono\ integral{\isacharunderscore}{\kern0pt}less{\isacharunderscore}{\kern0pt}AE{\isacharbrackleft}{\kern0pt}of\ {\isacharunderscore}{\kern0pt}\ {\isacharunderscore}{\kern0pt}\ A{\isacharbrackright}{\kern0pt}\ integrable{\isacharunderscore}{\kern0pt}mult{\isacharunderscore}{\kern0pt}indicator{\isacharparenright}{\kern0pt}\isanewline
\ \ \ \ \ \ \ \ \ \ \ \ {\isacharparenleft}{\kern0pt}fastforce\ simp\ add{\isacharcolon}{\kern0pt}\ dist{\isacharunderscore}{\kern0pt}less\ dist{\isacharunderscore}{\kern0pt}norm\ indicator{\isacharunderscore}{\kern0pt}def{\isacharparenright}{\kern0pt}{\isacharplus}{\kern0pt}\isanewline
\ \ \ \ \ \ \ \ \isacommand{also}\isamarkupfalse%
\ \isacommand{have}\isamarkupfalse%
\ {\isachardoublequoteopen}{\isachardot}{\kern0pt}{\isachardot}{\kern0pt}{\isachardot}{\kern0pt}\ {\isacharequal}{\kern0pt}\ r{\isachardoublequoteclose}\ \isacommand{using}\isamarkupfalse%
\ {\isacharasterisk}{\kern0pt}\ measure{\isacharunderscore}{\kern0pt}pos\ \isacommand{by}\isamarkupfalse%
\ {\isacharparenleft}{\kern0pt}simp\ add{\isacharcolon}{\kern0pt}\ set{\isacharunderscore}{\kern0pt}integral{\isacharunderscore}{\kern0pt}const{\isacharparenright}{\kern0pt}\isanewline
\ \ \ \ \ \ \ \ \isacommand{finally}\isamarkupfalse%
\ \isacommand{have}\isamarkupfalse%
\ {\isachardoublequoteopen}dist\ {\isacharparenleft}{\kern0pt}{\isacharparenleft}{\kern0pt}{\isadigit{1}}\ {\isacharslash}{\kern0pt}\ measure\ N\ A{\isacharparenright}{\kern0pt}\ {\isacharasterisk}{\kern0pt}\isactrlsub R\ set{\isacharunderscore}{\kern0pt}lebesgue{\isacharunderscore}{\kern0pt}integral\ N\ A\ f{\isacharparenright}{\kern0pt}\ v\ {\isacharless}{\kern0pt}\ r{\isachardoublequoteclose}\ \isacommand{by}\isamarkupfalse%
\ {\isacharparenleft}{\kern0pt}subst\ dist{\isacharunderscore}{\kern0pt}norm{\isacharparenright}{\kern0pt}\isanewline
\ \ \ \ \ \ \ \ \isacommand{hence}\isamarkupfalse%
\ {\isachardoublequoteopen}False{\isachardoublequoteclose}\ \isacommand{using}\isamarkupfalse%
\ average{\isacharbrackleft}{\kern0pt}OF\ {\isacharasterisk}{\kern0pt}\ measure{\isacharunderscore}{\kern0pt}pos{\isacharbrackright}{\kern0pt}\ \isacommand{by}\isamarkupfalse%
\ {\isacharparenleft}{\kern0pt}metis\ ComplD\ dist{\isacharunderscore}{\kern0pt}commute\ in{\isacharunderscore}{\kern0pt}mono\ mem{\isacharunderscore}{\kern0pt}ball\ ball{\isacharunderscore}{\kern0pt}in{\isacharunderscore}{\kern0pt}Compl{\isacharparenright}{\kern0pt}\isanewline
\ \ \ \ \ \ \isacommand{{\isacharbraceright}{\kern0pt}}\isamarkupfalse%
\isanewline
\ \ \ \ \ \ \isacommand{thus}\isamarkupfalse%
\ {\isacharquery}{\kern0pt}thesis\ \isacommand{by}\isamarkupfalse%
\ fastforce\isanewline
\ \ \ \ \isacommand{qed}\isamarkupfalse%
\isanewline
\ \ \isacommand{{\isacharbraceright}{\kern0pt}}\isamarkupfalse%
\isanewline
\ \ \isacommand{note}\isamarkupfalse%
\ {\isacharasterisk}{\kern0pt}\ {\isacharequal}{\kern0pt}\ this\isanewline
\ \ \isacommand{{\isacharbraceleft}{\kern0pt}}\isamarkupfalse%
\isanewline
\ \ \ \ \isacommand{fix}\isamarkupfalse%
\ b{\isacharprime}{\kern0pt}\ \isacommand{assume}\isamarkupfalse%
\ {\isachardoublequoteopen}b{\isacharprime}{\kern0pt}\ {\isasymin}\ B{\isachardoublequoteclose}\isanewline
\ \ \ \ \isacommand{hence}\isamarkupfalse%
\ ball{\isacharunderscore}{\kern0pt}subset{\isacharunderscore}{\kern0pt}Compl{\isacharcolon}{\kern0pt}\ {\isachardoublequoteopen}b{\isacharprime}{\kern0pt}\ {\isasymsubseteq}\ {\isacharminus}{\kern0pt}S{\isachardoublequoteclose}\ \isakeyword{and}\ ball{\isacharunderscore}{\kern0pt}radius{\isacharunderscore}{\kern0pt}pos{\isacharcolon}{\kern0pt}\ {\isachardoublequoteopen}{\isasymexists}v\ {\isasymin}\ D{\isachardot}{\kern0pt}\ {\isasymexists}r{\isachargreater}{\kern0pt}{\isadigit{0}}{\isachardot}{\kern0pt}\ b{\isacharprime}{\kern0pt}\ {\isacharequal}{\kern0pt}\ ball\ v\ r{\isachardoublequoteclose}\ \isacommand{using}\isamarkupfalse%
\ B{\isacharunderscore}{\kern0pt}balls\ \isacommand{by}\isamarkupfalse%
\ {\isacharparenleft}{\kern0pt}blast{\isacharcomma}{\kern0pt}\ fast{\isacharparenright}{\kern0pt}\isanewline
\ \ \isacommand{{\isacharbraceright}{\kern0pt}}\isamarkupfalse%
\isanewline
\ \ \isacommand{note}\isamarkupfalse%
\ {\isacharasterisk}{\kern0pt}{\isacharasterisk}{\kern0pt}\ {\isacharequal}{\kern0pt}\ this\isanewline
\ \ \isacommand{hence}\isamarkupfalse%
\ {\isachardoublequoteopen}emeasure\ N\ {\isacharparenleft}{\kern0pt}f\ {\isacharminus}{\kern0pt}{\isacharbackquote}{\kern0pt}\ b\ i\ {\isasyminter}\ space\ N{\isacharparenright}{\kern0pt}\ {\isacharequal}{\kern0pt}\ {\isadigit{0}}{\isachardoublequoteclose}\ \isakeyword{for}\ i\ \isacommand{by}\isamarkupfalse%
\ {\isacharparenleft}{\kern0pt}cases\ {\isachardoublequoteopen}b\ i\ {\isacharequal}{\kern0pt}\ {\isacharbraceleft}{\kern0pt}{\isacharbraceright}{\kern0pt}{\isachardoublequoteclose}{\isacharcomma}{\kern0pt}\ simp{\isacharparenright}{\kern0pt}\ {\isacharparenleft}{\kern0pt}metis\ UnE\ singletonD\ \ {\isacharasterisk}{\kern0pt}\ range{\isacharunderscore}{\kern0pt}b{\isacharbrackleft}{\kern0pt}THEN\ eq{\isacharunderscore}{\kern0pt}refl{\isacharcomma}{\kern0pt}\ THEN\ range{\isacharunderscore}{\kern0pt}subsetD{\isacharbrackright}{\kern0pt}{\isacharparenright}{\kern0pt}\isanewline
\ \ \isacommand{hence}\isamarkupfalse%
\ {\isachardoublequoteopen}emeasure\ N\ {\isacharparenleft}{\kern0pt}{\isasymUnion}i{\isachardot}{\kern0pt}\ f\ {\isacharminus}{\kern0pt}{\isacharbackquote}{\kern0pt}\ b\ i\ {\isasyminter}\ space\ N{\isacharparenright}{\kern0pt}\ {\isacharequal}{\kern0pt}\ {\isadigit{0}}{\isachardoublequoteclose}\ \isacommand{using}\isamarkupfalse%
\ open{\isacharunderscore}{\kern0pt}b\ \isacommand{by}\isamarkupfalse%
\ {\isacharparenleft}{\kern0pt}intro\ emeasure{\isacharunderscore}{\kern0pt}UN{\isacharunderscore}{\kern0pt}eq{\isacharunderscore}{\kern0pt}{\isadigit{0}}{\isacharparenright}{\kern0pt}\ fastforce{\isacharplus}{\kern0pt}\isanewline
\ \ \isacommand{moreover}\isamarkupfalse%
\ \isacommand{have}\isamarkupfalse%
\ {\isachardoublequoteopen}{\isacharparenleft}{\kern0pt}{\isasymUnion}i{\isachardot}{\kern0pt}\ f\ {\isacharminus}{\kern0pt}{\isacharbackquote}{\kern0pt}\ b\ i\ {\isasyminter}\ space\ N{\isacharparenright}{\kern0pt}\ {\isacharequal}{\kern0pt}\ f\ {\isacharminus}{\kern0pt}{\isacharbackquote}{\kern0pt}\ {\isacharparenleft}{\kern0pt}{\isasymUnion}{\isacharparenleft}{\kern0pt}range\ b{\isacharparenright}{\kern0pt}{\isacharparenright}{\kern0pt}\ {\isasyminter}\ space\ N{\isachardoublequoteclose}\ \isacommand{by}\isamarkupfalse%
\ blast\isanewline
\ \ \isacommand{ultimately}\isamarkupfalse%
\ \isacommand{have}\isamarkupfalse%
\ {\isachardoublequoteopen}emeasure\ N\ {\isacharparenleft}{\kern0pt}f\ {\isacharminus}{\kern0pt}{\isacharbackquote}{\kern0pt}\ {\isacharparenleft}{\kern0pt}{\isacharminus}{\kern0pt}S{\isacharparenright}{\kern0pt}\ {\isasyminter}\ space\ N{\isacharparenright}{\kern0pt}\ {\isacharequal}{\kern0pt}\ {\isadigit{0}}{\isachardoublequoteclose}\ \isacommand{using}\isamarkupfalse%
\ Union{\isacharunderscore}{\kern0pt}range{\isacharunderscore}{\kern0pt}b\ \isacommand{by}\isamarkupfalse%
\ argo\isanewline
\ \ \isacommand{hence}\isamarkupfalse%
\ {\isachardoublequoteopen}AE\ x\ in\ N{\isachardot}{\kern0pt}\ f\ x\ {\isasymnotin}\ {\isacharminus}{\kern0pt}S{\isachardoublequoteclose}\ \isacommand{using}\isamarkupfalse%
\ open{\isacharunderscore}{\kern0pt}Compl{\isacharbrackleft}{\kern0pt}OF\ assms{\isacharparenleft}{\kern0pt}{\isadigit{2}}{\isacharparenright}{\kern0pt}{\isacharbrackright}{\kern0pt}\ \isacommand{by}\isamarkupfalse%
\ {\isacharparenleft}{\kern0pt}intro\ AE{\isacharunderscore}{\kern0pt}iff{\isacharunderscore}{\kern0pt}measurable{\isacharbrackleft}{\kern0pt}THEN\ iffD{\isadigit{2}}{\isacharbrackright}{\kern0pt}{\isacharcomma}{\kern0pt}\ auto{\isacharparenright}{\kern0pt}\isanewline
\ \ \isacommand{thus}\isamarkupfalse%
\ {\isacharquery}{\kern0pt}case\ \isacommand{by}\isamarkupfalse%
\ force\isanewline
\isacommand{qed}\isamarkupfalse%
\ {\isacharparenleft}{\kern0pt}simp\ add{\isacharcolon}{\kern0pt}\ pred{\isacharunderscore}{\kern0pt}sets{\isadigit{2}}{\isacharbrackleft}{\kern0pt}OF\ borel{\isacharunderscore}{\kern0pt}closed{\isacharbrackright}{\kern0pt}\ assms{\isacharparenleft}{\kern0pt}{\isadigit{2}}{\isacharparenright}{\kern0pt}{\isacharparenright}{\kern0pt}%
\endisatagproof
{\isafoldproof}%
%
\isadelimproof
\isanewline
%
\endisadelimproof
\ \ \isanewline
\isacommand{lemma}\isamarkupfalse%
\ density{\isacharunderscore}{\kern0pt}zero{\isacharcolon}{\kern0pt}\isanewline
\ \ \isakeyword{fixes}\ f{\isacharcolon}{\kern0pt}{\isacharcolon}{\kern0pt}{\isachardoublequoteopen}{\isacharprime}{\kern0pt}a\ {\isasymRightarrow}\ {\isacharprime}{\kern0pt}b{\isacharcolon}{\kern0pt}{\isacharcolon}{\kern0pt}{\isacharbraceleft}{\kern0pt}second{\isacharunderscore}{\kern0pt}countable{\isacharunderscore}{\kern0pt}topology{\isacharcomma}{\kern0pt}\ banach{\isacharbraceright}{\kern0pt}{\isachardoublequoteclose}\isanewline
\ \ \isakeyword{assumes}\ {\isachardoublequoteopen}integrable\ M\ f{\isachardoublequoteclose}\isanewline
\ \ \ \ \ \ \isakeyword{and}\ density{\isacharunderscore}{\kern0pt}{\isadigit{0}}{\isacharcolon}{\kern0pt}\ {\isachardoublequoteopen}{\isasymAnd}A{\isachardot}{\kern0pt}\ A\ {\isasymin}\ sets\ M\ {\isasymLongrightarrow}\ set{\isacharunderscore}{\kern0pt}lebesgue{\isacharunderscore}{\kern0pt}integral\ M\ A\ f\ {\isacharequal}{\kern0pt}\ {\isadigit{0}}{\isachardoublequoteclose}\isanewline
\ \ \isakeyword{shows}\ {\isachardoublequoteopen}AE\ x\ in\ M{\isachardot}{\kern0pt}\ f\ x\ {\isacharequal}{\kern0pt}\ {\isadigit{0}}{\isachardoublequoteclose}\isanewline
%
\isadelimproof
\ \ %
\endisadelimproof
%
\isatagproof
\isacommand{using}\isamarkupfalse%
\ averaging{\isacharunderscore}{\kern0pt}theorem{\isacharbrackleft}{\kern0pt}OF\ assms{\isacharparenleft}{\kern0pt}{\isadigit{1}}{\isacharparenright}{\kern0pt}{\isacharcomma}{\kern0pt}\ of\ {\isachardoublequoteopen}{\isacharbraceleft}{\kern0pt}{\isadigit{0}}{\isacharbraceright}{\kern0pt}{\isachardoublequoteclose}{\isacharbrackright}{\kern0pt}\ assms{\isacharparenleft}{\kern0pt}{\isadigit{2}}{\isacharparenright}{\kern0pt}\isanewline
\ \ \isacommand{by}\isamarkupfalse%
\ {\isacharparenleft}{\kern0pt}simp\ add{\isacharcolon}{\kern0pt}\ scaleR{\isacharunderscore}{\kern0pt}nonneg{\isacharunderscore}{\kern0pt}nonneg{\isacharparenright}{\kern0pt}%
\endisatagproof
{\isafoldproof}%
%
\isadelimproof
\isanewline
%
\endisadelimproof
\isanewline
\isacommand{lemma}\isamarkupfalse%
\ density{\isacharunderscore}{\kern0pt}unique{\isacharcolon}{\kern0pt}\isanewline
\ \ \isakeyword{fixes}\ f\ f{\isacharprime}{\kern0pt}{\isacharcolon}{\kern0pt}{\isacharcolon}{\kern0pt}{\isachardoublequoteopen}{\isacharprime}{\kern0pt}a\ {\isasymRightarrow}\ {\isacharprime}{\kern0pt}b{\isacharcolon}{\kern0pt}{\isacharcolon}{\kern0pt}{\isacharbraceleft}{\kern0pt}second{\isacharunderscore}{\kern0pt}countable{\isacharunderscore}{\kern0pt}topology{\isacharcomma}{\kern0pt}\ banach{\isacharbraceright}{\kern0pt}{\isachardoublequoteclose}\isanewline
\ \ \isakeyword{assumes}\ {\isachardoublequoteopen}integrable\ M\ f{\isachardoublequoteclose}\ {\isachardoublequoteopen}integrable\ M\ f{\isacharprime}{\kern0pt}{\isachardoublequoteclose}\isanewline
\ \ \ \ \ \ \isakeyword{and}\ density{\isacharunderscore}{\kern0pt}eq{\isacharcolon}{\kern0pt}\ {\isachardoublequoteopen}{\isasymAnd}A{\isachardot}{\kern0pt}\ A\ {\isasymin}\ sets\ M\ {\isasymLongrightarrow}\ set{\isacharunderscore}{\kern0pt}lebesgue{\isacharunderscore}{\kern0pt}integral\ M\ A\ f\ {\isacharequal}{\kern0pt}\ set{\isacharunderscore}{\kern0pt}lebesgue{\isacharunderscore}{\kern0pt}integral\ M\ A\ f{\isacharprime}{\kern0pt}{\isachardoublequoteclose}\isanewline
\ \ \isakeyword{shows}\ {\isachardoublequoteopen}AE\ x\ in\ M{\isachardot}{\kern0pt}\ f\ x\ {\isacharequal}{\kern0pt}\ f{\isacharprime}{\kern0pt}\ x{\isachardoublequoteclose}\isanewline
%
\isadelimproof
%
\endisadelimproof
%
\isatagproof
\isacommand{proof}\isamarkupfalse%
{\isacharminus}{\kern0pt}\isanewline
\ \ \isacommand{{\isacharbraceleft}{\kern0pt}}\isamarkupfalse%
\ \isanewline
\ \ \ \ \isacommand{fix}\isamarkupfalse%
\ A\ \isacommand{assume}\isamarkupfalse%
\ asm{\isacharcolon}{\kern0pt}\ {\isachardoublequoteopen}A\ {\isasymin}\ sets\ M{\isachardoublequoteclose}\isanewline
\ \ \ \ \isacommand{hence}\isamarkupfalse%
\ {\isachardoublequoteopen}LINT\ x{\isacharbar}{\kern0pt}M{\isachardot}{\kern0pt}\ indicat{\isacharunderscore}{\kern0pt}real\ A\ x\ {\isacharasterisk}{\kern0pt}\isactrlsub R\ {\isacharparenleft}{\kern0pt}f\ x\ {\isacharminus}{\kern0pt}\ f{\isacharprime}{\kern0pt}\ x{\isacharparenright}{\kern0pt}\ {\isacharequal}{\kern0pt}\ {\isadigit{0}}{\isachardoublequoteclose}\ \isacommand{using}\isamarkupfalse%
\ density{\isacharunderscore}{\kern0pt}eq\ assms{\isacharparenleft}{\kern0pt}{\isadigit{1}}{\isacharcomma}{\kern0pt}{\isadigit{2}}{\isacharparenright}{\kern0pt}\ \isacommand{by}\isamarkupfalse%
\ {\isacharparenleft}{\kern0pt}simp\ add{\isacharcolon}{\kern0pt}\ set{\isacharunderscore}{\kern0pt}lebesgue{\isacharunderscore}{\kern0pt}integral{\isacharunderscore}{\kern0pt}def\ algebra{\isacharunderscore}{\kern0pt}simps\ Bochner{\isacharunderscore}{\kern0pt}Integration{\isachardot}{\kern0pt}integral{\isacharunderscore}{\kern0pt}diff{\isacharbrackleft}{\kern0pt}OF\ integrable{\isacharunderscore}{\kern0pt}mult{\isacharunderscore}{\kern0pt}indicator{\isacharparenleft}{\kern0pt}{\isadigit{1}}{\isacharcomma}{\kern0pt}{\isadigit{1}}{\isacharparenright}{\kern0pt}{\isacharbrackright}{\kern0pt}{\isacharparenright}{\kern0pt}\isanewline
\ \ \isacommand{{\isacharbraceright}{\kern0pt}}\isamarkupfalse%
\isanewline
\ \ \isacommand{thus}\isamarkupfalse%
\ {\isacharquery}{\kern0pt}thesis\ \isacommand{using}\isamarkupfalse%
\ density{\isacharunderscore}{\kern0pt}zero{\isacharbrackleft}{\kern0pt}OF\ Bochner{\isacharunderscore}{\kern0pt}Integration{\isachardot}{\kern0pt}integrable{\isacharunderscore}{\kern0pt}diff{\isacharbrackleft}{\kern0pt}OF\ assms{\isacharparenleft}{\kern0pt}{\isadigit{1}}{\isacharcomma}{\kern0pt}{\isadigit{2}}{\isacharparenright}{\kern0pt}{\isacharbrackright}{\kern0pt}{\isacharbrackright}{\kern0pt}\ \isacommand{by}\isamarkupfalse%
\ {\isacharparenleft}{\kern0pt}simp\ add{\isacharcolon}{\kern0pt}\ set{\isacharunderscore}{\kern0pt}lebesgue{\isacharunderscore}{\kern0pt}integral{\isacharunderscore}{\kern0pt}def{\isacharparenright}{\kern0pt}\isanewline
\isacommand{qed}\isamarkupfalse%
%
\endisatagproof
{\isafoldproof}%
%
\isadelimproof
\isanewline
%
\endisadelimproof
\isanewline
\isacommand{lemma}\isamarkupfalse%
\ density{\isacharunderscore}{\kern0pt}nonneg{\isacharcolon}{\kern0pt}\isanewline
\ \ \isakeyword{fixes}\ f{\isacharcolon}{\kern0pt}{\isacharcolon}{\kern0pt}{\isachardoublequoteopen}{\isacharunderscore}{\kern0pt}\ {\isasymRightarrow}\ {\isacharprime}{\kern0pt}b{\isacharcolon}{\kern0pt}{\isacharcolon}{\kern0pt}{\isacharbraceleft}{\kern0pt}second{\isacharunderscore}{\kern0pt}countable{\isacharunderscore}{\kern0pt}topology{\isacharcomma}{\kern0pt}\ banach{\isacharcomma}{\kern0pt}\ linorder{\isacharunderscore}{\kern0pt}topology{\isacharcomma}{\kern0pt}\ ordered{\isacharunderscore}{\kern0pt}real{\isacharunderscore}{\kern0pt}vector{\isacharbraceright}{\kern0pt}{\isachardoublequoteclose}\isanewline
\ \ \isakeyword{assumes}\ {\isachardoublequoteopen}integrable\ M\ f{\isachardoublequoteclose}\ \isanewline
\ \ \ \ \ \ \isakeyword{and}\ {\isachardoublequoteopen}{\isasymAnd}A{\isachardot}{\kern0pt}\ A\ {\isasymin}\ sets\ M\ {\isasymLongrightarrow}\ set{\isacharunderscore}{\kern0pt}lebesgue{\isacharunderscore}{\kern0pt}integral\ M\ A\ f\ {\isasymge}\ {\isadigit{0}}{\isachardoublequoteclose}\isanewline
\ \ \ \ \isakeyword{shows}\ {\isachardoublequoteopen}AE\ x\ in\ M{\isachardot}{\kern0pt}\ f\ x\ {\isasymge}\ {\isadigit{0}}{\isachardoublequoteclose}\isanewline
%
\isadelimproof
\ \ %
\endisadelimproof
%
\isatagproof
\isacommand{using}\isamarkupfalse%
\ averaging{\isacharunderscore}{\kern0pt}theorem{\isacharbrackleft}{\kern0pt}OF\ assms{\isacharparenleft}{\kern0pt}{\isadigit{1}}{\isacharparenright}{\kern0pt}{\isacharcomma}{\kern0pt}\ of\ {\isachardoublequoteopen}{\isacharbraceleft}{\kern0pt}{\isadigit{0}}{\isachardot}{\kern0pt}{\isachardot}{\kern0pt}{\isacharbraceright}{\kern0pt}{\isachardoublequoteclose}{\isacharcomma}{\kern0pt}\ OF\ closed{\isacharunderscore}{\kern0pt}atLeast{\isacharbrackright}{\kern0pt}\ assms{\isacharparenleft}{\kern0pt}{\isadigit{2}}{\isacharparenright}{\kern0pt}\isanewline
\ \ \isacommand{by}\isamarkupfalse%
\ {\isacharparenleft}{\kern0pt}simp\ add{\isacharcolon}{\kern0pt}\ scaleR{\isacharunderscore}{\kern0pt}nonneg{\isacharunderscore}{\kern0pt}nonneg{\isacharparenright}{\kern0pt}%
\endisatagproof
{\isafoldproof}%
%
\isadelimproof
\isanewline
%
\endisadelimproof
\isanewline
\isacommand{corollary}\isamarkupfalse%
\ integral{\isacharunderscore}{\kern0pt}nonneg{\isacharunderscore}{\kern0pt}AE{\isacharunderscore}{\kern0pt}eq{\isacharunderscore}{\kern0pt}{\isadigit{0}}{\isacharunderscore}{\kern0pt}iff{\isacharunderscore}{\kern0pt}AE{\isacharcolon}{\kern0pt}\isanewline
\ \ \isakeyword{fixes}\ f\ {\isacharcolon}{\kern0pt}{\isacharcolon}{\kern0pt}\ {\isachardoublequoteopen}{\isacharprime}{\kern0pt}a\ {\isasymRightarrow}\ {\isacharprime}{\kern0pt}b\ {\isacharcolon}{\kern0pt}{\isacharcolon}{\kern0pt}\ {\isacharbraceleft}{\kern0pt}second{\isacharunderscore}{\kern0pt}countable{\isacharunderscore}{\kern0pt}topology{\isacharcomma}{\kern0pt}\ banach{\isacharcomma}{\kern0pt}\ linorder{\isacharunderscore}{\kern0pt}topology{\isacharcomma}{\kern0pt}\ ordered{\isacharunderscore}{\kern0pt}real{\isacharunderscore}{\kern0pt}vector{\isacharbraceright}{\kern0pt}{\isachardoublequoteclose}\isanewline
\ \ \isakeyword{assumes}\ f{\isacharbrackleft}{\kern0pt}measurable{\isacharbrackright}{\kern0pt}{\isacharcolon}{\kern0pt}\ {\isachardoublequoteopen}integrable\ M\ f{\isachardoublequoteclose}\ \isakeyword{and}\ nonneg{\isacharcolon}{\kern0pt}\ {\isachardoublequoteopen}AE\ x\ in\ M{\isachardot}{\kern0pt}\ {\isadigit{0}}\ {\isasymle}\ f\ x{\isachardoublequoteclose}\isanewline
\ \ \isakeyword{shows}\ {\isachardoublequoteopen}integral\isactrlsup L\ M\ f\ {\isacharequal}{\kern0pt}\ {\isadigit{0}}\ {\isasymlongleftrightarrow}\ {\isacharparenleft}{\kern0pt}AE\ x\ in\ M{\isachardot}{\kern0pt}\ f\ x\ {\isacharequal}{\kern0pt}\ {\isadigit{0}}{\isacharparenright}{\kern0pt}{\isachardoublequoteclose}\isanewline
%
\isadelimproof
%
\endisadelimproof
%
\isatagproof
\isacommand{proof}\isamarkupfalse%
\ \isanewline
\ \ \isacommand{assume}\isamarkupfalse%
\ {\isacharasterisk}{\kern0pt}{\isacharcolon}{\kern0pt}\ {\isachardoublequoteopen}integral\isactrlsup L\ M\ f\ {\isacharequal}{\kern0pt}\ {\isadigit{0}}{\isachardoublequoteclose}\isanewline
\ \ \isacommand{{\isacharbraceleft}{\kern0pt}}\isamarkupfalse%
\isanewline
\ \ \ \ \isacommand{fix}\isamarkupfalse%
\ A\ \isacommand{assume}\isamarkupfalse%
\ asm{\isacharcolon}{\kern0pt}\ {\isachardoublequoteopen}A\ {\isasymin}\ sets\ M{\isachardoublequoteclose}\isanewline
\ \ \ \ \isacommand{have}\isamarkupfalse%
\ {\isachardoublequoteopen}{\isadigit{0}}\ {\isasymle}\ integral\isactrlsup L\ M\ {\isacharparenleft}{\kern0pt}{\isasymlambda}x{\isachardot}{\kern0pt}\ indicator\ A\ x\ {\isacharasterisk}{\kern0pt}\isactrlsub R\ f\ x{\isacharparenright}{\kern0pt}{\isachardoublequoteclose}\ \isacommand{using}\isamarkupfalse%
\ nonneg\ \isacommand{by}\isamarkupfalse%
\ {\isacharparenleft}{\kern0pt}subst\ integral{\isacharunderscore}{\kern0pt}zero{\isacharbrackleft}{\kern0pt}of\ M{\isacharcomma}{\kern0pt}\ symmetric{\isacharbrackright}{\kern0pt}{\isacharcomma}{\kern0pt}\ intro\ integral{\isacharunderscore}{\kern0pt}mono{\isacharunderscore}{\kern0pt}AE{\isacharunderscore}{\kern0pt}banach\ integrable{\isacharunderscore}{\kern0pt}mult{\isacharunderscore}{\kern0pt}indicator\ asm\ f\ integrable{\isacharunderscore}{\kern0pt}zero{\isacharcomma}{\kern0pt}\ auto\ simp\ add{\isacharcolon}{\kern0pt}\ indicator{\isacharunderscore}{\kern0pt}def{\isacharparenright}{\kern0pt}\isanewline
\ \ \ \ \isacommand{moreover}\isamarkupfalse%
\ \isacommand{have}\isamarkupfalse%
\ {\isachardoublequoteopen}{\isachardot}{\kern0pt}{\isachardot}{\kern0pt}{\isachardot}{\kern0pt}\ {\isasymle}\ integral\isactrlsup L\ M\ f{\isachardoublequoteclose}\ \isacommand{using}\isamarkupfalse%
\ nonneg\ \isacommand{by}\isamarkupfalse%
\ {\isacharparenleft}{\kern0pt}intro\ integral{\isacharunderscore}{\kern0pt}mono{\isacharunderscore}{\kern0pt}AE{\isacharunderscore}{\kern0pt}banach\ integrable{\isacharunderscore}{\kern0pt}mult{\isacharunderscore}{\kern0pt}indicator\ asm\ f{\isacharcomma}{\kern0pt}\ auto\ simp\ add{\isacharcolon}{\kern0pt}\ indicator{\isacharunderscore}{\kern0pt}def{\isacharparenright}{\kern0pt}\isanewline
\ \ \ \ \isacommand{ultimately}\isamarkupfalse%
\ \isacommand{have}\isamarkupfalse%
\ {\isachardoublequoteopen}set{\isacharunderscore}{\kern0pt}lebesgue{\isacharunderscore}{\kern0pt}integral\ M\ A\ f\ {\isacharequal}{\kern0pt}\ {\isadigit{0}}{\isachardoublequoteclose}\ \isacommand{unfolding}\isamarkupfalse%
\ set{\isacharunderscore}{\kern0pt}lebesgue{\isacharunderscore}{\kern0pt}integral{\isacharunderscore}{\kern0pt}def\ \isacommand{using}\isamarkupfalse%
\ {\isacharasterisk}{\kern0pt}\ \isacommand{by}\isamarkupfalse%
\ force\isanewline
\ \ \isacommand{{\isacharbraceright}{\kern0pt}}\isamarkupfalse%
\isanewline
\ \ \isacommand{thus}\isamarkupfalse%
\ {\isachardoublequoteopen}AE\ x\ in\ M{\isachardot}{\kern0pt}\ f\ x\ {\isacharequal}{\kern0pt}\ {\isadigit{0}}{\isachardoublequoteclose}\ \isacommand{by}\isamarkupfalse%
\ {\isacharparenleft}{\kern0pt}intro\ density{\isacharunderscore}{\kern0pt}zero\ f{\isacharcomma}{\kern0pt}\ blast{\isacharparenright}{\kern0pt}\isanewline
\isacommand{qed}\isamarkupfalse%
\ {\isacharparenleft}{\kern0pt}auto\ simp\ add{\isacharcolon}{\kern0pt}\ integral{\isacharunderscore}{\kern0pt}eq{\isacharunderscore}{\kern0pt}zero{\isacharunderscore}{\kern0pt}AE{\isacharparenright}{\kern0pt}%
\endisatagproof
{\isafoldproof}%
%
\isadelimproof
\isanewline
%
\endisadelimproof
\isanewline
\isacommand{corollary}\isamarkupfalse%
\ integral{\isacharunderscore}{\kern0pt}eq{\isacharunderscore}{\kern0pt}mono{\isacharunderscore}{\kern0pt}AE{\isacharunderscore}{\kern0pt}eq{\isacharunderscore}{\kern0pt}AE{\isacharcolon}{\kern0pt}\isanewline
\ \ \isakeyword{fixes}\ f\ g\ {\isacharcolon}{\kern0pt}{\isacharcolon}{\kern0pt}\ {\isachardoublequoteopen}{\isacharprime}{\kern0pt}a\ {\isasymRightarrow}\ {\isacharprime}{\kern0pt}b\ {\isacharcolon}{\kern0pt}{\isacharcolon}{\kern0pt}\ {\isacharbraceleft}{\kern0pt}second{\isacharunderscore}{\kern0pt}countable{\isacharunderscore}{\kern0pt}topology{\isacharcomma}{\kern0pt}\ banach{\isacharcomma}{\kern0pt}\ linorder{\isacharunderscore}{\kern0pt}topology{\isacharcomma}{\kern0pt}\ ordered{\isacharunderscore}{\kern0pt}real{\isacharunderscore}{\kern0pt}vector{\isacharbraceright}{\kern0pt}{\isachardoublequoteclose}\isanewline
\ \ \isakeyword{assumes}\ {\isachardoublequoteopen}integrable\ M\ f{\isachardoublequoteclose}\ {\isachardoublequoteopen}integrable\ M\ g{\isachardoublequoteclose}\ {\isachardoublequoteopen}integral\isactrlsup L\ M\ f\ {\isacharequal}{\kern0pt}\ integral\isactrlsup L\ M\ g{\isachardoublequoteclose}\ {\isachardoublequoteopen}AE\ x\ in\ M{\isachardot}{\kern0pt}\ f\ x\ {\isasymle}\ g\ x{\isachardoublequoteclose}\ \isanewline
\ \ \isakeyword{shows}\ {\isachardoublequoteopen}AE\ x\ in\ M{\isachardot}{\kern0pt}\ f\ x\ {\isacharequal}{\kern0pt}\ g\ x{\isachardoublequoteclose}\isanewline
%
\isadelimproof
%
\endisadelimproof
%
\isatagproof
\isacommand{proof}\isamarkupfalse%
\ {\isacharminus}{\kern0pt}\isanewline
\ \ \isacommand{define}\isamarkupfalse%
\ h\ \isakeyword{where}\ {\isachardoublequoteopen}h\ {\isacharequal}{\kern0pt}\ {\isacharparenleft}{\kern0pt}{\isasymlambda}x{\isachardot}{\kern0pt}\ g\ x\ {\isacharminus}{\kern0pt}\ f\ x{\isacharparenright}{\kern0pt}{\isachardoublequoteclose}\isanewline
\ \ \isacommand{have}\isamarkupfalse%
\ {\isachardoublequoteopen}AE\ x\ in\ M{\isachardot}{\kern0pt}\ h\ x\ {\isacharequal}{\kern0pt}\ {\isadigit{0}}{\isachardoublequoteclose}\ \isacommand{unfolding}\isamarkupfalse%
\ h{\isacharunderscore}{\kern0pt}def\ \isacommand{using}\isamarkupfalse%
\ assms\ \isacommand{by}\isamarkupfalse%
\ {\isacharparenleft}{\kern0pt}subst\ integral{\isacharunderscore}{\kern0pt}nonneg{\isacharunderscore}{\kern0pt}AE{\isacharunderscore}{\kern0pt}eq{\isacharunderscore}{\kern0pt}{\isadigit{0}}{\isacharunderscore}{\kern0pt}iff{\isacharunderscore}{\kern0pt}AE{\isacharbrackleft}{\kern0pt}symmetric{\isacharbrackright}{\kern0pt}{\isacharparenright}{\kern0pt}\ auto\isanewline
\ \ \isacommand{then}\isamarkupfalse%
\ \isacommand{show}\isamarkupfalse%
\ {\isacharquery}{\kern0pt}thesis\ \isacommand{unfolding}\isamarkupfalse%
\ h{\isacharunderscore}{\kern0pt}def\ \isacommand{by}\isamarkupfalse%
\ auto\isanewline
\isacommand{qed}\isamarkupfalse%
%
\endisatagproof
{\isafoldproof}%
%
\isadelimproof
\isanewline
%
\endisadelimproof
\isanewline
\isacommand{end}\isamarkupfalse%
\isanewline
%
\isadelimtheory
\isanewline
%
\endisadelimtheory
%
\isatagtheory
\isacommand{end}\isamarkupfalse%
%
\endisatagtheory
{\isafoldtheory}%
%
\isadelimtheory
%
\endisadelimtheory
%
\end{isabellebody}%
\endinput
%:%file=Sigma_Finite_Measure_Addendum.tex%:%
%:%10=1%:%
%:%11=1%:%
%:%12=2%:%
%:%13=3%:%
%:%27=5%:%
%:%37=8%:%
%:%38=8%:%
%:%39=9%:%
%:%40=10%:%
%:%41=11%:%
%:%42=12%:%
%:%49=13%:%
%:%50=13%:%
%:%51=14%:%
%:%52=14%:%
%:%53=14%:%
%:%54=14%:%
%:%55=15%:%
%:%56=15%:%
%:%57=16%:%
%:%58=16%:%
%:%59=17%:%
%:%60=17%:%
%:%61=17%:%
%:%62=18%:%
%:%63=18%:%
%:%64=18%:%
%:%65=18%:%
%:%66=19%:%
%:%67=19%:%
%:%68=19%:%
%:%69=19%:%
%:%70=20%:%
%:%71=20%:%
%:%72=20%:%
%:%73=20%:%
%:%74=20%:%
%:%75=21%:%
%:%76=22%:%
%:%77=22%:%
%:%78=22%:%
%:%79=22%:%
%:%80=23%:%
%:%81=23%:%
%:%82=23%:%
%:%83=23%:%
%:%84=24%:%
%:%85=24%:%
%:%86=24%:%
%:%87=24%:%
%:%88=24%:%
%:%89=25%:%
%:%90=25%:%
%:%91=25%:%
%:%92=25%:%
%:%93=25%:%
%:%94=26%:%
%:%95=26%:%
%:%96=27%:%
%:%97=27%:%
%:%98=27%:%
%:%99=27%:%
%:%100=28%:%
%:%106=28%:%
%:%109=29%:%
%:%110=30%:%
%:%111=30%:%
%:%112=31%:%
%:%113=32%:%
%:%114=33%:%
%:%115=33%:%
%:%116=34%:%
%:%121=39%:%
%:%122=40%:%
%:%123=41%:%
%:%130=42%:%
%:%131=42%:%
%:%132=43%:%
%:%133=43%:%
%:%134=43%:%
%:%135=43%:%
%:%136=44%:%
%:%137=45%:%
%:%138=45%:%
%:%139=45%:%
%:%140=45%:%
%:%141=46%:%
%:%142=46%:%
%:%143=47%:%
%:%144=47%:%
%:%145=47%:%
%:%146=47%:%
%:%147=48%:%
%:%148=48%:%
%:%149=49%:%
%:%150=49%:%
%:%151=50%:%
%:%152=50%:%
%:%153=50%:%
%:%154=51%:%
%:%155=51%:%
%:%156=51%:%
%:%157=51%:%
%:%158=52%:%
%:%159=52%:%
%:%160=53%:%
%:%161=53%:%
%:%162=54%:%
%:%163=54%:%
%:%164=55%:%
%:%165=55%:%
%:%166=56%:%
%:%167=56%:%
%:%168=56%:%
%:%169=56%:%
%:%170=57%:%
%:%171=57%:%
%:%172=57%:%
%:%173=57%:%
%:%174=58%:%
%:%175=58%:%
%:%176=58%:%
%:%177=59%:%
%:%178=59%:%
%:%179=60%:%
%:%180=60%:%
%:%181=60%:%
%:%182=61%:%
%:%183=61%:%
%:%184=61%:%
%:%185=61%:%
%:%186=61%:%
%:%187=62%:%
%:%188=62%:%
%:%189=62%:%
%:%190=62%:%
%:%191=63%:%
%:%197=63%:%
%:%200=64%:%
%:%201=65%:%
%:%202=66%:%
%:%203=66%:%
%:%204=67%:%
%:%205=68%:%
%:%206=69%:%
%:%207=70%:%
%:%208=71%:%
%:%215=72%:%
%:%216=72%:%
%:%217=73%:%
%:%218=73%:%
%:%219=74%:%
%:%220=75%:%
%:%221=75%:%
%:%222=75%:%
%:%223=76%:%
%:%224=77%:%
%:%225=77%:%
%:%226=77%:%
%:%227=77%:%
%:%228=78%:%
%:%229=78%:%
%:%230=79%:%
%:%231=79%:%
%:%232=80%:%
%:%233=80%:%
%:%234=80%:%
%:%235=80%:%
%:%236=81%:%
%:%237=81%:%
%:%238=81%:%
%:%239=82%:%
%:%240=82%:%
%:%241=82%:%
%:%242=82%:%
%:%243=83%:%
%:%244=83%:%
%:%245=83%:%
%:%246=83%:%
%:%247=83%:%
%:%248=84%:%
%:%249=84%:%
%:%250=84%:%
%:%251=84%:%
%:%252=84%:%
%:%253=85%:%
%:%254=85%:%
%:%255=86%:%
%:%256=87%:%
%:%257=87%:%
%:%258=87%:%
%:%259=87%:%
%:%260=88%:%
%:%261=88%:%
%:%262=88%:%
%:%263=88%:%
%:%264=89%:%
%:%265=90%:%
%:%266=90%:%
%:%267=90%:%
%:%268=90%:%
%:%269=91%:%
%:%270=91%:%
%:%271=91%:%
%:%272=91%:%
%:%273=92%:%
%:%274=93%:%
%:%275=93%:%
%:%276=94%:%
%:%277=94%:%
%:%278=94%:%
%:%279=95%:%
%:%280=95%:%
%:%281=95%:%
%:%282=95%:%
%:%283=96%:%
%:%284=96%:%
%:%285=96%:%
%:%286=96%:%
%:%287=96%:%
%:%288=97%:%
%:%289=97%:%
%:%290=97%:%
%:%291=97%:%
%:%292=98%:%
%:%293=99%:%
%:%294=99%:%
%:%295=100%:%
%:%296=100%:%
%:%297=100%:%
%:%298=101%:%
%:%299=101%:%
%:%300=102%:%
%:%301=102%:%
%:%302=102%:%
%:%303=102%:%
%:%304=102%:%
%:%305=103%:%
%:%306=103%:%
%:%307=103%:%
%:%308=104%:%
%:%309=104%:%
%:%310=104%:%
%:%311=104%:%
%:%312=105%:%
%:%313=105%:%
%:%314=106%:%
%:%315=106%:%
%:%316=107%:%
%:%317=107%:%
%:%318=108%:%
%:%319=108%:%
%:%320=109%:%
%:%321=109%:%
%:%322=109%:%
%:%323=109%:%
%:%324=110%:%
%:%325=110%:%
%:%326=110%:%
%:%327=110%:%
%:%328=111%:%
%:%329=111%:%
%:%330=111%:%
%:%331=111%:%
%:%332=111%:%
%:%333=112%:%
%:%334=112%:%
%:%335=112%:%
%:%336=112%:%
%:%337=112%:%
%:%338=113%:%
%:%339=113%:%
%:%340=113%:%
%:%341=114%:%
%:%342=114%:%
%:%343=115%:%
%:%344=115%:%
%:%345=116%:%
%:%346=116%:%
%:%347=117%:%
%:%348=118%:%
%:%349=118%:%
%:%350=118%:%
%:%351=118%:%
%:%352=118%:%
%:%353=119%:%
%:%354=119%:%
%:%355=119%:%
%:%356=119%:%
%:%357=120%:%
%:%358=120%:%
%:%359=120%:%
%:%360=120%:%
%:%361=121%:%
%:%362=121%:%
%:%363=122%:%
%:%364=122%:%
%:%365=122%:%
%:%366=123%:%
%:%367=123%:%
%:%368=124%:%
%:%369=124%:%
%:%370=125%:%
%:%371=125%:%
%:%372=126%:%
%:%373=126%:%
%:%374=127%:%
%:%375=127%:%
%:%376=127%:%
%:%377=128%:%
%:%378=128%:%
%:%379=128%:%
%:%380=128%:%
%:%381=129%:%
%:%382=129%:%
%:%383=130%:%
%:%384=130%:%
%:%385=131%:%
%:%386=131%:%
%:%387=131%:%
%:%388=132%:%
%:%389=132%:%
%:%390=132%:%
%:%391=132%:%
%:%392=133%:%
%:%393=133%:%
%:%394=133%:%
%:%395=133%:%
%:%396=134%:%
%:%397=134%:%
%:%398=134%:%
%:%399=134%:%
%:%400=134%:%
%:%401=135%:%
%:%402=135%:%
%:%403=135%:%
%:%404=135%:%
%:%405=136%:%
%:%406=136%:%
%:%407=136%:%
%:%408=137%:%
%:%409=137%:%
%:%414=137%:%
%:%417=138%:%
%:%418=139%:%
%:%419=139%:%
%:%420=140%:%
%:%421=141%:%
%:%422=142%:%
%:%423=143%:%
%:%426=144%:%
%:%430=144%:%
%:%431=144%:%
%:%432=145%:%
%:%433=145%:%
%:%438=145%:%
%:%441=146%:%
%:%442=147%:%
%:%443=147%:%
%:%444=148%:%
%:%445=149%:%
%:%446=150%:%
%:%447=151%:%
%:%454=152%:%
%:%455=152%:%
%:%456=153%:%
%:%457=153%:%
%:%458=154%:%
%:%459=154%:%
%:%460=154%:%
%:%461=155%:%
%:%462=155%:%
%:%463=155%:%
%:%464=155%:%
%:%465=156%:%
%:%466=156%:%
%:%467=157%:%
%:%468=157%:%
%:%469=157%:%
%:%470=157%:%
%:%471=158%:%
%:%477=158%:%
%:%480=159%:%
%:%481=160%:%
%:%482=160%:%
%:%483=161%:%
%:%484=162%:%
%:%485=163%:%
%:%486=164%:%
%:%489=165%:%
%:%493=165%:%
%:%494=165%:%
%:%495=166%:%
%:%496=166%:%
%:%501=166%:%
%:%504=167%:%
%:%505=168%:%
%:%506=168%:%
%:%507=169%:%
%:%508=170%:%
%:%509=171%:%
%:%516=172%:%
%:%517=172%:%
%:%518=173%:%
%:%519=173%:%
%:%520=174%:%
%:%521=174%:%
%:%522=175%:%
%:%523=175%:%
%:%524=175%:%
%:%525=176%:%
%:%526=176%:%
%:%527=176%:%
%:%528=176%:%
%:%529=177%:%
%:%530=177%:%
%:%531=177%:%
%:%532=177%:%
%:%533=177%:%
%:%534=178%:%
%:%535=178%:%
%:%536=178%:%
%:%537=178%:%
%:%538=178%:%
%:%539=178%:%
%:%540=179%:%
%:%541=179%:%
%:%542=180%:%
%:%543=180%:%
%:%544=180%:%
%:%545=181%:%
%:%546=181%:%
%:%551=181%:%
%:%554=182%:%
%:%555=183%:%
%:%556=183%:%
%:%557=184%:%
%:%558=185%:%
%:%559=186%:%
%:%566=187%:%
%:%567=187%:%
%:%568=188%:%
%:%569=188%:%
%:%570=189%:%
%:%571=189%:%
%:%572=189%:%
%:%573=189%:%
%:%574=189%:%
%:%575=190%:%
%:%576=190%:%
%:%577=190%:%
%:%578=190%:%
%:%579=190%:%
%:%580=191%:%
%:%586=191%:%
%:%589=192%:%
%:%590=193%:%
%:%591=193%:%
%:%594=194%:%
%:%599=195%:%

%
\begin{isabellebody}%
\setisabellecontext{Filtered{\isacharunderscore}{\kern0pt}Measure}%
%
\isadelimtheory
\isanewline
\isanewline
%
\endisadelimtheory
%
\isatagtheory
\isacommand{theory}\isamarkupfalse%
\ Filtered{\isacharunderscore}{\kern0pt}Measure\isanewline
\ \ \isakeyword{imports}\ {\isachardoublequoteopen}HOL{\isacharminus}{\kern0pt}Probability{\isachardot}{\kern0pt}Conditional{\isacharunderscore}{\kern0pt}Expectation{\isachardoublequoteclose}\isanewline
\isakeyword{begin}%
\endisatagtheory
{\isafoldtheory}%
%
\isadelimtheory
%
\endisadelimtheory
%
\isadelimdocument
%
\endisadelimdocument
%
\isatagdocument
%
\isamarkupsection{Filtered Measure Spaces%
}
\isamarkuptrue%
%
\isamarkupsubsection{Filtered Measure%
}
\isamarkuptrue%
%
\endisatagdocument
{\isafolddocument}%
%
\isadelimdocument
%
\endisadelimdocument
\isacommand{locale}\isamarkupfalse%
\ filtered{\isacharunderscore}{\kern0pt}measure\ {\isacharequal}{\kern0pt}\ \isanewline
\ \ \isakeyword{fixes}\ M\ F\ \isakeyword{and}\ t\isactrlsub {\isadigit{0}}\ {\isacharcolon}{\kern0pt}{\isacharcolon}{\kern0pt}\ {\isachardoublequoteopen}{\isacharprime}{\kern0pt}b\ {\isacharcolon}{\kern0pt}{\isacharcolon}{\kern0pt}\ {\isacharbraceleft}{\kern0pt}second{\isacharunderscore}{\kern0pt}countable{\isacharunderscore}{\kern0pt}topology{\isacharcomma}{\kern0pt}\ order{\isacharunderscore}{\kern0pt}topology{\isacharcomma}{\kern0pt}\ t{\isadigit{2}}{\isacharunderscore}{\kern0pt}space{\isacharbraceright}{\kern0pt}{\isachardoublequoteclose}\isanewline
\ \ \isakeyword{assumes}\ subalgebra{\isacharcolon}{\kern0pt}\ {\isachardoublequoteopen}{\isasymAnd}i{\isachardot}{\kern0pt}\ t\isactrlsub {\isadigit{0}}\ {\isasymle}\ i\ {\isasymLongrightarrow}\ subalgebra\ M\ {\isacharparenleft}{\kern0pt}F\ i{\isacharparenright}{\kern0pt}{\isachardoublequoteclose}\isanewline
\ \ \ \ \ \ \isakeyword{and}\ sets{\isacharunderscore}{\kern0pt}F{\isacharunderscore}{\kern0pt}mono{\isacharcolon}{\kern0pt}\ {\isachardoublequoteopen}{\isasymAnd}i\ j{\isachardot}{\kern0pt}\ t\isactrlsub {\isadigit{0}}\ {\isasymle}\ i\ {\isasymLongrightarrow}\ i\ {\isasymle}\ j\ {\isasymLongrightarrow}\ sets\ {\isacharparenleft}{\kern0pt}F\ i{\isacharparenright}{\kern0pt}\ {\isasymle}\ sets\ {\isacharparenleft}{\kern0pt}F\ j{\isacharparenright}{\kern0pt}{\isachardoublequoteclose}\isanewline
\isakeyword{begin}\isanewline
\isanewline
\isacommand{lemma}\isamarkupfalse%
\ space{\isacharunderscore}{\kern0pt}F{\isacharbrackleft}{\kern0pt}simp{\isacharbrackright}{\kern0pt}{\isacharcolon}{\kern0pt}\ \isanewline
\ \ \isakeyword{assumes}\ {\isachardoublequoteopen}t\isactrlsub {\isadigit{0}}\ {\isasymle}\ i{\isachardoublequoteclose}\isanewline
\ \ \isakeyword{shows}\ {\isachardoublequoteopen}space\ {\isacharparenleft}{\kern0pt}F\ i{\isacharparenright}{\kern0pt}\ {\isacharequal}{\kern0pt}\ space\ M{\isachardoublequoteclose}\isanewline
%
\isadelimproof
\ \ %
\endisadelimproof
%
\isatagproof
\isacommand{using}\isamarkupfalse%
\ subalgebra\ assms\ \isacommand{by}\isamarkupfalse%
\ {\isacharparenleft}{\kern0pt}simp\ add{\isacharcolon}{\kern0pt}\ subalgebra{\isacharunderscore}{\kern0pt}def{\isacharparenright}{\kern0pt}%
\endisatagproof
{\isafoldproof}%
%
\isadelimproof
\isanewline
%
\endisadelimproof
\isanewline
\isacommand{lemma}\isamarkupfalse%
\ subalgebra{\isacharunderscore}{\kern0pt}F{\isacharbrackleft}{\kern0pt}intro{\isacharbrackright}{\kern0pt}{\isacharcolon}{\kern0pt}\ \isanewline
\ \ \isakeyword{assumes}\ {\isachardoublequoteopen}t\isactrlsub {\isadigit{0}}\ {\isasymle}\ i{\isachardoublequoteclose}\ {\isachardoublequoteopen}i\ {\isasymle}\ j{\isachardoublequoteclose}\isanewline
\ \ \isakeyword{shows}\ {\isachardoublequoteopen}subalgebra\ {\isacharparenleft}{\kern0pt}F\ j{\isacharparenright}{\kern0pt}\ {\isacharparenleft}{\kern0pt}F\ i{\isacharparenright}{\kern0pt}{\isachardoublequoteclose}\isanewline
%
\isadelimproof
\ \ %
\endisadelimproof
%
\isatagproof
\isacommand{unfolding}\isamarkupfalse%
\ subalgebra{\isacharunderscore}{\kern0pt}def\ \isacommand{using}\isamarkupfalse%
\ assms\ \isacommand{by}\isamarkupfalse%
\ {\isacharparenleft}{\kern0pt}simp\ add{\isacharcolon}{\kern0pt}\ sets{\isacharunderscore}{\kern0pt}F{\isacharunderscore}{\kern0pt}mono{\isacharparenright}{\kern0pt}%
\endisatagproof
{\isafoldproof}%
%
\isadelimproof
\isanewline
%
\endisadelimproof
\isanewline
\isacommand{lemma}\isamarkupfalse%
\ borel{\isacharunderscore}{\kern0pt}measurable{\isacharunderscore}{\kern0pt}mono{\isacharcolon}{\kern0pt}\isanewline
\ \ \isakeyword{assumes}\ {\isachardoublequoteopen}t\isactrlsub {\isadigit{0}}\ {\isasymle}\ i{\isachardoublequoteclose}\ {\isachardoublequoteopen}i\ {\isasymle}\ j{\isachardoublequoteclose}\isanewline
\ \ \isakeyword{shows}\ {\isachardoublequoteopen}borel{\isacharunderscore}{\kern0pt}measurable\ {\isacharparenleft}{\kern0pt}F\ i{\isacharparenright}{\kern0pt}\ {\isasymsubseteq}\ borel{\isacharunderscore}{\kern0pt}measurable\ {\isacharparenleft}{\kern0pt}F\ j{\isacharparenright}{\kern0pt}{\isachardoublequoteclose}\isanewline
%
\isadelimproof
\ \ %
\endisadelimproof
%
\isatagproof
\isacommand{unfolding}\isamarkupfalse%
\ subset{\isacharunderscore}{\kern0pt}iff\ \isacommand{by}\isamarkupfalse%
\ {\isacharparenleft}{\kern0pt}metis\ assms\ subalgebra{\isacharunderscore}{\kern0pt}F\ measurable{\isacharunderscore}{\kern0pt}from{\isacharunderscore}{\kern0pt}subalg{\isacharparenright}{\kern0pt}%
\endisatagproof
{\isafoldproof}%
%
\isadelimproof
\isanewline
%
\endisadelimproof
\isanewline
\isacommand{end}\isamarkupfalse%
\isanewline
\isanewline
\isacommand{locale}\isamarkupfalse%
\ linearly{\isacharunderscore}{\kern0pt}filtered{\isacharunderscore}{\kern0pt}measure\ {\isacharequal}{\kern0pt}\ filtered{\isacharunderscore}{\kern0pt}measure\ M\ F\ t\isactrlsub {\isadigit{0}}\ \isakeyword{for}\ M\ \isakeyword{and}\ F\ {\isacharcolon}{\kern0pt}{\isacharcolon}{\kern0pt}\ {\isachardoublequoteopen}{\isacharunderscore}{\kern0pt}\ {\isacharcolon}{\kern0pt}{\isacharcolon}{\kern0pt}\ {\isacharbraceleft}{\kern0pt}linorder{\isacharunderscore}{\kern0pt}topology{\isacharbraceright}{\kern0pt}\ {\isasymRightarrow}\ {\isacharunderscore}{\kern0pt}{\isachardoublequoteclose}\ \isakeyword{and}\ t\isactrlsub {\isadigit{0}}\isanewline
\isanewline
\isacommand{locale}\isamarkupfalse%
\ nat{\isacharunderscore}{\kern0pt}filtered{\isacharunderscore}{\kern0pt}measure\ {\isacharequal}{\kern0pt}\ linearly{\isacharunderscore}{\kern0pt}filtered{\isacharunderscore}{\kern0pt}measure\ M\ F\ {\isadigit{0}}\ \isakeyword{for}\ M\ \isakeyword{and}\ F\ {\isacharcolon}{\kern0pt}{\isacharcolon}{\kern0pt}\ {\isachardoublequoteopen}nat\ {\isasymRightarrow}\ {\isacharunderscore}{\kern0pt}{\isachardoublequoteclose}\isanewline
\isacommand{locale}\isamarkupfalse%
\ real{\isacharunderscore}{\kern0pt}filtered{\isacharunderscore}{\kern0pt}measure\ {\isacharequal}{\kern0pt}\ linearly{\isacharunderscore}{\kern0pt}filtered{\isacharunderscore}{\kern0pt}measure\ M\ F\ {\isadigit{0}}\ \isakeyword{for}\ M\ \isakeyword{and}\ F\ {\isacharcolon}{\kern0pt}{\isacharcolon}{\kern0pt}\ {\isachardoublequoteopen}real\ {\isasymRightarrow}\ {\isacharunderscore}{\kern0pt}{\isachardoublequoteclose}%
\isadelimdocument
%
\endisadelimdocument
%
\isatagdocument
%
\isamarkupsubsection{Sigma Finite Filtered Measure%
}
\isamarkuptrue%
%
\endisatagdocument
{\isafolddocument}%
%
\isadelimdocument
%
\endisadelimdocument
%
\begin{isamarkuptext}%
The locale presented here is a generalization of the \isa{sigma{\isacharunderscore}{\kern0pt}finite{\isacharunderscore}{\kern0pt}subalgebra} for a particular filtration.%
\end{isamarkuptext}\isamarkuptrue%
\isacommand{locale}\isamarkupfalse%
\ sigma{\isacharunderscore}{\kern0pt}finite{\isacharunderscore}{\kern0pt}filtered{\isacharunderscore}{\kern0pt}measure\ {\isacharequal}{\kern0pt}\ filtered{\isacharunderscore}{\kern0pt}measure\ {\isacharplus}{\kern0pt}\isanewline
\ \ \isakeyword{assumes}\ sigma{\isacharunderscore}{\kern0pt}finite{\isacharunderscore}{\kern0pt}initial{\isacharcolon}{\kern0pt}\ {\isachardoublequoteopen}sigma{\isacharunderscore}{\kern0pt}finite{\isacharunderscore}{\kern0pt}subalgebra\ M\ {\isacharparenleft}{\kern0pt}F\ t\isactrlsub {\isadigit{0}}{\isacharparenright}{\kern0pt}{\isachardoublequoteclose}\isanewline
\isanewline
\isacommand{lemma}\isamarkupfalse%
\ {\isacharparenleft}{\kern0pt}\isakeyword{in}\ sigma{\isacharunderscore}{\kern0pt}finite{\isacharunderscore}{\kern0pt}filtered{\isacharunderscore}{\kern0pt}measure{\isacharparenright}{\kern0pt}\ sigma{\isacharunderscore}{\kern0pt}finite{\isacharunderscore}{\kern0pt}subalgebra{\isacharunderscore}{\kern0pt}F{\isacharbrackleft}{\kern0pt}intro{\isacharbrackright}{\kern0pt}{\isacharcolon}{\kern0pt}\isanewline
\ \ \isakeyword{assumes}\ {\isachardoublequoteopen}t\isactrlsub {\isadigit{0}}\ {\isasymle}\ i{\isachardoublequoteclose}\isanewline
\ \ \isakeyword{shows}\ {\isachardoublequoteopen}sigma{\isacharunderscore}{\kern0pt}finite{\isacharunderscore}{\kern0pt}subalgebra\ M\ {\isacharparenleft}{\kern0pt}F\ i{\isacharparenright}{\kern0pt}{\isachardoublequoteclose}\isanewline
%
\isadelimproof
\ \ %
\endisadelimproof
%
\isatagproof
\isacommand{using}\isamarkupfalse%
\ assms\ \isacommand{by}\isamarkupfalse%
\ {\isacharparenleft}{\kern0pt}metis\ dual{\isacharunderscore}{\kern0pt}order{\isachardot}{\kern0pt}refl\ sets{\isacharunderscore}{\kern0pt}F{\isacharunderscore}{\kern0pt}mono\ sigma{\isacharunderscore}{\kern0pt}finite{\isacharunderscore}{\kern0pt}initial\ sigma{\isacharunderscore}{\kern0pt}finite{\isacharunderscore}{\kern0pt}subalgebra{\isachardot}{\kern0pt}nested{\isacharunderscore}{\kern0pt}subalg{\isacharunderscore}{\kern0pt}is{\isacharunderscore}{\kern0pt}sigma{\isacharunderscore}{\kern0pt}finite\ subalgebra\ subalgebra{\isacharunderscore}{\kern0pt}def{\isacharparenright}{\kern0pt}%
\endisatagproof
{\isafoldproof}%
%
\isadelimproof
\isanewline
%
\endisadelimproof
\isanewline
\isacommand{locale}\isamarkupfalse%
\ nat{\isacharunderscore}{\kern0pt}sigma{\isacharunderscore}{\kern0pt}finite{\isacharunderscore}{\kern0pt}filtered{\isacharunderscore}{\kern0pt}measure\ {\isacharequal}{\kern0pt}\ sigma{\isacharunderscore}{\kern0pt}finite{\isacharunderscore}{\kern0pt}filtered{\isacharunderscore}{\kern0pt}measure\ M\ F\ {\isachardoublequoteopen}{\isadigit{0}}\ {\isacharcolon}{\kern0pt}{\isacharcolon}{\kern0pt}\ nat{\isachardoublequoteclose}\ \isakeyword{for}\ M\ F\isanewline
\isacommand{locale}\isamarkupfalse%
\ real{\isacharunderscore}{\kern0pt}sigma{\isacharunderscore}{\kern0pt}finite{\isacharunderscore}{\kern0pt}filtered{\isacharunderscore}{\kern0pt}measure\ {\isacharequal}{\kern0pt}\ sigma{\isacharunderscore}{\kern0pt}finite{\isacharunderscore}{\kern0pt}filtered{\isacharunderscore}{\kern0pt}measure\ M\ F\ {\isachardoublequoteopen}{\isadigit{0}}\ {\isacharcolon}{\kern0pt}{\isacharcolon}{\kern0pt}\ real{\isachardoublequoteclose}\ \isakeyword{for}\ M\ F\isanewline
\isanewline
\isacommand{sublocale}\isamarkupfalse%
\ nat{\isacharunderscore}{\kern0pt}sigma{\isacharunderscore}{\kern0pt}finite{\isacharunderscore}{\kern0pt}filtered{\isacharunderscore}{\kern0pt}measure\ {\isasymsubseteq}\ sigma{\isacharunderscore}{\kern0pt}finite{\isacharunderscore}{\kern0pt}subalgebra\ M\ {\isachardoublequoteopen}F\ i{\isachardoublequoteclose}%
\isadelimproof
\ %
\endisadelimproof
%
\isatagproof
\isacommand{by}\isamarkupfalse%
\ blast%
\endisatagproof
{\isafoldproof}%
%
\isadelimproof
%
\endisadelimproof
\isanewline
\isacommand{sublocale}\isamarkupfalse%
\ real{\isacharunderscore}{\kern0pt}sigma{\isacharunderscore}{\kern0pt}finite{\isacharunderscore}{\kern0pt}filtered{\isacharunderscore}{\kern0pt}measure\ {\isasymsubseteq}\ sigma{\isacharunderscore}{\kern0pt}finite{\isacharunderscore}{\kern0pt}subalgebra\ M\ {\isachardoublequoteopen}F\ {\isasymbar}i{\isasymbar}{\isachardoublequoteclose}%
\isadelimproof
\ %
\endisadelimproof
%
\isatagproof
\isacommand{by}\isamarkupfalse%
\ fastforce%
\endisatagproof
{\isafoldproof}%
%
\isadelimproof
%
\endisadelimproof
%
\isadelimdocument
%
\endisadelimdocument
%
\isatagdocument
%
\isamarkupsubsection{Finite Filtered Measure%
}
\isamarkuptrue%
%
\endisatagdocument
{\isafolddocument}%
%
\isadelimdocument
%
\endisadelimdocument
\isacommand{locale}\isamarkupfalse%
\ finite{\isacharunderscore}{\kern0pt}filtered{\isacharunderscore}{\kern0pt}measure\ {\isacharequal}{\kern0pt}\ filtered{\isacharunderscore}{\kern0pt}measure\ {\isacharplus}{\kern0pt}\ finite{\isacharunderscore}{\kern0pt}measure\isanewline
\isanewline
\isacommand{sublocale}\isamarkupfalse%
\ finite{\isacharunderscore}{\kern0pt}filtered{\isacharunderscore}{\kern0pt}measure\ {\isasymsubseteq}\ sigma{\isacharunderscore}{\kern0pt}finite{\isacharunderscore}{\kern0pt}filtered{\isacharunderscore}{\kern0pt}measure\ \isanewline
%
\isadelimproof
\ \ %
\endisadelimproof
%
\isatagproof
\isacommand{using}\isamarkupfalse%
\ subalgebra\ \isacommand{by}\isamarkupfalse%
\ {\isacharparenleft}{\kern0pt}unfold{\isacharunderscore}{\kern0pt}locales{\isacharcomma}{\kern0pt}\ blast{\isacharcomma}{\kern0pt}\ meson\ dual{\isacharunderscore}{\kern0pt}order{\isachardot}{\kern0pt}refl\ finite{\isacharunderscore}{\kern0pt}measure{\isacharunderscore}{\kern0pt}axioms\ finite{\isacharunderscore}{\kern0pt}measure{\isacharunderscore}{\kern0pt}def\ finite{\isacharunderscore}{\kern0pt}measure{\isacharunderscore}{\kern0pt}restr{\isacharunderscore}{\kern0pt}to{\isacharunderscore}{\kern0pt}subalg\ sigma{\isacharunderscore}{\kern0pt}finite{\isacharunderscore}{\kern0pt}measure{\isachardot}{\kern0pt}sigma{\isacharunderscore}{\kern0pt}finite{\isacharunderscore}{\kern0pt}countable\ subalgebra{\isacharparenright}{\kern0pt}%
\endisatagproof
{\isafoldproof}%
%
\isadelimproof
\isanewline
%
\endisadelimproof
\isanewline
\isacommand{locale}\isamarkupfalse%
\ nat{\isacharunderscore}{\kern0pt}finite{\isacharunderscore}{\kern0pt}filtered{\isacharunderscore}{\kern0pt}measure\ {\isacharequal}{\kern0pt}\ finite{\isacharunderscore}{\kern0pt}filtered{\isacharunderscore}{\kern0pt}measure\ M\ F\ {\isachardoublequoteopen}{\isadigit{0}}\ {\isacharcolon}{\kern0pt}{\isacharcolon}{\kern0pt}\ nat{\isachardoublequoteclose}\ \isakeyword{for}\ M\ F\isanewline
\isacommand{locale}\isamarkupfalse%
\ real{\isacharunderscore}{\kern0pt}finite{\isacharunderscore}{\kern0pt}filtered{\isacharunderscore}{\kern0pt}measure\ {\isacharequal}{\kern0pt}\ finite{\isacharunderscore}{\kern0pt}filtered{\isacharunderscore}{\kern0pt}measure\ M\ F\ {\isachardoublequoteopen}{\isadigit{0}}\ {\isacharcolon}{\kern0pt}{\isacharcolon}{\kern0pt}\ real{\isachardoublequoteclose}\ \isakeyword{for}\ M\ F\isanewline
\isanewline
\isacommand{sublocale}\isamarkupfalse%
\ nat{\isacharunderscore}{\kern0pt}finite{\isacharunderscore}{\kern0pt}filtered{\isacharunderscore}{\kern0pt}measure\ {\isasymsubseteq}\ nat{\isacharunderscore}{\kern0pt}sigma{\isacharunderscore}{\kern0pt}finite{\isacharunderscore}{\kern0pt}filtered{\isacharunderscore}{\kern0pt}measure%
\isadelimproof
\ %
\endisadelimproof
%
\isatagproof
\isacommand{{\isachardot}{\kern0pt}{\isachardot}{\kern0pt}}\isamarkupfalse%
%
\endisatagproof
{\isafoldproof}%
%
\isadelimproof
%
\endisadelimproof
\isanewline
\isacommand{sublocale}\isamarkupfalse%
\ real{\isacharunderscore}{\kern0pt}finite{\isacharunderscore}{\kern0pt}filtered{\isacharunderscore}{\kern0pt}measure\ {\isasymsubseteq}\ real{\isacharunderscore}{\kern0pt}sigma{\isacharunderscore}{\kern0pt}finite{\isacharunderscore}{\kern0pt}filtered{\isacharunderscore}{\kern0pt}measure%
\isadelimproof
\ %
\endisadelimproof
%
\isatagproof
\isacommand{{\isachardot}{\kern0pt}{\isachardot}{\kern0pt}}\isamarkupfalse%
%
\endisatagproof
{\isafoldproof}%
%
\isadelimproof
%
\endisadelimproof
%
\isadelimdocument
%
\endisadelimdocument
%
\isatagdocument
%
\isamarkupsubsection{Constant Filtration%
}
\isamarkuptrue%
%
\endisatagdocument
{\isafolddocument}%
%
\isadelimdocument
%
\endisadelimdocument
\isacommand{lemma}\isamarkupfalse%
\ filtered{\isacharunderscore}{\kern0pt}measure{\isacharunderscore}{\kern0pt}constant{\isacharunderscore}{\kern0pt}filtration{\isacharcolon}{\kern0pt}\isanewline
\ \ \isakeyword{assumes}\ {\isachardoublequoteopen}subalgebra\ M\ F{\isachardoublequoteclose}\isanewline
\ \ \isakeyword{shows}\ {\isachardoublequoteopen}filtered{\isacharunderscore}{\kern0pt}measure\ M\ {\isacharparenleft}{\kern0pt}{\isasymlambda}{\isacharunderscore}{\kern0pt}{\isachardot}{\kern0pt}\ F{\isacharparenright}{\kern0pt}\ t\isactrlsub {\isadigit{0}}{\isachardoublequoteclose}\isanewline
%
\isadelimproof
\ \ %
\endisadelimproof
%
\isatagproof
\isacommand{using}\isamarkupfalse%
\ assms\ \isacommand{by}\isamarkupfalse%
\ {\isacharparenleft}{\kern0pt}unfold{\isacharunderscore}{\kern0pt}locales{\isacharparenright}{\kern0pt}\ blast{\isacharplus}{\kern0pt}%
\endisatagproof
{\isafoldproof}%
%
\isadelimproof
\isanewline
%
\endisadelimproof
\isanewline
\isacommand{sublocale}\isamarkupfalse%
\ sigma{\isacharunderscore}{\kern0pt}finite{\isacharunderscore}{\kern0pt}subalgebra\ {\isasymsubseteq}\ constant{\isacharunderscore}{\kern0pt}filtration{\isacharcolon}{\kern0pt}\ sigma{\isacharunderscore}{\kern0pt}finite{\isacharunderscore}{\kern0pt}filtered{\isacharunderscore}{\kern0pt}measure\ M\ {\isachardoublequoteopen}{\isasymlambda}{\isacharunderscore}{\kern0pt}\ {\isacharcolon}{\kern0pt}{\isacharcolon}{\kern0pt}\ {\isacharprime}{\kern0pt}t\ {\isacharcolon}{\kern0pt}{\isacharcolon}{\kern0pt}\ {\isacharbraceleft}{\kern0pt}second{\isacharunderscore}{\kern0pt}countable{\isacharunderscore}{\kern0pt}topology{\isacharcomma}{\kern0pt}\ linorder{\isacharunderscore}{\kern0pt}topology{\isacharbraceright}{\kern0pt}{\isachardot}{\kern0pt}\ F{\isachardoublequoteclose}\ t\isactrlsub {\isadigit{0}}\isanewline
%
\isadelimproof
\ \ %
\endisadelimproof
%
\isatagproof
\isacommand{using}\isamarkupfalse%
\ subalg\ \isacommand{by}\isamarkupfalse%
\ {\isacharparenleft}{\kern0pt}unfold{\isacharunderscore}{\kern0pt}locales{\isacharparenright}{\kern0pt}\ blast{\isacharplus}{\kern0pt}%
\endisatagproof
{\isafoldproof}%
%
\isadelimproof
\isanewline
%
\endisadelimproof
\isanewline
\isacommand{lemma}\isamarkupfalse%
\ {\isacharparenleft}{\kern0pt}\isakeyword{in}\ finite{\isacharunderscore}{\kern0pt}measure{\isacharparenright}{\kern0pt}\ filtered{\isacharunderscore}{\kern0pt}measure{\isacharunderscore}{\kern0pt}constant{\isacharunderscore}{\kern0pt}filtration{\isacharcolon}{\kern0pt}\isanewline
\ \ \isakeyword{assumes}\ {\isachardoublequoteopen}subalgebra\ M\ F{\isachardoublequoteclose}\isanewline
\ \ \isakeyword{shows}\ {\isachardoublequoteopen}finite{\isacharunderscore}{\kern0pt}filtered{\isacharunderscore}{\kern0pt}measure\ M\ {\isacharparenleft}{\kern0pt}{\isasymlambda}{\isacharunderscore}{\kern0pt}{\isachardot}{\kern0pt}\ F{\isacharparenright}{\kern0pt}\ t\isactrlsub {\isadigit{0}}{\isachardoublequoteclose}\isanewline
%
\isadelimproof
\ \ %
\endisadelimproof
%
\isatagproof
\isacommand{using}\isamarkupfalse%
\ assms\ \isacommand{by}\isamarkupfalse%
\ {\isacharparenleft}{\kern0pt}unfold{\isacharunderscore}{\kern0pt}locales{\isacharparenright}{\kern0pt}\ blast{\isacharplus}{\kern0pt}%
\endisatagproof
{\isafoldproof}%
%
\isadelimproof
\isanewline
%
\endisadelimproof
%
\isadelimtheory
\isanewline
%
\endisadelimtheory
%
\isatagtheory
\isacommand{end}\isamarkupfalse%
%
\endisatagtheory
{\isafoldtheory}%
%
\isadelimtheory
%
\endisadelimtheory
%
\end{isabellebody}%
\endinput
%:%file=Filtered_Measure.tex%:%
%:%6=2%:%
%:%7=3%:%
%:%12=4%:%
%:%13=4%:%
%:%14=5%:%
%:%15=6%:%
%:%29=8%:%
%:%33=10%:%
%:%43=12%:%
%:%44=12%:%
%:%45=13%:%
%:%46=14%:%
%:%47=15%:%
%:%48=16%:%
%:%49=17%:%
%:%50=18%:%
%:%51=18%:%
%:%52=19%:%
%:%53=20%:%
%:%56=21%:%
%:%60=21%:%
%:%61=21%:%
%:%62=21%:%
%:%67=21%:%
%:%70=22%:%
%:%71=23%:%
%:%72=23%:%
%:%73=24%:%
%:%74=25%:%
%:%77=26%:%
%:%81=26%:%
%:%82=26%:%
%:%83=26%:%
%:%84=26%:%
%:%89=26%:%
%:%92=27%:%
%:%93=28%:%
%:%94=28%:%
%:%95=29%:%
%:%96=30%:%
%:%99=31%:%
%:%103=31%:%
%:%104=31%:%
%:%105=31%:%
%:%110=31%:%
%:%113=32%:%
%:%114=33%:%
%:%115=33%:%
%:%116=34%:%
%:%117=35%:%
%:%118=35%:%
%:%119=36%:%
%:%120=37%:%
%:%121=37%:%
%:%122=38%:%
%:%123=38%:%
%:%130=40%:%
%:%142=42%:%
%:%144=44%:%
%:%145=44%:%
%:%146=45%:%
%:%147=46%:%
%:%148=47%:%
%:%149=47%:%
%:%150=48%:%
%:%151=49%:%
%:%154=50%:%
%:%158=50%:%
%:%159=50%:%
%:%160=50%:%
%:%165=50%:%
%:%168=51%:%
%:%169=52%:%
%:%170=52%:%
%:%171=53%:%
%:%172=53%:%
%:%173=54%:%
%:%174=55%:%
%:%175=55%:%
%:%177=55%:%
%:%181=55%:%
%:%182=55%:%
%:%189=55%:%
%:%190=56%:%
%:%191=56%:%
%:%193=56%:%
%:%197=56%:%
%:%198=56%:%
%:%212=58%:%
%:%222=60%:%
%:%223=60%:%
%:%224=61%:%
%:%225=62%:%
%:%226=62%:%
%:%229=63%:%
%:%233=63%:%
%:%234=63%:%
%:%235=63%:%
%:%240=63%:%
%:%243=64%:%
%:%244=65%:%
%:%245=65%:%
%:%246=66%:%
%:%247=66%:%
%:%248=67%:%
%:%249=68%:%
%:%250=68%:%
%:%252=68%:%
%:%256=68%:%
%:%264=68%:%
%:%265=69%:%
%:%266=69%:%
%:%268=69%:%
%:%272=69%:%
%:%287=71%:%
%:%297=73%:%
%:%298=73%:%
%:%299=74%:%
%:%300=75%:%
%:%303=76%:%
%:%307=76%:%
%:%308=76%:%
%:%309=76%:%
%:%314=76%:%
%:%317=77%:%
%:%318=78%:%
%:%319=78%:%
%:%322=79%:%
%:%326=79%:%
%:%327=79%:%
%:%328=79%:%
%:%333=79%:%
%:%336=80%:%
%:%337=81%:%
%:%338=81%:%
%:%339=82%:%
%:%340=83%:%
%:%343=84%:%
%:%347=84%:%
%:%348=84%:%
%:%349=84%:%
%:%354=84%:%
%:%359=85%:%
%:%364=86%:%

%
\begin{isabellebody}%
\setisabellecontext{Conditional{\isacharunderscore}{\kern0pt}Expectation{\isacharunderscore}{\kern0pt}Banach}%
%
\isadelimtheory
%
\endisadelimtheory
%
\isatagtheory
\isacommand{theory}\isamarkupfalse%
\ Conditional{\isacharunderscore}{\kern0pt}Expectation{\isacharunderscore}{\kern0pt}Banach\ \ \ \ \ \ \ \ \ \ \ \ \ \ \ \ \ \ \ \ \ \ \ \ \ \ \ \ \ \ \ \ \ \ \ \ \ \ \ \ \ \ \ \ \ \ \ \ \ \ \ \ \ \ \ \ \ \ \ \ \ \ \ \ \ \isanewline
\isakeyword{imports}\ {\isachardoublequoteopen}HOL{\isacharminus}{\kern0pt}Probability{\isachardot}{\kern0pt}Conditional{\isacharunderscore}{\kern0pt}Expectation{\isachardoublequoteclose}\ Sigma{\isacharunderscore}{\kern0pt}Finite{\isacharunderscore}{\kern0pt}Measure{\isacharunderscore}{\kern0pt}Addendum\isanewline
\isakeyword{begin}%
\endisatagtheory
{\isafoldtheory}%
%
\isadelimtheory
%
\endisadelimtheory
%
\isadelimdocument
%
\endisadelimdocument
%
\isatagdocument
%
\isamarkupsection{Conditional Expectation in Banach Spaces%
}
\isamarkuptrue%
%
\endisatagdocument
{\isafolddocument}%
%
\isadelimdocument
%
\endisadelimdocument
\isacommand{definition}\isamarkupfalse%
\ has{\isacharunderscore}{\kern0pt}cond{\isacharunderscore}{\kern0pt}exp\ {\isacharcolon}{\kern0pt}{\isacharcolon}{\kern0pt}\ {\isachardoublequoteopen}{\isacharprime}{\kern0pt}a\ measure\ {\isasymRightarrow}\ {\isacharprime}{\kern0pt}a\ measure\ {\isasymRightarrow}\ {\isacharparenleft}{\kern0pt}{\isacharprime}{\kern0pt}a\ {\isasymRightarrow}\ {\isacharprime}{\kern0pt}b{\isacharparenright}{\kern0pt}\ {\isasymRightarrow}\ {\isacharparenleft}{\kern0pt}{\isacharprime}{\kern0pt}a\ {\isasymRightarrow}\ {\isacharprime}{\kern0pt}b{\isacharcolon}{\kern0pt}{\isacharcolon}{\kern0pt}{\isacharbraceleft}{\kern0pt}real{\isacharunderscore}{\kern0pt}normed{\isacharunderscore}{\kern0pt}vector{\isacharcomma}{\kern0pt}\ second{\isacharunderscore}{\kern0pt}countable{\isacharunderscore}{\kern0pt}topology{\isacharbraceright}{\kern0pt}{\isacharparenright}{\kern0pt}\ {\isasymRightarrow}\ bool{\isachardoublequoteclose}\ \isakeyword{where}\ \isanewline
\ \ {\isachardoublequoteopen}has{\isacharunderscore}{\kern0pt}cond{\isacharunderscore}{\kern0pt}exp\ M\ F\ f\ g\ {\isacharequal}{\kern0pt}\ {\isacharparenleft}{\kern0pt}{\isacharparenleft}{\kern0pt}{\isasymforall}A\ {\isasymin}\ sets\ F{\isachardot}{\kern0pt}\ {\isacharparenleft}{\kern0pt}{\isasymintegral}\ x\ {\isasymin}\ A{\isachardot}{\kern0pt}\ f\ x\ {\isasympartial}M{\isacharparenright}{\kern0pt}\ {\isacharequal}{\kern0pt}\ {\isacharparenleft}{\kern0pt}{\isasymintegral}\ x\ {\isasymin}\ A{\isachardot}{\kern0pt}\ g\ x\ {\isasympartial}M{\isacharparenright}{\kern0pt}{\isacharparenright}{\kern0pt}\isanewline
\ \ \ \ \ \ \ \ \ \ \ \ \ \ \ \ \ \ \ \ \ \ \ \ {\isasymand}\ integrable\ M\ f\ \isanewline
\ \ \ \ \ \ \ \ \ \ \ \ \ \ \ \ \ \ \ \ \ \ \ \ {\isasymand}\ integrable\ M\ g\ \isanewline
\ \ \ \ \ \ \ \ \ \ \ \ \ \ \ \ \ \ \ \ \ \ \ \ {\isasymand}\ g\ {\isasymin}\ borel{\isacharunderscore}{\kern0pt}measurable\ F{\isacharparenright}{\kern0pt}{\isachardoublequoteclose}\isanewline
\isanewline
\isacommand{lemma}\isamarkupfalse%
\ has{\isacharunderscore}{\kern0pt}cond{\isacharunderscore}{\kern0pt}expI{\isacharprime}{\kern0pt}{\isacharcolon}{\kern0pt}\isanewline
\ \ \isakeyword{assumes}\ {\isachardoublequoteopen}{\isasymAnd}A{\isachardot}{\kern0pt}\ A\ {\isasymin}\ sets\ F\ {\isasymLongrightarrow}\ {\isacharparenleft}{\kern0pt}{\isasymintegral}\ x\ {\isasymin}\ A{\isachardot}{\kern0pt}\ f\ x\ {\isasympartial}M{\isacharparenright}{\kern0pt}\ {\isacharequal}{\kern0pt}\ {\isacharparenleft}{\kern0pt}{\isasymintegral}\ x\ {\isasymin}\ A{\isachardot}{\kern0pt}\ g\ x\ {\isasympartial}M{\isacharparenright}{\kern0pt}{\isachardoublequoteclose}\isanewline
\ \ \ \ \ \ \ \ \ \ {\isachardoublequoteopen}integrable\ M\ f{\isachardoublequoteclose}\isanewline
\ \ \ \ \ \ \ \ \ \ {\isachardoublequoteopen}integrable\ M\ g{\isachardoublequoteclose}\isanewline
\ \ \ \ \ \ \ \ \ \ {\isachardoublequoteopen}g\ {\isasymin}\ borel{\isacharunderscore}{\kern0pt}measurable\ F{\isachardoublequoteclose}\isanewline
\ \ \isakeyword{shows}\ {\isachardoublequoteopen}has{\isacharunderscore}{\kern0pt}cond{\isacharunderscore}{\kern0pt}exp\ M\ F\ f\ g{\isachardoublequoteclose}\isanewline
%
\isadelimproof
\ \ %
\endisadelimproof
%
\isatagproof
\isacommand{using}\isamarkupfalse%
\ assms\ \isacommand{unfolding}\isamarkupfalse%
\ has{\isacharunderscore}{\kern0pt}cond{\isacharunderscore}{\kern0pt}exp{\isacharunderscore}{\kern0pt}def\ \isacommand{by}\isamarkupfalse%
\ simp%
\endisatagproof
{\isafoldproof}%
%
\isadelimproof
\isanewline
%
\endisadelimproof
\isanewline
\isacommand{lemma}\isamarkupfalse%
\ has{\isacharunderscore}{\kern0pt}cond{\isacharunderscore}{\kern0pt}expD{\isacharcolon}{\kern0pt}\isanewline
\ \ \isakeyword{assumes}\ {\isachardoublequoteopen}has{\isacharunderscore}{\kern0pt}cond{\isacharunderscore}{\kern0pt}exp\ M\ F\ f\ g{\isachardoublequoteclose}\isanewline
\ \ \isakeyword{shows}\ {\isachardoublequoteopen}{\isasymAnd}A{\isachardot}{\kern0pt}\ A\ {\isasymin}\ sets\ F\ {\isasymLongrightarrow}\ {\isacharparenleft}{\kern0pt}{\isasymintegral}\ x\ {\isasymin}\ A{\isachardot}{\kern0pt}\ f\ x\ {\isasympartial}M{\isacharparenright}{\kern0pt}\ {\isacharequal}{\kern0pt}\ {\isacharparenleft}{\kern0pt}{\isasymintegral}\ x\ {\isasymin}\ A{\isachardot}{\kern0pt}\ g\ x\ {\isasympartial}M{\isacharparenright}{\kern0pt}{\isachardoublequoteclose}\isanewline
\ \ \ \ \ \ \ \ {\isachardoublequoteopen}integrable\ M\ f{\isachardoublequoteclose}\isanewline
\ \ \ \ \ \ \ \ {\isachardoublequoteopen}integrable\ M\ g{\isachardoublequoteclose}\isanewline
\ \ \ \ \ \ \ \ {\isachardoublequoteopen}g\ {\isasymin}\ borel{\isacharunderscore}{\kern0pt}measurable\ F{\isachardoublequoteclose}\isanewline
%
\isadelimproof
\ \ %
\endisadelimproof
%
\isatagproof
\isacommand{using}\isamarkupfalse%
\ assms\ \isacommand{unfolding}\isamarkupfalse%
\ has{\isacharunderscore}{\kern0pt}cond{\isacharunderscore}{\kern0pt}exp{\isacharunderscore}{\kern0pt}def\ \isacommand{by}\isamarkupfalse%
\ simp{\isacharplus}{\kern0pt}%
\endisatagproof
{\isafoldproof}%
%
\isadelimproof
\isanewline
%
\endisadelimproof
\isanewline
\isanewline
\isanewline
\isacommand{definition}\isamarkupfalse%
\ cond{\isacharunderscore}{\kern0pt}exp\ {\isacharcolon}{\kern0pt}{\isacharcolon}{\kern0pt}\ {\isachardoublequoteopen}{\isacharprime}{\kern0pt}a\ measure\ {\isasymRightarrow}\ {\isacharprime}{\kern0pt}a\ measure\ {\isasymRightarrow}\ {\isacharparenleft}{\kern0pt}{\isacharprime}{\kern0pt}a\ {\isasymRightarrow}\ {\isacharprime}{\kern0pt}b{\isacharparenright}{\kern0pt}\ {\isasymRightarrow}\ {\isacharparenleft}{\kern0pt}{\isacharprime}{\kern0pt}a\ {\isasymRightarrow}\ {\isacharprime}{\kern0pt}b{\isacharcolon}{\kern0pt}{\isacharcolon}{\kern0pt}{\isacharbraceleft}{\kern0pt}banach{\isacharcomma}{\kern0pt}\ second{\isacharunderscore}{\kern0pt}countable{\isacharunderscore}{\kern0pt}topology{\isacharbraceright}{\kern0pt}{\isacharparenright}{\kern0pt}{\isachardoublequoteclose}\ \isakeyword{where}\isanewline
\ \ {\isachardoublequoteopen}cond{\isacharunderscore}{\kern0pt}exp\ M\ F\ f\ {\isacharequal}{\kern0pt}\ {\isacharparenleft}{\kern0pt}if\ {\isasymexists}g{\isachardot}{\kern0pt}\ has{\isacharunderscore}{\kern0pt}cond{\isacharunderscore}{\kern0pt}exp\ M\ F\ f\ g\ then\ {\isacharparenleft}{\kern0pt}SOME\ g{\isachardot}{\kern0pt}\ has{\isacharunderscore}{\kern0pt}cond{\isacharunderscore}{\kern0pt}exp\ M\ F\ f\ g{\isacharparenright}{\kern0pt}\ else\ {\isacharparenleft}{\kern0pt}{\isasymlambda}{\isacharunderscore}{\kern0pt}{\isachardot}{\kern0pt}\ {\isadigit{0}}{\isacharparenright}{\kern0pt}{\isacharparenright}{\kern0pt}{\isachardoublequoteclose}\isanewline
\isanewline
\isacommand{lemma}\isamarkupfalse%
\ borel{\isacharunderscore}{\kern0pt}measurable{\isacharunderscore}{\kern0pt}cond{\isacharunderscore}{\kern0pt}exp{\isacharbrackleft}{\kern0pt}measurable{\isacharbrackright}{\kern0pt}{\isacharcolon}{\kern0pt}\ {\isachardoublequoteopen}cond{\isacharunderscore}{\kern0pt}exp\ M\ F\ f\ {\isasymin}\ borel{\isacharunderscore}{\kern0pt}measurable\ F{\isachardoublequoteclose}\ \isanewline
%
\isadelimproof
\ \ %
\endisadelimproof
%
\isatagproof
\isacommand{by}\isamarkupfalse%
\ {\isacharparenleft}{\kern0pt}metis\ cond{\isacharunderscore}{\kern0pt}exp{\isacharunderscore}{\kern0pt}def\ someI\ has{\isacharunderscore}{\kern0pt}cond{\isacharunderscore}{\kern0pt}exp{\isacharunderscore}{\kern0pt}def\ borel{\isacharunderscore}{\kern0pt}measurable{\isacharunderscore}{\kern0pt}const{\isacharparenright}{\kern0pt}%
\endisatagproof
{\isafoldproof}%
%
\isadelimproof
\isanewline
%
\endisadelimproof
\isanewline
\isacommand{lemma}\isamarkupfalse%
\ integrable{\isacharunderscore}{\kern0pt}cond{\isacharunderscore}{\kern0pt}exp{\isacharbrackleft}{\kern0pt}intro{\isacharbrackright}{\kern0pt}{\isacharcolon}{\kern0pt}\ {\isachardoublequoteopen}integrable\ M\ {\isacharparenleft}{\kern0pt}cond{\isacharunderscore}{\kern0pt}exp\ M\ F\ f{\isacharparenright}{\kern0pt}{\isachardoublequoteclose}\ \isanewline
%
\isadelimproof
\ \ %
\endisadelimproof
%
\isatagproof
\isacommand{by}\isamarkupfalse%
\ {\isacharparenleft}{\kern0pt}metis\ cond{\isacharunderscore}{\kern0pt}exp{\isacharunderscore}{\kern0pt}def\ has{\isacharunderscore}{\kern0pt}cond{\isacharunderscore}{\kern0pt}expD{\isacharparenleft}{\kern0pt}{\isadigit{3}}{\isacharparenright}{\kern0pt}\ integrable{\isacharunderscore}{\kern0pt}zero\ someI{\isacharparenright}{\kern0pt}%
\endisatagproof
{\isafoldproof}%
%
\isadelimproof
\isanewline
%
\endisadelimproof
\isanewline
\isacommand{lemma}\isamarkupfalse%
\ set{\isacharunderscore}{\kern0pt}integrable{\isacharunderscore}{\kern0pt}cond{\isacharunderscore}{\kern0pt}exp{\isacharbrackleft}{\kern0pt}intro{\isacharbrackright}{\kern0pt}{\isacharcolon}{\kern0pt}\isanewline
\ \ \isakeyword{assumes}\ {\isachardoublequoteopen}A\ {\isasymin}\ sets\ M{\isachardoublequoteclose}\isanewline
\isakeyword{shows}\ {\isachardoublequoteopen}set{\isacharunderscore}{\kern0pt}integrable\ M\ A\ {\isacharparenleft}{\kern0pt}cond{\isacharunderscore}{\kern0pt}exp\ M\ F\ f{\isacharparenright}{\kern0pt}{\isachardoublequoteclose}%
\isadelimproof
\ %
\endisadelimproof
%
\isatagproof
\isacommand{using}\isamarkupfalse%
\ integrable{\isacharunderscore}{\kern0pt}mult{\isacharunderscore}{\kern0pt}indicator{\isacharbrackleft}{\kern0pt}OF\ assms\ integrable{\isacharunderscore}{\kern0pt}cond{\isacharunderscore}{\kern0pt}exp{\isacharcomma}{\kern0pt}\ of\ F\ f{\isacharbrackright}{\kern0pt}\ \isacommand{by}\isamarkupfalse%
\ {\isacharparenleft}{\kern0pt}auto\ simp\ add{\isacharcolon}{\kern0pt}\ set{\isacharunderscore}{\kern0pt}integrable{\isacharunderscore}{\kern0pt}def\ intro{\isacharbang}{\kern0pt}{\isacharcolon}{\kern0pt}\ integrable{\isacharunderscore}{\kern0pt}mult{\isacharunderscore}{\kern0pt}indicator{\isacharbrackleft}{\kern0pt}OF\ assms\ integrable{\isacharunderscore}{\kern0pt}cond{\isacharunderscore}{\kern0pt}exp{\isacharbrackright}{\kern0pt}{\isacharparenright}{\kern0pt}%
\endisatagproof
{\isafoldproof}%
%
\isadelimproof
%
\endisadelimproof
\isanewline
\isanewline
\isacommand{context}\isamarkupfalse%
\ sigma{\isacharunderscore}{\kern0pt}finite{\isacharunderscore}{\kern0pt}subalgebra\isanewline
\isakeyword{begin}\isanewline
\isanewline
\isacommand{lemma}\isamarkupfalse%
\ borel{\isacharunderscore}{\kern0pt}measurable{\isacharunderscore}{\kern0pt}cond{\isacharunderscore}{\kern0pt}exp{\isacharprime}{\kern0pt}{\isacharbrackleft}{\kern0pt}measurable{\isacharbrackright}{\kern0pt}{\isacharcolon}{\kern0pt}\ {\isachardoublequoteopen}cond{\isacharunderscore}{\kern0pt}exp\ M\ F\ f\ {\isasymin}\ borel{\isacharunderscore}{\kern0pt}measurable\ M{\isachardoublequoteclose}\isanewline
%
\isadelimproof
\ \ %
\endisadelimproof
%
\isatagproof
\isacommand{by}\isamarkupfalse%
\ {\isacharparenleft}{\kern0pt}metis\ cond{\isacharunderscore}{\kern0pt}exp{\isacharunderscore}{\kern0pt}def\ someI\ has{\isacharunderscore}{\kern0pt}cond{\isacharunderscore}{\kern0pt}exp{\isacharunderscore}{\kern0pt}def\ borel{\isacharunderscore}{\kern0pt}measurable{\isacharunderscore}{\kern0pt}const\ subalg\ measurable{\isacharunderscore}{\kern0pt}from{\isacharunderscore}{\kern0pt}subalg{\isacharparenright}{\kern0pt}%
\endisatagproof
{\isafoldproof}%
%
\isadelimproof
\isanewline
%
\endisadelimproof
\isanewline
\isacommand{lemma}\isamarkupfalse%
\ cond{\isacharunderscore}{\kern0pt}exp{\isacharunderscore}{\kern0pt}null{\isacharcolon}{\kern0pt}\ \isanewline
\ \ \isakeyword{assumes}\ {\isachardoublequoteopen}{\isasymnexists}g{\isachardot}{\kern0pt}\ has{\isacharunderscore}{\kern0pt}cond{\isacharunderscore}{\kern0pt}exp\ M\ F\ f\ g{\isachardoublequoteclose}\ \isanewline
\ \ \isakeyword{shows}\ {\isachardoublequoteopen}cond{\isacharunderscore}{\kern0pt}exp\ M\ F\ f\ {\isacharequal}{\kern0pt}\ {\isacharparenleft}{\kern0pt}{\isasymlambda}{\isacharunderscore}{\kern0pt}{\isachardot}{\kern0pt}\ {\isadigit{0}}{\isacharparenright}{\kern0pt}{\isachardoublequoteclose}\isanewline
%
\isadelimproof
\ \ %
\endisadelimproof
%
\isatagproof
\isacommand{unfolding}\isamarkupfalse%
\ cond{\isacharunderscore}{\kern0pt}exp{\isacharunderscore}{\kern0pt}def\ \isacommand{using}\isamarkupfalse%
\ assms\ \isacommand{by}\isamarkupfalse%
\ argo%
\endisatagproof
{\isafoldproof}%
%
\isadelimproof
\isanewline
%
\endisadelimproof
\isanewline
\isacommand{lemma}\isamarkupfalse%
\ has{\isacharunderscore}{\kern0pt}cond{\isacharunderscore}{\kern0pt}exp{\isacharunderscore}{\kern0pt}nested{\isacharunderscore}{\kern0pt}subalg{\isacharcolon}{\kern0pt}\isanewline
\ \ \isakeyword{fixes}\ f\ {\isacharcolon}{\kern0pt}{\isacharcolon}{\kern0pt}\ {\isachardoublequoteopen}{\isacharprime}{\kern0pt}a\ {\isasymRightarrow}\ {\isacharprime}{\kern0pt}b{\isacharcolon}{\kern0pt}{\isacharcolon}{\kern0pt}{\isacharbraceleft}{\kern0pt}second{\isacharunderscore}{\kern0pt}countable{\isacharunderscore}{\kern0pt}topology{\isacharcomma}{\kern0pt}\ banach{\isacharbraceright}{\kern0pt}{\isachardoublequoteclose}\isanewline
\ \ \isakeyword{assumes}\ {\isachardoublequoteopen}subalgebra\ G\ F{\isachardoublequoteclose}\ {\isachardoublequoteopen}has{\isacharunderscore}{\kern0pt}cond{\isacharunderscore}{\kern0pt}exp\ M\ F\ f\ h{\isachardoublequoteclose}\ {\isachardoublequoteopen}has{\isacharunderscore}{\kern0pt}cond{\isacharunderscore}{\kern0pt}exp\ M\ G\ f\ h{\isacharprime}{\kern0pt}{\isachardoublequoteclose}\isanewline
\ \ \isakeyword{shows}\ {\isachardoublequoteopen}has{\isacharunderscore}{\kern0pt}cond{\isacharunderscore}{\kern0pt}exp\ M\ F\ h{\isacharprime}{\kern0pt}\ h{\isachardoublequoteclose}\isanewline
%
\isadelimproof
\ \ %
\endisadelimproof
%
\isatagproof
\isacommand{by}\isamarkupfalse%
\ {\isacharparenleft}{\kern0pt}intro\ has{\isacharunderscore}{\kern0pt}cond{\isacharunderscore}{\kern0pt}expI{\isacharprime}{\kern0pt}{\isacharparenright}{\kern0pt}\ {\isacharparenleft}{\kern0pt}metis\ assms\ has{\isacharunderscore}{\kern0pt}cond{\isacharunderscore}{\kern0pt}expD\ in{\isacharunderscore}{\kern0pt}mono\ subalgebra{\isacharunderscore}{\kern0pt}def{\isacharparenright}{\kern0pt}{\isacharplus}{\kern0pt}%
\endisatagproof
{\isafoldproof}%
%
\isadelimproof
\isanewline
%
\endisadelimproof
\isanewline
\isacommand{lemma}\isamarkupfalse%
\ has{\isacharunderscore}{\kern0pt}cond{\isacharunderscore}{\kern0pt}exp{\isacharunderscore}{\kern0pt}charact{\isacharcolon}{\kern0pt}\isanewline
\ \ \isakeyword{fixes}\ f\ {\isacharcolon}{\kern0pt}{\isacharcolon}{\kern0pt}\ {\isachardoublequoteopen}{\isacharprime}{\kern0pt}a\ {\isasymRightarrow}\ {\isacharprime}{\kern0pt}b{\isacharcolon}{\kern0pt}{\isacharcolon}{\kern0pt}{\isacharbraceleft}{\kern0pt}second{\isacharunderscore}{\kern0pt}countable{\isacharunderscore}{\kern0pt}topology{\isacharcomma}{\kern0pt}\ banach{\isacharbraceright}{\kern0pt}{\isachardoublequoteclose}\isanewline
\ \ \isakeyword{assumes}\ {\isachardoublequoteopen}has{\isacharunderscore}{\kern0pt}cond{\isacharunderscore}{\kern0pt}exp\ M\ F\ f\ g{\isachardoublequoteclose}\isanewline
\ \ \isakeyword{shows}\ {\isachardoublequoteopen}has{\isacharunderscore}{\kern0pt}cond{\isacharunderscore}{\kern0pt}exp\ M\ F\ f\ {\isacharparenleft}{\kern0pt}cond{\isacharunderscore}{\kern0pt}exp\ M\ F\ f{\isacharparenright}{\kern0pt}{\isachardoublequoteclose}\isanewline
\ \ \ \ \ \ \ \ {\isachardoublequoteopen}AE\ x\ in\ M{\isachardot}{\kern0pt}\ cond{\isacharunderscore}{\kern0pt}exp\ M\ F\ f\ x\ {\isacharequal}{\kern0pt}\ g\ x{\isachardoublequoteclose}\isanewline
%
\isadelimproof
%
\endisadelimproof
%
\isatagproof
\isacommand{proof}\isamarkupfalse%
\ {\isacharminus}{\kern0pt}\isanewline
\ \ \isacommand{show}\isamarkupfalse%
\ cond{\isacharunderscore}{\kern0pt}exp{\isacharcolon}{\kern0pt}\ {\isachardoublequoteopen}has{\isacharunderscore}{\kern0pt}cond{\isacharunderscore}{\kern0pt}exp\ M\ F\ f\ {\isacharparenleft}{\kern0pt}cond{\isacharunderscore}{\kern0pt}exp\ M\ F\ f{\isacharparenright}{\kern0pt}{\isachardoublequoteclose}\ \isacommand{using}\isamarkupfalse%
\ assms\ someI\ cond{\isacharunderscore}{\kern0pt}exp{\isacharunderscore}{\kern0pt}def\ \isacommand{by}\isamarkupfalse%
\ metis\isanewline
\ \ \isacommand{let}\isamarkupfalse%
\ {\isacharquery}{\kern0pt}MF\ {\isacharequal}{\kern0pt}\ {\isachardoublequoteopen}restr{\isacharunderscore}{\kern0pt}to{\isacharunderscore}{\kern0pt}subalg\ M\ F{\isachardoublequoteclose}\isanewline
\ \ \isacommand{interpret}\isamarkupfalse%
\ sigma{\isacharunderscore}{\kern0pt}finite{\isacharunderscore}{\kern0pt}measure\ {\isacharquery}{\kern0pt}MF\ \isacommand{by}\isamarkupfalse%
\ {\isacharparenleft}{\kern0pt}rule\ sigma{\isacharunderscore}{\kern0pt}fin{\isacharunderscore}{\kern0pt}subalg{\isacharparenright}{\kern0pt}\isanewline
\ \ \isacommand{{\isacharbraceleft}{\kern0pt}}\isamarkupfalse%
\isanewline
\ \ \ \ \isacommand{fix}\isamarkupfalse%
\ A\ \isacommand{assume}\isamarkupfalse%
\ {\isachardoublequoteopen}A\ {\isasymin}\ sets\ {\isacharquery}{\kern0pt}MF{\isachardoublequoteclose}\isanewline
\ \ \ \ \isacommand{then}\isamarkupfalse%
\ \isacommand{have}\isamarkupfalse%
\ {\isacharbrackleft}{\kern0pt}measurable{\isacharbrackright}{\kern0pt}{\isacharcolon}{\kern0pt}\ {\isachardoublequoteopen}A\ {\isasymin}\ sets\ F{\isachardoublequoteclose}\ \isacommand{using}\isamarkupfalse%
\ sets{\isacharunderscore}{\kern0pt}restr{\isacharunderscore}{\kern0pt}to{\isacharunderscore}{\kern0pt}subalg{\isacharbrackleft}{\kern0pt}OF\ subalg{\isacharbrackright}{\kern0pt}\ \isacommand{by}\isamarkupfalse%
\ simp\isanewline
\ \ \ \ \isacommand{have}\isamarkupfalse%
\ {\isachardoublequoteopen}{\isacharparenleft}{\kern0pt}{\isasymintegral}x\ {\isasymin}\ A{\isachardot}{\kern0pt}\ g\ x\ {\isasympartial}{\isacharquery}{\kern0pt}MF{\isacharparenright}{\kern0pt}\ {\isacharequal}{\kern0pt}\ {\isacharparenleft}{\kern0pt}{\isasymintegral}x\ {\isasymin}\ A{\isachardot}{\kern0pt}\ g\ x\ {\isasympartial}M{\isacharparenright}{\kern0pt}{\isachardoublequoteclose}\ \isacommand{using}\isamarkupfalse%
\ assms\ subalg\ \isacommand{by}\isamarkupfalse%
\ {\isacharparenleft}{\kern0pt}auto\ simp\ add{\isacharcolon}{\kern0pt}\ integral{\isacharunderscore}{\kern0pt}subalgebra{\isadigit{2}}\ set{\isacharunderscore}{\kern0pt}lebesgue{\isacharunderscore}{\kern0pt}integral{\isacharunderscore}{\kern0pt}def\ dest{\isacharbang}{\kern0pt}{\isacharcolon}{\kern0pt}\ has{\isacharunderscore}{\kern0pt}cond{\isacharunderscore}{\kern0pt}expD{\isacharparenright}{\kern0pt}\isanewline
\ \ \ \ \isacommand{also}\isamarkupfalse%
\ \isacommand{have}\isamarkupfalse%
\ {\isachardoublequoteopen}{\isachardot}{\kern0pt}{\isachardot}{\kern0pt}{\isachardot}{\kern0pt}\ {\isacharequal}{\kern0pt}\ {\isacharparenleft}{\kern0pt}{\isasymintegral}x\ {\isasymin}\ A{\isachardot}{\kern0pt}\ cond{\isacharunderscore}{\kern0pt}exp\ M\ F\ f\ x\ {\isasympartial}M{\isacharparenright}{\kern0pt}{\isachardoublequoteclose}\ \isacommand{using}\isamarkupfalse%
\ assms\ cond{\isacharunderscore}{\kern0pt}exp\ \isacommand{by}\isamarkupfalse%
\ {\isacharparenleft}{\kern0pt}simp\ add{\isacharcolon}{\kern0pt}\ has{\isacharunderscore}{\kern0pt}cond{\isacharunderscore}{\kern0pt}exp{\isacharunderscore}{\kern0pt}def{\isacharparenright}{\kern0pt}\isanewline
\ \ \ \ \isacommand{also}\isamarkupfalse%
\ \isacommand{have}\isamarkupfalse%
\ {\isachardoublequoteopen}{\isachardot}{\kern0pt}{\isachardot}{\kern0pt}{\isachardot}{\kern0pt}\ {\isacharequal}{\kern0pt}\ {\isacharparenleft}{\kern0pt}{\isasymintegral}x\ {\isasymin}\ A{\isachardot}{\kern0pt}\ cond{\isacharunderscore}{\kern0pt}exp\ M\ F\ f\ x\ {\isasympartial}{\isacharquery}{\kern0pt}MF{\isacharparenright}{\kern0pt}{\isachardoublequoteclose}\ \isacommand{using}\isamarkupfalse%
\ subalg\ \isacommand{by}\isamarkupfalse%
\ {\isacharparenleft}{\kern0pt}auto\ simp\ add{\isacharcolon}{\kern0pt}\ integral{\isacharunderscore}{\kern0pt}subalgebra{\isadigit{2}}\ set{\isacharunderscore}{\kern0pt}lebesgue{\isacharunderscore}{\kern0pt}integral{\isacharunderscore}{\kern0pt}def{\isacharparenright}{\kern0pt}\isanewline
\ \ \ \ \isacommand{finally}\isamarkupfalse%
\ \isacommand{have}\isamarkupfalse%
\ {\isachardoublequoteopen}{\isacharparenleft}{\kern0pt}{\isasymintegral}x\ {\isasymin}\ A{\isachardot}{\kern0pt}\ g\ x\ {\isasympartial}{\isacharquery}{\kern0pt}MF{\isacharparenright}{\kern0pt}\ {\isacharequal}{\kern0pt}\ {\isacharparenleft}{\kern0pt}{\isasymintegral}x\ {\isasymin}\ A{\isachardot}{\kern0pt}\ cond{\isacharunderscore}{\kern0pt}exp\ M\ F\ f\ x\ {\isasympartial}{\isacharquery}{\kern0pt}MF{\isacharparenright}{\kern0pt}{\isachardoublequoteclose}\ \isacommand{by}\isamarkupfalse%
\ simp\isanewline
\ \ \isacommand{{\isacharbraceright}{\kern0pt}}\isamarkupfalse%
\isanewline
\ \ \isacommand{hence}\isamarkupfalse%
\ {\isachardoublequoteopen}AE\ x\ in\ {\isacharquery}{\kern0pt}MF{\isachardot}{\kern0pt}\ cond{\isacharunderscore}{\kern0pt}exp\ M\ F\ f\ x\ {\isacharequal}{\kern0pt}\ g\ x{\isachardoublequoteclose}\ \isacommand{using}\isamarkupfalse%
\ cond{\isacharunderscore}{\kern0pt}exp\ assms\ subalg\ \isacommand{by}\isamarkupfalse%
\ {\isacharparenleft}{\kern0pt}intro\ density{\isacharunderscore}{\kern0pt}unique{\isacharcomma}{\kern0pt}\ auto\ dest{\isacharcolon}{\kern0pt}\ has{\isacharunderscore}{\kern0pt}cond{\isacharunderscore}{\kern0pt}expD\ intro{\isacharbang}{\kern0pt}{\isacharcolon}{\kern0pt}\ integrable{\isacharunderscore}{\kern0pt}in{\isacharunderscore}{\kern0pt}subalg{\isacharparenright}{\kern0pt}\isanewline
\ \ \isacommand{then}\isamarkupfalse%
\ \isacommand{show}\isamarkupfalse%
\ {\isachardoublequoteopen}AE\ x\ in\ M{\isachardot}{\kern0pt}\ cond{\isacharunderscore}{\kern0pt}exp\ M\ F\ f\ x\ {\isacharequal}{\kern0pt}\ g\ x{\isachardoublequoteclose}\ \isacommand{using}\isamarkupfalse%
\ AE{\isacharunderscore}{\kern0pt}restr{\isacharunderscore}{\kern0pt}to{\isacharunderscore}{\kern0pt}subalg{\isacharbrackleft}{\kern0pt}OF\ subalg{\isacharbrackright}{\kern0pt}\ \isacommand{by}\isamarkupfalse%
\ simp\isanewline
\isacommand{qed}\isamarkupfalse%
%
\endisatagproof
{\isafoldproof}%
%
\isadelimproof
\isanewline
%
\endisadelimproof
\isanewline
\isacommand{lemma}\isamarkupfalse%
\ cond{\isacharunderscore}{\kern0pt}exp{\isacharunderscore}{\kern0pt}charact{\isacharcolon}{\kern0pt}\isanewline
\ \ \isakeyword{fixes}\ f\ {\isacharcolon}{\kern0pt}{\isacharcolon}{\kern0pt}\ {\isachardoublequoteopen}{\isacharprime}{\kern0pt}a\ {\isasymRightarrow}\ {\isacharprime}{\kern0pt}b{\isacharcolon}{\kern0pt}{\isacharcolon}{\kern0pt}{\isacharbraceleft}{\kern0pt}second{\isacharunderscore}{\kern0pt}countable{\isacharunderscore}{\kern0pt}topology{\isacharcomma}{\kern0pt}\ banach{\isacharbraceright}{\kern0pt}{\isachardoublequoteclose}\isanewline
\ \ \isakeyword{assumes}\ {\isachardoublequoteopen}{\isasymAnd}A{\isachardot}{\kern0pt}\ A\ {\isasymin}\ sets\ F\ {\isasymLongrightarrow}\ {\isacharparenleft}{\kern0pt}{\isasymintegral}\ x\ {\isasymin}\ A{\isachardot}{\kern0pt}\ f\ x\ {\isasympartial}M{\isacharparenright}{\kern0pt}\ {\isacharequal}{\kern0pt}\ {\isacharparenleft}{\kern0pt}{\isasymintegral}\ x\ {\isasymin}\ A{\isachardot}{\kern0pt}\ g\ x\ {\isasympartial}M{\isacharparenright}{\kern0pt}{\isachardoublequoteclose}\isanewline
\ \ \ \ \ \ \ \ \ \ {\isachardoublequoteopen}integrable\ M\ f{\isachardoublequoteclose}\isanewline
\ \ \ \ \ \ \ \ \ \ {\isachardoublequoteopen}integrable\ M\ g{\isachardoublequoteclose}\isanewline
\ \ \ \ \ \ \ \ \ \ {\isachardoublequoteopen}g\ {\isasymin}\ borel{\isacharunderscore}{\kern0pt}measurable\ F{\isachardoublequoteclose}\isanewline
\ \ \ \ \isakeyword{shows}\ {\isachardoublequoteopen}AE\ x\ in\ M{\isachardot}{\kern0pt}\ cond{\isacharunderscore}{\kern0pt}exp\ M\ F\ f\ x\ {\isacharequal}{\kern0pt}\ g\ x{\isachardoublequoteclose}\isanewline
%
\isadelimproof
\ \ %
\endisadelimproof
%
\isatagproof
\isacommand{by}\isamarkupfalse%
\ {\isacharparenleft}{\kern0pt}intro\ has{\isacharunderscore}{\kern0pt}cond{\isacharunderscore}{\kern0pt}exp{\isacharunderscore}{\kern0pt}charact\ has{\isacharunderscore}{\kern0pt}cond{\isacharunderscore}{\kern0pt}expI{\isacharprime}{\kern0pt}\ assms{\isacharparenright}{\kern0pt}\ auto%
\endisatagproof
{\isafoldproof}%
%
\isadelimproof
\isanewline
%
\endisadelimproof
\isanewline
\isacommand{corollary}\isamarkupfalse%
\ cond{\isacharunderscore}{\kern0pt}exp{\isacharunderscore}{\kern0pt}F{\isacharunderscore}{\kern0pt}meas{\isacharbrackleft}{\kern0pt}intro{\isacharcomma}{\kern0pt}\ simp{\isacharbrackright}{\kern0pt}{\isacharcolon}{\kern0pt}\isanewline
\ \ \isakeyword{fixes}\ f\ {\isacharcolon}{\kern0pt}{\isacharcolon}{\kern0pt}\ {\isachardoublequoteopen}{\isacharprime}{\kern0pt}a\ {\isasymRightarrow}\ {\isacharprime}{\kern0pt}b{\isacharcolon}{\kern0pt}{\isacharcolon}{\kern0pt}{\isacharbraceleft}{\kern0pt}second{\isacharunderscore}{\kern0pt}countable{\isacharunderscore}{\kern0pt}topology{\isacharcomma}{\kern0pt}\ banach{\isacharbraceright}{\kern0pt}{\isachardoublequoteclose}\isanewline
\ \ \isakeyword{assumes}\ {\isachardoublequoteopen}integrable\ M\ f{\isachardoublequoteclose}\isanewline
\ \ \ \ \ \ \ \ \ \ {\isachardoublequoteopen}f\ {\isasymin}\ borel{\isacharunderscore}{\kern0pt}measurable\ F{\isachardoublequoteclose}\isanewline
\ \ \ \ \isakeyword{shows}\ {\isachardoublequoteopen}AE\ x\ in\ M{\isachardot}{\kern0pt}\ cond{\isacharunderscore}{\kern0pt}exp\ M\ F\ f\ x\ {\isacharequal}{\kern0pt}\ f\ x{\isachardoublequoteclose}\isanewline
%
\isadelimproof
\ \ %
\endisadelimproof
%
\isatagproof
\isacommand{by}\isamarkupfalse%
\ {\isacharparenleft}{\kern0pt}rule\ cond{\isacharunderscore}{\kern0pt}exp{\isacharunderscore}{\kern0pt}charact{\isacharcomma}{\kern0pt}\ auto\ intro{\isacharcolon}{\kern0pt}\ assms{\isacharparenright}{\kern0pt}%
\endisatagproof
{\isafoldproof}%
%
\isadelimproof
%
\endisadelimproof
%
\begin{isamarkuptext}%
Congruence%
\end{isamarkuptext}\isamarkuptrue%
\isacommand{lemma}\isamarkupfalse%
\ has{\isacharunderscore}{\kern0pt}cond{\isacharunderscore}{\kern0pt}exp{\isacharunderscore}{\kern0pt}cong{\isacharcolon}{\kern0pt}\isanewline
\ \ \isakeyword{assumes}\ {\isachardoublequoteopen}integrable\ M\ f{\isachardoublequoteclose}\ {\isachardoublequoteopen}{\isasymAnd}x{\isachardot}{\kern0pt}\ x\ {\isasymin}\ space\ M\ {\isasymLongrightarrow}\ f\ x\ {\isacharequal}{\kern0pt}\ g\ x{\isachardoublequoteclose}\ {\isachardoublequoteopen}has{\isacharunderscore}{\kern0pt}cond{\isacharunderscore}{\kern0pt}exp\ M\ F\ g\ h{\isachardoublequoteclose}\isanewline
\ \ \isakeyword{shows}\ {\isachardoublequoteopen}has{\isacharunderscore}{\kern0pt}cond{\isacharunderscore}{\kern0pt}exp\ M\ F\ f\ h{\isachardoublequoteclose}\isanewline
%
\isadelimproof
%
\endisadelimproof
%
\isatagproof
\isacommand{proof}\isamarkupfalse%
\ {\isacharparenleft}{\kern0pt}intro\ has{\isacharunderscore}{\kern0pt}cond{\isacharunderscore}{\kern0pt}expI{\isacharprime}{\kern0pt}{\isacharbrackleft}{\kern0pt}OF\ {\isacharunderscore}{\kern0pt}\ assms{\isacharparenleft}{\kern0pt}{\isadigit{1}}{\isacharparenright}{\kern0pt}{\isacharbrackright}{\kern0pt}{\isacharcomma}{\kern0pt}\ goal{\isacharunderscore}{\kern0pt}cases{\isacharparenright}{\kern0pt}\isanewline
\ \ \isacommand{case}\isamarkupfalse%
\ {\isacharparenleft}{\kern0pt}{\isadigit{1}}\ A{\isacharparenright}{\kern0pt}\isanewline
\ \ \isacommand{hence}\isamarkupfalse%
\ {\isachardoublequoteopen}set{\isacharunderscore}{\kern0pt}lebesgue{\isacharunderscore}{\kern0pt}integral\ M\ A\ f\ {\isacharequal}{\kern0pt}\ set{\isacharunderscore}{\kern0pt}lebesgue{\isacharunderscore}{\kern0pt}integral\ M\ A\ g{\isachardoublequoteclose}\ \isacommand{by}\isamarkupfalse%
\ {\isacharparenleft}{\kern0pt}intro\ set{\isacharunderscore}{\kern0pt}lebesgue{\isacharunderscore}{\kern0pt}integral{\isacharunderscore}{\kern0pt}cong{\isacharparenright}{\kern0pt}\ {\isacharparenleft}{\kern0pt}meson\ assms{\isacharparenleft}{\kern0pt}{\isadigit{2}}{\isacharparenright}{\kern0pt}\ subalg\ in{\isacharunderscore}{\kern0pt}mono\ subalgebra{\isacharunderscore}{\kern0pt}def\ sets{\isachardot}{\kern0pt}sets{\isacharunderscore}{\kern0pt}into{\isacharunderscore}{\kern0pt}space\ subalgebra{\isacharunderscore}{\kern0pt}def\ subsetD{\isacharparenright}{\kern0pt}{\isacharplus}{\kern0pt}\isanewline
\ \ \isacommand{then}\isamarkupfalse%
\ \isacommand{show}\isamarkupfalse%
\ {\isacharquery}{\kern0pt}case\ \isacommand{using}\isamarkupfalse%
\ {\isadigit{1}}\ assms{\isacharparenleft}{\kern0pt}{\isadigit{3}}{\isacharparenright}{\kern0pt}\ \isacommand{by}\isamarkupfalse%
\ {\isacharparenleft}{\kern0pt}simp\ add{\isacharcolon}{\kern0pt}\ has{\isacharunderscore}{\kern0pt}cond{\isacharunderscore}{\kern0pt}exp{\isacharunderscore}{\kern0pt}def{\isacharparenright}{\kern0pt}\isanewline
\isacommand{qed}\isamarkupfalse%
\ {\isacharparenleft}{\kern0pt}auto\ simp\ add{\isacharcolon}{\kern0pt}\ has{\isacharunderscore}{\kern0pt}cond{\isacharunderscore}{\kern0pt}expD{\isacharbrackleft}{\kern0pt}OF\ assms{\isacharparenleft}{\kern0pt}{\isadigit{3}}{\isacharparenright}{\kern0pt}{\isacharbrackright}{\kern0pt}{\isacharparenright}{\kern0pt}%
\endisatagproof
{\isafoldproof}%
%
\isadelimproof
\isanewline
%
\endisadelimproof
\isanewline
\isacommand{lemma}\isamarkupfalse%
\ cond{\isacharunderscore}{\kern0pt}exp{\isacharunderscore}{\kern0pt}cong{\isacharcolon}{\kern0pt}\isanewline
\ \ \isakeyword{fixes}\ f\ {\isacharcolon}{\kern0pt}{\isacharcolon}{\kern0pt}\ {\isachardoublequoteopen}{\isacharprime}{\kern0pt}a\ {\isasymRightarrow}\ {\isacharprime}{\kern0pt}b{\isacharcolon}{\kern0pt}{\isacharcolon}{\kern0pt}{\isacharbraceleft}{\kern0pt}second{\isacharunderscore}{\kern0pt}countable{\isacharunderscore}{\kern0pt}topology{\isacharcomma}{\kern0pt}banach{\isacharbraceright}{\kern0pt}{\isachardoublequoteclose}\isanewline
\ \ \isakeyword{assumes}\ {\isachardoublequoteopen}integrable\ M\ f{\isachardoublequoteclose}\ {\isachardoublequoteopen}integrable\ M\ g{\isachardoublequoteclose}\ {\isachardoublequoteopen}{\isasymAnd}x{\isachardot}{\kern0pt}\ x\ {\isasymin}\ space\ M\ {\isasymLongrightarrow}\ f\ x\ {\isacharequal}{\kern0pt}\ g\ x{\isachardoublequoteclose}\isanewline
\ \ \isakeyword{shows}\ {\isachardoublequoteopen}AE\ x\ in\ M{\isachardot}{\kern0pt}\ cond{\isacharunderscore}{\kern0pt}exp\ M\ F\ f\ x\ {\isacharequal}{\kern0pt}\ cond{\isacharunderscore}{\kern0pt}exp\ M\ F\ g\ x{\isachardoublequoteclose}\isanewline
%
\isadelimproof
%
\endisadelimproof
%
\isatagproof
\isacommand{proof}\isamarkupfalse%
\ {\isacharparenleft}{\kern0pt}cases\ {\isachardoublequoteopen}{\isasymexists}h{\isachardot}{\kern0pt}\ has{\isacharunderscore}{\kern0pt}cond{\isacharunderscore}{\kern0pt}exp\ M\ F\ f\ h{\isachardoublequoteclose}{\isacharparenright}{\kern0pt}\isanewline
\ \ \isacommand{case}\isamarkupfalse%
\ True\isanewline
\ \ \isacommand{then}\isamarkupfalse%
\ \isacommand{obtain}\isamarkupfalse%
\ h\ \isakeyword{where}\ h{\isacharcolon}{\kern0pt}\ {\isachardoublequoteopen}has{\isacharunderscore}{\kern0pt}cond{\isacharunderscore}{\kern0pt}exp\ M\ F\ f\ h{\isachardoublequoteclose}\ {\isachardoublequoteopen}has{\isacharunderscore}{\kern0pt}cond{\isacharunderscore}{\kern0pt}exp\ M\ F\ g\ h{\isachardoublequoteclose}\ \isacommand{using}\isamarkupfalse%
\ has{\isacharunderscore}{\kern0pt}cond{\isacharunderscore}{\kern0pt}exp{\isacharunderscore}{\kern0pt}cong\ assms\ \isacommand{by}\isamarkupfalse%
\ metis\ \isanewline
\ \ \isacommand{show}\isamarkupfalse%
\ {\isacharquery}{\kern0pt}thesis\ \isacommand{using}\isamarkupfalse%
\ h{\isacharbrackleft}{\kern0pt}THEN\ has{\isacharunderscore}{\kern0pt}cond{\isacharunderscore}{\kern0pt}exp{\isacharunderscore}{\kern0pt}charact{\isacharparenleft}{\kern0pt}{\isadigit{2}}{\isacharparenright}{\kern0pt}{\isacharbrackright}{\kern0pt}\ \isacommand{by}\isamarkupfalse%
\ fastforce\isanewline
\isacommand{next}\isamarkupfalse%
\isanewline
\ \ \isacommand{case}\isamarkupfalse%
\ False\isanewline
\ \ \isacommand{moreover}\isamarkupfalse%
\ \isacommand{have}\isamarkupfalse%
\ {\isachardoublequoteopen}{\isasymnexists}h{\isachardot}{\kern0pt}\ has{\isacharunderscore}{\kern0pt}cond{\isacharunderscore}{\kern0pt}exp\ M\ F\ g\ h{\isachardoublequoteclose}\ \isacommand{using}\isamarkupfalse%
\ False\ has{\isacharunderscore}{\kern0pt}cond{\isacharunderscore}{\kern0pt}exp{\isacharunderscore}{\kern0pt}cong\ assms\ \isacommand{by}\isamarkupfalse%
\ auto\isanewline
\ \ \isacommand{ultimately}\isamarkupfalse%
\ \isacommand{show}\isamarkupfalse%
\ {\isacharquery}{\kern0pt}thesis\ \isacommand{unfolding}\isamarkupfalse%
\ cond{\isacharunderscore}{\kern0pt}exp{\isacharunderscore}{\kern0pt}def\ \isacommand{by}\isamarkupfalse%
\ auto\isanewline
\isacommand{qed}\isamarkupfalse%
%
\endisatagproof
{\isafoldproof}%
%
\isadelimproof
\isanewline
%
\endisadelimproof
\isanewline
\isacommand{lemma}\isamarkupfalse%
\ has{\isacharunderscore}{\kern0pt}cond{\isacharunderscore}{\kern0pt}exp{\isacharunderscore}{\kern0pt}cong{\isacharunderscore}{\kern0pt}AE{\isacharcolon}{\kern0pt}\isanewline
\ \ \isakeyword{assumes}\ {\isachardoublequoteopen}integrable\ M\ f{\isachardoublequoteclose}\ {\isachardoublequoteopen}AE\ x\ in\ M{\isachardot}{\kern0pt}\ f\ x\ {\isacharequal}{\kern0pt}\ g\ x{\isachardoublequoteclose}\ {\isachardoublequoteopen}has{\isacharunderscore}{\kern0pt}cond{\isacharunderscore}{\kern0pt}exp\ M\ F\ g\ h{\isachardoublequoteclose}\isanewline
\ \ \isakeyword{shows}\ {\isachardoublequoteopen}has{\isacharunderscore}{\kern0pt}cond{\isacharunderscore}{\kern0pt}exp\ M\ F\ f\ h{\isachardoublequoteclose}\isanewline
%
\isadelimproof
\ \ %
\endisadelimproof
%
\isatagproof
\isacommand{using}\isamarkupfalse%
\ assms{\isacharparenleft}{\kern0pt}{\isadigit{1}}{\isacharcomma}{\kern0pt}{\isadigit{2}}{\isacharparenright}{\kern0pt}\ subalg\ subalgebra{\isacharunderscore}{\kern0pt}def\ subset{\isacharunderscore}{\kern0pt}iff\ \isanewline
\ \ \isacommand{by}\isamarkupfalse%
\ {\isacharparenleft}{\kern0pt}intro\ has{\isacharunderscore}{\kern0pt}cond{\isacharunderscore}{\kern0pt}expI{\isacharprime}{\kern0pt}{\isacharcomma}{\kern0pt}\ subst\ set{\isacharunderscore}{\kern0pt}lebesgue{\isacharunderscore}{\kern0pt}integral{\isacharunderscore}{\kern0pt}cong{\isacharunderscore}{\kern0pt}AE{\isacharbrackleft}{\kern0pt}OF\ {\isacharunderscore}{\kern0pt}\ assms{\isacharparenleft}{\kern0pt}{\isadigit{1}}{\isacharparenright}{\kern0pt}{\isacharbrackleft}{\kern0pt}THEN\ borel{\isacharunderscore}{\kern0pt}measurable{\isacharunderscore}{\kern0pt}integrable{\isacharbrackright}{\kern0pt}\ borel{\isacharunderscore}{\kern0pt}measurable{\isacharunderscore}{\kern0pt}integrable{\isacharparenleft}{\kern0pt}{\isadigit{1}}{\isacharparenright}{\kern0pt}{\isacharbrackleft}{\kern0pt}OF\ has{\isacharunderscore}{\kern0pt}cond{\isacharunderscore}{\kern0pt}expD{\isacharparenleft}{\kern0pt}{\isadigit{2}}{\isacharparenright}{\kern0pt}{\isacharbrackleft}{\kern0pt}OF\ assms{\isacharparenleft}{\kern0pt}{\isadigit{3}}{\isacharparenright}{\kern0pt}{\isacharbrackright}{\kern0pt}{\isacharbrackright}{\kern0pt}{\isacharbrackright}{\kern0pt}{\isacharparenright}{\kern0pt}\ \isanewline
\ \ \ \ \ {\isacharparenleft}{\kern0pt}fast\ intro{\isacharcolon}{\kern0pt}\ has{\isacharunderscore}{\kern0pt}cond{\isacharunderscore}{\kern0pt}expD{\isacharbrackleft}{\kern0pt}OF\ assms{\isacharparenleft}{\kern0pt}{\isadigit{3}}{\isacharparenright}{\kern0pt}{\isacharbrackright}{\kern0pt}\ integrable{\isacharunderscore}{\kern0pt}cong{\isacharunderscore}{\kern0pt}AE{\isacharunderscore}{\kern0pt}imp{\isacharbrackleft}{\kern0pt}OF\ {\isacharunderscore}{\kern0pt}\ {\isacharunderscore}{\kern0pt}\ AE{\isacharunderscore}{\kern0pt}symmetric{\isacharbrackright}{\kern0pt}{\isacharparenright}{\kern0pt}{\isacharplus}{\kern0pt}%
\endisatagproof
{\isafoldproof}%
%
\isadelimproof
\isanewline
%
\endisadelimproof
\isanewline
\isacommand{lemma}\isamarkupfalse%
\ has{\isacharunderscore}{\kern0pt}cond{\isacharunderscore}{\kern0pt}exp{\isacharunderscore}{\kern0pt}cong{\isacharunderscore}{\kern0pt}AE{\isacharprime}{\kern0pt}{\isacharcolon}{\kern0pt}\isanewline
\ \ \isakeyword{assumes}\ {\isachardoublequoteopen}h\ {\isasymin}\ borel{\isacharunderscore}{\kern0pt}measurable\ F{\isachardoublequoteclose}\ {\isachardoublequoteopen}AE\ x\ in\ M{\isachardot}{\kern0pt}\ h\ x\ {\isacharequal}{\kern0pt}\ h{\isacharprime}{\kern0pt}\ x{\isachardoublequoteclose}\ {\isachardoublequoteopen}has{\isacharunderscore}{\kern0pt}cond{\isacharunderscore}{\kern0pt}exp\ M\ F\ f\ h{\isacharprime}{\kern0pt}{\isachardoublequoteclose}\isanewline
\ \ \isakeyword{shows}\ {\isachardoublequoteopen}has{\isacharunderscore}{\kern0pt}cond{\isacharunderscore}{\kern0pt}exp\ M\ F\ f\ h{\isachardoublequoteclose}\isanewline
%
\isadelimproof
\ \ %
\endisadelimproof
%
\isatagproof
\isacommand{using}\isamarkupfalse%
\ assms{\isacharparenleft}{\kern0pt}{\isadigit{1}}{\isacharcomma}{\kern0pt}\ {\isadigit{2}}{\isacharparenright}{\kern0pt}\ subalg\ subalgebra{\isacharunderscore}{\kern0pt}def\ subset{\isacharunderscore}{\kern0pt}iff\isanewline
\ \ \isacommand{using}\isamarkupfalse%
\ AE{\isacharunderscore}{\kern0pt}restr{\isacharunderscore}{\kern0pt}to{\isacharunderscore}{\kern0pt}subalg{\isadigit{2}}{\isacharbrackleft}{\kern0pt}OF\ subalg\ assms{\isacharparenleft}{\kern0pt}{\isadigit{2}}{\isacharparenright}{\kern0pt}{\isacharbrackright}{\kern0pt}\ measurable{\isacharunderscore}{\kern0pt}from{\isacharunderscore}{\kern0pt}subalg\isanewline
\ \ \isacommand{by}\isamarkupfalse%
\ {\isacharparenleft}{\kern0pt}intro\ has{\isacharunderscore}{\kern0pt}cond{\isacharunderscore}{\kern0pt}expI{\isacharprime}{\kern0pt}\ {\isacharcomma}{\kern0pt}\ subst\ set{\isacharunderscore}{\kern0pt}lebesgue{\isacharunderscore}{\kern0pt}integral{\isacharunderscore}{\kern0pt}cong{\isacharunderscore}{\kern0pt}AE{\isacharbrackleft}{\kern0pt}OF\ {\isacharunderscore}{\kern0pt}\ measurable{\isacharunderscore}{\kern0pt}from{\isacharunderscore}{\kern0pt}subalg{\isacharparenleft}{\kern0pt}{\isadigit{1}}{\isacharcomma}{\kern0pt}{\isadigit{1}}{\isacharparenright}{\kern0pt}{\isacharbrackleft}{\kern0pt}OF\ subalg{\isacharbrackright}{\kern0pt}{\isacharcomma}{\kern0pt}\ OF\ {\isacharunderscore}{\kern0pt}\ assms{\isacharparenleft}{\kern0pt}{\isadigit{1}}{\isacharparenright}{\kern0pt}\ has{\isacharunderscore}{\kern0pt}cond{\isacharunderscore}{\kern0pt}expD{\isacharparenleft}{\kern0pt}{\isadigit{4}}{\isacharparenright}{\kern0pt}{\isacharbrackleft}{\kern0pt}OF\ assms{\isacharparenleft}{\kern0pt}{\isadigit{3}}{\isacharparenright}{\kern0pt}{\isacharbrackright}{\kern0pt}{\isacharbrackright}{\kern0pt}{\isacharparenright}{\kern0pt}\isanewline
\ \ \ \ \ {\isacharparenleft}{\kern0pt}fast\ intro{\isacharcolon}{\kern0pt}\ has{\isacharunderscore}{\kern0pt}cond{\isacharunderscore}{\kern0pt}expD{\isacharbrackleft}{\kern0pt}OF\ assms{\isacharparenleft}{\kern0pt}{\isadigit{3}}{\isacharparenright}{\kern0pt}{\isacharbrackright}{\kern0pt}\ integrable{\isacharunderscore}{\kern0pt}cong{\isacharunderscore}{\kern0pt}AE{\isacharunderscore}{\kern0pt}imp{\isacharbrackleft}{\kern0pt}OF\ {\isacharunderscore}{\kern0pt}\ {\isacharunderscore}{\kern0pt}\ AE{\isacharunderscore}{\kern0pt}symmetric{\isacharbrackright}{\kern0pt}{\isacharparenright}{\kern0pt}{\isacharplus}{\kern0pt}%
\endisatagproof
{\isafoldproof}%
%
\isadelimproof
\isanewline
%
\endisadelimproof
\isanewline
\isacommand{lemma}\isamarkupfalse%
\ cond{\isacharunderscore}{\kern0pt}exp{\isacharunderscore}{\kern0pt}cong{\isacharunderscore}{\kern0pt}AE{\isacharcolon}{\kern0pt}\isanewline
\ \ \isakeyword{fixes}\ f\ {\isacharcolon}{\kern0pt}{\isacharcolon}{\kern0pt}\ {\isachardoublequoteopen}{\isacharprime}{\kern0pt}a\ {\isasymRightarrow}\ {\isacharprime}{\kern0pt}b{\isacharcolon}{\kern0pt}{\isacharcolon}{\kern0pt}{\isacharbraceleft}{\kern0pt}second{\isacharunderscore}{\kern0pt}countable{\isacharunderscore}{\kern0pt}topology{\isacharcomma}{\kern0pt}banach{\isacharbraceright}{\kern0pt}{\isachardoublequoteclose}\isanewline
\ \ \isakeyword{assumes}\ {\isachardoublequoteopen}integrable\ M\ f{\isachardoublequoteclose}\ {\isachardoublequoteopen}integrable\ M\ g{\isachardoublequoteclose}\ {\isachardoublequoteopen}AE\ x\ in\ M{\isachardot}{\kern0pt}\ f\ x\ {\isacharequal}{\kern0pt}\ g\ x{\isachardoublequoteclose}\isanewline
\ \ \isakeyword{shows}\ {\isachardoublequoteopen}AE\ x\ in\ M{\isachardot}{\kern0pt}\ cond{\isacharunderscore}{\kern0pt}exp\ M\ F\ f\ x\ {\isacharequal}{\kern0pt}\ cond{\isacharunderscore}{\kern0pt}exp\ M\ F\ g\ x{\isachardoublequoteclose}\isanewline
%
\isadelimproof
%
\endisadelimproof
%
\isatagproof
\isacommand{proof}\isamarkupfalse%
\ {\isacharparenleft}{\kern0pt}cases\ {\isachardoublequoteopen}{\isasymexists}h{\isachardot}{\kern0pt}\ has{\isacharunderscore}{\kern0pt}cond{\isacharunderscore}{\kern0pt}exp\ M\ F\ f\ h{\isachardoublequoteclose}{\isacharparenright}{\kern0pt}\isanewline
\ \ \isacommand{case}\isamarkupfalse%
\ True\isanewline
\ \ \isacommand{then}\isamarkupfalse%
\ \isacommand{obtain}\isamarkupfalse%
\ h\ \isakeyword{where}\ h{\isacharcolon}{\kern0pt}\ {\isachardoublequoteopen}has{\isacharunderscore}{\kern0pt}cond{\isacharunderscore}{\kern0pt}exp\ M\ F\ f\ h{\isachardoublequoteclose}\ {\isachardoublequoteopen}has{\isacharunderscore}{\kern0pt}cond{\isacharunderscore}{\kern0pt}exp\ M\ F\ g\ h{\isachardoublequoteclose}\ \isacommand{using}\isamarkupfalse%
\ has{\isacharunderscore}{\kern0pt}cond{\isacharunderscore}{\kern0pt}exp{\isacharunderscore}{\kern0pt}cong{\isacharunderscore}{\kern0pt}AE\ assms\ \isacommand{by}\isamarkupfalse%
\ {\isacharparenleft}{\kern0pt}metis\ {\isacharparenleft}{\kern0pt}mono{\isacharunderscore}{\kern0pt}tags{\isacharcomma}{\kern0pt}\ lifting{\isacharparenright}{\kern0pt}\ eventually{\isacharunderscore}{\kern0pt}mono{\isacharparenright}{\kern0pt}\isanewline
\ \ \isacommand{show}\isamarkupfalse%
\ {\isacharquery}{\kern0pt}thesis\ \isacommand{using}\isamarkupfalse%
\ h{\isacharbrackleft}{\kern0pt}THEN\ has{\isacharunderscore}{\kern0pt}cond{\isacharunderscore}{\kern0pt}exp{\isacharunderscore}{\kern0pt}charact{\isacharparenleft}{\kern0pt}{\isadigit{2}}{\isacharparenright}{\kern0pt}{\isacharbrackright}{\kern0pt}\ \isacommand{by}\isamarkupfalse%
\ fastforce\isanewline
\isacommand{next}\isamarkupfalse%
\isanewline
\ \ \isacommand{case}\isamarkupfalse%
\ False\isanewline
\ \ \isacommand{moreover}\isamarkupfalse%
\ \isacommand{have}\isamarkupfalse%
\ {\isachardoublequoteopen}{\isasymnexists}h{\isachardot}{\kern0pt}\ has{\isacharunderscore}{\kern0pt}cond{\isacharunderscore}{\kern0pt}exp\ M\ F\ g\ h{\isachardoublequoteclose}\ \isacommand{using}\isamarkupfalse%
\ False\ has{\isacharunderscore}{\kern0pt}cond{\isacharunderscore}{\kern0pt}exp{\isacharunderscore}{\kern0pt}cong{\isacharunderscore}{\kern0pt}AE\ assms\ \isacommand{by}\isamarkupfalse%
\ auto\isanewline
\ \ \isacommand{ultimately}\isamarkupfalse%
\ \isacommand{show}\isamarkupfalse%
\ {\isacharquery}{\kern0pt}thesis\ \isacommand{unfolding}\isamarkupfalse%
\ cond{\isacharunderscore}{\kern0pt}exp{\isacharunderscore}{\kern0pt}def\ \isacommand{by}\isamarkupfalse%
\ auto\isanewline
\isacommand{qed}\isamarkupfalse%
%
\endisatagproof
{\isafoldproof}%
%
\isadelimproof
\isanewline
%
\endisadelimproof
\isanewline
\isacommand{lemma}\isamarkupfalse%
\ has{\isacharunderscore}{\kern0pt}cond{\isacharunderscore}{\kern0pt}exp{\isacharunderscore}{\kern0pt}real{\isacharcolon}{\kern0pt}\isanewline
\ \ \isakeyword{fixes}\ f\ {\isacharcolon}{\kern0pt}{\isacharcolon}{\kern0pt}\ {\isachardoublequoteopen}{\isacharprime}{\kern0pt}a\ {\isasymRightarrow}\ real{\isachardoublequoteclose}\isanewline
\ \ \isakeyword{assumes}\ {\isachardoublequoteopen}integrable\ M\ f{\isachardoublequoteclose}\isanewline
\ \ \isakeyword{shows}\ {\isachardoublequoteopen}has{\isacharunderscore}{\kern0pt}cond{\isacharunderscore}{\kern0pt}exp\ M\ F\ f\ {\isacharparenleft}{\kern0pt}real{\isacharunderscore}{\kern0pt}cond{\isacharunderscore}{\kern0pt}exp\ M\ F\ f{\isacharparenright}{\kern0pt}{\isachardoublequoteclose}\isanewline
%
\isadelimproof
\ \ %
\endisadelimproof
%
\isatagproof
\isacommand{by}\isamarkupfalse%
\ {\isacharparenleft}{\kern0pt}intro\ has{\isacharunderscore}{\kern0pt}cond{\isacharunderscore}{\kern0pt}expI{\isacharprime}{\kern0pt}{\isacharcomma}{\kern0pt}\ auto\ intro{\isacharbang}{\kern0pt}{\isacharcolon}{\kern0pt}\ real{\isacharunderscore}{\kern0pt}cond{\isacharunderscore}{\kern0pt}exp{\isacharunderscore}{\kern0pt}intA\ assms{\isacharparenright}{\kern0pt}%
\endisatagproof
{\isafoldproof}%
%
\isadelimproof
\isanewline
%
\endisadelimproof
\isanewline
\isacommand{lemma}\isamarkupfalse%
\ cond{\isacharunderscore}{\kern0pt}exp{\isacharunderscore}{\kern0pt}real{\isacharbrackleft}{\kern0pt}intro{\isacharbrackright}{\kern0pt}{\isacharcolon}{\kern0pt}\isanewline
\ \ \isakeyword{fixes}\ f\ {\isacharcolon}{\kern0pt}{\isacharcolon}{\kern0pt}\ {\isachardoublequoteopen}{\isacharprime}{\kern0pt}a\ {\isasymRightarrow}\ real{\isachardoublequoteclose}\isanewline
\ \ \isakeyword{assumes}\ {\isachardoublequoteopen}integrable\ M\ f{\isachardoublequoteclose}\isanewline
\ \ \isakeyword{shows}\ {\isachardoublequoteopen}AE\ x\ in\ M{\isachardot}{\kern0pt}\ cond{\isacharunderscore}{\kern0pt}exp\ M\ F\ f\ x\ {\isacharequal}{\kern0pt}\ real{\isacharunderscore}{\kern0pt}cond{\isacharunderscore}{\kern0pt}exp\ M\ F\ f\ x{\isachardoublequoteclose}\ \isanewline
%
\isadelimproof
\ \ %
\endisadelimproof
%
\isatagproof
\isacommand{using}\isamarkupfalse%
\ has{\isacharunderscore}{\kern0pt}cond{\isacharunderscore}{\kern0pt}exp{\isacharunderscore}{\kern0pt}charact\ has{\isacharunderscore}{\kern0pt}cond{\isacharunderscore}{\kern0pt}exp{\isacharunderscore}{\kern0pt}real\ assms\ \isacommand{by}\isamarkupfalse%
\ blast%
\endisatagproof
{\isafoldproof}%
%
\isadelimproof
\isanewline
%
\endisadelimproof
\isanewline
\isacommand{lemma}\isamarkupfalse%
\ cond{\isacharunderscore}{\kern0pt}exp{\isacharunderscore}{\kern0pt}cmult{\isacharcolon}{\kern0pt}\isanewline
\ \ \isakeyword{fixes}\ f\ {\isacharcolon}{\kern0pt}{\isacharcolon}{\kern0pt}\ {\isachardoublequoteopen}{\isacharprime}{\kern0pt}a\ {\isasymRightarrow}\ real{\isachardoublequoteclose}\isanewline
\ \ \isakeyword{assumes}\ {\isachardoublequoteopen}integrable\ M\ f{\isachardoublequoteclose}\isanewline
\ \ \isakeyword{shows}\ {\isachardoublequoteopen}AE\ x\ in\ M{\isachardot}{\kern0pt}\ cond{\isacharunderscore}{\kern0pt}exp\ M\ F\ {\isacharparenleft}{\kern0pt}{\isasymlambda}x{\isachardot}{\kern0pt}\ c\ {\isacharasterisk}{\kern0pt}\ f\ x{\isacharparenright}{\kern0pt}\ x\ {\isacharequal}{\kern0pt}\ c\ {\isacharasterisk}{\kern0pt}\ cond{\isacharunderscore}{\kern0pt}exp\ M\ F\ f\ x{\isachardoublequoteclose}\isanewline
%
\isadelimproof
\ \ %
\endisadelimproof
%
\isatagproof
\isacommand{using}\isamarkupfalse%
\ real{\isacharunderscore}{\kern0pt}cond{\isacharunderscore}{\kern0pt}exp{\isacharunderscore}{\kern0pt}cmult{\isacharbrackleft}{\kern0pt}OF\ assms{\isacharparenleft}{\kern0pt}{\isadigit{1}}{\isacharparenright}{\kern0pt}{\isacharcomma}{\kern0pt}\ of\ c{\isacharbrackright}{\kern0pt}\ assms{\isacharparenleft}{\kern0pt}{\isadigit{1}}{\isacharparenright}{\kern0pt}{\isacharbrackleft}{\kern0pt}THEN\ cond{\isacharunderscore}{\kern0pt}exp{\isacharunderscore}{\kern0pt}real{\isacharbrackright}{\kern0pt}\ assms{\isacharparenleft}{\kern0pt}{\isadigit{1}}{\isacharparenright}{\kern0pt}{\isacharbrackleft}{\kern0pt}THEN\ integrable{\isacharunderscore}{\kern0pt}mult{\isacharunderscore}{\kern0pt}right{\isacharcomma}{\kern0pt}\ THEN\ cond{\isacharunderscore}{\kern0pt}exp{\isacharunderscore}{\kern0pt}real{\isacharcomma}{\kern0pt}\ of\ c{\isacharbrackright}{\kern0pt}\ \isacommand{by}\isamarkupfalse%
\ fastforce%
\endisatagproof
{\isafoldproof}%
%
\isadelimproof
%
\endisadelimproof
%
\begin{isamarkuptext}%
Indicator functions%
\end{isamarkuptext}\isamarkuptrue%
\isacommand{lemma}\isamarkupfalse%
\ has{\isacharunderscore}{\kern0pt}cond{\isacharunderscore}{\kern0pt}exp{\isacharunderscore}{\kern0pt}indicator{\isacharcolon}{\kern0pt}\isanewline
\ \ \isakeyword{assumes}\ {\isachardoublequoteopen}A\ {\isasymin}\ sets\ M{\isachardoublequoteclose}\ {\isachardoublequoteopen}emeasure\ M\ A\ {\isacharless}{\kern0pt}\ {\isasyminfinity}{\isachardoublequoteclose}\isanewline
\ \ \isakeyword{shows}\ {\isachardoublequoteopen}has{\isacharunderscore}{\kern0pt}cond{\isacharunderscore}{\kern0pt}exp\ M\ F\ {\isacharparenleft}{\kern0pt}{\isasymlambda}x{\isachardot}{\kern0pt}\ indicat{\isacharunderscore}{\kern0pt}real\ A\ x\ {\isacharasterisk}{\kern0pt}\isactrlsub R\ y{\isacharparenright}{\kern0pt}\ {\isacharparenleft}{\kern0pt}{\isasymlambda}x{\isachardot}{\kern0pt}\ real{\isacharunderscore}{\kern0pt}cond{\isacharunderscore}{\kern0pt}exp\ M\ F\ {\isacharparenleft}{\kern0pt}indicator\ A{\isacharparenright}{\kern0pt}\ x\ {\isacharasterisk}{\kern0pt}\isactrlsub R\ y{\isacharparenright}{\kern0pt}{\isachardoublequoteclose}\isanewline
%
\isadelimproof
%
\endisadelimproof
%
\isatagproof
\isacommand{proof}\isamarkupfalse%
\ {\isacharparenleft}{\kern0pt}intro\ has{\isacharunderscore}{\kern0pt}cond{\isacharunderscore}{\kern0pt}expI{\isacharprime}{\kern0pt}{\isacharcomma}{\kern0pt}\ goal{\isacharunderscore}{\kern0pt}cases{\isacharparenright}{\kern0pt}\isanewline
\ \ \isacommand{case}\isamarkupfalse%
\ {\isacharparenleft}{\kern0pt}{\isadigit{1}}\ B{\isacharparenright}{\kern0pt}\isanewline
\ \ \isacommand{have}\isamarkupfalse%
\ {\isachardoublequoteopen}{\isasymintegral}x{\isasymin}B{\isachardot}{\kern0pt}\ {\isacharparenleft}{\kern0pt}indicat{\isacharunderscore}{\kern0pt}real\ A\ x\ {\isacharasterisk}{\kern0pt}\isactrlsub R\ y{\isacharparenright}{\kern0pt}\ {\isasympartial}M\ \ {\isacharequal}{\kern0pt}\ {\isacharparenleft}{\kern0pt}{\isasymintegral}x{\isasymin}B{\isachardot}{\kern0pt}\ indicat{\isacharunderscore}{\kern0pt}real\ A\ x\ {\isasympartial}M{\isacharparenright}{\kern0pt}\ {\isacharasterisk}{\kern0pt}\isactrlsub R\ y{\isachardoublequoteclose}\ \isacommand{using}\isamarkupfalse%
\ assms\ \isacommand{by}\isamarkupfalse%
\ {\isacharparenleft}{\kern0pt}intro\ set{\isacharunderscore}{\kern0pt}integral{\isacharunderscore}{\kern0pt}scaleR{\isacharunderscore}{\kern0pt}left{\isacharcomma}{\kern0pt}\ meson\ {\isadigit{1}}\ in{\isacharunderscore}{\kern0pt}mono\ subalg\ subalgebra{\isacharunderscore}{\kern0pt}def{\isacharcomma}{\kern0pt}\ blast{\isacharparenright}{\kern0pt}\isanewline
\ \ \isacommand{also}\isamarkupfalse%
\ \isacommand{have}\isamarkupfalse%
\ {\isachardoublequoteopen}{\isachardot}{\kern0pt}{\isachardot}{\kern0pt}{\isachardot}{\kern0pt}\ {\isacharequal}{\kern0pt}\ {\isacharparenleft}{\kern0pt}{\isasymintegral}x{\isasymin}B{\isachardot}{\kern0pt}\ real{\isacharunderscore}{\kern0pt}cond{\isacharunderscore}{\kern0pt}exp\ M\ F\ {\isacharparenleft}{\kern0pt}indicator\ A{\isacharparenright}{\kern0pt}\ x\ {\isasympartial}M{\isacharparenright}{\kern0pt}\ {\isacharasterisk}{\kern0pt}\isactrlsub R\ y{\isachardoublequoteclose}\ \isacommand{using}\isamarkupfalse%
\ {\isadigit{1}}\ assms\ \isacommand{by}\isamarkupfalse%
\ {\isacharparenleft}{\kern0pt}subst\ real{\isacharunderscore}{\kern0pt}cond{\isacharunderscore}{\kern0pt}exp{\isacharunderscore}{\kern0pt}intA{\isacharcomma}{\kern0pt}\ auto{\isacharparenright}{\kern0pt}\isanewline
\ \ \isacommand{also}\isamarkupfalse%
\ \isacommand{have}\isamarkupfalse%
\ {\isachardoublequoteopen}{\isachardot}{\kern0pt}{\isachardot}{\kern0pt}{\isachardot}{\kern0pt}\ {\isacharequal}{\kern0pt}\ {\isasymintegral}x{\isasymin}B{\isachardot}{\kern0pt}\ {\isacharparenleft}{\kern0pt}real{\isacharunderscore}{\kern0pt}cond{\isacharunderscore}{\kern0pt}exp\ M\ F\ {\isacharparenleft}{\kern0pt}indicator\ A{\isacharparenright}{\kern0pt}\ x\ {\isacharasterisk}{\kern0pt}\isactrlsub R\ y{\isacharparenright}{\kern0pt}\ {\isasympartial}M{\isachardoublequoteclose}\ \isacommand{using}\isamarkupfalse%
\ assms\ \isacommand{by}\isamarkupfalse%
\ {\isacharparenleft}{\kern0pt}intro\ set{\isacharunderscore}{\kern0pt}integral{\isacharunderscore}{\kern0pt}scaleR{\isacharunderscore}{\kern0pt}left{\isacharbrackleft}{\kern0pt}symmetric{\isacharbrackright}{\kern0pt}{\isacharcomma}{\kern0pt}\ meson\ {\isadigit{1}}\ in{\isacharunderscore}{\kern0pt}mono\ subalg\ subalgebra{\isacharunderscore}{\kern0pt}def{\isacharcomma}{\kern0pt}\ blast{\isacharparenright}{\kern0pt}\isanewline
\ \ \isacommand{finally}\isamarkupfalse%
\ \isacommand{show}\isamarkupfalse%
\ {\isacharquery}{\kern0pt}case\ \isacommand{{\isachardot}{\kern0pt}}\isamarkupfalse%
\isanewline
\isacommand{next}\isamarkupfalse%
\isanewline
\ \ \isacommand{case}\isamarkupfalse%
\ {\isadigit{2}}\isanewline
\ \ \isacommand{then}\isamarkupfalse%
\ \isacommand{show}\isamarkupfalse%
\ {\isacharquery}{\kern0pt}case\ \isacommand{using}\isamarkupfalse%
\ integrable{\isacharunderscore}{\kern0pt}scaleR{\isacharunderscore}{\kern0pt}left\ integrable{\isacharunderscore}{\kern0pt}real{\isacharunderscore}{\kern0pt}indicator\ assms\ \isacommand{by}\isamarkupfalse%
\ blast\isanewline
\isacommand{next}\isamarkupfalse%
\isanewline
\ \ \isacommand{case}\isamarkupfalse%
\ {\isadigit{3}}\isanewline
\ \ \isacommand{show}\isamarkupfalse%
\ {\isacharquery}{\kern0pt}case\ \isacommand{using}\isamarkupfalse%
\ assms\ \isacommand{by}\isamarkupfalse%
\ {\isacharparenleft}{\kern0pt}intro\ integrable{\isacharunderscore}{\kern0pt}scaleR{\isacharunderscore}{\kern0pt}left{\isacharcomma}{\kern0pt}\ intro\ real{\isacharunderscore}{\kern0pt}cond{\isacharunderscore}{\kern0pt}exp{\isacharunderscore}{\kern0pt}int{\isacharcomma}{\kern0pt}\ blast{\isacharplus}{\kern0pt}{\isacharparenright}{\kern0pt}\isanewline
\isacommand{next}\isamarkupfalse%
\isanewline
\ \ \isacommand{case}\isamarkupfalse%
\ {\isadigit{4}}\isanewline
\ \ \isacommand{then}\isamarkupfalse%
\ \isacommand{show}\isamarkupfalse%
\ {\isacharquery}{\kern0pt}case\ \isacommand{by}\isamarkupfalse%
\ {\isacharparenleft}{\kern0pt}intro\ borel{\isacharunderscore}{\kern0pt}measurable{\isacharunderscore}{\kern0pt}scaleR{\isacharcomma}{\kern0pt}\ intro\ Conditional{\isacharunderscore}{\kern0pt}Expectation{\isachardot}{\kern0pt}borel{\isacharunderscore}{\kern0pt}measurable{\isacharunderscore}{\kern0pt}cond{\isacharunderscore}{\kern0pt}exp{\isacharcomma}{\kern0pt}\ simp{\isacharparenright}{\kern0pt}\isanewline
\isacommand{qed}\isamarkupfalse%
%
\endisatagproof
{\isafoldproof}%
%
\isadelimproof
\isanewline
%
\endisadelimproof
\isanewline
\isacommand{lemma}\isamarkupfalse%
\ cond{\isacharunderscore}{\kern0pt}exp{\isacharunderscore}{\kern0pt}indicator{\isacharbrackleft}{\kern0pt}intro{\isacharbrackright}{\kern0pt}{\isacharcolon}{\kern0pt}\isanewline
\ \ \isakeyword{fixes}\ y\ {\isacharcolon}{\kern0pt}{\isacharcolon}{\kern0pt}\ {\isachardoublequoteopen}{\isacharprime}{\kern0pt}b{\isacharcolon}{\kern0pt}{\isacharcolon}{\kern0pt}{\isacharbraceleft}{\kern0pt}second{\isacharunderscore}{\kern0pt}countable{\isacharunderscore}{\kern0pt}topology{\isacharcomma}{\kern0pt}banach{\isacharbraceright}{\kern0pt}{\isachardoublequoteclose}\isanewline
\ \ \isakeyword{assumes}\ {\isacharbrackleft}{\kern0pt}measurable{\isacharbrackright}{\kern0pt}{\isacharcolon}{\kern0pt}\ {\isachardoublequoteopen}A\ {\isasymin}\ sets\ M{\isachardoublequoteclose}\ {\isachardoublequoteopen}emeasure\ M\ A\ {\isacharless}{\kern0pt}\ {\isasyminfinity}{\isachardoublequoteclose}\isanewline
\ \ \isakeyword{shows}\ {\isachardoublequoteopen}AE\ x\ in\ M{\isachardot}{\kern0pt}\ cond{\isacharunderscore}{\kern0pt}exp\ M\ F\ {\isacharparenleft}{\kern0pt}{\isasymlambda}x{\isachardot}{\kern0pt}\ indicat{\isacharunderscore}{\kern0pt}real\ A\ x\ {\isacharasterisk}{\kern0pt}\isactrlsub R\ y{\isacharparenright}{\kern0pt}\ x\ {\isacharequal}{\kern0pt}\ cond{\isacharunderscore}{\kern0pt}exp\ M\ F\ {\isacharparenleft}{\kern0pt}indicator\ A{\isacharparenright}{\kern0pt}\ x\ {\isacharasterisk}{\kern0pt}\isactrlsub R\ y{\isachardoublequoteclose}\isanewline
%
\isadelimproof
%
\endisadelimproof
%
\isatagproof
\isacommand{proof}\isamarkupfalse%
\ {\isacharminus}{\kern0pt}\isanewline
\ \ \isacommand{have}\isamarkupfalse%
\ {\isachardoublequoteopen}AE\ x\ in\ M{\isachardot}{\kern0pt}\ cond{\isacharunderscore}{\kern0pt}exp\ M\ F\ {\isacharparenleft}{\kern0pt}{\isasymlambda}x{\isachardot}{\kern0pt}\ indicat{\isacharunderscore}{\kern0pt}real\ A\ x\ {\isacharasterisk}{\kern0pt}\isactrlsub R\ y{\isacharparenright}{\kern0pt}\ x\ {\isacharequal}{\kern0pt}\ real{\isacharunderscore}{\kern0pt}cond{\isacharunderscore}{\kern0pt}exp\ M\ F\ {\isacharparenleft}{\kern0pt}indicator\ A{\isacharparenright}{\kern0pt}\ x\ {\isacharasterisk}{\kern0pt}\isactrlsub R\ y{\isachardoublequoteclose}\ \isacommand{using}\isamarkupfalse%
\ has{\isacharunderscore}{\kern0pt}cond{\isacharunderscore}{\kern0pt}exp{\isacharunderscore}{\kern0pt}indicator{\isacharbrackleft}{\kern0pt}OF\ assms{\isacharbrackright}{\kern0pt}\ has{\isacharunderscore}{\kern0pt}cond{\isacharunderscore}{\kern0pt}exp{\isacharunderscore}{\kern0pt}charact\ \isacommand{by}\isamarkupfalse%
\ blast\isanewline
\ \ \isacommand{thus}\isamarkupfalse%
\ {\isacharquery}{\kern0pt}thesis\ \isacommand{using}\isamarkupfalse%
\ cond{\isacharunderscore}{\kern0pt}exp{\isacharunderscore}{\kern0pt}real{\isacharbrackleft}{\kern0pt}OF\ integrable{\isacharunderscore}{\kern0pt}real{\isacharunderscore}{\kern0pt}indicator{\isacharcomma}{\kern0pt}\ OF\ assms{\isacharbrackright}{\kern0pt}\ \isacommand{by}\isamarkupfalse%
\ fastforce\isanewline
\isacommand{qed}\isamarkupfalse%
%
\endisatagproof
{\isafoldproof}%
%
\isadelimproof
%
\endisadelimproof
%
\begin{isamarkuptext}%
Addition%
\end{isamarkuptext}\isamarkuptrue%
\isacommand{lemma}\isamarkupfalse%
\ has{\isacharunderscore}{\kern0pt}cond{\isacharunderscore}{\kern0pt}exp{\isacharunderscore}{\kern0pt}add{\isacharcolon}{\kern0pt}\isanewline
\ \ \isakeyword{fixes}\ f\ g\ {\isacharcolon}{\kern0pt}{\isacharcolon}{\kern0pt}\ {\isachardoublequoteopen}{\isacharprime}{\kern0pt}a\ {\isasymRightarrow}\ {\isacharprime}{\kern0pt}b{\isacharcolon}{\kern0pt}{\isacharcolon}{\kern0pt}{\isacharbraceleft}{\kern0pt}second{\isacharunderscore}{\kern0pt}countable{\isacharunderscore}{\kern0pt}topology{\isacharcomma}{\kern0pt}banach{\isacharbraceright}{\kern0pt}{\isachardoublequoteclose}\isanewline
\ \ \isakeyword{assumes}\ {\isachardoublequoteopen}has{\isacharunderscore}{\kern0pt}cond{\isacharunderscore}{\kern0pt}exp\ M\ F\ f\ f{\isacharprime}{\kern0pt}{\isachardoublequoteclose}\ {\isachardoublequoteopen}has{\isacharunderscore}{\kern0pt}cond{\isacharunderscore}{\kern0pt}exp\ M\ F\ g\ g{\isacharprime}{\kern0pt}{\isachardoublequoteclose}\isanewline
\ \ \isakeyword{shows}\ {\isachardoublequoteopen}has{\isacharunderscore}{\kern0pt}cond{\isacharunderscore}{\kern0pt}exp\ M\ F\ {\isacharparenleft}{\kern0pt}{\isasymlambda}x{\isachardot}{\kern0pt}\ f\ x\ {\isacharplus}{\kern0pt}\ g\ x{\isacharparenright}{\kern0pt}\ {\isacharparenleft}{\kern0pt}{\isasymlambda}x{\isachardot}{\kern0pt}\ f{\isacharprime}{\kern0pt}\ x\ {\isacharplus}{\kern0pt}\ g{\isacharprime}{\kern0pt}\ x{\isacharparenright}{\kern0pt}{\isachardoublequoteclose}\isanewline
%
\isadelimproof
%
\endisadelimproof
%
\isatagproof
\isacommand{proof}\isamarkupfalse%
\ {\isacharparenleft}{\kern0pt}intro\ has{\isacharunderscore}{\kern0pt}cond{\isacharunderscore}{\kern0pt}expI{\isacharprime}{\kern0pt}{\isacharcomma}{\kern0pt}\ goal{\isacharunderscore}{\kern0pt}cases{\isacharparenright}{\kern0pt}\isanewline
\ \ \isacommand{case}\isamarkupfalse%
\ {\isacharparenleft}{\kern0pt}{\isadigit{1}}\ A{\isacharparenright}{\kern0pt}\isanewline
\ \ \isacommand{have}\isamarkupfalse%
\ {\isachardoublequoteopen}{\isasymintegral}x{\isasymin}A{\isachardot}{\kern0pt}\ {\isacharparenleft}{\kern0pt}f\ x\ {\isacharplus}{\kern0pt}\ g\ x{\isacharparenright}{\kern0pt}{\isasympartial}M\ {\isacharequal}{\kern0pt}\ {\isacharparenleft}{\kern0pt}{\isasymintegral}x{\isasymin}A{\isachardot}{\kern0pt}\ f\ x\ {\isasympartial}M{\isacharparenright}{\kern0pt}\ {\isacharplus}{\kern0pt}\ {\isacharparenleft}{\kern0pt}{\isasymintegral}x{\isasymin}A{\isachardot}{\kern0pt}\ g\ x\ {\isasympartial}M{\isacharparenright}{\kern0pt}{\isachardoublequoteclose}\ \isacommand{using}\isamarkupfalse%
\ assms{\isacharbrackleft}{\kern0pt}THEN\ has{\isacharunderscore}{\kern0pt}cond{\isacharunderscore}{\kern0pt}expD{\isacharparenleft}{\kern0pt}{\isadigit{2}}{\isacharparenright}{\kern0pt}{\isacharbrackright}{\kern0pt}\ subalg\ {\isadigit{1}}\ \isacommand{by}\isamarkupfalse%
\ {\isacharparenleft}{\kern0pt}intro\ set{\isacharunderscore}{\kern0pt}integral{\isacharunderscore}{\kern0pt}add{\isacharparenleft}{\kern0pt}{\isadigit{2}}{\isacharparenright}{\kern0pt}{\isacharcomma}{\kern0pt}\ auto\ simp\ add{\isacharcolon}{\kern0pt}\ subalgebra{\isacharunderscore}{\kern0pt}def\ set{\isacharunderscore}{\kern0pt}integrable{\isacharunderscore}{\kern0pt}def\ intro{\isacharcolon}{\kern0pt}\ integrable{\isacharunderscore}{\kern0pt}mult{\isacharunderscore}{\kern0pt}indicator{\isacharparenright}{\kern0pt}\isanewline
\ \ \isacommand{also}\isamarkupfalse%
\ \isacommand{have}\isamarkupfalse%
\ {\isachardoublequoteopen}{\isachardot}{\kern0pt}{\isachardot}{\kern0pt}{\isachardot}{\kern0pt}\ {\isacharequal}{\kern0pt}\ {\isacharparenleft}{\kern0pt}{\isasymintegral}x{\isasymin}A{\isachardot}{\kern0pt}\ f{\isacharprime}{\kern0pt}\ x\ {\isasympartial}M{\isacharparenright}{\kern0pt}\ {\isacharplus}{\kern0pt}\ {\isacharparenleft}{\kern0pt}{\isasymintegral}x{\isasymin}A{\isachardot}{\kern0pt}\ g{\isacharprime}{\kern0pt}\ x\ {\isasympartial}M{\isacharparenright}{\kern0pt}{\isachardoublequoteclose}\ \isacommand{using}\isamarkupfalse%
\ assms{\isacharbrackleft}{\kern0pt}THEN\ has{\isacharunderscore}{\kern0pt}cond{\isacharunderscore}{\kern0pt}expD{\isacharparenleft}{\kern0pt}{\isadigit{1}}{\isacharparenright}{\kern0pt}{\isacharbrackleft}{\kern0pt}OF\ {\isacharunderscore}{\kern0pt}\ {\isadigit{1}}{\isacharbrackright}{\kern0pt}{\isacharbrackright}{\kern0pt}\ \isacommand{by}\isamarkupfalse%
\ argo\isanewline
\ \ \isacommand{also}\isamarkupfalse%
\ \isacommand{have}\isamarkupfalse%
\ {\isachardoublequoteopen}{\isachardot}{\kern0pt}{\isachardot}{\kern0pt}{\isachardot}{\kern0pt}\ {\isacharequal}{\kern0pt}\ {\isasymintegral}x{\isasymin}A{\isachardot}{\kern0pt}\ {\isacharparenleft}{\kern0pt}f{\isacharprime}{\kern0pt}\ x\ {\isacharplus}{\kern0pt}\ g{\isacharprime}{\kern0pt}\ x{\isacharparenright}{\kern0pt}{\isasympartial}M{\isachardoublequoteclose}\ \isacommand{using}\isamarkupfalse%
\ assms{\isacharbrackleft}{\kern0pt}THEN\ has{\isacharunderscore}{\kern0pt}cond{\isacharunderscore}{\kern0pt}expD{\isacharparenleft}{\kern0pt}{\isadigit{3}}{\isacharparenright}{\kern0pt}{\isacharbrackright}{\kern0pt}\ subalg\ {\isadigit{1}}\ \isacommand{by}\isamarkupfalse%
\ {\isacharparenleft}{\kern0pt}intro\ set{\isacharunderscore}{\kern0pt}integral{\isacharunderscore}{\kern0pt}add{\isacharparenleft}{\kern0pt}{\isadigit{2}}{\isacharparenright}{\kern0pt}{\isacharbrackleft}{\kern0pt}symmetric{\isacharbrackright}{\kern0pt}{\isacharcomma}{\kern0pt}\ auto\ simp\ add{\isacharcolon}{\kern0pt}\ subalgebra{\isacharunderscore}{\kern0pt}def\ set{\isacharunderscore}{\kern0pt}integrable{\isacharunderscore}{\kern0pt}def\ intro{\isacharcolon}{\kern0pt}\ integrable{\isacharunderscore}{\kern0pt}mult{\isacharunderscore}{\kern0pt}indicator{\isacharparenright}{\kern0pt}\isanewline
\ \ \isacommand{finally}\isamarkupfalse%
\ \isacommand{show}\isamarkupfalse%
\ {\isacharquery}{\kern0pt}case\ \isacommand{{\isachardot}{\kern0pt}}\isamarkupfalse%
\isanewline
\isacommand{next}\isamarkupfalse%
\isanewline
\ \ \isacommand{case}\isamarkupfalse%
\ {\isadigit{2}}\isanewline
\ \ \isacommand{then}\isamarkupfalse%
\ \isacommand{show}\isamarkupfalse%
\ {\isacharquery}{\kern0pt}case\ \isacommand{by}\isamarkupfalse%
\ {\isacharparenleft}{\kern0pt}metis\ Bochner{\isacharunderscore}{\kern0pt}Integration{\isachardot}{\kern0pt}integrable{\isacharunderscore}{\kern0pt}add\ assms\ has{\isacharunderscore}{\kern0pt}cond{\isacharunderscore}{\kern0pt}expD{\isacharparenleft}{\kern0pt}{\isadigit{2}}{\isacharparenright}{\kern0pt}{\isacharparenright}{\kern0pt}\isanewline
\isacommand{next}\isamarkupfalse%
\isanewline
\ \ \isacommand{case}\isamarkupfalse%
\ {\isadigit{3}}\isanewline
\ \ \isacommand{then}\isamarkupfalse%
\ \isacommand{show}\isamarkupfalse%
\ {\isacharquery}{\kern0pt}case\ \isacommand{by}\isamarkupfalse%
\ {\isacharparenleft}{\kern0pt}metis\ Bochner{\isacharunderscore}{\kern0pt}Integration{\isachardot}{\kern0pt}integrable{\isacharunderscore}{\kern0pt}add\ assms\ has{\isacharunderscore}{\kern0pt}cond{\isacharunderscore}{\kern0pt}expD{\isacharparenleft}{\kern0pt}{\isadigit{3}}{\isacharparenright}{\kern0pt}{\isacharparenright}{\kern0pt}\isanewline
\isacommand{next}\isamarkupfalse%
\isanewline
\ \ \isacommand{case}\isamarkupfalse%
\ {\isadigit{4}}\isanewline
\ \ \isacommand{then}\isamarkupfalse%
\ \isacommand{show}\isamarkupfalse%
\ {\isacharquery}{\kern0pt}case\ \isacommand{using}\isamarkupfalse%
\ assms\ borel{\isacharunderscore}{\kern0pt}measurable{\isacharunderscore}{\kern0pt}add\ has{\isacharunderscore}{\kern0pt}cond{\isacharunderscore}{\kern0pt}expD{\isacharparenleft}{\kern0pt}{\isadigit{4}}{\isacharparenright}{\kern0pt}\ \isacommand{by}\isamarkupfalse%
\ blast\isanewline
\isacommand{qed}\isamarkupfalse%
%
\endisatagproof
{\isafoldproof}%
%
\isadelimproof
\isanewline
%
\endisadelimproof
\isanewline
\isacommand{lemma}\isamarkupfalse%
\ has{\isacharunderscore}{\kern0pt}cond{\isacharunderscore}{\kern0pt}exp{\isacharunderscore}{\kern0pt}scaleR{\isacharunderscore}{\kern0pt}right{\isacharcolon}{\kern0pt}\isanewline
\ \ \isakeyword{fixes}\ f\ {\isacharcolon}{\kern0pt}{\isacharcolon}{\kern0pt}\ {\isachardoublequoteopen}{\isacharprime}{\kern0pt}a\ {\isasymRightarrow}\ {\isacharprime}{\kern0pt}b{\isacharcolon}{\kern0pt}{\isacharcolon}{\kern0pt}{\isacharbraceleft}{\kern0pt}second{\isacharunderscore}{\kern0pt}countable{\isacharunderscore}{\kern0pt}topology{\isacharcomma}{\kern0pt}banach{\isacharbraceright}{\kern0pt}{\isachardoublequoteclose}\isanewline
\ \ \isakeyword{assumes}\ {\isachardoublequoteopen}has{\isacharunderscore}{\kern0pt}cond{\isacharunderscore}{\kern0pt}exp\ M\ F\ f\ f{\isacharprime}{\kern0pt}{\isachardoublequoteclose}\isanewline
\ \ \isakeyword{shows}\ {\isachardoublequoteopen}has{\isacharunderscore}{\kern0pt}cond{\isacharunderscore}{\kern0pt}exp\ M\ F\ {\isacharparenleft}{\kern0pt}{\isasymlambda}x{\isachardot}{\kern0pt}\ c\ {\isacharasterisk}{\kern0pt}\isactrlsub R\ f\ x{\isacharparenright}{\kern0pt}\ {\isacharparenleft}{\kern0pt}{\isasymlambda}x{\isachardot}{\kern0pt}\ c\ {\isacharasterisk}{\kern0pt}\isactrlsub R\ f{\isacharprime}{\kern0pt}\ x{\isacharparenright}{\kern0pt}{\isachardoublequoteclose}\isanewline
%
\isadelimproof
\ \ %
\endisadelimproof
%
\isatagproof
\isacommand{using}\isamarkupfalse%
\ has{\isacharunderscore}{\kern0pt}cond{\isacharunderscore}{\kern0pt}expD{\isacharbrackleft}{\kern0pt}OF\ assms{\isacharbrackright}{\kern0pt}\ \isacommand{by}\isamarkupfalse%
\ {\isacharparenleft}{\kern0pt}intro\ has{\isacharunderscore}{\kern0pt}cond{\isacharunderscore}{\kern0pt}expI{\isacharprime}{\kern0pt}{\isacharcomma}{\kern0pt}\ auto{\isacharparenright}{\kern0pt}%
\endisatagproof
{\isafoldproof}%
%
\isadelimproof
\isanewline
%
\endisadelimproof
\isanewline
\isacommand{lemma}\isamarkupfalse%
\ cond{\isacharunderscore}{\kern0pt}exp{\isacharunderscore}{\kern0pt}scaleR{\isacharunderscore}{\kern0pt}right{\isacharcolon}{\kern0pt}\isanewline
\ \ \isakeyword{fixes}\ f\ {\isacharcolon}{\kern0pt}{\isacharcolon}{\kern0pt}\ {\isachardoublequoteopen}{\isacharprime}{\kern0pt}a\ {\isasymRightarrow}\ {\isacharprime}{\kern0pt}b{\isacharcolon}{\kern0pt}{\isacharcolon}{\kern0pt}{\isacharbraceleft}{\kern0pt}second{\isacharunderscore}{\kern0pt}countable{\isacharunderscore}{\kern0pt}topology{\isacharcomma}{\kern0pt}banach{\isacharbraceright}{\kern0pt}{\isachardoublequoteclose}\isanewline
\ \ \isakeyword{assumes}\ {\isachardoublequoteopen}integrable\ M\ f{\isachardoublequoteclose}\isanewline
\ \ \isakeyword{shows}\ {\isachardoublequoteopen}AE\ x\ in\ M{\isachardot}{\kern0pt}\ cond{\isacharunderscore}{\kern0pt}exp\ M\ F\ {\isacharparenleft}{\kern0pt}{\isasymlambda}x{\isachardot}{\kern0pt}\ c\ {\isacharasterisk}{\kern0pt}\isactrlsub R\ f\ x{\isacharparenright}{\kern0pt}\ x\ {\isacharequal}{\kern0pt}\ c\ {\isacharasterisk}{\kern0pt}\isactrlsub R\ cond{\isacharunderscore}{\kern0pt}exp\ M\ F\ f\ x{\isachardoublequoteclose}\isanewline
%
\isadelimproof
%
\endisadelimproof
%
\isatagproof
\isacommand{proof}\isamarkupfalse%
\ {\isacharparenleft}{\kern0pt}cases\ {\isachardoublequoteopen}{\isasymexists}f{\isacharprime}{\kern0pt}{\isachardot}{\kern0pt}\ has{\isacharunderscore}{\kern0pt}cond{\isacharunderscore}{\kern0pt}exp\ M\ F\ f\ f{\isacharprime}{\kern0pt}{\isachardoublequoteclose}{\isacharparenright}{\kern0pt}\isanewline
\ \ \isacommand{case}\isamarkupfalse%
\ True\isanewline
\ \ \isacommand{then}\isamarkupfalse%
\ \isacommand{show}\isamarkupfalse%
\ {\isacharquery}{\kern0pt}thesis\ \isacommand{using}\isamarkupfalse%
\ assms\ has{\isacharunderscore}{\kern0pt}cond{\isacharunderscore}{\kern0pt}exp{\isacharunderscore}{\kern0pt}charact\ has{\isacharunderscore}{\kern0pt}cond{\isacharunderscore}{\kern0pt}exp{\isacharunderscore}{\kern0pt}scaleR{\isacharunderscore}{\kern0pt}right\ \isacommand{by}\isamarkupfalse%
\ metis\isanewline
\isacommand{next}\isamarkupfalse%
\isanewline
\ \ \isacommand{case}\isamarkupfalse%
\ False\isanewline
\ \ \isacommand{show}\isamarkupfalse%
\ {\isacharquery}{\kern0pt}thesis\isanewline
\ \ \isacommand{proof}\isamarkupfalse%
\ {\isacharparenleft}{\kern0pt}cases\ {\isachardoublequoteopen}c\ {\isacharequal}{\kern0pt}\ {\isadigit{0}}{\isachardoublequoteclose}{\isacharparenright}{\kern0pt}\isanewline
\ \ \ \ \isacommand{case}\isamarkupfalse%
\ True\isanewline
\ \ \ \ \isacommand{then}\isamarkupfalse%
\ \isacommand{show}\isamarkupfalse%
\ {\isacharquery}{\kern0pt}thesis\ \isacommand{by}\isamarkupfalse%
\ simp\isanewline
\ \ \isacommand{next}\isamarkupfalse%
\isanewline
\ \ \ \ \isacommand{case}\isamarkupfalse%
\ c{\isacharunderscore}{\kern0pt}nonzero{\isacharcolon}{\kern0pt}\ False\isanewline
\ \ \ \ \isacommand{have}\isamarkupfalse%
\ {\isachardoublequoteopen}{\isasymnexists}f{\isacharprime}{\kern0pt}{\isachardot}{\kern0pt}\ has{\isacharunderscore}{\kern0pt}cond{\isacharunderscore}{\kern0pt}exp\ M\ F\ {\isacharparenleft}{\kern0pt}{\isasymlambda}x{\isachardot}{\kern0pt}\ c\ {\isacharasterisk}{\kern0pt}\isactrlsub R\ f\ x{\isacharparenright}{\kern0pt}\ f{\isacharprime}{\kern0pt}{\isachardoublequoteclose}\isanewline
\ \ \ \ \isacommand{proof}\isamarkupfalse%
\ {\isacharparenleft}{\kern0pt}standard{\isacharcomma}{\kern0pt}\ goal{\isacharunderscore}{\kern0pt}cases{\isacharparenright}{\kern0pt}\isanewline
\ \ \ \ \ \ \isacommand{case}\isamarkupfalse%
\ {\isadigit{1}}\isanewline
\ \ \ \ \ \ \isacommand{then}\isamarkupfalse%
\ \isacommand{obtain}\isamarkupfalse%
\ f{\isacharprime}{\kern0pt}\ \isakeyword{where}\ f{\isacharprime}{\kern0pt}{\isacharcolon}{\kern0pt}\ {\isachardoublequoteopen}has{\isacharunderscore}{\kern0pt}cond{\isacharunderscore}{\kern0pt}exp\ M\ F\ {\isacharparenleft}{\kern0pt}{\isasymlambda}x{\isachardot}{\kern0pt}\ c\ {\isacharasterisk}{\kern0pt}\isactrlsub R\ f\ x{\isacharparenright}{\kern0pt}\ f{\isacharprime}{\kern0pt}{\isachardoublequoteclose}\ \isacommand{by}\isamarkupfalse%
\ blast\isanewline
\ \ \ \ \ \ \isacommand{have}\isamarkupfalse%
\ {\isachardoublequoteopen}has{\isacharunderscore}{\kern0pt}cond{\isacharunderscore}{\kern0pt}exp\ M\ F\ f\ {\isacharparenleft}{\kern0pt}{\isasymlambda}x{\isachardot}{\kern0pt}\ inverse\ c\ {\isacharasterisk}{\kern0pt}\isactrlsub R\ f{\isacharprime}{\kern0pt}\ x{\isacharparenright}{\kern0pt}{\isachardoublequoteclose}\ \isacommand{using}\isamarkupfalse%
\ has{\isacharunderscore}{\kern0pt}cond{\isacharunderscore}{\kern0pt}expD{\isacharbrackleft}{\kern0pt}OF\ f{\isacharprime}{\kern0pt}{\isacharbrackright}{\kern0pt}\ divideR{\isacharunderscore}{\kern0pt}right{\isacharbrackleft}{\kern0pt}OF\ c{\isacharunderscore}{\kern0pt}nonzero{\isacharbrackright}{\kern0pt}\ assms\ \isacommand{by}\isamarkupfalse%
\ {\isacharparenleft}{\kern0pt}intro\ has{\isacharunderscore}{\kern0pt}cond{\isacharunderscore}{\kern0pt}expI{\isacharprime}{\kern0pt}{\isacharcomma}{\kern0pt}\ auto{\isacharparenright}{\kern0pt}\isanewline
\ \ \ \ \ \ \isacommand{then}\isamarkupfalse%
\ \isacommand{show}\isamarkupfalse%
\ {\isacharquery}{\kern0pt}case\ \isacommand{using}\isamarkupfalse%
\ False\ \isacommand{by}\isamarkupfalse%
\ blast\isanewline
\ \ \ \ \isacommand{qed}\isamarkupfalse%
\isanewline
\ \ \ \ \isacommand{then}\isamarkupfalse%
\ \isacommand{show}\isamarkupfalse%
\ {\isacharquery}{\kern0pt}thesis\ \isacommand{using}\isamarkupfalse%
\ cond{\isacharunderscore}{\kern0pt}exp{\isacharunderscore}{\kern0pt}null{\isacharbrackleft}{\kern0pt}OF\ False{\isacharbrackright}{\kern0pt}\ cond{\isacharunderscore}{\kern0pt}exp{\isacharunderscore}{\kern0pt}null\ \isacommand{by}\isamarkupfalse%
\ force\isanewline
\ \ \isacommand{qed}\isamarkupfalse%
\ \isanewline
\isacommand{qed}\isamarkupfalse%
%
\endisatagproof
{\isafoldproof}%
%
\isadelimproof
\isanewline
%
\endisadelimproof
\isanewline
\isacommand{lemma}\isamarkupfalse%
\ cond{\isacharunderscore}{\kern0pt}exp{\isacharunderscore}{\kern0pt}uminus{\isacharcolon}{\kern0pt}\isanewline
\ \ \isakeyword{fixes}\ f\ {\isacharcolon}{\kern0pt}{\isacharcolon}{\kern0pt}\ {\isachardoublequoteopen}{\isacharprime}{\kern0pt}a\ {\isasymRightarrow}\ {\isacharprime}{\kern0pt}b{\isacharcolon}{\kern0pt}{\isacharcolon}{\kern0pt}{\isacharbraceleft}{\kern0pt}second{\isacharunderscore}{\kern0pt}countable{\isacharunderscore}{\kern0pt}topology{\isacharcomma}{\kern0pt}banach{\isacharbraceright}{\kern0pt}{\isachardoublequoteclose}\isanewline
\ \ \isakeyword{assumes}\ {\isachardoublequoteopen}integrable\ M\ f{\isachardoublequoteclose}\isanewline
\ \ \isakeyword{shows}\ {\isachardoublequoteopen}AE\ x\ in\ M{\isachardot}{\kern0pt}\ cond{\isacharunderscore}{\kern0pt}exp\ M\ F\ {\isacharparenleft}{\kern0pt}{\isasymlambda}x{\isachardot}{\kern0pt}\ {\isacharminus}{\kern0pt}\ f\ x{\isacharparenright}{\kern0pt}\ x\ {\isacharequal}{\kern0pt}\ {\isacharminus}{\kern0pt}\ cond{\isacharunderscore}{\kern0pt}exp\ M\ F\ f\ x{\isachardoublequoteclose}\isanewline
%
\isadelimproof
\ \ %
\endisadelimproof
%
\isatagproof
\isacommand{using}\isamarkupfalse%
\ cond{\isacharunderscore}{\kern0pt}exp{\isacharunderscore}{\kern0pt}scaleR{\isacharunderscore}{\kern0pt}right{\isacharbrackleft}{\kern0pt}OF\ assms{\isacharcomma}{\kern0pt}\ of\ {\isachardoublequoteopen}{\isacharminus}{\kern0pt}{\isadigit{1}}{\isachardoublequoteclose}{\isacharbrackright}{\kern0pt}\ \isacommand{by}\isamarkupfalse%
\ force%
\endisatagproof
{\isafoldproof}%
%
\isadelimproof
\isanewline
%
\endisadelimproof
\isanewline
\isacommand{corollary}\isamarkupfalse%
\ has{\isacharunderscore}{\kern0pt}cond{\isacharunderscore}{\kern0pt}exp{\isacharunderscore}{\kern0pt}simple{\isacharcolon}{\kern0pt}\isanewline
\ \ \isakeyword{fixes}\ f\ {\isacharcolon}{\kern0pt}{\isacharcolon}{\kern0pt}\ {\isachardoublequoteopen}{\isacharprime}{\kern0pt}a\ {\isasymRightarrow}\ {\isacharprime}{\kern0pt}b{\isacharcolon}{\kern0pt}{\isacharcolon}{\kern0pt}{\isacharbraceleft}{\kern0pt}second{\isacharunderscore}{\kern0pt}countable{\isacharunderscore}{\kern0pt}topology{\isacharcomma}{\kern0pt}banach{\isacharbraceright}{\kern0pt}{\isachardoublequoteclose}\isanewline
\ \ \isakeyword{assumes}\ {\isachardoublequoteopen}simple{\isacharunderscore}{\kern0pt}function\ M\ f{\isachardoublequoteclose}\ {\isachardoublequoteopen}emeasure\ M\ {\isacharbraceleft}{\kern0pt}y\ {\isasymin}\ space\ M{\isachardot}{\kern0pt}\ f\ y\ {\isasymnoteq}\ {\isadigit{0}}{\isacharbraceright}{\kern0pt}\ {\isasymnoteq}\ {\isasyminfinity}{\isachardoublequoteclose}\isanewline
\ \ \isakeyword{shows}\ {\isachardoublequoteopen}has{\isacharunderscore}{\kern0pt}cond{\isacharunderscore}{\kern0pt}exp\ M\ F\ f\ {\isacharparenleft}{\kern0pt}cond{\isacharunderscore}{\kern0pt}exp\ M\ F\ f{\isacharparenright}{\kern0pt}{\isachardoublequoteclose}\isanewline
%
\isadelimproof
\ \ %
\endisadelimproof
%
\isatagproof
\isacommand{using}\isamarkupfalse%
\ assms\isanewline
\isacommand{proof}\isamarkupfalse%
\ {\isacharparenleft}{\kern0pt}induction\ rule{\isacharcolon}{\kern0pt}\ integrable{\isacharunderscore}{\kern0pt}simple{\isacharunderscore}{\kern0pt}function{\isacharunderscore}{\kern0pt}induct{\isacharparenright}{\kern0pt}\isanewline
\ \ \isacommand{case}\isamarkupfalse%
\ {\isacharparenleft}{\kern0pt}cong\ f\ g{\isacharparenright}{\kern0pt}\isanewline
\ \ \isacommand{then}\isamarkupfalse%
\ \isacommand{show}\isamarkupfalse%
\ {\isacharquery}{\kern0pt}case\ \isacommand{using}\isamarkupfalse%
\ has{\isacharunderscore}{\kern0pt}cond{\isacharunderscore}{\kern0pt}exp{\isacharunderscore}{\kern0pt}cong\ \isacommand{by}\isamarkupfalse%
\ {\isacharparenleft}{\kern0pt}metis\ {\isacharparenleft}{\kern0pt}no{\isacharunderscore}{\kern0pt}types{\isacharcomma}{\kern0pt}\ opaque{\isacharunderscore}{\kern0pt}lifting{\isacharparenright}{\kern0pt}\ Bochner{\isacharunderscore}{\kern0pt}Integration{\isachardot}{\kern0pt}integrable{\isacharunderscore}{\kern0pt}cong\ has{\isacharunderscore}{\kern0pt}cond{\isacharunderscore}{\kern0pt}expD{\isacharparenleft}{\kern0pt}{\isadigit{2}}{\isacharparenright}{\kern0pt}\ has{\isacharunderscore}{\kern0pt}cond{\isacharunderscore}{\kern0pt}exp{\isacharunderscore}{\kern0pt}charact{\isacharparenleft}{\kern0pt}{\isadigit{1}}{\isacharparenright}{\kern0pt}{\isacharparenright}{\kern0pt}\isanewline
\isacommand{next}\isamarkupfalse%
\isanewline
\ \ \isacommand{case}\isamarkupfalse%
\ {\isacharparenleft}{\kern0pt}indicator\ A\ y{\isacharparenright}{\kern0pt}\isanewline
\ \ \isacommand{then}\isamarkupfalse%
\ \isacommand{show}\isamarkupfalse%
\ {\isacharquery}{\kern0pt}case\ \isacommand{using}\isamarkupfalse%
\ has{\isacharunderscore}{\kern0pt}cond{\isacharunderscore}{\kern0pt}exp{\isacharunderscore}{\kern0pt}charact{\isacharbrackleft}{\kern0pt}OF\ has{\isacharunderscore}{\kern0pt}cond{\isacharunderscore}{\kern0pt}exp{\isacharunderscore}{\kern0pt}indicator{\isacharbrackright}{\kern0pt}\ \isacommand{by}\isamarkupfalse%
\ fast\isanewline
\isacommand{next}\isamarkupfalse%
\isanewline
\ \ \isacommand{case}\isamarkupfalse%
\ {\isacharparenleft}{\kern0pt}add\ u\ v{\isacharparenright}{\kern0pt}\isanewline
\ \ \isacommand{then}\isamarkupfalse%
\ \isacommand{show}\isamarkupfalse%
\ {\isacharquery}{\kern0pt}case\ \isacommand{using}\isamarkupfalse%
\ has{\isacharunderscore}{\kern0pt}cond{\isacharunderscore}{\kern0pt}exp{\isacharunderscore}{\kern0pt}add\ has{\isacharunderscore}{\kern0pt}cond{\isacharunderscore}{\kern0pt}exp{\isacharunderscore}{\kern0pt}charact{\isacharparenleft}{\kern0pt}{\isadigit{1}}{\isacharparenright}{\kern0pt}\ \isacommand{by}\isamarkupfalse%
\ blast\isanewline
\isacommand{qed}\isamarkupfalse%
%
\endisatagproof
{\isafoldproof}%
%
\isadelimproof
\isanewline
%
\endisadelimproof
\isanewline
\isacommand{lemma}\isamarkupfalse%
\ cond{\isacharunderscore}{\kern0pt}exp{\isacharunderscore}{\kern0pt}contraction{\isacharunderscore}{\kern0pt}real{\isacharcolon}{\kern0pt}\isanewline
\ \ \isakeyword{fixes}\ f\ {\isacharcolon}{\kern0pt}{\isacharcolon}{\kern0pt}\ {\isachardoublequoteopen}{\isacharprime}{\kern0pt}a\ {\isasymRightarrow}\ real{\isachardoublequoteclose}\isanewline
\ \ \isakeyword{assumes}\ integrable{\isacharbrackleft}{\kern0pt}measurable{\isacharbrackright}{\kern0pt}{\isacharcolon}{\kern0pt}\ {\isachardoublequoteopen}integrable\ M\ f{\isachardoublequoteclose}\isanewline
\ \ \isakeyword{shows}\ {\isachardoublequoteopen}AE\ x\ in\ M{\isachardot}{\kern0pt}\ norm\ {\isacharparenleft}{\kern0pt}cond{\isacharunderscore}{\kern0pt}exp\ M\ F\ f\ x{\isacharparenright}{\kern0pt}\ {\isasymle}\ cond{\isacharunderscore}{\kern0pt}exp\ M\ F\ {\isacharparenleft}{\kern0pt}{\isasymlambda}x{\isachardot}{\kern0pt}\ norm\ {\isacharparenleft}{\kern0pt}f\ x{\isacharparenright}{\kern0pt}{\isacharparenright}{\kern0pt}\ x{\isachardoublequoteclose}\isanewline
%
\isadelimproof
%
\endisadelimproof
%
\isatagproof
\isacommand{proof}\isamarkupfalse%
{\isacharminus}{\kern0pt}\isanewline
\ \ \isacommand{have}\isamarkupfalse%
\ int{\isacharcolon}{\kern0pt}\ {\isachardoublequoteopen}integrable\ M\ {\isacharparenleft}{\kern0pt}{\isasymlambda}x{\isachardot}{\kern0pt}\ norm\ {\isacharparenleft}{\kern0pt}f\ x{\isacharparenright}{\kern0pt}{\isacharparenright}{\kern0pt}{\isachardoublequoteclose}\ \isacommand{using}\isamarkupfalse%
\ assms\ \isacommand{by}\isamarkupfalse%
\ blast\isanewline
\ \ \isacommand{have}\isamarkupfalse%
\ {\isacharasterisk}{\kern0pt}{\isacharcolon}{\kern0pt}\ {\isachardoublequoteopen}AE\ x\ in\ M{\isachardot}{\kern0pt}\ {\isadigit{0}}\ {\isasymle}\ cond{\isacharunderscore}{\kern0pt}exp\ M\ F\ {\isacharparenleft}{\kern0pt}{\isasymlambda}x{\isachardot}{\kern0pt}\ norm\ {\isacharparenleft}{\kern0pt}f\ x{\isacharparenright}{\kern0pt}{\isacharparenright}{\kern0pt}\ x{\isachardoublequoteclose}\ \isacommand{using}\isamarkupfalse%
\ cond{\isacharunderscore}{\kern0pt}exp{\isacharunderscore}{\kern0pt}real{\isacharbrackleft}{\kern0pt}THEN\ AE{\isacharunderscore}{\kern0pt}symmetric{\isacharcomma}{\kern0pt}\ OF\ integrable{\isacharunderscore}{\kern0pt}norm{\isacharbrackleft}{\kern0pt}OF\ integrable{\isacharbrackright}{\kern0pt}{\isacharbrackright}{\kern0pt}\ real{\isacharunderscore}{\kern0pt}cond{\isacharunderscore}{\kern0pt}exp{\isacharunderscore}{\kern0pt}ge{\isacharunderscore}{\kern0pt}c{\isacharbrackleft}{\kern0pt}OF\ integrable{\isacharunderscore}{\kern0pt}norm{\isacharbrackleft}{\kern0pt}OF\ integrable{\isacharbrackright}{\kern0pt}{\isacharcomma}{\kern0pt}\ of\ {\isadigit{0}}{\isacharbrackright}{\kern0pt}\ norm{\isacharunderscore}{\kern0pt}ge{\isacharunderscore}{\kern0pt}zero\ \isacommand{by}\isamarkupfalse%
\ fastforce\isanewline
\ \ \isacommand{have}\isamarkupfalse%
\ {\isacharasterisk}{\kern0pt}{\isacharasterisk}{\kern0pt}{\isacharcolon}{\kern0pt}\ {\isachardoublequoteopen}A\ {\isasymin}\ sets\ F\ {\isasymLongrightarrow}\ {\isasymintegral}x{\isasymin}A{\isachardot}{\kern0pt}\ {\isasymbar}f\ x{\isasymbar}\ {\isasympartial}M\ {\isacharequal}{\kern0pt}\ {\isasymintegral}x{\isasymin}A{\isachardot}{\kern0pt}\ real{\isacharunderscore}{\kern0pt}cond{\isacharunderscore}{\kern0pt}exp\ M\ F\ {\isacharparenleft}{\kern0pt}{\isasymlambda}x{\isachardot}{\kern0pt}\ norm\ {\isacharparenleft}{\kern0pt}f\ x{\isacharparenright}{\kern0pt}{\isacharparenright}{\kern0pt}\ x\ {\isasympartial}M{\isachardoublequoteclose}\ \isakeyword{for}\ A\ \isacommand{unfolding}\isamarkupfalse%
\ real{\isacharunderscore}{\kern0pt}norm{\isacharunderscore}{\kern0pt}def\ \isacommand{using}\isamarkupfalse%
\ assms\ integrable{\isacharunderscore}{\kern0pt}abs\ real{\isacharunderscore}{\kern0pt}cond{\isacharunderscore}{\kern0pt}exp{\isacharunderscore}{\kern0pt}intA\ \isacommand{by}\isamarkupfalse%
\ blast\isanewline
\ \ \isanewline
\ \ \isacommand{have}\isamarkupfalse%
\ norm{\isacharunderscore}{\kern0pt}int{\isacharcolon}{\kern0pt}\ {\isachardoublequoteopen}A\ {\isasymin}\ sets\ F\ {\isasymLongrightarrow}\ {\isacharparenleft}{\kern0pt}{\isasymintegral}x{\isasymin}A{\isachardot}{\kern0pt}\ {\isasymbar}f\ x{\isasymbar}\ {\isasympartial}M{\isacharparenright}{\kern0pt}\ {\isacharequal}{\kern0pt}\ {\isacharparenleft}{\kern0pt}{\isasymintegral}\isactrlsup {\isacharplus}{\kern0pt}x{\isasymin}A{\isachardot}{\kern0pt}\ {\isasymbar}f\ x{\isasymbar}\ {\isasympartial}M{\isacharparenright}{\kern0pt}{\isachardoublequoteclose}\ \isakeyword{for}\ A\ \isacommand{using}\isamarkupfalse%
\ assms\ \isacommand{by}\isamarkupfalse%
\ {\isacharparenleft}{\kern0pt}intro\ nn{\isacharunderscore}{\kern0pt}set{\isacharunderscore}{\kern0pt}integral{\isacharunderscore}{\kern0pt}eq{\isacharunderscore}{\kern0pt}set{\isacharunderscore}{\kern0pt}integral{\isacharbrackleft}{\kern0pt}symmetric{\isacharbrackright}{\kern0pt}{\isacharcomma}{\kern0pt}\ blast{\isacharcomma}{\kern0pt}\ fastforce{\isacharparenright}{\kern0pt}\ {\isacharparenleft}{\kern0pt}meson\ subalg\ subalgebra{\isacharunderscore}{\kern0pt}def\ subsetD{\isacharparenright}{\kern0pt}\isanewline
\ \ \isanewline
\ \ \isacommand{have}\isamarkupfalse%
\ {\isachardoublequoteopen}AE\ x\ in\ M{\isachardot}{\kern0pt}\ real{\isacharunderscore}{\kern0pt}cond{\isacharunderscore}{\kern0pt}exp\ M\ F\ {\isacharparenleft}{\kern0pt}{\isasymlambda}x{\isachardot}{\kern0pt}\ norm\ {\isacharparenleft}{\kern0pt}f\ x{\isacharparenright}{\kern0pt}{\isacharparenright}{\kern0pt}\ x\ {\isasymge}\ {\isadigit{0}}{\isachardoublequoteclose}\ \isacommand{using}\isamarkupfalse%
\ int\ real{\isacharunderscore}{\kern0pt}cond{\isacharunderscore}{\kern0pt}exp{\isacharunderscore}{\kern0pt}ge{\isacharunderscore}{\kern0pt}c\ \isacommand{by}\isamarkupfalse%
\ force\isanewline
\ \ \isacommand{hence}\isamarkupfalse%
\ cond{\isacharunderscore}{\kern0pt}exp{\isacharunderscore}{\kern0pt}norm{\isacharunderscore}{\kern0pt}int{\isacharcolon}{\kern0pt}\ {\isachardoublequoteopen}A\ {\isasymin}\ sets\ F\ {\isasymLongrightarrow}\ {\isacharparenleft}{\kern0pt}{\isasymintegral}x{\isasymin}A{\isachardot}{\kern0pt}\ real{\isacharunderscore}{\kern0pt}cond{\isacharunderscore}{\kern0pt}exp\ M\ F\ {\isacharparenleft}{\kern0pt}{\isasymlambda}x{\isachardot}{\kern0pt}\ norm\ {\isacharparenleft}{\kern0pt}f\ x{\isacharparenright}{\kern0pt}{\isacharparenright}{\kern0pt}\ x\ {\isasympartial}M{\isacharparenright}{\kern0pt}\ {\isacharequal}{\kern0pt}\ {\isacharparenleft}{\kern0pt}{\isasymintegral}\isactrlsup {\isacharplus}{\kern0pt}x{\isasymin}A{\isachardot}{\kern0pt}\ real{\isacharunderscore}{\kern0pt}cond{\isacharunderscore}{\kern0pt}exp\ M\ F\ {\isacharparenleft}{\kern0pt}{\isasymlambda}x{\isachardot}{\kern0pt}\ norm\ {\isacharparenleft}{\kern0pt}f\ x{\isacharparenright}{\kern0pt}{\isacharparenright}{\kern0pt}\ x\ {\isasympartial}M{\isacharparenright}{\kern0pt}{\isachardoublequoteclose}\ \isakeyword{for}\ A\ \isacommand{using}\isamarkupfalse%
\ assms\ \isacommand{by}\isamarkupfalse%
\ {\isacharparenleft}{\kern0pt}intro\ nn{\isacharunderscore}{\kern0pt}set{\isacharunderscore}{\kern0pt}integral{\isacharunderscore}{\kern0pt}eq{\isacharunderscore}{\kern0pt}set{\isacharunderscore}{\kern0pt}integral{\isacharbrackleft}{\kern0pt}symmetric{\isacharbrackright}{\kern0pt}{\isacharcomma}{\kern0pt}\ blast{\isacharcomma}{\kern0pt}\ fastforce{\isacharparenright}{\kern0pt}\ {\isacharparenleft}{\kern0pt}meson\ subalg\ subalgebra{\isacharunderscore}{\kern0pt}def\ subsetD{\isacharparenright}{\kern0pt}\isanewline
\ \ \isanewline
\ \ \isacommand{have}\isamarkupfalse%
\ {\isachardoublequoteopen}A\ {\isasymin}\ sets\ F\ {\isasymLongrightarrow}\ {\isasymintegral}\isactrlsup {\isacharplus}{\kern0pt}x{\isasymin}A{\isachardot}{\kern0pt}\ {\isasymbar}f\ x{\isasymbar}{\isasympartial}M\ {\isacharequal}{\kern0pt}\ {\isasymintegral}\isactrlsup {\isacharplus}{\kern0pt}x{\isasymin}A{\isachardot}{\kern0pt}\ real{\isacharunderscore}{\kern0pt}cond{\isacharunderscore}{\kern0pt}exp\ M\ F\ {\isacharparenleft}{\kern0pt}{\isasymlambda}x{\isachardot}{\kern0pt}\ norm\ {\isacharparenleft}{\kern0pt}f\ x{\isacharparenright}{\kern0pt}{\isacharparenright}{\kern0pt}\ x\ {\isasympartial}M{\isachardoublequoteclose}\ \isakeyword{for}\ A\ \isacommand{using}\isamarkupfalse%
\ {\isacharasterisk}{\kern0pt}{\isacharasterisk}{\kern0pt}\ norm{\isacharunderscore}{\kern0pt}int\ cond{\isacharunderscore}{\kern0pt}exp{\isacharunderscore}{\kern0pt}norm{\isacharunderscore}{\kern0pt}int\ \isacommand{by}\isamarkupfalse%
\ {\isacharparenleft}{\kern0pt}auto\ simp\ add{\isacharcolon}{\kern0pt}\ nn{\isacharunderscore}{\kern0pt}integral{\isacharunderscore}{\kern0pt}set{\isacharunderscore}{\kern0pt}ennreal{\isacharparenright}{\kern0pt}\isanewline
\ \ \isacommand{moreover}\isamarkupfalse%
\ \isacommand{have}\isamarkupfalse%
\ {\isachardoublequoteopen}{\isacharparenleft}{\kern0pt}{\isasymlambda}x{\isachardot}{\kern0pt}\ ennreal\ {\isasymbar}f\ x{\isasymbar}{\isacharparenright}{\kern0pt}\ {\isasymin}\ borel{\isacharunderscore}{\kern0pt}measurable\ M{\isachardoublequoteclose}\ \isacommand{by}\isamarkupfalse%
\ measurable\isanewline
\ \ \isacommand{moreover}\isamarkupfalse%
\ \isacommand{have}\isamarkupfalse%
\ {\isachardoublequoteopen}{\isacharparenleft}{\kern0pt}{\isasymlambda}x{\isachardot}{\kern0pt}\ ennreal\ {\isacharparenleft}{\kern0pt}real{\isacharunderscore}{\kern0pt}cond{\isacharunderscore}{\kern0pt}exp\ M\ F\ {\isacharparenleft}{\kern0pt}{\isasymlambda}x{\isachardot}{\kern0pt}\ norm\ {\isacharparenleft}{\kern0pt}f\ x{\isacharparenright}{\kern0pt}{\isacharparenright}{\kern0pt}\ x{\isacharparenright}{\kern0pt}{\isacharparenright}{\kern0pt}\ {\isasymin}\ borel{\isacharunderscore}{\kern0pt}measurable\ F{\isachardoublequoteclose}\ \isacommand{by}\isamarkupfalse%
\ measurable\isanewline
\ \ \isacommand{ultimately}\isamarkupfalse%
\ \isacommand{have}\isamarkupfalse%
\ {\isachardoublequoteopen}AE\ x\ in\ M{\isachardot}{\kern0pt}\ nn{\isacharunderscore}{\kern0pt}cond{\isacharunderscore}{\kern0pt}exp\ M\ F\ {\isacharparenleft}{\kern0pt}{\isasymlambda}x{\isachardot}{\kern0pt}\ ennreal\ {\isasymbar}f\ x{\isasymbar}{\isacharparenright}{\kern0pt}\ x\ {\isacharequal}{\kern0pt}\ real{\isacharunderscore}{\kern0pt}cond{\isacharunderscore}{\kern0pt}exp\ M\ F\ {\isacharparenleft}{\kern0pt}{\isasymlambda}x{\isachardot}{\kern0pt}\ norm\ {\isacharparenleft}{\kern0pt}f\ x{\isacharparenright}{\kern0pt}{\isacharparenright}{\kern0pt}\ x{\isachardoublequoteclose}\ \isacommand{by}\isamarkupfalse%
\ {\isacharparenleft}{\kern0pt}intro\ nn{\isacharunderscore}{\kern0pt}cond{\isacharunderscore}{\kern0pt}exp{\isacharunderscore}{\kern0pt}charact{\isacharbrackleft}{\kern0pt}THEN\ AE{\isacharunderscore}{\kern0pt}symmetric{\isacharbrackright}{\kern0pt}{\isacharcomma}{\kern0pt}\ auto{\isacharparenright}{\kern0pt}\isanewline
\ \ \isacommand{hence}\isamarkupfalse%
\ {\isachardoublequoteopen}AE\ x\ in\ M{\isachardot}{\kern0pt}\ nn{\isacharunderscore}{\kern0pt}cond{\isacharunderscore}{\kern0pt}exp\ M\ F\ {\isacharparenleft}{\kern0pt}{\isasymlambda}x{\isachardot}{\kern0pt}\ ennreal\ {\isasymbar}f\ x{\isasymbar}{\isacharparenright}{\kern0pt}\ x\ {\isasymle}\ cond{\isacharunderscore}{\kern0pt}exp\ M\ F\ {\isacharparenleft}{\kern0pt}{\isasymlambda}x{\isachardot}{\kern0pt}\ norm\ {\isacharparenleft}{\kern0pt}f\ x{\isacharparenright}{\kern0pt}{\isacharparenright}{\kern0pt}\ x{\isachardoublequoteclose}\ \isacommand{using}\isamarkupfalse%
\ cond{\isacharunderscore}{\kern0pt}exp{\isacharunderscore}{\kern0pt}real{\isacharbrackleft}{\kern0pt}OF\ int{\isacharbrackright}{\kern0pt}\ \isacommand{by}\isamarkupfalse%
\ force\isanewline
\ \ \isacommand{moreover}\isamarkupfalse%
\ \isacommand{have}\isamarkupfalse%
\ {\isachardoublequoteopen}AE\ x\ in\ M{\isachardot}{\kern0pt}\ {\isasymbar}real{\isacharunderscore}{\kern0pt}cond{\isacharunderscore}{\kern0pt}exp\ M\ F\ f\ x{\isasymbar}\ {\isacharequal}{\kern0pt}\ norm\ {\isacharparenleft}{\kern0pt}cond{\isacharunderscore}{\kern0pt}exp\ M\ F\ f\ x{\isacharparenright}{\kern0pt}{\isachardoublequoteclose}\ \isacommand{unfolding}\isamarkupfalse%
\ real{\isacharunderscore}{\kern0pt}norm{\isacharunderscore}{\kern0pt}def\ \isacommand{using}\isamarkupfalse%
\ cond{\isacharunderscore}{\kern0pt}exp{\isacharunderscore}{\kern0pt}real{\isacharbrackleft}{\kern0pt}OF\ assms{\isacharbrackright}{\kern0pt}\ {\isacharasterisk}{\kern0pt}\ \isacommand{by}\isamarkupfalse%
\ force\isanewline
\ \ \isacommand{ultimately}\isamarkupfalse%
\ \isacommand{have}\isamarkupfalse%
\ {\isachardoublequoteopen}AE\ x\ in\ M{\isachardot}{\kern0pt}\ ennreal\ {\isacharparenleft}{\kern0pt}norm\ {\isacharparenleft}{\kern0pt}cond{\isacharunderscore}{\kern0pt}exp\ M\ F\ f\ x{\isacharparenright}{\kern0pt}{\isacharparenright}{\kern0pt}\ {\isasymle}\ cond{\isacharunderscore}{\kern0pt}exp\ M\ F\ {\isacharparenleft}{\kern0pt}{\isasymlambda}x{\isachardot}{\kern0pt}\ norm\ {\isacharparenleft}{\kern0pt}f\ x{\isacharparenright}{\kern0pt}{\isacharparenright}{\kern0pt}\ x{\isachardoublequoteclose}\ \isacommand{using}\isamarkupfalse%
\ real{\isacharunderscore}{\kern0pt}cond{\isacharunderscore}{\kern0pt}exp{\isacharunderscore}{\kern0pt}abs{\isacharbrackleft}{\kern0pt}OF\ assms{\isacharbrackleft}{\kern0pt}THEN\ borel{\isacharunderscore}{\kern0pt}measurable{\isacharunderscore}{\kern0pt}integrable{\isacharbrackright}{\kern0pt}{\isacharbrackright}{\kern0pt}\ \isacommand{by}\isamarkupfalse%
\ fastforce\isanewline
\ \ \isacommand{hence}\isamarkupfalse%
\ {\isachardoublequoteopen}AE\ x\ in\ M{\isachardot}{\kern0pt}\ enn{\isadigit{2}}real\ {\isacharparenleft}{\kern0pt}ennreal\ {\isacharparenleft}{\kern0pt}norm\ {\isacharparenleft}{\kern0pt}cond{\isacharunderscore}{\kern0pt}exp\ M\ F\ f\ x{\isacharparenright}{\kern0pt}{\isacharparenright}{\kern0pt}{\isacharparenright}{\kern0pt}\ {\isasymle}\ enn{\isadigit{2}}real\ {\isacharparenleft}{\kern0pt}cond{\isacharunderscore}{\kern0pt}exp\ M\ F\ {\isacharparenleft}{\kern0pt}{\isasymlambda}x{\isachardot}{\kern0pt}\ norm\ {\isacharparenleft}{\kern0pt}f\ x{\isacharparenright}{\kern0pt}{\isacharparenright}{\kern0pt}\ x{\isacharparenright}{\kern0pt}{\isachardoublequoteclose}\ \isacommand{using}\isamarkupfalse%
\ ennreal{\isacharunderscore}{\kern0pt}le{\isacharunderscore}{\kern0pt}iff{\isadigit{2}}\ \isacommand{by}\isamarkupfalse%
\ force\isanewline
\ \ \isacommand{thus}\isamarkupfalse%
\ {\isacharquery}{\kern0pt}thesis\ \isacommand{using}\isamarkupfalse%
\ {\isacharasterisk}{\kern0pt}\ \isacommand{by}\isamarkupfalse%
\ fastforce\isanewline
\isacommand{qed}\isamarkupfalse%
%
\endisatagproof
{\isafoldproof}%
%
\isadelimproof
\isanewline
%
\endisadelimproof
\isanewline
\isacommand{lemma}\isamarkupfalse%
\ cond{\isacharunderscore}{\kern0pt}exp{\isacharunderscore}{\kern0pt}contraction{\isacharunderscore}{\kern0pt}simple{\isacharcolon}{\kern0pt}\isanewline
\ \ \isakeyword{fixes}\ f\ {\isacharcolon}{\kern0pt}{\isacharcolon}{\kern0pt}\ {\isachardoublequoteopen}{\isacharprime}{\kern0pt}a\ {\isasymRightarrow}\ {\isacharprime}{\kern0pt}b{\isacharcolon}{\kern0pt}{\isacharcolon}{\kern0pt}{\isacharbraceleft}{\kern0pt}second{\isacharunderscore}{\kern0pt}countable{\isacharunderscore}{\kern0pt}topology{\isacharcomma}{\kern0pt}\ banach{\isacharbraceright}{\kern0pt}{\isachardoublequoteclose}\isanewline
\ \ \isakeyword{assumes}\ {\isachardoublequoteopen}simple{\isacharunderscore}{\kern0pt}function\ M\ f{\isachardoublequoteclose}\ {\isachardoublequoteopen}emeasure\ M\ {\isacharbraceleft}{\kern0pt}y\ {\isasymin}\ space\ M{\isachardot}{\kern0pt}\ f\ y\ {\isasymnoteq}\ {\isadigit{0}}{\isacharbraceright}{\kern0pt}\ {\isasymnoteq}\ {\isasyminfinity}{\isachardoublequoteclose}\isanewline
\ \ \isakeyword{shows}\ {\isachardoublequoteopen}AE\ x\ in\ M{\isachardot}{\kern0pt}\ norm\ {\isacharparenleft}{\kern0pt}cond{\isacharunderscore}{\kern0pt}exp\ M\ F\ f\ x{\isacharparenright}{\kern0pt}\ {\isasymle}\ cond{\isacharunderscore}{\kern0pt}exp\ M\ F\ {\isacharparenleft}{\kern0pt}{\isasymlambda}x{\isachardot}{\kern0pt}\ norm\ {\isacharparenleft}{\kern0pt}f\ x{\isacharparenright}{\kern0pt}{\isacharparenright}{\kern0pt}\ x{\isachardoublequoteclose}\isanewline
%
\isadelimproof
\ \ %
\endisadelimproof
%
\isatagproof
\isacommand{using}\isamarkupfalse%
\ assms\isanewline
\isacommand{proof}\isamarkupfalse%
\ {\isacharparenleft}{\kern0pt}induction\ rule{\isacharcolon}{\kern0pt}\ integrable{\isacharunderscore}{\kern0pt}simple{\isacharunderscore}{\kern0pt}function{\isacharunderscore}{\kern0pt}induct{\isacharparenright}{\kern0pt}\isanewline
\ \ \isacommand{case}\isamarkupfalse%
\ {\isacharparenleft}{\kern0pt}cong\ f\ g{\isacharparenright}{\kern0pt}\isanewline
\ \ \isacommand{hence}\isamarkupfalse%
\ ae{\isacharcolon}{\kern0pt}\ {\isachardoublequoteopen}AE\ x\ in\ M{\isachardot}{\kern0pt}\ f\ x\ {\isacharequal}{\kern0pt}\ g\ x{\isachardoublequoteclose}\ \isacommand{by}\isamarkupfalse%
\ blast\isanewline
\ \ \isacommand{hence}\isamarkupfalse%
\ {\isachardoublequoteopen}AE\ x\ in\ M{\isachardot}{\kern0pt}\ cond{\isacharunderscore}{\kern0pt}exp\ M\ F\ f\ x\ {\isacharequal}{\kern0pt}\ cond{\isacharunderscore}{\kern0pt}exp\ M\ F\ g\ x{\isachardoublequoteclose}\ \isacommand{using}\isamarkupfalse%
\ cong\ has{\isacharunderscore}{\kern0pt}cond{\isacharunderscore}{\kern0pt}exp{\isacharunderscore}{\kern0pt}simple\ \isacommand{by}\isamarkupfalse%
\ {\isacharparenleft}{\kern0pt}subst\ cond{\isacharunderscore}{\kern0pt}exp{\isacharunderscore}{\kern0pt}cong{\isacharunderscore}{\kern0pt}AE{\isacharparenright}{\kern0pt}\ {\isacharparenleft}{\kern0pt}auto\ intro{\isacharbang}{\kern0pt}{\isacharcolon}{\kern0pt}\ has{\isacharunderscore}{\kern0pt}cond{\isacharunderscore}{\kern0pt}expD{\isacharparenleft}{\kern0pt}{\isadigit{2}}{\isacharparenright}{\kern0pt}{\isacharparenright}{\kern0pt}\isanewline
\ \ \isacommand{hence}\isamarkupfalse%
\ {\isachardoublequoteopen}AE\ x\ in\ M{\isachardot}{\kern0pt}\ norm\ {\isacharparenleft}{\kern0pt}cond{\isacharunderscore}{\kern0pt}exp\ M\ F\ f\ x{\isacharparenright}{\kern0pt}\ {\isacharequal}{\kern0pt}\ norm\ {\isacharparenleft}{\kern0pt}cond{\isacharunderscore}{\kern0pt}exp\ M\ F\ g\ x{\isacharparenright}{\kern0pt}{\isachardoublequoteclose}\ \isacommand{by}\isamarkupfalse%
\ force\isanewline
\ \ \isacommand{moreover}\isamarkupfalse%
\ \isacommand{have}\isamarkupfalse%
\ {\isachardoublequoteopen}AE\ x\ in\ M{\isachardot}{\kern0pt}\ cond{\isacharunderscore}{\kern0pt}exp\ M\ F\ {\isacharparenleft}{\kern0pt}{\isasymlambda}x{\isachardot}{\kern0pt}\ norm\ {\isacharparenleft}{\kern0pt}f\ x{\isacharparenright}{\kern0pt}{\isacharparenright}{\kern0pt}\ x\ {\isacharequal}{\kern0pt}\ cond{\isacharunderscore}{\kern0pt}exp\ M\ F\ {\isacharparenleft}{\kern0pt}{\isasymlambda}x{\isachardot}{\kern0pt}\ norm\ {\isacharparenleft}{\kern0pt}g\ x{\isacharparenright}{\kern0pt}{\isacharparenright}{\kern0pt}\ x{\isachardoublequoteclose}\ \ \isacommand{using}\isamarkupfalse%
\ ae\ cong\ has{\isacharunderscore}{\kern0pt}cond{\isacharunderscore}{\kern0pt}exp{\isacharunderscore}{\kern0pt}simple\ \isacommand{by}\isamarkupfalse%
\ {\isacharparenleft}{\kern0pt}subst\ cond{\isacharunderscore}{\kern0pt}exp{\isacharunderscore}{\kern0pt}cong{\isacharunderscore}{\kern0pt}AE{\isacharparenright}{\kern0pt}\ {\isacharparenleft}{\kern0pt}auto\ dest{\isacharcolon}{\kern0pt}\ has{\isacharunderscore}{\kern0pt}cond{\isacharunderscore}{\kern0pt}expD{\isacharparenright}{\kern0pt}\isanewline
\ \ \isacommand{ultimately}\isamarkupfalse%
\ \isacommand{show}\isamarkupfalse%
\ {\isacharquery}{\kern0pt}case\ \isacommand{using}\isamarkupfalse%
\ cong{\isacharparenleft}{\kern0pt}{\isadigit{6}}{\isacharparenright}{\kern0pt}\ \isacommand{by}\isamarkupfalse%
\ fastforce\isanewline
\isacommand{next}\isamarkupfalse%
\isanewline
\ \ \isacommand{case}\isamarkupfalse%
\ {\isacharparenleft}{\kern0pt}indicator\ A\ y{\isacharparenright}{\kern0pt}\isanewline
\ \ \isacommand{hence}\isamarkupfalse%
\ {\isachardoublequoteopen}AE\ x\ in\ M{\isachardot}{\kern0pt}\ cond{\isacharunderscore}{\kern0pt}exp\ M\ F\ {\isacharparenleft}{\kern0pt}{\isasymlambda}a{\isachardot}{\kern0pt}\ indicator\ A\ a\ {\isacharasterisk}{\kern0pt}\isactrlsub R\ y{\isacharparenright}{\kern0pt}\ x\ {\isacharequal}{\kern0pt}\ cond{\isacharunderscore}{\kern0pt}exp\ M\ F\ {\isacharparenleft}{\kern0pt}indicator\ A{\isacharparenright}{\kern0pt}\ x\ {\isacharasterisk}{\kern0pt}\isactrlsub R\ y{\isachardoublequoteclose}\ \isacommand{by}\isamarkupfalse%
\ blast\isanewline
\ \ \isacommand{hence}\isamarkupfalse%
\ {\isacharasterisk}{\kern0pt}{\isacharcolon}{\kern0pt}\ {\isachardoublequoteopen}AE\ x\ in\ M{\isachardot}{\kern0pt}\ norm\ {\isacharparenleft}{\kern0pt}cond{\isacharunderscore}{\kern0pt}exp\ M\ F\ {\isacharparenleft}{\kern0pt}{\isasymlambda}a{\isachardot}{\kern0pt}\ indicat{\isacharunderscore}{\kern0pt}real\ A\ a\ {\isacharasterisk}{\kern0pt}\isactrlsub R\ y{\isacharparenright}{\kern0pt}\ x{\isacharparenright}{\kern0pt}\ {\isasymle}\ norm\ y\ {\isacharasterisk}{\kern0pt}\ cond{\isacharunderscore}{\kern0pt}exp\ M\ F\ {\isacharparenleft}{\kern0pt}{\isasymlambda}x{\isachardot}{\kern0pt}\ norm\ {\isacharparenleft}{\kern0pt}indicat{\isacharunderscore}{\kern0pt}real\ A\ x{\isacharparenright}{\kern0pt}{\isacharparenright}{\kern0pt}\ x{\isachardoublequoteclose}\ \isacommand{using}\isamarkupfalse%
\ cond{\isacharunderscore}{\kern0pt}exp{\isacharunderscore}{\kern0pt}contraction{\isacharunderscore}{\kern0pt}real{\isacharbrackleft}{\kern0pt}OF\ integrable{\isacharunderscore}{\kern0pt}real{\isacharunderscore}{\kern0pt}indicator{\isacharcomma}{\kern0pt}\ OF\ indicator{\isacharbrackright}{\kern0pt}\ \isacommand{by}\isamarkupfalse%
\ fastforce\isanewline
\isanewline
\ \ \isacommand{have}\isamarkupfalse%
\ {\isachardoublequoteopen}AE\ x\ in\ M{\isachardot}{\kern0pt}\ norm\ y\ {\isacharasterisk}{\kern0pt}\ cond{\isacharunderscore}{\kern0pt}exp\ M\ F\ {\isacharparenleft}{\kern0pt}{\isasymlambda}x{\isachardot}{\kern0pt}\ norm\ {\isacharparenleft}{\kern0pt}indicat{\isacharunderscore}{\kern0pt}real\ A\ x{\isacharparenright}{\kern0pt}{\isacharparenright}{\kern0pt}\ x\ {\isacharequal}{\kern0pt}\ norm\ y\ {\isacharasterisk}{\kern0pt}\ real{\isacharunderscore}{\kern0pt}cond{\isacharunderscore}{\kern0pt}exp\ M\ F\ {\isacharparenleft}{\kern0pt}{\isasymlambda}x{\isachardot}{\kern0pt}\ norm\ {\isacharparenleft}{\kern0pt}indicat{\isacharunderscore}{\kern0pt}real\ A\ x{\isacharparenright}{\kern0pt}{\isacharparenright}{\kern0pt}\ x{\isachardoublequoteclose}\ \isacommand{using}\isamarkupfalse%
\ cond{\isacharunderscore}{\kern0pt}exp{\isacharunderscore}{\kern0pt}real{\isacharbrackleft}{\kern0pt}OF\ integrable{\isacharunderscore}{\kern0pt}real{\isacharunderscore}{\kern0pt}indicator{\isacharcomma}{\kern0pt}\ OF\ indicator{\isacharbrackright}{\kern0pt}\ \isacommand{by}\isamarkupfalse%
\ fastforce\isanewline
\ \ \isacommand{moreover}\isamarkupfalse%
\ \isacommand{have}\isamarkupfalse%
\ {\isachardoublequoteopen}AE\ x\ in\ M{\isachardot}{\kern0pt}\ cond{\isacharunderscore}{\kern0pt}exp\ M\ F\ {\isacharparenleft}{\kern0pt}{\isasymlambda}x{\isachardot}{\kern0pt}\ norm\ y\ {\isacharasterisk}{\kern0pt}\ norm\ {\isacharparenleft}{\kern0pt}indicat{\isacharunderscore}{\kern0pt}real\ A\ x{\isacharparenright}{\kern0pt}{\isacharparenright}{\kern0pt}\ x\ {\isacharequal}{\kern0pt}\ real{\isacharunderscore}{\kern0pt}cond{\isacharunderscore}{\kern0pt}exp\ M\ F\ {\isacharparenleft}{\kern0pt}{\isasymlambda}x{\isachardot}{\kern0pt}\ norm\ y\ {\isacharasterisk}{\kern0pt}\ norm\ {\isacharparenleft}{\kern0pt}indicat{\isacharunderscore}{\kern0pt}real\ A\ x{\isacharparenright}{\kern0pt}{\isacharparenright}{\kern0pt}\ x{\isachardoublequoteclose}\ \isacommand{using}\isamarkupfalse%
\ indicator\ \isacommand{by}\isamarkupfalse%
\ {\isacharparenleft}{\kern0pt}intro\ cond{\isacharunderscore}{\kern0pt}exp{\isacharunderscore}{\kern0pt}real{\isacharcomma}{\kern0pt}\ auto{\isacharparenright}{\kern0pt}\isanewline
\ \ \isacommand{ultimately}\isamarkupfalse%
\ \isacommand{have}\isamarkupfalse%
\ {\isachardoublequoteopen}AE\ x\ in\ M{\isachardot}{\kern0pt}\ norm\ y\ {\isacharasterisk}{\kern0pt}\ cond{\isacharunderscore}{\kern0pt}exp\ M\ F\ {\isacharparenleft}{\kern0pt}{\isasymlambda}x{\isachardot}{\kern0pt}\ norm\ {\isacharparenleft}{\kern0pt}indicat{\isacharunderscore}{\kern0pt}real\ A\ x{\isacharparenright}{\kern0pt}{\isacharparenright}{\kern0pt}\ x\ {\isacharequal}{\kern0pt}\ cond{\isacharunderscore}{\kern0pt}exp\ M\ F\ {\isacharparenleft}{\kern0pt}{\isasymlambda}x{\isachardot}{\kern0pt}\ norm\ y\ {\isacharasterisk}{\kern0pt}\ norm\ {\isacharparenleft}{\kern0pt}indicat{\isacharunderscore}{\kern0pt}real\ A\ x{\isacharparenright}{\kern0pt}{\isacharparenright}{\kern0pt}\ x{\isachardoublequoteclose}\ \isacommand{using}\isamarkupfalse%
\ real{\isacharunderscore}{\kern0pt}cond{\isacharunderscore}{\kern0pt}exp{\isacharunderscore}{\kern0pt}cmult{\isacharbrackleft}{\kern0pt}of\ {\isachardoublequoteopen}{\isasymlambda}x{\isachardot}{\kern0pt}\ norm\ {\isacharparenleft}{\kern0pt}indicat{\isacharunderscore}{\kern0pt}real\ A\ x{\isacharparenright}{\kern0pt}{\isachardoublequoteclose}\ {\isachardoublequoteopen}norm\ y{\isachardoublequoteclose}{\isacharbrackright}{\kern0pt}\ indicator\ \isacommand{by}\isamarkupfalse%
\ fastforce\isanewline
\ \ \isacommand{moreover}\isamarkupfalse%
\ \isacommand{have}\isamarkupfalse%
\ {\isachardoublequoteopen}{\isacharparenleft}{\kern0pt}{\isasymlambda}x{\isachardot}{\kern0pt}\ norm\ y\ {\isacharasterisk}{\kern0pt}\ norm\ {\isacharparenleft}{\kern0pt}indicat{\isacharunderscore}{\kern0pt}real\ A\ x{\isacharparenright}{\kern0pt}{\isacharparenright}{\kern0pt}\ {\isacharequal}{\kern0pt}\ {\isacharparenleft}{\kern0pt}{\isasymlambda}x{\isachardot}{\kern0pt}\ norm\ {\isacharparenleft}{\kern0pt}indicat{\isacharunderscore}{\kern0pt}real\ A\ x\ {\isacharasterisk}{\kern0pt}\isactrlsub R\ y{\isacharparenright}{\kern0pt}{\isacharparenright}{\kern0pt}{\isachardoublequoteclose}\ \isacommand{by}\isamarkupfalse%
\ force\isanewline
\ \ \isacommand{ultimately}\isamarkupfalse%
\ \isacommand{show}\isamarkupfalse%
\ {\isacharquery}{\kern0pt}case\ \isacommand{using}\isamarkupfalse%
\ {\isacharasterisk}{\kern0pt}\ \isacommand{by}\isamarkupfalse%
\ force\isanewline
\isacommand{next}\isamarkupfalse%
\isanewline
\ \ \isacommand{case}\isamarkupfalse%
\ {\isacharparenleft}{\kern0pt}add\ u\ v{\isacharparenright}{\kern0pt}\isanewline
\ \ \isacommand{have}\isamarkupfalse%
\ {\isachardoublequoteopen}AE\ x\ in\ M{\isachardot}{\kern0pt}\ norm\ {\isacharparenleft}{\kern0pt}cond{\isacharunderscore}{\kern0pt}exp\ M\ F\ {\isacharparenleft}{\kern0pt}{\isasymlambda}a{\isachardot}{\kern0pt}\ u\ a\ {\isacharplus}{\kern0pt}\ v\ a{\isacharparenright}{\kern0pt}\ x{\isacharparenright}{\kern0pt}\ {\isacharequal}{\kern0pt}\ norm\ {\isacharparenleft}{\kern0pt}cond{\isacharunderscore}{\kern0pt}exp\ M\ F\ u\ x\ {\isacharplus}{\kern0pt}\ cond{\isacharunderscore}{\kern0pt}exp\ M\ F\ v\ x{\isacharparenright}{\kern0pt}{\isachardoublequoteclose}\ \isacommand{using}\isamarkupfalse%
\ has{\isacharunderscore}{\kern0pt}cond{\isacharunderscore}{\kern0pt}exp{\isacharunderscore}{\kern0pt}charact{\isacharparenleft}{\kern0pt}{\isadigit{2}}{\isacharparenright}{\kern0pt}{\isacharbrackleft}{\kern0pt}OF\ has{\isacharunderscore}{\kern0pt}cond{\isacharunderscore}{\kern0pt}exp{\isacharunderscore}{\kern0pt}add{\isacharcomma}{\kern0pt}\ OF\ has{\isacharunderscore}{\kern0pt}cond{\isacharunderscore}{\kern0pt}exp{\isacharunderscore}{\kern0pt}simple{\isacharparenleft}{\kern0pt}{\isadigit{1}}{\isacharcomma}{\kern0pt}{\isadigit{1}}{\isacharparenright}{\kern0pt}{\isacharcomma}{\kern0pt}\ OF\ add{\isacharparenleft}{\kern0pt}{\isadigit{1}}{\isacharcomma}{\kern0pt}{\isadigit{2}}{\isacharcomma}{\kern0pt}{\isadigit{3}}{\isacharcomma}{\kern0pt}{\isadigit{4}}{\isacharparenright}{\kern0pt}{\isacharbrackright}{\kern0pt}\ \isacommand{by}\isamarkupfalse%
\ fastforce\isanewline
\ \ \isacommand{moreover}\isamarkupfalse%
\ \isacommand{have}\isamarkupfalse%
\ {\isachardoublequoteopen}AE\ x\ in\ M{\isachardot}{\kern0pt}\ norm\ {\isacharparenleft}{\kern0pt}cond{\isacharunderscore}{\kern0pt}exp\ M\ F\ u\ x\ {\isacharplus}{\kern0pt}\ cond{\isacharunderscore}{\kern0pt}exp\ M\ F\ v\ x{\isacharparenright}{\kern0pt}\ {\isasymle}\ norm\ {\isacharparenleft}{\kern0pt}cond{\isacharunderscore}{\kern0pt}exp\ M\ F\ u\ x{\isacharparenright}{\kern0pt}\ {\isacharplus}{\kern0pt}\ norm\ {\isacharparenleft}{\kern0pt}cond{\isacharunderscore}{\kern0pt}exp\ M\ F\ v\ x{\isacharparenright}{\kern0pt}{\isachardoublequoteclose}\ \isacommand{using}\isamarkupfalse%
\ norm{\isacharunderscore}{\kern0pt}triangle{\isacharunderscore}{\kern0pt}ineq\ \isacommand{by}\isamarkupfalse%
\ blast\isanewline
\ \ \isacommand{moreover}\isamarkupfalse%
\ \isacommand{have}\isamarkupfalse%
\ {\isachardoublequoteopen}AE\ x\ in\ M{\isachardot}{\kern0pt}\ norm\ {\isacharparenleft}{\kern0pt}cond{\isacharunderscore}{\kern0pt}exp\ M\ F\ u\ x{\isacharparenright}{\kern0pt}\ {\isacharplus}{\kern0pt}\ norm\ {\isacharparenleft}{\kern0pt}cond{\isacharunderscore}{\kern0pt}exp\ M\ F\ v\ x{\isacharparenright}{\kern0pt}\ {\isasymle}\ cond{\isacharunderscore}{\kern0pt}exp\ M\ F\ {\isacharparenleft}{\kern0pt}{\isasymlambda}x{\isachardot}{\kern0pt}\ norm\ {\isacharparenleft}{\kern0pt}u\ x{\isacharparenright}{\kern0pt}{\isacharparenright}{\kern0pt}\ x\ {\isacharplus}{\kern0pt}\ cond{\isacharunderscore}{\kern0pt}exp\ M\ F\ {\isacharparenleft}{\kern0pt}{\isasymlambda}x{\isachardot}{\kern0pt}\ norm\ {\isacharparenleft}{\kern0pt}v\ x{\isacharparenright}{\kern0pt}{\isacharparenright}{\kern0pt}\ x{\isachardoublequoteclose}\ \isacommand{using}\isamarkupfalse%
\ add{\isacharparenleft}{\kern0pt}{\isadigit{6}}{\isacharcomma}{\kern0pt}{\isadigit{7}}{\isacharparenright}{\kern0pt}\ \isacommand{by}\isamarkupfalse%
\ fastforce\isanewline
\ \ \isacommand{moreover}\isamarkupfalse%
\ \isacommand{have}\isamarkupfalse%
\ {\isachardoublequoteopen}AE\ x\ in\ M{\isachardot}{\kern0pt}\ cond{\isacharunderscore}{\kern0pt}exp\ M\ F\ {\isacharparenleft}{\kern0pt}{\isasymlambda}x{\isachardot}{\kern0pt}\ norm\ {\isacharparenleft}{\kern0pt}u\ x{\isacharparenright}{\kern0pt}{\isacharparenright}{\kern0pt}\ x\ {\isacharplus}{\kern0pt}\ cond{\isacharunderscore}{\kern0pt}exp\ M\ F\ {\isacharparenleft}{\kern0pt}{\isasymlambda}x{\isachardot}{\kern0pt}\ norm\ {\isacharparenleft}{\kern0pt}v\ x{\isacharparenright}{\kern0pt}{\isacharparenright}{\kern0pt}\ x\ {\isacharequal}{\kern0pt}\ cond{\isacharunderscore}{\kern0pt}exp\ M\ F\ {\isacharparenleft}{\kern0pt}{\isasymlambda}x{\isachardot}{\kern0pt}\ norm\ {\isacharparenleft}{\kern0pt}u\ x{\isacharparenright}{\kern0pt}\ {\isacharplus}{\kern0pt}\ norm\ {\isacharparenleft}{\kern0pt}v\ x{\isacharparenright}{\kern0pt}{\isacharparenright}{\kern0pt}\ x{\isachardoublequoteclose}\ \isacommand{using}\isamarkupfalse%
\ integrable{\isacharunderscore}{\kern0pt}simple{\isacharunderscore}{\kern0pt}function{\isacharbrackleft}{\kern0pt}OF\ add{\isacharparenleft}{\kern0pt}{\isadigit{1}}{\isacharcomma}{\kern0pt}{\isadigit{2}}{\isacharparenright}{\kern0pt}{\isacharbrackright}{\kern0pt}\ integrable{\isacharunderscore}{\kern0pt}simple{\isacharunderscore}{\kern0pt}function{\isacharbrackleft}{\kern0pt}OF\ add{\isacharparenleft}{\kern0pt}{\isadigit{3}}{\isacharcomma}{\kern0pt}{\isadigit{4}}{\isacharparenright}{\kern0pt}{\isacharbrackright}{\kern0pt}\ \isacommand{by}\isamarkupfalse%
\ {\isacharparenleft}{\kern0pt}intro\ has{\isacharunderscore}{\kern0pt}cond{\isacharunderscore}{\kern0pt}exp{\isacharunderscore}{\kern0pt}charact{\isacharparenleft}{\kern0pt}{\isadigit{2}}{\isacharparenright}{\kern0pt}{\isacharbrackleft}{\kern0pt}OF\ has{\isacharunderscore}{\kern0pt}cond{\isacharunderscore}{\kern0pt}exp{\isacharunderscore}{\kern0pt}add{\isacharbrackleft}{\kern0pt}OF\ has{\isacharunderscore}{\kern0pt}cond{\isacharunderscore}{\kern0pt}exp{\isacharunderscore}{\kern0pt}charact{\isacharparenleft}{\kern0pt}{\isadigit{1}}{\isacharcomma}{\kern0pt}{\isadigit{1}}{\isacharparenright}{\kern0pt}{\isacharbrackright}{\kern0pt}{\isacharcomma}{\kern0pt}\ THEN\ AE{\isacharunderscore}{\kern0pt}symmetric{\isacharbrackright}{\kern0pt}{\isacharcomma}{\kern0pt}\ auto\ intro{\isacharcolon}{\kern0pt}\ has{\isacharunderscore}{\kern0pt}cond{\isacharunderscore}{\kern0pt}exp{\isacharunderscore}{\kern0pt}real{\isacharparenright}{\kern0pt}\isanewline
\ \ \isacommand{moreover}\isamarkupfalse%
\ \isacommand{have}\isamarkupfalse%
\ {\isachardoublequoteopen}AE\ x\ in\ M{\isachardot}{\kern0pt}\ cond{\isacharunderscore}{\kern0pt}exp\ M\ F\ {\isacharparenleft}{\kern0pt}{\isasymlambda}x{\isachardot}{\kern0pt}\ norm\ {\isacharparenleft}{\kern0pt}u\ x{\isacharparenright}{\kern0pt}\ {\isacharplus}{\kern0pt}\ norm\ {\isacharparenleft}{\kern0pt}v\ x{\isacharparenright}{\kern0pt}{\isacharparenright}{\kern0pt}\ x\ {\isacharequal}{\kern0pt}\ cond{\isacharunderscore}{\kern0pt}exp\ M\ F\ {\isacharparenleft}{\kern0pt}{\isasymlambda}x{\isachardot}{\kern0pt}\ norm\ {\isacharparenleft}{\kern0pt}u\ x\ {\isacharplus}{\kern0pt}\ v\ x{\isacharparenright}{\kern0pt}{\isacharparenright}{\kern0pt}\ x{\isachardoublequoteclose}\ \isacommand{using}\isamarkupfalse%
\ add{\isacharparenleft}{\kern0pt}{\isadigit{5}}{\isacharparenright}{\kern0pt}\ integrable{\isacharunderscore}{\kern0pt}simple{\isacharunderscore}{\kern0pt}function{\isacharbrackleft}{\kern0pt}OF\ add{\isacharparenleft}{\kern0pt}{\isadigit{1}}{\isacharcomma}{\kern0pt}{\isadigit{2}}{\isacharparenright}{\kern0pt}{\isacharbrackright}{\kern0pt}\ integrable{\isacharunderscore}{\kern0pt}simple{\isacharunderscore}{\kern0pt}function{\isacharbrackleft}{\kern0pt}OF\ add{\isacharparenleft}{\kern0pt}{\isadigit{3}}{\isacharcomma}{\kern0pt}{\isadigit{4}}{\isacharparenright}{\kern0pt}{\isacharbrackright}{\kern0pt}\ \isacommand{by}\isamarkupfalse%
\ {\isacharparenleft}{\kern0pt}intro\ cond{\isacharunderscore}{\kern0pt}exp{\isacharunderscore}{\kern0pt}cong{\isacharcomma}{\kern0pt}\ auto{\isacharparenright}{\kern0pt}\isanewline
\ \ \isacommand{ultimately}\isamarkupfalse%
\ \isacommand{show}\isamarkupfalse%
\ {\isacharquery}{\kern0pt}case\ \isacommand{by}\isamarkupfalse%
\ force\isanewline
\isacommand{qed}\isamarkupfalse%
%
\endisatagproof
{\isafoldproof}%
%
\isadelimproof
\isanewline
%
\endisadelimproof
\isanewline
\isacommand{lemma}\isamarkupfalse%
\ has{\isacharunderscore}{\kern0pt}cond{\isacharunderscore}{\kern0pt}exp{\isacharunderscore}{\kern0pt}simple{\isacharunderscore}{\kern0pt}lim{\isacharcolon}{\kern0pt}\isanewline
\ \ \ \ \isakeyword{fixes}\ f\ {\isacharcolon}{\kern0pt}{\isacharcolon}{\kern0pt}\ {\isachardoublequoteopen}{\isacharprime}{\kern0pt}a\ {\isasymRightarrow}\ {\isacharprime}{\kern0pt}b{\isacharcolon}{\kern0pt}{\isacharcolon}{\kern0pt}{\isacharbraceleft}{\kern0pt}second{\isacharunderscore}{\kern0pt}countable{\isacharunderscore}{\kern0pt}topology{\isacharcomma}{\kern0pt}\ banach{\isacharbraceright}{\kern0pt}{\isachardoublequoteclose}\isanewline
\ \ \isakeyword{assumes}\ integrable{\isacharbrackleft}{\kern0pt}measurable{\isacharbrackright}{\kern0pt}{\isacharcolon}{\kern0pt}\ {\isachardoublequoteopen}integrable\ M\ f{\isachardoublequoteclose}\isanewline
\ \ \ \ \ \ \isakeyword{and}\ {\isachardoublequoteopen}{\isasymAnd}i{\isachardot}{\kern0pt}\ simple{\isacharunderscore}{\kern0pt}function\ M\ {\isacharparenleft}{\kern0pt}s\ i{\isacharparenright}{\kern0pt}{\isachardoublequoteclose}\isanewline
\ \ \ \ \ \ \isakeyword{and}\ {\isachardoublequoteopen}{\isasymAnd}i{\isachardot}{\kern0pt}\ emeasure\ M\ {\isacharbraceleft}{\kern0pt}y\ {\isasymin}\ space\ M{\isachardot}{\kern0pt}\ s\ i\ y\ {\isasymnoteq}\ {\isadigit{0}}{\isacharbraceright}{\kern0pt}\ {\isasymnoteq}\ {\isasyminfinity}{\isachardoublequoteclose}\isanewline
\ \ \ \ \ \ \isakeyword{and}\ {\isachardoublequoteopen}{\isasymAnd}x{\isachardot}{\kern0pt}\ x\ {\isasymin}\ space\ M\ {\isasymLongrightarrow}\ {\isacharparenleft}{\kern0pt}{\isasymlambda}i{\isachardot}{\kern0pt}\ s\ i\ x{\isacharparenright}{\kern0pt}\ {\isasymlonglonglongrightarrow}\ f\ x{\isachardoublequoteclose}\isanewline
\ \ \ \ \ \ \isakeyword{and}\ {\isachardoublequoteopen}{\isasymAnd}x\ i{\isachardot}{\kern0pt}\ x\ {\isasymin}\ space\ M\ {\isasymLongrightarrow}\ norm\ {\isacharparenleft}{\kern0pt}s\ i\ x{\isacharparenright}{\kern0pt}\ {\isasymle}\ {\isadigit{2}}\ {\isacharasterisk}{\kern0pt}\ norm\ {\isacharparenleft}{\kern0pt}f\ x{\isacharparenright}{\kern0pt}{\isachardoublequoteclose}\isanewline
\ \ \isakeyword{obtains}\ r\ \isanewline
\ \ \isakeyword{where}\ {\isachardoublequoteopen}has{\isacharunderscore}{\kern0pt}cond{\isacharunderscore}{\kern0pt}exp\ M\ F\ f\ {\isacharparenleft}{\kern0pt}{\isasymlambda}x{\isachardot}{\kern0pt}\ lim\ {\isacharparenleft}{\kern0pt}{\isasymlambda}i{\isachardot}{\kern0pt}\ cond{\isacharunderscore}{\kern0pt}exp\ M\ F\ {\isacharparenleft}{\kern0pt}s\ {\isacharparenleft}{\kern0pt}r\ i{\isacharparenright}{\kern0pt}{\isacharparenright}{\kern0pt}\ x{\isacharparenright}{\kern0pt}{\isacharparenright}{\kern0pt}{\isachardoublequoteclose}\ \isanewline
\ \ \ \ \ \ \ \ {\isachardoublequoteopen}AE\ x\ in\ M{\isachardot}{\kern0pt}\ convergent\ {\isacharparenleft}{\kern0pt}{\isasymlambda}i{\isachardot}{\kern0pt}\ cond{\isacharunderscore}{\kern0pt}exp\ M\ F\ {\isacharparenleft}{\kern0pt}s\ {\isacharparenleft}{\kern0pt}r\ i{\isacharparenright}{\kern0pt}{\isacharparenright}{\kern0pt}\ x{\isacharparenright}{\kern0pt}{\isachardoublequoteclose}\isanewline
\ \ \ \ \ \ \ \ {\isachardoublequoteopen}strict{\isacharunderscore}{\kern0pt}mono\ r{\isachardoublequoteclose}\isanewline
%
\isadelimproof
%
\endisadelimproof
%
\isatagproof
\isacommand{proof}\isamarkupfalse%
\ {\isacharminus}{\kern0pt}\isanewline
\ \ \isacommand{have}\isamarkupfalse%
\ {\isacharbrackleft}{\kern0pt}measurable{\isacharbrackright}{\kern0pt}{\isacharcolon}{\kern0pt}\ {\isachardoublequoteopen}{\isacharparenleft}{\kern0pt}s\ i{\isacharparenright}{\kern0pt}\ {\isasymin}\ borel{\isacharunderscore}{\kern0pt}measurable\ M{\isachardoublequoteclose}\ \isakeyword{for}\ i\ \isacommand{using}\isamarkupfalse%
\ assms{\isacharparenleft}{\kern0pt}{\isadigit{2}}{\isacharparenright}{\kern0pt}\ \isacommand{by}\isamarkupfalse%
\ {\isacharparenleft}{\kern0pt}simp\ add{\isacharcolon}{\kern0pt}\ borel{\isacharunderscore}{\kern0pt}measurable{\isacharunderscore}{\kern0pt}simple{\isacharunderscore}{\kern0pt}function{\isacharparenright}{\kern0pt}\isanewline
\ \ \isacommand{have}\isamarkupfalse%
\ integrable{\isacharunderscore}{\kern0pt}s{\isacharcolon}{\kern0pt}\ {\isachardoublequoteopen}integrable\ M\ {\isacharparenleft}{\kern0pt}{\isasymlambda}x{\isachardot}{\kern0pt}\ s\ i\ x{\isacharparenright}{\kern0pt}{\isachardoublequoteclose}\ \isakeyword{for}\ i\ \isacommand{using}\isamarkupfalse%
\ assms{\isacharparenleft}{\kern0pt}{\isadigit{2}}{\isacharparenright}{\kern0pt}\ assms{\isacharparenleft}{\kern0pt}{\isadigit{3}}{\isacharparenright}{\kern0pt}\ integrable{\isacharunderscore}{\kern0pt}simple{\isacharunderscore}{\kern0pt}function\ \isacommand{by}\isamarkupfalse%
\ blast\isanewline
\ \ \isacommand{have}\isamarkupfalse%
\ integrable{\isacharunderscore}{\kern0pt}{\isadigit{4}}f{\isacharcolon}{\kern0pt}\ {\isachardoublequoteopen}integrable\ M\ {\isacharparenleft}{\kern0pt}{\isasymlambda}x{\isachardot}{\kern0pt}\ {\isadigit{4}}\ {\isacharasterisk}{\kern0pt}\ norm\ {\isacharparenleft}{\kern0pt}f\ x{\isacharparenright}{\kern0pt}{\isacharparenright}{\kern0pt}{\isachardoublequoteclose}\ \isacommand{using}\isamarkupfalse%
\ assms{\isacharparenleft}{\kern0pt}{\isadigit{1}}{\isacharparenright}{\kern0pt}\ \isacommand{by}\isamarkupfalse%
\ simp\isanewline
\ \ \isacommand{have}\isamarkupfalse%
\ integrable{\isacharunderscore}{\kern0pt}{\isadigit{2}}f{\isacharcolon}{\kern0pt}\ {\isachardoublequoteopen}integrable\ M\ {\isacharparenleft}{\kern0pt}{\isasymlambda}x{\isachardot}{\kern0pt}\ {\isadigit{2}}\ {\isacharasterisk}{\kern0pt}\ norm\ {\isacharparenleft}{\kern0pt}f\ x{\isacharparenright}{\kern0pt}{\isacharparenright}{\kern0pt}{\isachardoublequoteclose}\ \isacommand{using}\isamarkupfalse%
\ assms{\isacharparenleft}{\kern0pt}{\isadigit{1}}{\isacharparenright}{\kern0pt}\ \isacommand{by}\isamarkupfalse%
\ simp\isanewline
\ \ \isacommand{have}\isamarkupfalse%
\ integrable{\isacharunderscore}{\kern0pt}{\isadigit{2}}{\isacharunderscore}{\kern0pt}cond{\isacharunderscore}{\kern0pt}exp{\isacharunderscore}{\kern0pt}norm{\isacharunderscore}{\kern0pt}f{\isacharcolon}{\kern0pt}\ {\isachardoublequoteopen}integrable\ M\ {\isacharparenleft}{\kern0pt}{\isasymlambda}x{\isachardot}{\kern0pt}\ {\isadigit{2}}\ {\isacharasterisk}{\kern0pt}\ cond{\isacharunderscore}{\kern0pt}exp\ M\ F\ {\isacharparenleft}{\kern0pt}{\isasymlambda}x{\isachardot}{\kern0pt}\ norm\ {\isacharparenleft}{\kern0pt}f\ x{\isacharparenright}{\kern0pt}{\isacharparenright}{\kern0pt}\ x{\isacharparenright}{\kern0pt}{\isachardoublequoteclose}\ \isacommand{by}\isamarkupfalse%
\ fast\isanewline
\isanewline
\ \ \isacommand{have}\isamarkupfalse%
\ {\isachardoublequoteopen}emeasure\ M\ {\isacharbraceleft}{\kern0pt}y\ {\isasymin}\ space\ M{\isachardot}{\kern0pt}\ s\ i\ y\ {\isacharminus}{\kern0pt}\ s\ j\ y\ {\isasymnoteq}\ {\isadigit{0}}{\isacharbraceright}{\kern0pt}\ {\isasymle}\ \ emeasure\ M\ {\isacharbraceleft}{\kern0pt}y\ {\isasymin}\ space\ M{\isachardot}{\kern0pt}\ s\ i\ y\ {\isasymnoteq}\ {\isadigit{0}}{\isacharbraceright}{\kern0pt}\ {\isacharplus}{\kern0pt}\ emeasure\ M\ {\isacharbraceleft}{\kern0pt}y\ {\isasymin}\ space\ M{\isachardot}{\kern0pt}\ s\ j\ y\ {\isasymnoteq}\ {\isadigit{0}}{\isacharbraceright}{\kern0pt}{\isachardoublequoteclose}\ \isakeyword{for}\ i\ j\ \isacommand{using}\isamarkupfalse%
\ simple{\isacharunderscore}{\kern0pt}functionD{\isacharparenleft}{\kern0pt}{\isadigit{2}}{\isacharparenright}{\kern0pt}{\isacharbrackleft}{\kern0pt}OF\ assms{\isacharparenleft}{\kern0pt}{\isadigit{2}}{\isacharparenright}{\kern0pt}{\isacharbrackright}{\kern0pt}\ \isacommand{by}\isamarkupfalse%
\ {\isacharparenleft}{\kern0pt}intro\ order{\isacharunderscore}{\kern0pt}trans{\isacharbrackleft}{\kern0pt}OF\ emeasure{\isacharunderscore}{\kern0pt}mono\ emeasure{\isacharunderscore}{\kern0pt}subadditive{\isacharbrackright}{\kern0pt}{\isacharcomma}{\kern0pt}\ auto{\isacharparenright}{\kern0pt}\isanewline
\ \ \isacommand{hence}\isamarkupfalse%
\ fin{\isacharunderscore}{\kern0pt}sup{\isacharcolon}{\kern0pt}\ {\isachardoublequoteopen}emeasure\ M\ {\isacharbraceleft}{\kern0pt}y\ {\isasymin}\ space\ M{\isachardot}{\kern0pt}\ s\ i\ y\ {\isacharminus}{\kern0pt}\ s\ j\ y\ {\isasymnoteq}\ {\isadigit{0}}{\isacharbraceright}{\kern0pt}\ {\isasymnoteq}\ {\isasyminfinity}{\isachardoublequoteclose}\ \isakeyword{for}\ i\ j\ \isacommand{using}\isamarkupfalse%
\ assms{\isacharparenleft}{\kern0pt}{\isadigit{3}}{\isacharparenright}{\kern0pt}\ \isacommand{by}\isamarkupfalse%
\ {\isacharparenleft}{\kern0pt}metis\ {\isacharparenleft}{\kern0pt}mono{\isacharunderscore}{\kern0pt}tags{\isacharparenright}{\kern0pt}\ ennreal{\isacharunderscore}{\kern0pt}add{\isacharunderscore}{\kern0pt}eq{\isacharunderscore}{\kern0pt}top\ linorder{\isacharunderscore}{\kern0pt}not{\isacharunderscore}{\kern0pt}less\ top{\isachardot}{\kern0pt}not{\isacharunderscore}{\kern0pt}eq{\isacharunderscore}{\kern0pt}extremum\ infinity{\isacharunderscore}{\kern0pt}ennreal{\isacharunderscore}{\kern0pt}def{\isacharparenright}{\kern0pt}\isanewline
\isanewline
\ \ \isacommand{have}\isamarkupfalse%
\ {\isachardoublequoteopen}emeasure\ M\ {\isacharbraceleft}{\kern0pt}y\ {\isasymin}\ space\ M{\isachardot}{\kern0pt}\ norm\ {\isacharparenleft}{\kern0pt}s\ i\ y\ {\isacharminus}{\kern0pt}\ s\ j\ y{\isacharparenright}{\kern0pt}\ {\isasymnoteq}\ {\isadigit{0}}{\isacharbraceright}{\kern0pt}\ {\isasymle}\ \ emeasure\ M\ {\isacharbraceleft}{\kern0pt}y\ {\isasymin}\ space\ M{\isachardot}{\kern0pt}\ s\ i\ y\ {\isasymnoteq}\ {\isadigit{0}}{\isacharbraceright}{\kern0pt}\ {\isacharplus}{\kern0pt}\ emeasure\ M\ {\isacharbraceleft}{\kern0pt}y\ {\isasymin}\ space\ M{\isachardot}{\kern0pt}\ s\ j\ y\ {\isasymnoteq}\ {\isadigit{0}}{\isacharbraceright}{\kern0pt}{\isachardoublequoteclose}\ \isakeyword{for}\ i\ j\ \isacommand{using}\isamarkupfalse%
\ simple{\isacharunderscore}{\kern0pt}functionD{\isacharparenleft}{\kern0pt}{\isadigit{2}}{\isacharparenright}{\kern0pt}{\isacharbrackleft}{\kern0pt}OF\ assms{\isacharparenleft}{\kern0pt}{\isadigit{2}}{\isacharparenright}{\kern0pt}{\isacharbrackright}{\kern0pt}\ \isacommand{by}\isamarkupfalse%
\ {\isacharparenleft}{\kern0pt}intro\ order{\isacharunderscore}{\kern0pt}trans{\isacharbrackleft}{\kern0pt}OF\ emeasure{\isacharunderscore}{\kern0pt}mono\ emeasure{\isacharunderscore}{\kern0pt}subadditive{\isacharbrackright}{\kern0pt}{\isacharcomma}{\kern0pt}\ auto{\isacharparenright}{\kern0pt}\isanewline
\ \ \isacommand{hence}\isamarkupfalse%
\ fin{\isacharunderscore}{\kern0pt}sup{\isacharunderscore}{\kern0pt}norm{\isacharcolon}{\kern0pt}\ {\isachardoublequoteopen}emeasure\ M\ {\isacharbraceleft}{\kern0pt}y\ {\isasymin}\ space\ M{\isachardot}{\kern0pt}\ norm\ {\isacharparenleft}{\kern0pt}s\ i\ y\ {\isacharminus}{\kern0pt}\ s\ j\ y{\isacharparenright}{\kern0pt}\ {\isasymnoteq}\ {\isadigit{0}}{\isacharbraceright}{\kern0pt}\ {\isasymnoteq}\ {\isasyminfinity}{\isachardoublequoteclose}\ \isakeyword{for}\ i\ j\ \isacommand{using}\isamarkupfalse%
\ assms{\isacharparenleft}{\kern0pt}{\isadigit{3}}{\isacharparenright}{\kern0pt}\ \isacommand{by}\isamarkupfalse%
\ {\isacharparenleft}{\kern0pt}metis\ {\isacharparenleft}{\kern0pt}mono{\isacharunderscore}{\kern0pt}tags{\isacharparenright}{\kern0pt}\ ennreal{\isacharunderscore}{\kern0pt}add{\isacharunderscore}{\kern0pt}eq{\isacharunderscore}{\kern0pt}top\ linorder{\isacharunderscore}{\kern0pt}not{\isacharunderscore}{\kern0pt}less\ top{\isachardot}{\kern0pt}not{\isacharunderscore}{\kern0pt}eq{\isacharunderscore}{\kern0pt}extremum\ infinity{\isacharunderscore}{\kern0pt}ennreal{\isacharunderscore}{\kern0pt}def{\isacharparenright}{\kern0pt}\isanewline
\isanewline
\ \ \isacommand{have}\isamarkupfalse%
\ Cauchy{\isacharcolon}{\kern0pt}\ {\isachardoublequoteopen}Cauchy\ {\isacharparenleft}{\kern0pt}{\isasymlambda}n{\isachardot}{\kern0pt}\ s\ n\ x{\isacharparenright}{\kern0pt}{\isachardoublequoteclose}\ \isakeyword{if}\ {\isachardoublequoteopen}x\ {\isasymin}\ space\ M{\isachardoublequoteclose}\ \isakeyword{for}\ x\ \isacommand{using}\isamarkupfalse%
\ assms{\isacharparenleft}{\kern0pt}{\isadigit{4}}{\isacharparenright}{\kern0pt}\ LIMSEQ{\isacharunderscore}{\kern0pt}imp{\isacharunderscore}{\kern0pt}Cauchy\ that\ \isacommand{by}\isamarkupfalse%
\ blast\isanewline
\ \ \isacommand{hence}\isamarkupfalse%
\ bounded{\isacharunderscore}{\kern0pt}range{\isacharunderscore}{\kern0pt}s{\isacharcolon}{\kern0pt}\ {\isachardoublequoteopen}bounded\ {\isacharparenleft}{\kern0pt}range\ {\isacharparenleft}{\kern0pt}{\isasymlambda}n{\isachardot}{\kern0pt}\ s\ n\ x{\isacharparenright}{\kern0pt}{\isacharparenright}{\kern0pt}{\isachardoublequoteclose}\ \isakeyword{if}\ {\isachardoublequoteopen}x\ {\isasymin}\ space\ M{\isachardoublequoteclose}\ \isakeyword{for}\ x\ \isacommand{using}\isamarkupfalse%
\ that\ cauchy{\isacharunderscore}{\kern0pt}imp{\isacharunderscore}{\kern0pt}bounded\ \isacommand{by}\isamarkupfalse%
\ fast\isanewline
\isanewline
\ \ \isacommand{have}\isamarkupfalse%
\ {\isachardoublequoteopen}AE\ x\ in\ M{\isachardot}{\kern0pt}\ {\isacharparenleft}{\kern0pt}{\isasymlambda}n{\isachardot}{\kern0pt}\ diameter\ {\isacharbraceleft}{\kern0pt}s\ i\ x\ {\isacharbar}{\kern0pt}\ i{\isachardot}{\kern0pt}\ n\ {\isasymle}\ i{\isacharbraceright}{\kern0pt}{\isacharparenright}{\kern0pt}\ {\isasymlonglonglongrightarrow}\ {\isadigit{0}}{\isachardoublequoteclose}\ \isacommand{using}\isamarkupfalse%
\ Cauchy\ cauchy{\isacharunderscore}{\kern0pt}iff{\isacharunderscore}{\kern0pt}diameter{\isacharunderscore}{\kern0pt}tends{\isacharunderscore}{\kern0pt}to{\isacharunderscore}{\kern0pt}zero{\isacharunderscore}{\kern0pt}and{\isacharunderscore}{\kern0pt}bounded\ \isacommand{by}\isamarkupfalse%
\ fast\isanewline
\ \ \isacommand{moreover}\isamarkupfalse%
\ \isacommand{have}\isamarkupfalse%
\ {\isachardoublequoteopen}{\isacharparenleft}{\kern0pt}{\isasymlambda}x{\isachardot}{\kern0pt}\ diameter\ {\isacharbraceleft}{\kern0pt}s\ i\ x\ {\isacharbar}{\kern0pt}i{\isachardot}{\kern0pt}\ n\ {\isasymle}\ i{\isacharbraceright}{\kern0pt}{\isacharparenright}{\kern0pt}\ {\isasymin}\ borel{\isacharunderscore}{\kern0pt}measurable\ M{\isachardoublequoteclose}\ \isakeyword{for}\ n\ \isacommand{using}\isamarkupfalse%
\ bounded{\isacharunderscore}{\kern0pt}range{\isacharunderscore}{\kern0pt}s\ borel{\isacharunderscore}{\kern0pt}measurable{\isacharunderscore}{\kern0pt}diameter\ \isacommand{by}\isamarkupfalse%
\ measurable\isanewline
\ \ \isacommand{moreover}\isamarkupfalse%
\ \isacommand{have}\isamarkupfalse%
\ {\isachardoublequoteopen}AE\ x\ in\ M{\isachardot}{\kern0pt}\ norm\ {\isacharparenleft}{\kern0pt}diameter\ {\isacharbraceleft}{\kern0pt}s\ i\ x\ {\isacharbar}{\kern0pt}i{\isachardot}{\kern0pt}\ n\ {\isasymle}\ i{\isacharbraceright}{\kern0pt}{\isacharparenright}{\kern0pt}\ {\isasymle}\ {\isadigit{4}}\ {\isacharasterisk}{\kern0pt}\ norm\ {\isacharparenleft}{\kern0pt}f\ x{\isacharparenright}{\kern0pt}{\isachardoublequoteclose}\ \isakeyword{for}\ n\isanewline
\ \ \isacommand{proof}\isamarkupfalse%
\ {\isacharminus}{\kern0pt}\ \isanewline
\ \ \ \ \isacommand{{\isacharbraceleft}{\kern0pt}}\isamarkupfalse%
\isanewline
\ \ \ \ \ \ \isacommand{fix}\isamarkupfalse%
\ x\ \isacommand{assume}\isamarkupfalse%
\ x{\isacharcolon}{\kern0pt}\ {\isachardoublequoteopen}x\ {\isasymin}\ space\ M{\isachardoublequoteclose}\isanewline
\ \ \ \ \ \ \isacommand{have}\isamarkupfalse%
\ {\isachardoublequoteopen}diameter\ {\isacharbraceleft}{\kern0pt}s\ i\ x\ {\isacharbar}{\kern0pt}i{\isachardot}{\kern0pt}\ n\ {\isasymle}\ i{\isacharbraceright}{\kern0pt}\ {\isasymle}\ {\isadigit{2}}\ {\isacharasterisk}{\kern0pt}\ norm\ {\isacharparenleft}{\kern0pt}f\ x{\isacharparenright}{\kern0pt}\ {\isacharplus}{\kern0pt}\ {\isadigit{2}}\ {\isacharasterisk}{\kern0pt}\ norm\ {\isacharparenleft}{\kern0pt}f\ x{\isacharparenright}{\kern0pt}{\isachardoublequoteclose}\ \isacommand{by}\isamarkupfalse%
\ {\isacharparenleft}{\kern0pt}intro\ diameter{\isacharunderscore}{\kern0pt}le{\isacharcomma}{\kern0pt}\ blast{\isacharcomma}{\kern0pt}\ subst\ dist{\isacharunderscore}{\kern0pt}norm{\isacharbrackleft}{\kern0pt}symmetric{\isacharbrackright}{\kern0pt}{\isacharcomma}{\kern0pt}\ intro\ dist{\isacharunderscore}{\kern0pt}triangle{\isadigit{3}}{\isacharbrackleft}{\kern0pt}THEN\ order{\isacharunderscore}{\kern0pt}trans{\isacharcomma}{\kern0pt}\ of\ {\isadigit{0}}{\isacharbrackright}{\kern0pt}{\isacharcomma}{\kern0pt}\ intro\ add{\isacharunderscore}{\kern0pt}mono{\isacharparenright}{\kern0pt}\ {\isacharparenleft}{\kern0pt}auto\ intro{\isacharcolon}{\kern0pt}\ assms{\isacharparenleft}{\kern0pt}{\isadigit{5}}{\isacharparenright}{\kern0pt}{\isacharbrackleft}{\kern0pt}OF\ x{\isacharbrackright}{\kern0pt}{\isacharparenright}{\kern0pt}\isanewline
\ \ \ \ \ \ \isacommand{hence}\isamarkupfalse%
\ {\isachardoublequoteopen}norm\ {\isacharparenleft}{\kern0pt}diameter\ {\isacharbraceleft}{\kern0pt}s\ i\ x\ {\isacharbar}{\kern0pt}i{\isachardot}{\kern0pt}\ n\ {\isasymle}\ i{\isacharbraceright}{\kern0pt}{\isacharparenright}{\kern0pt}\ {\isasymle}\ {\isadigit{4}}\ {\isacharasterisk}{\kern0pt}\ norm\ {\isacharparenleft}{\kern0pt}f\ x{\isacharparenright}{\kern0pt}{\isachardoublequoteclose}\ \isacommand{using}\isamarkupfalse%
\ diameter{\isacharunderscore}{\kern0pt}ge{\isacharunderscore}{\kern0pt}{\isadigit{0}}{\isacharbrackleft}{\kern0pt}OF\ bounded{\isacharunderscore}{\kern0pt}subset{\isacharbrackleft}{\kern0pt}OF\ bounded{\isacharunderscore}{\kern0pt}range{\isacharunderscore}{\kern0pt}s{\isacharbrackright}{\kern0pt}{\isacharcomma}{\kern0pt}\ OF\ x{\isacharcomma}{\kern0pt}\ of\ {\isachardoublequoteopen}{\isacharbraceleft}{\kern0pt}s\ i\ x\ {\isacharbar}{\kern0pt}i{\isachardot}{\kern0pt}\ n\ {\isasymle}\ i{\isacharbraceright}{\kern0pt}{\isachardoublequoteclose}{\isacharbrackright}{\kern0pt}\ \isacommand{by}\isamarkupfalse%
\ force\isanewline
\ \ \ \ \isacommand{{\isacharbraceright}{\kern0pt}}\isamarkupfalse%
\isanewline
\ \ \ \ \isacommand{thus}\isamarkupfalse%
\ {\isacharquery}{\kern0pt}thesis\ \isacommand{by}\isamarkupfalse%
\ fast\isanewline
\ \ \isacommand{qed}\isamarkupfalse%
\isanewline
\ \ \isacommand{ultimately}\isamarkupfalse%
\ \isacommand{have}\isamarkupfalse%
\ diameter{\isacharunderscore}{\kern0pt}tendsto{\isacharunderscore}{\kern0pt}zero{\isacharcolon}{\kern0pt}\ {\isachardoublequoteopen}{\isacharparenleft}{\kern0pt}{\isasymlambda}n{\isachardot}{\kern0pt}\ LINT\ x{\isacharbar}{\kern0pt}M{\isachardot}{\kern0pt}\ diameter\ {\isacharbraceleft}{\kern0pt}s\ i\ x\ {\isacharbar}{\kern0pt}\ i{\isachardot}{\kern0pt}\ n\ {\isasymle}\ i{\isacharbraceright}{\kern0pt}{\isacharparenright}{\kern0pt}\ {\isasymlonglonglongrightarrow}\ {\isadigit{0}}{\isachardoublequoteclose}\ \isacommand{by}\isamarkupfalse%
\ {\isacharparenleft}{\kern0pt}intro\ integral{\isacharunderscore}{\kern0pt}dominated{\isacharunderscore}{\kern0pt}convergence{\isacharbrackleft}{\kern0pt}OF\ borel{\isacharunderscore}{\kern0pt}measurable{\isacharunderscore}{\kern0pt}const{\isacharbrackleft}{\kern0pt}of\ {\isadigit{0}}{\isacharbrackright}{\kern0pt}\ {\isacharunderscore}{\kern0pt}\ integrable{\isacharunderscore}{\kern0pt}{\isadigit{4}}f{\isacharcomma}{\kern0pt}\ simplified{\isacharbrackright}{\kern0pt}{\isacharparenright}{\kern0pt}\ {\isacharparenleft}{\kern0pt}fast{\isacharplus}{\kern0pt}{\isacharparenright}{\kern0pt}\isanewline
\ \ \isanewline
\ \ \isacommand{have}\isamarkupfalse%
\ diameter{\isacharunderscore}{\kern0pt}integrable{\isacharcolon}{\kern0pt}\ {\isachardoublequoteopen}integrable\ M\ {\isacharparenleft}{\kern0pt}{\isasymlambda}x{\isachardot}{\kern0pt}\ diameter\ {\isacharbraceleft}{\kern0pt}s\ i\ x\ {\isacharbar}{\kern0pt}\ i{\isachardot}{\kern0pt}\ n\ {\isasymle}\ i{\isacharbraceright}{\kern0pt}{\isacharparenright}{\kern0pt}{\isachardoublequoteclose}\ \isakeyword{for}\ n\ \isacommand{using}\isamarkupfalse%
\ assms{\isacharparenleft}{\kern0pt}{\isadigit{1}}{\isacharcomma}{\kern0pt}{\isadigit{5}}{\isacharparenright}{\kern0pt}\ \isacommand{by}\isamarkupfalse%
\ {\isacharparenleft}{\kern0pt}intro\ integrable{\isacharunderscore}{\kern0pt}bound{\isacharunderscore}{\kern0pt}diameter{\isacharbrackleft}{\kern0pt}OF\ bounded{\isacharunderscore}{\kern0pt}range{\isacharunderscore}{\kern0pt}s\ integrable{\isacharunderscore}{\kern0pt}{\isadigit{2}}f{\isacharbrackright}{\kern0pt}{\isacharcomma}{\kern0pt}\ auto{\isacharparenright}{\kern0pt}\isanewline
\isanewline
\ \ \isacommand{have}\isamarkupfalse%
\ dist{\isacharunderscore}{\kern0pt}integrable{\isacharcolon}{\kern0pt}\ {\isachardoublequoteopen}integrable\ M\ {\isacharparenleft}{\kern0pt}{\isasymlambda}x{\isachardot}{\kern0pt}\ dist\ {\isacharparenleft}{\kern0pt}s\ i\ x{\isacharparenright}{\kern0pt}\ {\isacharparenleft}{\kern0pt}s\ j\ x{\isacharparenright}{\kern0pt}{\isacharparenright}{\kern0pt}{\isachardoublequoteclose}\ \isakeyword{for}\ i\ j\ \isanewline
\ \ \ \ \isacommand{using}\isamarkupfalse%
\ assms{\isacharparenleft}{\kern0pt}{\isadigit{5}}{\isacharparenright}{\kern0pt}\ dist{\isacharunderscore}{\kern0pt}triangle{\isadigit{3}}{\isacharbrackleft}{\kern0pt}of\ {\isachardoublequoteopen}s\ i\ {\isacharunderscore}{\kern0pt}{\isachardoublequoteclose}\ {\isacharunderscore}{\kern0pt}\ {\isadigit{0}}{\isacharcomma}{\kern0pt}\ THEN\ order{\isacharunderscore}{\kern0pt}trans{\isacharcomma}{\kern0pt}\ OF\ add{\isacharunderscore}{\kern0pt}mono{\isacharcomma}{\kern0pt}\ of\ {\isacharunderscore}{\kern0pt}\ {\isachardoublequoteopen}{\isadigit{2}}\ {\isacharasterisk}{\kern0pt}\ norm\ {\isacharparenleft}{\kern0pt}f\ {\isacharunderscore}{\kern0pt}{\isacharparenright}{\kern0pt}{\isachardoublequoteclose}{\isacharbrackright}{\kern0pt}\isanewline
\ \ \ \ \isacommand{by}\isamarkupfalse%
\ {\isacharparenleft}{\kern0pt}intro\ Bochner{\isacharunderscore}{\kern0pt}Integration{\isachardot}{\kern0pt}integrable{\isacharunderscore}{\kern0pt}bound{\isacharbrackleft}{\kern0pt}OF\ integrable{\isacharunderscore}{\kern0pt}{\isadigit{4}}f{\isacharbrackright}{\kern0pt}{\isacharparenright}{\kern0pt}\ fastforce{\isacharplus}{\kern0pt}\isanewline
\ \ \ \isanewline
\ \ \isacommand{hence}\isamarkupfalse%
\ dist{\isacharunderscore}{\kern0pt}norm{\isacharunderscore}{\kern0pt}integrable{\isacharcolon}{\kern0pt}\ {\isachardoublequoteopen}integrable\ M\ {\isacharparenleft}{\kern0pt}{\isasymlambda}x{\isachardot}{\kern0pt}\ norm\ {\isacharparenleft}{\kern0pt}s\ i\ x\ {\isacharminus}{\kern0pt}\ s\ j\ x{\isacharparenright}{\kern0pt}{\isacharparenright}{\kern0pt}{\isachardoublequoteclose}\ \isakeyword{for}\ i\ j\ \isacommand{unfolding}\isamarkupfalse%
\ dist{\isacharunderscore}{\kern0pt}norm\ \isacommand{by}\isamarkupfalse%
\ presburger\isanewline
\isanewline
\ \ \isacommand{have}\isamarkupfalse%
\ {\isachardoublequoteopen}{\isasymexists}N{\isachardot}{\kern0pt}\ {\isasymforall}i{\isasymge}N{\isachardot}{\kern0pt}\ {\isasymforall}j{\isasymge}N{\isachardot}{\kern0pt}\ LINT\ x{\isacharbar}{\kern0pt}M{\isachardot}{\kern0pt}\ norm\ {\isacharparenleft}{\kern0pt}cond{\isacharunderscore}{\kern0pt}exp\ M\ F\ {\isacharparenleft}{\kern0pt}s\ i{\isacharparenright}{\kern0pt}\ x\ {\isacharminus}{\kern0pt}\ cond{\isacharunderscore}{\kern0pt}exp\ M\ F\ {\isacharparenleft}{\kern0pt}s\ j{\isacharparenright}{\kern0pt}\ x{\isacharparenright}{\kern0pt}\ {\isacharless}{\kern0pt}\ e{\isachardoublequoteclose}\ \isakeyword{if}\ e{\isacharunderscore}{\kern0pt}pos{\isacharcolon}{\kern0pt}\ {\isachardoublequoteopen}e\ {\isachargreater}{\kern0pt}\ {\isadigit{0}}{\isachardoublequoteclose}\ \isakeyword{for}\ e\isanewline
\ \ \isacommand{proof}\isamarkupfalse%
\ {\isacharminus}{\kern0pt}\isanewline
\ \ \ \ \isacommand{obtain}\isamarkupfalse%
\ N\ \isakeyword{where}\ {\isacharasterisk}{\kern0pt}{\isacharcolon}{\kern0pt}\ {\isachardoublequoteopen}LINT\ x{\isacharbar}{\kern0pt}M{\isachardot}{\kern0pt}\ diameter\ {\isacharbraceleft}{\kern0pt}s\ i\ x\ {\isacharbar}{\kern0pt}\ i{\isachardot}{\kern0pt}\ n\ {\isasymle}\ i{\isacharbraceright}{\kern0pt}\ {\isacharless}{\kern0pt}\ e{\isachardoublequoteclose}\ \isakeyword{if}\ {\isachardoublequoteopen}n\ {\isasymge}\ N{\isachardoublequoteclose}\ \isakeyword{for}\ n\ \isacommand{using}\isamarkupfalse%
\ that\ order{\isacharunderscore}{\kern0pt}tendsto{\isacharunderscore}{\kern0pt}iff{\isacharbrackleft}{\kern0pt}THEN\ iffD{\isadigit{1}}{\isacharcomma}{\kern0pt}\ OF\ diameter{\isacharunderscore}{\kern0pt}tendsto{\isacharunderscore}{\kern0pt}zero{\isacharcomma}{\kern0pt}\ unfolded\ eventually{\isacharunderscore}{\kern0pt}sequentially{\isacharbrackright}{\kern0pt}\ e{\isacharunderscore}{\kern0pt}pos\ \isacommand{by}\isamarkupfalse%
\ presburger\isanewline
\ \ \ \ \isacommand{{\isacharbraceleft}{\kern0pt}}\isamarkupfalse%
\isanewline
\ \ \ \ \ \ \isacommand{fix}\isamarkupfalse%
\ i\ j\ x\ \isacommand{assume}\isamarkupfalse%
\ asm{\isacharcolon}{\kern0pt}\ {\isachardoublequoteopen}i\ {\isasymge}\ N{\isachardoublequoteclose}\ {\isachardoublequoteopen}j\ {\isasymge}\ N{\isachardoublequoteclose}\ {\isachardoublequoteopen}x\ {\isasymin}\ space\ M{\isachardoublequoteclose}\isanewline
\ \ \ \ \ \ \isacommand{have}\isamarkupfalse%
\ {\isachardoublequoteopen}case{\isacharunderscore}{\kern0pt}prod\ dist\ {\isacharbackquote}{\kern0pt}\ {\isacharparenleft}{\kern0pt}{\isacharbraceleft}{\kern0pt}s\ i\ x\ {\isacharbar}{\kern0pt}i{\isachardot}{\kern0pt}\ N\ {\isasymle}\ i{\isacharbraceright}{\kern0pt}\ {\isasymtimes}\ {\isacharbraceleft}{\kern0pt}s\ i\ x\ {\isacharbar}{\kern0pt}i{\isachardot}{\kern0pt}\ N\ {\isasymle}\ i{\isacharbraceright}{\kern0pt}{\isacharparenright}{\kern0pt}\ {\isacharequal}{\kern0pt}\ case{\isacharunderscore}{\kern0pt}prod\ {\isacharparenleft}{\kern0pt}{\isasymlambda}i\ j{\isachardot}{\kern0pt}\ dist\ {\isacharparenleft}{\kern0pt}s\ i\ x{\isacharparenright}{\kern0pt}\ {\isacharparenleft}{\kern0pt}s\ j\ x{\isacharparenright}{\kern0pt}{\isacharparenright}{\kern0pt}\ {\isacharbackquote}{\kern0pt}\ {\isacharparenleft}{\kern0pt}{\isacharbraceleft}{\kern0pt}N{\isachardot}{\kern0pt}{\isachardot}{\kern0pt}{\isacharbraceright}{\kern0pt}\ {\isasymtimes}\ {\isacharbraceleft}{\kern0pt}N{\isachardot}{\kern0pt}{\isachardot}{\kern0pt}{\isacharbraceright}{\kern0pt}{\isacharparenright}{\kern0pt}{\isachardoublequoteclose}\ \isacommand{by}\isamarkupfalse%
\ fast\isanewline
\ \ \ \ \ \ \isacommand{hence}\isamarkupfalse%
\ {\isachardoublequoteopen}diameter\ {\isacharbraceleft}{\kern0pt}s\ i\ x\ {\isacharbar}{\kern0pt}\ i{\isachardot}{\kern0pt}\ N\ {\isasymle}\ i{\isacharbraceright}{\kern0pt}\ {\isacharequal}{\kern0pt}\ {\isacharparenleft}{\kern0pt}SUP\ {\isacharparenleft}{\kern0pt}i{\isacharcomma}{\kern0pt}\ j{\isacharparenright}{\kern0pt}\ {\isasymin}\ {\isacharbraceleft}{\kern0pt}N{\isachardot}{\kern0pt}{\isachardot}{\kern0pt}{\isacharbraceright}{\kern0pt}\ {\isasymtimes}\ {\isacharbraceleft}{\kern0pt}N{\isachardot}{\kern0pt}{\isachardot}{\kern0pt}{\isacharbraceright}{\kern0pt}{\isachardot}{\kern0pt}\ dist\ {\isacharparenleft}{\kern0pt}s\ i\ x{\isacharparenright}{\kern0pt}\ {\isacharparenleft}{\kern0pt}s\ j\ x{\isacharparenright}{\kern0pt}{\isacharparenright}{\kern0pt}{\isachardoublequoteclose}\ \isacommand{unfolding}\isamarkupfalse%
\ diameter{\isacharunderscore}{\kern0pt}def\ \isacommand{by}\isamarkupfalse%
\ auto\isanewline
\ \ \ \ \ \ \isacommand{moreover}\isamarkupfalse%
\ \isacommand{have}\isamarkupfalse%
\ {\isachardoublequoteopen}{\isacharparenleft}{\kern0pt}SUP\ {\isacharparenleft}{\kern0pt}i{\isacharcomma}{\kern0pt}\ j{\isacharparenright}{\kern0pt}\ {\isasymin}\ {\isacharbraceleft}{\kern0pt}N{\isachardot}{\kern0pt}{\isachardot}{\kern0pt}{\isacharbraceright}{\kern0pt}\ {\isasymtimes}\ {\isacharbraceleft}{\kern0pt}N{\isachardot}{\kern0pt}{\isachardot}{\kern0pt}{\isacharbraceright}{\kern0pt}{\isachardot}{\kern0pt}\ dist\ {\isacharparenleft}{\kern0pt}s\ i\ x{\isacharparenright}{\kern0pt}\ {\isacharparenleft}{\kern0pt}s\ j\ x{\isacharparenright}{\kern0pt}{\isacharparenright}{\kern0pt}\ {\isasymge}\ dist\ {\isacharparenleft}{\kern0pt}s\ i\ x{\isacharparenright}{\kern0pt}\ {\isacharparenleft}{\kern0pt}s\ j\ x{\isacharparenright}{\kern0pt}{\isachardoublequoteclose}\ \isacommand{using}\isamarkupfalse%
\ asm\ bounded{\isacharunderscore}{\kern0pt}imp{\isacharunderscore}{\kern0pt}bdd{\isacharunderscore}{\kern0pt}above{\isacharbrackleft}{\kern0pt}OF\ bounded{\isacharunderscore}{\kern0pt}imp{\isacharunderscore}{\kern0pt}dist{\isacharunderscore}{\kern0pt}bounded{\isacharcomma}{\kern0pt}\ OF\ bounded{\isacharunderscore}{\kern0pt}range{\isacharunderscore}{\kern0pt}s{\isacharbrackright}{\kern0pt}\ \isacommand{by}\isamarkupfalse%
\ {\isacharparenleft}{\kern0pt}intro\ cSup{\isacharunderscore}{\kern0pt}upper{\isacharcomma}{\kern0pt}\ auto{\isacharparenright}{\kern0pt}\isanewline
\ \ \ \ \ \ \isacommand{ultimately}\isamarkupfalse%
\ \isacommand{have}\isamarkupfalse%
\ {\isachardoublequoteopen}diameter\ {\isacharbraceleft}{\kern0pt}s\ i\ x\ {\isacharbar}{\kern0pt}\ i{\isachardot}{\kern0pt}\ N\ {\isasymle}\ i{\isacharbraceright}{\kern0pt}\ {\isasymge}\ dist\ {\isacharparenleft}{\kern0pt}s\ i\ x{\isacharparenright}{\kern0pt}\ {\isacharparenleft}{\kern0pt}s\ j\ x{\isacharparenright}{\kern0pt}{\isachardoublequoteclose}\ \isacommand{by}\isamarkupfalse%
\ presburger\isanewline
\ \ \ \ \isacommand{{\isacharbraceright}{\kern0pt}}\isamarkupfalse%
\isanewline
\ \ \ \ \isacommand{hence}\isamarkupfalse%
\ {\isachardoublequoteopen}LINT\ x{\isacharbar}{\kern0pt}M{\isachardot}{\kern0pt}\ dist\ {\isacharparenleft}{\kern0pt}s\ i\ x{\isacharparenright}{\kern0pt}\ {\isacharparenleft}{\kern0pt}s\ j\ x{\isacharparenright}{\kern0pt}\ {\isacharless}{\kern0pt}\ e{\isachardoublequoteclose}\ \isakeyword{if}\ {\isachardoublequoteopen}i\ {\isasymge}\ N{\isachardoublequoteclose}\ {\isachardoublequoteopen}j\ {\isasymge}\ N{\isachardoublequoteclose}\ \isakeyword{for}\ i\ j\ \isacommand{using}\isamarkupfalse%
\ that\ {\isacharasterisk}{\kern0pt}\ \isacommand{by}\isamarkupfalse%
\ {\isacharparenleft}{\kern0pt}intro\ integral{\isacharunderscore}{\kern0pt}mono{\isacharbrackleft}{\kern0pt}OF\ dist{\isacharunderscore}{\kern0pt}integrable\ diameter{\isacharunderscore}{\kern0pt}integrable{\isacharcomma}{\kern0pt}\ THEN\ order{\isachardot}{\kern0pt}strict{\isacharunderscore}{\kern0pt}trans{\isadigit{1}}{\isacharbrackright}{\kern0pt}{\isacharcomma}{\kern0pt}\ blast{\isacharplus}{\kern0pt}{\isacharparenright}{\kern0pt}\isanewline
\ \ \ \ \isacommand{moreover}\isamarkupfalse%
\ \isacommand{have}\isamarkupfalse%
\ {\isachardoublequoteopen}LINT\ x{\isacharbar}{\kern0pt}M{\isachardot}{\kern0pt}\ norm\ {\isacharparenleft}{\kern0pt}cond{\isacharunderscore}{\kern0pt}exp\ M\ F\ {\isacharparenleft}{\kern0pt}s\ i{\isacharparenright}{\kern0pt}\ x\ {\isacharminus}{\kern0pt}\ cond{\isacharunderscore}{\kern0pt}exp\ M\ F\ {\isacharparenleft}{\kern0pt}s\ j{\isacharparenright}{\kern0pt}\ x{\isacharparenright}{\kern0pt}\ {\isasymle}\ LINT\ x{\isacharbar}{\kern0pt}M{\isachardot}{\kern0pt}\ dist\ {\isacharparenleft}{\kern0pt}s\ i\ x{\isacharparenright}{\kern0pt}\ {\isacharparenleft}{\kern0pt}s\ j\ x{\isacharparenright}{\kern0pt}{\isachardoublequoteclose}\ \isakeyword{for}\ i\ j\isanewline
\ \ \ \ \isacommand{proof}\isamarkupfalse%
\ {\isacharminus}{\kern0pt}\isanewline
\ \ \ \ \ \ \isacommand{have}\isamarkupfalse%
\ {\isachardoublequoteopen}LINT\ x{\isacharbar}{\kern0pt}M{\isachardot}{\kern0pt}\ norm\ {\isacharparenleft}{\kern0pt}cond{\isacharunderscore}{\kern0pt}exp\ M\ F\ {\isacharparenleft}{\kern0pt}s\ i{\isacharparenright}{\kern0pt}\ x\ {\isacharminus}{\kern0pt}\ cond{\isacharunderscore}{\kern0pt}exp\ M\ F\ {\isacharparenleft}{\kern0pt}s\ j{\isacharparenright}{\kern0pt}\ x{\isacharparenright}{\kern0pt}\ {\isacharequal}{\kern0pt}\ LINT\ x{\isacharbar}{\kern0pt}M{\isachardot}{\kern0pt}\ norm\ {\isacharparenleft}{\kern0pt}cond{\isacharunderscore}{\kern0pt}exp\ M\ F\ {\isacharparenleft}{\kern0pt}s\ i{\isacharparenright}{\kern0pt}\ x\ {\isacharplus}{\kern0pt}\ {\isacharminus}{\kern0pt}\ {\isadigit{1}}\ {\isacharasterisk}{\kern0pt}\isactrlsub R\ cond{\isacharunderscore}{\kern0pt}exp\ M\ F\ {\isacharparenleft}{\kern0pt}s\ j{\isacharparenright}{\kern0pt}\ x{\isacharparenright}{\kern0pt}{\isachardoublequoteclose}\ \isacommand{unfolding}\isamarkupfalse%
\ dist{\isacharunderscore}{\kern0pt}norm\ \isacommand{by}\isamarkupfalse%
\ simp\isanewline
\ \ \ \ \ \ \isacommand{also}\isamarkupfalse%
\ \isacommand{have}\isamarkupfalse%
\ {\isachardoublequoteopen}{\isachardot}{\kern0pt}{\isachardot}{\kern0pt}{\isachardot}{\kern0pt}\ {\isacharequal}{\kern0pt}\ LINT\ x{\isacharbar}{\kern0pt}M{\isachardot}{\kern0pt}\ norm\ {\isacharparenleft}{\kern0pt}cond{\isacharunderscore}{\kern0pt}exp\ M\ F\ {\isacharparenleft}{\kern0pt}{\isasymlambda}x{\isachardot}{\kern0pt}\ s\ i\ x\ {\isacharminus}{\kern0pt}\ s\ j\ x{\isacharparenright}{\kern0pt}\ x{\isacharparenright}{\kern0pt}{\isachardoublequoteclose}\ \isacommand{using}\isamarkupfalse%
\ has{\isacharunderscore}{\kern0pt}cond{\isacharunderscore}{\kern0pt}exp{\isacharunderscore}{\kern0pt}charact{\isacharparenleft}{\kern0pt}{\isadigit{2}}{\isacharparenright}{\kern0pt}{\isacharbrackleft}{\kern0pt}OF\ has{\isacharunderscore}{\kern0pt}cond{\isacharunderscore}{\kern0pt}exp{\isacharunderscore}{\kern0pt}add{\isacharbrackleft}{\kern0pt}OF\ {\isacharunderscore}{\kern0pt}\ has{\isacharunderscore}{\kern0pt}cond{\isacharunderscore}{\kern0pt}exp{\isacharunderscore}{\kern0pt}scaleR{\isacharunderscore}{\kern0pt}right{\isacharcomma}{\kern0pt}\ OF\ has{\isacharunderscore}{\kern0pt}cond{\isacharunderscore}{\kern0pt}exp{\isacharunderscore}{\kern0pt}charact{\isacharparenleft}{\kern0pt}{\isadigit{1}}{\isacharcomma}{\kern0pt}{\isadigit{1}}{\isacharparenright}{\kern0pt}{\isacharcomma}{\kern0pt}\ OF\ has{\isacharunderscore}{\kern0pt}cond{\isacharunderscore}{\kern0pt}exp{\isacharunderscore}{\kern0pt}simple{\isacharparenleft}{\kern0pt}{\isadigit{1}}{\isacharcomma}{\kern0pt}{\isadigit{1}}{\isacharparenright}{\kern0pt}{\isacharbrackleft}{\kern0pt}OF\ assms{\isacharparenleft}{\kern0pt}{\isadigit{2}}{\isacharcomma}{\kern0pt}{\isadigit{3}}{\isacharparenright}{\kern0pt}{\isacharbrackright}{\kern0pt}{\isacharbrackright}{\kern0pt}{\isacharcomma}{\kern0pt}\ THEN\ AE{\isacharunderscore}{\kern0pt}symmetric{\isacharcomma}{\kern0pt}\ of\ i\ {\isachardoublequoteopen}{\isacharminus}{\kern0pt}{\isadigit{1}}{\isachardoublequoteclose}\ j{\isacharbrackright}{\kern0pt}\ \isacommand{by}\isamarkupfalse%
\ {\isacharparenleft}{\kern0pt}intro\ integral{\isacharunderscore}{\kern0pt}cong{\isacharunderscore}{\kern0pt}AE{\isacharparenright}{\kern0pt}\ force{\isacharplus}{\kern0pt}\ \ \ \ \ \ \isanewline
\ \ \ \ \ \ \isacommand{also}\isamarkupfalse%
\ \isacommand{have}\isamarkupfalse%
\ {\isachardoublequoteopen}{\isachardot}{\kern0pt}{\isachardot}{\kern0pt}{\isachardot}{\kern0pt}\ {\isasymle}\ LINT\ x{\isacharbar}{\kern0pt}M{\isachardot}{\kern0pt}\ cond{\isacharunderscore}{\kern0pt}exp\ M\ F\ {\isacharparenleft}{\kern0pt}{\isasymlambda}x{\isachardot}{\kern0pt}\ norm\ {\isacharparenleft}{\kern0pt}s\ i\ x\ {\isacharminus}{\kern0pt}\ s\ j\ x{\isacharparenright}{\kern0pt}{\isacharparenright}{\kern0pt}\ x{\isachardoublequoteclose}\ \isacommand{using}\isamarkupfalse%
\ cond{\isacharunderscore}{\kern0pt}exp{\isacharunderscore}{\kern0pt}contraction{\isacharunderscore}{\kern0pt}simple{\isacharbrackleft}{\kern0pt}OF\ {\isacharunderscore}{\kern0pt}\ fin{\isacharunderscore}{\kern0pt}sup{\isacharcomma}{\kern0pt}\ of\ i\ j{\isacharbrackright}{\kern0pt}\ integrable{\isacharunderscore}{\kern0pt}cond{\isacharunderscore}{\kern0pt}exp\ assms{\isacharparenleft}{\kern0pt}{\isadigit{2}}{\isacharparenright}{\kern0pt}\ \isacommand{by}\isamarkupfalse%
\ {\isacharparenleft}{\kern0pt}intro\ integral{\isacharunderscore}{\kern0pt}mono{\isacharunderscore}{\kern0pt}AE{\isacharcomma}{\kern0pt}\ fast{\isacharplus}{\kern0pt}{\isacharparenright}{\kern0pt}\isanewline
\ \ \ \ \ \ \isacommand{also}\isamarkupfalse%
\ \isacommand{have}\isamarkupfalse%
\ {\isachardoublequoteopen}{\isachardot}{\kern0pt}{\isachardot}{\kern0pt}{\isachardot}{\kern0pt}\ {\isacharequal}{\kern0pt}\ LINT\ x{\isacharbar}{\kern0pt}M{\isachardot}{\kern0pt}\ norm\ {\isacharparenleft}{\kern0pt}s\ i\ x\ {\isacharminus}{\kern0pt}\ s\ j\ x{\isacharparenright}{\kern0pt}{\isachardoublequoteclose}\ \isacommand{unfolding}\isamarkupfalse%
\ set{\isacharunderscore}{\kern0pt}integral{\isacharunderscore}{\kern0pt}space{\isacharparenleft}{\kern0pt}{\isadigit{1}}{\isacharparenright}{\kern0pt}{\isacharbrackleft}{\kern0pt}OF\ integrable{\isacharunderscore}{\kern0pt}cond{\isacharunderscore}{\kern0pt}exp{\isacharcomma}{\kern0pt}\ symmetric{\isacharbrackright}{\kern0pt}\ set{\isacharunderscore}{\kern0pt}integral{\isacharunderscore}{\kern0pt}space{\isacharbrackleft}{\kern0pt}OF\ dist{\isacharunderscore}{\kern0pt}norm{\isacharunderscore}{\kern0pt}integrable{\isacharcomma}{\kern0pt}\ symmetric{\isacharbrackright}{\kern0pt}\ \isacommand{by}\isamarkupfalse%
\ {\isacharparenleft}{\kern0pt}intro\ has{\isacharunderscore}{\kern0pt}cond{\isacharunderscore}{\kern0pt}expD{\isacharparenleft}{\kern0pt}{\isadigit{1}}{\isacharparenright}{\kern0pt}{\isacharbrackleft}{\kern0pt}OF\ has{\isacharunderscore}{\kern0pt}cond{\isacharunderscore}{\kern0pt}exp{\isacharunderscore}{\kern0pt}simple{\isacharbrackleft}{\kern0pt}OF\ {\isacharunderscore}{\kern0pt}\ fin{\isacharunderscore}{\kern0pt}sup{\isacharunderscore}{\kern0pt}norm{\isacharbrackright}{\kern0pt}{\isacharcomma}{\kern0pt}\ symmetric{\isacharbrackright}{\kern0pt}{\isacharparenright}{\kern0pt}\ {\isacharparenleft}{\kern0pt}metis\ assms{\isacharparenleft}{\kern0pt}{\isadigit{2}}{\isacharparenright}{\kern0pt}\ simple{\isacharunderscore}{\kern0pt}function{\isacharunderscore}{\kern0pt}compose{\isadigit{1}}\ simple{\isacharunderscore}{\kern0pt}function{\isacharunderscore}{\kern0pt}diff{\isacharcomma}{\kern0pt}\ metis\ sets{\isachardot}{\kern0pt}top\ subalg\ subalgebra{\isacharunderscore}{\kern0pt}def{\isacharparenright}{\kern0pt}\isanewline
\ \ \ \ \ \ \isacommand{finally}\isamarkupfalse%
\ \isacommand{show}\isamarkupfalse%
\ {\isacharquery}{\kern0pt}thesis\ \isacommand{unfolding}\isamarkupfalse%
\ dist{\isacharunderscore}{\kern0pt}norm\ \isacommand{{\isachardot}{\kern0pt}}\isamarkupfalse%
\ \ \isanewline
\ \ \ \ \isacommand{qed}\isamarkupfalse%
\isanewline
\ \ \ \ \isacommand{ultimately}\isamarkupfalse%
\ \isacommand{show}\isamarkupfalse%
\ {\isacharquery}{\kern0pt}thesis\ \isacommand{using}\isamarkupfalse%
\ order{\isachardot}{\kern0pt}strict{\isacharunderscore}{\kern0pt}trans{\isadigit{1}}\ \isacommand{by}\isamarkupfalse%
\ meson\isanewline
\ \ \isacommand{qed}\isamarkupfalse%
\isanewline
\ \ \isacommand{then}\isamarkupfalse%
\ \isacommand{obtain}\isamarkupfalse%
\ r\ \isakeyword{where}\ strict{\isacharunderscore}{\kern0pt}mono{\isacharunderscore}{\kern0pt}r{\isacharcolon}{\kern0pt}\ {\isachardoublequoteopen}strict{\isacharunderscore}{\kern0pt}mono\ r{\isachardoublequoteclose}\ \isakeyword{and}\ AE{\isacharunderscore}{\kern0pt}Cauchy{\isacharcolon}{\kern0pt}\ {\isachardoublequoteopen}AE\ x\ in\ M{\isachardot}{\kern0pt}\ Cauchy\ {\isacharparenleft}{\kern0pt}{\isasymlambda}i{\isachardot}{\kern0pt}\ cond{\isacharunderscore}{\kern0pt}exp\ M\ F\ {\isacharparenleft}{\kern0pt}s\ {\isacharparenleft}{\kern0pt}r\ i{\isacharparenright}{\kern0pt}{\isacharparenright}{\kern0pt}\ x{\isacharparenright}{\kern0pt}{\isachardoublequoteclose}\ \isacommand{by}\isamarkupfalse%
\ {\isacharparenleft}{\kern0pt}rule\ cauchy{\isacharunderscore}{\kern0pt}L{\isadigit{1}}{\isacharunderscore}{\kern0pt}AE{\isacharunderscore}{\kern0pt}cauchy{\isacharunderscore}{\kern0pt}subseq{\isacharbrackleft}{\kern0pt}OF\ integrable{\isacharunderscore}{\kern0pt}cond{\isacharunderscore}{\kern0pt}exp{\isacharbrackright}{\kern0pt}{\isacharcomma}{\kern0pt}\ auto{\isacharparenright}{\kern0pt}\isanewline
\ \ \isacommand{hence}\isamarkupfalse%
\ ae{\isacharunderscore}{\kern0pt}lim{\isacharunderscore}{\kern0pt}cond{\isacharunderscore}{\kern0pt}exp{\isacharcolon}{\kern0pt}\ {\isachardoublequoteopen}AE\ x\ in\ M{\isachardot}{\kern0pt}\ {\isacharparenleft}{\kern0pt}{\isasymlambda}n{\isachardot}{\kern0pt}\ cond{\isacharunderscore}{\kern0pt}exp\ M\ F\ {\isacharparenleft}{\kern0pt}s\ {\isacharparenleft}{\kern0pt}r\ n{\isacharparenright}{\kern0pt}{\isacharparenright}{\kern0pt}\ x{\isacharparenright}{\kern0pt}\ {\isasymlonglonglongrightarrow}\ lim\ {\isacharparenleft}{\kern0pt}{\isasymlambda}n{\isachardot}{\kern0pt}\ cond{\isacharunderscore}{\kern0pt}exp\ M\ F\ {\isacharparenleft}{\kern0pt}s\ {\isacharparenleft}{\kern0pt}r\ n{\isacharparenright}{\kern0pt}{\isacharparenright}{\kern0pt}\ x{\isacharparenright}{\kern0pt}{\isachardoublequoteclose}\ \isacommand{using}\isamarkupfalse%
\ Cauchy{\isacharunderscore}{\kern0pt}convergent{\isacharunderscore}{\kern0pt}iff\ convergent{\isacharunderscore}{\kern0pt}LIMSEQ{\isacharunderscore}{\kern0pt}iff\ \isacommand{by}\isamarkupfalse%
\ fastforce\isanewline
\isanewline
\ \ \isacommand{have}\isamarkupfalse%
\ cond{\isacharunderscore}{\kern0pt}exp{\isacharunderscore}{\kern0pt}bounded{\isacharcolon}{\kern0pt}\ {\isachardoublequoteopen}AE\ x\ in\ M{\isachardot}{\kern0pt}\ norm\ {\isacharparenleft}{\kern0pt}cond{\isacharunderscore}{\kern0pt}exp\ M\ F\ {\isacharparenleft}{\kern0pt}s\ {\isacharparenleft}{\kern0pt}r\ n{\isacharparenright}{\kern0pt}{\isacharparenright}{\kern0pt}\ x{\isacharparenright}{\kern0pt}\ {\isasymle}\ cond{\isacharunderscore}{\kern0pt}exp\ M\ F\ {\isacharparenleft}{\kern0pt}{\isasymlambda}x{\isachardot}{\kern0pt}\ {\isadigit{2}}\ {\isacharasterisk}{\kern0pt}\ norm\ {\isacharparenleft}{\kern0pt}f\ x{\isacharparenright}{\kern0pt}{\isacharparenright}{\kern0pt}\ x{\isachardoublequoteclose}\ \isakeyword{for}\ n\isanewline
\ \ \isacommand{proof}\isamarkupfalse%
\ {\isacharminus}{\kern0pt}\isanewline
\ \ \ \ \isacommand{have}\isamarkupfalse%
\ {\isachardoublequoteopen}AE\ x\ in\ M{\isachardot}{\kern0pt}\ norm\ {\isacharparenleft}{\kern0pt}cond{\isacharunderscore}{\kern0pt}exp\ M\ F\ {\isacharparenleft}{\kern0pt}s\ {\isacharparenleft}{\kern0pt}r\ n{\isacharparenright}{\kern0pt}{\isacharparenright}{\kern0pt}\ x{\isacharparenright}{\kern0pt}\ {\isasymle}\ cond{\isacharunderscore}{\kern0pt}exp\ M\ F\ {\isacharparenleft}{\kern0pt}{\isasymlambda}x{\isachardot}{\kern0pt}\ norm\ {\isacharparenleft}{\kern0pt}s\ {\isacharparenleft}{\kern0pt}r\ n{\isacharparenright}{\kern0pt}\ x{\isacharparenright}{\kern0pt}{\isacharparenright}{\kern0pt}\ x{\isachardoublequoteclose}\ \isacommand{by}\isamarkupfalse%
\ {\isacharparenleft}{\kern0pt}rule\ cond{\isacharunderscore}{\kern0pt}exp{\isacharunderscore}{\kern0pt}contraction{\isacharunderscore}{\kern0pt}simple{\isacharbrackleft}{\kern0pt}OF\ assms{\isacharparenleft}{\kern0pt}{\isadigit{2}}{\isacharcomma}{\kern0pt}{\isadigit{3}}{\isacharparenright}{\kern0pt}{\isacharbrackright}{\kern0pt}{\isacharparenright}{\kern0pt}\isanewline
\ \ \ \ \isacommand{moreover}\isamarkupfalse%
\ \isacommand{have}\isamarkupfalse%
\ {\isachardoublequoteopen}AE\ x\ in\ M{\isachardot}{\kern0pt}\ real{\isacharunderscore}{\kern0pt}cond{\isacharunderscore}{\kern0pt}exp\ M\ F\ {\isacharparenleft}{\kern0pt}{\isasymlambda}x{\isachardot}{\kern0pt}\ norm\ {\isacharparenleft}{\kern0pt}s\ {\isacharparenleft}{\kern0pt}r\ n{\isacharparenright}{\kern0pt}\ x{\isacharparenright}{\kern0pt}{\isacharparenright}{\kern0pt}\ x\ {\isasymle}\ real{\isacharunderscore}{\kern0pt}cond{\isacharunderscore}{\kern0pt}exp\ M\ F\ {\isacharparenleft}{\kern0pt}{\isasymlambda}x{\isachardot}{\kern0pt}\ {\isadigit{2}}\ {\isacharasterisk}{\kern0pt}\ norm\ {\isacharparenleft}{\kern0pt}f\ x{\isacharparenright}{\kern0pt}{\isacharparenright}{\kern0pt}\ x{\isachardoublequoteclose}\ \isacommand{using}\isamarkupfalse%
\ integrable{\isacharunderscore}{\kern0pt}s\ integrable{\isacharunderscore}{\kern0pt}{\isadigit{2}}f\ assms{\isacharparenleft}{\kern0pt}{\isadigit{5}}{\isacharparenright}{\kern0pt}\ \isacommand{by}\isamarkupfalse%
\ {\isacharparenleft}{\kern0pt}intro\ real{\isacharunderscore}{\kern0pt}cond{\isacharunderscore}{\kern0pt}exp{\isacharunderscore}{\kern0pt}mono{\isacharcomma}{\kern0pt}\ auto{\isacharparenright}{\kern0pt}\ \isanewline
\ \ \ \ \isacommand{ultimately}\isamarkupfalse%
\ \isacommand{show}\isamarkupfalse%
\ {\isacharquery}{\kern0pt}thesis\ \isacommand{using}\isamarkupfalse%
\ cond{\isacharunderscore}{\kern0pt}exp{\isacharunderscore}{\kern0pt}real{\isacharbrackleft}{\kern0pt}OF\ integrable{\isacharunderscore}{\kern0pt}norm{\isacharcomma}{\kern0pt}\ OF\ integrable{\isacharunderscore}{\kern0pt}s{\isacharcomma}{\kern0pt}\ of\ {\isachardoublequoteopen}r\ n{\isachardoublequoteclose}{\isacharbrackright}{\kern0pt}\ cond{\isacharunderscore}{\kern0pt}exp{\isacharunderscore}{\kern0pt}real{\isacharbrackleft}{\kern0pt}OF\ integrable{\isacharunderscore}{\kern0pt}{\isadigit{2}}f{\isacharbrackright}{\kern0pt}\ \isacommand{by}\isamarkupfalse%
\ force\isanewline
\ \ \isacommand{qed}\isamarkupfalse%
\isanewline
\ \ \isacommand{have}\isamarkupfalse%
\ lim{\isacharunderscore}{\kern0pt}integrable{\isacharcolon}{\kern0pt}\ {\isachardoublequoteopen}integrable\ M\ {\isacharparenleft}{\kern0pt}{\isasymlambda}x{\isachardot}{\kern0pt}\ lim\ {\isacharparenleft}{\kern0pt}{\isasymlambda}i{\isachardot}{\kern0pt}\ cond{\isacharunderscore}{\kern0pt}exp\ M\ F\ {\isacharparenleft}{\kern0pt}s\ {\isacharparenleft}{\kern0pt}r\ i{\isacharparenright}{\kern0pt}{\isacharparenright}{\kern0pt}\ x{\isacharparenright}{\kern0pt}{\isacharparenright}{\kern0pt}{\isachardoublequoteclose}\ \isacommand{by}\isamarkupfalse%
\ {\isacharparenleft}{\kern0pt}intro\ integrable{\isacharunderscore}{\kern0pt}dominated{\isacharunderscore}{\kern0pt}convergence{\isacharbrackleft}{\kern0pt}OF\ {\isacharunderscore}{\kern0pt}\ borel{\isacharunderscore}{\kern0pt}measurable{\isacharunderscore}{\kern0pt}cond{\isacharunderscore}{\kern0pt}exp{\isacharprime}{\kern0pt}\ integrable{\isacharunderscore}{\kern0pt}cond{\isacharunderscore}{\kern0pt}exp\ ae{\isacharunderscore}{\kern0pt}lim{\isacharunderscore}{\kern0pt}cond{\isacharunderscore}{\kern0pt}exp\ cond{\isacharunderscore}{\kern0pt}exp{\isacharunderscore}{\kern0pt}bounded{\isacharbrackright}{\kern0pt}{\isacharcomma}{\kern0pt}\ simp{\isacharparenright}{\kern0pt}\isanewline
\isanewline
\ \ \isacommand{{\isacharbraceleft}{\kern0pt}}\isamarkupfalse%
\isanewline
\ \ \ \ \isacommand{fix}\isamarkupfalse%
\ A\ \isacommand{assume}\isamarkupfalse%
\ A{\isacharunderscore}{\kern0pt}in{\isacharunderscore}{\kern0pt}sets{\isacharunderscore}{\kern0pt}F{\isacharcolon}{\kern0pt}\ {\isachardoublequoteopen}A\ {\isasymin}\ sets\ F{\isachardoublequoteclose}\isanewline
\ \ \ \ \isacommand{have}\isamarkupfalse%
\ {\isachardoublequoteopen}AE\ x\ in\ M{\isachardot}{\kern0pt}\ norm\ {\isacharparenleft}{\kern0pt}indicator\ A\ x\ {\isacharasterisk}{\kern0pt}\isactrlsub R\ cond{\isacharunderscore}{\kern0pt}exp\ M\ F\ {\isacharparenleft}{\kern0pt}s\ {\isacharparenleft}{\kern0pt}r\ n{\isacharparenright}{\kern0pt}{\isacharparenright}{\kern0pt}\ x{\isacharparenright}{\kern0pt}\ {\isasymle}\ cond{\isacharunderscore}{\kern0pt}exp\ M\ F\ {\isacharparenleft}{\kern0pt}{\isasymlambda}x{\isachardot}{\kern0pt}\ {\isadigit{2}}\ {\isacharasterisk}{\kern0pt}\ norm\ {\isacharparenleft}{\kern0pt}f\ x{\isacharparenright}{\kern0pt}{\isacharparenright}{\kern0pt}\ x{\isachardoublequoteclose}\ \isakeyword{for}\ n\isanewline
\ \ \ \ \isacommand{proof}\isamarkupfalse%
\ {\isacharminus}{\kern0pt}\isanewline
\ \ \ \ \ \ \isacommand{have}\isamarkupfalse%
\ {\isachardoublequoteopen}AE\ x\ in\ M{\isachardot}{\kern0pt}\ norm\ {\isacharparenleft}{\kern0pt}indicator\ A\ x\ {\isacharasterisk}{\kern0pt}\isactrlsub R\ cond{\isacharunderscore}{\kern0pt}exp\ M\ F\ {\isacharparenleft}{\kern0pt}s\ {\isacharparenleft}{\kern0pt}r\ n{\isacharparenright}{\kern0pt}{\isacharparenright}{\kern0pt}\ x{\isacharparenright}{\kern0pt}\ {\isasymle}\ norm\ {\isacharparenleft}{\kern0pt}cond{\isacharunderscore}{\kern0pt}exp\ M\ F\ {\isacharparenleft}{\kern0pt}s\ {\isacharparenleft}{\kern0pt}r\ n{\isacharparenright}{\kern0pt}{\isacharparenright}{\kern0pt}\ x{\isacharparenright}{\kern0pt}{\isachardoublequoteclose}\ \isacommand{unfolding}\isamarkupfalse%
\ indicator{\isacharunderscore}{\kern0pt}def\ \isacommand{by}\isamarkupfalse%
\ simp\isanewline
\ \ \ \ \ \ \isacommand{thus}\isamarkupfalse%
\ {\isacharquery}{\kern0pt}thesis\ \isacommand{using}\isamarkupfalse%
\ cond{\isacharunderscore}{\kern0pt}exp{\isacharunderscore}{\kern0pt}bounded{\isacharbrackleft}{\kern0pt}of\ n{\isacharbrackright}{\kern0pt}\ \isacommand{by}\isamarkupfalse%
\ force\isanewline
\ \ \ \ \isacommand{qed}\isamarkupfalse%
\isanewline
\ \ \ \ \isacommand{hence}\isamarkupfalse%
\ lim{\isacharunderscore}{\kern0pt}cond{\isacharunderscore}{\kern0pt}exp{\isacharunderscore}{\kern0pt}int{\isacharcolon}{\kern0pt}\ {\isachardoublequoteopen}{\isacharparenleft}{\kern0pt}{\isasymlambda}n{\isachardot}{\kern0pt}\ LINT\ x{\isacharcolon}{\kern0pt}A{\isacharbar}{\kern0pt}M{\isachardot}{\kern0pt}\ cond{\isacharunderscore}{\kern0pt}exp\ M\ F\ {\isacharparenleft}{\kern0pt}s\ {\isacharparenleft}{\kern0pt}r\ n{\isacharparenright}{\kern0pt}{\isacharparenright}{\kern0pt}\ x{\isacharparenright}{\kern0pt}\ {\isasymlonglonglongrightarrow}\ LINT\ x{\isacharcolon}{\kern0pt}A{\isacharbar}{\kern0pt}M{\isachardot}{\kern0pt}\ lim\ {\isacharparenleft}{\kern0pt}{\isasymlambda}n{\isachardot}{\kern0pt}\ cond{\isacharunderscore}{\kern0pt}exp\ M\ F\ {\isacharparenleft}{\kern0pt}s\ {\isacharparenleft}{\kern0pt}r\ n{\isacharparenright}{\kern0pt}{\isacharparenright}{\kern0pt}\ x{\isacharparenright}{\kern0pt}{\isachardoublequoteclose}\ \isanewline
\ \ \ \ \ \ \isacommand{using}\isamarkupfalse%
\ ae{\isacharunderscore}{\kern0pt}lim{\isacharunderscore}{\kern0pt}cond{\isacharunderscore}{\kern0pt}exp\ measurable{\isacharunderscore}{\kern0pt}from{\isacharunderscore}{\kern0pt}subalg{\isacharbrackleft}{\kern0pt}OF\ subalg\ borel{\isacharunderscore}{\kern0pt}measurable{\isacharunderscore}{\kern0pt}indicator{\isacharcomma}{\kern0pt}\ OF\ A{\isacharunderscore}{\kern0pt}in{\isacharunderscore}{\kern0pt}sets{\isacharunderscore}{\kern0pt}F{\isacharbrackright}{\kern0pt}\ cond{\isacharunderscore}{\kern0pt}exp{\isacharunderscore}{\kern0pt}bounded\isanewline
\ \ \ \ \ \ \isacommand{unfolding}\isamarkupfalse%
\ set{\isacharunderscore}{\kern0pt}lebesgue{\isacharunderscore}{\kern0pt}integral{\isacharunderscore}{\kern0pt}def\isanewline
\ \ \ \ \ \ \isacommand{by}\isamarkupfalse%
\ {\isacharparenleft}{\kern0pt}intro\ integral{\isacharunderscore}{\kern0pt}dominated{\isacharunderscore}{\kern0pt}convergence{\isacharbrackleft}{\kern0pt}OF\ borel{\isacharunderscore}{\kern0pt}measurable{\isacharunderscore}{\kern0pt}scaleR\ borel{\isacharunderscore}{\kern0pt}measurable{\isacharunderscore}{\kern0pt}scaleR\ integrable{\isacharunderscore}{\kern0pt}cond{\isacharunderscore}{\kern0pt}exp{\isacharbrackright}{\kern0pt}{\isacharparenright}{\kern0pt}\ {\isacharparenleft}{\kern0pt}fastforce\ simp\ add{\isacharcolon}{\kern0pt}\ tendsto{\isacharunderscore}{\kern0pt}scaleR{\isacharparenright}{\kern0pt}{\isacharplus}{\kern0pt}\isanewline
\isanewline
\ \ \ \ \isacommand{have}\isamarkupfalse%
\ {\isachardoublequoteopen}AE\ x\ in\ M{\isachardot}{\kern0pt}\ norm\ {\isacharparenleft}{\kern0pt}indicator\ A\ x\ {\isacharasterisk}{\kern0pt}\isactrlsub R\ s\ {\isacharparenleft}{\kern0pt}r\ n{\isacharparenright}{\kern0pt}\ x{\isacharparenright}{\kern0pt}\ {\isasymle}\ {\isadigit{2}}\ {\isacharasterisk}{\kern0pt}\ norm\ {\isacharparenleft}{\kern0pt}f\ x{\isacharparenright}{\kern0pt}{\isachardoublequoteclose}\ \isakeyword{for}\ n\isanewline
\ \ \ \ \isacommand{proof}\isamarkupfalse%
\ {\isacharminus}{\kern0pt}\isanewline
\ \ \ \ \ \ \isacommand{have}\isamarkupfalse%
\ {\isachardoublequoteopen}AE\ x\ in\ M{\isachardot}{\kern0pt}\ norm\ {\isacharparenleft}{\kern0pt}indicator\ A\ x\ {\isacharasterisk}{\kern0pt}\isactrlsub R\ s\ {\isacharparenleft}{\kern0pt}r\ n{\isacharparenright}{\kern0pt}\ x{\isacharparenright}{\kern0pt}\ {\isasymle}\ norm\ {\isacharparenleft}{\kern0pt}s\ {\isacharparenleft}{\kern0pt}r\ n{\isacharparenright}{\kern0pt}\ x{\isacharparenright}{\kern0pt}{\isachardoublequoteclose}\ \isacommand{unfolding}\isamarkupfalse%
\ indicator{\isacharunderscore}{\kern0pt}def\ \isacommand{by}\isamarkupfalse%
\ simp\isanewline
\ \ \ \ \ \ \isacommand{thus}\isamarkupfalse%
\ {\isacharquery}{\kern0pt}thesis\ \isacommand{using}\isamarkupfalse%
\ assms{\isacharparenleft}{\kern0pt}{\isadigit{5}}{\isacharparenright}{\kern0pt}{\isacharbrackleft}{\kern0pt}of\ {\isacharunderscore}{\kern0pt}\ {\isachardoublequoteopen}r\ n{\isachardoublequoteclose}{\isacharbrackright}{\kern0pt}\ \isacommand{by}\isamarkupfalse%
\ fastforce\isanewline
\ \ \ \ \isacommand{qed}\isamarkupfalse%
\isanewline
\ \ \ \ \isacommand{hence}\isamarkupfalse%
\ lim{\isacharunderscore}{\kern0pt}s{\isacharunderscore}{\kern0pt}int{\isacharcolon}{\kern0pt}\ {\isachardoublequoteopen}{\isacharparenleft}{\kern0pt}{\isasymlambda}n{\isachardot}{\kern0pt}\ LINT\ x{\isacharcolon}{\kern0pt}A{\isacharbar}{\kern0pt}M{\isachardot}{\kern0pt}\ s\ {\isacharparenleft}{\kern0pt}r\ n{\isacharparenright}{\kern0pt}\ x{\isacharparenright}{\kern0pt}\ {\isasymlonglonglongrightarrow}\ LINT\ x{\isacharcolon}{\kern0pt}A{\isacharbar}{\kern0pt}M{\isachardot}{\kern0pt}\ f\ x{\isachardoublequoteclose}\isanewline
\ \ \ \ \ \ \isacommand{using}\isamarkupfalse%
\ measurable{\isacharunderscore}{\kern0pt}from{\isacharunderscore}{\kern0pt}subalg{\isacharbrackleft}{\kern0pt}OF\ subalg\ borel{\isacharunderscore}{\kern0pt}measurable{\isacharunderscore}{\kern0pt}indicator{\isacharcomma}{\kern0pt}\ OF\ A{\isacharunderscore}{\kern0pt}in{\isacharunderscore}{\kern0pt}sets{\isacharunderscore}{\kern0pt}F{\isacharbrackright}{\kern0pt}\ LIMSEQ{\isacharunderscore}{\kern0pt}subseq{\isacharunderscore}{\kern0pt}LIMSEQ{\isacharbrackleft}{\kern0pt}OF\ assms{\isacharparenleft}{\kern0pt}{\isadigit{4}}{\isacharparenright}{\kern0pt}\ strict{\isacharunderscore}{\kern0pt}mono{\isacharunderscore}{\kern0pt}r{\isacharbrackright}{\kern0pt}\ assms{\isacharparenleft}{\kern0pt}{\isadigit{5}}{\isacharparenright}{\kern0pt}\isanewline
\ \ \ \ \ \ \isacommand{unfolding}\isamarkupfalse%
\ set{\isacharunderscore}{\kern0pt}lebesgue{\isacharunderscore}{\kern0pt}integral{\isacharunderscore}{\kern0pt}def\ comp{\isacharunderscore}{\kern0pt}def\isanewline
\ \ \ \ \ \ \isacommand{by}\isamarkupfalse%
\ {\isacharparenleft}{\kern0pt}intro\ integral{\isacharunderscore}{\kern0pt}dominated{\isacharunderscore}{\kern0pt}convergence{\isacharbrackleft}{\kern0pt}OF\ borel{\isacharunderscore}{\kern0pt}measurable{\isacharunderscore}{\kern0pt}scaleR\ borel{\isacharunderscore}{\kern0pt}measurable{\isacharunderscore}{\kern0pt}scaleR\ integrable{\isacharunderscore}{\kern0pt}{\isadigit{2}}f{\isacharbrackright}{\kern0pt}{\isacharparenright}{\kern0pt}\ {\isacharparenleft}{\kern0pt}fastforce\ simp\ add{\isacharcolon}{\kern0pt}\ tendsto{\isacharunderscore}{\kern0pt}scaleR{\isacharparenright}{\kern0pt}{\isacharplus}{\kern0pt}\isanewline
\isanewline
\ \ \ \ \isacommand{have}\isamarkupfalse%
\ {\isachardoublequoteopen}LINT\ x{\isacharcolon}{\kern0pt}A{\isacharbar}{\kern0pt}M{\isachardot}{\kern0pt}\ lim\ {\isacharparenleft}{\kern0pt}{\isasymlambda}n{\isachardot}{\kern0pt}\ cond{\isacharunderscore}{\kern0pt}exp\ M\ F\ {\isacharparenleft}{\kern0pt}s\ {\isacharparenleft}{\kern0pt}r\ n{\isacharparenright}{\kern0pt}{\isacharparenright}{\kern0pt}\ x{\isacharparenright}{\kern0pt}\ {\isacharequal}{\kern0pt}\ lim\ {\isacharparenleft}{\kern0pt}{\isasymlambda}n{\isachardot}{\kern0pt}\ LINT\ x{\isacharcolon}{\kern0pt}A{\isacharbar}{\kern0pt}M{\isachardot}{\kern0pt}\ cond{\isacharunderscore}{\kern0pt}exp\ M\ F\ {\isacharparenleft}{\kern0pt}s\ {\isacharparenleft}{\kern0pt}r\ n{\isacharparenright}{\kern0pt}{\isacharparenright}{\kern0pt}\ x{\isacharparenright}{\kern0pt}{\isachardoublequoteclose}\ \isacommand{using}\isamarkupfalse%
\ limI{\isacharbrackleft}{\kern0pt}OF\ lim{\isacharunderscore}{\kern0pt}cond{\isacharunderscore}{\kern0pt}exp{\isacharunderscore}{\kern0pt}int{\isacharbrackright}{\kern0pt}\ \isacommand{by}\isamarkupfalse%
\ argo\isanewline
\ \ \ \ \isacommand{also}\isamarkupfalse%
\ \isacommand{have}\isamarkupfalse%
\ {\isachardoublequoteopen}{\isachardot}{\kern0pt}{\isachardot}{\kern0pt}{\isachardot}{\kern0pt}\ {\isacharequal}{\kern0pt}\ lim\ {\isacharparenleft}{\kern0pt}{\isasymlambda}n{\isachardot}{\kern0pt}\ LINT\ x{\isacharcolon}{\kern0pt}A{\isacharbar}{\kern0pt}M{\isachardot}{\kern0pt}\ s\ {\isacharparenleft}{\kern0pt}r\ n{\isacharparenright}{\kern0pt}\ x{\isacharparenright}{\kern0pt}{\isachardoublequoteclose}\ \isacommand{using}\isamarkupfalse%
\ has{\isacharunderscore}{\kern0pt}cond{\isacharunderscore}{\kern0pt}expD{\isacharparenleft}{\kern0pt}{\isadigit{1}}{\isacharparenright}{\kern0pt}{\isacharbrackleft}{\kern0pt}OF\ has{\isacharunderscore}{\kern0pt}cond{\isacharunderscore}{\kern0pt}exp{\isacharunderscore}{\kern0pt}simple{\isacharbrackleft}{\kern0pt}OF\ assms{\isacharparenleft}{\kern0pt}{\isadigit{2}}{\isacharcomma}{\kern0pt}{\isadigit{3}}{\isacharparenright}{\kern0pt}{\isacharbrackright}{\kern0pt}\ A{\isacharunderscore}{\kern0pt}in{\isacharunderscore}{\kern0pt}sets{\isacharunderscore}{\kern0pt}F{\isacharcomma}{\kern0pt}\ symmetric{\isacharbrackright}{\kern0pt}\ \isacommand{by}\isamarkupfalse%
\ presburger\isanewline
\ \ \ \ \isacommand{also}\isamarkupfalse%
\ \isacommand{have}\isamarkupfalse%
\ {\isachardoublequoteopen}{\isachardot}{\kern0pt}{\isachardot}{\kern0pt}{\isachardot}{\kern0pt}\ {\isacharequal}{\kern0pt}\ LINT\ x{\isacharcolon}{\kern0pt}A{\isacharbar}{\kern0pt}M{\isachardot}{\kern0pt}\ f\ x{\isachardoublequoteclose}\ \isacommand{using}\isamarkupfalse%
\ limI{\isacharbrackleft}{\kern0pt}OF\ lim{\isacharunderscore}{\kern0pt}s{\isacharunderscore}{\kern0pt}int{\isacharbrackright}{\kern0pt}\ \isacommand{by}\isamarkupfalse%
\ argo\isanewline
\ \ \ \ \isacommand{finally}\isamarkupfalse%
\ \isacommand{have}\isamarkupfalse%
\ {\isachardoublequoteopen}LINT\ x{\isacharcolon}{\kern0pt}A{\isacharbar}{\kern0pt}M{\isachardot}{\kern0pt}\ lim\ {\isacharparenleft}{\kern0pt}{\isasymlambda}n{\isachardot}{\kern0pt}\ cond{\isacharunderscore}{\kern0pt}exp\ M\ F\ {\isacharparenleft}{\kern0pt}s\ {\isacharparenleft}{\kern0pt}r\ n{\isacharparenright}{\kern0pt}{\isacharparenright}{\kern0pt}\ x{\isacharparenright}{\kern0pt}\ {\isacharequal}{\kern0pt}\ LINT\ x{\isacharcolon}{\kern0pt}A{\isacharbar}{\kern0pt}M{\isachardot}{\kern0pt}\ f\ x{\isachardoublequoteclose}\ \isacommand{{\isachardot}{\kern0pt}}\isamarkupfalse%
\isanewline
\ \ \isacommand{{\isacharbraceright}{\kern0pt}}\isamarkupfalse%
\isanewline
\ \ \isacommand{hence}\isamarkupfalse%
\ {\isachardoublequoteopen}has{\isacharunderscore}{\kern0pt}cond{\isacharunderscore}{\kern0pt}exp\ M\ F\ f\ {\isacharparenleft}{\kern0pt}{\isasymlambda}x{\isachardot}{\kern0pt}\ lim\ {\isacharparenleft}{\kern0pt}{\isasymlambda}i{\isachardot}{\kern0pt}\ cond{\isacharunderscore}{\kern0pt}exp\ M\ F\ {\isacharparenleft}{\kern0pt}s\ {\isacharparenleft}{\kern0pt}r\ i{\isacharparenright}{\kern0pt}{\isacharparenright}{\kern0pt}\ x{\isacharparenright}{\kern0pt}{\isacharparenright}{\kern0pt}{\isachardoublequoteclose}\ \isacommand{using}\isamarkupfalse%
\ assms{\isacharparenleft}{\kern0pt}{\isadigit{1}}{\isacharparenright}{\kern0pt}\ lim{\isacharunderscore}{\kern0pt}integrable\ \isacommand{by}\isamarkupfalse%
\ {\isacharparenleft}{\kern0pt}intro\ has{\isacharunderscore}{\kern0pt}cond{\isacharunderscore}{\kern0pt}expI{\isacharprime}{\kern0pt}{\isacharcomma}{\kern0pt}\ auto{\isacharparenright}{\kern0pt}\ \isanewline
\ \ \isacommand{thus}\isamarkupfalse%
\ thesis\ \isacommand{using}\isamarkupfalse%
\ AE{\isacharunderscore}{\kern0pt}Cauchy\ Cauchy{\isacharunderscore}{\kern0pt}convergent\ strict{\isacharunderscore}{\kern0pt}mono{\isacharunderscore}{\kern0pt}r\ \isacommand{by}\isamarkupfalse%
\ {\isacharparenleft}{\kern0pt}auto\ intro{\isacharbang}{\kern0pt}{\isacharcolon}{\kern0pt}\ that{\isacharparenright}{\kern0pt}\isanewline
\isacommand{qed}\isamarkupfalse%
%
\endisatagproof
{\isafoldproof}%
%
\isadelimproof
\isanewline
%
\endisadelimproof
\isanewline
\isacommand{lemma}\isamarkupfalse%
\ cond{\isacharunderscore}{\kern0pt}exp{\isacharunderscore}{\kern0pt}simple{\isacharunderscore}{\kern0pt}lim{\isacharcolon}{\kern0pt}\isanewline
\ \ \ \ \isakeyword{fixes}\ f\ {\isacharcolon}{\kern0pt}{\isacharcolon}{\kern0pt}\ {\isachardoublequoteopen}{\isacharprime}{\kern0pt}a\ {\isasymRightarrow}\ {\isacharprime}{\kern0pt}b{\isacharcolon}{\kern0pt}{\isacharcolon}{\kern0pt}{\isacharbraceleft}{\kern0pt}second{\isacharunderscore}{\kern0pt}countable{\isacharunderscore}{\kern0pt}topology{\isacharcomma}{\kern0pt}\ banach{\isacharbraceright}{\kern0pt}{\isachardoublequoteclose}\isanewline
\ \ \isakeyword{assumes}\ {\isacharbrackleft}{\kern0pt}measurable{\isacharbrackright}{\kern0pt}{\isacharcolon}{\kern0pt}{\isachardoublequoteopen}integrable\ M\ f{\isachardoublequoteclose}\isanewline
\ \ \ \ \ \ \isakeyword{and}\ {\isachardoublequoteopen}{\isasymAnd}i{\isachardot}{\kern0pt}\ simple{\isacharunderscore}{\kern0pt}function\ M\ {\isacharparenleft}{\kern0pt}s\ i{\isacharparenright}{\kern0pt}{\isachardoublequoteclose}\isanewline
\ \ \ \ \ \ \isakeyword{and}\ {\isachardoublequoteopen}{\isasymAnd}i{\isachardot}{\kern0pt}\ emeasure\ M\ {\isacharbraceleft}{\kern0pt}y\ {\isasymin}\ space\ M{\isachardot}{\kern0pt}\ s\ i\ y\ {\isasymnoteq}\ {\isadigit{0}}{\isacharbraceright}{\kern0pt}\ {\isasymnoteq}\ {\isasyminfinity}{\isachardoublequoteclose}\isanewline
\ \ \ \ \ \ \isakeyword{and}\ {\isachardoublequoteopen}{\isasymAnd}x{\isachardot}{\kern0pt}\ x\ {\isasymin}\ space\ M\ {\isasymLongrightarrow}\ {\isacharparenleft}{\kern0pt}{\isasymlambda}i{\isachardot}{\kern0pt}\ s\ i\ x{\isacharparenright}{\kern0pt}\ {\isasymlonglonglongrightarrow}\ f\ x{\isachardoublequoteclose}\isanewline
\ \ \ \ \ \ \isakeyword{and}\ {\isachardoublequoteopen}{\isasymAnd}x\ i{\isachardot}{\kern0pt}\ x\ {\isasymin}\ space\ M\ {\isasymLongrightarrow}\ norm\ {\isacharparenleft}{\kern0pt}s\ i\ x{\isacharparenright}{\kern0pt}\ {\isasymle}\ {\isadigit{2}}\ {\isacharasterisk}{\kern0pt}\ norm\ {\isacharparenleft}{\kern0pt}f\ x{\isacharparenright}{\kern0pt}{\isachardoublequoteclose}\isanewline
\ \ \isakeyword{obtains}\ r\ \isakeyword{where}\ {\isachardoublequoteopen}AE\ x\ in\ M{\isachardot}{\kern0pt}\ {\isacharparenleft}{\kern0pt}{\isasymlambda}i{\isachardot}{\kern0pt}\ cond{\isacharunderscore}{\kern0pt}exp\ M\ F\ {\isacharparenleft}{\kern0pt}s\ {\isacharparenleft}{\kern0pt}r\ i{\isacharparenright}{\kern0pt}{\isacharparenright}{\kern0pt}\ x{\isacharparenright}{\kern0pt}\ {\isasymlonglonglongrightarrow}\ cond{\isacharunderscore}{\kern0pt}exp\ M\ F\ f\ x{\isachardoublequoteclose}\ {\isachardoublequoteopen}strict{\isacharunderscore}{\kern0pt}mono\ r{\isachardoublequoteclose}\isanewline
%
\isadelimproof
%
\endisadelimproof
%
\isatagproof
\isacommand{proof}\isamarkupfalse%
\ {\isacharminus}{\kern0pt}\isanewline
\ \ \isacommand{obtain}\isamarkupfalse%
\ r\ \isakeyword{where}\ {\isachardoublequoteopen}AE\ x\ in\ M{\isachardot}{\kern0pt}\ cond{\isacharunderscore}{\kern0pt}exp\ M\ F\ f\ x\ {\isacharequal}{\kern0pt}\ lim\ {\isacharparenleft}{\kern0pt}{\isasymlambda}i{\isachardot}{\kern0pt}\ cond{\isacharunderscore}{\kern0pt}exp\ M\ F\ {\isacharparenleft}{\kern0pt}s\ {\isacharparenleft}{\kern0pt}r\ i{\isacharparenright}{\kern0pt}{\isacharparenright}{\kern0pt}\ x{\isacharparenright}{\kern0pt}{\isachardoublequoteclose}\ {\isachardoublequoteopen}AE\ x\ in\ M{\isachardot}{\kern0pt}\ convergent\ {\isacharparenleft}{\kern0pt}{\isasymlambda}i{\isachardot}{\kern0pt}\ cond{\isacharunderscore}{\kern0pt}exp\ M\ F\ {\isacharparenleft}{\kern0pt}s\ {\isacharparenleft}{\kern0pt}r\ i{\isacharparenright}{\kern0pt}{\isacharparenright}{\kern0pt}\ x{\isacharparenright}{\kern0pt}{\isachardoublequoteclose}\ {\isachardoublequoteopen}strict{\isacharunderscore}{\kern0pt}mono\ r{\isachardoublequoteclose}\ \isacommand{using}\isamarkupfalse%
\ has{\isacharunderscore}{\kern0pt}cond{\isacharunderscore}{\kern0pt}exp{\isacharunderscore}{\kern0pt}charact{\isacharparenleft}{\kern0pt}{\isadigit{2}}{\isacharparenright}{\kern0pt}\ \isacommand{by}\isamarkupfalse%
\ {\isacharparenleft}{\kern0pt}auto\ intro{\isacharcolon}{\kern0pt}\ has{\isacharunderscore}{\kern0pt}cond{\isacharunderscore}{\kern0pt}exp{\isacharunderscore}{\kern0pt}simple{\isacharunderscore}{\kern0pt}lim{\isacharbrackleft}{\kern0pt}OF\ assms{\isacharbrackright}{\kern0pt}{\isacharparenright}{\kern0pt}\isanewline
\ \ \isacommand{thus}\isamarkupfalse%
\ {\isacharquery}{\kern0pt}thesis\ \isacommand{by}\isamarkupfalse%
\ {\isacharparenleft}{\kern0pt}auto\ intro{\isacharbang}{\kern0pt}{\isacharcolon}{\kern0pt}\ that{\isacharbrackleft}{\kern0pt}of\ r{\isacharbrackright}{\kern0pt}\ simp{\isacharcolon}{\kern0pt}\ convergent{\isacharunderscore}{\kern0pt}LIMSEQ{\isacharunderscore}{\kern0pt}iff{\isacharparenright}{\kern0pt}\isanewline
\isacommand{qed}\isamarkupfalse%
%
\endisatagproof
{\isafoldproof}%
%
\isadelimproof
\isanewline
%
\endisadelimproof
\isanewline
\isacommand{corollary}\isamarkupfalse%
\ has{\isacharunderscore}{\kern0pt}cond{\isacharunderscore}{\kern0pt}expI{\isacharcolon}{\kern0pt}\isanewline
\ \ \isakeyword{fixes}\ f\ {\isacharcolon}{\kern0pt}{\isacharcolon}{\kern0pt}\ {\isachardoublequoteopen}{\isacharprime}{\kern0pt}a\ {\isasymRightarrow}\ {\isacharprime}{\kern0pt}b{\isacharcolon}{\kern0pt}{\isacharcolon}{\kern0pt}{\isacharbraceleft}{\kern0pt}second{\isacharunderscore}{\kern0pt}countable{\isacharunderscore}{\kern0pt}topology{\isacharcomma}{\kern0pt}banach{\isacharbraceright}{\kern0pt}{\isachardoublequoteclose}\isanewline
\ \ \isakeyword{assumes}\ {\isachardoublequoteopen}integrable\ M\ f{\isachardoublequoteclose}\isanewline
\ \ \isakeyword{shows}\ {\isachardoublequoteopen}has{\isacharunderscore}{\kern0pt}cond{\isacharunderscore}{\kern0pt}exp\ M\ F\ f\ {\isacharparenleft}{\kern0pt}cond{\isacharunderscore}{\kern0pt}exp\ M\ F\ f{\isacharparenright}{\kern0pt}{\isachardoublequoteclose}\isanewline
%
\isadelimproof
%
\endisadelimproof
%
\isatagproof
\isacommand{proof}\isamarkupfalse%
\ {\isacharminus}{\kern0pt}\isanewline
\ \ \isacommand{obtain}\isamarkupfalse%
\ s\ \isakeyword{where}\ s{\isacharunderscore}{\kern0pt}is{\isacharcolon}{\kern0pt}\ {\isachardoublequoteopen}{\isasymAnd}i{\isachardot}{\kern0pt}\ simple{\isacharunderscore}{\kern0pt}function\ M\ {\isacharparenleft}{\kern0pt}s\ i{\isacharparenright}{\kern0pt}{\isachardoublequoteclose}\ {\isachardoublequoteopen}{\isasymAnd}i{\isachardot}{\kern0pt}\ emeasure\ M\ {\isacharbraceleft}{\kern0pt}y\ {\isasymin}\ space\ M{\isachardot}{\kern0pt}\ s\ i\ y\ {\isasymnoteq}\ {\isadigit{0}}{\isacharbraceright}{\kern0pt}\ {\isasymnoteq}\ {\isasyminfinity}{\isachardoublequoteclose}\ {\isachardoublequoteopen}{\isasymAnd}x{\isachardot}{\kern0pt}\ x\ {\isasymin}\ space\ M\ {\isasymLongrightarrow}\ {\isacharparenleft}{\kern0pt}{\isasymlambda}i{\isachardot}{\kern0pt}\ s\ i\ x{\isacharparenright}{\kern0pt}\ {\isasymlonglonglongrightarrow}\ f\ x{\isachardoublequoteclose}\ {\isachardoublequoteopen}{\isasymAnd}x\ i{\isachardot}{\kern0pt}\ x\ {\isasymin}\ space\ M\ {\isasymLongrightarrow}\ norm\ {\isacharparenleft}{\kern0pt}s\ i\ x{\isacharparenright}{\kern0pt}\ {\isasymle}\ {\isadigit{2}}\ {\isacharasterisk}{\kern0pt}\ norm\ {\isacharparenleft}{\kern0pt}f\ x{\isacharparenright}{\kern0pt}{\isachardoublequoteclose}\ \isacommand{using}\isamarkupfalse%
\ integrable{\isacharunderscore}{\kern0pt}implies{\isacharunderscore}{\kern0pt}simple{\isacharunderscore}{\kern0pt}function{\isacharunderscore}{\kern0pt}sequence{\isacharbrackleft}{\kern0pt}OF\ assms{\isacharbrackright}{\kern0pt}\ \isacommand{by}\isamarkupfalse%
\ blast\isanewline
\ \ \isacommand{show}\isamarkupfalse%
\ {\isacharquery}{\kern0pt}thesis\ \isacommand{using}\isamarkupfalse%
\ has{\isacharunderscore}{\kern0pt}cond{\isacharunderscore}{\kern0pt}exp{\isacharunderscore}{\kern0pt}simple{\isacharunderscore}{\kern0pt}lim{\isacharbrackleft}{\kern0pt}OF\ assms\ s{\isacharunderscore}{\kern0pt}is{\isacharbrackright}{\kern0pt}\ has{\isacharunderscore}{\kern0pt}cond{\isacharunderscore}{\kern0pt}exp{\isacharunderscore}{\kern0pt}charact{\isacharparenleft}{\kern0pt}{\isadigit{1}}{\isacharparenright}{\kern0pt}\ \isacommand{by}\isamarkupfalse%
\ metis\isanewline
\isacommand{qed}\isamarkupfalse%
%
\endisatagproof
{\isafoldproof}%
%
\isadelimproof
\isanewline
%
\endisadelimproof
\isanewline
\isanewline
\isanewline
\isacommand{lemma}\isamarkupfalse%
\ cond{\isacharunderscore}{\kern0pt}exp{\isacharunderscore}{\kern0pt}nested{\isacharunderscore}{\kern0pt}subalg{\isacharcolon}{\kern0pt}\isanewline
\ \ \isakeyword{fixes}\ f\ {\isacharcolon}{\kern0pt}{\isacharcolon}{\kern0pt}\ {\isachardoublequoteopen}{\isacharprime}{\kern0pt}a\ {\isasymRightarrow}\ {\isacharprime}{\kern0pt}b{\isacharcolon}{\kern0pt}{\isacharcolon}{\kern0pt}{\isacharbraceleft}{\kern0pt}second{\isacharunderscore}{\kern0pt}countable{\isacharunderscore}{\kern0pt}topology{\isacharcomma}{\kern0pt}banach{\isacharbraceright}{\kern0pt}{\isachardoublequoteclose}\isanewline
\ \ \isakeyword{assumes}\ {\isachardoublequoteopen}integrable\ M\ f{\isachardoublequoteclose}\ {\isachardoublequoteopen}subalgebra\ M\ G{\isachardoublequoteclose}\ {\isachardoublequoteopen}subalgebra\ G\ F{\isachardoublequoteclose}\isanewline
\ \ \isakeyword{shows}\ {\isachardoublequoteopen}AE\ {\isasymxi}\ in\ M{\isachardot}{\kern0pt}\ cond{\isacharunderscore}{\kern0pt}exp\ M\ F\ f\ {\isasymxi}\ {\isacharequal}{\kern0pt}\ cond{\isacharunderscore}{\kern0pt}exp\ M\ F\ {\isacharparenleft}{\kern0pt}cond{\isacharunderscore}{\kern0pt}exp\ M\ G\ f{\isacharparenright}{\kern0pt}\ {\isasymxi}{\isachardoublequoteclose}\isanewline
%
\isadelimproof
\ \ %
\endisadelimproof
%
\isatagproof
\isacommand{using}\isamarkupfalse%
\ has{\isacharunderscore}{\kern0pt}cond{\isacharunderscore}{\kern0pt}expI\ assms\ sigma{\isacharunderscore}{\kern0pt}finite{\isacharunderscore}{\kern0pt}subalgebra{\isacharunderscore}{\kern0pt}def\ \isacommand{by}\isamarkupfalse%
\ {\isacharparenleft}{\kern0pt}auto\ intro{\isacharbang}{\kern0pt}{\isacharcolon}{\kern0pt}\ has{\isacharunderscore}{\kern0pt}cond{\isacharunderscore}{\kern0pt}exp{\isacharunderscore}{\kern0pt}nested{\isacharunderscore}{\kern0pt}subalg{\isacharbrackleft}{\kern0pt}THEN\ has{\isacharunderscore}{\kern0pt}cond{\isacharunderscore}{\kern0pt}exp{\isacharunderscore}{\kern0pt}charact{\isacharparenleft}{\kern0pt}{\isadigit{2}}{\isacharparenright}{\kern0pt}{\isacharcomma}{\kern0pt}\ THEN\ AE{\isacharunderscore}{\kern0pt}symmetric{\isacharbrackright}{\kern0pt}\ sigma{\isacharunderscore}{\kern0pt}finite{\isacharunderscore}{\kern0pt}subalgebra{\isachardot}{\kern0pt}has{\isacharunderscore}{\kern0pt}cond{\isacharunderscore}{\kern0pt}expI{\isacharbrackleft}{\kern0pt}OF\ sigma{\isacharunderscore}{\kern0pt}finite{\isacharunderscore}{\kern0pt}subalgebra{\isachardot}{\kern0pt}intro{\isacharbrackleft}{\kern0pt}OF\ assms{\isacharparenleft}{\kern0pt}{\isadigit{2}}{\isacharparenright}{\kern0pt}{\isacharbrackright}{\kern0pt}{\isacharbrackright}{\kern0pt}\ nested{\isacharunderscore}{\kern0pt}subalg{\isacharunderscore}{\kern0pt}is{\isacharunderscore}{\kern0pt}sigma{\isacharunderscore}{\kern0pt}finite{\isacharparenright}{\kern0pt}%
\endisatagproof
{\isafoldproof}%
%
\isadelimproof
\isanewline
%
\endisadelimproof
\isanewline
\isacommand{lemma}\isamarkupfalse%
\ cond{\isacharunderscore}{\kern0pt}exp{\isacharunderscore}{\kern0pt}set{\isacharunderscore}{\kern0pt}integral{\isacharcolon}{\kern0pt}\isanewline
\ \ \isakeyword{fixes}\ f\ {\isacharcolon}{\kern0pt}{\isacharcolon}{\kern0pt}\ {\isachardoublequoteopen}{\isacharprime}{\kern0pt}a\ {\isasymRightarrow}\ {\isacharprime}{\kern0pt}b{\isacharcolon}{\kern0pt}{\isacharcolon}{\kern0pt}{\isacharbraceleft}{\kern0pt}second{\isacharunderscore}{\kern0pt}countable{\isacharunderscore}{\kern0pt}topology{\isacharcomma}{\kern0pt}banach{\isacharbraceright}{\kern0pt}{\isachardoublequoteclose}\isanewline
\ \ \isakeyword{assumes}\ {\isachardoublequoteopen}integrable\ M\ f{\isachardoublequoteclose}\ {\isachardoublequoteopen}A\ {\isasymin}\ sets\ F{\isachardoublequoteclose}\isanewline
\ \ \isakeyword{shows}\ {\isachardoublequoteopen}{\isacharparenleft}{\kern0pt}{\isasymintegral}\ x\ {\isasymin}\ A{\isachardot}{\kern0pt}\ f\ x\ {\isasympartial}M{\isacharparenright}{\kern0pt}\ {\isacharequal}{\kern0pt}\ {\isacharparenleft}{\kern0pt}{\isasymintegral}\ x\ {\isasymin}\ A{\isachardot}{\kern0pt}\ cond{\isacharunderscore}{\kern0pt}exp\ M\ F\ f\ x\ {\isasympartial}M{\isacharparenright}{\kern0pt}{\isachardoublequoteclose}\isanewline
%
\isadelimproof
\ \ %
\endisadelimproof
%
\isatagproof
\isacommand{using}\isamarkupfalse%
\ has{\isacharunderscore}{\kern0pt}cond{\isacharunderscore}{\kern0pt}expD{\isacharparenleft}{\kern0pt}{\isadigit{1}}{\isacharparenright}{\kern0pt}{\isacharbrackleft}{\kern0pt}OF\ has{\isacharunderscore}{\kern0pt}cond{\isacharunderscore}{\kern0pt}expI{\isacharcomma}{\kern0pt}\ OF\ assms{\isacharbrackright}{\kern0pt}\ \isacommand{by}\isamarkupfalse%
\ argo%
\endisatagproof
{\isafoldproof}%
%
\isadelimproof
\isanewline
%
\endisadelimproof
\isanewline
\isacommand{lemma}\isamarkupfalse%
\ cond{\isacharunderscore}{\kern0pt}exp{\isacharunderscore}{\kern0pt}add{\isacharcolon}{\kern0pt}\isanewline
\ \ \isakeyword{fixes}\ f\ {\isacharcolon}{\kern0pt}{\isacharcolon}{\kern0pt}\ {\isachardoublequoteopen}{\isacharprime}{\kern0pt}a\ {\isasymRightarrow}\ {\isacharprime}{\kern0pt}b{\isacharcolon}{\kern0pt}{\isacharcolon}{\kern0pt}{\isacharbraceleft}{\kern0pt}second{\isacharunderscore}{\kern0pt}countable{\isacharunderscore}{\kern0pt}topology{\isacharcomma}{\kern0pt}banach{\isacharbraceright}{\kern0pt}{\isachardoublequoteclose}\isanewline
\ \ \isakeyword{assumes}\ {\isachardoublequoteopen}integrable\ M\ f{\isachardoublequoteclose}\ {\isachardoublequoteopen}integrable\ M\ g{\isachardoublequoteclose}\isanewline
\ \ \isakeyword{shows}\ {\isachardoublequoteopen}AE\ x\ in\ M{\isachardot}{\kern0pt}\ cond{\isacharunderscore}{\kern0pt}exp\ M\ F\ {\isacharparenleft}{\kern0pt}{\isasymlambda}x{\isachardot}{\kern0pt}\ f\ x\ {\isacharplus}{\kern0pt}\ g\ x{\isacharparenright}{\kern0pt}\ x\ {\isacharequal}{\kern0pt}\ cond{\isacharunderscore}{\kern0pt}exp\ M\ F\ f\ x\ {\isacharplus}{\kern0pt}\ cond{\isacharunderscore}{\kern0pt}exp\ M\ F\ g\ x{\isachardoublequoteclose}\isanewline
%
\isadelimproof
\ \ %
\endisadelimproof
%
\isatagproof
\isacommand{using}\isamarkupfalse%
\ has{\isacharunderscore}{\kern0pt}cond{\isacharunderscore}{\kern0pt}exp{\isacharunderscore}{\kern0pt}add{\isacharbrackleft}{\kern0pt}OF\ has{\isacharunderscore}{\kern0pt}cond{\isacharunderscore}{\kern0pt}expI{\isacharparenleft}{\kern0pt}{\isadigit{1}}{\isacharcomma}{\kern0pt}{\isadigit{1}}{\isacharparenright}{\kern0pt}{\isacharcomma}{\kern0pt}\ OF\ assms{\isacharcomma}{\kern0pt}\ THEN\ has{\isacharunderscore}{\kern0pt}cond{\isacharunderscore}{\kern0pt}exp{\isacharunderscore}{\kern0pt}charact{\isacharparenleft}{\kern0pt}{\isadigit{2}}{\isacharparenright}{\kern0pt}{\isacharbrackright}{\kern0pt}\ \isacommand{{\isachardot}{\kern0pt}}\isamarkupfalse%
%
\endisatagproof
{\isafoldproof}%
%
\isadelimproof
\isanewline
%
\endisadelimproof
\isanewline
\isacommand{lemma}\isamarkupfalse%
\ cond{\isacharunderscore}{\kern0pt}exp{\isacharunderscore}{\kern0pt}diff{\isacharcolon}{\kern0pt}\isanewline
\ \ \isakeyword{fixes}\ f\ {\isacharcolon}{\kern0pt}{\isacharcolon}{\kern0pt}\ {\isachardoublequoteopen}{\isacharprime}{\kern0pt}a\ {\isasymRightarrow}\ {\isacharprime}{\kern0pt}b\ {\isacharcolon}{\kern0pt}{\isacharcolon}{\kern0pt}\ {\isacharbraceleft}{\kern0pt}second{\isacharunderscore}{\kern0pt}countable{\isacharunderscore}{\kern0pt}topology{\isacharcomma}{\kern0pt}\ banach{\isacharbraceright}{\kern0pt}{\isachardoublequoteclose}\isanewline
\ \ \isakeyword{assumes}\ {\isachardoublequoteopen}integrable\ M\ f{\isachardoublequoteclose}\ {\isachardoublequoteopen}integrable\ M\ g{\isachardoublequoteclose}\isanewline
\ \ \isakeyword{shows}\ {\isachardoublequoteopen}AE\ x\ in\ M{\isachardot}{\kern0pt}\ cond{\isacharunderscore}{\kern0pt}exp\ M\ F\ {\isacharparenleft}{\kern0pt}{\isasymlambda}x{\isachardot}{\kern0pt}\ f\ x\ {\isacharminus}{\kern0pt}\ g\ x{\isacharparenright}{\kern0pt}\ x\ {\isacharequal}{\kern0pt}\ cond{\isacharunderscore}{\kern0pt}exp\ M\ F\ f\ x\ {\isacharminus}{\kern0pt}\ cond{\isacharunderscore}{\kern0pt}exp\ M\ F\ g\ x{\isachardoublequoteclose}\isanewline
%
\isadelimproof
\ \ %
\endisadelimproof
%
\isatagproof
\isacommand{using}\isamarkupfalse%
\ has{\isacharunderscore}{\kern0pt}cond{\isacharunderscore}{\kern0pt}exp{\isacharunderscore}{\kern0pt}add{\isacharbrackleft}{\kern0pt}OF\ {\isacharunderscore}{\kern0pt}\ has{\isacharunderscore}{\kern0pt}cond{\isacharunderscore}{\kern0pt}exp{\isacharunderscore}{\kern0pt}scaleR{\isacharunderscore}{\kern0pt}right{\isacharcomma}{\kern0pt}\ OF\ has{\isacharunderscore}{\kern0pt}cond{\isacharunderscore}{\kern0pt}expI{\isacharparenleft}{\kern0pt}{\isadigit{1}}{\isacharcomma}{\kern0pt}{\isadigit{1}}{\isacharparenright}{\kern0pt}{\isacharcomma}{\kern0pt}\ OF\ assms{\isacharcomma}{\kern0pt}\ THEN\ has{\isacharunderscore}{\kern0pt}cond{\isacharunderscore}{\kern0pt}exp{\isacharunderscore}{\kern0pt}charact{\isacharparenleft}{\kern0pt}{\isadigit{2}}{\isacharparenright}{\kern0pt}{\isacharcomma}{\kern0pt}\ of\ {\isachardoublequoteopen}{\isacharminus}{\kern0pt}{\isadigit{1}}{\isachardoublequoteclose}{\isacharbrackright}{\kern0pt}\ \isacommand{by}\isamarkupfalse%
\ simp%
\endisatagproof
{\isafoldproof}%
%
\isadelimproof
\isanewline
%
\endisadelimproof
\isanewline
\isacommand{lemma}\isamarkupfalse%
\ cond{\isacharunderscore}{\kern0pt}exp{\isacharunderscore}{\kern0pt}diff{\isacharprime}{\kern0pt}{\isacharcolon}{\kern0pt}\isanewline
\ \ \isakeyword{fixes}\ f\ {\isacharcolon}{\kern0pt}{\isacharcolon}{\kern0pt}\ {\isachardoublequoteopen}{\isacharprime}{\kern0pt}a\ {\isasymRightarrow}\ {\isacharprime}{\kern0pt}b\ {\isacharcolon}{\kern0pt}{\isacharcolon}{\kern0pt}\ {\isacharbraceleft}{\kern0pt}second{\isacharunderscore}{\kern0pt}countable{\isacharunderscore}{\kern0pt}topology{\isacharcomma}{\kern0pt}\ banach{\isacharbraceright}{\kern0pt}{\isachardoublequoteclose}\isanewline
\ \ \isakeyword{assumes}\ {\isachardoublequoteopen}integrable\ M\ f{\isachardoublequoteclose}\ {\isachardoublequoteopen}integrable\ M\ g{\isachardoublequoteclose}\isanewline
\ \ \isakeyword{shows}\ {\isachardoublequoteopen}AE\ x\ in\ M{\isachardot}{\kern0pt}\ cond{\isacharunderscore}{\kern0pt}exp\ M\ F\ {\isacharparenleft}{\kern0pt}f\ {\isacharminus}{\kern0pt}\ g{\isacharparenright}{\kern0pt}\ x\ {\isacharequal}{\kern0pt}\ cond{\isacharunderscore}{\kern0pt}exp\ M\ F\ f\ x\ {\isacharminus}{\kern0pt}\ cond{\isacharunderscore}{\kern0pt}exp\ M\ F\ g\ x{\isachardoublequoteclose}\isanewline
%
\isadelimproof
\ \ %
\endisadelimproof
%
\isatagproof
\isacommand{unfolding}\isamarkupfalse%
\ fun{\isacharunderscore}{\kern0pt}diff{\isacharunderscore}{\kern0pt}def\ \isacommand{using}\isamarkupfalse%
\ assms\ \isacommand{by}\isamarkupfalse%
\ {\isacharparenleft}{\kern0pt}rule\ cond{\isacharunderscore}{\kern0pt}exp{\isacharunderscore}{\kern0pt}diff{\isacharparenright}{\kern0pt}%
\endisatagproof
{\isafoldproof}%
%
\isadelimproof
\isanewline
%
\endisadelimproof
\isanewline
\isacommand{lemma}\isamarkupfalse%
\ cond{\isacharunderscore}{\kern0pt}exp{\isacharunderscore}{\kern0pt}scaleR{\isacharunderscore}{\kern0pt}left{\isacharcolon}{\kern0pt}\isanewline
\ \ \isakeyword{fixes}\ f\ {\isacharcolon}{\kern0pt}{\isacharcolon}{\kern0pt}\ {\isachardoublequoteopen}{\isacharprime}{\kern0pt}a\ {\isasymRightarrow}\ real{\isachardoublequoteclose}\isanewline
\ \ \isakeyword{assumes}\ {\isachardoublequoteopen}integrable\ M\ f{\isachardoublequoteclose}\isanewline
\ \ \isakeyword{shows}\ {\isachardoublequoteopen}AE\ x\ in\ M{\isachardot}{\kern0pt}\ cond{\isacharunderscore}{\kern0pt}exp\ M\ F\ {\isacharparenleft}{\kern0pt}{\isasymlambda}x{\isachardot}{\kern0pt}\ f\ x\ {\isacharasterisk}{\kern0pt}\isactrlsub R\ c{\isacharparenright}{\kern0pt}\ x\ {\isacharequal}{\kern0pt}\ cond{\isacharunderscore}{\kern0pt}exp\ M\ F\ f\ x\ {\isacharasterisk}{\kern0pt}\isactrlsub R\ c{\isachardoublequoteclose}\ \isanewline
%
\isadelimproof
\ \ %
\endisadelimproof
%
\isatagproof
\isacommand{using}\isamarkupfalse%
\ cond{\isacharunderscore}{\kern0pt}exp{\isacharunderscore}{\kern0pt}set{\isacharunderscore}{\kern0pt}integral{\isacharbrackleft}{\kern0pt}OF\ assms{\isacharbrackright}{\kern0pt}\ subalg\ assms\ \isacommand{unfolding}\isamarkupfalse%
\ subalgebra{\isacharunderscore}{\kern0pt}def\isanewline
\ \ \isacommand{by}\isamarkupfalse%
\ {\isacharparenleft}{\kern0pt}intro\ cond{\isacharunderscore}{\kern0pt}exp{\isacharunderscore}{\kern0pt}charact{\isacharcomma}{\kern0pt}\isanewline
\ \ \ \ \ \ subst\ set{\isacharunderscore}{\kern0pt}integral{\isacharunderscore}{\kern0pt}scaleR{\isacharunderscore}{\kern0pt}left{\isacharcomma}{\kern0pt}\ blast{\isacharcomma}{\kern0pt}\ intro\ assms{\isacharcomma}{\kern0pt}\ \isanewline
\ \ \ \ \ \ subst\ set{\isacharunderscore}{\kern0pt}integral{\isacharunderscore}{\kern0pt}scaleR{\isacharunderscore}{\kern0pt}left{\isacharcomma}{\kern0pt}\ blast{\isacharcomma}{\kern0pt}\ intro\ integrable{\isacharunderscore}{\kern0pt}cond{\isacharunderscore}{\kern0pt}exp{\isacharparenright}{\kern0pt}\isanewline
\ \ \ \ \ \ auto%
\endisatagproof
{\isafoldproof}%
%
\isadelimproof
\isanewline
%
\endisadelimproof
\isanewline
\isacommand{lemma}\isamarkupfalse%
\ cond{\isacharunderscore}{\kern0pt}exp{\isacharunderscore}{\kern0pt}contraction{\isacharcolon}{\kern0pt}\isanewline
\ \ \isakeyword{fixes}\ f\ {\isacharcolon}{\kern0pt}{\isacharcolon}{\kern0pt}\ {\isachardoublequoteopen}{\isacharprime}{\kern0pt}a\ {\isasymRightarrow}\ {\isacharprime}{\kern0pt}b{\isacharcolon}{\kern0pt}{\isacharcolon}{\kern0pt}{\isacharbraceleft}{\kern0pt}second{\isacharunderscore}{\kern0pt}countable{\isacharunderscore}{\kern0pt}topology{\isacharcomma}{\kern0pt}\ banach{\isacharbraceright}{\kern0pt}{\isachardoublequoteclose}\isanewline
\ \ \isakeyword{assumes}\ {\isachardoublequoteopen}integrable\ M\ f{\isachardoublequoteclose}\isanewline
\ \ \isakeyword{shows}\ {\isachardoublequoteopen}AE\ x\ in\ M{\isachardot}{\kern0pt}\ norm\ {\isacharparenleft}{\kern0pt}cond{\isacharunderscore}{\kern0pt}exp\ M\ F\ f\ x{\isacharparenright}{\kern0pt}\ {\isasymle}\ cond{\isacharunderscore}{\kern0pt}exp\ M\ F\ {\isacharparenleft}{\kern0pt}{\isasymlambda}x{\isachardot}{\kern0pt}\ norm\ {\isacharparenleft}{\kern0pt}f\ x{\isacharparenright}{\kern0pt}{\isacharparenright}{\kern0pt}\ x{\isachardoublequoteclose}\ \isanewline
%
\isadelimproof
%
\endisadelimproof
%
\isatagproof
\isacommand{proof}\isamarkupfalse%
\ {\isacharminus}{\kern0pt}\isanewline
\ \ \isacommand{obtain}\isamarkupfalse%
\ s\ \isakeyword{where}\ s{\isacharcolon}{\kern0pt}\ {\isachardoublequoteopen}{\isasymAnd}i{\isachardot}{\kern0pt}\ simple{\isacharunderscore}{\kern0pt}function\ M\ {\isacharparenleft}{\kern0pt}s\ i{\isacharparenright}{\kern0pt}{\isachardoublequoteclose}\ {\isachardoublequoteopen}{\isasymAnd}i{\isachardot}{\kern0pt}\ emeasure\ M\ {\isacharbraceleft}{\kern0pt}y\ {\isasymin}\ space\ M{\isachardot}{\kern0pt}\ s\ i\ y\ {\isasymnoteq}\ {\isadigit{0}}{\isacharbraceright}{\kern0pt}\ {\isasymnoteq}\ {\isasyminfinity}{\isachardoublequoteclose}\ {\isachardoublequoteopen}{\isasymAnd}x{\isachardot}{\kern0pt}\ x\ {\isasymin}\ space\ M\ {\isasymLongrightarrow}\ {\isacharparenleft}{\kern0pt}{\isasymlambda}i{\isachardot}{\kern0pt}\ s\ i\ x{\isacharparenright}{\kern0pt}\ {\isasymlonglonglongrightarrow}\ f\ x{\isachardoublequoteclose}\ {\isachardoublequoteopen}{\isasymAnd}i\ x{\isachardot}{\kern0pt}\ x\ {\isasymin}\ space\ M\ {\isasymLongrightarrow}\ norm\ {\isacharparenleft}{\kern0pt}s\ i\ x{\isacharparenright}{\kern0pt}\ {\isasymle}\ {\isadigit{2}}\ {\isacharasterisk}{\kern0pt}\ norm\ {\isacharparenleft}{\kern0pt}f\ x{\isacharparenright}{\kern0pt}{\isachardoublequoteclose}\ \isanewline
\ \ \ \ \isacommand{by}\isamarkupfalse%
\ {\isacharparenleft}{\kern0pt}blast\ intro{\isacharcolon}{\kern0pt}\ integrable{\isacharunderscore}{\kern0pt}implies{\isacharunderscore}{\kern0pt}simple{\isacharunderscore}{\kern0pt}function{\isacharunderscore}{\kern0pt}sequence{\isacharbrackleft}{\kern0pt}OF\ assms{\isacharbrackright}{\kern0pt}{\isacharparenright}{\kern0pt}\isanewline
\isanewline
\ \ \isacommand{obtain}\isamarkupfalse%
\ r\ \isakeyword{where}\ r{\isacharcolon}{\kern0pt}\ {\isachardoublequoteopen}AE\ x\ in\ M{\isachardot}{\kern0pt}\ {\isacharparenleft}{\kern0pt}{\isasymlambda}i{\isachardot}{\kern0pt}\ cond{\isacharunderscore}{\kern0pt}exp\ M\ F\ {\isacharparenleft}{\kern0pt}s\ {\isacharparenleft}{\kern0pt}r\ i{\isacharparenright}{\kern0pt}{\isacharparenright}{\kern0pt}\ x{\isacharparenright}{\kern0pt}\ {\isasymlonglonglongrightarrow}\ cond{\isacharunderscore}{\kern0pt}exp\ M\ F\ f\ x{\isachardoublequoteclose}\ {\isachardoublequoteopen}strict{\isacharunderscore}{\kern0pt}mono\ r{\isachardoublequoteclose}\ \isacommand{using}\isamarkupfalse%
\ cond{\isacharunderscore}{\kern0pt}exp{\isacharunderscore}{\kern0pt}simple{\isacharunderscore}{\kern0pt}lim{\isacharbrackleft}{\kern0pt}OF\ assms\ s{\isacharbrackright}{\kern0pt}\ \isacommand{by}\isamarkupfalse%
\ blast\isanewline
\isanewline
\ \ \isacommand{have}\isamarkupfalse%
\ norm{\isacharunderscore}{\kern0pt}s{\isacharunderscore}{\kern0pt}r{\isacharcolon}{\kern0pt}\ {\isachardoublequoteopen}{\isasymAnd}i{\isachardot}{\kern0pt}\ simple{\isacharunderscore}{\kern0pt}function\ M\ {\isacharparenleft}{\kern0pt}{\isasymlambda}x{\isachardot}{\kern0pt}\ norm\ {\isacharparenleft}{\kern0pt}s\ {\isacharparenleft}{\kern0pt}r\ i{\isacharparenright}{\kern0pt}\ x{\isacharparenright}{\kern0pt}{\isacharparenright}{\kern0pt}{\isachardoublequoteclose}\ {\isachardoublequoteopen}{\isasymAnd}i{\isachardot}{\kern0pt}\ emeasure\ M\ {\isacharbraceleft}{\kern0pt}y\ {\isasymin}\ space\ M{\isachardot}{\kern0pt}\ norm\ {\isacharparenleft}{\kern0pt}s\ {\isacharparenleft}{\kern0pt}r\ i{\isacharparenright}{\kern0pt}\ y{\isacharparenright}{\kern0pt}\ {\isasymnoteq}\ {\isadigit{0}}{\isacharbraceright}{\kern0pt}\ {\isasymnoteq}\ {\isasyminfinity}{\isachardoublequoteclose}\ {\isachardoublequoteopen}{\isasymAnd}x{\isachardot}{\kern0pt}\ x\ {\isasymin}\ space\ M\ {\isasymLongrightarrow}\ {\isacharparenleft}{\kern0pt}{\isasymlambda}i{\isachardot}{\kern0pt}\ norm\ {\isacharparenleft}{\kern0pt}s\ {\isacharparenleft}{\kern0pt}r\ i{\isacharparenright}{\kern0pt}\ x{\isacharparenright}{\kern0pt}{\isacharparenright}{\kern0pt}\ {\isasymlonglonglongrightarrow}\ norm\ {\isacharparenleft}{\kern0pt}f\ x{\isacharparenright}{\kern0pt}{\isachardoublequoteclose}\ {\isachardoublequoteopen}{\isasymAnd}i\ x{\isachardot}{\kern0pt}\ x\ {\isasymin}\ space\ M\ {\isasymLongrightarrow}\ norm\ {\isacharparenleft}{\kern0pt}norm\ {\isacharparenleft}{\kern0pt}s\ {\isacharparenleft}{\kern0pt}r\ i{\isacharparenright}{\kern0pt}\ x{\isacharparenright}{\kern0pt}{\isacharparenright}{\kern0pt}\ {\isasymle}\ {\isadigit{2}}\ {\isacharasterisk}{\kern0pt}\ norm\ {\isacharparenleft}{\kern0pt}norm\ {\isacharparenleft}{\kern0pt}f\ x{\isacharparenright}{\kern0pt}{\isacharparenright}{\kern0pt}{\isachardoublequoteclose}\ \isanewline
\ \ \ \ \isacommand{using}\isamarkupfalse%
\ s\ \isacommand{by}\isamarkupfalse%
\ {\isacharparenleft}{\kern0pt}auto\ intro{\isacharcolon}{\kern0pt}\ LIMSEQ{\isacharunderscore}{\kern0pt}subseq{\isacharunderscore}{\kern0pt}LIMSEQ{\isacharbrackleft}{\kern0pt}OF\ tendsto{\isacharunderscore}{\kern0pt}norm\ r{\isacharparenleft}{\kern0pt}{\isadigit{2}}{\isacharparenright}{\kern0pt}{\isacharcomma}{\kern0pt}\ unfolded\ comp{\isacharunderscore}{\kern0pt}def{\isacharbrackright}{\kern0pt}\ simple{\isacharunderscore}{\kern0pt}function{\isacharunderscore}{\kern0pt}compose{\isadigit{1}}{\isacharparenright}{\kern0pt}\ \isanewline
\ \ \isanewline
\ \ \isacommand{obtain}\isamarkupfalse%
\ r{\isacharprime}{\kern0pt}\ \isakeyword{where}\ r{\isacharprime}{\kern0pt}{\isacharcolon}{\kern0pt}\ {\isachardoublequoteopen}AE\ x\ in\ M{\isachardot}{\kern0pt}\ {\isacharparenleft}{\kern0pt}{\isasymlambda}i{\isachardot}{\kern0pt}\ {\isacharparenleft}{\kern0pt}cond{\isacharunderscore}{\kern0pt}exp\ M\ F\ {\isacharparenleft}{\kern0pt}{\isasymlambda}x{\isachardot}{\kern0pt}\ norm\ {\isacharparenleft}{\kern0pt}s\ {\isacharparenleft}{\kern0pt}r\ {\isacharparenleft}{\kern0pt}r{\isacharprime}{\kern0pt}\ i{\isacharparenright}{\kern0pt}{\isacharparenright}{\kern0pt}\ x{\isacharparenright}{\kern0pt}{\isacharparenright}{\kern0pt}\ x{\isacharparenright}{\kern0pt}{\isacharparenright}{\kern0pt}\ {\isasymlonglonglongrightarrow}\ cond{\isacharunderscore}{\kern0pt}exp\ M\ F\ {\isacharparenleft}{\kern0pt}{\isasymlambda}x{\isachardot}{\kern0pt}\ norm\ {\isacharparenleft}{\kern0pt}f\ x{\isacharparenright}{\kern0pt}{\isacharparenright}{\kern0pt}\ x{\isachardoublequoteclose}\ {\isachardoublequoteopen}strict{\isacharunderscore}{\kern0pt}mono\ r{\isacharprime}{\kern0pt}{\isachardoublequoteclose}\ \isacommand{using}\isamarkupfalse%
\ cond{\isacharunderscore}{\kern0pt}exp{\isacharunderscore}{\kern0pt}simple{\isacharunderscore}{\kern0pt}lim{\isacharbrackleft}{\kern0pt}OF\ integrable{\isacharunderscore}{\kern0pt}norm\ norm{\isacharunderscore}{\kern0pt}s{\isacharunderscore}{\kern0pt}r{\isacharcomma}{\kern0pt}\ OF\ assms{\isacharbrackright}{\kern0pt}\ \isacommand{by}\isamarkupfalse%
\ blast\isanewline
\isanewline
\ \ \isacommand{have}\isamarkupfalse%
\ {\isachardoublequoteopen}AE\ x\ in\ M{\isachardot}{\kern0pt}\ {\isasymforall}i{\isachardot}{\kern0pt}\ norm\ {\isacharparenleft}{\kern0pt}cond{\isacharunderscore}{\kern0pt}exp\ M\ F\ {\isacharparenleft}{\kern0pt}s\ {\isacharparenleft}{\kern0pt}r\ {\isacharparenleft}{\kern0pt}r{\isacharprime}{\kern0pt}\ i{\isacharparenright}{\kern0pt}{\isacharparenright}{\kern0pt}{\isacharparenright}{\kern0pt}\ x{\isacharparenright}{\kern0pt}\ {\isasymle}\ cond{\isacharunderscore}{\kern0pt}exp\ M\ F\ {\isacharparenleft}{\kern0pt}{\isasymlambda}x{\isachardot}{\kern0pt}\ norm\ {\isacharparenleft}{\kern0pt}s\ {\isacharparenleft}{\kern0pt}r\ {\isacharparenleft}{\kern0pt}r{\isacharprime}{\kern0pt}\ i{\isacharparenright}{\kern0pt}{\isacharparenright}{\kern0pt}\ x{\isacharparenright}{\kern0pt}{\isacharparenright}{\kern0pt}\ x{\isachardoublequoteclose}\ \isacommand{using}\isamarkupfalse%
\ s\ \isacommand{by}\isamarkupfalse%
\ {\isacharparenleft}{\kern0pt}auto\ intro{\isacharcolon}{\kern0pt}\ cond{\isacharunderscore}{\kern0pt}exp{\isacharunderscore}{\kern0pt}contraction{\isacharunderscore}{\kern0pt}simple\ simp\ add{\isacharcolon}{\kern0pt}\ AE{\isacharunderscore}{\kern0pt}all{\isacharunderscore}{\kern0pt}countable{\isacharparenright}{\kern0pt}\isanewline
\ \ \isacommand{moreover}\isamarkupfalse%
\ \isacommand{have}\isamarkupfalse%
\ {\isachardoublequoteopen}AE\ x\ in\ M{\isachardot}{\kern0pt}\ {\isacharparenleft}{\kern0pt}{\isasymlambda}i{\isachardot}{\kern0pt}\ norm\ {\isacharparenleft}{\kern0pt}cond{\isacharunderscore}{\kern0pt}exp\ M\ F\ {\isacharparenleft}{\kern0pt}s\ {\isacharparenleft}{\kern0pt}r\ {\isacharparenleft}{\kern0pt}r{\isacharprime}{\kern0pt}\ i{\isacharparenright}{\kern0pt}{\isacharparenright}{\kern0pt}{\isacharparenright}{\kern0pt}\ x{\isacharparenright}{\kern0pt}{\isacharparenright}{\kern0pt}\ {\isasymlonglonglongrightarrow}\ norm\ {\isacharparenleft}{\kern0pt}cond{\isacharunderscore}{\kern0pt}exp\ M\ F\ f\ x{\isacharparenright}{\kern0pt}{\isachardoublequoteclose}\ \isacommand{using}\isamarkupfalse%
\ r\ LIMSEQ{\isacharunderscore}{\kern0pt}subseq{\isacharunderscore}{\kern0pt}LIMSEQ{\isacharbrackleft}{\kern0pt}OF\ tendsto{\isacharunderscore}{\kern0pt}norm\ r{\isacharprime}{\kern0pt}{\isacharparenleft}{\kern0pt}{\isadigit{2}}{\isacharparenright}{\kern0pt}{\isacharcomma}{\kern0pt}\ unfolded\ comp{\isacharunderscore}{\kern0pt}def{\isacharbrackright}{\kern0pt}\ \isacommand{by}\isamarkupfalse%
\ fast\isanewline
\ \ \isacommand{ultimately}\isamarkupfalse%
\ \isacommand{show}\isamarkupfalse%
\ {\isacharquery}{\kern0pt}thesis\ \isacommand{using}\isamarkupfalse%
\ LIMSEQ{\isacharunderscore}{\kern0pt}le\ r{\isacharprime}{\kern0pt}{\isacharparenleft}{\kern0pt}{\isadigit{1}}{\isacharparenright}{\kern0pt}\ \isacommand{by}\isamarkupfalse%
\ fast\isanewline
\isacommand{qed}\isamarkupfalse%
%
\endisatagproof
{\isafoldproof}%
%
\isadelimproof
\isanewline
%
\endisadelimproof
\isanewline
\isanewline
\isanewline
\isacommand{lemma}\isamarkupfalse%
\ cond{\isacharunderscore}{\kern0pt}exp{\isacharunderscore}{\kern0pt}measurable{\isacharunderscore}{\kern0pt}mult{\isacharcolon}{\kern0pt}\isanewline
\ \ \isakeyword{fixes}\ f\ g\ {\isacharcolon}{\kern0pt}{\isacharcolon}{\kern0pt}\ {\isachardoublequoteopen}{\isacharprime}{\kern0pt}a\ {\isasymRightarrow}\ real{\isachardoublequoteclose}\isanewline
\ \ \isakeyword{assumes}\ {\isacharbrackleft}{\kern0pt}measurable{\isacharbrackright}{\kern0pt}{\isacharcolon}{\kern0pt}\ {\isachardoublequoteopen}integrable\ M\ {\isacharparenleft}{\kern0pt}{\isasymlambda}x{\isachardot}{\kern0pt}\ f\ x\ {\isacharasterisk}{\kern0pt}\ g\ x{\isacharparenright}{\kern0pt}{\isachardoublequoteclose}\ {\isachardoublequoteopen}integrable\ M\ g{\isachardoublequoteclose}\ {\isachardoublequoteopen}f\ {\isasymin}\ borel{\isacharunderscore}{\kern0pt}measurable\ F{\isachardoublequoteclose}\ \isanewline
\ \ \isakeyword{shows}\ {\isachardoublequoteopen}integrable\ M\ {\isacharparenleft}{\kern0pt}{\isasymlambda}x{\isachardot}{\kern0pt}\ f\ x\ {\isacharasterisk}{\kern0pt}\ cond{\isacharunderscore}{\kern0pt}exp\ M\ F\ g\ x{\isacharparenright}{\kern0pt}{\isachardoublequoteclose}\isanewline
\ \ \ \ \ \ \ \ {\isachardoublequoteopen}AE\ x\ in\ M{\isachardot}{\kern0pt}\ cond{\isacharunderscore}{\kern0pt}exp\ M\ F\ {\isacharparenleft}{\kern0pt}{\isasymlambda}x{\isachardot}{\kern0pt}\ f\ x\ {\isacharasterisk}{\kern0pt}\ g\ x{\isacharparenright}{\kern0pt}\ x\ {\isacharequal}{\kern0pt}\ f\ x\ {\isacharasterisk}{\kern0pt}\ cond{\isacharunderscore}{\kern0pt}exp\ M\ F\ g\ x{\isachardoublequoteclose}\isanewline
%
\isadelimproof
%
\endisadelimproof
%
\isatagproof
\isacommand{proof}\isamarkupfalse%
{\isacharminus}{\kern0pt}\isanewline
\ \ \isacommand{show}\isamarkupfalse%
\ integrable{\isacharcolon}{\kern0pt}\ {\isachardoublequoteopen}integrable\ M\ {\isacharparenleft}{\kern0pt}{\isasymlambda}x{\isachardot}{\kern0pt}\ f\ x\ {\isacharasterisk}{\kern0pt}\ cond{\isacharunderscore}{\kern0pt}exp\ M\ F\ g\ x{\isacharparenright}{\kern0pt}{\isachardoublequoteclose}\ \isacommand{using}\isamarkupfalse%
\ cond{\isacharunderscore}{\kern0pt}exp{\isacharunderscore}{\kern0pt}real{\isacharbrackleft}{\kern0pt}OF\ assms{\isacharparenleft}{\kern0pt}{\isadigit{2}}{\isacharparenright}{\kern0pt}{\isacharbrackright}{\kern0pt}\ \isacommand{by}\isamarkupfalse%
\ {\isacharparenleft}{\kern0pt}intro\ integrable{\isacharunderscore}{\kern0pt}cong{\isacharunderscore}{\kern0pt}AE{\isacharunderscore}{\kern0pt}imp{\isacharbrackleft}{\kern0pt}OF\ real{\isacharunderscore}{\kern0pt}cond{\isacharunderscore}{\kern0pt}exp{\isacharunderscore}{\kern0pt}intg{\isacharparenleft}{\kern0pt}{\isadigit{1}}{\isacharparenright}{\kern0pt}{\isacharcomma}{\kern0pt}\ OF\ assms{\isacharparenleft}{\kern0pt}{\isadigit{1}}{\isacharcomma}{\kern0pt}{\isadigit{3}}{\isacharparenright}{\kern0pt}\ assms{\isacharparenleft}{\kern0pt}{\isadigit{2}}{\isacharparenright}{\kern0pt}{\isacharbrackleft}{\kern0pt}THEN\ borel{\isacharunderscore}{\kern0pt}measurable{\isacharunderscore}{\kern0pt}integrable{\isacharbrackright}{\kern0pt}{\isacharbrackright}{\kern0pt}\ measurable{\isacharunderscore}{\kern0pt}from{\isacharunderscore}{\kern0pt}subalg{\isacharbrackleft}{\kern0pt}OF\ subalg{\isacharbrackright}{\kern0pt}{\isacharparenright}{\kern0pt}\ auto\isanewline
\ \ \isacommand{interpret}\isamarkupfalse%
\ sigma{\isacharunderscore}{\kern0pt}finite{\isacharunderscore}{\kern0pt}measure\ {\isachardoublequoteopen}restr{\isacharunderscore}{\kern0pt}to{\isacharunderscore}{\kern0pt}subalg\ M\ F{\isachardoublequoteclose}\ \isacommand{by}\isamarkupfalse%
\ {\isacharparenleft}{\kern0pt}rule\ sigma{\isacharunderscore}{\kern0pt}fin{\isacharunderscore}{\kern0pt}subalg{\isacharparenright}{\kern0pt}\isanewline
\ \ \isacommand{{\isacharbraceleft}{\kern0pt}}\isamarkupfalse%
\isanewline
\ \ \ \ \isacommand{fix}\isamarkupfalse%
\ A\ \isacommand{assume}\isamarkupfalse%
\ asm{\isacharcolon}{\kern0pt}\ {\isachardoublequoteopen}A\ {\isasymin}\ sets\ F{\isachardoublequoteclose}\isanewline
\ \ \ \ \isacommand{hence}\isamarkupfalse%
\ asm{\isacharprime}{\kern0pt}{\isacharcolon}{\kern0pt}\ {\isachardoublequoteopen}A\ {\isasymin}\ sets\ M{\isachardoublequoteclose}\ \isacommand{using}\isamarkupfalse%
\ subalg\ \isacommand{by}\isamarkupfalse%
\ {\isacharparenleft}{\kern0pt}fastforce\ simp\ add{\isacharcolon}{\kern0pt}\ subalgebra{\isacharunderscore}{\kern0pt}def{\isacharparenright}{\kern0pt}\isanewline
\ \ \ \ \isacommand{have}\isamarkupfalse%
\ {\isachardoublequoteopen}set{\isacharunderscore}{\kern0pt}lebesgue{\isacharunderscore}{\kern0pt}integral\ M\ A\ {\isacharparenleft}{\kern0pt}cond{\isacharunderscore}{\kern0pt}exp\ M\ F\ {\isacharparenleft}{\kern0pt}{\isasymlambda}x{\isachardot}{\kern0pt}\ f\ x\ {\isacharasterisk}{\kern0pt}\ g\ x{\isacharparenright}{\kern0pt}{\isacharparenright}{\kern0pt}\ {\isacharequal}{\kern0pt}\ set{\isacharunderscore}{\kern0pt}lebesgue{\isacharunderscore}{\kern0pt}integral\ M\ A\ {\isacharparenleft}{\kern0pt}{\isasymlambda}x{\isachardot}{\kern0pt}\ f\ x\ {\isacharasterisk}{\kern0pt}\ g\ x{\isacharparenright}{\kern0pt}{\isachardoublequoteclose}\ \isacommand{by}\isamarkupfalse%
\ {\isacharparenleft}{\kern0pt}simp\ add{\isacharcolon}{\kern0pt}\ cond{\isacharunderscore}{\kern0pt}exp{\isacharunderscore}{\kern0pt}set{\isacharunderscore}{\kern0pt}integral{\isacharbrackleft}{\kern0pt}OF\ assms{\isacharparenleft}{\kern0pt}{\isadigit{1}}{\isacharparenright}{\kern0pt}\ asm{\isacharbrackright}{\kern0pt}{\isacharparenright}{\kern0pt}\isanewline
\ \ \ \ \isacommand{also}\isamarkupfalse%
\ \isacommand{have}\isamarkupfalse%
\ {\isachardoublequoteopen}{\isachardot}{\kern0pt}{\isachardot}{\kern0pt}{\isachardot}{\kern0pt}\ {\isacharequal}{\kern0pt}\ set{\isacharunderscore}{\kern0pt}lebesgue{\isacharunderscore}{\kern0pt}integral\ M\ A\ {\isacharparenleft}{\kern0pt}{\isasymlambda}x{\isachardot}{\kern0pt}\ f\ x\ {\isacharasterisk}{\kern0pt}\ real{\isacharunderscore}{\kern0pt}cond{\isacharunderscore}{\kern0pt}exp\ M\ F\ g\ x{\isacharparenright}{\kern0pt}{\isachardoublequoteclose}\ \isacommand{using}\isamarkupfalse%
\ borel{\isacharunderscore}{\kern0pt}measurable{\isacharunderscore}{\kern0pt}times{\isacharbrackleft}{\kern0pt}OF\ borel{\isacharunderscore}{\kern0pt}measurable{\isacharunderscore}{\kern0pt}indicator{\isacharbrackleft}{\kern0pt}OF\ asm{\isacharbrackright}{\kern0pt}\ assms{\isacharparenleft}{\kern0pt}{\isadigit{3}}{\isacharparenright}{\kern0pt}{\isacharbrackright}{\kern0pt}\ borel{\isacharunderscore}{\kern0pt}measurable{\isacharunderscore}{\kern0pt}integrable{\isacharbrackleft}{\kern0pt}OF\ assms{\isacharparenleft}{\kern0pt}{\isadigit{2}}{\isacharparenright}{\kern0pt}{\isacharbrackright}{\kern0pt}\ integrable{\isacharunderscore}{\kern0pt}mult{\isacharunderscore}{\kern0pt}indicator{\isacharbrackleft}{\kern0pt}OF\ asm{\isacharprime}{\kern0pt}\ assms{\isacharparenleft}{\kern0pt}{\isadigit{1}}{\isacharparenright}{\kern0pt}{\isacharbrackright}{\kern0pt}\ \isacommand{by}\isamarkupfalse%
\ {\isacharparenleft}{\kern0pt}fastforce\ simp\ add{\isacharcolon}{\kern0pt}\ set{\isacharunderscore}{\kern0pt}lebesgue{\isacharunderscore}{\kern0pt}integral{\isacharunderscore}{\kern0pt}def\ mult{\isachardot}{\kern0pt}assoc{\isacharbrackleft}{\kern0pt}symmetric{\isacharbrackright}{\kern0pt}\ intro{\isacharcolon}{\kern0pt}\ real{\isacharunderscore}{\kern0pt}cond{\isacharunderscore}{\kern0pt}exp{\isacharunderscore}{\kern0pt}intg{\isacharparenleft}{\kern0pt}{\isadigit{2}}{\isacharparenright}{\kern0pt}{\isacharbrackleft}{\kern0pt}symmetric{\isacharbrackright}{\kern0pt}{\isacharparenright}{\kern0pt}\isanewline
\ \ \ \ \isacommand{also}\isamarkupfalse%
\ \isacommand{have}\isamarkupfalse%
\ {\isachardoublequoteopen}{\isachardot}{\kern0pt}{\isachardot}{\kern0pt}{\isachardot}{\kern0pt}\ {\isacharequal}{\kern0pt}\ set{\isacharunderscore}{\kern0pt}lebesgue{\isacharunderscore}{\kern0pt}integral\ M\ A\ {\isacharparenleft}{\kern0pt}{\isasymlambda}x{\isachardot}{\kern0pt}\ f\ x\ {\isacharasterisk}{\kern0pt}\ cond{\isacharunderscore}{\kern0pt}exp\ M\ F\ g\ x{\isacharparenright}{\kern0pt}{\isachardoublequoteclose}\ \isacommand{using}\isamarkupfalse%
\ cond{\isacharunderscore}{\kern0pt}exp{\isacharunderscore}{\kern0pt}real{\isacharbrackleft}{\kern0pt}OF\ assms{\isacharparenleft}{\kern0pt}{\isadigit{2}}{\isacharparenright}{\kern0pt}{\isacharbrackright}{\kern0pt}\ asm{\isacharprime}{\kern0pt}\ borel{\isacharunderscore}{\kern0pt}measurable{\isacharunderscore}{\kern0pt}cond{\isacharunderscore}{\kern0pt}exp{\isacharprime}{\kern0pt}\ borel{\isacharunderscore}{\kern0pt}measurable{\isacharunderscore}{\kern0pt}cond{\isacharunderscore}{\kern0pt}exp{\isadigit{2}}\ measurable{\isacharunderscore}{\kern0pt}from{\isacharunderscore}{\kern0pt}subalg{\isacharbrackleft}{\kern0pt}OF\ subalg\ assms{\isacharparenleft}{\kern0pt}{\isadigit{3}}{\isacharparenright}{\kern0pt}{\isacharbrackright}{\kern0pt}\ \isacommand{by}\isamarkupfalse%
\ {\isacharparenleft}{\kern0pt}auto\ simp\ add{\isacharcolon}{\kern0pt}\ set{\isacharunderscore}{\kern0pt}lebesgue{\isacharunderscore}{\kern0pt}integral{\isacharunderscore}{\kern0pt}def\ intro{\isacharcolon}{\kern0pt}\ integral{\isacharunderscore}{\kern0pt}cong{\isacharunderscore}{\kern0pt}AE{\isacharparenright}{\kern0pt}\isanewline
\ \ \ \ \isacommand{finally}\isamarkupfalse%
\ \isacommand{have}\isamarkupfalse%
\ {\isachardoublequoteopen}set{\isacharunderscore}{\kern0pt}lebesgue{\isacharunderscore}{\kern0pt}integral\ M\ A\ {\isacharparenleft}{\kern0pt}cond{\isacharunderscore}{\kern0pt}exp\ M\ F\ {\isacharparenleft}{\kern0pt}{\isasymlambda}x{\isachardot}{\kern0pt}\ f\ x\ {\isacharasterisk}{\kern0pt}\ g\ x{\isacharparenright}{\kern0pt}{\isacharparenright}{\kern0pt}\ {\isacharequal}{\kern0pt}\ {\isasymintegral}x{\isasymin}A{\isachardot}{\kern0pt}\ {\isacharparenleft}{\kern0pt}f\ x\ {\isacharasterisk}{\kern0pt}\ cond{\isacharunderscore}{\kern0pt}exp\ M\ F\ g\ x{\isacharparenright}{\kern0pt}{\isasympartial}M{\isachardoublequoteclose}\ \isacommand{{\isachardot}{\kern0pt}}\isamarkupfalse%
\isanewline
\ \ \isacommand{{\isacharbraceright}{\kern0pt}}\isamarkupfalse%
\isanewline
\ \ \isacommand{hence}\isamarkupfalse%
\ {\isachardoublequoteopen}AE\ x\ in\ restr{\isacharunderscore}{\kern0pt}to{\isacharunderscore}{\kern0pt}subalg\ M\ F{\isachardot}{\kern0pt}\ cond{\isacharunderscore}{\kern0pt}exp\ M\ F\ {\isacharparenleft}{\kern0pt}{\isasymlambda}x{\isachardot}{\kern0pt}\ f\ x\ {\isacharasterisk}{\kern0pt}\ g\ x{\isacharparenright}{\kern0pt}\ x\ {\isacharequal}{\kern0pt}\ f\ x\ {\isacharasterisk}{\kern0pt}\ cond{\isacharunderscore}{\kern0pt}exp\ M\ F\ g\ x{\isachardoublequoteclose}\ \isacommand{by}\isamarkupfalse%
\ {\isacharparenleft}{\kern0pt}intro\ density{\isacharunderscore}{\kern0pt}unique\ integrable{\isacharunderscore}{\kern0pt}cond{\isacharunderscore}{\kern0pt}exp\ integrable\ integrable{\isacharunderscore}{\kern0pt}in{\isacharunderscore}{\kern0pt}subalg\ subalg{\isacharcomma}{\kern0pt}\ measurable{\isacharcomma}{\kern0pt}\ simp\ add{\isacharcolon}{\kern0pt}\ set{\isacharunderscore}{\kern0pt}lebesgue{\isacharunderscore}{\kern0pt}integral{\isacharunderscore}{\kern0pt}def\ integral{\isacharunderscore}{\kern0pt}subalgebra{\isadigit{2}}{\isacharbrackleft}{\kern0pt}OF\ subalg{\isacharbrackright}{\kern0pt}\ sets{\isacharunderscore}{\kern0pt}restr{\isacharunderscore}{\kern0pt}to{\isacharunderscore}{\kern0pt}subalg{\isacharbrackleft}{\kern0pt}OF\ subalg{\isacharbrackright}{\kern0pt}{\isacharparenright}{\kern0pt}\isanewline
\ \ \isacommand{thus}\isamarkupfalse%
\ {\isachardoublequoteopen}AE\ x\ in\ M{\isachardot}{\kern0pt}\ cond{\isacharunderscore}{\kern0pt}exp\ M\ F\ {\isacharparenleft}{\kern0pt}{\isasymlambda}x{\isachardot}{\kern0pt}\ f\ x\ {\isacharasterisk}{\kern0pt}\ g\ x{\isacharparenright}{\kern0pt}\ x\ {\isacharequal}{\kern0pt}\ f\ x\ {\isacharasterisk}{\kern0pt}\ cond{\isacharunderscore}{\kern0pt}exp\ M\ F\ g\ x{\isachardoublequoteclose}\ \isacommand{by}\isamarkupfalse%
\ {\isacharparenleft}{\kern0pt}rule\ AE{\isacharunderscore}{\kern0pt}restr{\isacharunderscore}{\kern0pt}to{\isacharunderscore}{\kern0pt}subalg{\isacharbrackleft}{\kern0pt}OF\ subalg{\isacharbrackright}{\kern0pt}{\isacharparenright}{\kern0pt}\isanewline
\isacommand{qed}\isamarkupfalse%
%
\endisatagproof
{\isafoldproof}%
%
\isadelimproof
\isanewline
%
\endisadelimproof
\isanewline
\isacommand{lemma}\isamarkupfalse%
\ cond{\isacharunderscore}{\kern0pt}exp{\isacharunderscore}{\kern0pt}measurable{\isacharunderscore}{\kern0pt}scaleR{\isacharcolon}{\kern0pt}\isanewline
\ \ \isakeyword{fixes}\ f\ {\isacharcolon}{\kern0pt}{\isacharcolon}{\kern0pt}\ {\isachardoublequoteopen}{\isacharprime}{\kern0pt}a\ {\isasymRightarrow}\ real{\isachardoublequoteclose}\ \isakeyword{and}\ g\ {\isacharcolon}{\kern0pt}{\isacharcolon}{\kern0pt}\ {\isachardoublequoteopen}{\isacharprime}{\kern0pt}a\ {\isasymRightarrow}\ {\isacharprime}{\kern0pt}b\ {\isacharcolon}{\kern0pt}{\isacharcolon}{\kern0pt}\ {\isacharbraceleft}{\kern0pt}second{\isacharunderscore}{\kern0pt}countable{\isacharunderscore}{\kern0pt}topology{\isacharcomma}{\kern0pt}\ banach{\isacharbraceright}{\kern0pt}{\isachardoublequoteclose}\isanewline
\ \ \isakeyword{assumes}\ {\isacharbrackleft}{\kern0pt}measurable{\isacharbrackright}{\kern0pt}{\isacharcolon}{\kern0pt}\ {\isachardoublequoteopen}integrable\ M\ {\isacharparenleft}{\kern0pt}{\isasymlambda}x{\isachardot}{\kern0pt}\ f\ x\ {\isacharasterisk}{\kern0pt}\isactrlsub R\ g\ x{\isacharparenright}{\kern0pt}{\isachardoublequoteclose}\ {\isachardoublequoteopen}integrable\ M\ g{\isachardoublequoteclose}\ {\isachardoublequoteopen}f\ {\isasymin}\ borel{\isacharunderscore}{\kern0pt}measurable\ F{\isachardoublequoteclose}\isanewline
\ \ \isakeyword{shows}\ {\isachardoublequoteopen}integrable\ M\ {\isacharparenleft}{\kern0pt}{\isasymlambda}x{\isachardot}{\kern0pt}\ f\ x\ {\isacharasterisk}{\kern0pt}\isactrlsub R\ cond{\isacharunderscore}{\kern0pt}exp\ M\ F\ g\ x{\isacharparenright}{\kern0pt}{\isachardoublequoteclose}\isanewline
\ \ \ \ \ \ \ \ {\isachardoublequoteopen}AE\ x\ in\ M{\isachardot}{\kern0pt}\ cond{\isacharunderscore}{\kern0pt}exp\ M\ F\ {\isacharparenleft}{\kern0pt}{\isasymlambda}x{\isachardot}{\kern0pt}\ f\ x\ {\isacharasterisk}{\kern0pt}\isactrlsub R\ g\ x{\isacharparenright}{\kern0pt}\ x\ {\isacharequal}{\kern0pt}\ f\ x\ {\isacharasterisk}{\kern0pt}\isactrlsub R\ cond{\isacharunderscore}{\kern0pt}exp\ M\ F\ g\ x{\isachardoublequoteclose}\isanewline
%
\isadelimproof
%
\endisadelimproof
%
\isatagproof
\isacommand{proof}\isamarkupfalse%
\ {\isacharminus}{\kern0pt}\isanewline
\ \ \isacommand{let}\isamarkupfalse%
\ {\isacharquery}{\kern0pt}F\ {\isacharequal}{\kern0pt}\ {\isachardoublequoteopen}restr{\isacharunderscore}{\kern0pt}to{\isacharunderscore}{\kern0pt}subalg\ M\ F{\isachardoublequoteclose}\isanewline
\ \ \isacommand{have}\isamarkupfalse%
\ subalg{\isacharprime}{\kern0pt}{\isacharcolon}{\kern0pt}\ {\isachardoublequoteopen}subalgebra\ M\ {\isacharparenleft}{\kern0pt}restr{\isacharunderscore}{\kern0pt}to{\isacharunderscore}{\kern0pt}subalg\ M\ F{\isacharparenright}{\kern0pt}{\isachardoublequoteclose}\ \isacommand{by}\isamarkupfalse%
\ {\isacharparenleft}{\kern0pt}metis\ sets{\isacharunderscore}{\kern0pt}eq{\isacharunderscore}{\kern0pt}imp{\isacharunderscore}{\kern0pt}space{\isacharunderscore}{\kern0pt}eq\ sets{\isacharunderscore}{\kern0pt}restr{\isacharunderscore}{\kern0pt}to{\isacharunderscore}{\kern0pt}subalg\ subalg\ subalgebra{\isacharunderscore}{\kern0pt}def{\isacharparenright}{\kern0pt}\isanewline
\ \ \isacommand{{\isacharbraceleft}{\kern0pt}}\isamarkupfalse%
\isanewline
\ \ \ \ \isacommand{fix}\isamarkupfalse%
\ z\ \isacommand{assume}\isamarkupfalse%
\ asm{\isacharbrackleft}{\kern0pt}measurable{\isacharbrackright}{\kern0pt}{\isacharcolon}{\kern0pt}\ {\isachardoublequoteopen}integrable\ M\ {\isacharparenleft}{\kern0pt}{\isasymlambda}x{\isachardot}{\kern0pt}\ z\ x\ {\isacharasterisk}{\kern0pt}\isactrlsub R\ g\ x{\isacharparenright}{\kern0pt}{\isachardoublequoteclose}\ {\isachardoublequoteopen}z\ {\isasymin}\ borel{\isacharunderscore}{\kern0pt}measurable\ {\isacharquery}{\kern0pt}F{\isachardoublequoteclose}\isanewline
\ \ \ \ \isacommand{hence}\isamarkupfalse%
\ asm{\isacharprime}{\kern0pt}{\isacharbrackleft}{\kern0pt}measurable{\isacharbrackright}{\kern0pt}{\isacharcolon}{\kern0pt}\ {\isachardoublequoteopen}z\ {\isasymin}\ borel{\isacharunderscore}{\kern0pt}measurable\ F{\isachardoublequoteclose}\ \isacommand{using}\isamarkupfalse%
\ measurable{\isacharunderscore}{\kern0pt}in{\isacharunderscore}{\kern0pt}subalg{\isacharprime}{\kern0pt}\ subalg\ \isacommand{by}\isamarkupfalse%
\ blast\isanewline
\ \ \ \ \isacommand{have}\isamarkupfalse%
\ {\isachardoublequoteopen}integrable\ M\ {\isacharparenleft}{\kern0pt}{\isasymlambda}x{\isachardot}{\kern0pt}\ z\ x\ {\isacharasterisk}{\kern0pt}\isactrlsub R\ cond{\isacharunderscore}{\kern0pt}exp\ M\ F\ g\ x{\isacharparenright}{\kern0pt}{\isachardoublequoteclose}\ {\isachardoublequoteopen}LINT\ x{\isacharbar}{\kern0pt}M{\isachardot}{\kern0pt}\ z\ x\ {\isacharasterisk}{\kern0pt}\isactrlsub R\ g\ x\ {\isacharequal}{\kern0pt}\ LINT\ x{\isacharbar}{\kern0pt}M{\isachardot}{\kern0pt}\ z\ x\ {\isacharasterisk}{\kern0pt}\isactrlsub R\ cond{\isacharunderscore}{\kern0pt}exp\ M\ F\ g\ x{\isachardoublequoteclose}\isanewline
\ \ \ \ \isacommand{proof}\isamarkupfalse%
\ {\isacharminus}{\kern0pt}\isanewline
\ \ \ \ \ \ \isacommand{obtain}\isamarkupfalse%
\ s\ \isakeyword{where}\ s{\isacharunderscore}{\kern0pt}is{\isacharcolon}{\kern0pt}\ {\isachardoublequoteopen}{\isasymAnd}i{\isachardot}{\kern0pt}\ simple{\isacharunderscore}{\kern0pt}function\ {\isacharquery}{\kern0pt}F\ {\isacharparenleft}{\kern0pt}s\ i{\isacharparenright}{\kern0pt}{\isachardoublequoteclose}\ {\isachardoublequoteopen}{\isasymAnd}x{\isachardot}{\kern0pt}\ x\ {\isasymin}\ space\ {\isacharquery}{\kern0pt}F\ {\isasymLongrightarrow}\ {\isacharparenleft}{\kern0pt}{\isasymlambda}i{\isachardot}{\kern0pt}\ s\ i\ x{\isacharparenright}{\kern0pt}\ {\isasymlonglonglongrightarrow}\ z\ x{\isachardoublequoteclose}\ {\isachardoublequoteopen}{\isasymAnd}i\ x{\isachardot}{\kern0pt}\ x\ {\isasymin}\ space\ {\isacharquery}{\kern0pt}F\ {\isasymLongrightarrow}\ norm\ {\isacharparenleft}{\kern0pt}s\ i\ x{\isacharparenright}{\kern0pt}\ {\isasymle}\ {\isadigit{2}}\ {\isacharasterisk}{\kern0pt}\ norm\ {\isacharparenleft}{\kern0pt}z\ x{\isacharparenright}{\kern0pt}{\isachardoublequoteclose}\ \isacommand{using}\isamarkupfalse%
\ borel{\isacharunderscore}{\kern0pt}measurable{\isacharunderscore}{\kern0pt}implies{\isacharunderscore}{\kern0pt}sequence{\isacharunderscore}{\kern0pt}metric{\isacharbrackleft}{\kern0pt}OF\ asm{\isacharparenleft}{\kern0pt}{\isadigit{2}}{\isacharparenright}{\kern0pt}{\isacharcomma}{\kern0pt}\ of\ {\isadigit{0}}{\isacharbrackright}{\kern0pt}\ \isacommand{by}\isamarkupfalse%
\ force\isanewline
\isanewline
\ \ \ \ \ \ \isanewline
\isanewline
\ \ \ \ \ \ \isacommand{have}\isamarkupfalse%
\ s{\isacharunderscore}{\kern0pt}scaleR{\isacharunderscore}{\kern0pt}g{\isacharunderscore}{\kern0pt}tendsto{\isacharcolon}{\kern0pt}\ {\isachardoublequoteopen}AE\ x\ in\ M{\isachardot}{\kern0pt}\ {\isacharparenleft}{\kern0pt}{\isasymlambda}i{\isachardot}{\kern0pt}\ s\ i\ x\ {\isacharasterisk}{\kern0pt}\isactrlsub R\ g\ x{\isacharparenright}{\kern0pt}\ {\isasymlonglonglongrightarrow}\ z\ x\ {\isacharasterisk}{\kern0pt}\isactrlsub R\ g\ x{\isachardoublequoteclose}\ \isacommand{using}\isamarkupfalse%
\ s{\isacharunderscore}{\kern0pt}is{\isacharparenleft}{\kern0pt}{\isadigit{2}}{\isacharparenright}{\kern0pt}\ \isacommand{by}\isamarkupfalse%
\ {\isacharparenleft}{\kern0pt}simp\ add{\isacharcolon}{\kern0pt}\ space{\isacharunderscore}{\kern0pt}restr{\isacharunderscore}{\kern0pt}to{\isacharunderscore}{\kern0pt}subalg\ tendsto{\isacharunderscore}{\kern0pt}scaleR{\isacharparenright}{\kern0pt}\isanewline
\ \ \ \ \ \ \isacommand{have}\isamarkupfalse%
\ s{\isacharunderscore}{\kern0pt}scaleR{\isacharunderscore}{\kern0pt}cond{\isacharunderscore}{\kern0pt}exp{\isacharunderscore}{\kern0pt}g{\isacharunderscore}{\kern0pt}tendsto{\isacharcolon}{\kern0pt}\ {\isachardoublequoteopen}AE\ x\ in\ {\isacharquery}{\kern0pt}F{\isachardot}{\kern0pt}\ {\isacharparenleft}{\kern0pt}{\isasymlambda}i{\isachardot}{\kern0pt}\ s\ i\ x\ {\isacharasterisk}{\kern0pt}\isactrlsub R\ cond{\isacharunderscore}{\kern0pt}exp\ M\ F\ g\ x{\isacharparenright}{\kern0pt}\ {\isasymlonglonglongrightarrow}\ z\ x\ {\isacharasterisk}{\kern0pt}\isactrlsub R\ cond{\isacharunderscore}{\kern0pt}exp\ M\ F\ g\ x{\isachardoublequoteclose}\ \isacommand{using}\isamarkupfalse%
\ s{\isacharunderscore}{\kern0pt}is{\isacharparenleft}{\kern0pt}{\isadigit{2}}{\isacharparenright}{\kern0pt}\ \isacommand{by}\isamarkupfalse%
\ {\isacharparenleft}{\kern0pt}simp\ add{\isacharcolon}{\kern0pt}\ tendsto{\isacharunderscore}{\kern0pt}scaleR{\isacharparenright}{\kern0pt}\isanewline
\isanewline
\ \ \ \ \ \ \isacommand{have}\isamarkupfalse%
\ s{\isacharunderscore}{\kern0pt}scaleR{\isacharunderscore}{\kern0pt}g{\isacharunderscore}{\kern0pt}meas{\isacharcolon}{\kern0pt}\ {\isachardoublequoteopen}{\isacharparenleft}{\kern0pt}{\isasymlambda}x{\isachardot}{\kern0pt}\ s\ i\ x\ {\isacharasterisk}{\kern0pt}\isactrlsub R\ g\ x{\isacharparenright}{\kern0pt}\ {\isasymin}\ borel{\isacharunderscore}{\kern0pt}measurable\ M{\isachardoublequoteclose}\ \isakeyword{for}\ i\ \isacommand{using}\isamarkupfalse%
\ s{\isacharunderscore}{\kern0pt}is{\isacharparenleft}{\kern0pt}{\isadigit{1}}{\isacharparenright}{\kern0pt}{\isacharbrackleft}{\kern0pt}THEN\ borel{\isacharunderscore}{\kern0pt}measurable{\isacharunderscore}{\kern0pt}simple{\isacharunderscore}{\kern0pt}function{\isacharcomma}{\kern0pt}\ THEN\ subalg{\isacharprime}{\kern0pt}{\isacharbrackleft}{\kern0pt}THEN\ measurable{\isacharunderscore}{\kern0pt}from{\isacharunderscore}{\kern0pt}subalg{\isacharbrackright}{\kern0pt}{\isacharbrackright}{\kern0pt}\ \isacommand{by}\isamarkupfalse%
\ simp\isanewline
\ \ \ \ \ \ \isacommand{have}\isamarkupfalse%
\ s{\isacharunderscore}{\kern0pt}scaleR{\isacharunderscore}{\kern0pt}cond{\isacharunderscore}{\kern0pt}exp{\isacharunderscore}{\kern0pt}g{\isacharunderscore}{\kern0pt}meas{\isacharcolon}{\kern0pt}\ {\isachardoublequoteopen}{\isacharparenleft}{\kern0pt}{\isasymlambda}x{\isachardot}{\kern0pt}\ s\ i\ x\ {\isacharasterisk}{\kern0pt}\isactrlsub R\ cond{\isacharunderscore}{\kern0pt}exp\ M\ F\ g\ x{\isacharparenright}{\kern0pt}\ {\isasymin}\ borel{\isacharunderscore}{\kern0pt}measurable\ {\isacharquery}{\kern0pt}F{\isachardoublequoteclose}\ \isakeyword{for}\ i\ \isacommand{using}\isamarkupfalse%
\ s{\isacharunderscore}{\kern0pt}is{\isacharparenleft}{\kern0pt}{\isadigit{1}}{\isacharparenright}{\kern0pt}{\isacharbrackleft}{\kern0pt}THEN\ borel{\isacharunderscore}{\kern0pt}measurable{\isacharunderscore}{\kern0pt}simple{\isacharunderscore}{\kern0pt}function{\isacharbrackright}{\kern0pt}\ measurable{\isacharunderscore}{\kern0pt}in{\isacharunderscore}{\kern0pt}subalg{\isacharbrackleft}{\kern0pt}OF\ subalg\ borel{\isacharunderscore}{\kern0pt}measurable{\isacharunderscore}{\kern0pt}cond{\isacharunderscore}{\kern0pt}exp{\isacharbrackright}{\kern0pt}\ \isacommand{by}\isamarkupfalse%
\ {\isacharparenleft}{\kern0pt}fastforce\ intro{\isacharcolon}{\kern0pt}\ borel{\isacharunderscore}{\kern0pt}measurable{\isacharunderscore}{\kern0pt}scaleR{\isacharparenright}{\kern0pt}\isanewline
\isanewline
\ \ \ \ \ \ \isacommand{have}\isamarkupfalse%
\ s{\isacharunderscore}{\kern0pt}scaleR{\isacharunderscore}{\kern0pt}g{\isacharunderscore}{\kern0pt}AE{\isacharunderscore}{\kern0pt}bdd{\isacharcolon}{\kern0pt}\ {\isachardoublequoteopen}AE\ x\ in\ M{\isachardot}{\kern0pt}\ norm\ {\isacharparenleft}{\kern0pt}s\ i\ x\ {\isacharasterisk}{\kern0pt}\isactrlsub R\ g\ x{\isacharparenright}{\kern0pt}\ {\isasymle}\ {\isadigit{2}}\ {\isacharasterisk}{\kern0pt}\ norm\ {\isacharparenleft}{\kern0pt}z\ x\ {\isacharasterisk}{\kern0pt}\isactrlsub R\ g\ x{\isacharparenright}{\kern0pt}{\isachardoublequoteclose}\ \isakeyword{for}\ i\ \isacommand{using}\isamarkupfalse%
\ s{\isacharunderscore}{\kern0pt}is{\isacharparenleft}{\kern0pt}{\isadigit{3}}{\isacharparenright}{\kern0pt}\ \isacommand{by}\isamarkupfalse%
\ {\isacharparenleft}{\kern0pt}fastforce\ simp\ add{\isacharcolon}{\kern0pt}\ space{\isacharunderscore}{\kern0pt}restr{\isacharunderscore}{\kern0pt}to{\isacharunderscore}{\kern0pt}subalg\ mult{\isachardot}{\kern0pt}assoc{\isacharbrackleft}{\kern0pt}symmetric{\isacharbrackright}{\kern0pt}\ mult{\isacharunderscore}{\kern0pt}right{\isacharunderscore}{\kern0pt}mono{\isacharparenright}{\kern0pt}\isanewline
\ \ \ \ \ \ \isacommand{{\isacharbraceleft}{\kern0pt}}\isamarkupfalse%
\isanewline
\ \ \ \ \ \ \ \ \isacommand{fix}\isamarkupfalse%
\ i\isanewline
\ \ \ \ \ \ \ \ \isacommand{have}\isamarkupfalse%
\ asm{\isacharcolon}{\kern0pt}\ {\isachardoublequoteopen}integrable\ M\ {\isacharparenleft}{\kern0pt}{\isasymlambda}x{\isachardot}{\kern0pt}\ norm\ {\isacharparenleft}{\kern0pt}z\ x{\isacharparenright}{\kern0pt}\ {\isacharasterisk}{\kern0pt}\ norm\ {\isacharparenleft}{\kern0pt}g\ x{\isacharparenright}{\kern0pt}{\isacharparenright}{\kern0pt}{\isachardoublequoteclose}\ \isacommand{using}\isamarkupfalse%
\ asm{\isacharparenleft}{\kern0pt}{\isadigit{1}}{\isacharparenright}{\kern0pt}{\isacharbrackleft}{\kern0pt}THEN\ integrable{\isacharunderscore}{\kern0pt}norm{\isacharbrackright}{\kern0pt}\ \isacommand{by}\isamarkupfalse%
\ simp\isanewline
\ \ \ \ \ \ \ \ \isacommand{have}\isamarkupfalse%
\ {\isachardoublequoteopen}AE\ x\ in\ {\isacharquery}{\kern0pt}F{\isachardot}{\kern0pt}\ norm\ {\isacharparenleft}{\kern0pt}s\ i\ x\ {\isacharasterisk}{\kern0pt}\isactrlsub R\ cond{\isacharunderscore}{\kern0pt}exp\ M\ F\ g\ x{\isacharparenright}{\kern0pt}\ {\isasymle}\ {\isadigit{2}}\ {\isacharasterisk}{\kern0pt}\ norm\ {\isacharparenleft}{\kern0pt}z\ x{\isacharparenright}{\kern0pt}\ {\isacharasterisk}{\kern0pt}\ norm\ {\isacharparenleft}{\kern0pt}cond{\isacharunderscore}{\kern0pt}exp\ M\ F\ g\ x{\isacharparenright}{\kern0pt}{\isachardoublequoteclose}\ \isacommand{using}\isamarkupfalse%
\ s{\isacharunderscore}{\kern0pt}is{\isacharparenleft}{\kern0pt}{\isadigit{3}}{\isacharparenright}{\kern0pt}\ \isacommand{by}\isamarkupfalse%
\ {\isacharparenleft}{\kern0pt}fastforce\ simp\ add{\isacharcolon}{\kern0pt}\ mult{\isacharunderscore}{\kern0pt}mono{\isacharparenright}{\kern0pt}\isanewline
\ \ \ \ \ \ \ \ \isacommand{moreover}\isamarkupfalse%
\ \isacommand{have}\isamarkupfalse%
\ {\isachardoublequoteopen}AE\ x\ in\ {\isacharquery}{\kern0pt}F{\isachardot}{\kern0pt}\ norm\ {\isacharparenleft}{\kern0pt}z\ x{\isacharparenright}{\kern0pt}\ {\isacharasterisk}{\kern0pt}\ cond{\isacharunderscore}{\kern0pt}exp\ M\ F\ {\isacharparenleft}{\kern0pt}{\isasymlambda}x{\isachardot}{\kern0pt}\ norm\ {\isacharparenleft}{\kern0pt}g\ x{\isacharparenright}{\kern0pt}{\isacharparenright}{\kern0pt}\ x\ {\isacharequal}{\kern0pt}\ cond{\isacharunderscore}{\kern0pt}exp\ M\ F\ {\isacharparenleft}{\kern0pt}{\isasymlambda}x{\isachardot}{\kern0pt}\ norm\ {\isacharparenleft}{\kern0pt}z\ x{\isacharparenright}{\kern0pt}\ {\isacharasterisk}{\kern0pt}\ norm\ {\isacharparenleft}{\kern0pt}g\ x{\isacharparenright}{\kern0pt}{\isacharparenright}{\kern0pt}\ x{\isachardoublequoteclose}\ \isacommand{by}\isamarkupfalse%
\ {\isacharparenleft}{\kern0pt}rule\ cond{\isacharunderscore}{\kern0pt}exp{\isacharunderscore}{\kern0pt}measurable{\isacharunderscore}{\kern0pt}mult{\isacharparenleft}{\kern0pt}{\isadigit{2}}{\isacharparenright}{\kern0pt}{\isacharbrackleft}{\kern0pt}THEN\ AE{\isacharunderscore}{\kern0pt}symmetric{\isacharcomma}{\kern0pt}\ OF\ asm\ integrable{\isacharunderscore}{\kern0pt}norm{\isacharcomma}{\kern0pt}\ OF\ assms{\isacharparenleft}{\kern0pt}{\isadigit{2}}{\isacharparenright}{\kern0pt}{\isacharcomma}{\kern0pt}\ THEN\ AE{\isacharunderscore}{\kern0pt}restr{\isacharunderscore}{\kern0pt}to{\isacharunderscore}{\kern0pt}subalg{\isadigit{2}}{\isacharbrackleft}{\kern0pt}OF\ subalg{\isacharbrackright}{\kern0pt}{\isacharbrackright}{\kern0pt}{\isacharcomma}{\kern0pt}\ auto{\isacharparenright}{\kern0pt}\isanewline
\ \ \ \ \ \ \ \ \isacommand{ultimately}\isamarkupfalse%
\ \isacommand{have}\isamarkupfalse%
\ {\isachardoublequoteopen}AE\ x\ in\ {\isacharquery}{\kern0pt}F{\isachardot}{\kern0pt}\ norm\ {\isacharparenleft}{\kern0pt}s\ i\ x\ {\isacharasterisk}{\kern0pt}\isactrlsub R\ cond{\isacharunderscore}{\kern0pt}exp\ M\ F\ g\ x{\isacharparenright}{\kern0pt}\ {\isasymle}\ {\isadigit{2}}\ {\isacharasterisk}{\kern0pt}\ cond{\isacharunderscore}{\kern0pt}exp\ M\ F\ {\isacharparenleft}{\kern0pt}{\isasymlambda}x{\isachardot}{\kern0pt}\ norm\ {\isacharparenleft}{\kern0pt}z\ x\ {\isacharasterisk}{\kern0pt}\isactrlsub R\ g\ x{\isacharparenright}{\kern0pt}{\isacharparenright}{\kern0pt}\ x{\isachardoublequoteclose}\ \isacommand{using}\isamarkupfalse%
\ cond{\isacharunderscore}{\kern0pt}exp{\isacharunderscore}{\kern0pt}contraction{\isacharbrackleft}{\kern0pt}OF\ assms{\isacharparenleft}{\kern0pt}{\isadigit{2}}{\isacharparenright}{\kern0pt}{\isacharcomma}{\kern0pt}\ THEN\ AE{\isacharunderscore}{\kern0pt}restr{\isacharunderscore}{\kern0pt}to{\isacharunderscore}{\kern0pt}subalg{\isadigit{2}}{\isacharbrackleft}{\kern0pt}OF\ subalg{\isacharbrackright}{\kern0pt}{\isacharbrackright}{\kern0pt}\ order{\isacharunderscore}{\kern0pt}trans{\isacharbrackleft}{\kern0pt}OF\ {\isacharunderscore}{\kern0pt}\ mult{\isacharunderscore}{\kern0pt}mono{\isacharbrackright}{\kern0pt}\ \isacommand{by}\isamarkupfalse%
\ fastforce\isanewline
\ \ \ \ \ \ \isacommand{{\isacharbraceright}{\kern0pt}}\isamarkupfalse%
\isanewline
\ \ \ \ \ \ \isacommand{note}\isamarkupfalse%
\ s{\isacharunderscore}{\kern0pt}scaleR{\isacharunderscore}{\kern0pt}cond{\isacharunderscore}{\kern0pt}exp{\isacharunderscore}{\kern0pt}g{\isacharunderscore}{\kern0pt}AE{\isacharunderscore}{\kern0pt}bdd\ {\isacharequal}{\kern0pt}\ this\isanewline
\isanewline
\ \ \ \ \ \ \isanewline
\isanewline
\ \ \ \ \ \ \isacommand{{\isacharbraceleft}{\kern0pt}}\isamarkupfalse%
\isanewline
\ \ \ \ \ \ \ \ \isacommand{fix}\isamarkupfalse%
\ i\isanewline
\ \ \ \ \ \ \ \ \isacommand{have}\isamarkupfalse%
\ s{\isacharunderscore}{\kern0pt}meas{\isacharunderscore}{\kern0pt}M{\isacharbrackleft}{\kern0pt}measurable{\isacharbrackright}{\kern0pt}{\isacharcolon}{\kern0pt}\ {\isachardoublequoteopen}s\ i\ {\isasymin}\ borel{\isacharunderscore}{\kern0pt}measurable\ M{\isachardoublequoteclose}\ \isacommand{by}\isamarkupfalse%
\ {\isacharparenleft}{\kern0pt}meson\ borel{\isacharunderscore}{\kern0pt}measurable{\isacharunderscore}{\kern0pt}simple{\isacharunderscore}{\kern0pt}function\ measurable{\isacharunderscore}{\kern0pt}from{\isacharunderscore}{\kern0pt}subalg\ s{\isacharunderscore}{\kern0pt}is{\isacharparenleft}{\kern0pt}{\isadigit{1}}{\isacharparenright}{\kern0pt}\ subalg{\isacharprime}{\kern0pt}{\isacharparenright}{\kern0pt}\isanewline
\ \ \ \ \ \ \ \ \isacommand{have}\isamarkupfalse%
\ s{\isacharunderscore}{\kern0pt}meas{\isacharunderscore}{\kern0pt}F{\isacharbrackleft}{\kern0pt}measurable{\isacharbrackright}{\kern0pt}{\isacharcolon}{\kern0pt}\ {\isachardoublequoteopen}s\ i\ {\isasymin}\ borel{\isacharunderscore}{\kern0pt}measurable\ F{\isachardoublequoteclose}\ \isacommand{by}\isamarkupfalse%
\ {\isacharparenleft}{\kern0pt}meson\ borel{\isacharunderscore}{\kern0pt}measurable{\isacharunderscore}{\kern0pt}simple{\isacharunderscore}{\kern0pt}function\ measurable{\isacharunderscore}{\kern0pt}in{\isacharunderscore}{\kern0pt}subalg{\isacharprime}{\kern0pt}\ s{\isacharunderscore}{\kern0pt}is{\isacharparenleft}{\kern0pt}{\isadigit{1}}{\isacharparenright}{\kern0pt}\ subalg{\isacharparenright}{\kern0pt}\isanewline
\isanewline
\ \ \ \ \ \ \ \ \isacommand{have}\isamarkupfalse%
\ s{\isacharunderscore}{\kern0pt}scaleR{\isacharunderscore}{\kern0pt}eq{\isacharcolon}{\kern0pt}\ {\isachardoublequoteopen}s\ i\ x\ {\isacharasterisk}{\kern0pt}\isactrlsub R\ h\ x\ {\isacharequal}{\kern0pt}\ {\isacharparenleft}{\kern0pt}{\isasymSum}y{\isasymin}s\ i\ {\isacharbackquote}{\kern0pt}\ space\ M{\isachardot}{\kern0pt}\ {\isacharparenleft}{\kern0pt}indicator\ {\isacharparenleft}{\kern0pt}s\ i\ {\isacharminus}{\kern0pt}{\isacharbackquote}{\kern0pt}\ {\isacharbraceleft}{\kern0pt}y{\isacharbraceright}{\kern0pt}\ {\isasyminter}\ space\ M{\isacharparenright}{\kern0pt}\ x\ {\isacharasterisk}{\kern0pt}\isactrlsub R\ y{\isacharparenright}{\kern0pt}\ {\isacharasterisk}{\kern0pt}\isactrlsub R\ h\ x{\isacharparenright}{\kern0pt}{\isachardoublequoteclose}\ \isakeyword{if}\ {\isachardoublequoteopen}x\ {\isasymin}\ space\ M{\isachardoublequoteclose}\ \isakeyword{for}\ x\ \isakeyword{and}\ h\ {\isacharcolon}{\kern0pt}{\isacharcolon}{\kern0pt}\ {\isachardoublequoteopen}{\isacharprime}{\kern0pt}a\ {\isasymRightarrow}\ {\isacharprime}{\kern0pt}b{\isachardoublequoteclose}\ \isacommand{using}\isamarkupfalse%
\ simple{\isacharunderscore}{\kern0pt}function{\isacharunderscore}{\kern0pt}indicator{\isacharunderscore}{\kern0pt}representation{\isacharbrackleft}{\kern0pt}OF\ s{\isacharunderscore}{\kern0pt}is{\isacharparenleft}{\kern0pt}{\isadigit{1}}{\isacharparenright}{\kern0pt}{\isacharcomma}{\kern0pt}\ of\ x\ i{\isacharbrackright}{\kern0pt}\ that\ \isacommand{unfolding}\isamarkupfalse%
\ space{\isacharunderscore}{\kern0pt}restr{\isacharunderscore}{\kern0pt}to{\isacharunderscore}{\kern0pt}subalg\ scaleR{\isacharunderscore}{\kern0pt}left{\isachardot}{\kern0pt}sum{\isacharbrackleft}{\kern0pt}of\ {\isacharunderscore}{\kern0pt}\ {\isacharunderscore}{\kern0pt}\ {\isachardoublequoteopen}h\ x{\isachardoublequoteclose}{\isacharcomma}{\kern0pt}\ symmetric{\isacharbrackright}{\kern0pt}\ \isacommand{by}\isamarkupfalse%
\ presburger\isanewline
\ \ \ \ \ \ \ \ \isanewline
\ \ \ \ \ \ \ \ \isacommand{have}\isamarkupfalse%
\ {\isachardoublequoteopen}LINT\ x{\isacharbar}{\kern0pt}M{\isachardot}{\kern0pt}\ s\ i\ x\ {\isacharasterisk}{\kern0pt}\isactrlsub R\ g\ x\ {\isacharequal}{\kern0pt}\ LINT\ x{\isacharbar}{\kern0pt}M{\isachardot}{\kern0pt}\ {\isacharparenleft}{\kern0pt}{\isasymSum}y{\isasymin}s\ i\ {\isacharbackquote}{\kern0pt}\ space\ M{\isachardot}{\kern0pt}\ indicator\ {\isacharparenleft}{\kern0pt}s\ i\ {\isacharminus}{\kern0pt}{\isacharbackquote}{\kern0pt}\ {\isacharbraceleft}{\kern0pt}y{\isacharbraceright}{\kern0pt}\ {\isasyminter}\ space\ M{\isacharparenright}{\kern0pt}\ x\ {\isacharasterisk}{\kern0pt}\isactrlsub R\ y\ {\isacharasterisk}{\kern0pt}\isactrlsub R\ g\ x{\isacharparenright}{\kern0pt}{\isachardoublequoteclose}\ \isacommand{using}\isamarkupfalse%
\ s{\isacharunderscore}{\kern0pt}scaleR{\isacharunderscore}{\kern0pt}eq\ \isacommand{by}\isamarkupfalse%
\ {\isacharparenleft}{\kern0pt}intro\ Bochner{\isacharunderscore}{\kern0pt}Integration{\isachardot}{\kern0pt}integral{\isacharunderscore}{\kern0pt}cong{\isacharparenright}{\kern0pt}\ auto\isanewline
\ \ \ \ \ \ \ \ \isacommand{also}\isamarkupfalse%
\ \isacommand{have}\isamarkupfalse%
\ {\isachardoublequoteopen}{\isachardot}{\kern0pt}{\isachardot}{\kern0pt}{\isachardot}{\kern0pt}\ {\isacharequal}{\kern0pt}\ {\isacharparenleft}{\kern0pt}{\isasymSum}y{\isasymin}s\ i\ {\isacharbackquote}{\kern0pt}\ space\ M{\isachardot}{\kern0pt}\ LINT\ x{\isacharbar}{\kern0pt}M{\isachardot}{\kern0pt}\ indicator\ {\isacharparenleft}{\kern0pt}s\ i\ {\isacharminus}{\kern0pt}{\isacharbackquote}{\kern0pt}\ {\isacharbraceleft}{\kern0pt}y{\isacharbraceright}{\kern0pt}\ {\isasyminter}\ space\ M{\isacharparenright}{\kern0pt}\ x\ {\isacharasterisk}{\kern0pt}\isactrlsub R\ y\ {\isacharasterisk}{\kern0pt}\isactrlsub R\ g\ x{\isacharparenright}{\kern0pt}{\isachardoublequoteclose}\ \isacommand{by}\isamarkupfalse%
\ {\isacharparenleft}{\kern0pt}intro\ Bochner{\isacharunderscore}{\kern0pt}Integration{\isachardot}{\kern0pt}integral{\isacharunderscore}{\kern0pt}sum\ integrable{\isacharunderscore}{\kern0pt}mult{\isacharunderscore}{\kern0pt}indicator{\isacharbrackleft}{\kern0pt}OF\ {\isacharunderscore}{\kern0pt}\ integrable{\isacharunderscore}{\kern0pt}scaleR{\isacharunderscore}{\kern0pt}right{\isacharbrackright}{\kern0pt}\ assms{\isacharparenleft}{\kern0pt}{\isadigit{2}}{\isacharparenright}{\kern0pt}{\isacharparenright}{\kern0pt}\ simp\isanewline
\ \ \ \ \ \ \ \ \isacommand{also}\isamarkupfalse%
\ \isacommand{have}\isamarkupfalse%
\ {\isachardoublequoteopen}{\isachardot}{\kern0pt}{\isachardot}{\kern0pt}{\isachardot}{\kern0pt}\ {\isacharequal}{\kern0pt}\ {\isacharparenleft}{\kern0pt}{\isasymSum}y{\isasymin}s\ i\ {\isacharbackquote}{\kern0pt}\ space\ M{\isachardot}{\kern0pt}\ \ y\ {\isacharasterisk}{\kern0pt}\isactrlsub R\ set{\isacharunderscore}{\kern0pt}lebesgue{\isacharunderscore}{\kern0pt}integral\ M\ {\isacharparenleft}{\kern0pt}s\ i\ {\isacharminus}{\kern0pt}{\isacharbackquote}{\kern0pt}\ {\isacharbraceleft}{\kern0pt}y{\isacharbraceright}{\kern0pt}\ {\isasyminter}\ space\ M{\isacharparenright}{\kern0pt}\ g{\isacharparenright}{\kern0pt}{\isachardoublequoteclose}\ \isacommand{by}\isamarkupfalse%
\ {\isacharparenleft}{\kern0pt}simp\ only{\isacharcolon}{\kern0pt}\ set{\isacharunderscore}{\kern0pt}lebesgue{\isacharunderscore}{\kern0pt}integral{\isacharunderscore}{\kern0pt}def{\isacharbrackleft}{\kern0pt}symmetric{\isacharbrackright}{\kern0pt}{\isacharparenright}{\kern0pt}\ simp\isanewline
\ \ \ \ \ \ \ \ \isacommand{also}\isamarkupfalse%
\ \isacommand{have}\isamarkupfalse%
\ {\isachardoublequoteopen}{\isachardot}{\kern0pt}{\isachardot}{\kern0pt}{\isachardot}{\kern0pt}\ {\isacharequal}{\kern0pt}\ {\isacharparenleft}{\kern0pt}{\isasymSum}y{\isasymin}s\ i\ {\isacharbackquote}{\kern0pt}\ space\ M{\isachardot}{\kern0pt}\ \ y\ {\isacharasterisk}{\kern0pt}\isactrlsub R\ set{\isacharunderscore}{\kern0pt}lebesgue{\isacharunderscore}{\kern0pt}integral\ M\ {\isacharparenleft}{\kern0pt}s\ i\ {\isacharminus}{\kern0pt}{\isacharbackquote}{\kern0pt}\ {\isacharbraceleft}{\kern0pt}y{\isacharbraceright}{\kern0pt}\ {\isasyminter}\ space\ M{\isacharparenright}{\kern0pt}\ {\isacharparenleft}{\kern0pt}cond{\isacharunderscore}{\kern0pt}exp\ M\ F\ g{\isacharparenright}{\kern0pt}{\isacharparenright}{\kern0pt}{\isachardoublequoteclose}\ \isacommand{using}\isamarkupfalse%
\ assms{\isacharparenleft}{\kern0pt}{\isadigit{2}}{\isacharparenright}{\kern0pt}\ subalg\ borel{\isacharunderscore}{\kern0pt}measurable{\isacharunderscore}{\kern0pt}vimage{\isacharbrackleft}{\kern0pt}OF\ s{\isacharunderscore}{\kern0pt}meas{\isacharunderscore}{\kern0pt}F{\isacharbrackright}{\kern0pt}\ \isacommand{by}\isamarkupfalse%
\ {\isacharparenleft}{\kern0pt}subst\ cond{\isacharunderscore}{\kern0pt}exp{\isacharunderscore}{\kern0pt}set{\isacharunderscore}{\kern0pt}integral{\isacharcomma}{\kern0pt}\ auto\ simp\ add{\isacharcolon}{\kern0pt}\ subalgebra{\isacharunderscore}{\kern0pt}def{\isacharparenright}{\kern0pt}\ \isanewline
\ \ \ \ \ \ \ \ \isacommand{also}\isamarkupfalse%
\ \isacommand{have}\isamarkupfalse%
\ {\isachardoublequoteopen}{\isachardot}{\kern0pt}{\isachardot}{\kern0pt}{\isachardot}{\kern0pt}\ {\isacharequal}{\kern0pt}\ {\isacharparenleft}{\kern0pt}{\isasymSum}y{\isasymin}s\ i\ {\isacharbackquote}{\kern0pt}\ space\ M{\isachardot}{\kern0pt}\ LINT\ x{\isacharbar}{\kern0pt}M{\isachardot}{\kern0pt}\ indicator\ {\isacharparenleft}{\kern0pt}s\ i\ {\isacharminus}{\kern0pt}{\isacharbackquote}{\kern0pt}\ {\isacharbraceleft}{\kern0pt}y{\isacharbraceright}{\kern0pt}\ {\isasyminter}\ space\ M{\isacharparenright}{\kern0pt}\ x\ {\isacharasterisk}{\kern0pt}\isactrlsub R\ y\ {\isacharasterisk}{\kern0pt}\isactrlsub R\ cond{\isacharunderscore}{\kern0pt}exp\ M\ F\ g\ x{\isacharparenright}{\kern0pt}{\isachardoublequoteclose}\ \isacommand{by}\isamarkupfalse%
\ {\isacharparenleft}{\kern0pt}simp\ only{\isacharcolon}{\kern0pt}\ set{\isacharunderscore}{\kern0pt}lebesgue{\isacharunderscore}{\kern0pt}integral{\isacharunderscore}{\kern0pt}def{\isacharbrackleft}{\kern0pt}symmetric{\isacharbrackright}{\kern0pt}{\isacharparenright}{\kern0pt}\ simp\isanewline
\ \ \ \ \ \ \ \ \isacommand{also}\isamarkupfalse%
\ \isacommand{have}\isamarkupfalse%
\ {\isachardoublequoteopen}{\isachardot}{\kern0pt}{\isachardot}{\kern0pt}{\isachardot}{\kern0pt}\ {\isacharequal}{\kern0pt}\ LINT\ x{\isacharbar}{\kern0pt}M{\isachardot}{\kern0pt}\ {\isacharparenleft}{\kern0pt}{\isasymSum}y{\isasymin}s\ i\ {\isacharbackquote}{\kern0pt}\ space\ M{\isachardot}{\kern0pt}\ indicator\ {\isacharparenleft}{\kern0pt}s\ i\ {\isacharminus}{\kern0pt}{\isacharbackquote}{\kern0pt}\ {\isacharbraceleft}{\kern0pt}y{\isacharbraceright}{\kern0pt}\ {\isasyminter}\ space\ M{\isacharparenright}{\kern0pt}\ x\ {\isacharasterisk}{\kern0pt}\isactrlsub R\ y\ {\isacharasterisk}{\kern0pt}\isactrlsub R\ cond{\isacharunderscore}{\kern0pt}exp\ M\ F\ g\ x{\isacharparenright}{\kern0pt}{\isachardoublequoteclose}\ \isacommand{by}\isamarkupfalse%
\ {\isacharparenleft}{\kern0pt}intro\ Bochner{\isacharunderscore}{\kern0pt}Integration{\isachardot}{\kern0pt}integral{\isacharunderscore}{\kern0pt}sum{\isacharbrackleft}{\kern0pt}symmetric{\isacharbrackright}{\kern0pt}\ integrable{\isacharunderscore}{\kern0pt}mult{\isacharunderscore}{\kern0pt}indicator{\isacharbrackleft}{\kern0pt}OF\ {\isacharunderscore}{\kern0pt}\ integrable{\isacharunderscore}{\kern0pt}scaleR{\isacharunderscore}{\kern0pt}right{\isacharbrackright}{\kern0pt}{\isacharparenright}{\kern0pt}\ auto\isanewline
\ \ \ \ \ \ \ \ \isacommand{also}\isamarkupfalse%
\ \isacommand{have}\isamarkupfalse%
\ {\isachardoublequoteopen}{\isachardot}{\kern0pt}{\isachardot}{\kern0pt}{\isachardot}{\kern0pt}\ {\isacharequal}{\kern0pt}\ LINT\ x{\isacharbar}{\kern0pt}M{\isachardot}{\kern0pt}\ s\ i\ x\ {\isacharasterisk}{\kern0pt}\isactrlsub R\ cond{\isacharunderscore}{\kern0pt}exp\ M\ F\ g\ x{\isachardoublequoteclose}\ \isacommand{using}\isamarkupfalse%
\ s{\isacharunderscore}{\kern0pt}scaleR{\isacharunderscore}{\kern0pt}eq\ \isacommand{by}\isamarkupfalse%
\ {\isacharparenleft}{\kern0pt}intro\ Bochner{\isacharunderscore}{\kern0pt}Integration{\isachardot}{\kern0pt}integral{\isacharunderscore}{\kern0pt}cong{\isacharparenright}{\kern0pt}\ auto\isanewline
\ \ \ \ \ \ \ \ \isacommand{finally}\isamarkupfalse%
\ \isacommand{have}\isamarkupfalse%
\ {\isachardoublequoteopen}LINT\ x{\isacharbar}{\kern0pt}M{\isachardot}{\kern0pt}\ s\ i\ x\ {\isacharasterisk}{\kern0pt}\isactrlsub R\ g\ x\ {\isacharequal}{\kern0pt}\ LINT\ x{\isacharbar}{\kern0pt}{\isacharquery}{\kern0pt}F{\isachardot}{\kern0pt}\ s\ i\ x\ {\isacharasterisk}{\kern0pt}\isactrlsub R\ cond{\isacharunderscore}{\kern0pt}exp\ M\ F\ g\ x{\isachardoublequoteclose}\ \isacommand{by}\isamarkupfalse%
\ {\isacharparenleft}{\kern0pt}simp\ add{\isacharcolon}{\kern0pt}\ integral{\isacharunderscore}{\kern0pt}subalgebra{\isadigit{2}}{\isacharbrackleft}{\kern0pt}OF\ subalg{\isacharbrackright}{\kern0pt}{\isacharparenright}{\kern0pt}\isanewline
\ \ \ \ \ \ \isacommand{{\isacharbraceright}{\kern0pt}}\isamarkupfalse%
\isanewline
\ \ \ \ \ \ \isacommand{note}\isamarkupfalse%
\ integral{\isacharunderscore}{\kern0pt}s{\isacharunderscore}{\kern0pt}eq\ {\isacharequal}{\kern0pt}\ this\isanewline
\isanewline
\ \ \ \ \ \ \isanewline
\isanewline
\ \ \ \ \ \ \isacommand{show}\isamarkupfalse%
\ {\isachardoublequoteopen}integrable\ M\ {\isacharparenleft}{\kern0pt}{\isasymlambda}x{\isachardot}{\kern0pt}\ z\ x\ {\isacharasterisk}{\kern0pt}\isactrlsub R\ cond{\isacharunderscore}{\kern0pt}exp\ M\ F\ g\ x{\isacharparenright}{\kern0pt}{\isachardoublequoteclose}\ \isacommand{using}\isamarkupfalse%
\ s{\isacharunderscore}{\kern0pt}scaleR{\isacharunderscore}{\kern0pt}cond{\isacharunderscore}{\kern0pt}exp{\isacharunderscore}{\kern0pt}g{\isacharunderscore}{\kern0pt}meas\ asm{\isacharparenleft}{\kern0pt}{\isadigit{2}}{\isacharparenright}{\kern0pt}\ borel{\isacharunderscore}{\kern0pt}measurable{\isacharunderscore}{\kern0pt}cond{\isacharunderscore}{\kern0pt}exp{\isacharprime}{\kern0pt}\ \isacommand{by}\isamarkupfalse%
\ {\isacharparenleft}{\kern0pt}intro\ integrable{\isacharunderscore}{\kern0pt}from{\isacharunderscore}{\kern0pt}subalg{\isacharbrackleft}{\kern0pt}OF\ subalg{\isacharbrackright}{\kern0pt}\ integrable{\isacharunderscore}{\kern0pt}cond{\isacharunderscore}{\kern0pt}exp\ integrable{\isacharunderscore}{\kern0pt}dominated{\isacharunderscore}{\kern0pt}convergence{\isacharbrackleft}{\kern0pt}OF\ {\isacharunderscore}{\kern0pt}\ {\isacharunderscore}{\kern0pt}\ {\isacharunderscore}{\kern0pt}\ s{\isacharunderscore}{\kern0pt}scaleR{\isacharunderscore}{\kern0pt}cond{\isacharunderscore}{\kern0pt}exp{\isacharunderscore}{\kern0pt}g{\isacharunderscore}{\kern0pt}tendsto\ s{\isacharunderscore}{\kern0pt}scaleR{\isacharunderscore}{\kern0pt}cond{\isacharunderscore}{\kern0pt}exp{\isacharunderscore}{\kern0pt}g{\isacharunderscore}{\kern0pt}AE{\isacharunderscore}{\kern0pt}bdd{\isacharbrackright}{\kern0pt}{\isacharparenright}{\kern0pt}\ {\isacharparenleft}{\kern0pt}auto\ intro{\isacharcolon}{\kern0pt}\ measurable{\isacharunderscore}{\kern0pt}from{\isacharunderscore}{\kern0pt}subalg{\isacharbrackleft}{\kern0pt}OF\ subalg{\isacharbrackright}{\kern0pt}\ integrable{\isacharunderscore}{\kern0pt}in{\isacharunderscore}{\kern0pt}subalg\ measurable{\isacharunderscore}{\kern0pt}in{\isacharunderscore}{\kern0pt}subalg\ subalg{\isacharparenright}{\kern0pt}\isanewline
\ \ \ \ \ \ \ \ \ \isanewline
\ \ \ \ \ \ \isacommand{have}\isamarkupfalse%
\ {\isachardoublequoteopen}{\isacharparenleft}{\kern0pt}{\isasymlambda}i{\isachardot}{\kern0pt}\ LINT\ x{\isacharbar}{\kern0pt}M{\isachardot}{\kern0pt}\ s\ i\ x\ {\isacharasterisk}{\kern0pt}\isactrlsub R\ g\ x{\isacharparenright}{\kern0pt}\ {\isasymlonglonglongrightarrow}\ LINT\ x{\isacharbar}{\kern0pt}M{\isachardot}{\kern0pt}\ z\ x\ {\isacharasterisk}{\kern0pt}\isactrlsub R\ g\ x{\isachardoublequoteclose}\ \isacommand{using}\isamarkupfalse%
\ s{\isacharunderscore}{\kern0pt}scaleR{\isacharunderscore}{\kern0pt}g{\isacharunderscore}{\kern0pt}meas\ asm{\isacharparenleft}{\kern0pt}{\isadigit{1}}{\isacharparenright}{\kern0pt}{\isacharbrackleft}{\kern0pt}THEN\ integrable{\isacharunderscore}{\kern0pt}norm{\isacharbrackright}{\kern0pt}\ asm{\isacharprime}{\kern0pt}\ borel{\isacharunderscore}{\kern0pt}measurable{\isacharunderscore}{\kern0pt}cond{\isacharunderscore}{\kern0pt}exp{\isacharprime}{\kern0pt}\ \isacommand{by}\isamarkupfalse%
\ {\isacharparenleft}{\kern0pt}intro\ integral{\isacharunderscore}{\kern0pt}dominated{\isacharunderscore}{\kern0pt}convergence{\isacharbrackleft}{\kern0pt}OF\ {\isacharunderscore}{\kern0pt}\ {\isacharunderscore}{\kern0pt}\ {\isacharunderscore}{\kern0pt}\ s{\isacharunderscore}{\kern0pt}scaleR{\isacharunderscore}{\kern0pt}g{\isacharunderscore}{\kern0pt}tendsto\ s{\isacharunderscore}{\kern0pt}scaleR{\isacharunderscore}{\kern0pt}g{\isacharunderscore}{\kern0pt}AE{\isacharunderscore}{\kern0pt}bdd{\isacharbrackright}{\kern0pt}{\isacharparenright}{\kern0pt}\ {\isacharparenleft}{\kern0pt}auto\ intro{\isacharcolon}{\kern0pt}\ measurable{\isacharunderscore}{\kern0pt}from{\isacharunderscore}{\kern0pt}subalg{\isacharbrackleft}{\kern0pt}OF\ subalg{\isacharbrackright}{\kern0pt}{\isacharparenright}{\kern0pt}\isanewline
\ \ \ \ \ \ \isacommand{moreover}\isamarkupfalse%
\ \isacommand{have}\isamarkupfalse%
\ {\isachardoublequoteopen}{\isacharparenleft}{\kern0pt}{\isasymlambda}i{\isachardot}{\kern0pt}\ LINT\ x{\isacharbar}{\kern0pt}{\isacharquery}{\kern0pt}F{\isachardot}{\kern0pt}\ s\ i\ x\ {\isacharasterisk}{\kern0pt}\isactrlsub R\ cond{\isacharunderscore}{\kern0pt}exp\ M\ F\ g\ x{\isacharparenright}{\kern0pt}\ {\isasymlonglonglongrightarrow}\ LINT\ x{\isacharbar}{\kern0pt}{\isacharquery}{\kern0pt}F{\isachardot}{\kern0pt}\ z\ x\ {\isacharasterisk}{\kern0pt}\isactrlsub R\ cond{\isacharunderscore}{\kern0pt}exp\ M\ F\ g\ x{\isachardoublequoteclose}\ \isacommand{using}\isamarkupfalse%
\ s{\isacharunderscore}{\kern0pt}scaleR{\isacharunderscore}{\kern0pt}cond{\isacharunderscore}{\kern0pt}exp{\isacharunderscore}{\kern0pt}g{\isacharunderscore}{\kern0pt}meas\ asm{\isacharparenleft}{\kern0pt}{\isadigit{2}}{\isacharparenright}{\kern0pt}\ borel{\isacharunderscore}{\kern0pt}measurable{\isacharunderscore}{\kern0pt}cond{\isacharunderscore}{\kern0pt}exp{\isacharprime}{\kern0pt}\ \isacommand{by}\isamarkupfalse%
\ {\isacharparenleft}{\kern0pt}intro\ integral{\isacharunderscore}{\kern0pt}dominated{\isacharunderscore}{\kern0pt}convergence{\isacharbrackleft}{\kern0pt}OF\ {\isacharunderscore}{\kern0pt}\ {\isacharunderscore}{\kern0pt}\ {\isacharunderscore}{\kern0pt}\ s{\isacharunderscore}{\kern0pt}scaleR{\isacharunderscore}{\kern0pt}cond{\isacharunderscore}{\kern0pt}exp{\isacharunderscore}{\kern0pt}g{\isacharunderscore}{\kern0pt}tendsto\ s{\isacharunderscore}{\kern0pt}scaleR{\isacharunderscore}{\kern0pt}cond{\isacharunderscore}{\kern0pt}exp{\isacharunderscore}{\kern0pt}g{\isacharunderscore}{\kern0pt}AE{\isacharunderscore}{\kern0pt}bdd{\isacharbrackright}{\kern0pt}{\isacharparenright}{\kern0pt}\ {\isacharparenleft}{\kern0pt}auto\ intro{\isacharcolon}{\kern0pt}\ measurable{\isacharunderscore}{\kern0pt}from{\isacharunderscore}{\kern0pt}subalg{\isacharbrackleft}{\kern0pt}OF\ subalg{\isacharbrackright}{\kern0pt}\ integrable{\isacharunderscore}{\kern0pt}in{\isacharunderscore}{\kern0pt}subalg\ measurable{\isacharunderscore}{\kern0pt}in{\isacharunderscore}{\kern0pt}subalg\ subalg{\isacharparenright}{\kern0pt}\isanewline
\ \ \ \ \ \ \isacommand{ultimately}\isamarkupfalse%
\ \isacommand{show}\isamarkupfalse%
\ {\isachardoublequoteopen}LINT\ x{\isacharbar}{\kern0pt}M{\isachardot}{\kern0pt}\ z\ x\ {\isacharasterisk}{\kern0pt}\isactrlsub R\ g\ x\ {\isacharequal}{\kern0pt}\ LINT\ x{\isacharbar}{\kern0pt}M{\isachardot}{\kern0pt}\ z\ x\ {\isacharasterisk}{\kern0pt}\isactrlsub R\ cond{\isacharunderscore}{\kern0pt}exp\ M\ F\ g\ x{\isachardoublequoteclose}\ \isacommand{using}\isamarkupfalse%
\ integral{\isacharunderscore}{\kern0pt}s{\isacharunderscore}{\kern0pt}eq\ \isacommand{using}\isamarkupfalse%
\ subalg\ \isacommand{by}\isamarkupfalse%
\ {\isacharparenleft}{\kern0pt}simp\ add{\isacharcolon}{\kern0pt}\ LIMSEQ{\isacharunderscore}{\kern0pt}unique\ integral{\isacharunderscore}{\kern0pt}subalgebra{\isadigit{2}}{\isacharparenright}{\kern0pt}\isanewline
\ \ \ \ \isacommand{qed}\isamarkupfalse%
\isanewline
\ \ \isacommand{{\isacharbraceright}{\kern0pt}}\isamarkupfalse%
\isanewline
\ \ \isacommand{note}\isamarkupfalse%
\ {\isacharasterisk}{\kern0pt}\ {\isacharequal}{\kern0pt}\ this\isanewline
\isanewline
\ \ \isanewline
\ \ \isanewline
\ \ \isacommand{show}\isamarkupfalse%
\ {\isachardoublequoteopen}integrable\ M\ {\isacharparenleft}{\kern0pt}{\isasymlambda}x{\isachardot}{\kern0pt}\ f\ x\ {\isacharasterisk}{\kern0pt}\isactrlsub R\ cond{\isacharunderscore}{\kern0pt}exp\ M\ F\ g\ x{\isacharparenright}{\kern0pt}{\isachardoublequoteclose}\ \isacommand{using}\isamarkupfalse%
\ {\isacharasterisk}{\kern0pt}\ assms\ measurable{\isacharunderscore}{\kern0pt}in{\isacharunderscore}{\kern0pt}subalg{\isacharbrackleft}{\kern0pt}OF\ subalg{\isacharbrackright}{\kern0pt}\ \isacommand{by}\isamarkupfalse%
\ blast\isanewline
\isanewline
\ \ \isacommand{{\isacharbraceleft}{\kern0pt}}\isamarkupfalse%
\isanewline
\ \ \ \ \isacommand{fix}\isamarkupfalse%
\ A\ \isacommand{assume}\isamarkupfalse%
\ asm{\isacharcolon}{\kern0pt}\ {\isachardoublequoteopen}A\ {\isasymin}\ F{\isachardoublequoteclose}\isanewline
\ \ \ \ \isacommand{hence}\isamarkupfalse%
\ {\isachardoublequoteopen}integrable\ M\ {\isacharparenleft}{\kern0pt}{\isasymlambda}x{\isachardot}{\kern0pt}\ indicat{\isacharunderscore}{\kern0pt}real\ A\ x\ {\isacharasterisk}{\kern0pt}\isactrlsub R\ f\ x\ {\isacharasterisk}{\kern0pt}\isactrlsub R\ g\ x{\isacharparenright}{\kern0pt}{\isachardoublequoteclose}\ \isacommand{using}\isamarkupfalse%
\ subalg\ \isacommand{by}\isamarkupfalse%
\ {\isacharparenleft}{\kern0pt}fastforce\ simp\ add{\isacharcolon}{\kern0pt}\ subalgebra{\isacharunderscore}{\kern0pt}def\ intro{\isacharbang}{\kern0pt}{\isacharcolon}{\kern0pt}\ integrable{\isacharunderscore}{\kern0pt}mult{\isacharunderscore}{\kern0pt}indicator\ assms{\isacharparenleft}{\kern0pt}{\isadigit{1}}{\isacharparenright}{\kern0pt}{\isacharparenright}{\kern0pt}\isanewline
\ \ \ \ \isacommand{hence}\isamarkupfalse%
\ {\isachardoublequoteopen}set{\isacharunderscore}{\kern0pt}lebesgue{\isacharunderscore}{\kern0pt}integral\ M\ A\ {\isacharparenleft}{\kern0pt}{\isasymlambda}x{\isachardot}{\kern0pt}\ f\ x\ {\isacharasterisk}{\kern0pt}\isactrlsub R\ g\ x{\isacharparenright}{\kern0pt}\ {\isacharequal}{\kern0pt}\ set{\isacharunderscore}{\kern0pt}lebesgue{\isacharunderscore}{\kern0pt}integral\ M\ A\ {\isacharparenleft}{\kern0pt}{\isasymlambda}x{\isachardot}{\kern0pt}\ f\ x\ {\isacharasterisk}{\kern0pt}\isactrlsub R\ cond{\isacharunderscore}{\kern0pt}exp\ M\ F\ g\ x{\isacharparenright}{\kern0pt}{\isachardoublequoteclose}\ \isacommand{unfolding}\isamarkupfalse%
\ set{\isacharunderscore}{\kern0pt}lebesgue{\isacharunderscore}{\kern0pt}integral{\isacharunderscore}{\kern0pt}def\ \isacommand{using}\isamarkupfalse%
\ asm\ \isacommand{by}\isamarkupfalse%
\ {\isacharparenleft}{\kern0pt}auto\ intro{\isacharbang}{\kern0pt}{\isacharcolon}{\kern0pt}\ {\isacharasterisk}{\kern0pt}\ measurable{\isacharunderscore}{\kern0pt}in{\isacharunderscore}{\kern0pt}subalg{\isacharbrackleft}{\kern0pt}OF\ subalg{\isacharbrackright}{\kern0pt}{\isacharparenright}{\kern0pt}\isanewline
\ \ \isacommand{{\isacharbraceright}{\kern0pt}}\isamarkupfalse%
\isanewline
\ \ \isacommand{thus}\isamarkupfalse%
\ {\isachardoublequoteopen}AE\ x\ in\ M{\isachardot}{\kern0pt}\ cond{\isacharunderscore}{\kern0pt}exp\ M\ F\ {\isacharparenleft}{\kern0pt}{\isasymlambda}x{\isachardot}{\kern0pt}\ f\ x\ {\isacharasterisk}{\kern0pt}\isactrlsub R\ g\ x{\isacharparenright}{\kern0pt}\ x\ {\isacharequal}{\kern0pt}\ f\ x\ {\isacharasterisk}{\kern0pt}\isactrlsub R\ cond{\isacharunderscore}{\kern0pt}exp\ M\ F\ g\ x{\isachardoublequoteclose}\ \isacommand{using}\isamarkupfalse%
\ borel{\isacharunderscore}{\kern0pt}measurable{\isacharunderscore}{\kern0pt}cond{\isacharunderscore}{\kern0pt}exp\ \isacommand{by}\isamarkupfalse%
\ {\isacharparenleft}{\kern0pt}intro\ cond{\isacharunderscore}{\kern0pt}exp{\isacharunderscore}{\kern0pt}charact{\isacharcomma}{\kern0pt}\ auto\ intro{\isacharbang}{\kern0pt}{\isacharcolon}{\kern0pt}\ {\isacharasterisk}{\kern0pt}\ assms\ measurable{\isacharunderscore}{\kern0pt}in{\isacharunderscore}{\kern0pt}subalg{\isacharbrackleft}{\kern0pt}OF\ subalg{\isacharbrackright}{\kern0pt}{\isacharparenright}{\kern0pt}\isanewline
\isacommand{qed}\isamarkupfalse%
%
\endisatagproof
{\isafoldproof}%
%
\isadelimproof
\isanewline
%
\endisadelimproof
\ \ \isanewline
\isacommand{lemma}\isamarkupfalse%
\ cond{\isacharunderscore}{\kern0pt}exp{\isacharunderscore}{\kern0pt}sum\ {\isacharbrackleft}{\kern0pt}intro{\isacharcomma}{\kern0pt}\ simp{\isacharbrackright}{\kern0pt}{\isacharcolon}{\kern0pt}\isanewline
\ \ \isakeyword{fixes}\ f\ {\isacharcolon}{\kern0pt}{\isacharcolon}{\kern0pt}\ {\isachardoublequoteopen}{\isacharprime}{\kern0pt}t\ {\isasymRightarrow}\ {\isacharprime}{\kern0pt}a\ {\isasymRightarrow}\ {\isacharprime}{\kern0pt}b\ {\isacharcolon}{\kern0pt}{\isacharcolon}{\kern0pt}\ {\isacharbraceleft}{\kern0pt}second{\isacharunderscore}{\kern0pt}countable{\isacharunderscore}{\kern0pt}topology{\isacharcomma}{\kern0pt}banach{\isacharbraceright}{\kern0pt}{\isachardoublequoteclose}\isanewline
\ \ \isakeyword{assumes}\ {\isacharbrackleft}{\kern0pt}measurable{\isacharbrackright}{\kern0pt}{\isacharcolon}{\kern0pt}\ {\isachardoublequoteopen}{\isasymAnd}i{\isachardot}{\kern0pt}\ integrable\ M\ {\isacharparenleft}{\kern0pt}f\ i{\isacharparenright}{\kern0pt}{\isachardoublequoteclose}\isanewline
\ \ \isakeyword{shows}\ {\isachardoublequoteopen}AE\ x\ in\ M{\isachardot}{\kern0pt}\ cond{\isacharunderscore}{\kern0pt}exp\ M\ F\ {\isacharparenleft}{\kern0pt}{\isasymlambda}x{\isachardot}{\kern0pt}\ {\isasymSum}i{\isasymin}I{\isachardot}{\kern0pt}\ f\ i\ x{\isacharparenright}{\kern0pt}\ x\ {\isacharequal}{\kern0pt}\ {\isacharparenleft}{\kern0pt}{\isasymSum}i{\isasymin}I{\isachardot}{\kern0pt}\ cond{\isacharunderscore}{\kern0pt}exp\ M\ F\ {\isacharparenleft}{\kern0pt}f\ i{\isacharparenright}{\kern0pt}\ x{\isacharparenright}{\kern0pt}{\isachardoublequoteclose}\isanewline
%
\isadelimproof
%
\endisadelimproof
%
\isatagproof
\isacommand{proof}\isamarkupfalse%
\ {\isacharparenleft}{\kern0pt}rule\ has{\isacharunderscore}{\kern0pt}cond{\isacharunderscore}{\kern0pt}exp{\isacharunderscore}{\kern0pt}charact{\isacharcomma}{\kern0pt}\ intro\ has{\isacharunderscore}{\kern0pt}cond{\isacharunderscore}{\kern0pt}expI{\isacharprime}{\kern0pt}{\isacharparenright}{\kern0pt}\isanewline
\ \ \isacommand{fix}\isamarkupfalse%
\ A\ \isacommand{assume}\isamarkupfalse%
\ {\isacharbrackleft}{\kern0pt}measurable{\isacharbrackright}{\kern0pt}{\isacharcolon}{\kern0pt}\ {\isachardoublequoteopen}A\ {\isasymin}\ sets\ F{\isachardoublequoteclose}\isanewline
\ \ \isacommand{then}\isamarkupfalse%
\ \isacommand{have}\isamarkupfalse%
\ A{\isacharunderscore}{\kern0pt}meas\ {\isacharbrackleft}{\kern0pt}measurable{\isacharbrackright}{\kern0pt}{\isacharcolon}{\kern0pt}\ {\isachardoublequoteopen}A\ {\isasymin}\ sets\ M{\isachardoublequoteclose}\ \isacommand{by}\isamarkupfalse%
\ {\isacharparenleft}{\kern0pt}meson\ subsetD\ subalg\ subalgebra{\isacharunderscore}{\kern0pt}def{\isacharparenright}{\kern0pt}\isanewline
\isanewline
\ \ \isacommand{have}\isamarkupfalse%
\ {\isachardoublequoteopen}{\isacharparenleft}{\kern0pt}{\isasymintegral}x{\isasymin}A{\isachardot}{\kern0pt}\ {\isacharparenleft}{\kern0pt}{\isasymSum}i{\isasymin}I{\isachardot}{\kern0pt}\ f\ i\ x{\isacharparenright}{\kern0pt}{\isasympartial}M{\isacharparenright}{\kern0pt}\ {\isacharequal}{\kern0pt}\ {\isacharparenleft}{\kern0pt}{\isasymintegral}x{\isachardot}{\kern0pt}\ {\isacharparenleft}{\kern0pt}{\isasymSum}i{\isasymin}I{\isachardot}{\kern0pt}\ indicator\ A\ x\ {\isacharasterisk}{\kern0pt}\isactrlsub R\ f\ i\ x{\isacharparenright}{\kern0pt}{\isasympartial}M{\isacharparenright}{\kern0pt}{\isachardoublequoteclose}\ \isacommand{unfolding}\isamarkupfalse%
\ set{\isacharunderscore}{\kern0pt}lebesgue{\isacharunderscore}{\kern0pt}integral{\isacharunderscore}{\kern0pt}def\ \isacommand{by}\isamarkupfalse%
\ {\isacharparenleft}{\kern0pt}simp\ add{\isacharcolon}{\kern0pt}\ scaleR{\isacharunderscore}{\kern0pt}sum{\isacharunderscore}{\kern0pt}right{\isacharparenright}{\kern0pt}\isanewline
\ \ \isacommand{also}\isamarkupfalse%
\ \isacommand{have}\isamarkupfalse%
\ {\isachardoublequoteopen}{\isachardot}{\kern0pt}{\isachardot}{\kern0pt}{\isachardot}{\kern0pt}\ {\isacharequal}{\kern0pt}\ {\isacharparenleft}{\kern0pt}{\isasymSum}i{\isasymin}I{\isachardot}{\kern0pt}\ {\isacharparenleft}{\kern0pt}{\isasymintegral}x{\isachardot}{\kern0pt}\ indicator\ A\ x\ {\isacharasterisk}{\kern0pt}\isactrlsub R\ f\ i\ x\ {\isasympartial}M{\isacharparenright}{\kern0pt}{\isacharparenright}{\kern0pt}{\isachardoublequoteclose}\ \isacommand{using}\isamarkupfalse%
\ assms\ \isacommand{by}\isamarkupfalse%
\ {\isacharparenleft}{\kern0pt}auto\ intro{\isacharbang}{\kern0pt}{\isacharcolon}{\kern0pt}\ Bochner{\isacharunderscore}{\kern0pt}Integration{\isachardot}{\kern0pt}integral{\isacharunderscore}{\kern0pt}sum\ integrable{\isacharunderscore}{\kern0pt}mult{\isacharunderscore}{\kern0pt}indicator{\isacharparenright}{\kern0pt}\isanewline
\ \ \isacommand{also}\isamarkupfalse%
\ \isacommand{have}\isamarkupfalse%
\ {\isachardoublequoteopen}{\isachardot}{\kern0pt}{\isachardot}{\kern0pt}{\isachardot}{\kern0pt}\ {\isacharequal}{\kern0pt}\ {\isacharparenleft}{\kern0pt}{\isasymSum}i{\isasymin}I{\isachardot}{\kern0pt}\ {\isacharparenleft}{\kern0pt}{\isasymintegral}x{\isachardot}{\kern0pt}\ indicator\ A\ x\ {\isacharasterisk}{\kern0pt}\isactrlsub R\ cond{\isacharunderscore}{\kern0pt}exp\ M\ F\ {\isacharparenleft}{\kern0pt}f\ i{\isacharparenright}{\kern0pt}\ x\ {\isasympartial}M{\isacharparenright}{\kern0pt}{\isacharparenright}{\kern0pt}{\isachardoublequoteclose}\ \isacommand{using}\isamarkupfalse%
\ cond{\isacharunderscore}{\kern0pt}exp{\isacharunderscore}{\kern0pt}set{\isacharunderscore}{\kern0pt}integral{\isacharbrackleft}{\kern0pt}OF\ assms{\isacharbrackright}{\kern0pt}\ \isacommand{by}\isamarkupfalse%
\ {\isacharparenleft}{\kern0pt}simp\ add{\isacharcolon}{\kern0pt}\ set{\isacharunderscore}{\kern0pt}lebesgue{\isacharunderscore}{\kern0pt}integral{\isacharunderscore}{\kern0pt}def{\isacharparenright}{\kern0pt}\isanewline
\ \ \isacommand{also}\isamarkupfalse%
\ \isacommand{have}\isamarkupfalse%
\ {\isachardoublequoteopen}{\isachardot}{\kern0pt}{\isachardot}{\kern0pt}{\isachardot}{\kern0pt}\ {\isacharequal}{\kern0pt}\ {\isacharparenleft}{\kern0pt}{\isasymintegral}x{\isachardot}{\kern0pt}\ {\isacharparenleft}{\kern0pt}{\isasymSum}i{\isasymin}I{\isachardot}{\kern0pt}\ indicator\ A\ x\ {\isacharasterisk}{\kern0pt}\isactrlsub R\ cond{\isacharunderscore}{\kern0pt}exp\ M\ F\ {\isacharparenleft}{\kern0pt}f\ i{\isacharparenright}{\kern0pt}\ x{\isacharparenright}{\kern0pt}{\isasympartial}M{\isacharparenright}{\kern0pt}{\isachardoublequoteclose}\ \isacommand{using}\isamarkupfalse%
\ assms\ \isacommand{by}\isamarkupfalse%
\ {\isacharparenleft}{\kern0pt}auto\ intro{\isacharbang}{\kern0pt}{\isacharcolon}{\kern0pt}\ Bochner{\isacharunderscore}{\kern0pt}Integration{\isachardot}{\kern0pt}integral{\isacharunderscore}{\kern0pt}sum{\isacharbrackleft}{\kern0pt}symmetric{\isacharbrackright}{\kern0pt}\ integrable{\isacharunderscore}{\kern0pt}mult{\isacharunderscore}{\kern0pt}indicator{\isacharparenright}{\kern0pt}\isanewline
\ \ \isacommand{also}\isamarkupfalse%
\ \isacommand{have}\isamarkupfalse%
\ {\isachardoublequoteopen}{\isachardot}{\kern0pt}{\isachardot}{\kern0pt}{\isachardot}{\kern0pt}\ {\isacharequal}{\kern0pt}\ {\isacharparenleft}{\kern0pt}{\isasymintegral}x{\isasymin}A{\isachardot}{\kern0pt}\ {\isacharparenleft}{\kern0pt}{\isasymSum}i{\isasymin}I{\isachardot}{\kern0pt}\ cond{\isacharunderscore}{\kern0pt}exp\ M\ F\ {\isacharparenleft}{\kern0pt}f\ i{\isacharparenright}{\kern0pt}\ x{\isacharparenright}{\kern0pt}{\isasympartial}M{\isacharparenright}{\kern0pt}{\isachardoublequoteclose}\ \isacommand{unfolding}\isamarkupfalse%
\ set{\isacharunderscore}{\kern0pt}lebesgue{\isacharunderscore}{\kern0pt}integral{\isacharunderscore}{\kern0pt}def\ \isacommand{by}\isamarkupfalse%
\ {\isacharparenleft}{\kern0pt}simp\ add{\isacharcolon}{\kern0pt}\ scaleR{\isacharunderscore}{\kern0pt}sum{\isacharunderscore}{\kern0pt}right{\isacharparenright}{\kern0pt}\isanewline
\ \ \isacommand{finally}\isamarkupfalse%
\ \isacommand{show}\isamarkupfalse%
\ {\isachardoublequoteopen}{\isacharparenleft}{\kern0pt}{\isasymintegral}x{\isasymin}A{\isachardot}{\kern0pt}\ {\isacharparenleft}{\kern0pt}{\isasymSum}i{\isasymin}I{\isachardot}{\kern0pt}\ f\ i\ x{\isacharparenright}{\kern0pt}{\isasympartial}M{\isacharparenright}{\kern0pt}\ {\isacharequal}{\kern0pt}\ {\isacharparenleft}{\kern0pt}{\isasymintegral}x{\isasymin}A{\isachardot}{\kern0pt}\ {\isacharparenleft}{\kern0pt}{\isasymSum}i{\isasymin}I{\isachardot}{\kern0pt}\ cond{\isacharunderscore}{\kern0pt}exp\ M\ F\ {\isacharparenleft}{\kern0pt}f\ i{\isacharparenright}{\kern0pt}\ x{\isacharparenright}{\kern0pt}{\isasympartial}M{\isacharparenright}{\kern0pt}{\isachardoublequoteclose}\ \isacommand{by}\isamarkupfalse%
\ auto\isanewline
\isacommand{qed}\isamarkupfalse%
\ {\isacharparenleft}{\kern0pt}auto\ simp\ add{\isacharcolon}{\kern0pt}\ assms\ integrable{\isacharunderscore}{\kern0pt}cond{\isacharunderscore}{\kern0pt}exp{\isacharparenright}{\kern0pt}%
\endisatagproof
{\isafoldproof}%
%
\isadelimproof
%
\endisadelimproof
%
\isadelimdocument
%
\endisadelimdocument
%
\isatagdocument
%
\isamarkupsubsection{Linearly Ordered Banach Spaces%
}
\isamarkuptrue%
%
\endisatagdocument
{\isafolddocument}%
%
\isadelimdocument
%
\endisadelimdocument
\isacommand{lemma}\isamarkupfalse%
\ cond{\isacharunderscore}{\kern0pt}exp{\isacharunderscore}{\kern0pt}gr{\isacharunderscore}{\kern0pt}c{\isacharcolon}{\kern0pt}\isanewline
\ \ \isakeyword{fixes}\ f\ {\isacharcolon}{\kern0pt}{\isacharcolon}{\kern0pt}\ {\isachardoublequoteopen}{\isacharprime}{\kern0pt}a\ {\isasymRightarrow}\ {\isacharprime}{\kern0pt}b\ {\isacharcolon}{\kern0pt}{\isacharcolon}{\kern0pt}\ {\isacharbraceleft}{\kern0pt}second{\isacharunderscore}{\kern0pt}countable{\isacharunderscore}{\kern0pt}topology{\isacharcomma}{\kern0pt}\ banach{\isacharcomma}{\kern0pt}\ linorder{\isacharunderscore}{\kern0pt}topology{\isacharcomma}{\kern0pt}\ ordered{\isacharunderscore}{\kern0pt}real{\isacharunderscore}{\kern0pt}vector{\isacharbraceright}{\kern0pt}{\isachardoublequoteclose}\isanewline
\ \ \isakeyword{assumes}\ {\isachardoublequoteopen}integrable\ M\ f{\isachardoublequoteclose}\ \ {\isachardoublequoteopen}AE\ x\ in\ M{\isachardot}{\kern0pt}\ f\ x\ {\isachargreater}{\kern0pt}\ c{\isachardoublequoteclose}\isanewline
\ \ \isakeyword{shows}\ {\isachardoublequoteopen}AE\ x\ in\ M{\isachardot}{\kern0pt}\ cond{\isacharunderscore}{\kern0pt}exp\ M\ F\ f\ x\ {\isachargreater}{\kern0pt}\ c{\isachardoublequoteclose}\isanewline
%
\isadelimproof
%
\endisadelimproof
%
\isatagproof
\isacommand{proof}\isamarkupfalse%
\ {\isacharminus}{\kern0pt}\isanewline
\ \ \isacommand{define}\isamarkupfalse%
\ X\ \isakeyword{where}\ {\isachardoublequoteopen}X\ {\isacharequal}{\kern0pt}\ {\isacharbraceleft}{\kern0pt}x\ {\isasymin}\ space\ M{\isachardot}{\kern0pt}\ cond{\isacharunderscore}{\kern0pt}exp\ M\ F\ f\ x\ {\isasymle}\ c{\isacharbraceright}{\kern0pt}{\isachardoublequoteclose}\isanewline
\ \ \isacommand{have}\isamarkupfalse%
\ {\isacharbrackleft}{\kern0pt}measurable{\isacharbrackright}{\kern0pt}{\isacharcolon}{\kern0pt}\ {\isachardoublequoteopen}X\ {\isasymin}\ sets\ F{\isachardoublequoteclose}\ \isacommand{unfolding}\isamarkupfalse%
\ X{\isacharunderscore}{\kern0pt}def\ \isacommand{by}\isamarkupfalse%
\ measurable\ {\isacharparenleft}{\kern0pt}metis\ sets{\isachardot}{\kern0pt}top\ subalg\ subalgebra{\isacharunderscore}{\kern0pt}def{\isacharparenright}{\kern0pt}\isanewline
\ \ \isacommand{hence}\isamarkupfalse%
\ X{\isacharunderscore}{\kern0pt}in{\isacharunderscore}{\kern0pt}M{\isacharcolon}{\kern0pt}\ {\isachardoublequoteopen}X\ {\isasymin}\ sets\ M{\isachardoublequoteclose}\ \isacommand{using}\isamarkupfalse%
\ sets{\isacharunderscore}{\kern0pt}restr{\isacharunderscore}{\kern0pt}to{\isacharunderscore}{\kern0pt}subalg\ subalg\ subalgebra{\isacharunderscore}{\kern0pt}def\ \isacommand{by}\isamarkupfalse%
\ blast\isanewline
\ \ \isacommand{have}\isamarkupfalse%
\ {\isachardoublequoteopen}emeasure\ M\ X\ {\isacharequal}{\kern0pt}\ {\isadigit{0}}{\isachardoublequoteclose}\isanewline
\ \ \isacommand{proof}\isamarkupfalse%
\ {\isacharparenleft}{\kern0pt}rule\ ccontr{\isacharparenright}{\kern0pt}\isanewline
\ \ \ \ \isacommand{assume}\isamarkupfalse%
\ {\isachardoublequoteopen}emeasure\ M\ X\ {\isasymnoteq}\ {\isadigit{0}}{\isachardoublequoteclose}\isanewline
\ \ \ \ \isacommand{have}\isamarkupfalse%
\ {\isachardoublequoteopen}emeasure\ {\isacharparenleft}{\kern0pt}restr{\isacharunderscore}{\kern0pt}to{\isacharunderscore}{\kern0pt}subalg\ M\ F{\isacharparenright}{\kern0pt}\ X\ {\isacharequal}{\kern0pt}\ emeasure\ M\ X{\isachardoublequoteclose}\ \isacommand{by}\isamarkupfalse%
\ {\isacharparenleft}{\kern0pt}simp\ add{\isacharcolon}{\kern0pt}\ emeasure{\isacharunderscore}{\kern0pt}restr{\isacharunderscore}{\kern0pt}to{\isacharunderscore}{\kern0pt}subalg\ subalg{\isacharparenright}{\kern0pt}\isanewline
\ \ \ \ \isacommand{hence}\isamarkupfalse%
\ {\isachardoublequoteopen}emeasure\ {\isacharparenleft}{\kern0pt}restr{\isacharunderscore}{\kern0pt}to{\isacharunderscore}{\kern0pt}subalg\ M\ F{\isacharparenright}{\kern0pt}\ X\ {\isachargreater}{\kern0pt}\ {\isadigit{0}}{\isachardoublequoteclose}\ \isacommand{using}\isamarkupfalse%
\ {\isacartoucheopen}{\isasymnot}{\isacharparenleft}{\kern0pt}emeasure\ M\ X{\isacharparenright}{\kern0pt}\ {\isacharequal}{\kern0pt}\ {\isadigit{0}}{\isacartoucheclose}\ gr{\isacharunderscore}{\kern0pt}zeroI\ \isacommand{by}\isamarkupfalse%
\ auto\isanewline
\ \ \ \ \isacommand{then}\isamarkupfalse%
\ \isacommand{obtain}\isamarkupfalse%
\ A\ \isakeyword{where}\ A{\isacharcolon}{\kern0pt}\ {\isachardoublequoteopen}A\ {\isasymin}\ sets\ {\isacharparenleft}{\kern0pt}restr{\isacharunderscore}{\kern0pt}to{\isacharunderscore}{\kern0pt}subalg\ M\ F{\isacharparenright}{\kern0pt}{\isachardoublequoteclose}\ {\isachardoublequoteopen}A\ {\isasymsubseteq}\ X{\isachardoublequoteclose}\ {\isachardoublequoteopen}emeasure\ {\isacharparenleft}{\kern0pt}restr{\isacharunderscore}{\kern0pt}to{\isacharunderscore}{\kern0pt}subalg\ M\ F{\isacharparenright}{\kern0pt}\ A\ {\isachargreater}{\kern0pt}\ {\isadigit{0}}{\isachardoublequoteclose}\ {\isachardoublequoteopen}emeasure\ {\isacharparenleft}{\kern0pt}restr{\isacharunderscore}{\kern0pt}to{\isacharunderscore}{\kern0pt}subalg\ M\ F{\isacharparenright}{\kern0pt}\ A\ {\isacharless}{\kern0pt}\ {\isasyminfinity}{\isachardoublequoteclose}\isanewline
\ \ \ \ \ \ \isacommand{using}\isamarkupfalse%
\ sigma{\isacharunderscore}{\kern0pt}fin{\isacharunderscore}{\kern0pt}subalg\ \isacommand{by}\isamarkupfalse%
\ {\isacharparenleft}{\kern0pt}metis\ emeasure{\isacharunderscore}{\kern0pt}notin{\isacharunderscore}{\kern0pt}sets\ ennreal{\isacharunderscore}{\kern0pt}{\isadigit{0}}\ infinity{\isacharunderscore}{\kern0pt}ennreal{\isacharunderscore}{\kern0pt}def\ le{\isacharunderscore}{\kern0pt}less{\isacharunderscore}{\kern0pt}linear\ neq{\isacharunderscore}{\kern0pt}top{\isacharunderscore}{\kern0pt}trans\ not{\isacharunderscore}{\kern0pt}gr{\isacharunderscore}{\kern0pt}zero\ order{\isacharunderscore}{\kern0pt}refl\ sigma{\isacharunderscore}{\kern0pt}finite{\isacharunderscore}{\kern0pt}measure{\isachardot}{\kern0pt}approx{\isacharunderscore}{\kern0pt}PInf{\isacharunderscore}{\kern0pt}emeasure{\isacharunderscore}{\kern0pt}with{\isacharunderscore}{\kern0pt}finite{\isacharparenright}{\kern0pt}\isanewline
\ \ \ \ \isacommand{hence}\isamarkupfalse%
\ {\isacharbrackleft}{\kern0pt}simp{\isacharbrackright}{\kern0pt}{\isacharcolon}{\kern0pt}\ {\isachardoublequoteopen}A\ {\isasymin}\ sets\ F{\isachardoublequoteclose}\ \isacommand{using}\isamarkupfalse%
\ subalg\ sets{\isacharunderscore}{\kern0pt}restr{\isacharunderscore}{\kern0pt}to{\isacharunderscore}{\kern0pt}subalg\ \isacommand{by}\isamarkupfalse%
\ blast\isanewline
\ \ \ \ \isacommand{hence}\isamarkupfalse%
\ A{\isacharunderscore}{\kern0pt}in{\isacharunderscore}{\kern0pt}sets{\isacharunderscore}{\kern0pt}M{\isacharbrackleft}{\kern0pt}simp{\isacharbrackright}{\kern0pt}{\isacharcolon}{\kern0pt}\ {\isachardoublequoteopen}A\ {\isasymin}\ sets\ M{\isachardoublequoteclose}\ \isacommand{using}\isamarkupfalse%
\ sets{\isacharunderscore}{\kern0pt}restr{\isacharunderscore}{\kern0pt}to{\isacharunderscore}{\kern0pt}subalg\ subalg\ subalgebra{\isacharunderscore}{\kern0pt}def\ \isacommand{by}\isamarkupfalse%
\ blast\isanewline
\ \ \ \ \isacommand{have}\isamarkupfalse%
\ {\isacharbrackleft}{\kern0pt}simp{\isacharbrackright}{\kern0pt}{\isacharcolon}{\kern0pt}\ {\isachardoublequoteopen}set{\isacharunderscore}{\kern0pt}integrable\ M\ A\ {\isacharparenleft}{\kern0pt}{\isasymlambda}x{\isachardot}{\kern0pt}\ c{\isacharparenright}{\kern0pt}{\isachardoublequoteclose}\ \isacommand{using}\isamarkupfalse%
\ A\ subalg\ \isacommand{by}\isamarkupfalse%
\ {\isacharparenleft}{\kern0pt}auto\ simp\ add{\isacharcolon}{\kern0pt}\ set{\isacharunderscore}{\kern0pt}integrable{\isacharunderscore}{\kern0pt}def\ emeasure{\isacharunderscore}{\kern0pt}restr{\isacharunderscore}{\kern0pt}to{\isacharunderscore}{\kern0pt}subalg{\isacharparenright}{\kern0pt}\ \isanewline
\ \ \ \ \isacommand{have}\isamarkupfalse%
\ {\isacharbrackleft}{\kern0pt}simp{\isacharbrackright}{\kern0pt}{\isacharcolon}{\kern0pt}\ {\isachardoublequoteopen}set{\isacharunderscore}{\kern0pt}integrable\ M\ A\ f{\isachardoublequoteclose}\ \isacommand{unfolding}\isamarkupfalse%
\ set{\isacharunderscore}{\kern0pt}integrable{\isacharunderscore}{\kern0pt}def\ \isacommand{by}\isamarkupfalse%
\ {\isacharparenleft}{\kern0pt}rule\ integrable{\isacharunderscore}{\kern0pt}mult{\isacharunderscore}{\kern0pt}indicator{\isacharcomma}{\kern0pt}\ auto\ simp\ add{\isacharcolon}{\kern0pt}\ assms{\isacharparenleft}{\kern0pt}{\isadigit{1}}{\isacharparenright}{\kern0pt}{\isacharparenright}{\kern0pt}\isanewline
\ \ \ \ \isacommand{have}\isamarkupfalse%
\ {\isachardoublequoteopen}AE\ x\ in\ M{\isachardot}{\kern0pt}\ indicator\ A\ x\ {\isacharasterisk}{\kern0pt}\isactrlsub R\ c\ {\isacharequal}{\kern0pt}\ indicator\ A\ x\ {\isacharasterisk}{\kern0pt}\isactrlsub R\ f\ x{\isachardoublequoteclose}\isanewline
\ \ \ \ \isacommand{proof}\isamarkupfalse%
\ {\isacharparenleft}{\kern0pt}rule\ integral{\isacharunderscore}{\kern0pt}eq{\isacharunderscore}{\kern0pt}mono{\isacharunderscore}{\kern0pt}AE{\isacharunderscore}{\kern0pt}eq{\isacharunderscore}{\kern0pt}AE{\isacharparenright}{\kern0pt}\isanewline
\ \ \ \ \ \ \isacommand{show}\isamarkupfalse%
\ {\isachardoublequoteopen}LINT\ x{\isacharbar}{\kern0pt}M{\isachardot}{\kern0pt}\ indicator\ A\ x\ {\isacharasterisk}{\kern0pt}\isactrlsub R\ c\ {\isacharequal}{\kern0pt}\ LINT\ x{\isacharbar}{\kern0pt}M{\isachardot}{\kern0pt}\ indicator\ A\ x\ {\isacharasterisk}{\kern0pt}\isactrlsub R\ f\ x{\isachardoublequoteclose}\ \isanewline
\ \ \ \ \ \ \isacommand{proof}\isamarkupfalse%
\ {\isacharparenleft}{\kern0pt}simp\ only{\isacharcolon}{\kern0pt}\ set{\isacharunderscore}{\kern0pt}lebesgue{\isacharunderscore}{\kern0pt}integral{\isacharunderscore}{\kern0pt}def{\isacharbrackleft}{\kern0pt}symmetric{\isacharbrackright}{\kern0pt}{\isacharcomma}{\kern0pt}\ rule\ antisym{\isacharparenright}{\kern0pt}\isanewline
\ \ \ \ \ \ \ \ \isacommand{show}\isamarkupfalse%
\ {\isachardoublequoteopen}{\isacharparenleft}{\kern0pt}{\isasymintegral}x{\isasymin}A{\isachardot}{\kern0pt}\ c\ {\isasympartial}M{\isacharparenright}{\kern0pt}\ {\isasymle}\ {\isacharparenleft}{\kern0pt}{\isasymintegral}x{\isasymin}A{\isachardot}{\kern0pt}\ f\ x\ {\isasympartial}M{\isacharparenright}{\kern0pt}{\isachardoublequoteclose}\ \isacommand{using}\isamarkupfalse%
\ assms{\isacharparenleft}{\kern0pt}{\isadigit{2}}{\isacharparenright}{\kern0pt}\ \isacommand{by}\isamarkupfalse%
\ {\isacharparenleft}{\kern0pt}intro\ set{\isacharunderscore}{\kern0pt}integral{\isacharunderscore}{\kern0pt}mono{\isacharunderscore}{\kern0pt}AE{\isacharunderscore}{\kern0pt}banach{\isacharparenright}{\kern0pt}\ auto\isanewline
\ \ \ \ \ \ \ \ \isacommand{have}\isamarkupfalse%
\ {\isachardoublequoteopen}{\isacharparenleft}{\kern0pt}{\isasymintegral}x{\isasymin}A{\isachardot}{\kern0pt}\ f\ x\ {\isasympartial}M{\isacharparenright}{\kern0pt}\ {\isacharequal}{\kern0pt}\ {\isacharparenleft}{\kern0pt}{\isasymintegral}x{\isasymin}A{\isachardot}{\kern0pt}\ cond{\isacharunderscore}{\kern0pt}exp\ M\ F\ f\ x\ {\isasympartial}M{\isacharparenright}{\kern0pt}{\isachardoublequoteclose}\ \isacommand{by}\isamarkupfalse%
\ {\isacharparenleft}{\kern0pt}rule\ cond{\isacharunderscore}{\kern0pt}exp{\isacharunderscore}{\kern0pt}set{\isacharunderscore}{\kern0pt}integral{\isacharcomma}{\kern0pt}\ auto\ simp\ add{\isacharcolon}{\kern0pt}\ assms{\isacharparenright}{\kern0pt}\isanewline
\ \ \ \ \ \ \ \ \isacommand{also}\isamarkupfalse%
\ \isacommand{have}\isamarkupfalse%
\ {\isachardoublequoteopen}{\isachardot}{\kern0pt}{\isachardot}{\kern0pt}{\isachardot}{\kern0pt}\ {\isasymle}\ {\isacharparenleft}{\kern0pt}{\isasymintegral}x{\isasymin}A{\isachardot}{\kern0pt}\ c\ {\isasympartial}M{\isacharparenright}{\kern0pt}{\isachardoublequoteclose}\ \isacommand{using}\isamarkupfalse%
\ A\ \isacommand{by}\isamarkupfalse%
\ {\isacharparenleft}{\kern0pt}auto\ intro{\isacharbang}{\kern0pt}{\isacharcolon}{\kern0pt}\ set{\isacharunderscore}{\kern0pt}integral{\isacharunderscore}{\kern0pt}mono{\isacharunderscore}{\kern0pt}banach\ simp\ add{\isacharcolon}{\kern0pt}\ X{\isacharunderscore}{\kern0pt}def{\isacharparenright}{\kern0pt}\isanewline
\ \ \ \ \ \ \ \ \isacommand{finally}\isamarkupfalse%
\ \isacommand{show}\isamarkupfalse%
\ {\isachardoublequoteopen}{\isacharparenleft}{\kern0pt}{\isasymintegral}x{\isasymin}A{\isachardot}{\kern0pt}\ f\ x\ {\isasympartial}M{\isacharparenright}{\kern0pt}\ {\isasymle}\ {\isacharparenleft}{\kern0pt}{\isasymintegral}x{\isasymin}A{\isachardot}{\kern0pt}\ c\ {\isasympartial}M{\isacharparenright}{\kern0pt}{\isachardoublequoteclose}\ \isacommand{by}\isamarkupfalse%
\ simp\isanewline
\ \ \ \ \ \ \isacommand{qed}\isamarkupfalse%
\isanewline
\ \ \ \ \ \ \isacommand{show}\isamarkupfalse%
\ {\isachardoublequoteopen}AE\ x\ in\ M{\isachardot}{\kern0pt}\ indicator\ A\ x\ {\isacharasterisk}{\kern0pt}\isactrlsub R\ c\ {\isasymle}\ indicator\ A\ x\ {\isacharasterisk}{\kern0pt}\isactrlsub R\ f\ x{\isachardoublequoteclose}\ \isacommand{using}\isamarkupfalse%
\ assms\ \isacommand{by}\isamarkupfalse%
\ {\isacharparenleft}{\kern0pt}auto\ simp\ add{\isacharcolon}{\kern0pt}\ X{\isacharunderscore}{\kern0pt}def\ indicator{\isacharunderscore}{\kern0pt}def{\isacharparenright}{\kern0pt}\isanewline
\ \ \ \ \isacommand{qed}\isamarkupfalse%
\ {\isacharparenleft}{\kern0pt}auto\ simp\ add{\isacharcolon}{\kern0pt}\ set{\isacharunderscore}{\kern0pt}integrable{\isacharunderscore}{\kern0pt}def{\isacharbrackleft}{\kern0pt}symmetric{\isacharbrackright}{\kern0pt}{\isacharparenright}{\kern0pt}\isanewline
\ \ \ \ \isacommand{hence}\isamarkupfalse%
\ {\isachardoublequoteopen}AE\ x{\isasymin}A\ in\ M{\isachardot}{\kern0pt}\ c\ {\isacharequal}{\kern0pt}\ f\ x{\isachardoublequoteclose}\ \isacommand{by}\isamarkupfalse%
\ auto\isanewline
\ \ \ \ \isacommand{hence}\isamarkupfalse%
\ {\isachardoublequoteopen}AE\ x{\isasymin}A\ in\ M{\isachardot}{\kern0pt}\ False{\isachardoublequoteclose}\ \isacommand{using}\isamarkupfalse%
\ assms{\isacharparenleft}{\kern0pt}{\isadigit{2}}{\isacharparenright}{\kern0pt}\ \isacommand{by}\isamarkupfalse%
\ auto\isanewline
\ \ \ \ \isacommand{hence}\isamarkupfalse%
\ {\isachardoublequoteopen}A\ {\isasymin}\ null{\isacharunderscore}{\kern0pt}sets\ M{\isachardoublequoteclose}\ \isacommand{using}\isamarkupfalse%
\ AE{\isacharunderscore}{\kern0pt}iff{\isacharunderscore}{\kern0pt}null{\isacharunderscore}{\kern0pt}sets\ A{\isacharunderscore}{\kern0pt}in{\isacharunderscore}{\kern0pt}sets{\isacharunderscore}{\kern0pt}M\ \isacommand{by}\isamarkupfalse%
\ metis\isanewline
\ \ \ \ \isacommand{thus}\isamarkupfalse%
\ False\ \isacommand{using}\isamarkupfalse%
\ A{\isacharparenleft}{\kern0pt}{\isadigit{3}}{\isacharparenright}{\kern0pt}\ \isacommand{by}\isamarkupfalse%
\ {\isacharparenleft}{\kern0pt}simp\ add{\isacharcolon}{\kern0pt}\ emeasure{\isacharunderscore}{\kern0pt}restr{\isacharunderscore}{\kern0pt}to{\isacharunderscore}{\kern0pt}subalg\ null{\isacharunderscore}{\kern0pt}setsD{\isadigit{1}}\ subalg{\isacharparenright}{\kern0pt}\isanewline
\ \ \isacommand{qed}\isamarkupfalse%
\isanewline
\ \ \isacommand{thus}\isamarkupfalse%
\ {\isacharquery}{\kern0pt}thesis\ \isacommand{using}\isamarkupfalse%
\ AE{\isacharunderscore}{\kern0pt}iff{\isacharunderscore}{\kern0pt}null{\isacharunderscore}{\kern0pt}sets{\isacharbrackleft}{\kern0pt}OF\ X{\isacharunderscore}{\kern0pt}in{\isacharunderscore}{\kern0pt}M{\isacharbrackright}{\kern0pt}\ \isacommand{unfolding}\isamarkupfalse%
\ X{\isacharunderscore}{\kern0pt}def\ \isacommand{by}\isamarkupfalse%
\ auto\isanewline
\isacommand{qed}\isamarkupfalse%
%
\endisatagproof
{\isafoldproof}%
%
\isadelimproof
\isanewline
%
\endisadelimproof
\isanewline
\isacommand{corollary}\isamarkupfalse%
\ cond{\isacharunderscore}{\kern0pt}exp{\isacharunderscore}{\kern0pt}less{\isacharunderscore}{\kern0pt}c{\isacharcolon}{\kern0pt}\isanewline
\ \ \isakeyword{fixes}\ f\ {\isacharcolon}{\kern0pt}{\isacharcolon}{\kern0pt}\ {\isachardoublequoteopen}{\isacharprime}{\kern0pt}a\ {\isasymRightarrow}\ {\isacharprime}{\kern0pt}b\ {\isacharcolon}{\kern0pt}{\isacharcolon}{\kern0pt}\ {\isacharbraceleft}{\kern0pt}second{\isacharunderscore}{\kern0pt}countable{\isacharunderscore}{\kern0pt}topology{\isacharcomma}{\kern0pt}\ banach{\isacharcomma}{\kern0pt}\ linorder{\isacharunderscore}{\kern0pt}topology{\isacharcomma}{\kern0pt}\ ordered{\isacharunderscore}{\kern0pt}real{\isacharunderscore}{\kern0pt}vector{\isacharbraceright}{\kern0pt}{\isachardoublequoteclose}\isanewline
\ \ \isakeyword{assumes}\ {\isachardoublequoteopen}integrable\ M\ f{\isachardoublequoteclose}\ {\isachardoublequoteopen}AE\ x\ in\ M{\isachardot}{\kern0pt}\ f\ x\ {\isacharless}{\kern0pt}\ c{\isachardoublequoteclose}\isanewline
\ \ \isakeyword{shows}\ {\isachardoublequoteopen}AE\ x\ in\ M{\isachardot}{\kern0pt}\ cond{\isacharunderscore}{\kern0pt}exp\ M\ F\ f\ x\ {\isacharless}{\kern0pt}\ c{\isachardoublequoteclose}\isanewline
%
\isadelimproof
%
\endisadelimproof
%
\isatagproof
\isacommand{proof}\isamarkupfalse%
\ {\isacharminus}{\kern0pt}\isanewline
\ \ \isacommand{have}\isamarkupfalse%
\ {\isachardoublequoteopen}AE\ x\ in\ M{\isachardot}{\kern0pt}\ cond{\isacharunderscore}{\kern0pt}exp\ M\ F\ f\ x\ {\isacharequal}{\kern0pt}\ {\isacharminus}{\kern0pt}\ cond{\isacharunderscore}{\kern0pt}exp\ M\ F\ {\isacharparenleft}{\kern0pt}{\isasymlambda}x{\isachardot}{\kern0pt}\ {\isacharminus}{\kern0pt}\ f\ x{\isacharparenright}{\kern0pt}\ x{\isachardoublequoteclose}\ \isacommand{using}\isamarkupfalse%
\ cond{\isacharunderscore}{\kern0pt}exp{\isacharunderscore}{\kern0pt}uminus{\isacharbrackleft}{\kern0pt}OF\ assms{\isacharparenleft}{\kern0pt}{\isadigit{1}}{\isacharparenright}{\kern0pt}{\isacharbrackright}{\kern0pt}\ \isacommand{by}\isamarkupfalse%
\ auto\isanewline
\ \ \isacommand{moreover}\isamarkupfalse%
\ \isacommand{have}\isamarkupfalse%
\ {\isachardoublequoteopen}AE\ x\ in\ M{\isachardot}{\kern0pt}\ cond{\isacharunderscore}{\kern0pt}exp\ M\ F\ {\isacharparenleft}{\kern0pt}{\isasymlambda}x{\isachardot}{\kern0pt}\ {\isacharminus}{\kern0pt}\ f\ x{\isacharparenright}{\kern0pt}\ x\ {\isachargreater}{\kern0pt}\ {\isacharminus}{\kern0pt}\ c{\isachardoublequoteclose}\ \ \isacommand{using}\isamarkupfalse%
\ assms\ \isacommand{by}\isamarkupfalse%
\ {\isacharparenleft}{\kern0pt}intro\ cond{\isacharunderscore}{\kern0pt}exp{\isacharunderscore}{\kern0pt}gr{\isacharunderscore}{\kern0pt}c{\isacharparenright}{\kern0pt}\ auto\isanewline
\ \ \isacommand{ultimately}\isamarkupfalse%
\ \isacommand{show}\isamarkupfalse%
\ {\isacharquery}{\kern0pt}thesis\ \isacommand{by}\isamarkupfalse%
\ {\isacharparenleft}{\kern0pt}force\ simp\ add{\isacharcolon}{\kern0pt}\ minus{\isacharunderscore}{\kern0pt}less{\isacharunderscore}{\kern0pt}iff{\isacharparenright}{\kern0pt}\isanewline
\isacommand{qed}\isamarkupfalse%
%
\endisatagproof
{\isafoldproof}%
%
\isadelimproof
\isanewline
%
\endisadelimproof
\isanewline
\isacommand{lemma}\isamarkupfalse%
\ cond{\isacharunderscore}{\kern0pt}exp{\isacharunderscore}{\kern0pt}mono{\isacharunderscore}{\kern0pt}strict{\isacharcolon}{\kern0pt}\isanewline
\ \ \isakeyword{fixes}\ f\ {\isacharcolon}{\kern0pt}{\isacharcolon}{\kern0pt}\ {\isachardoublequoteopen}{\isacharprime}{\kern0pt}a\ {\isasymRightarrow}\ {\isacharprime}{\kern0pt}b\ {\isacharcolon}{\kern0pt}{\isacharcolon}{\kern0pt}\ {\isacharbraceleft}{\kern0pt}second{\isacharunderscore}{\kern0pt}countable{\isacharunderscore}{\kern0pt}topology{\isacharcomma}{\kern0pt}\ banach{\isacharcomma}{\kern0pt}\ linorder{\isacharunderscore}{\kern0pt}topology{\isacharcomma}{\kern0pt}\ ordered{\isacharunderscore}{\kern0pt}real{\isacharunderscore}{\kern0pt}vector{\isacharbraceright}{\kern0pt}{\isachardoublequoteclose}\isanewline
\ \ \isakeyword{assumes}\ {\isachardoublequoteopen}integrable\ M\ f{\isachardoublequoteclose}\ {\isachardoublequoteopen}integrable\ M\ g{\isachardoublequoteclose}\ {\isachardoublequoteopen}AE\ x\ in\ M{\isachardot}{\kern0pt}\ f\ x\ {\isacharless}{\kern0pt}\ g\ x{\isachardoublequoteclose}\isanewline
\ \ \isakeyword{shows}\ {\isachardoublequoteopen}AE\ x\ in\ M{\isachardot}{\kern0pt}\ cond{\isacharunderscore}{\kern0pt}exp\ M\ F\ f\ x\ {\isacharless}{\kern0pt}\ cond{\isacharunderscore}{\kern0pt}exp\ M\ F\ g\ x{\isachardoublequoteclose}\isanewline
%
\isadelimproof
\ \ %
\endisadelimproof
%
\isatagproof
\isacommand{using}\isamarkupfalse%
\ cond{\isacharunderscore}{\kern0pt}exp{\isacharunderscore}{\kern0pt}less{\isacharunderscore}{\kern0pt}c{\isacharbrackleft}{\kern0pt}OF\ Bochner{\isacharunderscore}{\kern0pt}Integration{\isachardot}{\kern0pt}integrable{\isacharunderscore}{\kern0pt}diff{\isacharcomma}{\kern0pt}\ OF\ assms{\isacharparenleft}{\kern0pt}{\isadigit{1}}{\isacharcomma}{\kern0pt}{\isadigit{2}}{\isacharparenright}{\kern0pt}{\isacharcomma}{\kern0pt}\ of\ {\isadigit{0}}{\isacharbrackright}{\kern0pt}\ \isanewline
\ \ \ \ \ \ \ \ cond{\isacharunderscore}{\kern0pt}exp{\isacharunderscore}{\kern0pt}diff{\isacharbrackleft}{\kern0pt}OF\ assms{\isacharparenleft}{\kern0pt}{\isadigit{1}}{\isacharcomma}{\kern0pt}{\isadigit{2}}{\isacharparenright}{\kern0pt}{\isacharbrackright}{\kern0pt}\ assms{\isacharparenleft}{\kern0pt}{\isadigit{3}}{\isacharparenright}{\kern0pt}\ \isacommand{by}\isamarkupfalse%
\ auto%
\endisatagproof
{\isafoldproof}%
%
\isadelimproof
\isanewline
%
\endisadelimproof
\isanewline
\isacommand{lemma}\isamarkupfalse%
\ cond{\isacharunderscore}{\kern0pt}exp{\isacharunderscore}{\kern0pt}ge{\isacharunderscore}{\kern0pt}c{\isacharcolon}{\kern0pt}\isanewline
\ \ \isakeyword{fixes}\ f\ {\isacharcolon}{\kern0pt}{\isacharcolon}{\kern0pt}\ {\isachardoublequoteopen}{\isacharprime}{\kern0pt}a\ {\isasymRightarrow}\ {\isacharprime}{\kern0pt}b\ {\isacharcolon}{\kern0pt}{\isacharcolon}{\kern0pt}\ {\isacharbraceleft}{\kern0pt}second{\isacharunderscore}{\kern0pt}countable{\isacharunderscore}{\kern0pt}topology{\isacharcomma}{\kern0pt}\ banach{\isacharcomma}{\kern0pt}\ linorder{\isacharunderscore}{\kern0pt}topology{\isacharcomma}{\kern0pt}\ ordered{\isacharunderscore}{\kern0pt}real{\isacharunderscore}{\kern0pt}vector{\isacharbraceright}{\kern0pt}{\isachardoublequoteclose}\isanewline
\ \ \isakeyword{assumes}\ {\isacharbrackleft}{\kern0pt}measurable{\isacharbrackright}{\kern0pt}{\isacharcolon}{\kern0pt}\ {\isachardoublequoteopen}integrable\ M\ f{\isachardoublequoteclose}\ \ \ \ \ \ \ \ \ \ \ \ \ \ \ \ \ \ \ \ \ \ \ \ \ \ \ \ \ \ \ \ \ \ \ \ \ \ \ \ \ \ \ \ \ \ \ \ \ \ \ \ \ \ \ \ \ \ \ \ \ \ \ \isanewline
\ \ \ \ \ \ \isakeyword{and}\ {\isachardoublequoteopen}AE\ x\ in\ M{\isachardot}{\kern0pt}\ f\ x\ {\isasymge}\ c{\isachardoublequoteclose}\isanewline
\ \ \isakeyword{shows}\ {\isachardoublequoteopen}AE\ x\ in\ M{\isachardot}{\kern0pt}\ cond{\isacharunderscore}{\kern0pt}exp\ M\ F\ f\ x\ {\isasymge}\ c{\isachardoublequoteclose}\isanewline
%
\isadelimproof
%
\endisadelimproof
%
\isatagproof
\isacommand{proof}\isamarkupfalse%
\ {\isacharminus}{\kern0pt}\isanewline
\ \ \isacommand{let}\isamarkupfalse%
\ {\isacharquery}{\kern0pt}F\ {\isacharequal}{\kern0pt}\ {\isachardoublequoteopen}restr{\isacharunderscore}{\kern0pt}to{\isacharunderscore}{\kern0pt}subalg\ M\ F{\isachardoublequoteclose}\isanewline
\ \ \isacommand{interpret}\isamarkupfalse%
\ sigma{\isacharunderscore}{\kern0pt}finite{\isacharunderscore}{\kern0pt}measure\ {\isachardoublequoteopen}restr{\isacharunderscore}{\kern0pt}to{\isacharunderscore}{\kern0pt}subalg\ M\ F{\isachardoublequoteclose}\ \isacommand{using}\isamarkupfalse%
\ sigma{\isacharunderscore}{\kern0pt}fin{\isacharunderscore}{\kern0pt}subalg\ \isacommand{by}\isamarkupfalse%
\ auto\isanewline
\ \ \isacommand{{\isacharbraceleft}{\kern0pt}}\isamarkupfalse%
\ \isanewline
\ \ \ \ \isacommand{fix}\isamarkupfalse%
\ A\ \isacommand{assume}\isamarkupfalse%
\ asm{\isacharcolon}{\kern0pt}\ {\isachardoublequoteopen}A\ {\isasymin}\ sets\ {\isacharquery}{\kern0pt}F{\isachardoublequoteclose}\ {\isachardoublequoteopen}{\isadigit{0}}\ {\isacharless}{\kern0pt}\ measure\ {\isacharquery}{\kern0pt}F\ A{\isachardoublequoteclose}\isanewline
\ \ \ \ \isacommand{have}\isamarkupfalse%
\ {\isacharbrackleft}{\kern0pt}simp{\isacharbrackright}{\kern0pt}{\isacharcolon}{\kern0pt}\ {\isachardoublequoteopen}sets\ {\isacharquery}{\kern0pt}F\ {\isacharequal}{\kern0pt}\ sets\ F{\isachardoublequoteclose}\ {\isachardoublequoteopen}measure\ {\isacharquery}{\kern0pt}F\ A\ {\isacharequal}{\kern0pt}\ measure\ M\ A{\isachardoublequoteclose}\ \isacommand{using}\isamarkupfalse%
\ asm\ \isacommand{by}\isamarkupfalse%
\ {\isacharparenleft}{\kern0pt}auto\ simp\ add{\isacharcolon}{\kern0pt}\ measure{\isacharunderscore}{\kern0pt}def\ sets{\isacharunderscore}{\kern0pt}restr{\isacharunderscore}{\kern0pt}to{\isacharunderscore}{\kern0pt}subalg{\isacharbrackleft}{\kern0pt}OF\ subalg{\isacharbrackright}{\kern0pt}\ emeasure{\isacharunderscore}{\kern0pt}restr{\isacharunderscore}{\kern0pt}to{\isacharunderscore}{\kern0pt}subalg{\isacharbrackleft}{\kern0pt}OF\ subalg{\isacharbrackright}{\kern0pt}{\isacharparenright}{\kern0pt}\isanewline
\ \ \ \ \isacommand{have}\isamarkupfalse%
\ M{\isacharunderscore}{\kern0pt}A{\isacharcolon}{\kern0pt}\ {\isachardoublequoteopen}emeasure\ M\ A\ {\isacharless}{\kern0pt}\ {\isasyminfinity}{\isachardoublequoteclose}\ \isacommand{using}\isamarkupfalse%
\ measure{\isacharunderscore}{\kern0pt}zero{\isacharunderscore}{\kern0pt}top\ asm\ \isacommand{by}\isamarkupfalse%
\ {\isacharparenleft}{\kern0pt}force\ simp\ add{\isacharcolon}{\kern0pt}\ top{\isachardot}{\kern0pt}not{\isacharunderscore}{\kern0pt}eq{\isacharunderscore}{\kern0pt}extremum{\isacharparenright}{\kern0pt}\isanewline
\ \ \ \ \isacommand{hence}\isamarkupfalse%
\ F{\isacharunderscore}{\kern0pt}A{\isacharcolon}{\kern0pt}\ {\isachardoublequoteopen}emeasure\ {\isacharquery}{\kern0pt}F\ A\ {\isacharless}{\kern0pt}\ {\isasyminfinity}{\isachardoublequoteclose}\ \isacommand{using}\isamarkupfalse%
\ asm{\isacharparenleft}{\kern0pt}{\isadigit{1}}{\isacharparenright}{\kern0pt}\ emeasure{\isacharunderscore}{\kern0pt}restr{\isacharunderscore}{\kern0pt}to{\isacharunderscore}{\kern0pt}subalg\ subalg\ \isacommand{by}\isamarkupfalse%
\ fastforce\isanewline
\ \ \ \ \isacommand{have}\isamarkupfalse%
\ {\isachardoublequoteopen}set{\isacharunderscore}{\kern0pt}lebesgue{\isacharunderscore}{\kern0pt}integral\ M\ A\ {\isacharparenleft}{\kern0pt}{\isasymlambda}{\isacharunderscore}{\kern0pt}{\isachardot}{\kern0pt}\ c{\isacharparenright}{\kern0pt}\ {\isasymle}\ set{\isacharunderscore}{\kern0pt}lebesgue{\isacharunderscore}{\kern0pt}integral\ M\ A\ f{\isachardoublequoteclose}\ \isacommand{using}\isamarkupfalse%
\ assms\ asm\ M{\isacharunderscore}{\kern0pt}A\ subalg\ \isacommand{by}\isamarkupfalse%
\ {\isacharparenleft}{\kern0pt}intro\ set{\isacharunderscore}{\kern0pt}integral{\isacharunderscore}{\kern0pt}mono{\isacharunderscore}{\kern0pt}AE{\isacharunderscore}{\kern0pt}banach{\isacharcomma}{\kern0pt}\ auto\ simp\ add{\isacharcolon}{\kern0pt}\ set{\isacharunderscore}{\kern0pt}integrable{\isacharunderscore}{\kern0pt}def\ integrable{\isacharunderscore}{\kern0pt}mult{\isacharunderscore}{\kern0pt}indicator\ subalgebra{\isacharunderscore}{\kern0pt}def\ sets{\isacharunderscore}{\kern0pt}restr{\isacharunderscore}{\kern0pt}to{\isacharunderscore}{\kern0pt}subalg{\isacharparenright}{\kern0pt}\isanewline
\ \ \ \ \isacommand{also}\isamarkupfalse%
\ \isacommand{have}\isamarkupfalse%
\ {\isachardoublequoteopen}{\isachardot}{\kern0pt}{\isachardot}{\kern0pt}{\isachardot}{\kern0pt}\ {\isacharequal}{\kern0pt}\ set{\isacharunderscore}{\kern0pt}lebesgue{\isacharunderscore}{\kern0pt}integral\ M\ A\ {\isacharparenleft}{\kern0pt}cond{\isacharunderscore}{\kern0pt}exp\ M\ F\ f{\isacharparenright}{\kern0pt}{\isachardoublequoteclose}\ \isacommand{using}\isamarkupfalse%
\ cond{\isacharunderscore}{\kern0pt}exp{\isacharunderscore}{\kern0pt}set{\isacharunderscore}{\kern0pt}integral{\isacharbrackleft}{\kern0pt}OF\ assms{\isacharparenleft}{\kern0pt}{\isadigit{1}}{\isacharparenright}{\kern0pt}{\isacharbrackright}{\kern0pt}\ asm\ \isacommand{by}\isamarkupfalse%
\ auto\isanewline
\ \ \ \ \isacommand{also}\isamarkupfalse%
\ \isacommand{have}\isamarkupfalse%
\ {\isachardoublequoteopen}{\isachardot}{\kern0pt}{\isachardot}{\kern0pt}{\isachardot}{\kern0pt}\ {\isacharequal}{\kern0pt}\ set{\isacharunderscore}{\kern0pt}lebesgue{\isacharunderscore}{\kern0pt}integral\ {\isacharquery}{\kern0pt}F\ A\ {\isacharparenleft}{\kern0pt}cond{\isacharunderscore}{\kern0pt}exp\ M\ F\ f{\isacharparenright}{\kern0pt}{\isachardoublequoteclose}\ \isacommand{unfolding}\isamarkupfalse%
\ set{\isacharunderscore}{\kern0pt}lebesgue{\isacharunderscore}{\kern0pt}integral{\isacharunderscore}{\kern0pt}def\ \isacommand{using}\isamarkupfalse%
\ asm\ borel{\isacharunderscore}{\kern0pt}measurable{\isacharunderscore}{\kern0pt}cond{\isacharunderscore}{\kern0pt}exp\ \isacommand{by}\isamarkupfalse%
\ {\isacharparenleft}{\kern0pt}intro\ integral{\isacharunderscore}{\kern0pt}subalgebra{\isadigit{2}}{\isacharbrackleft}{\kern0pt}OF\ subalg{\isacharcomma}{\kern0pt}\ symmetric{\isacharbrackright}{\kern0pt}{\isacharcomma}{\kern0pt}\ simp{\isacharparenright}{\kern0pt}\isanewline
\ \ \ \ \isacommand{finally}\isamarkupfalse%
\ \isacommand{have}\isamarkupfalse%
\ {\isachardoublequoteopen}{\isacharparenleft}{\kern0pt}{\isadigit{1}}\ {\isacharslash}{\kern0pt}\ measure\ {\isacharquery}{\kern0pt}F\ A{\isacharparenright}{\kern0pt}\ {\isacharasterisk}{\kern0pt}\isactrlsub R\ set{\isacharunderscore}{\kern0pt}lebesgue{\isacharunderscore}{\kern0pt}integral\ {\isacharquery}{\kern0pt}F\ A\ {\isacharparenleft}{\kern0pt}cond{\isacharunderscore}{\kern0pt}exp\ M\ F\ f{\isacharparenright}{\kern0pt}\ {\isasymin}\ {\isacharbraceleft}{\kern0pt}c{\isachardot}{\kern0pt}{\isachardot}{\kern0pt}{\isacharbraceright}{\kern0pt}{\isachardoublequoteclose}\ \isacommand{using}\isamarkupfalse%
\ asm\ subalg\ M{\isacharunderscore}{\kern0pt}A\ \isacommand{by}\isamarkupfalse%
\ {\isacharparenleft}{\kern0pt}auto\ simp\ add{\isacharcolon}{\kern0pt}\ set{\isacharunderscore}{\kern0pt}integral{\isacharunderscore}{\kern0pt}const\ subalgebra{\isacharunderscore}{\kern0pt}def\ intro{\isacharbang}{\kern0pt}{\isacharcolon}{\kern0pt}\ pos{\isacharunderscore}{\kern0pt}divideR{\isacharunderscore}{\kern0pt}le{\isacharunderscore}{\kern0pt}eq{\isacharbrackleft}{\kern0pt}THEN\ iffD{\isadigit{1}}{\isacharbrackright}{\kern0pt}{\isacharparenright}{\kern0pt}\ \isanewline
\ \ \isacommand{{\isacharbraceright}{\kern0pt}}\isamarkupfalse%
\isanewline
\ \ \isacommand{thus}\isamarkupfalse%
\ {\isacharquery}{\kern0pt}thesis\ \isacommand{using}\isamarkupfalse%
\ AE{\isacharunderscore}{\kern0pt}restr{\isacharunderscore}{\kern0pt}to{\isacharunderscore}{\kern0pt}subalg{\isacharbrackleft}{\kern0pt}OF\ subalg{\isacharbrackright}{\kern0pt}\ averaging{\isacharunderscore}{\kern0pt}theorem{\isacharbrackleft}{\kern0pt}OF\ integrable{\isacharunderscore}{\kern0pt}in{\isacharunderscore}{\kern0pt}subalg\ closed{\isacharunderscore}{\kern0pt}atLeast{\isacharcomma}{\kern0pt}\ OF\ subalg\ borel{\isacharunderscore}{\kern0pt}measurable{\isacharunderscore}{\kern0pt}cond{\isacharunderscore}{\kern0pt}exp\ integrable{\isacharunderscore}{\kern0pt}cond{\isacharunderscore}{\kern0pt}exp{\isacharbrackright}{\kern0pt}\ \isacommand{by}\isamarkupfalse%
\ auto\isanewline
\isacommand{qed}\isamarkupfalse%
%
\endisatagproof
{\isafoldproof}%
%
\isadelimproof
\isanewline
%
\endisadelimproof
\isanewline
\isacommand{corollary}\isamarkupfalse%
\ cond{\isacharunderscore}{\kern0pt}exp{\isacharunderscore}{\kern0pt}le{\isacharunderscore}{\kern0pt}c{\isacharcolon}{\kern0pt}\isanewline
\ \ \isakeyword{fixes}\ f\ {\isacharcolon}{\kern0pt}{\isacharcolon}{\kern0pt}\ {\isachardoublequoteopen}{\isacharprime}{\kern0pt}a\ {\isasymRightarrow}\ {\isacharprime}{\kern0pt}b\ {\isacharcolon}{\kern0pt}{\isacharcolon}{\kern0pt}\ {\isacharbraceleft}{\kern0pt}second{\isacharunderscore}{\kern0pt}countable{\isacharunderscore}{\kern0pt}topology{\isacharcomma}{\kern0pt}\ banach{\isacharcomma}{\kern0pt}\ linorder{\isacharunderscore}{\kern0pt}topology{\isacharcomma}{\kern0pt}\ ordered{\isacharunderscore}{\kern0pt}real{\isacharunderscore}{\kern0pt}vector{\isacharbraceright}{\kern0pt}{\isachardoublequoteclose}\isanewline
\ \ \isakeyword{assumes}\ {\isachardoublequoteopen}integrable\ M\ f{\isachardoublequoteclose}\isanewline
\ \ \ \ \ \ \isakeyword{and}\ {\isachardoublequoteopen}AE\ x\ in\ M{\isachardot}{\kern0pt}\ f\ x\ {\isasymle}\ c{\isachardoublequoteclose}\isanewline
\ \ \isakeyword{shows}\ {\isachardoublequoteopen}AE\ x\ in\ M{\isachardot}{\kern0pt}\ cond{\isacharunderscore}{\kern0pt}exp\ M\ F\ f\ x\ {\isasymle}\ c{\isachardoublequoteclose}\isanewline
%
\isadelimproof
%
\endisadelimproof
%
\isatagproof
\isacommand{proof}\isamarkupfalse%
\ {\isacharminus}{\kern0pt}\isanewline
\ \ \isacommand{have}\isamarkupfalse%
\ {\isachardoublequoteopen}AE\ x\ in\ M{\isachardot}{\kern0pt}\ cond{\isacharunderscore}{\kern0pt}exp\ M\ F\ f\ x\ {\isacharequal}{\kern0pt}\ {\isacharminus}{\kern0pt}\ cond{\isacharunderscore}{\kern0pt}exp\ M\ F\ {\isacharparenleft}{\kern0pt}{\isasymlambda}x{\isachardot}{\kern0pt}\ {\isacharminus}{\kern0pt}\ f\ x{\isacharparenright}{\kern0pt}\ x{\isachardoublequoteclose}\ \isacommand{using}\isamarkupfalse%
\ cond{\isacharunderscore}{\kern0pt}exp{\isacharunderscore}{\kern0pt}uminus{\isacharbrackleft}{\kern0pt}OF\ assms{\isacharparenleft}{\kern0pt}{\isadigit{1}}{\isacharparenright}{\kern0pt}{\isacharbrackright}{\kern0pt}\ \isacommand{by}\isamarkupfalse%
\ force\isanewline
\ \ \isacommand{moreover}\isamarkupfalse%
\ \isacommand{have}\isamarkupfalse%
\ {\isachardoublequoteopen}AE\ x\ in\ M{\isachardot}{\kern0pt}\ cond{\isacharunderscore}{\kern0pt}exp\ M\ F\ {\isacharparenleft}{\kern0pt}{\isasymlambda}x{\isachardot}{\kern0pt}\ {\isacharminus}{\kern0pt}\ f\ x{\isacharparenright}{\kern0pt}\ x\ {\isasymge}\ {\isacharminus}{\kern0pt}\ c{\isachardoublequoteclose}\ \isacommand{using}\isamarkupfalse%
\ assms\ \isacommand{by}\isamarkupfalse%
\ {\isacharparenleft}{\kern0pt}intro\ cond{\isacharunderscore}{\kern0pt}exp{\isacharunderscore}{\kern0pt}ge{\isacharunderscore}{\kern0pt}c{\isacharparenright}{\kern0pt}\ auto\isanewline
\ \ \isacommand{ultimately}\isamarkupfalse%
\ \isacommand{show}\isamarkupfalse%
\ {\isacharquery}{\kern0pt}thesis\ \isacommand{by}\isamarkupfalse%
\ {\isacharparenleft}{\kern0pt}force\ simp\ add{\isacharcolon}{\kern0pt}\ minus{\isacharunderscore}{\kern0pt}le{\isacharunderscore}{\kern0pt}iff{\isacharparenright}{\kern0pt}\isanewline
\isacommand{qed}\isamarkupfalse%
%
\endisatagproof
{\isafoldproof}%
%
\isadelimproof
\isanewline
%
\endisadelimproof
\isanewline
\isacommand{corollary}\isamarkupfalse%
\ cond{\isacharunderscore}{\kern0pt}exp{\isacharunderscore}{\kern0pt}mono{\isacharcolon}{\kern0pt}\isanewline
\ \ \isakeyword{fixes}\ f\ {\isacharcolon}{\kern0pt}{\isacharcolon}{\kern0pt}\ {\isachardoublequoteopen}{\isacharprime}{\kern0pt}a\ {\isasymRightarrow}\ {\isacharprime}{\kern0pt}b\ {\isacharcolon}{\kern0pt}{\isacharcolon}{\kern0pt}\ {\isacharbraceleft}{\kern0pt}second{\isacharunderscore}{\kern0pt}countable{\isacharunderscore}{\kern0pt}topology{\isacharcomma}{\kern0pt}\ banach{\isacharcomma}{\kern0pt}\ linorder{\isacharunderscore}{\kern0pt}topology{\isacharcomma}{\kern0pt}\ ordered{\isacharunderscore}{\kern0pt}real{\isacharunderscore}{\kern0pt}vector{\isacharbraceright}{\kern0pt}{\isachardoublequoteclose}\isanewline
\ \ \isakeyword{assumes}\ {\isachardoublequoteopen}integrable\ M\ f{\isachardoublequoteclose}\ {\isachardoublequoteopen}integrable\ M\ g{\isachardoublequoteclose}\ {\isachardoublequoteopen}AE\ x\ in\ M{\isachardot}{\kern0pt}\ f\ x\ {\isasymle}\ g\ x{\isachardoublequoteclose}\isanewline
\ \ \isakeyword{shows}\ {\isachardoublequoteopen}AE\ x\ in\ M{\isachardot}{\kern0pt}\ cond{\isacharunderscore}{\kern0pt}exp\ M\ F\ f\ x\ {\isasymle}\ cond{\isacharunderscore}{\kern0pt}exp\ M\ F\ g\ x{\isachardoublequoteclose}\isanewline
%
\isadelimproof
\ \ %
\endisadelimproof
%
\isatagproof
\isacommand{using}\isamarkupfalse%
\ cond{\isacharunderscore}{\kern0pt}exp{\isacharunderscore}{\kern0pt}le{\isacharunderscore}{\kern0pt}c{\isacharbrackleft}{\kern0pt}OF\ Bochner{\isacharunderscore}{\kern0pt}Integration{\isachardot}{\kern0pt}integrable{\isacharunderscore}{\kern0pt}diff{\isacharcomma}{\kern0pt}\ OF\ assms{\isacharparenleft}{\kern0pt}{\isadigit{1}}{\isacharcomma}{\kern0pt}{\isadigit{2}}{\isacharparenright}{\kern0pt}{\isacharcomma}{\kern0pt}\ of\ {\isadigit{0}}{\isacharbrackright}{\kern0pt}\ \isanewline
\ \ \ \ \ \ \ \ cond{\isacharunderscore}{\kern0pt}exp{\isacharunderscore}{\kern0pt}diff{\isacharbrackleft}{\kern0pt}OF\ assms{\isacharparenleft}{\kern0pt}{\isadigit{1}}{\isacharcomma}{\kern0pt}{\isadigit{2}}{\isacharparenright}{\kern0pt}{\isacharbrackright}{\kern0pt}\ assms{\isacharparenleft}{\kern0pt}{\isadigit{3}}{\isacharparenright}{\kern0pt}\ \isacommand{by}\isamarkupfalse%
\ auto%
\endisatagproof
{\isafoldproof}%
%
\isadelimproof
\isanewline
%
\endisadelimproof
\ \ \ \ \ \ \ \ \ \ \ \ \ \ \ \ \ \ \ \ \ \ \ \ \ \ \ \ \ \ \ \ \ \ \ \ \ \ \isanewline
\isacommand{corollary}\isamarkupfalse%
\ cond{\isacharunderscore}{\kern0pt}exp{\isacharunderscore}{\kern0pt}min{\isacharcolon}{\kern0pt}\isanewline
\ \ \isakeyword{fixes}\ f\ {\isacharcolon}{\kern0pt}{\isacharcolon}{\kern0pt}\ {\isachardoublequoteopen}{\isacharprime}{\kern0pt}a\ {\isasymRightarrow}\ {\isacharprime}{\kern0pt}b\ {\isacharcolon}{\kern0pt}{\isacharcolon}{\kern0pt}\ {\isacharbraceleft}{\kern0pt}second{\isacharunderscore}{\kern0pt}countable{\isacharunderscore}{\kern0pt}topology{\isacharcomma}{\kern0pt}\ banach{\isacharcomma}{\kern0pt}\ linorder{\isacharunderscore}{\kern0pt}topology{\isacharcomma}{\kern0pt}\ ordered{\isacharunderscore}{\kern0pt}real{\isacharunderscore}{\kern0pt}vector{\isacharbraceright}{\kern0pt}{\isachardoublequoteclose}\isanewline
\ \ \isakeyword{assumes}\ {\isachardoublequoteopen}integrable\ M\ f{\isachardoublequoteclose}\ {\isachardoublequoteopen}integrable\ M\ g{\isachardoublequoteclose}\isanewline
\ \ \isakeyword{shows}\ {\isachardoublequoteopen}AE\ {\isasymxi}\ in\ M{\isachardot}{\kern0pt}\ cond{\isacharunderscore}{\kern0pt}exp\ M\ F\ {\isacharparenleft}{\kern0pt}{\isasymlambda}x{\isachardot}{\kern0pt}\ min\ {\isacharparenleft}{\kern0pt}f\ x{\isacharparenright}{\kern0pt}\ {\isacharparenleft}{\kern0pt}g\ x{\isacharparenright}{\kern0pt}{\isacharparenright}{\kern0pt}\ {\isasymxi}\ {\isasymle}\ min\ {\isacharparenleft}{\kern0pt}cond{\isacharunderscore}{\kern0pt}exp\ M\ F\ f\ {\isasymxi}{\isacharparenright}{\kern0pt}\ {\isacharparenleft}{\kern0pt}cond{\isacharunderscore}{\kern0pt}exp\ M\ F\ g\ {\isasymxi}{\isacharparenright}{\kern0pt}{\isachardoublequoteclose}\isanewline
%
\isadelimproof
%
\endisadelimproof
%
\isatagproof
\isacommand{proof}\isamarkupfalse%
\ {\isacharminus}{\kern0pt}\isanewline
\ \ \isacommand{have}\isamarkupfalse%
\ {\isachardoublequoteopen}AE\ {\isasymxi}\ in\ M{\isachardot}{\kern0pt}\ cond{\isacharunderscore}{\kern0pt}exp\ M\ F\ {\isacharparenleft}{\kern0pt}{\isasymlambda}x{\isachardot}{\kern0pt}\ min\ {\isacharparenleft}{\kern0pt}f\ x{\isacharparenright}{\kern0pt}\ {\isacharparenleft}{\kern0pt}g\ x{\isacharparenright}{\kern0pt}{\isacharparenright}{\kern0pt}\ {\isasymxi}\ {\isasymle}\ cond{\isacharunderscore}{\kern0pt}exp\ M\ F\ f\ {\isasymxi}{\isachardoublequoteclose}\ \isacommand{by}\isamarkupfalse%
\ {\isacharparenleft}{\kern0pt}intro\ cond{\isacharunderscore}{\kern0pt}exp{\isacharunderscore}{\kern0pt}mono\ integrable{\isacharunderscore}{\kern0pt}min\ assms{\isacharcomma}{\kern0pt}\ simp{\isacharparenright}{\kern0pt}\isanewline
\ \ \isacommand{moreover}\isamarkupfalse%
\ \isacommand{have}\isamarkupfalse%
\ {\isachardoublequoteopen}AE\ {\isasymxi}\ in\ M{\isachardot}{\kern0pt}\ cond{\isacharunderscore}{\kern0pt}exp\ M\ F\ {\isacharparenleft}{\kern0pt}{\isasymlambda}x{\isachardot}{\kern0pt}\ min\ {\isacharparenleft}{\kern0pt}f\ x{\isacharparenright}{\kern0pt}\ {\isacharparenleft}{\kern0pt}g\ x{\isacharparenright}{\kern0pt}{\isacharparenright}{\kern0pt}\ {\isasymxi}\ {\isasymle}\ cond{\isacharunderscore}{\kern0pt}exp\ M\ F\ g\ {\isasymxi}{\isachardoublequoteclose}\ \isacommand{by}\isamarkupfalse%
\ {\isacharparenleft}{\kern0pt}intro\ cond{\isacharunderscore}{\kern0pt}exp{\isacharunderscore}{\kern0pt}mono\ integrable{\isacharunderscore}{\kern0pt}min\ assms{\isacharcomma}{\kern0pt}\ simp{\isacharparenright}{\kern0pt}\isanewline
\ \ \isacommand{ultimately}\isamarkupfalse%
\ \isacommand{show}\isamarkupfalse%
\ {\isachardoublequoteopen}AE\ {\isasymxi}\ in\ M{\isachardot}{\kern0pt}\ cond{\isacharunderscore}{\kern0pt}exp\ M\ F\ {\isacharparenleft}{\kern0pt}{\isasymlambda}x{\isachardot}{\kern0pt}\ min\ {\isacharparenleft}{\kern0pt}f\ x{\isacharparenright}{\kern0pt}\ {\isacharparenleft}{\kern0pt}g\ x{\isacharparenright}{\kern0pt}{\isacharparenright}{\kern0pt}\ {\isasymxi}\ {\isasymle}\ min\ {\isacharparenleft}{\kern0pt}cond{\isacharunderscore}{\kern0pt}exp\ M\ F\ f\ {\isasymxi}{\isacharparenright}{\kern0pt}\ {\isacharparenleft}{\kern0pt}cond{\isacharunderscore}{\kern0pt}exp\ M\ F\ g\ {\isasymxi}{\isacharparenright}{\kern0pt}{\isachardoublequoteclose}\ \isacommand{by}\isamarkupfalse%
\ fastforce\isanewline
\isacommand{qed}\isamarkupfalse%
%
\endisatagproof
{\isafoldproof}%
%
\isadelimproof
\isanewline
%
\endisadelimproof
\isanewline
\isacommand{corollary}\isamarkupfalse%
\ cond{\isacharunderscore}{\kern0pt}exp{\isacharunderscore}{\kern0pt}max{\isacharcolon}{\kern0pt}\isanewline
\ \ \isakeyword{fixes}\ f\ {\isacharcolon}{\kern0pt}{\isacharcolon}{\kern0pt}\ {\isachardoublequoteopen}{\isacharprime}{\kern0pt}a\ {\isasymRightarrow}\ {\isacharprime}{\kern0pt}b\ {\isacharcolon}{\kern0pt}{\isacharcolon}{\kern0pt}\ {\isacharbraceleft}{\kern0pt}second{\isacharunderscore}{\kern0pt}countable{\isacharunderscore}{\kern0pt}topology{\isacharcomma}{\kern0pt}\ banach{\isacharcomma}{\kern0pt}\ linorder{\isacharunderscore}{\kern0pt}topology{\isacharcomma}{\kern0pt}\ ordered{\isacharunderscore}{\kern0pt}real{\isacharunderscore}{\kern0pt}vector{\isacharbraceright}{\kern0pt}{\isachardoublequoteclose}\isanewline
\ \ \isakeyword{assumes}\ {\isachardoublequoteopen}integrable\ M\ f{\isachardoublequoteclose}\ {\isachardoublequoteopen}integrable\ M\ g{\isachardoublequoteclose}\isanewline
\ \ \isakeyword{shows}\ {\isachardoublequoteopen}AE\ {\isasymxi}\ in\ M{\isachardot}{\kern0pt}\ cond{\isacharunderscore}{\kern0pt}exp\ M\ F\ {\isacharparenleft}{\kern0pt}{\isasymlambda}x{\isachardot}{\kern0pt}\ max\ {\isacharparenleft}{\kern0pt}f\ x{\isacharparenright}{\kern0pt}\ {\isacharparenleft}{\kern0pt}g\ x{\isacharparenright}{\kern0pt}{\isacharparenright}{\kern0pt}\ {\isasymxi}\ {\isasymge}\ max\ {\isacharparenleft}{\kern0pt}cond{\isacharunderscore}{\kern0pt}exp\ M\ F\ f\ {\isasymxi}{\isacharparenright}{\kern0pt}\ {\isacharparenleft}{\kern0pt}cond{\isacharunderscore}{\kern0pt}exp\ M\ F\ g\ {\isasymxi}{\isacharparenright}{\kern0pt}{\isachardoublequoteclose}\isanewline
%
\isadelimproof
%
\endisadelimproof
%
\isatagproof
\isacommand{proof}\isamarkupfalse%
\ {\isacharminus}{\kern0pt}\isanewline
\ \ \isacommand{have}\isamarkupfalse%
\ {\isachardoublequoteopen}AE\ {\isasymxi}\ in\ M{\isachardot}{\kern0pt}\ cond{\isacharunderscore}{\kern0pt}exp\ M\ F\ {\isacharparenleft}{\kern0pt}{\isasymlambda}x{\isachardot}{\kern0pt}\ max\ {\isacharparenleft}{\kern0pt}f\ x{\isacharparenright}{\kern0pt}\ {\isacharparenleft}{\kern0pt}g\ x{\isacharparenright}{\kern0pt}{\isacharparenright}{\kern0pt}\ {\isasymxi}\ {\isasymge}\ cond{\isacharunderscore}{\kern0pt}exp\ M\ F\ f\ {\isasymxi}{\isachardoublequoteclose}\ \isacommand{by}\isamarkupfalse%
\ {\isacharparenleft}{\kern0pt}intro\ cond{\isacharunderscore}{\kern0pt}exp{\isacharunderscore}{\kern0pt}mono\ integrable{\isacharunderscore}{\kern0pt}max\ assms{\isacharcomma}{\kern0pt}\ simp{\isacharparenright}{\kern0pt}\isanewline
\ \ \isacommand{moreover}\isamarkupfalse%
\ \isacommand{have}\isamarkupfalse%
\ {\isachardoublequoteopen}AE\ {\isasymxi}\ in\ M{\isachardot}{\kern0pt}\ cond{\isacharunderscore}{\kern0pt}exp\ M\ F\ {\isacharparenleft}{\kern0pt}{\isasymlambda}x{\isachardot}{\kern0pt}\ max\ {\isacharparenleft}{\kern0pt}f\ x{\isacharparenright}{\kern0pt}\ {\isacharparenleft}{\kern0pt}g\ x{\isacharparenright}{\kern0pt}{\isacharparenright}{\kern0pt}\ {\isasymxi}\ {\isasymge}\ cond{\isacharunderscore}{\kern0pt}exp\ M\ F\ g\ {\isasymxi}{\isachardoublequoteclose}\ \isacommand{by}\isamarkupfalse%
\ {\isacharparenleft}{\kern0pt}intro\ cond{\isacharunderscore}{\kern0pt}exp{\isacharunderscore}{\kern0pt}mono\ integrable{\isacharunderscore}{\kern0pt}max\ assms{\isacharcomma}{\kern0pt}\ simp{\isacharparenright}{\kern0pt}\isanewline
\ \ \isacommand{ultimately}\isamarkupfalse%
\ \isacommand{show}\isamarkupfalse%
\ {\isachardoublequoteopen}AE\ {\isasymxi}\ in\ M{\isachardot}{\kern0pt}\ cond{\isacharunderscore}{\kern0pt}exp\ M\ F\ {\isacharparenleft}{\kern0pt}{\isasymlambda}x{\isachardot}{\kern0pt}\ max\ {\isacharparenleft}{\kern0pt}f\ x{\isacharparenright}{\kern0pt}\ {\isacharparenleft}{\kern0pt}g\ x{\isacharparenright}{\kern0pt}{\isacharparenright}{\kern0pt}\ {\isasymxi}\ {\isasymge}\ max\ {\isacharparenleft}{\kern0pt}cond{\isacharunderscore}{\kern0pt}exp\ M\ F\ f\ {\isasymxi}{\isacharparenright}{\kern0pt}\ {\isacharparenleft}{\kern0pt}cond{\isacharunderscore}{\kern0pt}exp\ M\ F\ g\ {\isasymxi}{\isacharparenright}{\kern0pt}{\isachardoublequoteclose}\ \isacommand{by}\isamarkupfalse%
\ fastforce\isanewline
\isacommand{qed}\isamarkupfalse%
%
\endisatagproof
{\isafoldproof}%
%
\isadelimproof
\isanewline
%
\endisadelimproof
\isanewline
\isacommand{corollary}\isamarkupfalse%
\ cond{\isacharunderscore}{\kern0pt}exp{\isacharunderscore}{\kern0pt}inf{\isacharcolon}{\kern0pt}\isanewline
\ \ \isakeyword{fixes}\ f\ {\isacharcolon}{\kern0pt}{\isacharcolon}{\kern0pt}\ {\isachardoublequoteopen}{\isacharprime}{\kern0pt}a\ {\isasymRightarrow}\ {\isacharprime}{\kern0pt}b\ {\isacharcolon}{\kern0pt}{\isacharcolon}{\kern0pt}\ {\isacharbraceleft}{\kern0pt}second{\isacharunderscore}{\kern0pt}countable{\isacharunderscore}{\kern0pt}topology{\isacharcomma}{\kern0pt}\ banach{\isacharcomma}{\kern0pt}\ linorder{\isacharunderscore}{\kern0pt}topology{\isacharcomma}{\kern0pt}\ ordered{\isacharunderscore}{\kern0pt}real{\isacharunderscore}{\kern0pt}vector{\isacharcomma}{\kern0pt}\ lattice{\isacharbraceright}{\kern0pt}{\isachardoublequoteclose}\isanewline
\ \ \isakeyword{assumes}\ {\isachardoublequoteopen}integrable\ M\ f{\isachardoublequoteclose}\ {\isachardoublequoteopen}integrable\ M\ g{\isachardoublequoteclose}\isanewline
\ \ \isakeyword{shows}\ {\isachardoublequoteopen}AE\ {\isasymxi}\ in\ M{\isachardot}{\kern0pt}\ cond{\isacharunderscore}{\kern0pt}exp\ M\ F\ {\isacharparenleft}{\kern0pt}{\isasymlambda}x{\isachardot}{\kern0pt}\ inf\ {\isacharparenleft}{\kern0pt}f\ x{\isacharparenright}{\kern0pt}\ {\isacharparenleft}{\kern0pt}g\ x{\isacharparenright}{\kern0pt}{\isacharparenright}{\kern0pt}\ {\isasymxi}\ {\isasymle}\ inf\ {\isacharparenleft}{\kern0pt}cond{\isacharunderscore}{\kern0pt}exp\ M\ F\ f\ {\isasymxi}{\isacharparenright}{\kern0pt}\ {\isacharparenleft}{\kern0pt}cond{\isacharunderscore}{\kern0pt}exp\ M\ F\ g\ {\isasymxi}{\isacharparenright}{\kern0pt}{\isachardoublequoteclose}\isanewline
%
\isadelimproof
\ \ %
\endisadelimproof
%
\isatagproof
\isacommand{unfolding}\isamarkupfalse%
\ inf{\isacharunderscore}{\kern0pt}min\ \isacommand{using}\isamarkupfalse%
\ assms\ \isacommand{by}\isamarkupfalse%
\ {\isacharparenleft}{\kern0pt}rule\ cond{\isacharunderscore}{\kern0pt}exp{\isacharunderscore}{\kern0pt}min{\isacharparenright}{\kern0pt}%
\endisatagproof
{\isafoldproof}%
%
\isadelimproof
\isanewline
%
\endisadelimproof
\isanewline
\isacommand{corollary}\isamarkupfalse%
\ cond{\isacharunderscore}{\kern0pt}exp{\isacharunderscore}{\kern0pt}sup{\isacharcolon}{\kern0pt}\isanewline
\ \ \isakeyword{fixes}\ f\ {\isacharcolon}{\kern0pt}{\isacharcolon}{\kern0pt}\ {\isachardoublequoteopen}{\isacharprime}{\kern0pt}a\ {\isasymRightarrow}\ {\isacharprime}{\kern0pt}b\ {\isacharcolon}{\kern0pt}{\isacharcolon}{\kern0pt}\ {\isacharbraceleft}{\kern0pt}second{\isacharunderscore}{\kern0pt}countable{\isacharunderscore}{\kern0pt}topology{\isacharcomma}{\kern0pt}\ banach{\isacharcomma}{\kern0pt}\ linorder{\isacharunderscore}{\kern0pt}topology{\isacharcomma}{\kern0pt}\ ordered{\isacharunderscore}{\kern0pt}real{\isacharunderscore}{\kern0pt}vector{\isacharcomma}{\kern0pt}\ lattice{\isacharbraceright}{\kern0pt}{\isachardoublequoteclose}\isanewline
\ \ \isakeyword{assumes}\ {\isachardoublequoteopen}integrable\ M\ f{\isachardoublequoteclose}\ {\isachardoublequoteopen}integrable\ M\ g{\isachardoublequoteclose}\isanewline
\ \ \isakeyword{shows}\ {\isachardoublequoteopen}AE\ {\isasymxi}\ in\ M{\isachardot}{\kern0pt}\ cond{\isacharunderscore}{\kern0pt}exp\ M\ F\ {\isacharparenleft}{\kern0pt}{\isasymlambda}x{\isachardot}{\kern0pt}\ sup\ {\isacharparenleft}{\kern0pt}f\ x{\isacharparenright}{\kern0pt}\ {\isacharparenleft}{\kern0pt}g\ x{\isacharparenright}{\kern0pt}{\isacharparenright}{\kern0pt}\ {\isasymxi}\ {\isasymge}\ sup\ {\isacharparenleft}{\kern0pt}cond{\isacharunderscore}{\kern0pt}exp\ M\ F\ f\ {\isasymxi}{\isacharparenright}{\kern0pt}\ {\isacharparenleft}{\kern0pt}cond{\isacharunderscore}{\kern0pt}exp\ M\ F\ g\ {\isasymxi}{\isacharparenright}{\kern0pt}{\isachardoublequoteclose}\isanewline
%
\isadelimproof
\ \ %
\endisadelimproof
%
\isatagproof
\isacommand{unfolding}\isamarkupfalse%
\ sup{\isacharunderscore}{\kern0pt}max\ \isacommand{using}\isamarkupfalse%
\ assms\ \isacommand{by}\isamarkupfalse%
\ {\isacharparenleft}{\kern0pt}rule\ cond{\isacharunderscore}{\kern0pt}exp{\isacharunderscore}{\kern0pt}max{\isacharparenright}{\kern0pt}%
\endisatagproof
{\isafoldproof}%
%
\isadelimproof
\isanewline
%
\endisadelimproof
\isanewline
\isacommand{end}\isamarkupfalse%
\isanewline
%
\isadelimtheory
\isanewline
%
\endisadelimtheory
%
\isatagtheory
\isacommand{end}\isamarkupfalse%
%
\endisatagtheory
{\isafoldtheory}%
%
\isadelimtheory
%
\endisadelimtheory
%
\end{isabellebody}%
\endinput
%:%file=Conditional_Expectation_Banach.tex%:%
%:%10=1%:%
%:%11=1%:%
%:%12=2%:%
%:%13=3%:%
%:%27=5%:%
%:%37=7%:%
%:%38=7%:%
%:%39=8%:%
%:%42=11%:%
%:%43=12%:%
%:%44=13%:%
%:%45=13%:%
%:%46=14%:%
%:%47=15%:%
%:%48=16%:%
%:%49=17%:%
%:%50=18%:%
%:%53=19%:%
%:%57=19%:%
%:%58=19%:%
%:%59=19%:%
%:%60=19%:%
%:%65=19%:%
%:%68=20%:%
%:%69=21%:%
%:%70=21%:%
%:%71=22%:%
%:%72=23%:%
%:%73=24%:%
%:%74=25%:%
%:%75=26%:%
%:%78=27%:%
%:%82=27%:%
%:%83=27%:%
%:%84=27%:%
%:%85=27%:%
%:%90=27%:%
%:%93=28%:%
%:%94=31%:%
%:%95=32%:%
%:%96=33%:%
%:%97=33%:%
%:%98=34%:%
%:%99=35%:%
%:%100=36%:%
%:%101=36%:%
%:%104=37%:%
%:%108=37%:%
%:%109=37%:%
%:%114=37%:%
%:%117=38%:%
%:%118=39%:%
%:%119=39%:%
%:%122=40%:%
%:%126=40%:%
%:%127=40%:%
%:%132=40%:%
%:%135=41%:%
%:%136=42%:%
%:%137=42%:%
%:%138=43%:%
%:%139=44%:%
%:%141=44%:%
%:%145=44%:%
%:%146=44%:%
%:%147=44%:%
%:%154=44%:%
%:%155=45%:%
%:%156=46%:%
%:%157=46%:%
%:%158=47%:%
%:%159=48%:%
%:%160=49%:%
%:%161=49%:%
%:%164=50%:%
%:%168=50%:%
%:%169=50%:%
%:%174=50%:%
%:%177=51%:%
%:%178=52%:%
%:%179=52%:%
%:%180=53%:%
%:%181=54%:%
%:%184=55%:%
%:%188=55%:%
%:%189=55%:%
%:%190=55%:%
%:%191=55%:%
%:%196=55%:%
%:%199=56%:%
%:%200=57%:%
%:%201=57%:%
%:%202=58%:%
%:%203=59%:%
%:%204=60%:%
%:%207=61%:%
%:%211=61%:%
%:%212=61%:%
%:%217=61%:%
%:%220=62%:%
%:%221=63%:%
%:%222=63%:%
%:%223=64%:%
%:%224=65%:%
%:%225=66%:%
%:%226=67%:%
%:%233=68%:%
%:%234=68%:%
%:%235=69%:%
%:%236=69%:%
%:%237=69%:%
%:%238=69%:%
%:%239=70%:%
%:%240=70%:%
%:%241=71%:%
%:%242=71%:%
%:%243=71%:%
%:%244=72%:%
%:%245=72%:%
%:%246=73%:%
%:%247=73%:%
%:%248=73%:%
%:%249=74%:%
%:%250=74%:%
%:%251=74%:%
%:%252=74%:%
%:%253=74%:%
%:%254=75%:%
%:%255=75%:%
%:%256=75%:%
%:%257=75%:%
%:%258=76%:%
%:%259=76%:%
%:%260=76%:%
%:%261=76%:%
%:%262=76%:%
%:%263=77%:%
%:%264=77%:%
%:%265=77%:%
%:%266=77%:%
%:%267=77%:%
%:%268=78%:%
%:%269=78%:%
%:%270=78%:%
%:%271=78%:%
%:%272=79%:%
%:%273=79%:%
%:%274=80%:%
%:%275=80%:%
%:%276=80%:%
%:%277=80%:%
%:%278=81%:%
%:%279=81%:%
%:%280=81%:%
%:%281=81%:%
%:%282=81%:%
%:%283=82%:%
%:%289=82%:%
%:%292=83%:%
%:%293=84%:%
%:%294=84%:%
%:%295=85%:%
%:%296=86%:%
%:%297=87%:%
%:%298=88%:%
%:%299=89%:%
%:%300=90%:%
%:%303=91%:%
%:%307=91%:%
%:%308=91%:%
%:%313=91%:%
%:%316=92%:%
%:%317=93%:%
%:%318=93%:%
%:%319=94%:%
%:%320=95%:%
%:%321=96%:%
%:%322=97%:%
%:%325=98%:%
%:%329=98%:%
%:%330=98%:%
%:%339=100%:%
%:%341=102%:%
%:%342=102%:%
%:%343=103%:%
%:%344=104%:%
%:%351=105%:%
%:%352=105%:%
%:%353=106%:%
%:%354=106%:%
%:%355=107%:%
%:%356=107%:%
%:%357=107%:%
%:%358=108%:%
%:%359=108%:%
%:%360=108%:%
%:%361=108%:%
%:%362=108%:%
%:%363=109%:%
%:%364=109%:%
%:%369=109%:%
%:%372=110%:%
%:%373=111%:%
%:%374=111%:%
%:%375=112%:%
%:%376=113%:%
%:%377=114%:%
%:%384=115%:%
%:%385=115%:%
%:%386=116%:%
%:%387=116%:%
%:%388=117%:%
%:%389=117%:%
%:%390=117%:%
%:%391=117%:%
%:%392=117%:%
%:%393=118%:%
%:%394=118%:%
%:%395=118%:%
%:%396=118%:%
%:%397=119%:%
%:%398=119%:%
%:%399=120%:%
%:%400=120%:%
%:%401=121%:%
%:%402=121%:%
%:%403=121%:%
%:%404=121%:%
%:%405=121%:%
%:%406=122%:%
%:%407=122%:%
%:%408=122%:%
%:%409=122%:%
%:%410=122%:%
%:%411=123%:%
%:%417=123%:%
%:%420=124%:%
%:%421=125%:%
%:%422=125%:%
%:%423=126%:%
%:%424=127%:%
%:%427=128%:%
%:%431=128%:%
%:%432=128%:%
%:%433=129%:%
%:%434=129%:%
%:%435=130%:%
%:%440=130%:%
%:%443=131%:%
%:%444=132%:%
%:%445=132%:%
%:%446=133%:%
%:%447=134%:%
%:%450=135%:%
%:%454=135%:%
%:%455=135%:%
%:%456=136%:%
%:%457=136%:%
%:%458=137%:%
%:%459=137%:%
%:%460=138%:%
%:%465=138%:%
%:%468=139%:%
%:%469=140%:%
%:%470=140%:%
%:%471=141%:%
%:%472=142%:%
%:%473=143%:%
%:%480=144%:%
%:%481=144%:%
%:%482=145%:%
%:%483=145%:%
%:%484=146%:%
%:%485=146%:%
%:%486=146%:%
%:%487=146%:%
%:%488=146%:%
%:%489=147%:%
%:%490=147%:%
%:%491=147%:%
%:%492=147%:%
%:%493=148%:%
%:%494=148%:%
%:%495=149%:%
%:%496=149%:%
%:%497=150%:%
%:%498=150%:%
%:%499=150%:%
%:%500=150%:%
%:%501=150%:%
%:%502=151%:%
%:%503=151%:%
%:%504=151%:%
%:%505=151%:%
%:%506=151%:%
%:%507=152%:%
%:%513=152%:%
%:%516=153%:%
%:%517=154%:%
%:%518=154%:%
%:%519=155%:%
%:%520=156%:%
%:%521=157%:%
%:%524=158%:%
%:%528=158%:%
%:%529=158%:%
%:%534=158%:%
%:%537=159%:%
%:%538=160%:%
%:%539=160%:%
%:%540=161%:%
%:%541=162%:%
%:%542=163%:%
%:%545=164%:%
%:%549=164%:%
%:%550=164%:%
%:%551=164%:%
%:%556=164%:%
%:%559=165%:%
%:%560=166%:%
%:%561=166%:%
%:%562=167%:%
%:%563=168%:%
%:%564=169%:%
%:%567=170%:%
%:%571=170%:%
%:%572=170%:%
%:%573=170%:%
%:%582=172%:%
%:%584=174%:%
%:%585=174%:%
%:%586=175%:%
%:%587=176%:%
%:%594=177%:%
%:%595=177%:%
%:%596=178%:%
%:%597=178%:%
%:%598=179%:%
%:%599=179%:%
%:%600=179%:%
%:%601=179%:%
%:%602=180%:%
%:%603=180%:%
%:%604=180%:%
%:%605=180%:%
%:%606=180%:%
%:%607=181%:%
%:%608=181%:%
%:%609=181%:%
%:%610=181%:%
%:%611=181%:%
%:%612=182%:%
%:%613=182%:%
%:%614=182%:%
%:%615=182%:%
%:%616=183%:%
%:%617=183%:%
%:%618=184%:%
%:%619=184%:%
%:%620=185%:%
%:%621=185%:%
%:%622=185%:%
%:%623=185%:%
%:%624=185%:%
%:%625=186%:%
%:%626=186%:%
%:%627=187%:%
%:%628=187%:%
%:%629=188%:%
%:%630=188%:%
%:%631=188%:%
%:%632=188%:%
%:%633=189%:%
%:%634=189%:%
%:%635=190%:%
%:%636=190%:%
%:%637=191%:%
%:%638=191%:%
%:%639=191%:%
%:%640=191%:%
%:%641=192%:%
%:%647=192%:%
%:%650=193%:%
%:%651=194%:%
%:%652=194%:%
%:%653=195%:%
%:%654=196%:%
%:%655=197%:%
%:%662=198%:%
%:%663=198%:%
%:%664=199%:%
%:%665=199%:%
%:%666=199%:%
%:%667=199%:%
%:%668=200%:%
%:%669=200%:%
%:%670=200%:%
%:%671=200%:%
%:%672=201%:%
%:%682=203%:%
%:%684=205%:%
%:%685=205%:%
%:%686=206%:%
%:%687=207%:%
%:%688=208%:%
%:%695=209%:%
%:%696=209%:%
%:%697=210%:%
%:%698=210%:%
%:%699=211%:%
%:%700=211%:%
%:%701=211%:%
%:%702=211%:%
%:%703=212%:%
%:%704=212%:%
%:%705=212%:%
%:%706=212%:%
%:%707=212%:%
%:%708=213%:%
%:%709=213%:%
%:%710=213%:%
%:%711=213%:%
%:%712=213%:%
%:%713=214%:%
%:%714=214%:%
%:%715=214%:%
%:%716=214%:%
%:%717=215%:%
%:%718=215%:%
%:%719=216%:%
%:%720=216%:%
%:%721=217%:%
%:%722=217%:%
%:%723=217%:%
%:%724=217%:%
%:%725=218%:%
%:%726=218%:%
%:%727=219%:%
%:%728=219%:%
%:%729=220%:%
%:%730=220%:%
%:%731=220%:%
%:%732=220%:%
%:%733=221%:%
%:%734=221%:%
%:%735=222%:%
%:%736=222%:%
%:%737=223%:%
%:%738=223%:%
%:%739=223%:%
%:%740=223%:%
%:%741=223%:%
%:%742=224%:%
%:%748=224%:%
%:%751=225%:%
%:%752=226%:%
%:%753=226%:%
%:%754=227%:%
%:%755=228%:%
%:%756=229%:%
%:%759=230%:%
%:%763=230%:%
%:%764=230%:%
%:%765=230%:%
%:%770=230%:%
%:%773=231%:%
%:%774=232%:%
%:%775=232%:%
%:%776=233%:%
%:%777=234%:%
%:%778=235%:%
%:%785=236%:%
%:%786=236%:%
%:%787=237%:%
%:%788=237%:%
%:%789=238%:%
%:%790=238%:%
%:%791=238%:%
%:%792=238%:%
%:%793=238%:%
%:%794=239%:%
%:%795=239%:%
%:%796=240%:%
%:%797=240%:%
%:%798=241%:%
%:%799=241%:%
%:%800=242%:%
%:%801=242%:%
%:%802=243%:%
%:%803=243%:%
%:%804=244%:%
%:%805=244%:%
%:%806=244%:%
%:%807=244%:%
%:%808=245%:%
%:%809=245%:%
%:%810=246%:%
%:%811=246%:%
%:%812=247%:%
%:%813=247%:%
%:%814=248%:%
%:%815=248%:%
%:%816=249%:%
%:%817=249%:%
%:%818=250%:%
%:%819=250%:%
%:%820=250%:%
%:%821=250%:%
%:%822=251%:%
%:%823=251%:%
%:%824=251%:%
%:%825=251%:%
%:%826=252%:%
%:%827=252%:%
%:%828=252%:%
%:%829=252%:%
%:%830=252%:%
%:%831=253%:%
%:%832=253%:%
%:%833=254%:%
%:%834=254%:%
%:%835=254%:%
%:%836=254%:%
%:%837=254%:%
%:%838=255%:%
%:%839=255%:%
%:%840=256%:%
%:%846=256%:%
%:%849=257%:%
%:%850=258%:%
%:%851=258%:%
%:%852=259%:%
%:%853=260%:%
%:%854=261%:%
%:%857=262%:%
%:%861=262%:%
%:%862=262%:%
%:%863=262%:%
%:%868=262%:%
%:%871=263%:%
%:%872=264%:%
%:%873=264%:%
%:%874=265%:%
%:%875=266%:%
%:%876=267%:%
%:%879=268%:%
%:%883=268%:%
%:%884=268%:%
%:%885=269%:%
%:%886=269%:%
%:%887=270%:%
%:%888=270%:%
%:%889=271%:%
%:%890=271%:%
%:%891=271%:%
%:%892=271%:%
%:%893=271%:%
%:%894=272%:%
%:%895=272%:%
%:%896=273%:%
%:%897=273%:%
%:%898=274%:%
%:%899=274%:%
%:%900=274%:%
%:%901=274%:%
%:%902=274%:%
%:%903=275%:%
%:%904=275%:%
%:%905=276%:%
%:%906=276%:%
%:%907=277%:%
%:%908=277%:%
%:%909=277%:%
%:%910=277%:%
%:%911=277%:%
%:%912=278%:%
%:%918=278%:%
%:%921=279%:%
%:%922=280%:%
%:%923=280%:%
%:%924=281%:%
%:%925=282%:%
%:%926=283%:%
%:%933=284%:%
%:%934=284%:%
%:%935=285%:%
%:%936=285%:%
%:%937=285%:%
%:%938=285%:%
%:%939=286%:%
%:%940=286%:%
%:%941=286%:%
%:%942=286%:%
%:%943=287%:%
%:%944=287%:%
%:%945=287%:%
%:%946=287%:%
%:%947=287%:%
%:%948=288%:%
%:%949=289%:%
%:%950=289%:%
%:%951=289%:%
%:%952=289%:%
%:%953=290%:%
%:%954=291%:%
%:%955=291%:%
%:%956=291%:%
%:%957=291%:%
%:%958=292%:%
%:%959=292%:%
%:%960=292%:%
%:%961=292%:%
%:%962=293%:%
%:%963=294%:%
%:%964=294%:%
%:%965=294%:%
%:%966=294%:%
%:%967=295%:%
%:%968=295%:%
%:%969=295%:%
%:%970=295%:%
%:%971=296%:%
%:%972=296%:%
%:%973=296%:%
%:%974=296%:%
%:%975=297%:%
%:%976=297%:%
%:%977=297%:%
%:%978=297%:%
%:%979=298%:%
%:%980=298%:%
%:%981=298%:%
%:%982=298%:%
%:%983=299%:%
%:%984=299%:%
%:%985=299%:%
%:%986=299%:%
%:%987=299%:%
%:%988=299%:%
%:%989=300%:%
%:%990=300%:%
%:%991=300%:%
%:%992=300%:%
%:%993=300%:%
%:%994=301%:%
%:%995=301%:%
%:%996=301%:%
%:%997=301%:%
%:%998=302%:%
%:%999=302%:%
%:%1000=302%:%
%:%1001=302%:%
%:%1002=303%:%
%:%1008=303%:%
%:%1011=304%:%
%:%1012=305%:%
%:%1013=305%:%
%:%1014=306%:%
%:%1015=307%:%
%:%1016=308%:%
%:%1019=309%:%
%:%1023=309%:%
%:%1024=309%:%
%:%1025=310%:%
%:%1026=310%:%
%:%1027=311%:%
%:%1028=311%:%
%:%1029=312%:%
%:%1030=312%:%
%:%1031=312%:%
%:%1032=313%:%
%:%1033=313%:%
%:%1034=313%:%
%:%1035=313%:%
%:%1036=314%:%
%:%1037=314%:%
%:%1038=314%:%
%:%1039=315%:%
%:%1040=315%:%
%:%1041=315%:%
%:%1042=315%:%
%:%1043=315%:%
%:%1044=316%:%
%:%1045=316%:%
%:%1046=316%:%
%:%1047=316%:%
%:%1048=316%:%
%:%1049=317%:%
%:%1050=317%:%
%:%1051=318%:%
%:%1052=318%:%
%:%1053=319%:%
%:%1054=319%:%
%:%1055=319%:%
%:%1056=320%:%
%:%1057=320%:%
%:%1058=320%:%
%:%1059=320%:%
%:%1060=321%:%
%:%1061=322%:%
%:%1062=322%:%
%:%1063=322%:%
%:%1064=322%:%
%:%1065=323%:%
%:%1066=323%:%
%:%1067=323%:%
%:%1068=323%:%
%:%1069=323%:%
%:%1070=324%:%
%:%1071=324%:%
%:%1072=324%:%
%:%1073=324%:%
%:%1074=324%:%
%:%1075=325%:%
%:%1076=325%:%
%:%1077=325%:%
%:%1078=325%:%
%:%1079=326%:%
%:%1080=326%:%
%:%1081=326%:%
%:%1082=326%:%
%:%1083=326%:%
%:%1084=327%:%
%:%1085=327%:%
%:%1086=328%:%
%:%1087=328%:%
%:%1088=329%:%
%:%1089=329%:%
%:%1090=329%:%
%:%1091=329%:%
%:%1092=330%:%
%:%1093=330%:%
%:%1094=330%:%
%:%1095=330%:%
%:%1096=330%:%
%:%1097=331%:%
%:%1098=331%:%
%:%1099=331%:%
%:%1100=331%:%
%:%1101=331%:%
%:%1102=332%:%
%:%1103=332%:%
%:%1104=332%:%
%:%1105=332%:%
%:%1106=332%:%
%:%1107=333%:%
%:%1108=333%:%
%:%1109=333%:%
%:%1110=333%:%
%:%1111=333%:%
%:%1112=334%:%
%:%1113=334%:%
%:%1114=334%:%
%:%1115=334%:%
%:%1116=335%:%
%:%1122=335%:%
%:%1125=336%:%
%:%1126=337%:%
%:%1127=337%:%
%:%1128=338%:%
%:%1129=339%:%
%:%1130=340%:%
%:%1131=341%:%
%:%1132=342%:%
%:%1133=343%:%
%:%1134=344%:%
%:%1135=345%:%
%:%1136=346%:%
%:%1137=347%:%
%:%1144=348%:%
%:%1145=348%:%
%:%1146=349%:%
%:%1147=349%:%
%:%1148=349%:%
%:%1149=349%:%
%:%1150=350%:%
%:%1151=350%:%
%:%1152=350%:%
%:%1153=350%:%
%:%1154=351%:%
%:%1155=351%:%
%:%1156=351%:%
%:%1157=351%:%
%:%1158=352%:%
%:%1159=352%:%
%:%1160=352%:%
%:%1161=352%:%
%:%1162=353%:%
%:%1163=353%:%
%:%1164=353%:%
%:%1165=354%:%
%:%1166=355%:%
%:%1167=355%:%
%:%1168=355%:%
%:%1169=355%:%
%:%1170=356%:%
%:%1171=356%:%
%:%1172=356%:%
%:%1173=356%:%
%:%1174=357%:%
%:%1175=358%:%
%:%1176=358%:%
%:%1177=358%:%
%:%1178=358%:%
%:%1179=359%:%
%:%1180=359%:%
%:%1181=359%:%
%:%1182=359%:%
%:%1183=360%:%
%:%1184=361%:%
%:%1185=361%:%
%:%1186=361%:%
%:%1187=361%:%
%:%1188=362%:%
%:%1189=362%:%
%:%1190=362%:%
%:%1191=362%:%
%:%1192=363%:%
%:%1193=364%:%
%:%1194=364%:%
%:%1195=364%:%
%:%1196=364%:%
%:%1197=365%:%
%:%1198=365%:%
%:%1199=365%:%
%:%1200=365%:%
%:%1201=365%:%
%:%1202=366%:%
%:%1203=366%:%
%:%1204=366%:%
%:%1205=367%:%
%:%1206=367%:%
%:%1207=368%:%
%:%1208=368%:%
%:%1209=369%:%
%:%1210=369%:%
%:%1211=369%:%
%:%1212=370%:%
%:%1213=370%:%
%:%1214=370%:%
%:%1215=371%:%
%:%1216=371%:%
%:%1217=371%:%
%:%1218=371%:%
%:%1219=372%:%
%:%1220=372%:%
%:%1221=373%:%
%:%1222=373%:%
%:%1223=373%:%
%:%1224=374%:%
%:%1225=374%:%
%:%1226=375%:%
%:%1227=375%:%
%:%1228=375%:%
%:%1229=375%:%
%:%1230=376%:%
%:%1231=377%:%
%:%1232=377%:%
%:%1233=377%:%
%:%1234=377%:%
%:%1235=378%:%
%:%1236=379%:%
%:%1237=379%:%
%:%1238=380%:%
%:%1239=380%:%
%:%1240=381%:%
%:%1241=381%:%
%:%1242=382%:%
%:%1243=383%:%
%:%1244=383%:%
%:%1245=383%:%
%:%1246=383%:%
%:%1247=384%:%
%:%1248=385%:%
%:%1249=385%:%
%:%1250=386%:%
%:%1251=386%:%
%:%1252=387%:%
%:%1253=387%:%
%:%1254=387%:%
%:%1255=387%:%
%:%1256=388%:%
%:%1257=388%:%
%:%1258=389%:%
%:%1259=389%:%
%:%1260=389%:%
%:%1261=390%:%
%:%1262=390%:%
%:%1263=390%:%
%:%1264=391%:%
%:%1265=391%:%
%:%1266=391%:%
%:%1267=391%:%
%:%1268=392%:%
%:%1269=392%:%
%:%1270=392%:%
%:%1271=392%:%
%:%1272=392%:%
%:%1273=393%:%
%:%1274=393%:%
%:%1275=393%:%
%:%1276=393%:%
%:%1277=394%:%
%:%1278=394%:%
%:%1279=395%:%
%:%1280=395%:%
%:%1281=395%:%
%:%1282=395%:%
%:%1283=396%:%
%:%1284=396%:%
%:%1285=396%:%
%:%1286=397%:%
%:%1287=397%:%
%:%1288=398%:%
%:%1289=398%:%
%:%1290=398%:%
%:%1291=398%:%
%:%1292=399%:%
%:%1293=399%:%
%:%1294=399%:%
%:%1295=399%:%
%:%1296=399%:%
%:%1297=400%:%
%:%1298=400%:%
%:%1299=400%:%
%:%1300=400%:%
%:%1301=400%:%
%:%1302=401%:%
%:%1303=401%:%
%:%1304=401%:%
%:%1305=401%:%
%:%1306=401%:%
%:%1307=402%:%
%:%1308=402%:%
%:%1309=402%:%
%:%1310=402%:%
%:%1311=402%:%
%:%1312=403%:%
%:%1313=403%:%
%:%1314=404%:%
%:%1315=404%:%
%:%1316=404%:%
%:%1317=404%:%
%:%1318=404%:%
%:%1319=405%:%
%:%1320=405%:%
%:%1321=406%:%
%:%1322=406%:%
%:%1323=406%:%
%:%1324=406%:%
%:%1325=407%:%
%:%1326=407%:%
%:%1327=407%:%
%:%1328=407%:%
%:%1329=408%:%
%:%1330=409%:%
%:%1331=409%:%
%:%1332=410%:%
%:%1333=410%:%
%:%1334=411%:%
%:%1335=411%:%
%:%1336=411%:%
%:%1337=412%:%
%:%1338=412%:%
%:%1339=412%:%
%:%1340=412%:%
%:%1341=412%:%
%:%1342=413%:%
%:%1343=413%:%
%:%1344=413%:%
%:%1345=413%:%
%:%1346=413%:%
%:%1347=414%:%
%:%1348=414%:%
%:%1349=415%:%
%:%1350=415%:%
%:%1351=415%:%
%:%1352=416%:%
%:%1353=417%:%
%:%1354=417%:%
%:%1355=418%:%
%:%1356=418%:%
%:%1357=418%:%
%:%1358=419%:%
%:%1359=419%:%
%:%1360=420%:%
%:%1361=420%:%
%:%1362=421%:%
%:%1363=421%:%
%:%1364=421%:%
%:%1365=421%:%
%:%1366=422%:%
%:%1367=422%:%
%:%1368=422%:%
%:%1369=422%:%
%:%1370=423%:%
%:%1371=423%:%
%:%1372=424%:%
%:%1373=424%:%
%:%1374=425%:%
%:%1375=425%:%
%:%1376=426%:%
%:%1377=426%:%
%:%1378=427%:%
%:%1379=427%:%
%:%1380=428%:%
%:%1381=429%:%
%:%1382=429%:%
%:%1383=430%:%
%:%1384=430%:%
%:%1385=431%:%
%:%1386=431%:%
%:%1387=431%:%
%:%1388=431%:%
%:%1389=432%:%
%:%1390=432%:%
%:%1391=432%:%
%:%1392=432%:%
%:%1393=433%:%
%:%1394=433%:%
%:%1395=434%:%
%:%1396=434%:%
%:%1397=435%:%
%:%1398=435%:%
%:%1399=436%:%
%:%1400=436%:%
%:%1401=437%:%
%:%1402=437%:%
%:%1403=438%:%
%:%1404=439%:%
%:%1405=439%:%
%:%1406=439%:%
%:%1407=439%:%
%:%1408=440%:%
%:%1409=440%:%
%:%1410=440%:%
%:%1411=440%:%
%:%1412=440%:%
%:%1413=441%:%
%:%1414=441%:%
%:%1415=441%:%
%:%1416=441%:%
%:%1417=441%:%
%:%1418=442%:%
%:%1419=442%:%
%:%1420=442%:%
%:%1421=442%:%
%:%1422=443%:%
%:%1423=443%:%
%:%1424=444%:%
%:%1425=444%:%
%:%1426=444%:%
%:%1427=444%:%
%:%1428=445%:%
%:%1429=445%:%
%:%1430=445%:%
%:%1431=445%:%
%:%1432=446%:%
%:%1438=446%:%
%:%1441=447%:%
%:%1442=448%:%
%:%1443=448%:%
%:%1444=449%:%
%:%1445=450%:%
%:%1446=451%:%
%:%1447=452%:%
%:%1448=453%:%
%:%1449=454%:%
%:%1450=455%:%
%:%1457=456%:%
%:%1458=456%:%
%:%1459=457%:%
%:%1460=457%:%
%:%1461=457%:%
%:%1462=457%:%
%:%1463=458%:%
%:%1464=458%:%
%:%1465=458%:%
%:%1466=459%:%
%:%1472=459%:%
%:%1475=460%:%
%:%1476=461%:%
%:%1477=461%:%
%:%1478=462%:%
%:%1479=463%:%
%:%1480=464%:%
%:%1487=465%:%
%:%1488=465%:%
%:%1489=466%:%
%:%1490=466%:%
%:%1491=466%:%
%:%1492=466%:%
%:%1493=467%:%
%:%1494=467%:%
%:%1495=467%:%
%:%1496=467%:%
%:%1497=468%:%
%:%1503=468%:%
%:%1506=469%:%
%:%1507=471%:%
%:%1508=472%:%
%:%1509=473%:%
%:%1510=473%:%
%:%1511=474%:%
%:%1512=475%:%
%:%1513=476%:%
%:%1516=477%:%
%:%1520=477%:%
%:%1521=477%:%
%:%1522=477%:%
%:%1527=477%:%
%:%1530=478%:%
%:%1531=479%:%
%:%1532=479%:%
%:%1533=480%:%
%:%1534=481%:%
%:%1535=482%:%
%:%1538=483%:%
%:%1542=483%:%
%:%1543=483%:%
%:%1544=483%:%
%:%1549=483%:%
%:%1552=484%:%
%:%1553=485%:%
%:%1554=485%:%
%:%1555=486%:%
%:%1556=487%:%
%:%1557=488%:%
%:%1560=489%:%
%:%1564=489%:%
%:%1565=489%:%
%:%1571=489%:%
%:%1574=490%:%
%:%1575=491%:%
%:%1576=491%:%
%:%1577=492%:%
%:%1578=493%:%
%:%1579=494%:%
%:%1582=495%:%
%:%1586=495%:%
%:%1587=495%:%
%:%1588=495%:%
%:%1593=495%:%
%:%1596=496%:%
%:%1597=497%:%
%:%1598=497%:%
%:%1599=498%:%
%:%1600=499%:%
%:%1601=500%:%
%:%1604=501%:%
%:%1608=501%:%
%:%1609=501%:%
%:%1610=501%:%
%:%1611=501%:%
%:%1616=501%:%
%:%1619=502%:%
%:%1620=503%:%
%:%1621=503%:%
%:%1622=504%:%
%:%1623=505%:%
%:%1624=506%:%
%:%1627=507%:%
%:%1631=507%:%
%:%1632=507%:%
%:%1633=507%:%
%:%1634=508%:%
%:%1635=508%:%
%:%1636=509%:%
%:%1637=510%:%
%:%1638=511%:%
%:%1643=511%:%
%:%1646=512%:%
%:%1647=513%:%
%:%1648=513%:%
%:%1649=514%:%
%:%1650=515%:%
%:%1651=516%:%
%:%1658=517%:%
%:%1659=517%:%
%:%1660=518%:%
%:%1661=518%:%
%:%1662=519%:%
%:%1663=519%:%
%:%1664=520%:%
%:%1665=521%:%
%:%1666=521%:%
%:%1667=521%:%
%:%1668=521%:%
%:%1669=522%:%
%:%1670=523%:%
%:%1671=523%:%
%:%1672=524%:%
%:%1673=524%:%
%:%1674=524%:%
%:%1675=525%:%
%:%1676=526%:%
%:%1677=526%:%
%:%1678=526%:%
%:%1679=526%:%
%:%1680=527%:%
%:%1681=528%:%
%:%1682=528%:%
%:%1683=528%:%
%:%1684=528%:%
%:%1685=529%:%
%:%1686=529%:%
%:%1687=529%:%
%:%1688=529%:%
%:%1689=529%:%
%:%1690=530%:%
%:%1691=530%:%
%:%1692=530%:%
%:%1693=530%:%
%:%1694=530%:%
%:%1695=531%:%
%:%1701=531%:%
%:%1704=532%:%
%:%1705=533%:%
%:%1706=534%:%
%:%1707=535%:%
%:%1708=535%:%
%:%1709=536%:%
%:%1710=537%:%
%:%1711=538%:%
%:%1712=539%:%
%:%1719=540%:%
%:%1720=540%:%
%:%1721=541%:%
%:%1722=541%:%
%:%1723=541%:%
%:%1724=541%:%
%:%1725=542%:%
%:%1726=542%:%
%:%1727=542%:%
%:%1728=543%:%
%:%1729=543%:%
%:%1730=544%:%
%:%1731=544%:%
%:%1732=544%:%
%:%1733=545%:%
%:%1734=545%:%
%:%1735=545%:%
%:%1736=545%:%
%:%1737=546%:%
%:%1738=546%:%
%:%1739=546%:%
%:%1740=547%:%
%:%1741=547%:%
%:%1742=547%:%
%:%1743=547%:%
%:%1744=547%:%
%:%1745=548%:%
%:%1746=548%:%
%:%1747=548%:%
%:%1748=548%:%
%:%1749=548%:%
%:%1750=549%:%
%:%1751=549%:%
%:%1752=549%:%
%:%1753=549%:%
%:%1754=550%:%
%:%1755=550%:%
%:%1756=551%:%
%:%1757=551%:%
%:%1758=551%:%
%:%1759=552%:%
%:%1760=552%:%
%:%1761=552%:%
%:%1762=553%:%
%:%1768=553%:%
%:%1771=554%:%
%:%1772=555%:%
%:%1773=555%:%
%:%1774=556%:%
%:%1775=557%:%
%:%1776=558%:%
%:%1777=559%:%
%:%1784=560%:%
%:%1785=560%:%
%:%1786=561%:%
%:%1787=561%:%
%:%1788=562%:%
%:%1789=562%:%
%:%1790=562%:%
%:%1791=563%:%
%:%1792=563%:%
%:%1793=564%:%
%:%1794=564%:%
%:%1795=564%:%
%:%1796=565%:%
%:%1797=565%:%
%:%1798=565%:%
%:%1799=565%:%
%:%1800=566%:%
%:%1801=566%:%
%:%1802=567%:%
%:%1803=567%:%
%:%1804=568%:%
%:%1805=568%:%
%:%1806=568%:%
%:%1807=568%:%
%:%1808=569%:%
%:%1809=570%:%
%:%1810=571%:%
%:%1811=572%:%
%:%1812=572%:%
%:%1813=572%:%
%:%1814=572%:%
%:%1815=573%:%
%:%1816=573%:%
%:%1817=573%:%
%:%1818=573%:%
%:%1819=574%:%
%:%1820=575%:%
%:%1821=575%:%
%:%1822=575%:%
%:%1823=575%:%
%:%1824=576%:%
%:%1825=576%:%
%:%1826=576%:%
%:%1827=576%:%
%:%1828=577%:%
%:%1829=578%:%
%:%1830=578%:%
%:%1831=578%:%
%:%1832=578%:%
%:%1833=579%:%
%:%1834=579%:%
%:%1835=580%:%
%:%1836=580%:%
%:%1837=581%:%
%:%1838=581%:%
%:%1839=581%:%
%:%1840=581%:%
%:%1841=582%:%
%:%1842=582%:%
%:%1843=582%:%
%:%1844=582%:%
%:%1845=583%:%
%:%1846=583%:%
%:%1847=583%:%
%:%1848=583%:%
%:%1849=584%:%
%:%1850=584%:%
%:%1851=584%:%
%:%1852=584%:%
%:%1853=584%:%
%:%1854=585%:%
%:%1855=585%:%
%:%1856=586%:%
%:%1857=586%:%
%:%1858=587%:%
%:%1859=588%:%
%:%1860=589%:%
%:%1861=590%:%
%:%1862=590%:%
%:%1863=591%:%
%:%1864=591%:%
%:%1865=592%:%
%:%1866=592%:%
%:%1867=592%:%
%:%1868=593%:%
%:%1869=593%:%
%:%1870=593%:%
%:%1871=594%:%
%:%1872=595%:%
%:%1873=595%:%
%:%1874=595%:%
%:%1875=595%:%
%:%1876=595%:%
%:%1877=596%:%
%:%1878=597%:%
%:%1879=597%:%
%:%1880=597%:%
%:%1881=597%:%
%:%1882=598%:%
%:%1883=598%:%
%:%1884=598%:%
%:%1885=598%:%
%:%1886=599%:%
%:%1887=599%:%
%:%1888=599%:%
%:%1889=599%:%
%:%1890=600%:%
%:%1891=600%:%
%:%1892=600%:%
%:%1893=600%:%
%:%1894=600%:%
%:%1895=601%:%
%:%1896=601%:%
%:%1897=601%:%
%:%1898=601%:%
%:%1899=602%:%
%:%1900=602%:%
%:%1901=602%:%
%:%1902=602%:%
%:%1903=603%:%
%:%1904=603%:%
%:%1905=603%:%
%:%1906=603%:%
%:%1907=603%:%
%:%1908=604%:%
%:%1909=604%:%
%:%1910=604%:%
%:%1911=604%:%
%:%1912=605%:%
%:%1913=605%:%
%:%1914=606%:%
%:%1915=606%:%
%:%1916=607%:%
%:%1917=608%:%
%:%1918=609%:%
%:%1919=610%:%
%:%1920=610%:%
%:%1921=610%:%
%:%1922=610%:%
%:%1923=611%:%
%:%1924=612%:%
%:%1925=612%:%
%:%1926=612%:%
%:%1927=612%:%
%:%1928=613%:%
%:%1929=613%:%
%:%1930=613%:%
%:%1931=613%:%
%:%1932=613%:%
%:%1933=614%:%
%:%1934=614%:%
%:%1935=614%:%
%:%1936=614%:%
%:%1937=614%:%
%:%1938=614%:%
%:%1939=615%:%
%:%1940=615%:%
%:%1941=616%:%
%:%1942=616%:%
%:%1943=617%:%
%:%1944=617%:%
%:%1945=618%:%
%:%1946=619%:%
%:%1947=620%:%
%:%1948=621%:%
%:%1949=621%:%
%:%1950=621%:%
%:%1951=621%:%
%:%1952=622%:%
%:%1953=623%:%
%:%1954=623%:%
%:%1955=624%:%
%:%1956=624%:%
%:%1957=624%:%
%:%1958=625%:%
%:%1959=625%:%
%:%1960=625%:%
%:%1961=625%:%
%:%1962=626%:%
%:%1963=626%:%
%:%1964=626%:%
%:%1965=626%:%
%:%1966=626%:%
%:%1967=627%:%
%:%1968=627%:%
%:%1969=628%:%
%:%1970=628%:%
%:%1971=628%:%
%:%1972=628%:%
%:%1973=629%:%
%:%1979=629%:%
%:%1982=630%:%
%:%1983=631%:%
%:%1984=631%:%
%:%1985=632%:%
%:%1986=633%:%
%:%1987=634%:%
%:%1994=635%:%
%:%1995=635%:%
%:%1996=636%:%
%:%1997=636%:%
%:%1998=636%:%
%:%1999=637%:%
%:%2000=637%:%
%:%2001=637%:%
%:%2002=637%:%
%:%2003=638%:%
%:%2004=639%:%
%:%2005=639%:%
%:%2006=639%:%
%:%2007=639%:%
%:%2008=640%:%
%:%2009=640%:%
%:%2010=640%:%
%:%2011=640%:%
%:%2012=640%:%
%:%2013=641%:%
%:%2014=641%:%
%:%2015=641%:%
%:%2016=641%:%
%:%2017=641%:%
%:%2018=642%:%
%:%2019=642%:%
%:%2020=642%:%
%:%2021=642%:%
%:%2022=642%:%
%:%2023=643%:%
%:%2024=643%:%
%:%2025=643%:%
%:%2026=643%:%
%:%2027=643%:%
%:%2028=644%:%
%:%2029=644%:%
%:%2030=644%:%
%:%2031=644%:%
%:%2032=645%:%
%:%2033=645%:%
%:%2047=647%:%
%:%2057=649%:%
%:%2058=649%:%
%:%2059=650%:%
%:%2060=651%:%
%:%2061=652%:%
%:%2068=653%:%
%:%2069=653%:%
%:%2070=654%:%
%:%2071=654%:%
%:%2072=655%:%
%:%2073=655%:%
%:%2074=655%:%
%:%2075=655%:%
%:%2076=656%:%
%:%2077=656%:%
%:%2078=656%:%
%:%2079=656%:%
%:%2080=657%:%
%:%2081=657%:%
%:%2082=658%:%
%:%2083=658%:%
%:%2084=659%:%
%:%2085=659%:%
%:%2086=660%:%
%:%2087=660%:%
%:%2088=660%:%
%:%2089=661%:%
%:%2090=661%:%
%:%2091=661%:%
%:%2092=661%:%
%:%2093=662%:%
%:%2094=662%:%
%:%2095=662%:%
%:%2096=663%:%
%:%2097=663%:%
%:%2098=663%:%
%:%2099=664%:%
%:%2100=664%:%
%:%2101=664%:%
%:%2102=664%:%
%:%2103=665%:%
%:%2104=665%:%
%:%2105=665%:%
%:%2106=665%:%
%:%2107=666%:%
%:%2108=666%:%
%:%2109=666%:%
%:%2110=666%:%
%:%2111=667%:%
%:%2112=667%:%
%:%2113=667%:%
%:%2114=667%:%
%:%2115=668%:%
%:%2116=668%:%
%:%2117=669%:%
%:%2118=669%:%
%:%2119=670%:%
%:%2120=670%:%
%:%2121=671%:%
%:%2122=671%:%
%:%2123=672%:%
%:%2124=672%:%
%:%2125=672%:%
%:%2126=672%:%
%:%2127=673%:%
%:%2128=673%:%
%:%2129=673%:%
%:%2130=674%:%
%:%2131=674%:%
%:%2132=674%:%
%:%2133=674%:%
%:%2134=674%:%
%:%2135=675%:%
%:%2136=675%:%
%:%2137=675%:%
%:%2138=675%:%
%:%2139=676%:%
%:%2140=676%:%
%:%2141=677%:%
%:%2142=677%:%
%:%2143=677%:%
%:%2144=677%:%
%:%2145=678%:%
%:%2146=678%:%
%:%2147=679%:%
%:%2148=679%:%
%:%2149=679%:%
%:%2150=680%:%
%:%2151=680%:%
%:%2152=680%:%
%:%2153=680%:%
%:%2154=681%:%
%:%2155=681%:%
%:%2156=681%:%
%:%2157=681%:%
%:%2158=682%:%
%:%2159=682%:%
%:%2160=682%:%
%:%2161=682%:%
%:%2162=683%:%
%:%2163=683%:%
%:%2164=684%:%
%:%2165=684%:%
%:%2166=684%:%
%:%2167=684%:%
%:%2168=684%:%
%:%2169=685%:%
%:%2175=685%:%
%:%2178=686%:%
%:%2179=687%:%
%:%2180=687%:%
%:%2181=688%:%
%:%2182=689%:%
%:%2183=690%:%
%:%2190=691%:%
%:%2191=691%:%
%:%2192=692%:%
%:%2193=692%:%
%:%2194=692%:%
%:%2195=692%:%
%:%2196=693%:%
%:%2197=693%:%
%:%2198=693%:%
%:%2199=693%:%
%:%2200=693%:%
%:%2201=694%:%
%:%2202=694%:%
%:%2203=694%:%
%:%2204=694%:%
%:%2205=695%:%
%:%2211=695%:%
%:%2214=696%:%
%:%2215=697%:%
%:%2216=697%:%
%:%2217=698%:%
%:%2218=699%:%
%:%2219=700%:%
%:%2222=701%:%
%:%2226=701%:%
%:%2227=701%:%
%:%2228=702%:%
%:%2229=702%:%
%:%2234=702%:%
%:%2237=703%:%
%:%2238=704%:%
%:%2239=704%:%
%:%2240=705%:%
%:%2241=706%:%
%:%2242=707%:%
%:%2243=708%:%
%:%2250=709%:%
%:%2251=709%:%
%:%2252=710%:%
%:%2253=710%:%
%:%2254=711%:%
%:%2255=711%:%
%:%2256=711%:%
%:%2257=711%:%
%:%2258=712%:%
%:%2259=712%:%
%:%2260=713%:%
%:%2261=713%:%
%:%2262=713%:%
%:%2263=714%:%
%:%2264=714%:%
%:%2265=714%:%
%:%2266=714%:%
%:%2267=715%:%
%:%2268=715%:%
%:%2269=715%:%
%:%2270=715%:%
%:%2271=716%:%
%:%2272=716%:%
%:%2273=716%:%
%:%2274=716%:%
%:%2275=717%:%
%:%2276=717%:%
%:%2277=717%:%
%:%2278=717%:%
%:%2279=718%:%
%:%2280=718%:%
%:%2281=718%:%
%:%2282=718%:%
%:%2283=718%:%
%:%2284=719%:%
%:%2285=719%:%
%:%2286=719%:%
%:%2287=719%:%
%:%2288=719%:%
%:%2289=719%:%
%:%2290=720%:%
%:%2291=720%:%
%:%2292=720%:%
%:%2293=720%:%
%:%2294=720%:%
%:%2295=721%:%
%:%2296=721%:%
%:%2297=722%:%
%:%2298=722%:%
%:%2299=722%:%
%:%2300=722%:%
%:%2301=723%:%
%:%2307=723%:%
%:%2310=724%:%
%:%2311=725%:%
%:%2312=725%:%
%:%2313=726%:%
%:%2314=727%:%
%:%2315=728%:%
%:%2316=729%:%
%:%2323=730%:%
%:%2324=730%:%
%:%2325=731%:%
%:%2326=731%:%
%:%2327=731%:%
%:%2328=731%:%
%:%2329=732%:%
%:%2330=732%:%
%:%2331=732%:%
%:%2332=732%:%
%:%2333=732%:%
%:%2334=733%:%
%:%2335=733%:%
%:%2336=733%:%
%:%2337=733%:%
%:%2338=734%:%
%:%2344=734%:%
%:%2347=735%:%
%:%2348=736%:%
%:%2349=736%:%
%:%2350=737%:%
%:%2351=738%:%
%:%2352=739%:%
%:%2355=740%:%
%:%2359=740%:%
%:%2360=740%:%
%:%2361=741%:%
%:%2362=741%:%
%:%2367=741%:%
%:%2370=742%:%
%:%2371=743%:%
%:%2372=743%:%
%:%2373=744%:%
%:%2374=745%:%
%:%2375=746%:%
%:%2382=747%:%
%:%2383=747%:%
%:%2384=748%:%
%:%2385=748%:%
%:%2386=748%:%
%:%2387=749%:%
%:%2388=749%:%
%:%2389=749%:%
%:%2390=749%:%
%:%2391=750%:%
%:%2392=750%:%
%:%2393=750%:%
%:%2394=750%:%
%:%2395=751%:%
%:%2401=751%:%
%:%2404=752%:%
%:%2405=753%:%
%:%2406=753%:%
%:%2407=754%:%
%:%2408=755%:%
%:%2409=756%:%
%:%2416=757%:%
%:%2417=757%:%
%:%2418=758%:%
%:%2419=758%:%
%:%2420=758%:%
%:%2421=759%:%
%:%2422=759%:%
%:%2423=759%:%
%:%2424=759%:%
%:%2425=760%:%
%:%2426=760%:%
%:%2427=760%:%
%:%2428=760%:%
%:%2429=761%:%
%:%2435=761%:%
%:%2438=762%:%
%:%2439=763%:%
%:%2440=763%:%
%:%2441=764%:%
%:%2442=765%:%
%:%2443=766%:%
%:%2446=767%:%
%:%2450=767%:%
%:%2451=767%:%
%:%2452=767%:%
%:%2453=767%:%
%:%2458=767%:%
%:%2461=768%:%
%:%2462=769%:%
%:%2463=769%:%
%:%2464=770%:%
%:%2465=771%:%
%:%2466=772%:%
%:%2469=773%:%
%:%2473=773%:%
%:%2474=773%:%
%:%2475=773%:%
%:%2476=773%:%
%:%2481=773%:%
%:%2484=774%:%
%:%2485=775%:%
%:%2486=775%:%
%:%2489=776%:%
%:%2494=777%:%

%
\begin{isabellebody}%
\setisabellecontext{Stochastic{\isacharunderscore}{\kern0pt}Process}%
%
\isadelimtheory
%
\endisadelimtheory
%
\isatagtheory
\isacommand{theory}\isamarkupfalse%
\ Stochastic{\isacharunderscore}{\kern0pt}Process\isanewline
\isakeyword{imports}\ Filtration\isanewline
\isakeyword{begin}%
\endisatagtheory
{\isafoldtheory}%
%
\isadelimtheory
%
\endisadelimtheory
%
\isadelimdocument
%
\endisadelimdocument
%
\isatagdocument
%
\isamarkupsubsection{Stochastic Process%
}
\isamarkuptrue%
%
\endisatagdocument
{\isafolddocument}%
%
\isadelimdocument
%
\endisadelimdocument
\isacommand{locale}\isamarkupfalse%
\ stochastic{\isacharunderscore}{\kern0pt}process\ {\isacharequal}{\kern0pt}\ sigma{\isacharunderscore}{\kern0pt}finite{\isacharunderscore}{\kern0pt}measure\ M\ \isakeyword{for}\ M\ {\isacharplus}{\kern0pt}\isanewline
\ \ \isakeyword{fixes}\ X\ {\isacharcolon}{\kern0pt}{\isacharcolon}{\kern0pt}\ {\isachardoublequoteopen}{\isacharprime}{\kern0pt}t\ {\isacharcolon}{\kern0pt}{\isacharcolon}{\kern0pt}\ {\isacharbraceleft}{\kern0pt}second{\isacharunderscore}{\kern0pt}countable{\isacharunderscore}{\kern0pt}topology{\isacharcomma}{\kern0pt}linorder{\isacharunderscore}{\kern0pt}topology{\isacharbraceright}{\kern0pt}\ {\isasymRightarrow}\ {\isacharprime}{\kern0pt}a\ {\isasymRightarrow}\ {\isacharprime}{\kern0pt}b{\isacharcolon}{\kern0pt}{\isacharcolon}{\kern0pt}{\isacharbraceleft}{\kern0pt}real{\isacharunderscore}{\kern0pt}normed{\isacharunderscore}{\kern0pt}vector{\isacharcomma}{\kern0pt}\ second{\isacharunderscore}{\kern0pt}countable{\isacharunderscore}{\kern0pt}topology{\isacharbraceright}{\kern0pt}{\isachardoublequoteclose}\isanewline
\ \ \isakeyword{assumes}\ random{\isacharunderscore}{\kern0pt}variable{\isacharbrackleft}{\kern0pt}measurable{\isacharbrackright}{\kern0pt}{\isacharcolon}{\kern0pt}\ {\isachardoublequoteopen}{\isasymAnd}i{\isachardot}{\kern0pt}\ X\ i\ {\isasymin}\ borel{\isacharunderscore}{\kern0pt}measurable\ M{\isachardoublequoteclose}\isanewline
\isakeyword{begin}\isanewline
\isanewline
\isacommand{definition}\isamarkupfalse%
\ left{\isacharunderscore}{\kern0pt}continuous\ \isakeyword{where}\ {\isachardoublequoteopen}left{\isacharunderscore}{\kern0pt}continuous\ {\isacharequal}{\kern0pt}\ {\isacharparenleft}{\kern0pt}AE\ {\isasymxi}\ in\ M{\isachardot}{\kern0pt}\ {\isasymforall}i{\isachardot}{\kern0pt}\ continuous\ {\isacharparenleft}{\kern0pt}at{\isacharunderscore}{\kern0pt}left\ i{\isacharparenright}{\kern0pt}\ {\isacharparenleft}{\kern0pt}{\isasymlambda}i{\isachardot}{\kern0pt}\ X\ i\ {\isasymxi}{\isacharparenright}{\kern0pt}{\isacharparenright}{\kern0pt}{\isachardoublequoteclose}\isanewline
\isacommand{definition}\isamarkupfalse%
\ right{\isacharunderscore}{\kern0pt}continuous\ \isakeyword{where}\ {\isachardoublequoteopen}right{\isacharunderscore}{\kern0pt}continuous\ {\isacharequal}{\kern0pt}\ {\isacharparenleft}{\kern0pt}AE\ {\isasymxi}\ in\ M{\isachardot}{\kern0pt}\ {\isasymforall}i{\isachardot}{\kern0pt}\ continuous\ {\isacharparenleft}{\kern0pt}at{\isacharunderscore}{\kern0pt}right\ i{\isacharparenright}{\kern0pt}\ {\isacharparenleft}{\kern0pt}{\isasymlambda}i{\isachardot}{\kern0pt}\ X\ i\ {\isasymxi}{\isacharparenright}{\kern0pt}{\isacharparenright}{\kern0pt}{\isachardoublequoteclose}\isanewline
\isanewline
\isacommand{lemma}\isamarkupfalse%
\ compose{\isacharcolon}{\kern0pt}\isanewline
\ \ \isakeyword{assumes}\ {\isachardoublequoteopen}{\isasymAnd}i{\isachardot}{\kern0pt}\ f\ i\ {\isasymin}\ borel{\isacharunderscore}{\kern0pt}measurable\ borel{\isachardoublequoteclose}\isanewline
\ \ \isakeyword{shows}\ {\isachardoublequoteopen}stochastic{\isacharunderscore}{\kern0pt}process\ M\ {\isacharparenleft}{\kern0pt}{\isasymlambda}i\ {\isasymxi}{\isachardot}{\kern0pt}\ {\isacharparenleft}{\kern0pt}f\ i{\isacharparenright}{\kern0pt}\ {\isacharparenleft}{\kern0pt}X\ i\ {\isasymxi}{\isacharparenright}{\kern0pt}{\isacharparenright}{\kern0pt}{\isachardoublequoteclose}\isanewline
%
\isadelimproof
\ \ %
\endisadelimproof
%
\isatagproof
\isacommand{by}\isamarkupfalse%
\ {\isacharparenleft}{\kern0pt}unfold{\isacharunderscore}{\kern0pt}locales{\isacharcomma}{\kern0pt}\ intro\ measurable{\isacharunderscore}{\kern0pt}compose{\isacharbrackleft}{\kern0pt}OF\ random{\isacharunderscore}{\kern0pt}variable\ assms{\isacharbrackright}{\kern0pt}{\isacharparenright}{\kern0pt}%
\endisatagproof
{\isafoldproof}%
%
\isadelimproof
\ \isanewline
%
\endisadelimproof
\isanewline
\isacommand{lemma}\isamarkupfalse%
\ norm{\isacharcolon}{\kern0pt}\ {\isachardoublequoteopen}stochastic{\isacharunderscore}{\kern0pt}process\ M\ {\isacharparenleft}{\kern0pt}{\isasymlambda}i\ {\isasymxi}{\isachardot}{\kern0pt}\ norm\ {\isacharparenleft}{\kern0pt}X\ i\ {\isasymxi}{\isacharparenright}{\kern0pt}{\isacharparenright}{\kern0pt}{\isachardoublequoteclose}%
\isadelimproof
\ %
\endisadelimproof
%
\isatagproof
\isacommand{by}\isamarkupfalse%
\ {\isacharparenleft}{\kern0pt}auto\ intro{\isacharcolon}{\kern0pt}\ compose\ borel{\isacharunderscore}{\kern0pt}measurable{\isacharunderscore}{\kern0pt}norm{\isacharparenright}{\kern0pt}%
\endisatagproof
{\isafoldproof}%
%
\isadelimproof
%
\endisadelimproof
\isanewline
\isanewline
\isacommand{lemma}\isamarkupfalse%
\ scaleR{\isacharcolon}{\kern0pt}\isanewline
\ \ \isakeyword{assumes}\ {\isachardoublequoteopen}stochastic{\isacharunderscore}{\kern0pt}process\ M\ R{\isachardoublequoteclose}\isanewline
\ \ \isakeyword{shows}\ {\isachardoublequoteopen}stochastic{\isacharunderscore}{\kern0pt}process\ M\ {\isacharparenleft}{\kern0pt}{\isasymlambda}i\ {\isasymxi}{\isachardot}{\kern0pt}\ {\isacharparenleft}{\kern0pt}R\ i\ {\isasymxi}{\isacharparenright}{\kern0pt}\ {\isacharasterisk}{\kern0pt}\isactrlsub R\ {\isacharparenleft}{\kern0pt}X\ i\ {\isasymxi}{\isacharparenright}{\kern0pt}{\isacharparenright}{\kern0pt}{\isachardoublequoteclose}\isanewline
%
\isadelimproof
\ \ %
\endisadelimproof
%
\isatagproof
\isacommand{by}\isamarkupfalse%
\ {\isacharparenleft}{\kern0pt}unfold{\isacharunderscore}{\kern0pt}locales{\isacharparenright}{\kern0pt}\ {\isacharparenleft}{\kern0pt}simp\ add{\isacharcolon}{\kern0pt}\ borel{\isacharunderscore}{\kern0pt}measurable{\isacharunderscore}{\kern0pt}scaleR\ random{\isacharunderscore}{\kern0pt}variable\ assms\ stochastic{\isacharunderscore}{\kern0pt}process{\isachardot}{\kern0pt}random{\isacharunderscore}{\kern0pt}variable{\isacharparenright}{\kern0pt}%
\endisatagproof
{\isafoldproof}%
%
\isadelimproof
\isanewline
%
\endisadelimproof
\isanewline
\isacommand{lemma}\isamarkupfalse%
\ scaleR{\isacharunderscore}{\kern0pt}const{\isacharunderscore}{\kern0pt}fun{\isacharcolon}{\kern0pt}\ \isanewline
\ \ \isakeyword{assumes}\ {\isachardoublequoteopen}f\ {\isasymin}\ borel{\isacharunderscore}{\kern0pt}measurable\ M{\isachardoublequoteclose}\ \isanewline
\ \ \isakeyword{shows}\ {\isachardoublequoteopen}stochastic{\isacharunderscore}{\kern0pt}process\ M\ {\isacharparenleft}{\kern0pt}{\isasymlambda}i\ {\isasymxi}{\isachardot}{\kern0pt}\ f\ {\isasymxi}\ {\isacharasterisk}{\kern0pt}\isactrlsub R\ {\isacharparenleft}{\kern0pt}X\ i\ {\isasymxi}{\isacharparenright}{\kern0pt}{\isacharparenright}{\kern0pt}{\isachardoublequoteclose}\isanewline
%
\isadelimproof
\ \ %
\endisadelimproof
%
\isatagproof
\isacommand{by}\isamarkupfalse%
\ {\isacharparenleft}{\kern0pt}unfold{\isacharunderscore}{\kern0pt}locales{\isacharcomma}{\kern0pt}\ intro\ borel{\isacharunderscore}{\kern0pt}measurable{\isacharunderscore}{\kern0pt}scaleR\ assms\ random{\isacharunderscore}{\kern0pt}variable{\isacharparenright}{\kern0pt}%
\endisatagproof
{\isafoldproof}%
%
\isadelimproof
\isanewline
%
\endisadelimproof
\isanewline
\isacommand{lemma}\isamarkupfalse%
\ scaleR{\isacharunderscore}{\kern0pt}const{\isacharcolon}{\kern0pt}\ {\isachardoublequoteopen}stochastic{\isacharunderscore}{\kern0pt}process\ M\ {\isacharparenleft}{\kern0pt}{\isasymlambda}i\ {\isasymxi}{\isachardot}{\kern0pt}\ c\ {\isacharasterisk}{\kern0pt}\isactrlsub R\ {\isacharparenleft}{\kern0pt}X\ i\ {\isasymxi}{\isacharparenright}{\kern0pt}{\isacharparenright}{\kern0pt}{\isachardoublequoteclose}%
\isadelimproof
\ %
\endisadelimproof
%
\isatagproof
\isacommand{by}\isamarkupfalse%
\ {\isacharparenleft}{\kern0pt}auto\ intro{\isacharcolon}{\kern0pt}\ scaleR{\isacharunderscore}{\kern0pt}const{\isacharunderscore}{\kern0pt}fun\ borel{\isacharunderscore}{\kern0pt}measurable{\isacharunderscore}{\kern0pt}const{\isacharparenright}{\kern0pt}%
\endisatagproof
{\isafoldproof}%
%
\isadelimproof
%
\endisadelimproof
\isanewline
\isanewline
\isacommand{lemma}\isamarkupfalse%
\ add{\isacharcolon}{\kern0pt}\isanewline
\ \ \isakeyword{assumes}\ {\isachardoublequoteopen}stochastic{\isacharunderscore}{\kern0pt}process\ M\ Y{\isachardoublequoteclose}\isanewline
\ \ \isakeyword{shows}\ {\isachardoublequoteopen}stochastic{\isacharunderscore}{\kern0pt}process\ M\ {\isacharparenleft}{\kern0pt}{\isasymlambda}i\ {\isasymxi}{\isachardot}{\kern0pt}\ X\ i\ {\isasymxi}\ {\isacharplus}{\kern0pt}\ Y\ i\ {\isasymxi}{\isacharparenright}{\kern0pt}{\isachardoublequoteclose}\isanewline
%
\isadelimproof
\ \ %
\endisadelimproof
%
\isatagproof
\isacommand{by}\isamarkupfalse%
\ {\isacharparenleft}{\kern0pt}unfold{\isacharunderscore}{\kern0pt}locales{\isacharparenright}{\kern0pt}\ {\isacharparenleft}{\kern0pt}simp\ add{\isacharcolon}{\kern0pt}\ borel{\isacharunderscore}{\kern0pt}measurable{\isacharunderscore}{\kern0pt}add\ random{\isacharunderscore}{\kern0pt}variable\ assms\ stochastic{\isacharunderscore}{\kern0pt}process{\isachardot}{\kern0pt}random{\isacharunderscore}{\kern0pt}variable{\isacharparenright}{\kern0pt}%
\endisatagproof
{\isafoldproof}%
%
\isadelimproof
\isanewline
%
\endisadelimproof
\isanewline
\isacommand{lemma}\isamarkupfalse%
\ diff{\isacharcolon}{\kern0pt}\isanewline
\ \ \isakeyword{assumes}\ {\isachardoublequoteopen}stochastic{\isacharunderscore}{\kern0pt}process\ M\ Y{\isachardoublequoteclose}\isanewline
\ \ \isakeyword{shows}\ {\isachardoublequoteopen}stochastic{\isacharunderscore}{\kern0pt}process\ M\ {\isacharparenleft}{\kern0pt}{\isasymlambda}i\ {\isasymxi}{\isachardot}{\kern0pt}\ X\ i\ {\isasymxi}\ {\isacharminus}{\kern0pt}\ Y\ i\ {\isasymxi}{\isacharparenright}{\kern0pt}{\isachardoublequoteclose}\isanewline
%
\isadelimproof
\ \ %
\endisadelimproof
%
\isatagproof
\isacommand{by}\isamarkupfalse%
\ {\isacharparenleft}{\kern0pt}unfold{\isacharunderscore}{\kern0pt}locales{\isacharparenright}{\kern0pt}\ {\isacharparenleft}{\kern0pt}simp\ add{\isacharcolon}{\kern0pt}\ borel{\isacharunderscore}{\kern0pt}measurable{\isacharunderscore}{\kern0pt}diff\ random{\isacharunderscore}{\kern0pt}variable\ assms\ stochastic{\isacharunderscore}{\kern0pt}process{\isachardot}{\kern0pt}random{\isacharunderscore}{\kern0pt}variable{\isacharparenright}{\kern0pt}%
\endisatagproof
{\isafoldproof}%
%
\isadelimproof
\isanewline
%
\endisadelimproof
\isanewline
\isacommand{lemma}\isamarkupfalse%
\ uminus{\isacharcolon}{\kern0pt}\ {\isachardoublequoteopen}stochastic{\isacharunderscore}{\kern0pt}process\ M\ {\isacharparenleft}{\kern0pt}{\isacharminus}{\kern0pt}X{\isacharparenright}{\kern0pt}{\isachardoublequoteclose}%
\isadelimproof
\ %
\endisadelimproof
%
\isatagproof
\isacommand{using}\isamarkupfalse%
\ scaleR{\isacharunderscore}{\kern0pt}const{\isacharbrackleft}{\kern0pt}of\ {\isachardoublequoteopen}{\isacharminus}{\kern0pt}{\isadigit{1}}{\isachardoublequoteclose}{\isacharbrackright}{\kern0pt}\ \isacommand{by}\isamarkupfalse%
\ {\isacharparenleft}{\kern0pt}simp\ add{\isacharcolon}{\kern0pt}\ fun{\isacharunderscore}{\kern0pt}Compl{\isacharunderscore}{\kern0pt}def{\isacharparenright}{\kern0pt}%
\endisatagproof
{\isafoldproof}%
%
\isadelimproof
%
\endisadelimproof
\isanewline
\isanewline
\isacommand{end}\isamarkupfalse%
%
\isadelimdocument
%
\endisadelimdocument
%
\isatagdocument
%
\isamarkupsubsection{Adapted Process%
}
\isamarkuptrue%
%
\endisatagdocument
{\isafolddocument}%
%
\isadelimdocument
%
\endisadelimdocument
\isacommand{locale}\isamarkupfalse%
\ adapted{\isacharunderscore}{\kern0pt}process\ {\isacharequal}{\kern0pt}\ filtered{\isacharunderscore}{\kern0pt}sigma{\isacharunderscore}{\kern0pt}finite{\isacharunderscore}{\kern0pt}measure\ M\ F\ {\isacharplus}{\kern0pt}\ stochastic{\isacharunderscore}{\kern0pt}process\ M\ X\ \isakeyword{for}\ M\ \isakeyword{and}\ F\ {\isacharcolon}{\kern0pt}{\isacharcolon}{\kern0pt}\ {\isachardoublequoteopen}{\isacharprime}{\kern0pt}t\ {\isacharcolon}{\kern0pt}{\isacharcolon}{\kern0pt}\ {\isacharbraceleft}{\kern0pt}second{\isacharunderscore}{\kern0pt}countable{\isacharunderscore}{\kern0pt}topology{\isacharcomma}{\kern0pt}\ linorder{\isacharunderscore}{\kern0pt}topology{\isacharcomma}{\kern0pt}\ order{\isacharunderscore}{\kern0pt}bot{\isacharbraceright}{\kern0pt}\ {\isasymRightarrow}\ {\isacharunderscore}{\kern0pt}{\isachardoublequoteclose}\ \isakeyword{and}\ X\ {\isacharcolon}{\kern0pt}{\isacharcolon}{\kern0pt}\ {\isachardoublequoteopen}{\isacharprime}{\kern0pt}t\ {\isasymRightarrow}\ {\isacharunderscore}{\kern0pt}\ {\isasymRightarrow}\ {\isacharunderscore}{\kern0pt}\ {\isacharcolon}{\kern0pt}{\isacharcolon}{\kern0pt}\ {\isacharbraceleft}{\kern0pt}second{\isacharunderscore}{\kern0pt}countable{\isacharunderscore}{\kern0pt}topology{\isacharcomma}{\kern0pt}\ banach{\isacharbraceright}{\kern0pt}{\isachardoublequoteclose}\ {\isacharplus}{\kern0pt}\isanewline
\ \ \isakeyword{assumes}\ adapted{\isacharbrackleft}{\kern0pt}measurable{\isacharbrackright}{\kern0pt}{\isacharcolon}{\kern0pt}\ {\isachardoublequoteopen}{\isasymAnd}i{\isachardot}{\kern0pt}\ X\ i\ {\isasymin}\ borel{\isacharunderscore}{\kern0pt}measurable\ {\isacharparenleft}{\kern0pt}F\ i{\isacharparenright}{\kern0pt}{\isachardoublequoteclose}\isanewline
\isakeyword{begin}\isanewline
\isanewline
\isacommand{lemma}\isamarkupfalse%
\ const{\isacharunderscore}{\kern0pt}fun{\isacharcolon}{\kern0pt}\isanewline
\ \ \isakeyword{assumes}\ {\isachardoublequoteopen}f\ {\isasymin}\ borel{\isacharunderscore}{\kern0pt}measurable\ {\isacharparenleft}{\kern0pt}F\ {\isasymbottom}{\isacharparenright}{\kern0pt}{\isachardoublequoteclose}\isanewline
\ \ \isakeyword{shows}\ {\isachardoublequoteopen}adapted{\isacharunderscore}{\kern0pt}process\ M\ F\ {\isacharparenleft}{\kern0pt}{\isasymlambda}{\isacharunderscore}{\kern0pt}{\isachardot}{\kern0pt}\ f{\isacharparenright}{\kern0pt}{\isachardoublequoteclose}\isanewline
%
\isadelimproof
\ \ %
\endisadelimproof
%
\isatagproof
\isacommand{using}\isamarkupfalse%
\ assms\ \isacommand{by}\isamarkupfalse%
\ {\isacharparenleft}{\kern0pt}unfold{\isacharunderscore}{\kern0pt}locales{\isacharparenright}{\kern0pt}\ {\isacharparenleft}{\kern0pt}blast\ intro{\isacharcolon}{\kern0pt}\ measurable{\isacharunderscore}{\kern0pt}from{\isacharunderscore}{\kern0pt}subalg\ subalgebra{\isacharcomma}{\kern0pt}\ metis\ borel{\isacharunderscore}{\kern0pt}measurable{\isacharunderscore}{\kern0pt}subalgebra\ bot{\isachardot}{\kern0pt}extremum\ sets{\isacharunderscore}{\kern0pt}F{\isacharunderscore}{\kern0pt}mono\ space{\isacharunderscore}{\kern0pt}F{\isacharparenright}{\kern0pt}%
\endisatagproof
{\isafoldproof}%
%
\isadelimproof
\isanewline
%
\endisadelimproof
\isanewline
\isacommand{lemma}\isamarkupfalse%
\ compose{\isacharcolon}{\kern0pt}\isanewline
\ \ \isakeyword{assumes}\ {\isachardoublequoteopen}{\isasymAnd}i{\isachardot}{\kern0pt}\ f\ i\ {\isasymin}\ borel{\isacharunderscore}{\kern0pt}measurable\ borel{\isachardoublequoteclose}\isanewline
\ \ \isakeyword{shows}\ {\isachardoublequoteopen}adapted{\isacharunderscore}{\kern0pt}process\ M\ F\ {\isacharparenleft}{\kern0pt}{\isasymlambda}i\ {\isasymxi}{\isachardot}{\kern0pt}\ {\isacharparenleft}{\kern0pt}f\ i{\isacharparenright}{\kern0pt}\ {\isacharparenleft}{\kern0pt}X\ i\ {\isasymxi}{\isacharparenright}{\kern0pt}{\isacharparenright}{\kern0pt}{\isachardoublequoteclose}\isanewline
%
\isadelimproof
\ \ %
\endisadelimproof
%
\isatagproof
\isacommand{by}\isamarkupfalse%
\ {\isacharparenleft}{\kern0pt}unfold{\isacharunderscore}{\kern0pt}locales{\isacharcomma}{\kern0pt}\ intro\ measurable{\isacharunderscore}{\kern0pt}compose{\isacharbrackleft}{\kern0pt}OF\ random{\isacharunderscore}{\kern0pt}variable\ assms{\isacharbrackright}{\kern0pt}{\isacharcomma}{\kern0pt}\ intro\ measurable{\isacharunderscore}{\kern0pt}compose{\isacharbrackleft}{\kern0pt}OF\ adapted\ assms{\isacharbrackright}{\kern0pt}{\isacharparenright}{\kern0pt}%
\endisatagproof
{\isafoldproof}%
%
\isadelimproof
\isanewline
%
\endisadelimproof
\isanewline
\isacommand{lemma}\isamarkupfalse%
\ norm{\isacharcolon}{\kern0pt}\ {\isachardoublequoteopen}adapted{\isacharunderscore}{\kern0pt}process\ M\ F\ {\isacharparenleft}{\kern0pt}{\isasymlambda}i\ {\isasymxi}{\isachardot}{\kern0pt}\ norm\ {\isacharparenleft}{\kern0pt}X\ i\ {\isasymxi}{\isacharparenright}{\kern0pt}{\isacharparenright}{\kern0pt}{\isachardoublequoteclose}%
\isadelimproof
\ %
\endisadelimproof
%
\isatagproof
\isacommand{by}\isamarkupfalse%
\ {\isacharparenleft}{\kern0pt}auto\ intro{\isacharcolon}{\kern0pt}\ compose\ borel{\isacharunderscore}{\kern0pt}measurable{\isacharunderscore}{\kern0pt}norm{\isacharparenright}{\kern0pt}%
\endisatagproof
{\isafoldproof}%
%
\isadelimproof
%
\endisadelimproof
\isanewline
\isanewline
\isacommand{lemma}\isamarkupfalse%
\ scaleR{\isacharcolon}{\kern0pt}\isanewline
\ \ \isakeyword{assumes}\ {\isachardoublequoteopen}adapted{\isacharunderscore}{\kern0pt}process\ M\ F\ R{\isachardoublequoteclose}\isanewline
\ \ \isakeyword{shows}\ {\isachardoublequoteopen}adapted{\isacharunderscore}{\kern0pt}process\ M\ F\ {\isacharparenleft}{\kern0pt}{\isasymlambda}i\ {\isasymxi}{\isachardot}{\kern0pt}\ {\isacharparenleft}{\kern0pt}R\ i\ {\isasymxi}{\isacharparenright}{\kern0pt}\ {\isacharasterisk}{\kern0pt}\isactrlsub R\ {\isacharparenleft}{\kern0pt}X\ i\ {\isasymxi}{\isacharparenright}{\kern0pt}{\isacharparenright}{\kern0pt}{\isachardoublequoteclose}\isanewline
%
\isadelimproof
%
\endisadelimproof
%
\isatagproof
\isacommand{proof}\isamarkupfalse%
\ {\isacharminus}{\kern0pt}\isanewline
\ \ \isacommand{interpret}\isamarkupfalse%
\ R{\isacharcolon}{\kern0pt}\ adapted{\isacharunderscore}{\kern0pt}process\ M\ F\ R\ \isacommand{by}\isamarkupfalse%
\ {\isacharparenleft}{\kern0pt}rule\ assms{\isacharparenright}{\kern0pt}\isanewline
\ \ \isacommand{show}\isamarkupfalse%
\ {\isacharquery}{\kern0pt}thesis\ \isacommand{by}\isamarkupfalse%
\ {\isacharparenleft}{\kern0pt}unfold{\isacharunderscore}{\kern0pt}locales{\isacharparenright}{\kern0pt}\ {\isacharparenleft}{\kern0pt}auto\ simp\ add{\isacharcolon}{\kern0pt}\ borel{\isacharunderscore}{\kern0pt}measurable{\isacharunderscore}{\kern0pt}scaleR\ adapted\ random{\isacharunderscore}{\kern0pt}variable\ assms\ R{\isachardot}{\kern0pt}random{\isacharunderscore}{\kern0pt}variable\ R{\isachardot}{\kern0pt}adapted{\isacharparenright}{\kern0pt}\isanewline
\isacommand{qed}\isamarkupfalse%
%
\endisatagproof
{\isafoldproof}%
%
\isadelimproof
\isanewline
%
\endisadelimproof
\ \ \isanewline
\isacommand{lemma}\isamarkupfalse%
\ scaleR{\isacharunderscore}{\kern0pt}const{\isacharunderscore}{\kern0pt}fun{\isacharcolon}{\kern0pt}\ \isanewline
\ \ \isakeyword{assumes}\ {\isachardoublequoteopen}f\ {\isasymin}\ borel{\isacharunderscore}{\kern0pt}measurable\ {\isacharparenleft}{\kern0pt}F\ {\isasymbottom}{\isacharparenright}{\kern0pt}{\isachardoublequoteclose}\ \isanewline
\ \ \isakeyword{shows}\ {\isachardoublequoteopen}adapted{\isacharunderscore}{\kern0pt}process\ M\ F\ {\isacharparenleft}{\kern0pt}{\isasymlambda}i\ {\isasymxi}{\isachardot}{\kern0pt}\ f\ {\isasymxi}\ {\isacharasterisk}{\kern0pt}\isactrlsub R\ {\isacharparenleft}{\kern0pt}X\ i\ {\isasymxi}{\isacharparenright}{\kern0pt}{\isacharparenright}{\kern0pt}{\isachardoublequoteclose}\isanewline
%
\isadelimproof
\ \ %
\endisadelimproof
%
\isatagproof
\isacommand{using}\isamarkupfalse%
\ assms\ \isacommand{by}\isamarkupfalse%
\ {\isacharparenleft}{\kern0pt}fast\ intro{\isacharcolon}{\kern0pt}\ scaleR\ const{\isacharunderscore}{\kern0pt}fun{\isacharparenright}{\kern0pt}%
\endisatagproof
{\isafoldproof}%
%
\isadelimproof
\isanewline
%
\endisadelimproof
\isanewline
\isacommand{lemma}\isamarkupfalse%
\ scaleR{\isacharunderscore}{\kern0pt}const{\isacharcolon}{\kern0pt}\ {\isachardoublequoteopen}adapted{\isacharunderscore}{\kern0pt}process\ M\ F\ {\isacharparenleft}{\kern0pt}{\isasymlambda}i\ {\isasymxi}{\isachardot}{\kern0pt}\ c\ {\isacharasterisk}{\kern0pt}\isactrlsub R\ {\isacharparenleft}{\kern0pt}X\ i\ {\isasymxi}{\isacharparenright}{\kern0pt}{\isacharparenright}{\kern0pt}{\isachardoublequoteclose}%
\isadelimproof
\ %
\endisadelimproof
%
\isatagproof
\isacommand{by}\isamarkupfalse%
\ {\isacharparenleft}{\kern0pt}auto\ intro{\isacharcolon}{\kern0pt}\ scaleR{\isacharunderscore}{\kern0pt}const{\isacharunderscore}{\kern0pt}fun\ borel{\isacharunderscore}{\kern0pt}measurable{\isacharunderscore}{\kern0pt}const{\isacharparenright}{\kern0pt}%
\endisatagproof
{\isafoldproof}%
%
\isadelimproof
%
\endisadelimproof
\isanewline
\isanewline
\isacommand{lemma}\isamarkupfalse%
\ add{\isacharcolon}{\kern0pt}\isanewline
\ \ \isakeyword{assumes}\ {\isachardoublequoteopen}adapted{\isacharunderscore}{\kern0pt}process\ M\ F\ Y{\isachardoublequoteclose}\isanewline
\ \ \isakeyword{shows}\ {\isachardoublequoteopen}adapted{\isacharunderscore}{\kern0pt}process\ M\ F\ {\isacharparenleft}{\kern0pt}{\isasymlambda}i\ {\isasymxi}{\isachardot}{\kern0pt}\ X\ i\ {\isasymxi}\ {\isacharplus}{\kern0pt}\ Y\ i\ {\isasymxi}{\isacharparenright}{\kern0pt}{\isachardoublequoteclose}\isanewline
%
\isadelimproof
%
\endisadelimproof
%
\isatagproof
\isacommand{proof}\isamarkupfalse%
\ {\isacharminus}{\kern0pt}\isanewline
\ \ \isacommand{interpret}\isamarkupfalse%
\ Y{\isacharcolon}{\kern0pt}\ adapted{\isacharunderscore}{\kern0pt}process\ M\ F\ Y\ \isacommand{by}\isamarkupfalse%
\ {\isacharparenleft}{\kern0pt}rule\ assms{\isacharparenright}{\kern0pt}\isanewline
\ \ \isacommand{show}\isamarkupfalse%
\ {\isacharquery}{\kern0pt}thesis\ \isacommand{by}\isamarkupfalse%
\ {\isacharparenleft}{\kern0pt}unfold{\isacharunderscore}{\kern0pt}locales{\isacharparenright}{\kern0pt}\ {\isacharparenleft}{\kern0pt}auto\ simp\ add{\isacharcolon}{\kern0pt}\ borel{\isacharunderscore}{\kern0pt}measurable{\isacharunderscore}{\kern0pt}add\ adapted\ random{\isacharunderscore}{\kern0pt}variable\ Y{\isachardot}{\kern0pt}random{\isacharunderscore}{\kern0pt}variable\ Y{\isachardot}{\kern0pt}adapted{\isacharparenright}{\kern0pt}\isanewline
\isacommand{qed}\isamarkupfalse%
%
\endisatagproof
{\isafoldproof}%
%
\isadelimproof
\isanewline
%
\endisadelimproof
\isanewline
\isacommand{lemma}\isamarkupfalse%
\ diff{\isacharcolon}{\kern0pt}\isanewline
\ \ \isakeyword{assumes}\ {\isachardoublequoteopen}adapted{\isacharunderscore}{\kern0pt}process\ M\ F\ Y{\isachardoublequoteclose}\isanewline
\ \ \isakeyword{shows}\ {\isachardoublequoteopen}adapted{\isacharunderscore}{\kern0pt}process\ M\ F\ {\isacharparenleft}{\kern0pt}{\isasymlambda}i\ {\isasymxi}{\isachardot}{\kern0pt}\ X\ i\ {\isasymxi}\ {\isacharminus}{\kern0pt}\ Y\ i\ {\isasymxi}{\isacharparenright}{\kern0pt}{\isachardoublequoteclose}\isanewline
%
\isadelimproof
%
\endisadelimproof
%
\isatagproof
\isacommand{proof}\isamarkupfalse%
\ {\isacharminus}{\kern0pt}\isanewline
\ \ \isacommand{interpret}\isamarkupfalse%
\ Y{\isacharcolon}{\kern0pt}\ adapted{\isacharunderscore}{\kern0pt}process\ M\ F\ Y\ \isacommand{by}\isamarkupfalse%
\ {\isacharparenleft}{\kern0pt}rule\ assms{\isacharparenright}{\kern0pt}\isanewline
\ \ \isacommand{show}\isamarkupfalse%
\ {\isacharquery}{\kern0pt}thesis\ \isacommand{by}\isamarkupfalse%
\ {\isacharparenleft}{\kern0pt}unfold{\isacharunderscore}{\kern0pt}locales{\isacharparenright}{\kern0pt}\ {\isacharparenleft}{\kern0pt}auto\ simp\ add{\isacharcolon}{\kern0pt}\ borel{\isacharunderscore}{\kern0pt}measurable{\isacharunderscore}{\kern0pt}diff\ adapted\ random{\isacharunderscore}{\kern0pt}variable\ Y{\isachardot}{\kern0pt}random{\isacharunderscore}{\kern0pt}variable\ Y{\isachardot}{\kern0pt}adapted{\isacharparenright}{\kern0pt}\isanewline
\isacommand{qed}\isamarkupfalse%
%
\endisatagproof
{\isafoldproof}%
%
\isadelimproof
\isanewline
%
\endisadelimproof
\isanewline
\isacommand{lemma}\isamarkupfalse%
\ uminus{\isacharcolon}{\kern0pt}\ {\isachardoublequoteopen}adapted{\isacharunderscore}{\kern0pt}process\ M\ F\ {\isacharparenleft}{\kern0pt}{\isacharminus}{\kern0pt}X{\isacharparenright}{\kern0pt}{\isachardoublequoteclose}%
\isadelimproof
\ %
\endisadelimproof
%
\isatagproof
\isacommand{using}\isamarkupfalse%
\ scaleR{\isacharunderscore}{\kern0pt}const{\isacharbrackleft}{\kern0pt}of\ {\isachardoublequoteopen}{\isacharminus}{\kern0pt}{\isadigit{1}}{\isachardoublequoteclose}{\isacharbrackright}{\kern0pt}\ \isacommand{by}\isamarkupfalse%
\ {\isacharparenleft}{\kern0pt}simp\ add{\isacharcolon}{\kern0pt}\ fun{\isacharunderscore}{\kern0pt}Compl{\isacharunderscore}{\kern0pt}def{\isacharparenright}{\kern0pt}%
\endisatagproof
{\isafoldproof}%
%
\isadelimproof
%
\endisadelimproof
\isanewline
\isanewline
\isacommand{end}\isamarkupfalse%
\isanewline
\isanewline
\isacommand{locale}\isamarkupfalse%
\ adapted{\isacharunderscore}{\kern0pt}process{\isacharunderscore}{\kern0pt}order\ {\isacharequal}{\kern0pt}\ adapted{\isacharunderscore}{\kern0pt}process\ M\ F\ X\ \isakeyword{for}\ M\ F\ \isakeyword{and}\ X\ {\isacharcolon}{\kern0pt}{\isacharcolon}{\kern0pt}\ {\isachardoublequoteopen}{\isacharprime}{\kern0pt}t\ {\isacharcolon}{\kern0pt}{\isacharcolon}{\kern0pt}\ {\isacharbraceleft}{\kern0pt}second{\isacharunderscore}{\kern0pt}countable{\isacharunderscore}{\kern0pt}topology{\isacharcomma}{\kern0pt}\ linorder{\isacharunderscore}{\kern0pt}topology{\isacharcomma}{\kern0pt}\ order{\isacharunderscore}{\kern0pt}bot{\isacharbraceright}{\kern0pt}\ {\isasymRightarrow}\ {\isacharunderscore}{\kern0pt}\ {\isasymRightarrow}\ {\isacharunderscore}{\kern0pt}\ {\isacharcolon}{\kern0pt}{\isacharcolon}{\kern0pt}\ {\isacharbraceleft}{\kern0pt}linorder{\isacharunderscore}{\kern0pt}topology{\isacharcomma}{\kern0pt}\ ordered{\isacharunderscore}{\kern0pt}real{\isacharunderscore}{\kern0pt}vector{\isacharbraceright}{\kern0pt}{\isachardoublequoteclose}%
\isadelimdocument
%
\endisadelimdocument
%
\isatagdocument
%
\isamarkupsubsection{Discrete-Time Processes%
}
\isamarkuptrue%
%
\endisatagdocument
{\isafolddocument}%
%
\isadelimdocument
%
\endisadelimdocument
\isacommand{locale}\isamarkupfalse%
\ discrete{\isacharunderscore}{\kern0pt}time{\isacharunderscore}{\kern0pt}stochastic{\isacharunderscore}{\kern0pt}process\ {\isacharequal}{\kern0pt}\ stochastic{\isacharunderscore}{\kern0pt}process\ M\ X\ \isakeyword{for}\ M\ \isakeyword{and}\ X\ {\isacharcolon}{\kern0pt}{\isacharcolon}{\kern0pt}\ {\isachardoublequoteopen}nat\ {\isasymRightarrow}\ {\isacharunderscore}{\kern0pt}\ {\isasymRightarrow}\ {\isacharunderscore}{\kern0pt}{\isachardoublequoteclose}\isanewline
\isacommand{locale}\isamarkupfalse%
\ discrete{\isacharunderscore}{\kern0pt}time{\isacharunderscore}{\kern0pt}adapted{\isacharunderscore}{\kern0pt}process\ {\isacharequal}{\kern0pt}\ adapted{\isacharunderscore}{\kern0pt}process\ M\ F\ X\ \isakeyword{for}\ M\ F\ \isakeyword{and}\ X\ {\isacharcolon}{\kern0pt}{\isacharcolon}{\kern0pt}\ {\isachardoublequoteopen}nat\ {\isasymRightarrow}\ {\isacharunderscore}{\kern0pt}\ {\isasymRightarrow}\ {\isacharunderscore}{\kern0pt}{\isachardoublequoteclose}\isanewline
\isacommand{locale}\isamarkupfalse%
\ discrete{\isacharunderscore}{\kern0pt}time{\isacharunderscore}{\kern0pt}adapted{\isacharunderscore}{\kern0pt}process{\isacharunderscore}{\kern0pt}order\ {\isacharequal}{\kern0pt}\ adapted{\isacharunderscore}{\kern0pt}process{\isacharunderscore}{\kern0pt}order\ M\ F\ X\ \isakeyword{for}\ M\ F\ \isakeyword{and}\ X\ {\isacharcolon}{\kern0pt}{\isacharcolon}{\kern0pt}\ {\isachardoublequoteopen}nat\ {\isasymRightarrow}\ {\isacharunderscore}{\kern0pt}\ {\isasymRightarrow}\ {\isacharunderscore}{\kern0pt}{\isachardoublequoteclose}\isanewline
\isanewline
\isacommand{sublocale}\isamarkupfalse%
\ discrete{\isacharunderscore}{\kern0pt}time{\isacharunderscore}{\kern0pt}adapted{\isacharunderscore}{\kern0pt}process{\isacharunderscore}{\kern0pt}order\ {\isasymsubseteq}\ discrete{\isacharunderscore}{\kern0pt}time{\isacharunderscore}{\kern0pt}adapted{\isacharunderscore}{\kern0pt}process%
\isadelimproof
\ %
\endisadelimproof
%
\isatagproof
\isacommand{by}\isamarkupfalse%
\ {\isacharparenleft}{\kern0pt}unfold{\isacharunderscore}{\kern0pt}locales{\isacharparenright}{\kern0pt}%
\endisatagproof
{\isafoldproof}%
%
\isadelimproof
%
\endisadelimproof
\isanewline
\isacommand{sublocale}\isamarkupfalse%
\ discrete{\isacharunderscore}{\kern0pt}time{\isacharunderscore}{\kern0pt}adapted{\isacharunderscore}{\kern0pt}process\ {\isasymsubseteq}\ discrete{\isacharunderscore}{\kern0pt}time{\isacharunderscore}{\kern0pt}stochastic{\isacharunderscore}{\kern0pt}process%
\isadelimproof
\ %
\endisadelimproof
%
\isatagproof
\isacommand{by}\isamarkupfalse%
\ {\isacharparenleft}{\kern0pt}unfold{\isacharunderscore}{\kern0pt}locales{\isacharparenright}{\kern0pt}%
\endisatagproof
{\isafoldproof}%
%
\isadelimproof
%
\endisadelimproof
\isanewline
\isacommand{sublocale}\isamarkupfalse%
\ discrete{\isacharunderscore}{\kern0pt}time{\isacharunderscore}{\kern0pt}adapted{\isacharunderscore}{\kern0pt}process\ {\isasymsubseteq}\ nat{\isacharunderscore}{\kern0pt}filtered{\isacharunderscore}{\kern0pt}sigma{\isacharunderscore}{\kern0pt}finite{\isacharunderscore}{\kern0pt}measure%
\isadelimproof
\ %
\endisadelimproof
%
\isatagproof
\isacommand{by}\isamarkupfalse%
\ {\isacharparenleft}{\kern0pt}unfold{\isacharunderscore}{\kern0pt}locales{\isacharparenright}{\kern0pt}%
\endisatagproof
{\isafoldproof}%
%
\isadelimproof
%
\endisadelimproof
%
\isadelimdocument
%
\endisadelimdocument
%
\isatagdocument
%
\isamarkupsubsection{Predictable Processes%
}
\isamarkuptrue%
%
\endisatagdocument
{\isafolddocument}%
%
\isadelimdocument
%
\endisadelimdocument
\isacommand{context}\isamarkupfalse%
\ filtered{\isacharunderscore}{\kern0pt}sigma{\isacharunderscore}{\kern0pt}finite{\isacharunderscore}{\kern0pt}measure\isanewline
\isakeyword{begin}\isanewline
\isanewline
\isacommand{definition}\isamarkupfalse%
\ predictable{\isacharunderscore}{\kern0pt}sigma\ {\isacharcolon}{\kern0pt}{\isacharcolon}{\kern0pt}\ {\isachardoublequoteopen}{\isacharparenleft}{\kern0pt}{\isacharprime}{\kern0pt}t\ {\isasymtimes}\ {\isacharprime}{\kern0pt}a{\isacharparenright}{\kern0pt}\ measure{\isachardoublequoteclose}\ \isakeyword{where}\isanewline
\ \ {\isachardoublequoteopen}predictable{\isacharunderscore}{\kern0pt}sigma\ {\isacharequal}{\kern0pt}\ sigma\ {\isacharparenleft}{\kern0pt}UNIV\ {\isasymtimes}\ space\ M{\isacharparenright}{\kern0pt}\ {\isacharparenleft}{\kern0pt}{\isacharbraceleft}{\kern0pt}{\isacharbraceleft}{\kern0pt}s{\isacharless}{\kern0pt}{\isachardot}{\kern0pt}{\isachardot}{\kern0pt}t{\isacharbraceright}{\kern0pt}\ {\isasymtimes}\ A\ {\isacharbar}{\kern0pt}\ A\ s\ t{\isachardot}{\kern0pt}\ A\ {\isasymin}\ F\ s\ {\isasymand}\ s\ {\isacharless}{\kern0pt}\ t{\isacharbraceright}{\kern0pt}\ {\isasymunion}\ {\isacharbraceleft}{\kern0pt}{\isacharbraceleft}{\kern0pt}bot{\isacharbraceright}{\kern0pt}\ {\isasymtimes}\ A\ {\isacharbar}{\kern0pt}\ A{\isachardot}{\kern0pt}\ A\ {\isasymin}\ F\ bot{\isacharbraceright}{\kern0pt}{\isacharparenright}{\kern0pt}{\isachardoublequoteclose}\isanewline
\isanewline
\isacommand{lemma}\isamarkupfalse%
\ space{\isacharunderscore}{\kern0pt}predictable{\isacharunderscore}{\kern0pt}sigma{\isacharbrackleft}{\kern0pt}simp{\isacharbrackright}{\kern0pt}{\isacharcolon}{\kern0pt}\ {\isachardoublequoteopen}space\ predictable{\isacharunderscore}{\kern0pt}sigma\ {\isacharequal}{\kern0pt}\ {\isacharparenleft}{\kern0pt}UNIV\ {\isasymtimes}\ space\ M{\isacharparenright}{\kern0pt}{\isachardoublequoteclose}%
\isadelimproof
\ %
\endisadelimproof
%
\isatagproof
\isacommand{unfolding}\isamarkupfalse%
\ predictable{\isacharunderscore}{\kern0pt}sigma{\isacharunderscore}{\kern0pt}def\ space{\isacharunderscore}{\kern0pt}measure{\isacharunderscore}{\kern0pt}of{\isacharunderscore}{\kern0pt}conv\ \isacommand{by}\isamarkupfalse%
\ blast%
\endisatagproof
{\isafoldproof}%
%
\isadelimproof
%
\endisadelimproof
\isanewline
\isanewline
\isacommand{lemma}\isamarkupfalse%
\ sets{\isacharunderscore}{\kern0pt}predictable{\isacharunderscore}{\kern0pt}sigma{\isacharbrackleft}{\kern0pt}simp{\isacharbrackright}{\kern0pt}{\isacharcolon}{\kern0pt}\ {\isachardoublequoteopen}sets\ predictable{\isacharunderscore}{\kern0pt}sigma\ {\isacharequal}{\kern0pt}\ sigma{\isacharunderscore}{\kern0pt}sets\ {\isacharparenleft}{\kern0pt}UNIV\ {\isasymtimes}\ space\ M{\isacharparenright}{\kern0pt}\ {\isacharparenleft}{\kern0pt}{\isacharbraceleft}{\kern0pt}{\isacharbraceleft}{\kern0pt}s{\isacharless}{\kern0pt}{\isachardot}{\kern0pt}{\isachardot}{\kern0pt}t{\isacharbraceright}{\kern0pt}\ {\isasymtimes}\ A\ {\isacharbar}{\kern0pt}\ A\ s\ t{\isachardot}{\kern0pt}\ A\ {\isasymin}\ F\ s\ {\isasymand}\ s\ {\isacharless}{\kern0pt}\ t{\isacharbraceright}{\kern0pt}\ {\isasymunion}\ {\isacharbraceleft}{\kern0pt}{\isacharbraceleft}{\kern0pt}bot{\isacharbraceright}{\kern0pt}\ {\isasymtimes}\ A\ {\isacharbar}{\kern0pt}\ A{\isachardot}{\kern0pt}\ A\ {\isasymin}\ F\ bot{\isacharbraceright}{\kern0pt}{\isacharparenright}{\kern0pt}{\isachardoublequoteclose}\ \isanewline
%
\isadelimproof
\ \ %
\endisadelimproof
%
\isatagproof
\isacommand{unfolding}\isamarkupfalse%
\ predictable{\isacharunderscore}{\kern0pt}sigma{\isacharunderscore}{\kern0pt}def\ sets{\isacharunderscore}{\kern0pt}measure{\isacharunderscore}{\kern0pt}of{\isacharunderscore}{\kern0pt}conv\ \isanewline
\ \ \isacommand{using}\isamarkupfalse%
\ space{\isacharunderscore}{\kern0pt}F\ sets{\isachardot}{\kern0pt}sets{\isacharunderscore}{\kern0pt}into{\isacharunderscore}{\kern0pt}space\isanewline
\ \ \isacommand{by}\isamarkupfalse%
\ {\isacharparenleft}{\kern0pt}fastforce\ intro{\isacharbang}{\kern0pt}{\isacharcolon}{\kern0pt}\ if{\isacharunderscore}{\kern0pt}P{\isacharparenright}{\kern0pt}%
\endisatagproof
{\isafoldproof}%
%
\isadelimproof
\isanewline
%
\endisadelimproof
\isanewline
\isacommand{definition}\isamarkupfalse%
\ predictable\ {\isacharcolon}{\kern0pt}{\isacharcolon}{\kern0pt}\ {\isachardoublequoteopen}{\isacharparenleft}{\kern0pt}{\isacharprime}{\kern0pt}t\ {\isasymRightarrow}\ {\isacharprime}{\kern0pt}a\ {\isasymRightarrow}\ {\isacharprime}{\kern0pt}b\ {\isacharcolon}{\kern0pt}{\isacharcolon}{\kern0pt}\ {\isacharbraceleft}{\kern0pt}second{\isacharunderscore}{\kern0pt}countable{\isacharunderscore}{\kern0pt}topology{\isacharcomma}{\kern0pt}banach{\isacharbraceright}{\kern0pt}{\isacharparenright}{\kern0pt}\ {\isasymRightarrow}\ bool{\isachardoublequoteclose}\ \isakeyword{where}\isanewline
\ \ {\isachardoublequoteopen}predictable\ X\ {\isacharequal}{\kern0pt}\ {\isacharparenleft}{\kern0pt}case{\isacharunderscore}{\kern0pt}prod\ X\ {\isasymin}\ borel{\isacharunderscore}{\kern0pt}measurable\ {\isacharparenleft}{\kern0pt}predictable{\isacharunderscore}{\kern0pt}sigma{\isacharparenright}{\kern0pt}{\isacharparenright}{\kern0pt}{\isachardoublequoteclose}\isanewline
\isanewline
\isacommand{lemmas}\isamarkupfalse%
\ predictableD\ {\isacharequal}{\kern0pt}\ measurable{\isacharunderscore}{\kern0pt}sets{\isacharbrackleft}{\kern0pt}OF\ predictable{\isacharunderscore}{\kern0pt}def{\isacharbrackleft}{\kern0pt}THEN\ iffD{\isadigit{1}}{\isacharbrackright}{\kern0pt}{\isacharcomma}{\kern0pt}\ unfolded\ space{\isacharunderscore}{\kern0pt}predictable{\isacharunderscore}{\kern0pt}sigma{\isacharbrackright}{\kern0pt}\isanewline
\isanewline
\isacommand{lemma}\isamarkupfalse%
\ {\isacharparenleft}{\kern0pt}\isakeyword{in}\ nat{\isacharunderscore}{\kern0pt}filtered{\isacharunderscore}{\kern0pt}sigma{\isacharunderscore}{\kern0pt}finite{\isacharunderscore}{\kern0pt}measure{\isacharparenright}{\kern0pt}\ predictable{\isacharunderscore}{\kern0pt}sets{\isacharunderscore}{\kern0pt}in{\isacharunderscore}{\kern0pt}F{\isacharcolon}{\kern0pt}\isanewline
\ \ \isakeyword{assumes}\ {\isachardoublequoteopen}{\isacharparenleft}{\kern0pt}{\isasymUnion}i{\isachardot}{\kern0pt}\ {\isacharbraceleft}{\kern0pt}i{\isacharbraceright}{\kern0pt}\ {\isasymtimes}\ A\ i{\isacharparenright}{\kern0pt}\ {\isasymin}\ predictable{\isacharunderscore}{\kern0pt}sigma{\isachardoublequoteclose}\isanewline
\ \ \isakeyword{shows}\ {\isachardoublequoteopen}A\ {\isacharparenleft}{\kern0pt}Suc\ i{\isacharparenright}{\kern0pt}\ {\isasymin}\ F\ i{\isachardoublequoteclose}\ {\isachardoublequoteopen}A\ {\isadigit{0}}\ {\isasymin}\ F\ {\isadigit{0}}{\isachardoublequoteclose}\isanewline
%
\isadelimproof
\ \ %
\endisadelimproof
%
\isatagproof
\isacommand{using}\isamarkupfalse%
\ assms\ \isacommand{unfolding}\isamarkupfalse%
\ sets{\isacharunderscore}{\kern0pt}predictable{\isacharunderscore}{\kern0pt}sigma\isanewline
\isacommand{proof}\isamarkupfalse%
\ {\isacharparenleft}{\kern0pt}induction\ {\isachardoublequoteopen}{\isacharparenleft}{\kern0pt}{\isasymUnion}i{\isachardot}{\kern0pt}\ {\isacharbraceleft}{\kern0pt}i{\isacharbraceright}{\kern0pt}\ {\isasymtimes}\ A\ i{\isacharparenright}{\kern0pt}{\isachardoublequoteclose}\ arbitrary{\isacharcolon}{\kern0pt}\ A{\isacharparenright}{\kern0pt}\isanewline
\ \ \isacommand{case}\isamarkupfalse%
\ Basic\isanewline
\ \ \isacommand{{\isacharbraceleft}{\kern0pt}}\isamarkupfalse%
\isanewline
\ \ \ \ \isacommand{assume}\isamarkupfalse%
\ {\isachardoublequoteopen}{\isasymexists}S{\isachardot}{\kern0pt}\ {\isacharparenleft}{\kern0pt}{\isasymUnion}i{\isachardot}{\kern0pt}\ {\isacharbraceleft}{\kern0pt}i{\isacharbraceright}{\kern0pt}\ {\isasymtimes}\ A\ i{\isacharparenright}{\kern0pt}\ {\isacharequal}{\kern0pt}\ {\isacharbraceleft}{\kern0pt}bot{\isacharbraceright}{\kern0pt}\ {\isasymtimes}\ S{\isachardoublequoteclose}\isanewline
\ \ \ \ \isacommand{then}\isamarkupfalse%
\ \isacommand{obtain}\isamarkupfalse%
\ S\ \isakeyword{where}\ S{\isacharcolon}{\kern0pt}\ {\isachardoublequoteopen}{\isacharparenleft}{\kern0pt}{\isasymUnion}i{\isachardot}{\kern0pt}\ {\isacharbraceleft}{\kern0pt}i{\isacharbraceright}{\kern0pt}\ {\isasymtimes}\ A\ i{\isacharparenright}{\kern0pt}\ {\isacharequal}{\kern0pt}\ {\isacharbraceleft}{\kern0pt}bot{\isacharbraceright}{\kern0pt}\ {\isasymtimes}\ S{\isachardoublequoteclose}\ \isacommand{by}\isamarkupfalse%
\ blast\isanewline
\ \ \ \ \isacommand{hence}\isamarkupfalse%
\ {\isachardoublequoteopen}S\ {\isasymin}\ F\ {\isadigit{0}}{\isachardoublequoteclose}\ \isacommand{using}\isamarkupfalse%
\ Basic\ \isacommand{by}\isamarkupfalse%
\ {\isacharparenleft}{\kern0pt}fastforce\ simp\ add{\isacharcolon}{\kern0pt}\ times{\isacharunderscore}{\kern0pt}eq{\isacharunderscore}{\kern0pt}iff\ bot{\isacharunderscore}{\kern0pt}nat{\isacharunderscore}{\kern0pt}def{\isacharparenright}{\kern0pt}\isanewline
\ \ \ \ \isacommand{moreover}\isamarkupfalse%
\ \isacommand{have}\isamarkupfalse%
\ {\isachardoublequoteopen}A\ i\ {\isacharequal}{\kern0pt}\ {\isacharbraceleft}{\kern0pt}{\isacharbraceright}{\kern0pt}{\isachardoublequoteclose}\ \isakeyword{if}\ {\isachardoublequoteopen}i\ {\isasymnoteq}\ bot{\isachardoublequoteclose}\ \isakeyword{for}\ i\ \isacommand{using}\isamarkupfalse%
\ that\ S\ \isacommand{by}\isamarkupfalse%
\ blast\isanewline
\ \ \ \ \isacommand{moreover}\isamarkupfalse%
\ \isacommand{have}\isamarkupfalse%
\ {\isachardoublequoteopen}A\ bot\ {\isacharequal}{\kern0pt}\ S{\isachardoublequoteclose}\ \isacommand{using}\isamarkupfalse%
\ S\ \isacommand{by}\isamarkupfalse%
\ blast\isanewline
\ \ \ \ \isacommand{ultimately}\isamarkupfalse%
\ \isacommand{have}\isamarkupfalse%
\ {\isachardoublequoteopen}A\ {\isacharparenleft}{\kern0pt}Suc\ i{\isacharparenright}{\kern0pt}\ {\isasymin}\ F\ i{\isachardoublequoteclose}\ {\isachardoublequoteopen}A\ {\isadigit{0}}\ {\isasymin}\ F\ {\isadigit{0}}{\isachardoublequoteclose}\ \isakeyword{for}\ i\ \isacommand{unfolding}\isamarkupfalse%
\ bot{\isacharunderscore}{\kern0pt}nat{\isacharunderscore}{\kern0pt}def\ \isacommand{by}\isamarkupfalse%
\ {\isacharparenleft}{\kern0pt}auto\ simp\ add{\isacharcolon}{\kern0pt}\ bot{\isacharunderscore}{\kern0pt}nat{\isacharunderscore}{\kern0pt}def{\isacharparenright}{\kern0pt}\isanewline
\ \ \isacommand{{\isacharbraceright}{\kern0pt}}\isamarkupfalse%
\isanewline
\ \ \isacommand{note}\isamarkupfalse%
\ {\isacharasterisk}{\kern0pt}\ {\isacharequal}{\kern0pt}\ this\isanewline
\ \ \isacommand{{\isacharbraceleft}{\kern0pt}}\isamarkupfalse%
\isanewline
\ \ \ \ \isacommand{assume}\isamarkupfalse%
\ {\isachardoublequoteopen}{\isasymnexists}S{\isachardot}{\kern0pt}\ {\isacharparenleft}{\kern0pt}{\isasymUnion}i{\isachardot}{\kern0pt}\ {\isacharbraceleft}{\kern0pt}i{\isacharbraceright}{\kern0pt}\ {\isasymtimes}\ A\ i{\isacharparenright}{\kern0pt}\ {\isacharequal}{\kern0pt}\ {\isacharbraceleft}{\kern0pt}bot{\isacharbraceright}{\kern0pt}\ {\isasymtimes}\ S{\isachardoublequoteclose}\isanewline
\ \ \ \ \isacommand{then}\isamarkupfalse%
\ \isacommand{obtain}\isamarkupfalse%
\ s\ t\ B\ \isakeyword{where}\ B{\isacharcolon}{\kern0pt}\ {\isachardoublequoteopen}{\isacharparenleft}{\kern0pt}{\isasymUnion}i{\isachardot}{\kern0pt}\ {\isacharbraceleft}{\kern0pt}i{\isacharbraceright}{\kern0pt}\ {\isasymtimes}\ A\ i{\isacharparenright}{\kern0pt}\ {\isacharequal}{\kern0pt}\ {\isacharbraceleft}{\kern0pt}s{\isacharless}{\kern0pt}{\isachardot}{\kern0pt}{\isachardot}{\kern0pt}t{\isacharbraceright}{\kern0pt}\ {\isasymtimes}\ B{\isachardoublequoteclose}\ {\isachardoublequoteopen}B\ {\isasymin}\ sets\ {\isacharparenleft}{\kern0pt}F\ s{\isacharparenright}{\kern0pt}{\isachardoublequoteclose}\ {\isachardoublequoteopen}s\ {\isacharless}{\kern0pt}\ t{\isachardoublequoteclose}\ \isacommand{using}\isamarkupfalse%
\ Basic\ \isacommand{by}\isamarkupfalse%
\ auto\isanewline
\ \ \ \ \isacommand{hence}\isamarkupfalse%
\ {\isachardoublequoteopen}A\ i\ {\isacharequal}{\kern0pt}\ B{\isachardoublequoteclose}\ \isakeyword{if}\ {\isachardoublequoteopen}i\ {\isasymin}\ {\isacharbraceleft}{\kern0pt}s{\isacharless}{\kern0pt}{\isachardot}{\kern0pt}{\isachardot}{\kern0pt}t{\isacharbraceright}{\kern0pt}{\isachardoublequoteclose}\ \isakeyword{for}\ i\ \isacommand{using}\isamarkupfalse%
\ that\ \isacommand{by}\isamarkupfalse%
\ fast\isanewline
\ \ \ \ \isacommand{moreover}\isamarkupfalse%
\ \isacommand{have}\isamarkupfalse%
\ {\isachardoublequoteopen}A\ i\ {\isacharequal}{\kern0pt}\ {\isacharbraceleft}{\kern0pt}{\isacharbraceright}{\kern0pt}{\isachardoublequoteclose}\ \isakeyword{if}\ {\isachardoublequoteopen}i\ {\isasymnotin}\ {\isacharbraceleft}{\kern0pt}s{\isacharless}{\kern0pt}{\isachardot}{\kern0pt}{\isachardot}{\kern0pt}t{\isacharbraceright}{\kern0pt}{\isachardoublequoteclose}\ \isakeyword{for}\ i\ \isacommand{using}\isamarkupfalse%
\ B\ that\ \isacommand{by}\isamarkupfalse%
\ fastforce\isanewline
\ \ \ \ \isacommand{ultimately}\isamarkupfalse%
\ \isacommand{have}\isamarkupfalse%
\ {\isachardoublequoteopen}A\ {\isacharparenleft}{\kern0pt}Suc\ i{\isacharparenright}{\kern0pt}\ {\isasymin}\ F\ i{\isachardoublequoteclose}\ {\isachardoublequoteopen}A\ {\isadigit{0}}\ {\isasymin}\ F\ {\isadigit{0}}{\isachardoublequoteclose}\ \isakeyword{for}\ i\ \isacommand{unfolding}\isamarkupfalse%
\ bot{\isacharunderscore}{\kern0pt}nat{\isacharunderscore}{\kern0pt}def\ \isacommand{using}\isamarkupfalse%
\ B\ sets{\isacharunderscore}{\kern0pt}F{\isacharunderscore}{\kern0pt}mono\ \isacommand{by}\isamarkupfalse%
\ {\isacharparenleft}{\kern0pt}auto\ simp\ add{\isacharcolon}{\kern0pt}\ bot{\isacharunderscore}{\kern0pt}nat{\isacharunderscore}{\kern0pt}def{\isacharparenright}{\kern0pt}\ {\isacharparenleft}{\kern0pt}metis\ less{\isacharunderscore}{\kern0pt}Suc{\isacharunderscore}{\kern0pt}eq{\isacharunderscore}{\kern0pt}le\ sets{\isachardot}{\kern0pt}empty{\isacharunderscore}{\kern0pt}sets\ subset{\isacharunderscore}{\kern0pt}eq{\isacharparenright}{\kern0pt}\isanewline
\ \ \isacommand{{\isacharbraceright}{\kern0pt}}\isamarkupfalse%
\isanewline
\ \ \isacommand{note}\isamarkupfalse%
\ {\isacharasterisk}{\kern0pt}{\isacharasterisk}{\kern0pt}\ {\isacharequal}{\kern0pt}\ this\isanewline
\ \ \isacommand{show}\isamarkupfalse%
\ {\isachardoublequoteopen}A\ {\isacharparenleft}{\kern0pt}Suc\ i{\isacharparenright}{\kern0pt}\ {\isasymin}\ sets\ {\isacharparenleft}{\kern0pt}F\ i{\isacharparenright}{\kern0pt}{\isachardoublequoteclose}\ {\isachardoublequoteopen}A\ {\isadigit{0}}\ {\isasymin}\ sets\ {\isacharparenleft}{\kern0pt}F\ {\isadigit{0}}{\isacharparenright}{\kern0pt}{\isachardoublequoteclose}\ \isacommand{using}\isamarkupfalse%
\ {\isacharasterisk}{\kern0pt}{\isacharparenleft}{\kern0pt}{\isadigit{1}}{\isacharparenright}{\kern0pt}{\isacharbrackleft}{\kern0pt}of\ i{\isacharbrackright}{\kern0pt}\ {\isacharasterisk}{\kern0pt}{\isacharparenleft}{\kern0pt}{\isadigit{2}}{\isacharparenright}{\kern0pt}\ {\isacharasterisk}{\kern0pt}{\isacharasterisk}{\kern0pt}{\isacharparenleft}{\kern0pt}{\isadigit{1}}{\isacharparenright}{\kern0pt}{\isacharbrackleft}{\kern0pt}of\ i{\isacharbrackright}{\kern0pt}\ {\isacharasterisk}{\kern0pt}{\isacharasterisk}{\kern0pt}{\isacharparenleft}{\kern0pt}{\isadigit{2}}{\isacharparenright}{\kern0pt}\ \isacommand{by}\isamarkupfalse%
\ auto\ blast{\isacharplus}{\kern0pt}\ \isanewline
\isacommand{next}\isamarkupfalse%
\isanewline
\ \ \isacommand{case}\isamarkupfalse%
\ Empty\isanewline
\ \ \isacommand{{\isacharbraceleft}{\kern0pt}}\isamarkupfalse%
\isanewline
\ \ \ \ \isacommand{case}\isamarkupfalse%
\ {\isadigit{1}}\isanewline
\ \ \ \ \isacommand{then}\isamarkupfalse%
\ \isacommand{show}\isamarkupfalse%
\ {\isacharquery}{\kern0pt}case\ \isacommand{using}\isamarkupfalse%
\ Empty\ \isacommand{by}\isamarkupfalse%
\ simp\isanewline
\ \ \isacommand{next}\isamarkupfalse%
\isanewline
\ \ \ \ \isacommand{case}\isamarkupfalse%
\ {\isadigit{2}}\isanewline
\ \ \ \ \isacommand{then}\isamarkupfalse%
\ \isacommand{show}\isamarkupfalse%
\ {\isacharquery}{\kern0pt}case\ \isacommand{using}\isamarkupfalse%
\ Empty\ \isacommand{by}\isamarkupfalse%
\ simp\isanewline
\ \ \isacommand{{\isacharbraceright}{\kern0pt}}\isamarkupfalse%
\isanewline
\isacommand{next}\isamarkupfalse%
\isanewline
\ \ \isacommand{case}\isamarkupfalse%
\ {\isacharparenleft}{\kern0pt}Compl\ a{\isacharparenright}{\kern0pt}\isanewline
\ \ \isacommand{have}\isamarkupfalse%
\ a{\isacharunderscore}{\kern0pt}in{\isacharcolon}{\kern0pt}\ {\isachardoublequoteopen}a\ {\isasymsubseteq}\ UNIV\ {\isasymtimes}\ space\ M{\isachardoublequoteclose}\ \isacommand{using}\isamarkupfalse%
\ Compl{\isacharparenleft}{\kern0pt}{\isadigit{1}}{\isacharparenright}{\kern0pt}\ sets{\isachardot}{\kern0pt}sets{\isacharunderscore}{\kern0pt}into{\isacharunderscore}{\kern0pt}space\ sets{\isacharunderscore}{\kern0pt}predictable{\isacharunderscore}{\kern0pt}sigma\ space{\isacharunderscore}{\kern0pt}predictable{\isacharunderscore}{\kern0pt}sigma\ \isacommand{by}\isamarkupfalse%
\ metis\isanewline
\ \ \isacommand{hence}\isamarkupfalse%
\ A{\isacharunderscore}{\kern0pt}in{\isacharcolon}{\kern0pt}\ {\isachardoublequoteopen}A\ i\ {\isasymsubseteq}\ space\ M{\isachardoublequoteclose}\ \isakeyword{for}\ i\ \isacommand{using}\isamarkupfalse%
\ Compl{\isacharparenleft}{\kern0pt}{\isadigit{4}}{\isacharparenright}{\kern0pt}\ \isacommand{by}\isamarkupfalse%
\ blast\isanewline
\ \ \isacommand{have}\isamarkupfalse%
\ a{\isacharcolon}{\kern0pt}\ {\isachardoublequoteopen}a\ {\isacharequal}{\kern0pt}\ UNIV\ {\isasymtimes}\ space\ M\ {\isacharminus}{\kern0pt}\ {\isacharparenleft}{\kern0pt}{\isasymUnion}i{\isachardot}{\kern0pt}\ {\isacharbraceleft}{\kern0pt}i{\isacharbraceright}{\kern0pt}\ {\isasymtimes}\ A\ i{\isacharparenright}{\kern0pt}{\isachardoublequoteclose}\ \isacommand{using}\isamarkupfalse%
\ a{\isacharunderscore}{\kern0pt}in\ Compl{\isacharparenleft}{\kern0pt}{\isadigit{4}}{\isacharparenright}{\kern0pt}\ \isacommand{by}\isamarkupfalse%
\ blast\isanewline
\ \ \isacommand{also}\isamarkupfalse%
\ \isacommand{have}\isamarkupfalse%
\ {\isachardoublequoteopen}{\isachardot}{\kern0pt}{\isachardot}{\kern0pt}{\isachardot}{\kern0pt}\ {\isacharequal}{\kern0pt}\ {\isacharparenleft}{\kern0pt}{\isasymUnion}j{\isachardot}{\kern0pt}\ {\isacharbraceleft}{\kern0pt}j{\isacharbraceright}{\kern0pt}\ {\isasymtimes}\ {\isacharparenleft}{\kern0pt}space\ M\ {\isacharminus}{\kern0pt}\ A\ j{\isacharparenright}{\kern0pt}{\isacharparenright}{\kern0pt}{\isachardoublequoteclose}\ \isacommand{by}\isamarkupfalse%
\ blast\isanewline
\ \ \isacommand{finally}\isamarkupfalse%
\ \isacommand{have}\isamarkupfalse%
\ {\isacharasterisk}{\kern0pt}{\isacharcolon}{\kern0pt}\ {\isachardoublequoteopen}{\isacharparenleft}{\kern0pt}space\ M\ {\isacharminus}{\kern0pt}\ A\ {\isacharparenleft}{\kern0pt}Suc\ i{\isacharparenright}{\kern0pt}{\isacharparenright}{\kern0pt}\ {\isasymin}\ F\ i{\isachardoublequoteclose}\ {\isachardoublequoteopen}{\isacharparenleft}{\kern0pt}space\ M\ {\isacharminus}{\kern0pt}\ A\ {\isadigit{0}}{\isacharparenright}{\kern0pt}\ {\isasymin}\ F\ {\isadigit{0}}{\isachardoublequoteclose}\ \isacommand{using}\isamarkupfalse%
\ Compl{\isacharparenleft}{\kern0pt}{\isadigit{2}}{\isacharcomma}{\kern0pt}{\isadigit{3}}{\isacharparenright}{\kern0pt}\ \isacommand{by}\isamarkupfalse%
\ auto\isanewline
\ \ \isacommand{{\isacharbraceleft}{\kern0pt}}\isamarkupfalse%
\isanewline
\ \ \ \ \isacommand{case}\isamarkupfalse%
\ {\isadigit{1}}\isanewline
\ \ \ \ \isacommand{then}\isamarkupfalse%
\ \isacommand{show}\isamarkupfalse%
\ {\isacharquery}{\kern0pt}case\ \isacommand{using}\isamarkupfalse%
\ {\isacharasterisk}{\kern0pt}\ A{\isacharunderscore}{\kern0pt}in\ \isacommand{by}\isamarkupfalse%
\ {\isacharparenleft}{\kern0pt}metis\ double{\isacharunderscore}{\kern0pt}diff\ sets{\isachardot}{\kern0pt}compl{\isacharunderscore}{\kern0pt}sets\ space{\isacharunderscore}{\kern0pt}F\ subset{\isacharunderscore}{\kern0pt}refl{\isacharparenright}{\kern0pt}\isanewline
\ \ \isacommand{next}\isamarkupfalse%
\isanewline
\ \ \ \ \isacommand{case}\isamarkupfalse%
\ {\isadigit{2}}\isanewline
\ \ \ \ \isacommand{then}\isamarkupfalse%
\ \isacommand{show}\isamarkupfalse%
\ {\isacharquery}{\kern0pt}case\ \isacommand{using}\isamarkupfalse%
\ {\isacharasterisk}{\kern0pt}\ A{\isacharunderscore}{\kern0pt}in\ \isacommand{by}\isamarkupfalse%
\ {\isacharparenleft}{\kern0pt}metis\ double{\isacharunderscore}{\kern0pt}diff\ sets{\isachardot}{\kern0pt}compl{\isacharunderscore}{\kern0pt}sets\ space{\isacharunderscore}{\kern0pt}F\ subset{\isacharunderscore}{\kern0pt}refl{\isacharparenright}{\kern0pt}\isanewline
\ \ \isacommand{{\isacharbraceright}{\kern0pt}}\isamarkupfalse%
\isanewline
\isacommand{next}\isamarkupfalse%
\isanewline
\ \ \isacommand{case}\isamarkupfalse%
\ {\isacharparenleft}{\kern0pt}Union\ a{\isacharparenright}{\kern0pt}\isanewline
\ \ \isacommand{have}\isamarkupfalse%
\ a{\isacharunderscore}{\kern0pt}in{\isacharcolon}{\kern0pt}\ {\isachardoublequoteopen}a\ i\ {\isasymsubseteq}\ UNIV\ {\isasymtimes}\ space\ M{\isachardoublequoteclose}\ \isakeyword{for}\ i\ \isacommand{using}\isamarkupfalse%
\ Union{\isacharparenleft}{\kern0pt}{\isadigit{1}}{\isacharparenright}{\kern0pt}\ sets{\isachardot}{\kern0pt}sets{\isacharunderscore}{\kern0pt}into{\isacharunderscore}{\kern0pt}space\ sets{\isacharunderscore}{\kern0pt}predictable{\isacharunderscore}{\kern0pt}sigma\ space{\isacharunderscore}{\kern0pt}predictable{\isacharunderscore}{\kern0pt}sigma\ \isacommand{by}\isamarkupfalse%
\ metis\isanewline
\ \ \isacommand{hence}\isamarkupfalse%
\ A{\isacharunderscore}{\kern0pt}in{\isacharcolon}{\kern0pt}\ {\isachardoublequoteopen}A\ i\ {\isasymsubseteq}\ space\ M{\isachardoublequoteclose}\ \isakeyword{for}\ i\ \isacommand{using}\isamarkupfalse%
\ Union{\isacharparenleft}{\kern0pt}{\isadigit{4}}{\isacharparenright}{\kern0pt}\ \isacommand{by}\isamarkupfalse%
\ blast\isanewline
\ \ \isacommand{have}\isamarkupfalse%
\ {\isachardoublequoteopen}snd\ x\ {\isasymin}\ snd\ {\isacharbackquote}{\kern0pt}\ {\isacharparenleft}{\kern0pt}a\ i\ {\isasyminter}\ {\isacharparenleft}{\kern0pt}{\isacharbraceleft}{\kern0pt}fst\ x{\isacharbraceright}{\kern0pt}\ {\isasymtimes}\ space\ M{\isacharparenright}{\kern0pt}{\isacharparenright}{\kern0pt}{\isachardoublequoteclose}\ \isakeyword{if}\ {\isachardoublequoteopen}x\ {\isasymin}\ a\ i{\isachardoublequoteclose}\ \isakeyword{for}\ i\ x\ \isacommand{using}\isamarkupfalse%
\ that\ a{\isacharunderscore}{\kern0pt}in\ \isacommand{by}\isamarkupfalse%
\ fastforce\isanewline
\ \ \isacommand{hence}\isamarkupfalse%
\ a{\isacharunderscore}{\kern0pt}i{\isacharcolon}{\kern0pt}\ {\isachardoublequoteopen}a\ i\ {\isacharequal}{\kern0pt}\ {\isacharparenleft}{\kern0pt}{\isasymUnion}j{\isachardot}{\kern0pt}\ {\isacharbraceleft}{\kern0pt}j{\isacharbraceright}{\kern0pt}\ {\isasymtimes}\ {\isacharparenleft}{\kern0pt}snd\ {\isacharbackquote}{\kern0pt}\ {\isacharparenleft}{\kern0pt}a\ i\ {\isasyminter}\ {\isacharparenleft}{\kern0pt}{\isacharbraceleft}{\kern0pt}j{\isacharbraceright}{\kern0pt}\ {\isasymtimes}\ space\ M{\isacharparenright}{\kern0pt}{\isacharparenright}{\kern0pt}{\isacharparenright}{\kern0pt}{\isacharparenright}{\kern0pt}{\isachardoublequoteclose}\ \isakeyword{for}\ i\ \isacommand{by}\isamarkupfalse%
\ force\isanewline
\ \ \isacommand{have}\isamarkupfalse%
\ A{\isacharunderscore}{\kern0pt}i{\isacharcolon}{\kern0pt}\ {\isachardoublequoteopen}A\ i\ {\isacharequal}{\kern0pt}\ snd\ {\isacharbackquote}{\kern0pt}\ {\isacharparenleft}{\kern0pt}{\isasymUnion}\ {\isacharparenleft}{\kern0pt}range\ a{\isacharparenright}{\kern0pt}\ {\isasyminter}\ {\isacharparenleft}{\kern0pt}{\isacharbraceleft}{\kern0pt}i{\isacharbraceright}{\kern0pt}\ {\isasymtimes}\ space\ M{\isacharparenright}{\kern0pt}{\isacharparenright}{\kern0pt}{\isachardoublequoteclose}\ \isakeyword{for}\ i\ \isacommand{unfolding}\isamarkupfalse%
\ Union{\isacharparenleft}{\kern0pt}{\isadigit{4}}{\isacharparenright}{\kern0pt}\ \isacommand{using}\isamarkupfalse%
\ A{\isacharunderscore}{\kern0pt}in\ \isacommand{by}\isamarkupfalse%
\ force\ \isanewline
\ \ \isacommand{have}\isamarkupfalse%
\ {\isacharasterisk}{\kern0pt}{\isacharcolon}{\kern0pt}\ {\isachardoublequoteopen}snd\ {\isacharbackquote}{\kern0pt}\ {\isacharparenleft}{\kern0pt}a\ j\ {\isasyminter}\ {\isacharparenleft}{\kern0pt}{\isacharbraceleft}{\kern0pt}Suc\ i{\isacharbraceright}{\kern0pt}\ {\isasymtimes}\ space\ M{\isacharparenright}{\kern0pt}{\isacharparenright}{\kern0pt}\ {\isasymin}\ F\ i{\isachardoublequoteclose}\ {\isachardoublequoteopen}snd\ {\isacharbackquote}{\kern0pt}\ {\isacharparenleft}{\kern0pt}a\ j\ {\isasyminter}\ {\isacharparenleft}{\kern0pt}{\isacharbraceleft}{\kern0pt}{\isadigit{0}}{\isacharbraceright}{\kern0pt}\ {\isasymtimes}\ space\ M{\isacharparenright}{\kern0pt}{\isacharparenright}{\kern0pt}\ {\isasymin}\ F\ {\isadigit{0}}{\isachardoublequoteclose}\ \isakeyword{for}\ j\ \isacommand{using}\isamarkupfalse%
\ Union{\isacharparenleft}{\kern0pt}{\isadigit{2}}{\isacharcomma}{\kern0pt}{\isadigit{3}}{\isacharparenright}{\kern0pt}{\isacharbrackleft}{\kern0pt}OF\ a{\isacharunderscore}{\kern0pt}i{\isacharbrackright}{\kern0pt}\ \isacommand{by}\isamarkupfalse%
\ auto\isanewline
\ \ \isacommand{{\isacharbraceleft}{\kern0pt}}\isamarkupfalse%
\isanewline
\ \ \ \ \isacommand{case}\isamarkupfalse%
\ {\isadigit{1}}\isanewline
\ \ \ \ \isacommand{have}\isamarkupfalse%
\ {\isachardoublequoteopen}{\isacharparenleft}{\kern0pt}{\isasymUnion}j{\isachardot}{\kern0pt}\ snd\ {\isacharbackquote}{\kern0pt}\ {\isacharparenleft}{\kern0pt}a\ j\ {\isasyminter}\ {\isacharparenleft}{\kern0pt}{\isacharbraceleft}{\kern0pt}Suc\ i{\isacharbraceright}{\kern0pt}\ {\isasymtimes}\ space\ M{\isacharparenright}{\kern0pt}{\isacharparenright}{\kern0pt}{\isacharparenright}{\kern0pt}\ {\isasymin}\ F\ i{\isachardoublequoteclose}\ \isacommand{using}\isamarkupfalse%
\ {\isacharasterisk}{\kern0pt}\ \isacommand{by}\isamarkupfalse%
\ fast\isanewline
\ \ \ \ \isacommand{moreover}\isamarkupfalse%
\ \isacommand{have}\isamarkupfalse%
\ {\isachardoublequoteopen}{\isacharparenleft}{\kern0pt}{\isasymUnion}j{\isachardot}{\kern0pt}\ snd\ {\isacharbackquote}{\kern0pt}\ {\isacharparenleft}{\kern0pt}a\ j\ {\isasyminter}\ {\isacharparenleft}{\kern0pt}{\isacharbraceleft}{\kern0pt}Suc\ i{\isacharbraceright}{\kern0pt}\ {\isasymtimes}\ space\ M{\isacharparenright}{\kern0pt}{\isacharparenright}{\kern0pt}{\isacharparenright}{\kern0pt}\ {\isacharequal}{\kern0pt}\ snd\ {\isacharbackquote}{\kern0pt}\ {\isacharparenleft}{\kern0pt}{\isasymUnion}\ {\isacharparenleft}{\kern0pt}range\ a{\isacharparenright}{\kern0pt}\ {\isasyminter}\ {\isacharparenleft}{\kern0pt}{\isacharbraceleft}{\kern0pt}Suc\ i{\isacharbraceright}{\kern0pt}\ {\isasymtimes}\ space\ M{\isacharparenright}{\kern0pt}{\isacharparenright}{\kern0pt}{\isachardoublequoteclose}\ \isacommand{by}\isamarkupfalse%
\ fast\isanewline
\ \ \ \ \isacommand{ultimately}\isamarkupfalse%
\ \isacommand{show}\isamarkupfalse%
\ {\isacharquery}{\kern0pt}case\ \isacommand{using}\isamarkupfalse%
\ A{\isacharunderscore}{\kern0pt}i\ \isacommand{by}\isamarkupfalse%
\ metis\isanewline
\ \ \isacommand{next}\isamarkupfalse%
\isanewline
\ \ \ \ \isacommand{case}\isamarkupfalse%
\ {\isadigit{2}}\isanewline
\ \ \ \ \isacommand{have}\isamarkupfalse%
\ {\isachardoublequoteopen}{\isacharparenleft}{\kern0pt}{\isasymUnion}j{\isachardot}{\kern0pt}\ snd\ {\isacharbackquote}{\kern0pt}\ {\isacharparenleft}{\kern0pt}a\ j\ {\isasyminter}\ {\isacharparenleft}{\kern0pt}{\isacharbraceleft}{\kern0pt}{\isadigit{0}}{\isacharbraceright}{\kern0pt}\ {\isasymtimes}\ space\ M{\isacharparenright}{\kern0pt}{\isacharparenright}{\kern0pt}{\isacharparenright}{\kern0pt}\ {\isasymin}\ F\ {\isadigit{0}}{\isachardoublequoteclose}\ \isacommand{using}\isamarkupfalse%
\ {\isacharasterisk}{\kern0pt}\ \isacommand{by}\isamarkupfalse%
\ fast\isanewline
\ \ \ \ \isacommand{moreover}\isamarkupfalse%
\ \isacommand{have}\isamarkupfalse%
\ {\isachardoublequoteopen}{\isacharparenleft}{\kern0pt}{\isasymUnion}j{\isachardot}{\kern0pt}\ snd\ {\isacharbackquote}{\kern0pt}\ {\isacharparenleft}{\kern0pt}a\ j\ {\isasyminter}\ {\isacharparenleft}{\kern0pt}{\isacharbraceleft}{\kern0pt}{\isadigit{0}}{\isacharbraceright}{\kern0pt}\ {\isasymtimes}\ space\ M{\isacharparenright}{\kern0pt}{\isacharparenright}{\kern0pt}{\isacharparenright}{\kern0pt}\ {\isacharequal}{\kern0pt}\ snd\ {\isacharbackquote}{\kern0pt}\ {\isacharparenleft}{\kern0pt}{\isasymUnion}\ {\isacharparenleft}{\kern0pt}range\ a{\isacharparenright}{\kern0pt}\ {\isasyminter}\ {\isacharparenleft}{\kern0pt}{\isacharbraceleft}{\kern0pt}{\isadigit{0}}{\isacharbraceright}{\kern0pt}\ {\isasymtimes}\ space\ M{\isacharparenright}{\kern0pt}{\isacharparenright}{\kern0pt}{\isachardoublequoteclose}\ \isacommand{by}\isamarkupfalse%
\ fast\isanewline
\ \ \ \ \isacommand{ultimately}\isamarkupfalse%
\ \isacommand{show}\isamarkupfalse%
\ {\isacharquery}{\kern0pt}case\ \isacommand{using}\isamarkupfalse%
\ A{\isacharunderscore}{\kern0pt}i\ \isacommand{by}\isamarkupfalse%
\ metis\isanewline
\ \ \isacommand{{\isacharbraceright}{\kern0pt}}\isamarkupfalse%
\isanewline
\isacommand{qed}\isamarkupfalse%
%
\endisatagproof
{\isafoldproof}%
%
\isadelimproof
\isanewline
%
\endisadelimproof
\isanewline
\isacommand{lemma}\isamarkupfalse%
\ {\isacharparenleft}{\kern0pt}\isakeyword{in}\ nat{\isacharunderscore}{\kern0pt}filtered{\isacharunderscore}{\kern0pt}sigma{\isacharunderscore}{\kern0pt}finite{\isacharunderscore}{\kern0pt}measure{\isacharparenright}{\kern0pt}\ predictable{\isacharunderscore}{\kern0pt}discrete{\isacharunderscore}{\kern0pt}time{\isacharunderscore}{\kern0pt}process{\isacharunderscore}{\kern0pt}measurable{\isacharcolon}{\kern0pt}\isanewline
\ \ \isakeyword{assumes}\ {\isachardoublequoteopen}predictable\ X{\isachardoublequoteclose}\isanewline
\ \ \isakeyword{shows}\ {\isachardoublequoteopen}X\ i\ {\isasymin}\ borel{\isacharunderscore}{\kern0pt}measurable\ {\isacharparenleft}{\kern0pt}F\ {\isacharparenleft}{\kern0pt}i\ {\isacharminus}{\kern0pt}\ {\isadigit{1}}{\isacharparenright}{\kern0pt}{\isacharparenright}{\kern0pt}{\isachardoublequoteclose}\isanewline
%
\isadelimproof
%
\endisadelimproof
%
\isatagproof
\isacommand{proof}\isamarkupfalse%
\ {\isacharparenleft}{\kern0pt}cases\ i{\isacharparenright}{\kern0pt}\isanewline
\ \ \isacommand{case}\isamarkupfalse%
\ {\isadigit{0}}\isanewline
\ \ \isacommand{{\isacharbraceleft}{\kern0pt}}\isamarkupfalse%
\isanewline
\ \ \ \ \isacommand{fix}\isamarkupfalse%
\ S\ {\isacharcolon}{\kern0pt}{\isacharcolon}{\kern0pt}\ {\isachardoublequoteopen}{\isacharprime}{\kern0pt}b\ set{\isachardoublequoteclose}\ \isacommand{assume}\isamarkupfalse%
\ open{\isacharunderscore}{\kern0pt}S{\isacharcolon}{\kern0pt}\ {\isachardoublequoteopen}open\ S{\isachardoublequoteclose}\isanewline
\ \ \ \ \isacommand{hence}\isamarkupfalse%
\ {\isachardoublequoteopen}{\isacharbraceleft}{\kern0pt}{\isadigit{0}}{\isacharbraceright}{\kern0pt}\ {\isasymtimes}\ space\ M\ {\isasymin}\ predictable{\isacharunderscore}{\kern0pt}sigma{\isachardoublequoteclose}\ \isacommand{by}\isamarkupfalse%
\ {\isacharparenleft}{\kern0pt}auto\ simp\ add{\isacharcolon}{\kern0pt}\ bot{\isacharunderscore}{\kern0pt}nat{\isacharunderscore}{\kern0pt}def\ space{\isacharunderscore}{\kern0pt}F{\isacharbrackleft}{\kern0pt}symmetric{\isacharcomma}{\kern0pt}\ of\ bot{\isacharbrackright}{\kern0pt}{\isacharparenright}{\kern0pt}\isanewline
\ \ \ \ \isacommand{moreover}\isamarkupfalse%
\ \isacommand{have}\isamarkupfalse%
\ {\isachardoublequoteopen}case{\isacharunderscore}{\kern0pt}prod\ X\ {\isacharminus}{\kern0pt}{\isacharbackquote}{\kern0pt}\ S\ {\isasyminter}\ {\isacharparenleft}{\kern0pt}UNIV\ {\isasymtimes}\ space\ M{\isacharparenright}{\kern0pt}\ {\isasymin}\ predictable{\isacharunderscore}{\kern0pt}sigma{\isachardoublequoteclose}\ \isacommand{using}\isamarkupfalse%
\ open{\isacharunderscore}{\kern0pt}S\ \isacommand{by}\isamarkupfalse%
\ {\isacharparenleft}{\kern0pt}intro\ predictableD{\isacharbrackleft}{\kern0pt}OF\ assms{\isacharbrackright}{\kern0pt}{\isacharcomma}{\kern0pt}\ simp\ add{\isacharcolon}{\kern0pt}\ borel{\isacharunderscore}{\kern0pt}open{\isacharparenright}{\kern0pt}\ \ \isanewline
\ \ \ \ \isacommand{ultimately}\isamarkupfalse%
\ \isacommand{have}\isamarkupfalse%
\ {\isachardoublequoteopen}case{\isacharunderscore}{\kern0pt}prod\ X\ {\isacharminus}{\kern0pt}{\isacharbackquote}{\kern0pt}\ S\ {\isasyminter}\ {\isacharparenleft}{\kern0pt}{\isacharbraceleft}{\kern0pt}{\isadigit{0}}{\isacharbraceright}{\kern0pt}\ {\isasymtimes}\ space\ M{\isacharparenright}{\kern0pt}\ {\isasymin}\ predictable{\isacharunderscore}{\kern0pt}sigma{\isachardoublequoteclose}\ \isacommand{unfolding}\isamarkupfalse%
\ sets{\isacharunderscore}{\kern0pt}predictable{\isacharunderscore}{\kern0pt}sigma\ \isacommand{using}\isamarkupfalse%
\ space{\isacharunderscore}{\kern0pt}F\ sets{\isachardot}{\kern0pt}sets{\isacharunderscore}{\kern0pt}into{\isacharunderscore}{\kern0pt}space\isanewline
\ \ \ \ \ \ \isacommand{by}\isamarkupfalse%
\ {\isacharparenleft}{\kern0pt}subst\ Times{\isacharunderscore}{\kern0pt}Int{\isacharunderscore}{\kern0pt}distrib{\isadigit{1}}{\isacharbrackleft}{\kern0pt}of\ {\isachardoublequoteopen}{\isacharbraceleft}{\kern0pt}{\isadigit{0}}{\isacharbraceright}{\kern0pt}{\isachardoublequoteclose}\ UNIV\ {\isachardoublequoteopen}space\ M{\isachardoublequoteclose}{\isacharcomma}{\kern0pt}\ simplified{\isacharbrackright}{\kern0pt}{\isacharcomma}{\kern0pt}\ subst\ inf{\isachardot}{\kern0pt}commute{\isacharbrackleft}{\kern0pt}of\ {\isachardoublequoteopen}{\isacharunderscore}{\kern0pt}\ {\isasymtimes}\ {\isacharunderscore}{\kern0pt}{\isachardoublequoteclose}{\isacharbrackright}{\kern0pt}{\isacharcomma}{\kern0pt}\ subst\ Int{\isacharunderscore}{\kern0pt}assoc{\isacharbrackleft}{\kern0pt}symmetric{\isacharbrackright}{\kern0pt}{\isacharcomma}{\kern0pt}\ subst\ Int{\isacharunderscore}{\kern0pt}range{\isacharunderscore}{\kern0pt}binary{\isacharparenright}{\kern0pt}\ \isanewline
\ \ \ \ \ \ \ \ \ {\isacharparenleft}{\kern0pt}intro\ sigma{\isacharunderscore}{\kern0pt}sets{\isacharunderscore}{\kern0pt}Inter\ binary{\isacharunderscore}{\kern0pt}in{\isacharunderscore}{\kern0pt}sigma{\isacharunderscore}{\kern0pt}sets{\isacharcomma}{\kern0pt}\ fast{\isacharparenright}{\kern0pt}{\isacharplus}{\kern0pt}\isanewline
\ \ \ \ \isacommand{moreover}\isamarkupfalse%
\ \isacommand{have}\isamarkupfalse%
\ {\isachardoublequoteopen}case{\isacharunderscore}{\kern0pt}prod\ X\ {\isacharminus}{\kern0pt}{\isacharbackquote}{\kern0pt}\ S\ {\isasyminter}\ {\isacharparenleft}{\kern0pt}{\isacharbraceleft}{\kern0pt}{\isadigit{0}}{\isacharbraceright}{\kern0pt}\ {\isasymtimes}\ space\ M{\isacharparenright}{\kern0pt}\ {\isacharequal}{\kern0pt}\ {\isacharbraceleft}{\kern0pt}{\isadigit{0}}{\isacharbraceright}{\kern0pt}\ {\isasymtimes}\ {\isacharparenleft}{\kern0pt}X\ {\isadigit{0}}\ {\isacharminus}{\kern0pt}{\isacharbackquote}{\kern0pt}\ S\ {\isasyminter}\ space\ M{\isacharparenright}{\kern0pt}{\isachardoublequoteclose}\ \isacommand{by}\isamarkupfalse%
\ {\isacharparenleft}{\kern0pt}auto\ simp\ add{\isacharcolon}{\kern0pt}\ le{\isacharunderscore}{\kern0pt}Suc{\isacharunderscore}{\kern0pt}eq{\isacharparenright}{\kern0pt}\isanewline
\ \ \ \ \isacommand{moreover}\isamarkupfalse%
\ \isacommand{have}\isamarkupfalse%
\ {\isachardoublequoteopen}{\isachardot}{\kern0pt}{\isachardot}{\kern0pt}{\isachardot}{\kern0pt}\ {\isacharequal}{\kern0pt}\ {\isacharparenleft}{\kern0pt}{\isasymUnion}i{\isachardot}{\kern0pt}\ {\isacharbraceleft}{\kern0pt}i{\isacharbraceright}{\kern0pt}\ {\isasymtimes}\ {\isacharparenleft}{\kern0pt}if\ i\ {\isacharequal}{\kern0pt}\ {\isadigit{0}}\ then\ X\ {\isadigit{0}}\ {\isacharminus}{\kern0pt}{\isacharbackquote}{\kern0pt}\ S\ {\isasyminter}\ space\ M\ else\ {\isacharbraceleft}{\kern0pt}{\isacharbraceright}{\kern0pt}{\isacharparenright}{\kern0pt}{\isacharparenright}{\kern0pt}{\isachardoublequoteclose}\ \isacommand{by}\isamarkupfalse%
\ {\isacharparenleft}{\kern0pt}auto\ split{\isacharcolon}{\kern0pt}\ if{\isacharunderscore}{\kern0pt}splits{\isacharparenright}{\kern0pt}\isanewline
\ \ \ \ \isacommand{ultimately}\isamarkupfalse%
\ \isacommand{have}\isamarkupfalse%
\ {\isachardoublequoteopen}{\isacharparenleft}{\kern0pt}{\isasymUnion}i{\isachardot}{\kern0pt}\ {\isacharbraceleft}{\kern0pt}i{\isacharbraceright}{\kern0pt}\ {\isasymtimes}\ {\isacharparenleft}{\kern0pt}if\ i\ {\isacharequal}{\kern0pt}\ {\isadigit{0}}\ then\ X\ {\isadigit{0}}\ {\isacharminus}{\kern0pt}{\isacharbackquote}{\kern0pt}\ S\ {\isasyminter}\ space\ M\ else\ {\isacharbraceleft}{\kern0pt}{\isacharbraceright}{\kern0pt}{\isacharparenright}{\kern0pt}{\isacharparenright}{\kern0pt}\ {\isasymin}\ predictable{\isacharunderscore}{\kern0pt}sigma{\isachardoublequoteclose}\ \isacommand{by}\isamarkupfalse%
\ argo\isanewline
\ \ \ \ \isacommand{then}\isamarkupfalse%
\ \isacommand{have}\isamarkupfalse%
\ {\isachardoublequoteopen}X\ {\isadigit{0}}\ {\isacharminus}{\kern0pt}{\isacharbackquote}{\kern0pt}\ S\ {\isasyminter}\ space\ M\ {\isasymin}\ sets\ {\isacharparenleft}{\kern0pt}F\ {\isadigit{0}}{\isacharparenright}{\kern0pt}{\isachardoublequoteclose}\ \isacommand{using}\isamarkupfalse%
\ predictable{\isacharunderscore}{\kern0pt}sets{\isacharunderscore}{\kern0pt}in{\isacharunderscore}{\kern0pt}F{\isacharbrackleft}{\kern0pt}of\ {\isachardoublequoteopen}{\isasymlambda}i{\isachardot}{\kern0pt}\ if\ i\ {\isacharequal}{\kern0pt}\ {\isadigit{0}}\ then\ X\ {\isadigit{0}}\ {\isacharminus}{\kern0pt}{\isacharbackquote}{\kern0pt}\ S\ {\isasyminter}\ space\ M\ else\ {\isacharbraceleft}{\kern0pt}{\isacharbraceright}{\kern0pt}{\isachardoublequoteclose}{\isacharbrackright}{\kern0pt}\ \isacommand{by}\isamarkupfalse%
\ presburger\isanewline
\ \ \isacommand{{\isacharbraceright}{\kern0pt}}\isamarkupfalse%
\isanewline
\ \ \isacommand{hence}\isamarkupfalse%
\ {\isachardoublequoteopen}X\ {\isadigit{0}}\ {\isasymin}\ borel{\isacharunderscore}{\kern0pt}measurable\ {\isacharparenleft}{\kern0pt}F\ {\isadigit{0}}{\isacharparenright}{\kern0pt}{\isachardoublequoteclose}\ \isacommand{by}\isamarkupfalse%
\ {\isacharparenleft}{\kern0pt}fastforce\ simp\ add{\isacharcolon}{\kern0pt}\ bot{\isacharunderscore}{\kern0pt}nat{\isacharunderscore}{\kern0pt}def\ space{\isacharunderscore}{\kern0pt}F\ intro{\isacharbang}{\kern0pt}{\isacharcolon}{\kern0pt}\ borel{\isacharunderscore}{\kern0pt}measurableI{\isacharparenright}{\kern0pt}\isanewline
\ \ \isacommand{thus}\isamarkupfalse%
\ {\isacharquery}{\kern0pt}thesis\ \isacommand{using}\isamarkupfalse%
\ {\isadigit{0}}\ \isacommand{by}\isamarkupfalse%
\ force\isanewline
\isacommand{next}\isamarkupfalse%
\isanewline
\ \ \isacommand{case}\isamarkupfalse%
\ {\isacharparenleft}{\kern0pt}Suc\ i{\isacharparenright}{\kern0pt}\isanewline
\ \ \isacommand{{\isacharbraceleft}{\kern0pt}}\isamarkupfalse%
\isanewline
\ \ \ \ \isacommand{fix}\isamarkupfalse%
\ S\ {\isacharcolon}{\kern0pt}{\isacharcolon}{\kern0pt}\ {\isachardoublequoteopen}{\isacharprime}{\kern0pt}b\ set{\isachardoublequoteclose}\ \isacommand{assume}\isamarkupfalse%
\ open{\isacharunderscore}{\kern0pt}S{\isacharcolon}{\kern0pt}\ {\isachardoublequoteopen}open\ S{\isachardoublequoteclose}\isanewline
\ \ \ \ \isacommand{have}\isamarkupfalse%
\ {\isachardoublequoteopen}{\isacharbraceleft}{\kern0pt}Suc\ i{\isacharbraceright}{\kern0pt}\ {\isacharequal}{\kern0pt}\ {\isacharbraceleft}{\kern0pt}i{\isacharless}{\kern0pt}{\isachardot}{\kern0pt}{\isachardot}{\kern0pt}Suc\ i{\isacharbraceright}{\kern0pt}{\isachardoublequoteclose}\ \isacommand{by}\isamarkupfalse%
\ fastforce\isanewline
\ \ \ \ \isacommand{hence}\isamarkupfalse%
\ {\isachardoublequoteopen}{\isacharbraceleft}{\kern0pt}Suc\ i{\isacharbraceright}{\kern0pt}\ {\isasymtimes}\ space\ M\ {\isasymin}\ predictable{\isacharunderscore}{\kern0pt}sigma{\isachardoublequoteclose}\ \isacommand{unfolding}\isamarkupfalse%
\ space{\isacharunderscore}{\kern0pt}F{\isacharbrackleft}{\kern0pt}symmetric{\isacharcomma}{\kern0pt}\ of\ i{\isacharbrackright}{\kern0pt}\ \isacommand{by}\isamarkupfalse%
\ {\isacharparenleft}{\kern0pt}auto\ intro{\isacharbang}{\kern0pt}{\isacharcolon}{\kern0pt}\ sigma{\isacharunderscore}{\kern0pt}sets{\isachardot}{\kern0pt}Basic{\isacharparenright}{\kern0pt}\isanewline
\ \ \ \ \isacommand{moreover}\isamarkupfalse%
\ \isacommand{have}\isamarkupfalse%
\ {\isachardoublequoteopen}case{\isacharunderscore}{\kern0pt}prod\ X\ {\isacharminus}{\kern0pt}{\isacharbackquote}{\kern0pt}\ S\ {\isasyminter}\ {\isacharparenleft}{\kern0pt}UNIV\ {\isasymtimes}\ space\ M{\isacharparenright}{\kern0pt}\ {\isasymin}\ predictable{\isacharunderscore}{\kern0pt}sigma{\isachardoublequoteclose}\ \isacommand{using}\isamarkupfalse%
\ open{\isacharunderscore}{\kern0pt}S\ \isacommand{by}\isamarkupfalse%
\ {\isacharparenleft}{\kern0pt}intro\ predictableD{\isacharbrackleft}{\kern0pt}OF\ assms{\isacharbrackright}{\kern0pt}{\isacharcomma}{\kern0pt}\ simp\ add{\isacharcolon}{\kern0pt}\ borel{\isacharunderscore}{\kern0pt}open{\isacharparenright}{\kern0pt}\isanewline
\ \ \ \ \isacommand{ultimately}\isamarkupfalse%
\ \isacommand{have}\isamarkupfalse%
\ {\isachardoublequoteopen}case{\isacharunderscore}{\kern0pt}prod\ X\ {\isacharminus}{\kern0pt}{\isacharbackquote}{\kern0pt}\ S\ {\isasyminter}\ {\isacharparenleft}{\kern0pt}{\isacharbraceleft}{\kern0pt}Suc\ i{\isacharbraceright}{\kern0pt}\ {\isasymtimes}\ space\ M{\isacharparenright}{\kern0pt}\ {\isasymin}\ predictable{\isacharunderscore}{\kern0pt}sigma{\isachardoublequoteclose}\ \isacommand{unfolding}\isamarkupfalse%
\ sets{\isacharunderscore}{\kern0pt}predictable{\isacharunderscore}{\kern0pt}sigma\ \isacommand{using}\isamarkupfalse%
\ space{\isacharunderscore}{\kern0pt}F\ sets{\isachardot}{\kern0pt}sets{\isacharunderscore}{\kern0pt}into{\isacharunderscore}{\kern0pt}space\isanewline
\ \ \ \ \ \ \isacommand{by}\isamarkupfalse%
\ {\isacharparenleft}{\kern0pt}subst\ Times{\isacharunderscore}{\kern0pt}Int{\isacharunderscore}{\kern0pt}distrib{\isadigit{1}}{\isacharbrackleft}{\kern0pt}of\ {\isachardoublequoteopen}{\isacharbraceleft}{\kern0pt}Suc\ i{\isacharbraceright}{\kern0pt}{\isachardoublequoteclose}\ UNIV\ {\isachardoublequoteopen}space\ M{\isachardoublequoteclose}{\isacharcomma}{\kern0pt}\ simplified{\isacharbrackright}{\kern0pt}{\isacharcomma}{\kern0pt}\ subst\ inf{\isachardot}{\kern0pt}commute{\isacharbrackleft}{\kern0pt}of\ {\isachardoublequoteopen}{\isacharunderscore}{\kern0pt}\ {\isasymtimes}\ {\isacharunderscore}{\kern0pt}{\isachardoublequoteclose}{\isacharbrackright}{\kern0pt}{\isacharcomma}{\kern0pt}\ subst\ Int{\isacharunderscore}{\kern0pt}assoc{\isacharbrackleft}{\kern0pt}symmetric{\isacharbrackright}{\kern0pt}{\isacharcomma}{\kern0pt}\ subst\ Int{\isacharunderscore}{\kern0pt}range{\isacharunderscore}{\kern0pt}binary{\isacharparenright}{\kern0pt}\ \isanewline
\ \ \ \ \ \ \ \ \ {\isacharparenleft}{\kern0pt}intro\ sigma{\isacharunderscore}{\kern0pt}sets{\isacharunderscore}{\kern0pt}Inter\ binary{\isacharunderscore}{\kern0pt}in{\isacharunderscore}{\kern0pt}sigma{\isacharunderscore}{\kern0pt}sets{\isacharcomma}{\kern0pt}\ fast{\isacharparenright}{\kern0pt}{\isacharplus}{\kern0pt}\isanewline
\ \ \ \ \isacommand{moreover}\isamarkupfalse%
\ \isacommand{have}\isamarkupfalse%
\ {\isachardoublequoteopen}case{\isacharunderscore}{\kern0pt}prod\ X\ {\isacharminus}{\kern0pt}{\isacharbackquote}{\kern0pt}\ S\ {\isasyminter}\ {\isacharparenleft}{\kern0pt}{\isacharbraceleft}{\kern0pt}Suc\ i{\isacharbraceright}{\kern0pt}\ {\isasymtimes}\ space\ M{\isacharparenright}{\kern0pt}\ {\isacharequal}{\kern0pt}\ {\isacharbraceleft}{\kern0pt}Suc\ i{\isacharbraceright}{\kern0pt}\ {\isasymtimes}\ {\isacharparenleft}{\kern0pt}X\ {\isacharparenleft}{\kern0pt}Suc\ i{\isacharparenright}{\kern0pt}\ {\isacharminus}{\kern0pt}{\isacharbackquote}{\kern0pt}\ S\ {\isasyminter}\ space\ M{\isacharparenright}{\kern0pt}{\isachardoublequoteclose}\ \isacommand{by}\isamarkupfalse%
\ {\isacharparenleft}{\kern0pt}auto\ simp\ add{\isacharcolon}{\kern0pt}\ le{\isacharunderscore}{\kern0pt}Suc{\isacharunderscore}{\kern0pt}eq{\isacharparenright}{\kern0pt}\isanewline
\ \ \ \ \isacommand{moreover}\isamarkupfalse%
\ \isacommand{have}\isamarkupfalse%
\ {\isachardoublequoteopen}{\isachardot}{\kern0pt}{\isachardot}{\kern0pt}{\isachardot}{\kern0pt}\ {\isacharequal}{\kern0pt}\ {\isacharparenleft}{\kern0pt}{\isasymUnion}j{\isachardot}{\kern0pt}\ {\isacharbraceleft}{\kern0pt}j{\isacharbraceright}{\kern0pt}\ {\isasymtimes}\ {\isacharparenleft}{\kern0pt}if\ j\ {\isacharequal}{\kern0pt}\ Suc\ i\ then\ {\isacharparenleft}{\kern0pt}X\ {\isacharparenleft}{\kern0pt}Suc\ i{\isacharparenright}{\kern0pt}\ {\isacharminus}{\kern0pt}{\isacharbackquote}{\kern0pt}\ S\ {\isasyminter}\ space\ M{\isacharparenright}{\kern0pt}\ else\ {\isacharbraceleft}{\kern0pt}{\isacharbraceright}{\kern0pt}{\isacharparenright}{\kern0pt}{\isacharparenright}{\kern0pt}{\isachardoublequoteclose}\ \isacommand{by}\isamarkupfalse%
\ {\isacharparenleft}{\kern0pt}auto\ split{\isacharcolon}{\kern0pt}\ if{\isacharunderscore}{\kern0pt}splits{\isacharparenright}{\kern0pt}\isanewline
\ \ \ \ \isacommand{ultimately}\isamarkupfalse%
\ \isacommand{have}\isamarkupfalse%
\ {\isachardoublequoteopen}{\isacharparenleft}{\kern0pt}{\isasymUnion}j{\isachardot}{\kern0pt}\ {\isacharbraceleft}{\kern0pt}j{\isacharbraceright}{\kern0pt}\ {\isasymtimes}\ {\isacharparenleft}{\kern0pt}if\ j\ {\isacharequal}{\kern0pt}\ Suc\ i\ then\ {\isacharparenleft}{\kern0pt}X\ {\isacharparenleft}{\kern0pt}Suc\ i{\isacharparenright}{\kern0pt}\ {\isacharminus}{\kern0pt}{\isacharbackquote}{\kern0pt}\ S\ {\isasyminter}\ space\ M{\isacharparenright}{\kern0pt}\ else\ {\isacharbraceleft}{\kern0pt}{\isacharbraceright}{\kern0pt}{\isacharparenright}{\kern0pt}{\isacharparenright}{\kern0pt}\ {\isasymin}\ predictable{\isacharunderscore}{\kern0pt}sigma{\isachardoublequoteclose}\ \isacommand{by}\isamarkupfalse%
\ argo\isanewline
\ \ \ \ \isacommand{then}\isamarkupfalse%
\ \isacommand{have}\isamarkupfalse%
\ {\isachardoublequoteopen}X\ {\isacharparenleft}{\kern0pt}Suc\ i{\isacharparenright}{\kern0pt}\ {\isacharminus}{\kern0pt}{\isacharbackquote}{\kern0pt}\ S\ {\isasyminter}\ space\ M\ {\isasymin}\ sets\ {\isacharparenleft}{\kern0pt}F\ i{\isacharparenright}{\kern0pt}{\isachardoublequoteclose}\ \isacommand{using}\isamarkupfalse%
\ predictable{\isacharunderscore}{\kern0pt}sets{\isacharunderscore}{\kern0pt}in{\isacharunderscore}{\kern0pt}F{\isacharbrackleft}{\kern0pt}of\ {\isachardoublequoteopen}{\isasymlambda}j{\isachardot}{\kern0pt}\ if\ j\ {\isacharequal}{\kern0pt}\ Suc\ i\ then\ {\isacharparenleft}{\kern0pt}X\ {\isacharparenleft}{\kern0pt}Suc\ i{\isacharparenright}{\kern0pt}\ {\isacharminus}{\kern0pt}{\isacharbackquote}{\kern0pt}\ S\ {\isasyminter}\ space\ M{\isacharparenright}{\kern0pt}\ else\ {\isacharbraceleft}{\kern0pt}{\isacharbraceright}{\kern0pt}{\isachardoublequoteclose}{\isacharbrackright}{\kern0pt}\ \isacommand{by}\isamarkupfalse%
\ presburger\isanewline
\ \ \isacommand{{\isacharbraceright}{\kern0pt}}\isamarkupfalse%
\isanewline
\ \ \isacommand{hence}\isamarkupfalse%
\ {\isachardoublequoteopen}X\ {\isacharparenleft}{\kern0pt}Suc\ i{\isacharparenright}{\kern0pt}\ {\isasymin}\ borel{\isacharunderscore}{\kern0pt}measurable\ {\isacharparenleft}{\kern0pt}F\ i{\isacharparenright}{\kern0pt}{\isachardoublequoteclose}\ \isacommand{by}\isamarkupfalse%
\ {\isacharparenleft}{\kern0pt}fastforce\ simp\ add{\isacharcolon}{\kern0pt}\ space{\isacharunderscore}{\kern0pt}F\ intro{\isacharbang}{\kern0pt}{\isacharcolon}{\kern0pt}\ borel{\isacharunderscore}{\kern0pt}measurableI{\isacharparenright}{\kern0pt}\isanewline
\ \ \isacommand{then}\isamarkupfalse%
\ \isacommand{show}\isamarkupfalse%
\ {\isacharquery}{\kern0pt}thesis\ \isacommand{using}\isamarkupfalse%
\ Suc\ \isacommand{by}\isamarkupfalse%
\ force\isanewline
\isacommand{qed}\isamarkupfalse%
%
\endisatagproof
{\isafoldproof}%
%
\isadelimproof
\isanewline
%
\endisadelimproof
\isanewline
\isacommand{end}\isamarkupfalse%
\isanewline
%
\isadelimtheory
\isanewline
%
\endisadelimtheory
%
\isatagtheory
\isacommand{end}\isamarkupfalse%
%
\endisatagtheory
{\isafoldtheory}%
%
\isadelimtheory
%
\endisadelimtheory
%
\end{isabellebody}%
\endinput
%:%file=Stochastic_Process.tex%:%
%:%10=1%:%
%:%11=1%:%
%:%12=2%:%
%:%13=3%:%
%:%27=5%:%
%:%37=7%:%
%:%38=7%:%
%:%39=8%:%
%:%40=9%:%
%:%41=10%:%
%:%42=11%:%
%:%43=12%:%
%:%44=12%:%
%:%45=13%:%
%:%46=13%:%
%:%47=14%:%
%:%48=15%:%
%:%49=15%:%
%:%50=16%:%
%:%51=17%:%
%:%54=18%:%
%:%58=18%:%
%:%59=18%:%
%:%64=18%:%
%:%67=19%:%
%:%68=20%:%
%:%69=20%:%
%:%71=20%:%
%:%75=20%:%
%:%76=20%:%
%:%83=20%:%
%:%84=21%:%
%:%85=22%:%
%:%86=22%:%
%:%87=23%:%
%:%88=24%:%
%:%91=25%:%
%:%95=25%:%
%:%96=25%:%
%:%101=25%:%
%:%104=26%:%
%:%105=27%:%
%:%106=27%:%
%:%107=28%:%
%:%108=29%:%
%:%111=30%:%
%:%115=30%:%
%:%116=30%:%
%:%121=30%:%
%:%124=31%:%
%:%125=32%:%
%:%126=32%:%
%:%128=32%:%
%:%132=32%:%
%:%133=32%:%
%:%140=32%:%
%:%141=33%:%
%:%142=34%:%
%:%143=34%:%
%:%144=35%:%
%:%145=36%:%
%:%148=37%:%
%:%152=37%:%
%:%153=37%:%
%:%158=37%:%
%:%161=38%:%
%:%162=39%:%
%:%163=39%:%
%:%164=40%:%
%:%165=41%:%
%:%168=42%:%
%:%172=42%:%
%:%173=42%:%
%:%178=42%:%
%:%181=43%:%
%:%182=44%:%
%:%183=44%:%
%:%185=44%:%
%:%189=44%:%
%:%190=44%:%
%:%191=44%:%
%:%198=44%:%
%:%199=45%:%
%:%200=46%:%
%:%208=48%:%
%:%218=50%:%
%:%219=50%:%
%:%220=51%:%
%:%221=52%:%
%:%222=53%:%
%:%223=54%:%
%:%224=54%:%
%:%225=55%:%
%:%226=56%:%
%:%229=57%:%
%:%233=57%:%
%:%234=57%:%
%:%235=57%:%
%:%240=57%:%
%:%243=58%:%
%:%244=59%:%
%:%245=59%:%
%:%246=60%:%
%:%247=61%:%
%:%250=62%:%
%:%254=62%:%
%:%255=62%:%
%:%260=62%:%
%:%263=63%:%
%:%264=64%:%
%:%265=64%:%
%:%267=64%:%
%:%271=64%:%
%:%272=64%:%
%:%279=64%:%
%:%280=65%:%
%:%281=66%:%
%:%282=66%:%
%:%283=67%:%
%:%284=68%:%
%:%291=69%:%
%:%292=69%:%
%:%293=70%:%
%:%294=70%:%
%:%295=70%:%
%:%296=71%:%
%:%297=71%:%
%:%298=71%:%
%:%299=72%:%
%:%305=72%:%
%:%308=73%:%
%:%309=74%:%
%:%310=74%:%
%:%311=75%:%
%:%312=76%:%
%:%315=77%:%
%:%319=77%:%
%:%320=77%:%
%:%321=77%:%
%:%326=77%:%
%:%329=78%:%
%:%330=79%:%
%:%331=79%:%
%:%333=79%:%
%:%337=79%:%
%:%338=79%:%
%:%345=79%:%
%:%346=80%:%
%:%347=81%:%
%:%348=81%:%
%:%349=82%:%
%:%350=83%:%
%:%357=84%:%
%:%358=84%:%
%:%359=85%:%
%:%360=85%:%
%:%361=85%:%
%:%362=86%:%
%:%363=86%:%
%:%364=86%:%
%:%365=87%:%
%:%371=87%:%
%:%374=88%:%
%:%375=89%:%
%:%376=89%:%
%:%377=90%:%
%:%378=91%:%
%:%385=92%:%
%:%386=92%:%
%:%387=93%:%
%:%388=93%:%
%:%389=93%:%
%:%390=94%:%
%:%391=94%:%
%:%392=94%:%
%:%393=95%:%
%:%399=95%:%
%:%402=96%:%
%:%403=97%:%
%:%404=97%:%
%:%406=97%:%
%:%410=97%:%
%:%411=97%:%
%:%412=97%:%
%:%419=97%:%
%:%420=98%:%
%:%421=99%:%
%:%422=99%:%
%:%423=100%:%
%:%424=101%:%
%:%425=101%:%
%:%432=103%:%
%:%442=105%:%
%:%443=105%:%
%:%444=106%:%
%:%445=106%:%
%:%446=107%:%
%:%447=107%:%
%:%448=108%:%
%:%449=109%:%
%:%450=109%:%
%:%452=109%:%
%:%456=109%:%
%:%457=109%:%
%:%464=109%:%
%:%465=110%:%
%:%466=110%:%
%:%468=110%:%
%:%472=110%:%
%:%473=110%:%
%:%480=110%:%
%:%481=111%:%
%:%482=111%:%
%:%484=111%:%
%:%488=111%:%
%:%489=111%:%
%:%503=113%:%
%:%513=115%:%
%:%514=115%:%
%:%515=116%:%
%:%516=117%:%
%:%517=118%:%
%:%518=118%:%
%:%519=119%:%
%:%520=120%:%
%:%521=121%:%
%:%522=121%:%
%:%524=121%:%
%:%528=121%:%
%:%529=121%:%
%:%530=121%:%
%:%537=121%:%
%:%538=122%:%
%:%539=123%:%
%:%540=123%:%
%:%543=124%:%
%:%547=124%:%
%:%548=124%:%
%:%549=125%:%
%:%550=125%:%
%:%551=126%:%
%:%552=126%:%
%:%557=126%:%
%:%560=127%:%
%:%561=128%:%
%:%562=128%:%
%:%563=129%:%
%:%564=130%:%
%:%565=131%:%
%:%566=131%:%
%:%567=132%:%
%:%568=133%:%
%:%569=133%:%
%:%570=134%:%
%:%571=135%:%
%:%574=136%:%
%:%578=136%:%
%:%579=136%:%
%:%580=136%:%
%:%581=137%:%
%:%582=137%:%
%:%583=138%:%
%:%584=138%:%
%:%585=139%:%
%:%586=139%:%
%:%587=140%:%
%:%588=140%:%
%:%589=141%:%
%:%590=141%:%
%:%591=141%:%
%:%592=141%:%
%:%593=142%:%
%:%594=142%:%
%:%595=142%:%
%:%596=142%:%
%:%597=143%:%
%:%598=143%:%
%:%599=143%:%
%:%600=143%:%
%:%601=143%:%
%:%602=144%:%
%:%603=144%:%
%:%604=144%:%
%:%605=144%:%
%:%606=144%:%
%:%607=145%:%
%:%608=145%:%
%:%609=145%:%
%:%610=145%:%
%:%611=145%:%
%:%612=146%:%
%:%613=146%:%
%:%614=147%:%
%:%615=147%:%
%:%616=148%:%
%:%617=148%:%
%:%618=149%:%
%:%619=149%:%
%:%620=150%:%
%:%621=150%:%
%:%622=150%:%
%:%623=150%:%
%:%624=150%:%
%:%625=151%:%
%:%626=151%:%
%:%627=151%:%
%:%628=151%:%
%:%629=152%:%
%:%630=152%:%
%:%631=152%:%
%:%632=152%:%
%:%633=152%:%
%:%634=153%:%
%:%635=153%:%
%:%636=153%:%
%:%637=153%:%
%:%638=153%:%
%:%639=153%:%
%:%640=154%:%
%:%641=154%:%
%:%642=155%:%
%:%643=155%:%
%:%644=156%:%
%:%645=156%:%
%:%646=156%:%
%:%647=156%:%
%:%648=157%:%
%:%649=157%:%
%:%650=158%:%
%:%651=158%:%
%:%652=159%:%
%:%653=159%:%
%:%654=160%:%
%:%655=160%:%
%:%656=161%:%
%:%657=161%:%
%:%658=161%:%
%:%659=161%:%
%:%660=161%:%
%:%661=162%:%
%:%662=162%:%
%:%663=163%:%
%:%664=163%:%
%:%665=164%:%
%:%666=164%:%
%:%667=164%:%
%:%668=164%:%
%:%669=164%:%
%:%670=165%:%
%:%671=165%:%
%:%672=166%:%
%:%673=166%:%
%:%674=167%:%
%:%675=167%:%
%:%676=168%:%
%:%677=168%:%
%:%678=168%:%
%:%679=168%:%
%:%680=169%:%
%:%681=169%:%
%:%682=169%:%
%:%683=169%:%
%:%684=170%:%
%:%685=170%:%
%:%686=170%:%
%:%687=170%:%
%:%688=171%:%
%:%689=171%:%
%:%690=171%:%
%:%691=171%:%
%:%692=172%:%
%:%693=172%:%
%:%694=172%:%
%:%695=172%:%
%:%696=172%:%
%:%697=173%:%
%:%698=173%:%
%:%699=174%:%
%:%700=174%:%
%:%701=175%:%
%:%702=175%:%
%:%703=175%:%
%:%704=175%:%
%:%705=175%:%
%:%706=176%:%
%:%707=176%:%
%:%708=177%:%
%:%709=177%:%
%:%710=178%:%
%:%711=178%:%
%:%712=178%:%
%:%713=178%:%
%:%714=178%:%
%:%715=179%:%
%:%716=179%:%
%:%717=180%:%
%:%718=180%:%
%:%719=181%:%
%:%720=181%:%
%:%721=182%:%
%:%722=182%:%
%:%723=182%:%
%:%724=182%:%
%:%725=183%:%
%:%726=183%:%
%:%727=183%:%
%:%728=183%:%
%:%729=184%:%
%:%730=184%:%
%:%731=184%:%
%:%732=184%:%
%:%733=185%:%
%:%734=185%:%
%:%735=185%:%
%:%736=186%:%
%:%737=186%:%
%:%738=186%:%
%:%739=186%:%
%:%740=186%:%
%:%741=187%:%
%:%742=187%:%
%:%743=187%:%
%:%744=187%:%
%:%745=188%:%
%:%746=188%:%
%:%747=189%:%
%:%748=189%:%
%:%749=190%:%
%:%750=190%:%
%:%751=190%:%
%:%752=190%:%
%:%753=191%:%
%:%754=191%:%
%:%755=191%:%
%:%756=191%:%
%:%757=192%:%
%:%758=192%:%
%:%759=192%:%
%:%760=192%:%
%:%761=192%:%
%:%762=193%:%
%:%763=193%:%
%:%764=194%:%
%:%765=194%:%
%:%766=195%:%
%:%767=195%:%
%:%768=195%:%
%:%769=195%:%
%:%770=196%:%
%:%771=196%:%
%:%772=196%:%
%:%773=196%:%
%:%774=197%:%
%:%775=197%:%
%:%776=197%:%
%:%777=197%:%
%:%778=197%:%
%:%779=198%:%
%:%780=198%:%
%:%781=199%:%
%:%787=199%:%
%:%790=200%:%
%:%791=201%:%
%:%792=201%:%
%:%793=202%:%
%:%794=203%:%
%:%801=204%:%
%:%802=204%:%
%:%803=205%:%
%:%804=205%:%
%:%805=206%:%
%:%806=206%:%
%:%807=207%:%
%:%808=207%:%
%:%809=207%:%
%:%810=208%:%
%:%811=208%:%
%:%812=208%:%
%:%813=209%:%
%:%814=209%:%
%:%815=209%:%
%:%816=209%:%
%:%817=209%:%
%:%818=210%:%
%:%819=210%:%
%:%820=210%:%
%:%821=210%:%
%:%822=210%:%
%:%823=211%:%
%:%824=211%:%
%:%825=212%:%
%:%826=213%:%
%:%827=213%:%
%:%828=213%:%
%:%829=213%:%
%:%830=214%:%
%:%831=214%:%
%:%832=214%:%
%:%833=214%:%
%:%834=215%:%
%:%835=215%:%
%:%836=215%:%
%:%837=215%:%
%:%838=216%:%
%:%839=216%:%
%:%840=216%:%
%:%841=216%:%
%:%842=216%:%
%:%843=217%:%
%:%844=217%:%
%:%845=218%:%
%:%846=218%:%
%:%847=218%:%
%:%848=219%:%
%:%849=219%:%
%:%850=219%:%
%:%851=219%:%
%:%852=220%:%
%:%853=220%:%
%:%854=221%:%
%:%855=221%:%
%:%856=222%:%
%:%857=222%:%
%:%858=223%:%
%:%859=223%:%
%:%860=223%:%
%:%861=224%:%
%:%862=224%:%
%:%863=224%:%
%:%864=225%:%
%:%865=225%:%
%:%866=225%:%
%:%867=225%:%
%:%868=226%:%
%:%869=226%:%
%:%870=226%:%
%:%871=226%:%
%:%872=226%:%
%:%873=227%:%
%:%874=227%:%
%:%875=227%:%
%:%876=227%:%
%:%877=227%:%
%:%878=228%:%
%:%879=228%:%
%:%880=229%:%
%:%881=230%:%
%:%882=230%:%
%:%883=230%:%
%:%884=230%:%
%:%885=231%:%
%:%886=231%:%
%:%887=231%:%
%:%888=231%:%
%:%889=232%:%
%:%890=232%:%
%:%891=232%:%
%:%892=232%:%
%:%893=233%:%
%:%894=233%:%
%:%895=233%:%
%:%896=233%:%
%:%897=233%:%
%:%898=234%:%
%:%899=234%:%
%:%900=235%:%
%:%901=235%:%
%:%902=235%:%
%:%903=236%:%
%:%904=236%:%
%:%905=236%:%
%:%906=236%:%
%:%907=236%:%
%:%908=237%:%
%:%914=237%:%
%:%917=238%:%
%:%918=239%:%
%:%919=239%:%
%:%922=240%:%
%:%927=241%:%

%
\begin{isabellebody}%
\setisabellecontext{Martingale}%
%
\isadelimtheory
\isanewline
\isanewline
%
\endisadelimtheory
%
\isatagtheory
\isacommand{theory}\isamarkupfalse%
\ Martingale\isanewline
\ \ \isakeyword{imports}\ Stochastic{\isacharunderscore}{\kern0pt}Process\ Conditional{\isacharunderscore}{\kern0pt}Expectation{\isacharunderscore}{\kern0pt}Banach\isanewline
\isakeyword{begin}%
\endisatagtheory
{\isafoldtheory}%
%
\isadelimtheory
%
\endisadelimtheory
%
\isadelimdocument
%
\endisadelimdocument
%
\isatagdocument
%
\isamarkupsection{Martingales%
}
\isamarkuptrue%
%
\endisatagdocument
{\isafolddocument}%
%
\isadelimdocument
%
\endisadelimdocument
%
\begin{isamarkuptext}%
The following locales are necessary for defining martingales.%
\end{isamarkuptext}\isamarkuptrue%
%
\isadelimdocument
%
\endisadelimdocument
%
\isatagdocument
%
\isamarkupsubsection{Additional Locale Definitions%
}
\isamarkuptrue%
%
\endisatagdocument
{\isafolddocument}%
%
\isadelimdocument
%
\endisadelimdocument
\isacommand{locale}\isamarkupfalse%
\ sigma{\isacharunderscore}{\kern0pt}finite{\isacharunderscore}{\kern0pt}adapted{\isacharunderscore}{\kern0pt}process\ {\isacharequal}{\kern0pt}\ sigma{\isacharunderscore}{\kern0pt}finite{\isacharunderscore}{\kern0pt}filtered{\isacharunderscore}{\kern0pt}measure\ M\ F\ t\isactrlsub {\isadigit{0}}\ {\isacharplus}{\kern0pt}\ adapted{\isacharunderscore}{\kern0pt}process\ M\ F\ t\isactrlsub {\isadigit{0}}\ X\ \isakeyword{for}\ M\ F\ t\isactrlsub {\isadigit{0}}\ X\isanewline
\isanewline
\isacommand{locale}\isamarkupfalse%
\ nat{\isacharunderscore}{\kern0pt}sigma{\isacharunderscore}{\kern0pt}finite{\isacharunderscore}{\kern0pt}adapted{\isacharunderscore}{\kern0pt}process\ {\isacharequal}{\kern0pt}\ sigma{\isacharunderscore}{\kern0pt}finite{\isacharunderscore}{\kern0pt}adapted{\isacharunderscore}{\kern0pt}process\ M\ F\ {\isachardoublequoteopen}{\isadigit{0}}\ {\isacharcolon}{\kern0pt}{\isacharcolon}{\kern0pt}\ nat{\isachardoublequoteclose}\ X\ \isakeyword{for}\ M\ F\ X\isanewline
\isacommand{locale}\isamarkupfalse%
\ real{\isacharunderscore}{\kern0pt}sigma{\isacharunderscore}{\kern0pt}finite{\isacharunderscore}{\kern0pt}adapted{\isacharunderscore}{\kern0pt}process\ {\isacharequal}{\kern0pt}\ sigma{\isacharunderscore}{\kern0pt}finite{\isacharunderscore}{\kern0pt}adapted{\isacharunderscore}{\kern0pt}process\ M\ F\ {\isachardoublequoteopen}{\isadigit{0}}\ {\isacharcolon}{\kern0pt}{\isacharcolon}{\kern0pt}\ real{\isachardoublequoteclose}\ X\ \isakeyword{for}\ M\ F\ X\isanewline
\isanewline
\isacommand{sublocale}\isamarkupfalse%
\ nat{\isacharunderscore}{\kern0pt}sigma{\isacharunderscore}{\kern0pt}finite{\isacharunderscore}{\kern0pt}adapted{\isacharunderscore}{\kern0pt}process\ {\isasymsubseteq}\ nat{\isacharunderscore}{\kern0pt}sigma{\isacharunderscore}{\kern0pt}finite{\isacharunderscore}{\kern0pt}filtered{\isacharunderscore}{\kern0pt}measure%
\isadelimproof
\ %
\endisadelimproof
%
\isatagproof
\isacommand{{\isachardot}{\kern0pt}{\isachardot}{\kern0pt}}\isamarkupfalse%
%
\endisatagproof
{\isafoldproof}%
%
\isadelimproof
%
\endisadelimproof
\isanewline
\isacommand{sublocale}\isamarkupfalse%
\ real{\isacharunderscore}{\kern0pt}sigma{\isacharunderscore}{\kern0pt}finite{\isacharunderscore}{\kern0pt}adapted{\isacharunderscore}{\kern0pt}process\ {\isasymsubseteq}\ real{\isacharunderscore}{\kern0pt}sigma{\isacharunderscore}{\kern0pt}finite{\isacharunderscore}{\kern0pt}filtered{\isacharunderscore}{\kern0pt}measure%
\isadelimproof
\ %
\endisadelimproof
%
\isatagproof
\isacommand{{\isachardot}{\kern0pt}{\isachardot}{\kern0pt}}\isamarkupfalse%
%
\endisatagproof
{\isafoldproof}%
%
\isadelimproof
%
\endisadelimproof
\isanewline
\isanewline
\isacommand{locale}\isamarkupfalse%
\ finite{\isacharunderscore}{\kern0pt}adapted{\isacharunderscore}{\kern0pt}process\ {\isacharequal}{\kern0pt}\ finite{\isacharunderscore}{\kern0pt}filtered{\isacharunderscore}{\kern0pt}measure\ M\ F\ t\isactrlsub {\isadigit{0}}\ {\isacharplus}{\kern0pt}\ adapted{\isacharunderscore}{\kern0pt}process\ M\ F\ t\isactrlsub {\isadigit{0}}\ X\ \isakeyword{for}\ M\ F\ t\isactrlsub {\isadigit{0}}\ X\isanewline
\isanewline
\isacommand{sublocale}\isamarkupfalse%
\ finite{\isacharunderscore}{\kern0pt}adapted{\isacharunderscore}{\kern0pt}process\ {\isasymsubseteq}\ sigma{\isacharunderscore}{\kern0pt}finite{\isacharunderscore}{\kern0pt}adapted{\isacharunderscore}{\kern0pt}process%
\isadelimproof
\ %
\endisadelimproof
%
\isatagproof
\isacommand{{\isachardot}{\kern0pt}{\isachardot}{\kern0pt}}\isamarkupfalse%
%
\endisatagproof
{\isafoldproof}%
%
\isadelimproof
%
\endisadelimproof
\isanewline
\isanewline
\isacommand{locale}\isamarkupfalse%
\ nat{\isacharunderscore}{\kern0pt}finite{\isacharunderscore}{\kern0pt}adapted{\isacharunderscore}{\kern0pt}process\ {\isacharequal}{\kern0pt}\ finite{\isacharunderscore}{\kern0pt}adapted{\isacharunderscore}{\kern0pt}process\ M\ F\ {\isachardoublequoteopen}{\isadigit{0}}\ {\isacharcolon}{\kern0pt}{\isacharcolon}{\kern0pt}\ nat{\isachardoublequoteclose}\ X\ \isakeyword{for}\ M\ F\ X\isanewline
\isacommand{locale}\isamarkupfalse%
\ real{\isacharunderscore}{\kern0pt}finite{\isacharunderscore}{\kern0pt}adapted{\isacharunderscore}{\kern0pt}process\ {\isacharequal}{\kern0pt}\ finite{\isacharunderscore}{\kern0pt}adapted{\isacharunderscore}{\kern0pt}process\ M\ F\ {\isachardoublequoteopen}{\isadigit{0}}\ {\isacharcolon}{\kern0pt}{\isacharcolon}{\kern0pt}\ real{\isachardoublequoteclose}\ X\ \isakeyword{for}\ M\ F\ X\isanewline
\isanewline
\isacommand{sublocale}\isamarkupfalse%
\ nat{\isacharunderscore}{\kern0pt}finite{\isacharunderscore}{\kern0pt}adapted{\isacharunderscore}{\kern0pt}process\ {\isasymsubseteq}\ nat{\isacharunderscore}{\kern0pt}sigma{\isacharunderscore}{\kern0pt}finite{\isacharunderscore}{\kern0pt}adapted{\isacharunderscore}{\kern0pt}process%
\isadelimproof
\ %
\endisadelimproof
%
\isatagproof
\isacommand{{\isachardot}{\kern0pt}{\isachardot}{\kern0pt}}\isamarkupfalse%
%
\endisatagproof
{\isafoldproof}%
%
\isadelimproof
%
\endisadelimproof
\isanewline
\isacommand{sublocale}\isamarkupfalse%
\ real{\isacharunderscore}{\kern0pt}finite{\isacharunderscore}{\kern0pt}adapted{\isacharunderscore}{\kern0pt}process\ {\isasymsubseteq}\ real{\isacharunderscore}{\kern0pt}sigma{\isacharunderscore}{\kern0pt}finite{\isacharunderscore}{\kern0pt}adapted{\isacharunderscore}{\kern0pt}process%
\isadelimproof
\ %
\endisadelimproof
%
\isatagproof
\isacommand{{\isachardot}{\kern0pt}{\isachardot}{\kern0pt}}\isamarkupfalse%
%
\endisatagproof
{\isafoldproof}%
%
\isadelimproof
%
\endisadelimproof
\isanewline
\isanewline
\isanewline
\isanewline
\isacommand{locale}\isamarkupfalse%
\ sigma{\isacharunderscore}{\kern0pt}finite{\isacharunderscore}{\kern0pt}adapted{\isacharunderscore}{\kern0pt}process{\isacharunderscore}{\kern0pt}order\ {\isacharequal}{\kern0pt}\ sigma{\isacharunderscore}{\kern0pt}finite{\isacharunderscore}{\kern0pt}adapted{\isacharunderscore}{\kern0pt}process\ M\ F\ t\isactrlsub {\isadigit{0}}\ X\ \isakeyword{for}\ M\ F\ t\isactrlsub {\isadigit{0}}\ \isakeyword{and}\ X\ {\isacharcolon}{\kern0pt}{\isacharcolon}{\kern0pt}\ {\isachardoublequoteopen}{\isacharunderscore}{\kern0pt}\ \ {\isasymRightarrow}\ {\isacharunderscore}{\kern0pt}\ {\isasymRightarrow}\ {\isacharunderscore}{\kern0pt}\ {\isacharcolon}{\kern0pt}{\isacharcolon}{\kern0pt}\ {\isacharbraceleft}{\kern0pt}order{\isacharunderscore}{\kern0pt}topology{\isacharcomma}{\kern0pt}\ ordered{\isacharunderscore}{\kern0pt}real{\isacharunderscore}{\kern0pt}vector{\isacharbraceright}{\kern0pt}{\isachardoublequoteclose}\ \isanewline
\isanewline
\isacommand{locale}\isamarkupfalse%
\ nat{\isacharunderscore}{\kern0pt}sigma{\isacharunderscore}{\kern0pt}finite{\isacharunderscore}{\kern0pt}adapted{\isacharunderscore}{\kern0pt}process{\isacharunderscore}{\kern0pt}order\ {\isacharequal}{\kern0pt}\ sigma{\isacharunderscore}{\kern0pt}finite{\isacharunderscore}{\kern0pt}adapted{\isacharunderscore}{\kern0pt}process{\isacharunderscore}{\kern0pt}order\ M\ F\ {\isachardoublequoteopen}{\isadigit{0}}\ {\isacharcolon}{\kern0pt}{\isacharcolon}{\kern0pt}\ nat{\isachardoublequoteclose}\ X\ \isakeyword{for}\ M\ F\ X\isanewline
\isacommand{locale}\isamarkupfalse%
\ real{\isacharunderscore}{\kern0pt}sigma{\isacharunderscore}{\kern0pt}finite{\isacharunderscore}{\kern0pt}adapted{\isacharunderscore}{\kern0pt}process{\isacharunderscore}{\kern0pt}order\ {\isacharequal}{\kern0pt}\ sigma{\isacharunderscore}{\kern0pt}finite{\isacharunderscore}{\kern0pt}adapted{\isacharunderscore}{\kern0pt}process{\isacharunderscore}{\kern0pt}order\ M\ F\ {\isachardoublequoteopen}{\isadigit{0}}\ {\isacharcolon}{\kern0pt}{\isacharcolon}{\kern0pt}\ real{\isachardoublequoteclose}\ X\ \isakeyword{for}\ M\ F\ X\isanewline
\isanewline
\isacommand{sublocale}\isamarkupfalse%
\ nat{\isacharunderscore}{\kern0pt}sigma{\isacharunderscore}{\kern0pt}finite{\isacharunderscore}{\kern0pt}adapted{\isacharunderscore}{\kern0pt}process{\isacharunderscore}{\kern0pt}order\ {\isasymsubseteq}\ nat{\isacharunderscore}{\kern0pt}sigma{\isacharunderscore}{\kern0pt}finite{\isacharunderscore}{\kern0pt}adapted{\isacharunderscore}{\kern0pt}process%
\isadelimproof
\ %
\endisadelimproof
%
\isatagproof
\isacommand{{\isachardot}{\kern0pt}{\isachardot}{\kern0pt}}\isamarkupfalse%
%
\endisatagproof
{\isafoldproof}%
%
\isadelimproof
%
\endisadelimproof
\isanewline
\isacommand{sublocale}\isamarkupfalse%
\ real{\isacharunderscore}{\kern0pt}sigma{\isacharunderscore}{\kern0pt}finite{\isacharunderscore}{\kern0pt}adapted{\isacharunderscore}{\kern0pt}process{\isacharunderscore}{\kern0pt}order\ {\isasymsubseteq}\ real{\isacharunderscore}{\kern0pt}sigma{\isacharunderscore}{\kern0pt}finite{\isacharunderscore}{\kern0pt}adapted{\isacharunderscore}{\kern0pt}process%
\isadelimproof
\ %
\endisadelimproof
%
\isatagproof
\isacommand{{\isachardot}{\kern0pt}{\isachardot}{\kern0pt}}\isamarkupfalse%
%
\endisatagproof
{\isafoldproof}%
%
\isadelimproof
%
\endisadelimproof
\isanewline
\ \ \ \ \ \ \ \ \ \ \ \ \ \ \ \ \ \ \ \ \ \ \ \ \ \ \ \ \ \ \ \ \ \ \ \ \ \ \ \ \ \ \ \ \ \ \ \ \isanewline
\isacommand{locale}\isamarkupfalse%
\ finite{\isacharunderscore}{\kern0pt}adapted{\isacharunderscore}{\kern0pt}process{\isacharunderscore}{\kern0pt}order\ {\isacharequal}{\kern0pt}\ finite{\isacharunderscore}{\kern0pt}adapted{\isacharunderscore}{\kern0pt}process\ M\ F\ t\isactrlsub {\isadigit{0}}\ X\ \isakeyword{for}\ M\ F\ t\isactrlsub {\isadigit{0}}\ \isakeyword{and}\ X\ {\isacharcolon}{\kern0pt}{\isacharcolon}{\kern0pt}\ {\isachardoublequoteopen}{\isacharunderscore}{\kern0pt}\ \ {\isasymRightarrow}\ {\isacharunderscore}{\kern0pt}\ {\isasymRightarrow}\ {\isacharunderscore}{\kern0pt}\ {\isacharcolon}{\kern0pt}{\isacharcolon}{\kern0pt}\ {\isacharbraceleft}{\kern0pt}order{\isacharunderscore}{\kern0pt}topology{\isacharcomma}{\kern0pt}\ ordered{\isacharunderscore}{\kern0pt}real{\isacharunderscore}{\kern0pt}vector{\isacharbraceright}{\kern0pt}{\isachardoublequoteclose}\ \isanewline
\isanewline
\isacommand{locale}\isamarkupfalse%
\ nat{\isacharunderscore}{\kern0pt}finite{\isacharunderscore}{\kern0pt}adapted{\isacharunderscore}{\kern0pt}process{\isacharunderscore}{\kern0pt}order\ {\isacharequal}{\kern0pt}\ finite{\isacharunderscore}{\kern0pt}adapted{\isacharunderscore}{\kern0pt}process{\isacharunderscore}{\kern0pt}order\ M\ F\ {\isachardoublequoteopen}{\isadigit{0}}\ {\isacharcolon}{\kern0pt}{\isacharcolon}{\kern0pt}\ nat{\isachardoublequoteclose}\ X\ \isakeyword{for}\ M\ F\ X\isanewline
\isacommand{locale}\isamarkupfalse%
\ real{\isacharunderscore}{\kern0pt}finite{\isacharunderscore}{\kern0pt}adapted{\isacharunderscore}{\kern0pt}process{\isacharunderscore}{\kern0pt}order\ {\isacharequal}{\kern0pt}\ finite{\isacharunderscore}{\kern0pt}adapted{\isacharunderscore}{\kern0pt}process{\isacharunderscore}{\kern0pt}order\ M\ F\ {\isachardoublequoteopen}{\isadigit{0}}\ {\isacharcolon}{\kern0pt}{\isacharcolon}{\kern0pt}\ real{\isachardoublequoteclose}\ X\ \isakeyword{for}\ M\ F\ X\isanewline
\isanewline
\isacommand{sublocale}\isamarkupfalse%
\ nat{\isacharunderscore}{\kern0pt}finite{\isacharunderscore}{\kern0pt}adapted{\isacharunderscore}{\kern0pt}process{\isacharunderscore}{\kern0pt}order\ {\isasymsubseteq}\ nat{\isacharunderscore}{\kern0pt}sigma{\isacharunderscore}{\kern0pt}finite{\isacharunderscore}{\kern0pt}adapted{\isacharunderscore}{\kern0pt}process{\isacharunderscore}{\kern0pt}order%
\isadelimproof
\ %
\endisadelimproof
%
\isatagproof
\isacommand{{\isachardot}{\kern0pt}{\isachardot}{\kern0pt}}\isamarkupfalse%
%
\endisatagproof
{\isafoldproof}%
%
\isadelimproof
%
\endisadelimproof
\isanewline
\isacommand{sublocale}\isamarkupfalse%
\ real{\isacharunderscore}{\kern0pt}finite{\isacharunderscore}{\kern0pt}adapted{\isacharunderscore}{\kern0pt}process{\isacharunderscore}{\kern0pt}order\ {\isasymsubseteq}\ real{\isacharunderscore}{\kern0pt}sigma{\isacharunderscore}{\kern0pt}finite{\isacharunderscore}{\kern0pt}adapted{\isacharunderscore}{\kern0pt}process{\isacharunderscore}{\kern0pt}order%
\isadelimproof
\ %
\endisadelimproof
%
\isatagproof
\isacommand{{\isachardot}{\kern0pt}{\isachardot}{\kern0pt}}\isamarkupfalse%
%
\endisatagproof
{\isafoldproof}%
%
\isadelimproof
%
\endisadelimproof
\isanewline
\isanewline
\isanewline
\isanewline
\isacommand{locale}\isamarkupfalse%
\ sigma{\isacharunderscore}{\kern0pt}finite{\isacharunderscore}{\kern0pt}adapted{\isacharunderscore}{\kern0pt}process{\isacharunderscore}{\kern0pt}linorder\ {\isacharequal}{\kern0pt}\ sigma{\isacharunderscore}{\kern0pt}finite{\isacharunderscore}{\kern0pt}adapted{\isacharunderscore}{\kern0pt}process{\isacharunderscore}{\kern0pt}order\ M\ F\ t\isactrlsub {\isadigit{0}}\ X\ \isakeyword{for}\ M\ F\ t\isactrlsub {\isadigit{0}}\ \isakeyword{and}\ X\ {\isacharcolon}{\kern0pt}{\isacharcolon}{\kern0pt}\ {\isachardoublequoteopen}{\isacharunderscore}{\kern0pt}\ \ {\isasymRightarrow}\ {\isacharunderscore}{\kern0pt}\ {\isasymRightarrow}\ {\isacharunderscore}{\kern0pt}\ {\isacharcolon}{\kern0pt}{\isacharcolon}{\kern0pt}\ {\isacharbraceleft}{\kern0pt}linorder{\isacharunderscore}{\kern0pt}topology{\isacharbraceright}{\kern0pt}{\isachardoublequoteclose}\isanewline
\isanewline
\isacommand{locale}\isamarkupfalse%
\ nat{\isacharunderscore}{\kern0pt}sigma{\isacharunderscore}{\kern0pt}finite{\isacharunderscore}{\kern0pt}adapted{\isacharunderscore}{\kern0pt}process{\isacharunderscore}{\kern0pt}linorder\ {\isacharequal}{\kern0pt}\ sigma{\isacharunderscore}{\kern0pt}finite{\isacharunderscore}{\kern0pt}adapted{\isacharunderscore}{\kern0pt}process{\isacharunderscore}{\kern0pt}linorder\ M\ F\ {\isachardoublequoteopen}{\isadigit{0}}\ {\isacharcolon}{\kern0pt}{\isacharcolon}{\kern0pt}\ nat{\isachardoublequoteclose}\ X\ \isakeyword{for}\ M\ F\ X\isanewline
\isacommand{locale}\isamarkupfalse%
\ real{\isacharunderscore}{\kern0pt}sigma{\isacharunderscore}{\kern0pt}finite{\isacharunderscore}{\kern0pt}adapted{\isacharunderscore}{\kern0pt}process{\isacharunderscore}{\kern0pt}linorder\ {\isacharequal}{\kern0pt}\ sigma{\isacharunderscore}{\kern0pt}finite{\isacharunderscore}{\kern0pt}adapted{\isacharunderscore}{\kern0pt}process{\isacharunderscore}{\kern0pt}linorder\ M\ F\ {\isachardoublequoteopen}{\isadigit{0}}\ {\isacharcolon}{\kern0pt}{\isacharcolon}{\kern0pt}\ real{\isachardoublequoteclose}\ X\ \isakeyword{for}\ M\ F\ X\isanewline
\isanewline
\isacommand{sublocale}\isamarkupfalse%
\ nat{\isacharunderscore}{\kern0pt}sigma{\isacharunderscore}{\kern0pt}finite{\isacharunderscore}{\kern0pt}adapted{\isacharunderscore}{\kern0pt}process{\isacharunderscore}{\kern0pt}linorder\ {\isasymsubseteq}\ nat{\isacharunderscore}{\kern0pt}sigma{\isacharunderscore}{\kern0pt}finite{\isacharunderscore}{\kern0pt}adapted{\isacharunderscore}{\kern0pt}process{\isacharunderscore}{\kern0pt}order%
\isadelimproof
\ %
\endisadelimproof
%
\isatagproof
\isacommand{{\isachardot}{\kern0pt}{\isachardot}{\kern0pt}}\isamarkupfalse%
%
\endisatagproof
{\isafoldproof}%
%
\isadelimproof
%
\endisadelimproof
\isanewline
\isacommand{sublocale}\isamarkupfalse%
\ real{\isacharunderscore}{\kern0pt}sigma{\isacharunderscore}{\kern0pt}finite{\isacharunderscore}{\kern0pt}adapted{\isacharunderscore}{\kern0pt}process{\isacharunderscore}{\kern0pt}linorder\ {\isasymsubseteq}\ real{\isacharunderscore}{\kern0pt}sigma{\isacharunderscore}{\kern0pt}finite{\isacharunderscore}{\kern0pt}adapted{\isacharunderscore}{\kern0pt}process{\isacharunderscore}{\kern0pt}order%
\isadelimproof
\ %
\endisadelimproof
%
\isatagproof
\isacommand{{\isachardot}{\kern0pt}{\isachardot}{\kern0pt}}\isamarkupfalse%
%
\endisatagproof
{\isafoldproof}%
%
\isadelimproof
%
\endisadelimproof
\isanewline
\isanewline
\isacommand{locale}\isamarkupfalse%
\ finite{\isacharunderscore}{\kern0pt}adapted{\isacharunderscore}{\kern0pt}process{\isacharunderscore}{\kern0pt}linorder\ {\isacharequal}{\kern0pt}\ finite{\isacharunderscore}{\kern0pt}adapted{\isacharunderscore}{\kern0pt}process{\isacharunderscore}{\kern0pt}order\ M\ F\ t\isactrlsub {\isadigit{0}}\ X\ \isakeyword{for}\ M\ F\ t\isactrlsub {\isadigit{0}}\ \isakeyword{and}\ X\ {\isacharcolon}{\kern0pt}{\isacharcolon}{\kern0pt}\ {\isachardoublequoteopen}{\isacharunderscore}{\kern0pt}\ \ {\isasymRightarrow}\ {\isacharunderscore}{\kern0pt}\ {\isasymRightarrow}\ {\isacharunderscore}{\kern0pt}\ {\isacharcolon}{\kern0pt}{\isacharcolon}{\kern0pt}\ {\isacharbraceleft}{\kern0pt}linorder{\isacharunderscore}{\kern0pt}topology{\isacharbraceright}{\kern0pt}{\isachardoublequoteclose}\isanewline
\isanewline
\isacommand{locale}\isamarkupfalse%
\ nat{\isacharunderscore}{\kern0pt}finite{\isacharunderscore}{\kern0pt}adapted{\isacharunderscore}{\kern0pt}process{\isacharunderscore}{\kern0pt}linorder\ {\isacharequal}{\kern0pt}\ finite{\isacharunderscore}{\kern0pt}adapted{\isacharunderscore}{\kern0pt}process{\isacharunderscore}{\kern0pt}linorder\ M\ F\ {\isachardoublequoteopen}{\isadigit{0}}\ {\isacharcolon}{\kern0pt}{\isacharcolon}{\kern0pt}\ nat{\isachardoublequoteclose}\ X\ \isakeyword{for}\ M\ F\ X\isanewline
\isacommand{locale}\isamarkupfalse%
\ real{\isacharunderscore}{\kern0pt}finite{\isacharunderscore}{\kern0pt}adapted{\isacharunderscore}{\kern0pt}process{\isacharunderscore}{\kern0pt}linorder\ {\isacharequal}{\kern0pt}\ finite{\isacharunderscore}{\kern0pt}adapted{\isacharunderscore}{\kern0pt}process{\isacharunderscore}{\kern0pt}linorder\ M\ F\ {\isachardoublequoteopen}{\isadigit{0}}\ {\isacharcolon}{\kern0pt}{\isacharcolon}{\kern0pt}\ real{\isachardoublequoteclose}\ X\ \isakeyword{for}\ M\ F\ X\isanewline
\isanewline
\isacommand{sublocale}\isamarkupfalse%
\ nat{\isacharunderscore}{\kern0pt}finite{\isacharunderscore}{\kern0pt}adapted{\isacharunderscore}{\kern0pt}process{\isacharunderscore}{\kern0pt}linorder\ {\isasymsubseteq}\ nat{\isacharunderscore}{\kern0pt}sigma{\isacharunderscore}{\kern0pt}finite{\isacharunderscore}{\kern0pt}adapted{\isacharunderscore}{\kern0pt}process{\isacharunderscore}{\kern0pt}linorder%
\isadelimproof
\ %
\endisadelimproof
%
\isatagproof
\isacommand{{\isachardot}{\kern0pt}{\isachardot}{\kern0pt}}\isamarkupfalse%
%
\endisatagproof
{\isafoldproof}%
%
\isadelimproof
%
\endisadelimproof
\isanewline
\isacommand{sublocale}\isamarkupfalse%
\ real{\isacharunderscore}{\kern0pt}finite{\isacharunderscore}{\kern0pt}adapted{\isacharunderscore}{\kern0pt}process{\isacharunderscore}{\kern0pt}linorder\ {\isasymsubseteq}\ real{\isacharunderscore}{\kern0pt}sigma{\isacharunderscore}{\kern0pt}finite{\isacharunderscore}{\kern0pt}adapted{\isacharunderscore}{\kern0pt}process{\isacharunderscore}{\kern0pt}linorder%
\isadelimproof
\ %
\endisadelimproof
%
\isatagproof
\isacommand{{\isachardot}{\kern0pt}{\isachardot}{\kern0pt}}\isamarkupfalse%
%
\endisatagproof
{\isafoldproof}%
%
\isadelimproof
%
\endisadelimproof
%
\isadelimdocument
%
\endisadelimdocument
%
\isatagdocument
%
\isamarkupsubsection{Martingale%
}
\isamarkuptrue%
%
\endisatagdocument
{\isafolddocument}%
%
\isadelimdocument
%
\endisadelimdocument
\isacommand{locale}\isamarkupfalse%
\ martingale\ {\isacharequal}{\kern0pt}\ sigma{\isacharunderscore}{\kern0pt}finite{\isacharunderscore}{\kern0pt}adapted{\isacharunderscore}{\kern0pt}process\ {\isacharplus}{\kern0pt}\isanewline
\ \ \isakeyword{assumes}\ integrable{\isacharcolon}{\kern0pt}\ {\isachardoublequoteopen}{\isasymAnd}i{\isachardot}{\kern0pt}\ t\isactrlsub {\isadigit{0}}\ {\isasymle}\ i\ {\isasymLongrightarrow}\ integrable\ M\ {\isacharparenleft}{\kern0pt}X\ i{\isacharparenright}{\kern0pt}{\isachardoublequoteclose}\isanewline
\ \ \ \ \ \ \isakeyword{and}\ martingale{\isacharunderscore}{\kern0pt}property{\isacharcolon}{\kern0pt}\ {\isachardoublequoteopen}{\isasymAnd}i\ j{\isachardot}{\kern0pt}\ t\isactrlsub {\isadigit{0}}\ {\isasymle}\ i\ {\isasymLongrightarrow}\ i\ {\isasymle}\ j\ {\isasymLongrightarrow}\ AE\ {\isasymxi}\ in\ M{\isachardot}{\kern0pt}\ X\ i\ {\isasymxi}\ {\isacharequal}{\kern0pt}\ cond{\isacharunderscore}{\kern0pt}exp\ M\ {\isacharparenleft}{\kern0pt}F\ i{\isacharparenright}{\kern0pt}\ {\isacharparenleft}{\kern0pt}X\ j{\isacharparenright}{\kern0pt}\ {\isasymxi}{\isachardoublequoteclose}\isanewline
\isanewline
\isacommand{locale}\isamarkupfalse%
\ martingale{\isacharunderscore}{\kern0pt}order\ {\isacharequal}{\kern0pt}\ martingale\ M\ F\ t\isactrlsub {\isadigit{0}}\ X\ \isakeyword{for}\ M\ F\ t\isactrlsub {\isadigit{0}}\ \isakeyword{and}\ X\ {\isacharcolon}{\kern0pt}{\isacharcolon}{\kern0pt}\ {\isachardoublequoteopen}{\isacharunderscore}{\kern0pt}\ {\isasymRightarrow}\ {\isacharunderscore}{\kern0pt}\ {\isasymRightarrow}\ {\isacharunderscore}{\kern0pt}\ {\isacharcolon}{\kern0pt}{\isacharcolon}{\kern0pt}\ {\isacharbraceleft}{\kern0pt}order{\isacharunderscore}{\kern0pt}topology{\isacharcomma}{\kern0pt}\ ordered{\isacharunderscore}{\kern0pt}real{\isacharunderscore}{\kern0pt}vector{\isacharbraceright}{\kern0pt}{\isachardoublequoteclose}\isanewline
\isacommand{locale}\isamarkupfalse%
\ martingale{\isacharunderscore}{\kern0pt}linorder\ {\isacharequal}{\kern0pt}\ martingale\ M\ F\ t\isactrlsub {\isadigit{0}}\ X\ \isakeyword{for}\ M\ F\ t\isactrlsub {\isadigit{0}}\ \isakeyword{and}\ X\ {\isacharcolon}{\kern0pt}{\isacharcolon}{\kern0pt}\ {\isachardoublequoteopen}{\isacharunderscore}{\kern0pt}\ {\isasymRightarrow}\ {\isacharunderscore}{\kern0pt}\ {\isasymRightarrow}\ {\isacharunderscore}{\kern0pt}\ {\isacharcolon}{\kern0pt}{\isacharcolon}{\kern0pt}\ {\isacharbraceleft}{\kern0pt}linorder{\isacharunderscore}{\kern0pt}topology{\isacharcomma}{\kern0pt}\ ordered{\isacharunderscore}{\kern0pt}real{\isacharunderscore}{\kern0pt}vector{\isacharbraceright}{\kern0pt}{\isachardoublequoteclose}\isanewline
\isacommand{sublocale}\isamarkupfalse%
\ martingale{\isacharunderscore}{\kern0pt}linorder\ {\isasymsubseteq}\ martingale{\isacharunderscore}{\kern0pt}order%
\isadelimproof
\ %
\endisadelimproof
%
\isatagproof
\isacommand{{\isachardot}{\kern0pt}{\isachardot}{\kern0pt}}\isamarkupfalse%
%
\endisatagproof
{\isafoldproof}%
%
\isadelimproof
%
\endisadelimproof
\isanewline
\isanewline
\isacommand{lemma}\isamarkupfalse%
\ {\isacharparenleft}{\kern0pt}\isakeyword{in}\ sigma{\isacharunderscore}{\kern0pt}finite{\isacharunderscore}{\kern0pt}filtered{\isacharunderscore}{\kern0pt}measure{\isacharparenright}{\kern0pt}\ martingale{\isacharunderscore}{\kern0pt}const{\isacharunderscore}{\kern0pt}fun{\isacharbrackleft}{\kern0pt}intro{\isacharbrackright}{\kern0pt}{\isacharcolon}{\kern0pt}\ \ \isanewline
\ \ \isakeyword{assumes}\ {\isachardoublequoteopen}integrable\ M\ f{\isachardoublequoteclose}\ {\isachardoublequoteopen}f\ {\isasymin}\ borel{\isacharunderscore}{\kern0pt}measurable\ {\isacharparenleft}{\kern0pt}F\ t\isactrlsub {\isadigit{0}}{\isacharparenright}{\kern0pt}{\isachardoublequoteclose}\isanewline
\ \ \isakeyword{shows}\ {\isachardoublequoteopen}martingale\ M\ F\ t\isactrlsub {\isadigit{0}}\ {\isacharparenleft}{\kern0pt}{\isasymlambda}{\isacharunderscore}{\kern0pt}{\isachardot}{\kern0pt}\ f{\isacharparenright}{\kern0pt}{\isachardoublequoteclose}\isanewline
%
\isadelimproof
\ \ %
\endisadelimproof
%
\isatagproof
\isacommand{using}\isamarkupfalse%
\ assms\ sigma{\isacharunderscore}{\kern0pt}finite{\isacharunderscore}{\kern0pt}subalgebra{\isachardot}{\kern0pt}cond{\isacharunderscore}{\kern0pt}exp{\isacharunderscore}{\kern0pt}F{\isacharunderscore}{\kern0pt}meas{\isacharbrackleft}{\kern0pt}OF\ {\isacharunderscore}{\kern0pt}\ assms{\isacharparenleft}{\kern0pt}{\isadigit{1}}{\isacharparenright}{\kern0pt}{\isacharcomma}{\kern0pt}\ THEN\ AE{\isacharunderscore}{\kern0pt}symmetric{\isacharbrackright}{\kern0pt}\ borel{\isacharunderscore}{\kern0pt}measurable{\isacharunderscore}{\kern0pt}mono\isanewline
\ \ \isacommand{by}\isamarkupfalse%
\ {\isacharparenleft}{\kern0pt}unfold{\isacharunderscore}{\kern0pt}locales{\isacharparenright}{\kern0pt}\ blast{\isacharplus}{\kern0pt}%
\endisatagproof
{\isafoldproof}%
%
\isadelimproof
\isanewline
%
\endisadelimproof
\isanewline
\isacommand{lemma}\isamarkupfalse%
\ {\isacharparenleft}{\kern0pt}\isakeyword{in}\ sigma{\isacharunderscore}{\kern0pt}finite{\isacharunderscore}{\kern0pt}filtered{\isacharunderscore}{\kern0pt}measure{\isacharparenright}{\kern0pt}\ martingale{\isacharunderscore}{\kern0pt}cond{\isacharunderscore}{\kern0pt}exp{\isacharbrackleft}{\kern0pt}intro{\isacharbrackright}{\kern0pt}{\isacharcolon}{\kern0pt}\ \ \isanewline
\ \ \isakeyword{assumes}\ {\isachardoublequoteopen}integrable\ M\ f{\isachardoublequoteclose}\isanewline
\ \ \isakeyword{shows}\ {\isachardoublequoteopen}martingale\ M\ F\ t\isactrlsub {\isadigit{0}}\ {\isacharparenleft}{\kern0pt}{\isasymlambda}i{\isachardot}{\kern0pt}\ cond{\isacharunderscore}{\kern0pt}exp\ M\ {\isacharparenleft}{\kern0pt}F\ i{\isacharparenright}{\kern0pt}\ f{\isacharparenright}{\kern0pt}{\isachardoublequoteclose}\isanewline
%
\isadelimproof
\ \ %
\endisadelimproof
%
\isatagproof
\isacommand{using}\isamarkupfalse%
\ sigma{\isacharunderscore}{\kern0pt}finite{\isacharunderscore}{\kern0pt}subalgebra{\isachardot}{\kern0pt}borel{\isacharunderscore}{\kern0pt}measurable{\isacharunderscore}{\kern0pt}cond{\isacharunderscore}{\kern0pt}exp{\isacharprime}{\kern0pt}\ borel{\isacharunderscore}{\kern0pt}measurable{\isacharunderscore}{\kern0pt}cond{\isacharunderscore}{\kern0pt}exp\ \isanewline
\ \ \isacommand{by}\isamarkupfalse%
\ {\isacharparenleft}{\kern0pt}unfold{\isacharunderscore}{\kern0pt}locales{\isacharparenright}{\kern0pt}\ {\isacharparenleft}{\kern0pt}auto\ intro{\isacharcolon}{\kern0pt}\ sigma{\isacharunderscore}{\kern0pt}finite{\isacharunderscore}{\kern0pt}subalgebra{\isachardot}{\kern0pt}cond{\isacharunderscore}{\kern0pt}exp{\isacharunderscore}{\kern0pt}nested{\isacharunderscore}{\kern0pt}subalg{\isacharbrackleft}{\kern0pt}OF\ {\isacharunderscore}{\kern0pt}\ assms{\isacharbrackright}{\kern0pt}\ simp\ add{\isacharcolon}{\kern0pt}\ subalgebra{\isacharunderscore}{\kern0pt}F\ subalgebras{\isacharparenright}{\kern0pt}%
\endisatagproof
{\isafoldproof}%
%
\isadelimproof
\isanewline
%
\endisadelimproof
\isanewline
\isacommand{corollary}\isamarkupfalse%
\ {\isacharparenleft}{\kern0pt}\isakeyword{in}\ sigma{\isacharunderscore}{\kern0pt}finite{\isacharunderscore}{\kern0pt}filtered{\isacharunderscore}{\kern0pt}measure{\isacharparenright}{\kern0pt}\ martingale{\isacharunderscore}{\kern0pt}zero{\isacharbrackleft}{\kern0pt}intro{\isacharbrackright}{\kern0pt}{\isacharcolon}{\kern0pt}\ {\isachardoublequoteopen}martingale\ M\ F\ t\isactrlsub {\isadigit{0}}\ {\isacharparenleft}{\kern0pt}{\isasymlambda}{\isacharunderscore}{\kern0pt}\ {\isacharunderscore}{\kern0pt}{\isachardot}{\kern0pt}\ {\isadigit{0}}{\isacharparenright}{\kern0pt}{\isachardoublequoteclose}%
\isadelimproof
\ %
\endisadelimproof
%
\isatagproof
\isacommand{by}\isamarkupfalse%
\ fastforce%
\endisatagproof
{\isafoldproof}%
%
\isadelimproof
%
\endisadelimproof
\isanewline
\isanewline
\isacommand{corollary}\isamarkupfalse%
\ {\isacharparenleft}{\kern0pt}\isakeyword{in}\ finite{\isacharunderscore}{\kern0pt}filtered{\isacharunderscore}{\kern0pt}measure{\isacharparenright}{\kern0pt}\ martingale{\isacharunderscore}{\kern0pt}const{\isacharbrackleft}{\kern0pt}intro{\isacharbrackright}{\kern0pt}{\isacharcolon}{\kern0pt}\ {\isachardoublequoteopen}martingale\ M\ F\ t\isactrlsub {\isadigit{0}}\ {\isacharparenleft}{\kern0pt}{\isasymlambda}{\isacharunderscore}{\kern0pt}\ {\isacharunderscore}{\kern0pt}{\isachardot}{\kern0pt}\ c{\isacharparenright}{\kern0pt}{\isachardoublequoteclose}%
\isadelimproof
\ %
\endisadelimproof
%
\isatagproof
\isacommand{by}\isamarkupfalse%
\ fastforce%
\endisatagproof
{\isafoldproof}%
%
\isadelimproof
%
\endisadelimproof
%
\isadelimdocument
%
\endisadelimdocument
%
\isatagdocument
%
\isamarkupsubsection{Submartingale%
}
\isamarkuptrue%
%
\endisatagdocument
{\isafolddocument}%
%
\isadelimdocument
%
\endisadelimdocument
\isacommand{locale}\isamarkupfalse%
\ submartingale\ {\isacharequal}{\kern0pt}\ sigma{\isacharunderscore}{\kern0pt}finite{\isacharunderscore}{\kern0pt}adapted{\isacharunderscore}{\kern0pt}process{\isacharunderscore}{\kern0pt}order\ {\isacharplus}{\kern0pt}\isanewline
\ \ \isakeyword{assumes}\ integrable{\isacharcolon}{\kern0pt}\ {\isachardoublequoteopen}{\isasymAnd}i{\isachardot}{\kern0pt}\ t\isactrlsub {\isadigit{0}}\ {\isasymle}\ i\ {\isasymLongrightarrow}\ integrable\ M\ {\isacharparenleft}{\kern0pt}X\ i{\isacharparenright}{\kern0pt}{\isachardoublequoteclose}\isanewline
\ \ \ \ \ \ \isakeyword{and}\ submartingale{\isacharunderscore}{\kern0pt}property{\isacharcolon}{\kern0pt}\ {\isachardoublequoteopen}{\isasymAnd}i\ j{\isachardot}{\kern0pt}\ t\isactrlsub {\isadigit{0}}\ {\isasymle}\ i\ {\isasymLongrightarrow}\ i\ {\isasymle}\ j\ {\isasymLongrightarrow}\ AE\ {\isasymxi}\ in\ M{\isachardot}{\kern0pt}\ X\ i\ {\isasymxi}\ {\isasymle}\ cond{\isacharunderscore}{\kern0pt}exp\ M\ {\isacharparenleft}{\kern0pt}F\ i{\isacharparenright}{\kern0pt}\ {\isacharparenleft}{\kern0pt}X\ j{\isacharparenright}{\kern0pt}\ {\isasymxi}{\isachardoublequoteclose}\isanewline
\isanewline
\isacommand{locale}\isamarkupfalse%
\ submartingale{\isacharunderscore}{\kern0pt}linorder\ {\isacharequal}{\kern0pt}\ submartingale\ M\ F\ t\isactrlsub {\isadigit{0}}\ X\ \isakeyword{for}\ M\ F\ t\isactrlsub {\isadigit{0}}\ \isakeyword{and}\ X\ {\isacharcolon}{\kern0pt}{\isacharcolon}{\kern0pt}\ {\isachardoublequoteopen}{\isacharunderscore}{\kern0pt}\ {\isasymRightarrow}\ {\isacharunderscore}{\kern0pt}\ {\isasymRightarrow}\ {\isacharunderscore}{\kern0pt}\ {\isacharcolon}{\kern0pt}{\isacharcolon}{\kern0pt}\ {\isacharbraceleft}{\kern0pt}linorder{\isacharunderscore}{\kern0pt}topology{\isacharbraceright}{\kern0pt}{\isachardoublequoteclose}\isanewline
\isanewline
\isacommand{sublocale}\isamarkupfalse%
\ martingale{\isacharunderscore}{\kern0pt}order\ {\isasymsubseteq}\ submartingale%
\isadelimproof
\ %
\endisadelimproof
%
\isatagproof
\isacommand{using}\isamarkupfalse%
\ martingale{\isacharunderscore}{\kern0pt}property\ \isacommand{by}\isamarkupfalse%
\ {\isacharparenleft}{\kern0pt}unfold{\isacharunderscore}{\kern0pt}locales{\isacharparenright}{\kern0pt}\ {\isacharparenleft}{\kern0pt}force\ simp\ add{\isacharcolon}{\kern0pt}\ integrable{\isacharparenright}{\kern0pt}{\isacharplus}{\kern0pt}%
\endisatagproof
{\isafoldproof}%
%
\isadelimproof
%
\endisadelimproof
\isanewline
\isacommand{sublocale}\isamarkupfalse%
\ martingale{\isacharunderscore}{\kern0pt}linorder\ {\isasymsubseteq}\ submartingale{\isacharunderscore}{\kern0pt}linorder%
\isadelimproof
\ %
\endisadelimproof
%
\isatagproof
\isacommand{{\isachardot}{\kern0pt}{\isachardot}{\kern0pt}}\isamarkupfalse%
%
\endisatagproof
{\isafoldproof}%
%
\isadelimproof
%
\endisadelimproof
%
\isadelimdocument
%
\endisadelimdocument
%
\isatagdocument
%
\isamarkupsubsection{Supermartingale%
}
\isamarkuptrue%
%
\endisatagdocument
{\isafolddocument}%
%
\isadelimdocument
%
\endisadelimdocument
\isacommand{locale}\isamarkupfalse%
\ supermartingale\ {\isacharequal}{\kern0pt}\ sigma{\isacharunderscore}{\kern0pt}finite{\isacharunderscore}{\kern0pt}adapted{\isacharunderscore}{\kern0pt}process{\isacharunderscore}{\kern0pt}order\ {\isacharplus}{\kern0pt}\isanewline
\ \ \isakeyword{assumes}\ integrable{\isacharcolon}{\kern0pt}\ {\isachardoublequoteopen}{\isasymAnd}i{\isachardot}{\kern0pt}\ t\isactrlsub {\isadigit{0}}\ {\isasymle}\ i\ {\isasymLongrightarrow}\ integrable\ M\ {\isacharparenleft}{\kern0pt}X\ i{\isacharparenright}{\kern0pt}{\isachardoublequoteclose}\isanewline
\ \ \ \ \ \ \isakeyword{and}\ supermartingale{\isacharunderscore}{\kern0pt}property{\isacharcolon}{\kern0pt}\ {\isachardoublequoteopen}{\isasymAnd}i\ j{\isachardot}{\kern0pt}\ t\isactrlsub {\isadigit{0}}\ {\isasymle}\ i\ {\isasymLongrightarrow}\ i\ {\isasymle}\ j\ {\isasymLongrightarrow}\ AE\ {\isasymxi}\ in\ M{\isachardot}{\kern0pt}\ X\ i\ {\isasymxi}\ {\isasymge}\ cond{\isacharunderscore}{\kern0pt}exp\ M\ {\isacharparenleft}{\kern0pt}F\ i{\isacharparenright}{\kern0pt}\ {\isacharparenleft}{\kern0pt}X\ j{\isacharparenright}{\kern0pt}\ {\isasymxi}{\isachardoublequoteclose}\isanewline
\isanewline
\isacommand{locale}\isamarkupfalse%
\ supermartingale{\isacharunderscore}{\kern0pt}linorder\ {\isacharequal}{\kern0pt}\ supermartingale\ M\ F\ t\isactrlsub {\isadigit{0}}\ X\ \isakeyword{for}\ M\ F\ t\isactrlsub {\isadigit{0}}\ \isakeyword{and}\ X\ {\isacharcolon}{\kern0pt}{\isacharcolon}{\kern0pt}\ {\isachardoublequoteopen}{\isacharunderscore}{\kern0pt}\ {\isasymRightarrow}\ {\isacharunderscore}{\kern0pt}\ {\isasymRightarrow}\ {\isacharunderscore}{\kern0pt}\ {\isacharcolon}{\kern0pt}{\isacharcolon}{\kern0pt}\ {\isacharbraceleft}{\kern0pt}linorder{\isacharunderscore}{\kern0pt}topology{\isacharbraceright}{\kern0pt}{\isachardoublequoteclose}\isanewline
\isanewline
\isacommand{sublocale}\isamarkupfalse%
\ martingale{\isacharunderscore}{\kern0pt}order\ {\isasymsubseteq}\ supermartingale%
\isadelimproof
\ %
\endisadelimproof
%
\isatagproof
\isacommand{using}\isamarkupfalse%
\ martingale{\isacharunderscore}{\kern0pt}property\ \isacommand{by}\isamarkupfalse%
\ {\isacharparenleft}{\kern0pt}unfold{\isacharunderscore}{\kern0pt}locales{\isacharparenright}{\kern0pt}\ {\isacharparenleft}{\kern0pt}force\ simp\ add{\isacharcolon}{\kern0pt}\ integrable{\isacharparenright}{\kern0pt}{\isacharplus}{\kern0pt}%
\endisatagproof
{\isafoldproof}%
%
\isadelimproof
%
\endisadelimproof
\isanewline
\isacommand{sublocale}\isamarkupfalse%
\ martingale{\isacharunderscore}{\kern0pt}linorder\ {\isasymsubseteq}\ supermartingale{\isacharunderscore}{\kern0pt}linorder%
\isadelimproof
\ %
\endisadelimproof
%
\isatagproof
\isacommand{{\isachardot}{\kern0pt}{\isachardot}{\kern0pt}}\isamarkupfalse%
%
\endisatagproof
{\isafoldproof}%
%
\isadelimproof
%
\endisadelimproof
%
\begin{isamarkuptext}%
A stochastic process is a martingale, if and only if it is both a submartingale and a supermartingale.%
\end{isamarkuptext}\isamarkuptrue%
\isacommand{lemma}\isamarkupfalse%
\ martingale{\isacharunderscore}{\kern0pt}iff{\isacharcolon}{\kern0pt}\ \isanewline
\ \ \isakeyword{shows}\ {\isachardoublequoteopen}martingale\ M\ F\ t\isactrlsub {\isadigit{0}}\ X\ {\isasymlongleftrightarrow}\ submartingale\ M\ F\ t\isactrlsub {\isadigit{0}}\ X\ {\isasymand}\ supermartingale\ M\ F\ t\isactrlsub {\isadigit{0}}\ X{\isachardoublequoteclose}\isanewline
%
\isadelimproof
%
\endisadelimproof
%
\isatagproof
\isacommand{proof}\isamarkupfalse%
\ {\isacharparenleft}{\kern0pt}rule\ iffI{\isacharparenright}{\kern0pt}\isanewline
\ \ \isacommand{assume}\isamarkupfalse%
\ asm{\isacharcolon}{\kern0pt}\ {\isachardoublequoteopen}martingale\ M\ F\ t\isactrlsub {\isadigit{0}}\ X{\isachardoublequoteclose}\isanewline
\ \ \isacommand{interpret}\isamarkupfalse%
\ martingale{\isacharunderscore}{\kern0pt}order\ M\ F\ t\isactrlsub {\isadigit{0}}\ X\ \isacommand{by}\isamarkupfalse%
\ {\isacharparenleft}{\kern0pt}intro\ martingale{\isacharunderscore}{\kern0pt}order{\isachardot}{\kern0pt}intro\ asm{\isacharparenright}{\kern0pt}\isanewline
\ \ \isacommand{show}\isamarkupfalse%
\ {\isachardoublequoteopen}submartingale\ M\ F\ t\isactrlsub {\isadigit{0}}\ X\ {\isasymand}\ supermartingale\ M\ F\ t\isactrlsub {\isadigit{0}}\ X{\isachardoublequoteclose}\ \isacommand{using}\isamarkupfalse%
\ submartingale{\isacharunderscore}{\kern0pt}axioms\ supermartingale{\isacharunderscore}{\kern0pt}axioms\ \isacommand{by}\isamarkupfalse%
\ blast\isanewline
\isacommand{next}\isamarkupfalse%
\isanewline
\ \ \isacommand{assume}\isamarkupfalse%
\ asm{\isacharcolon}{\kern0pt}\ {\isachardoublequoteopen}submartingale\ M\ F\ t\isactrlsub {\isadigit{0}}\ X\ {\isasymand}\ supermartingale\ M\ F\ t\isactrlsub {\isadigit{0}}\ X{\isachardoublequoteclose}\isanewline
\ \ \isacommand{interpret}\isamarkupfalse%
\ submartingale\ M\ F\ t\isactrlsub {\isadigit{0}}\ X\ \isacommand{by}\isamarkupfalse%
\ {\isacharparenleft}{\kern0pt}simp\ add{\isacharcolon}{\kern0pt}\ asm{\isacharparenright}{\kern0pt}\isanewline
\ \ \isacommand{interpret}\isamarkupfalse%
\ supermartingale\ M\ F\ t\isactrlsub {\isadigit{0}}\ X\ \isacommand{by}\isamarkupfalse%
\ {\isacharparenleft}{\kern0pt}simp\ add{\isacharcolon}{\kern0pt}\ asm{\isacharparenright}{\kern0pt}\isanewline
\ \ \isacommand{show}\isamarkupfalse%
\ {\isachardoublequoteopen}martingale\ M\ F\ t\isactrlsub {\isadigit{0}}\ X{\isachardoublequoteclose}\ \isacommand{using}\isamarkupfalse%
\ submartingale{\isacharunderscore}{\kern0pt}property\ supermartingale{\isacharunderscore}{\kern0pt}property\ \isacommand{by}\isamarkupfalse%
\ {\isacharparenleft}{\kern0pt}unfold{\isacharunderscore}{\kern0pt}locales{\isacharparenright}{\kern0pt}\ {\isacharparenleft}{\kern0pt}intro\ integrable{\isacharcomma}{\kern0pt}\ blast{\isacharcomma}{\kern0pt}\ force{\isacharparenright}{\kern0pt}\isanewline
\isacommand{qed}\isamarkupfalse%
%
\endisatagproof
{\isafoldproof}%
%
\isadelimproof
%
\endisadelimproof
%
\isadelimdocument
%
\endisadelimdocument
%
\isatagdocument
%
\isamarkupsubsection{Martingale Lemmas%
}
\isamarkuptrue%
%
\endisatagdocument
{\isafolddocument}%
%
\isadelimdocument
%
\endisadelimdocument
\isacommand{context}\isamarkupfalse%
\ martingale\isanewline
\isakeyword{begin}\isanewline
\isanewline
\isacommand{lemma}\isamarkupfalse%
\ cond{\isacharunderscore}{\kern0pt}exp{\isacharunderscore}{\kern0pt}diff{\isacharunderscore}{\kern0pt}eq{\isacharunderscore}{\kern0pt}zero{\isacharcolon}{\kern0pt}\isanewline
\ \ \isakeyword{assumes}\ {\isachardoublequoteopen}t\isactrlsub {\isadigit{0}}\ {\isasymle}\ i{\isachardoublequoteclose}\ {\isachardoublequoteopen}i\ {\isasymle}\ j{\isachardoublequoteclose}\isanewline
\ \ \isakeyword{shows}\ {\isachardoublequoteopen}AE\ {\isasymxi}\ in\ M{\isachardot}{\kern0pt}\ cond{\isacharunderscore}{\kern0pt}exp\ M\ {\isacharparenleft}{\kern0pt}F\ i{\isacharparenright}{\kern0pt}\ {\isacharparenleft}{\kern0pt}{\isasymlambda}{\isasymxi}{\isachardot}{\kern0pt}\ X\ j\ {\isasymxi}\ {\isacharminus}{\kern0pt}\ X\ i\ {\isasymxi}{\isacharparenright}{\kern0pt}\ {\isasymxi}\ {\isacharequal}{\kern0pt}\ {\isadigit{0}}{\isachardoublequoteclose}\isanewline
%
\isadelimproof
\ \ %
\endisadelimproof
%
\isatagproof
\isacommand{using}\isamarkupfalse%
\ martingale{\isacharunderscore}{\kern0pt}property{\isacharbrackleft}{\kern0pt}OF\ assms{\isacharbrackright}{\kern0pt}\ assms\isanewline
\ \ \ \ \ \ \ \ sigma{\isacharunderscore}{\kern0pt}finite{\isacharunderscore}{\kern0pt}subalgebra{\isachardot}{\kern0pt}cond{\isacharunderscore}{\kern0pt}exp{\isacharunderscore}{\kern0pt}F{\isacharunderscore}{\kern0pt}meas{\isacharbrackleft}{\kern0pt}OF\ {\isacharunderscore}{\kern0pt}\ integrable\ adapted{\isacharcomma}{\kern0pt}\ of\ i{\isacharbrackright}{\kern0pt}\isanewline
\ \ \ \ \ \ \ \ sigma{\isacharunderscore}{\kern0pt}finite{\isacharunderscore}{\kern0pt}subalgebra{\isachardot}{\kern0pt}cond{\isacharunderscore}{\kern0pt}exp{\isacharunderscore}{\kern0pt}diff{\isacharbrackleft}{\kern0pt}OF\ {\isacharunderscore}{\kern0pt}\ integrable{\isacharparenleft}{\kern0pt}{\isadigit{1}}{\isacharcomma}{\kern0pt}{\isadigit{1}}{\isacharparenright}{\kern0pt}{\isacharcomma}{\kern0pt}\ of\ {\isachardoublequoteopen}F\ i{\isachardoublequoteclose}\ j\ i{\isacharbrackright}{\kern0pt}\ \isacommand{by}\isamarkupfalse%
\ fastforce%
\endisatagproof
{\isafoldproof}%
%
\isadelimproof
\isanewline
%
\endisadelimproof
\isanewline
\isacommand{lemma}\isamarkupfalse%
\ set{\isacharunderscore}{\kern0pt}integral{\isacharunderscore}{\kern0pt}eq{\isacharcolon}{\kern0pt}\isanewline
\ \ \isakeyword{assumes}\ {\isachardoublequoteopen}A\ {\isasymin}\ F\ i{\isachardoublequoteclose}\ {\isachardoublequoteopen}t\isactrlsub {\isadigit{0}}\ {\isasymle}\ i{\isachardoublequoteclose}\ {\isachardoublequoteopen}i\ {\isasymle}\ j{\isachardoublequoteclose}\isanewline
\ \ \isakeyword{shows}\ {\isachardoublequoteopen}set{\isacharunderscore}{\kern0pt}lebesgue{\isacharunderscore}{\kern0pt}integral\ M\ A\ {\isacharparenleft}{\kern0pt}X\ i{\isacharparenright}{\kern0pt}\ {\isacharequal}{\kern0pt}\ set{\isacharunderscore}{\kern0pt}lebesgue{\isacharunderscore}{\kern0pt}integral\ M\ A\ {\isacharparenleft}{\kern0pt}X\ j{\isacharparenright}{\kern0pt}{\isachardoublequoteclose}\isanewline
%
\isadelimproof
%
\endisadelimproof
%
\isatagproof
\isacommand{proof}\isamarkupfalse%
\ {\isacharminus}{\kern0pt}\isanewline
\ \ \isacommand{interpret}\isamarkupfalse%
\ sigma{\isacharunderscore}{\kern0pt}finite{\isacharunderscore}{\kern0pt}subalgebra\ M\ {\isachardoublequoteopen}F\ i{\isachardoublequoteclose}\ \isacommand{using}\isamarkupfalse%
\ assms{\isacharparenleft}{\kern0pt}{\isadigit{2}}{\isacharparenright}{\kern0pt}\ \isacommand{by}\isamarkupfalse%
\ blast\isanewline
\ \ \isacommand{have}\isamarkupfalse%
\ {\isachardoublequoteopen}{\isasymintegral}x\ {\isasymin}\ A{\isachardot}{\kern0pt}\ X\ i\ x\ {\isasympartial}M\ {\isacharequal}{\kern0pt}\ {\isasymintegral}x\ {\isasymin}\ A{\isachardot}{\kern0pt}\ cond{\isacharunderscore}{\kern0pt}exp\ M\ {\isacharparenleft}{\kern0pt}F\ i{\isacharparenright}{\kern0pt}\ {\isacharparenleft}{\kern0pt}X\ j{\isacharparenright}{\kern0pt}\ x\ {\isasympartial}M{\isachardoublequoteclose}\ \isacommand{using}\isamarkupfalse%
\ martingale{\isacharunderscore}{\kern0pt}property{\isacharbrackleft}{\kern0pt}OF\ assms{\isacharparenleft}{\kern0pt}{\isadigit{2}}{\isacharcomma}{\kern0pt}{\isadigit{3}}{\isacharparenright}{\kern0pt}{\isacharbrackright}{\kern0pt}\ borel{\isacharunderscore}{\kern0pt}measurable{\isacharunderscore}{\kern0pt}cond{\isacharunderscore}{\kern0pt}exp{\isacharprime}{\kern0pt}\ assms\ subalgebras\ subalgebra{\isacharunderscore}{\kern0pt}def\ \isacommand{by}\isamarkupfalse%
\ {\isacharparenleft}{\kern0pt}intro\ set{\isacharunderscore}{\kern0pt}lebesgue{\isacharunderscore}{\kern0pt}integral{\isacharunderscore}{\kern0pt}cong{\isacharunderscore}{\kern0pt}AE{\isacharbrackleft}{\kern0pt}OF\ {\isacharunderscore}{\kern0pt}\ random{\isacharunderscore}{\kern0pt}variable{\isacharbrackright}{\kern0pt}{\isacharparenright}{\kern0pt}\ fastforce{\isacharplus}{\kern0pt}\isanewline
\ \ \isacommand{also}\isamarkupfalse%
\ \isacommand{have}\isamarkupfalse%
\ {\isachardoublequoteopen}{\isachardot}{\kern0pt}{\isachardot}{\kern0pt}{\isachardot}{\kern0pt}\ {\isacharequal}{\kern0pt}\ {\isasymintegral}x\ {\isasymin}\ A{\isachardot}{\kern0pt}\ X\ j\ x\ {\isasympartial}M{\isachardoublequoteclose}\ \isacommand{using}\isamarkupfalse%
\ assms\ \isacommand{by}\isamarkupfalse%
\ {\isacharparenleft}{\kern0pt}auto\ simp{\isacharcolon}{\kern0pt}\ integrable\ intro{\isacharcolon}{\kern0pt}\ cond{\isacharunderscore}{\kern0pt}exp{\isacharunderscore}{\kern0pt}set{\isacharunderscore}{\kern0pt}integral{\isacharbrackleft}{\kern0pt}symmetric{\isacharbrackright}{\kern0pt}{\isacharparenright}{\kern0pt}\isanewline
\ \ \isacommand{finally}\isamarkupfalse%
\ \isacommand{show}\isamarkupfalse%
\ {\isacharquery}{\kern0pt}thesis\ \isacommand{{\isachardot}{\kern0pt}}\isamarkupfalse%
\isanewline
\isacommand{qed}\isamarkupfalse%
%
\endisatagproof
{\isafoldproof}%
%
\isadelimproof
\isanewline
%
\endisadelimproof
\isanewline
\isacommand{lemma}\isamarkupfalse%
\ scaleR{\isacharunderscore}{\kern0pt}const{\isacharbrackleft}{\kern0pt}intro{\isacharbrackright}{\kern0pt}{\isacharcolon}{\kern0pt}\isanewline
\ \ \isakeyword{shows}\ {\isachardoublequoteopen}martingale\ M\ F\ t\isactrlsub {\isadigit{0}}\ {\isacharparenleft}{\kern0pt}{\isasymlambda}i\ x{\isachardot}{\kern0pt}\ c\ {\isacharasterisk}{\kern0pt}\isactrlsub R\ X\ i\ x{\isacharparenright}{\kern0pt}{\isachardoublequoteclose}\isanewline
%
\isadelimproof
%
\endisadelimproof
%
\isatagproof
\isacommand{proof}\isamarkupfalse%
\ {\isacharminus}{\kern0pt}\isanewline
\ \ \isacommand{{\isacharbraceleft}{\kern0pt}}\isamarkupfalse%
\isanewline
\ \ \ \ \isacommand{fix}\isamarkupfalse%
\ i\ j\ {\isacharcolon}{\kern0pt}{\isacharcolon}{\kern0pt}\ {\isacharprime}{\kern0pt}b\ \isacommand{assume}\isamarkupfalse%
\ asm{\isacharcolon}{\kern0pt}\ {\isachardoublequoteopen}t\isactrlsub {\isadigit{0}}\ {\isasymle}\ i{\isachardoublequoteclose}\ {\isachardoublequoteopen}i\ {\isasymle}\ j{\isachardoublequoteclose}\isanewline
\ \ \ \ \isacommand{interpret}\isamarkupfalse%
\ sigma{\isacharunderscore}{\kern0pt}finite{\isacharunderscore}{\kern0pt}subalgebra\ M\ {\isachardoublequoteopen}F\ i{\isachardoublequoteclose}\ \isacommand{using}\isamarkupfalse%
\ asm\ \isacommand{by}\isamarkupfalse%
\ blast\isanewline
\ \ \ \ \isacommand{have}\isamarkupfalse%
\ {\isachardoublequoteopen}AE\ x\ in\ M{\isachardot}{\kern0pt}\ c\ {\isacharasterisk}{\kern0pt}\isactrlsub R\ X\ i\ x\ {\isacharequal}{\kern0pt}\ cond{\isacharunderscore}{\kern0pt}exp\ M\ {\isacharparenleft}{\kern0pt}F\ i{\isacharparenright}{\kern0pt}\ {\isacharparenleft}{\kern0pt}{\isasymlambda}x{\isachardot}{\kern0pt}\ c\ {\isacharasterisk}{\kern0pt}\isactrlsub R\ X\ j\ x{\isacharparenright}{\kern0pt}\ x{\isachardoublequoteclose}\ \isacommand{using}\isamarkupfalse%
\ asm\ cond{\isacharunderscore}{\kern0pt}exp{\isacharunderscore}{\kern0pt}scaleR{\isacharunderscore}{\kern0pt}right{\isacharbrackleft}{\kern0pt}OF\ integrable{\isacharcomma}{\kern0pt}\ of\ j{\isacharcomma}{\kern0pt}\ THEN\ AE{\isacharunderscore}{\kern0pt}symmetric{\isacharbrackright}{\kern0pt}\ martingale{\isacharunderscore}{\kern0pt}property{\isacharbrackleft}{\kern0pt}OF\ asm{\isacharbrackright}{\kern0pt}\ \isacommand{by}\isamarkupfalse%
\ force\isanewline
\ \ \isacommand{{\isacharbraceright}{\kern0pt}}\isamarkupfalse%
\isanewline
\ \ \isacommand{thus}\isamarkupfalse%
\ {\isacharquery}{\kern0pt}thesis\ \isacommand{by}\isamarkupfalse%
\ {\isacharparenleft}{\kern0pt}unfold{\isacharunderscore}{\kern0pt}locales{\isacharparenright}{\kern0pt}\ {\isacharparenleft}{\kern0pt}auto\ simp\ add{\isacharcolon}{\kern0pt}\ integrable\ martingale{\isachardot}{\kern0pt}integrable{\isacharparenright}{\kern0pt}\isanewline
\isacommand{qed}\isamarkupfalse%
%
\endisatagproof
{\isafoldproof}%
%
\isadelimproof
\isanewline
%
\endisadelimproof
\isanewline
\isacommand{lemma}\isamarkupfalse%
\ uminus{\isacharbrackleft}{\kern0pt}intro{\isacharbrackright}{\kern0pt}{\isacharcolon}{\kern0pt}\isanewline
\ \ \isakeyword{shows}\ {\isachardoublequoteopen}martingale\ M\ F\ t\isactrlsub {\isadigit{0}}\ {\isacharparenleft}{\kern0pt}{\isacharminus}{\kern0pt}\ X{\isacharparenright}{\kern0pt}{\isachardoublequoteclose}\ \isanewline
%
\isadelimproof
\ \ %
\endisadelimproof
%
\isatagproof
\isacommand{using}\isamarkupfalse%
\ scaleR{\isacharunderscore}{\kern0pt}const{\isacharbrackleft}{\kern0pt}of\ {\isachardoublequoteopen}{\isacharminus}{\kern0pt}{\isadigit{1}}{\isachardoublequoteclose}{\isacharbrackright}{\kern0pt}\ \isacommand{by}\isamarkupfalse%
\ {\isacharparenleft}{\kern0pt}force\ intro{\isacharcolon}{\kern0pt}\ back{\isacharunderscore}{\kern0pt}subst{\isacharbrackleft}{\kern0pt}of\ {\isachardoublequoteopen}martingale\ M\ F\ t\isactrlsub {\isadigit{0}}{\isachardoublequoteclose}{\isacharbrackright}{\kern0pt}{\isacharparenright}{\kern0pt}%
\endisatagproof
{\isafoldproof}%
%
\isadelimproof
\isanewline
%
\endisadelimproof
\isanewline
\isacommand{lemma}\isamarkupfalse%
\ add{\isacharbrackleft}{\kern0pt}intro{\isacharbrackright}{\kern0pt}{\isacharcolon}{\kern0pt}\isanewline
\ \ \isakeyword{assumes}\ {\isachardoublequoteopen}martingale\ M\ F\ t\isactrlsub {\isadigit{0}}\ Y{\isachardoublequoteclose}\isanewline
\ \ \isakeyword{shows}\ {\isachardoublequoteopen}martingale\ M\ F\ t\isactrlsub {\isadigit{0}}\ {\isacharparenleft}{\kern0pt}{\isasymlambda}i\ {\isasymxi}{\isachardot}{\kern0pt}\ X\ i\ {\isasymxi}\ {\isacharplus}{\kern0pt}\ Y\ i\ {\isasymxi}{\isacharparenright}{\kern0pt}{\isachardoublequoteclose}\isanewline
%
\isadelimproof
%
\endisadelimproof
%
\isatagproof
\isacommand{proof}\isamarkupfalse%
\ {\isacharminus}{\kern0pt}\isanewline
\ \ \isacommand{interpret}\isamarkupfalse%
\ Y{\isacharcolon}{\kern0pt}\ martingale\ M\ F\ t\isactrlsub {\isadigit{0}}\ Y\ \isacommand{by}\isamarkupfalse%
\ {\isacharparenleft}{\kern0pt}rule\ assms{\isacharparenright}{\kern0pt}\isanewline
\ \ \isacommand{{\isacharbraceleft}{\kern0pt}}\isamarkupfalse%
\isanewline
\ \ \ \ \isacommand{fix}\isamarkupfalse%
\ i\ j\ {\isacharcolon}{\kern0pt}{\isacharcolon}{\kern0pt}\ {\isacharprime}{\kern0pt}b\ \isacommand{assume}\isamarkupfalse%
\ asm{\isacharcolon}{\kern0pt}\ {\isachardoublequoteopen}t\isactrlsub {\isadigit{0}}\ {\isasymle}\ i{\isachardoublequoteclose}\ {\isachardoublequoteopen}i\ {\isasymle}\ j{\isachardoublequoteclose}\isanewline
\ \ \ \ \isacommand{hence}\isamarkupfalse%
\ {\isachardoublequoteopen}AE\ {\isasymxi}\ in\ M{\isachardot}{\kern0pt}\ X\ i\ {\isasymxi}\ {\isacharplus}{\kern0pt}\ Y\ i\ {\isasymxi}\ {\isacharequal}{\kern0pt}\ cond{\isacharunderscore}{\kern0pt}exp\ M\ {\isacharparenleft}{\kern0pt}F\ i{\isacharparenright}{\kern0pt}\ {\isacharparenleft}{\kern0pt}{\isasymlambda}x{\isachardot}{\kern0pt}\ X\ j\ x\ {\isacharplus}{\kern0pt}\ Y\ j\ x{\isacharparenright}{\kern0pt}\ {\isasymxi}{\isachardoublequoteclose}\ \isanewline
\ \ \ \ \ \ \isacommand{using}\isamarkupfalse%
\ sigma{\isacharunderscore}{\kern0pt}finite{\isacharunderscore}{\kern0pt}subalgebra{\isachardot}{\kern0pt}cond{\isacharunderscore}{\kern0pt}exp{\isacharunderscore}{\kern0pt}add{\isacharbrackleft}{\kern0pt}OF\ {\isacharunderscore}{\kern0pt}\ integrable\ martingale{\isachardot}{\kern0pt}integrable{\isacharbrackleft}{\kern0pt}OF\ assms{\isacharbrackright}{\kern0pt}{\isacharcomma}{\kern0pt}\ of\ {\isachardoublequoteopen}F\ i{\isachardoublequoteclose}\ j\ j{\isacharcomma}{\kern0pt}\ THEN\ AE{\isacharunderscore}{\kern0pt}symmetric{\isacharbrackright}{\kern0pt}\isanewline
\ \ \ \ \ \ \ \ \ \ \ \ martingale{\isacharunderscore}{\kern0pt}property{\isacharbrackleft}{\kern0pt}OF\ asm{\isacharbrackright}{\kern0pt}\ martingale{\isachardot}{\kern0pt}martingale{\isacharunderscore}{\kern0pt}property{\isacharbrackleft}{\kern0pt}OF\ assms\ asm{\isacharbrackright}{\kern0pt}\ \isacommand{by}\isamarkupfalse%
\ force\isanewline
\ \ \isacommand{{\isacharbraceright}{\kern0pt}}\isamarkupfalse%
\isanewline
\ \ \isacommand{thus}\isamarkupfalse%
\ {\isacharquery}{\kern0pt}thesis\ \isacommand{using}\isamarkupfalse%
\ assms\isanewline
\ \ \isacommand{by}\isamarkupfalse%
\ {\isacharparenleft}{\kern0pt}unfold{\isacharunderscore}{\kern0pt}locales{\isacharparenright}{\kern0pt}\ {\isacharparenleft}{\kern0pt}auto\ simp\ add{\isacharcolon}{\kern0pt}\ integrable\ martingale{\isachardot}{\kern0pt}integrable{\isacharparenright}{\kern0pt}\isanewline
\isacommand{qed}\isamarkupfalse%
%
\endisatagproof
{\isafoldproof}%
%
\isadelimproof
\isanewline
%
\endisadelimproof
\isanewline
\isacommand{lemma}\isamarkupfalse%
\ diff{\isacharbrackleft}{\kern0pt}intro{\isacharbrackright}{\kern0pt}{\isacharcolon}{\kern0pt}\isanewline
\ \ \isakeyword{assumes}\ {\isachardoublequoteopen}martingale\ M\ F\ t\isactrlsub {\isadigit{0}}\ Y{\isachardoublequoteclose}\isanewline
\ \ \isakeyword{shows}\ {\isachardoublequoteopen}martingale\ M\ F\ t\isactrlsub {\isadigit{0}}\ {\isacharparenleft}{\kern0pt}{\isasymlambda}i\ x{\isachardot}{\kern0pt}\ X\ i\ x\ {\isacharminus}{\kern0pt}\ Y\ i\ x{\isacharparenright}{\kern0pt}{\isachardoublequoteclose}\isanewline
%
\isadelimproof
%
\endisadelimproof
%
\isatagproof
\isacommand{proof}\isamarkupfalse%
\ {\isacharminus}{\kern0pt}\isanewline
\ \ \isacommand{interpret}\isamarkupfalse%
\ Y{\isacharcolon}{\kern0pt}\ martingale\ M\ F\ t\isactrlsub {\isadigit{0}}\ Y\ \isacommand{by}\isamarkupfalse%
\ {\isacharparenleft}{\kern0pt}rule\ assms{\isacharparenright}{\kern0pt}\isanewline
\ \ \isacommand{{\isacharbraceleft}{\kern0pt}}\isamarkupfalse%
\isanewline
\ \ \ \ \isacommand{fix}\isamarkupfalse%
\ i\ j\ {\isacharcolon}{\kern0pt}{\isacharcolon}{\kern0pt}\ {\isacharprime}{\kern0pt}b\ \isacommand{assume}\isamarkupfalse%
\ asm{\isacharcolon}{\kern0pt}\ {\isachardoublequoteopen}t\isactrlsub {\isadigit{0}}\ {\isasymle}\ i{\isachardoublequoteclose}\ {\isachardoublequoteopen}i\ {\isasymle}\ j{\isachardoublequoteclose}\isanewline
\ \ \ \ \isacommand{hence}\isamarkupfalse%
\ {\isachardoublequoteopen}AE\ {\isasymxi}\ in\ M{\isachardot}{\kern0pt}\ X\ i\ {\isasymxi}\ {\isacharminus}{\kern0pt}\ Y\ i\ {\isasymxi}\ {\isacharequal}{\kern0pt}\ cond{\isacharunderscore}{\kern0pt}exp\ M\ {\isacharparenleft}{\kern0pt}F\ i{\isacharparenright}{\kern0pt}\ {\isacharparenleft}{\kern0pt}{\isasymlambda}x{\isachardot}{\kern0pt}\ X\ j\ x\ {\isacharminus}{\kern0pt}\ Y\ j\ x{\isacharparenright}{\kern0pt}\ {\isasymxi}{\isachardoublequoteclose}\ \isanewline
\ \ \ \ \ \ \isacommand{using}\isamarkupfalse%
\ sigma{\isacharunderscore}{\kern0pt}finite{\isacharunderscore}{\kern0pt}subalgebra{\isachardot}{\kern0pt}cond{\isacharunderscore}{\kern0pt}exp{\isacharunderscore}{\kern0pt}diff{\isacharbrackleft}{\kern0pt}OF\ {\isacharunderscore}{\kern0pt}\ integrable\ martingale{\isachardot}{\kern0pt}integrable{\isacharbrackleft}{\kern0pt}OF\ assms{\isacharbrackright}{\kern0pt}{\isacharcomma}{\kern0pt}\ of\ {\isachardoublequoteopen}F\ i{\isachardoublequoteclose}\ j\ j{\isacharcomma}{\kern0pt}\ THEN\ AE{\isacharunderscore}{\kern0pt}symmetric{\isacharbrackright}{\kern0pt}\ \isanewline
\ \ \ \ \ \ \ \ \ \ \ \ martingale{\isacharunderscore}{\kern0pt}property{\isacharbrackleft}{\kern0pt}OF\ asm{\isacharbrackright}{\kern0pt}\ martingale{\isachardot}{\kern0pt}martingale{\isacharunderscore}{\kern0pt}property{\isacharbrackleft}{\kern0pt}OF\ assms\ asm{\isacharbrackright}{\kern0pt}\ \isacommand{by}\isamarkupfalse%
\ fastforce\isanewline
\ \ \isacommand{{\isacharbraceright}{\kern0pt}}\isamarkupfalse%
\isanewline
\ \ \isacommand{thus}\isamarkupfalse%
\ {\isacharquery}{\kern0pt}thesis\ \isacommand{using}\isamarkupfalse%
\ assms\ \isacommand{by}\isamarkupfalse%
\ {\isacharparenleft}{\kern0pt}unfold{\isacharunderscore}{\kern0pt}locales{\isacharparenright}{\kern0pt}\ {\isacharparenleft}{\kern0pt}auto\ simp\ add{\isacharcolon}{\kern0pt}\ integrable\ martingale{\isachardot}{\kern0pt}integrable{\isacharparenright}{\kern0pt}\ \ \isanewline
\isacommand{qed}\isamarkupfalse%
%
\endisatagproof
{\isafoldproof}%
%
\isadelimproof
\isanewline
%
\endisadelimproof
\isanewline
\isacommand{end}\isamarkupfalse%
\isanewline
\isanewline
\isacommand{lemma}\isamarkupfalse%
\ {\isacharparenleft}{\kern0pt}\isakeyword{in}\ sigma{\isacharunderscore}{\kern0pt}finite{\isacharunderscore}{\kern0pt}adapted{\isacharunderscore}{\kern0pt}process{\isacharparenright}{\kern0pt}\ martingale{\isacharunderscore}{\kern0pt}of{\isacharunderscore}{\kern0pt}cond{\isacharunderscore}{\kern0pt}exp{\isacharunderscore}{\kern0pt}diff{\isacharunderscore}{\kern0pt}eq{\isacharunderscore}{\kern0pt}zero{\isacharcolon}{\kern0pt}\ \isanewline
\ \ \isakeyword{assumes}\ integrable{\isacharcolon}{\kern0pt}\ {\isachardoublequoteopen}{\isasymAnd}i{\isachardot}{\kern0pt}\ t\isactrlsub {\isadigit{0}}\ {\isasymle}\ i\ {\isasymLongrightarrow}\ integrable\ M\ {\isacharparenleft}{\kern0pt}X\ i{\isacharparenright}{\kern0pt}{\isachardoublequoteclose}\ \isanewline
\ \ \ \ \ \ \isakeyword{and}\ diff{\isacharunderscore}{\kern0pt}zero{\isacharcolon}{\kern0pt}\ {\isachardoublequoteopen}{\isasymAnd}i\ j{\isachardot}{\kern0pt}\ t\isactrlsub {\isadigit{0}}\ {\isasymle}\ i\ {\isasymLongrightarrow}\ i\ {\isasymle}\ j\ {\isasymLongrightarrow}\ AE\ x\ in\ M{\isachardot}{\kern0pt}\ cond{\isacharunderscore}{\kern0pt}exp\ M\ {\isacharparenleft}{\kern0pt}F\ i{\isacharparenright}{\kern0pt}\ {\isacharparenleft}{\kern0pt}{\isasymlambda}{\isasymxi}{\isachardot}{\kern0pt}\ X\ j\ {\isasymxi}\ {\isacharminus}{\kern0pt}\ X\ i\ {\isasymxi}{\isacharparenright}{\kern0pt}\ x\ {\isacharequal}{\kern0pt}\ {\isadigit{0}}{\isachardoublequoteclose}\isanewline
\ \ \ \ \isakeyword{shows}\ {\isachardoublequoteopen}martingale\ M\ F\ t\isactrlsub {\isadigit{0}}\ X{\isachardoublequoteclose}\isanewline
%
\isadelimproof
%
\endisadelimproof
%
\isatagproof
\isacommand{proof}\isamarkupfalse%
\ \isanewline
\ \ \isacommand{{\isacharbraceleft}{\kern0pt}}\isamarkupfalse%
\isanewline
\ \ \ \ \isacommand{fix}\isamarkupfalse%
\ i\ j\ {\isacharcolon}{\kern0pt}{\isacharcolon}{\kern0pt}\ {\isacharprime}{\kern0pt}b\ \isacommand{assume}\isamarkupfalse%
\ asm{\isacharcolon}{\kern0pt}\ {\isachardoublequoteopen}t\isactrlsub {\isadigit{0}}\ {\isasymle}\ i{\isachardoublequoteclose}\ {\isachardoublequoteopen}i\ {\isasymle}\ j{\isachardoublequoteclose}\isanewline
\ \ \ \ \isacommand{thus}\isamarkupfalse%
\ {\isachardoublequoteopen}AE\ {\isasymxi}\ in\ M{\isachardot}{\kern0pt}\ X\ i\ {\isasymxi}\ {\isacharequal}{\kern0pt}\ cond{\isacharunderscore}{\kern0pt}exp\ M\ {\isacharparenleft}{\kern0pt}F\ i{\isacharparenright}{\kern0pt}\ {\isacharparenleft}{\kern0pt}X\ j{\isacharparenright}{\kern0pt}\ {\isasymxi}{\isachardoublequoteclose}\ \isanewline
\ \ \ \ \ \ \isacommand{using}\isamarkupfalse%
\ diff{\isacharunderscore}{\kern0pt}zero{\isacharbrackleft}{\kern0pt}OF\ asm{\isacharbrackright}{\kern0pt}\ sigma{\isacharunderscore}{\kern0pt}finite{\isacharunderscore}{\kern0pt}subalgebra{\isachardot}{\kern0pt}cond{\isacharunderscore}{\kern0pt}exp{\isacharunderscore}{\kern0pt}diff{\isacharbrackleft}{\kern0pt}OF\ {\isacharunderscore}{\kern0pt}\ integrable{\isacharparenleft}{\kern0pt}{\isadigit{1}}{\isacharcomma}{\kern0pt}{\isadigit{1}}{\isacharparenright}{\kern0pt}{\isacharcomma}{\kern0pt}\ of\ {\isachardoublequoteopen}F\ i{\isachardoublequoteclose}\ j\ i{\isacharbrackright}{\kern0pt}\ \isanewline
\ \ \ \ \ \ \ \ \ \ \ \ sigma{\isacharunderscore}{\kern0pt}finite{\isacharunderscore}{\kern0pt}subalgebra{\isachardot}{\kern0pt}cond{\isacharunderscore}{\kern0pt}exp{\isacharunderscore}{\kern0pt}F{\isacharunderscore}{\kern0pt}meas{\isacharbrackleft}{\kern0pt}OF\ {\isacharunderscore}{\kern0pt}\ integrable\ adapted{\isacharcomma}{\kern0pt}\ of\ i{\isacharbrackright}{\kern0pt}\ \isacommand{by}\isamarkupfalse%
\ fastforce\isanewline
\ \ \isacommand{{\isacharbraceright}{\kern0pt}}\isamarkupfalse%
\isanewline
\isacommand{qed}\isamarkupfalse%
\ {\isacharparenleft}{\kern0pt}intro\ integrable{\isacharparenright}{\kern0pt}%
\endisatagproof
{\isafoldproof}%
%
\isadelimproof
\isanewline
%
\endisadelimproof
\isanewline
\isacommand{lemma}\isamarkupfalse%
\ {\isacharparenleft}{\kern0pt}\isakeyword{in}\ sigma{\isacharunderscore}{\kern0pt}finite{\isacharunderscore}{\kern0pt}adapted{\isacharunderscore}{\kern0pt}process{\isacharparenright}{\kern0pt}\ martingale{\isacharunderscore}{\kern0pt}of{\isacharunderscore}{\kern0pt}set{\isacharunderscore}{\kern0pt}integral{\isacharunderscore}{\kern0pt}eq{\isacharcolon}{\kern0pt}\isanewline
\ \ \isakeyword{assumes}\ integrable{\isacharcolon}{\kern0pt}\ {\isachardoublequoteopen}{\isasymAnd}i{\isachardot}{\kern0pt}\ t\isactrlsub {\isadigit{0}}\ {\isasymle}\ i\ {\isasymLongrightarrow}\ integrable\ M\ {\isacharparenleft}{\kern0pt}X\ i{\isacharparenright}{\kern0pt}{\isachardoublequoteclose}\isanewline
\ \ \ \ \ \ \isakeyword{and}\ {\isachardoublequoteopen}{\isasymAnd}A\ i\ j{\isachardot}{\kern0pt}\ t\isactrlsub {\isadigit{0}}\ {\isasymle}\ i\ {\isasymLongrightarrow}\ i\ {\isasymle}\ j\ {\isasymLongrightarrow}\ A\ {\isasymin}\ F\ i\ {\isasymLongrightarrow}\ set{\isacharunderscore}{\kern0pt}lebesgue{\isacharunderscore}{\kern0pt}integral\ M\ A\ {\isacharparenleft}{\kern0pt}X\ i{\isacharparenright}{\kern0pt}\ {\isacharequal}{\kern0pt}\ set{\isacharunderscore}{\kern0pt}lebesgue{\isacharunderscore}{\kern0pt}integral\ M\ A\ {\isacharparenleft}{\kern0pt}X\ j{\isacharparenright}{\kern0pt}{\isachardoublequoteclose}\ \isanewline
\ \ \ \ \isakeyword{shows}\ {\isachardoublequoteopen}martingale\ M\ F\ t\isactrlsub {\isadigit{0}}\ X{\isachardoublequoteclose}\isanewline
%
\isadelimproof
%
\endisadelimproof
%
\isatagproof
\isacommand{proof}\isamarkupfalse%
\ {\isacharparenleft}{\kern0pt}unfold{\isacharunderscore}{\kern0pt}locales{\isacharparenright}{\kern0pt}\isanewline
\ \ \isacommand{fix}\isamarkupfalse%
\ i\ j\ {\isacharcolon}{\kern0pt}{\isacharcolon}{\kern0pt}\ {\isacharprime}{\kern0pt}b\ \isacommand{assume}\isamarkupfalse%
\ asm{\isacharcolon}{\kern0pt}\ {\isachardoublequoteopen}t\isactrlsub {\isadigit{0}}\ {\isasymle}\ i{\isachardoublequoteclose}\ {\isachardoublequoteopen}i\ {\isasymle}\ j{\isachardoublequoteclose}\isanewline
\ \ \isacommand{interpret}\isamarkupfalse%
\ sigma{\isacharunderscore}{\kern0pt}finite{\isacharunderscore}{\kern0pt}subalgebra\ M\ {\isachardoublequoteopen}F\ i{\isachardoublequoteclose}\ \isacommand{using}\isamarkupfalse%
\ asm\ \isacommand{by}\isamarkupfalse%
\ blast\isanewline
\ \ \isacommand{interpret}\isamarkupfalse%
\ r{\isacharcolon}{\kern0pt}\ sigma{\isacharunderscore}{\kern0pt}finite{\isacharunderscore}{\kern0pt}measure\ {\isachardoublequoteopen}restr{\isacharunderscore}{\kern0pt}to{\isacharunderscore}{\kern0pt}subalg\ M\ {\isacharparenleft}{\kern0pt}F\ i{\isacharparenright}{\kern0pt}{\isachardoublequoteclose}\ \isacommand{by}\isamarkupfalse%
\ {\isacharparenleft}{\kern0pt}simp\ add{\isacharcolon}{\kern0pt}\ sigma{\isacharunderscore}{\kern0pt}fin{\isacharunderscore}{\kern0pt}subalg{\isacharparenright}{\kern0pt}\isanewline
\ \ \isacommand{{\isacharbraceleft}{\kern0pt}}\isamarkupfalse%
\isanewline
\ \ \ \ \isacommand{fix}\isamarkupfalse%
\ A\ \isacommand{assume}\isamarkupfalse%
\ {\isachardoublequoteopen}A\ {\isasymin}\ restr{\isacharunderscore}{\kern0pt}to{\isacharunderscore}{\kern0pt}subalg\ M\ {\isacharparenleft}{\kern0pt}F\ i{\isacharparenright}{\kern0pt}{\isachardoublequoteclose}\isanewline
\ \ \ \ \isacommand{hence}\isamarkupfalse%
\ {\isacharasterisk}{\kern0pt}{\isacharcolon}{\kern0pt}\ {\isachardoublequoteopen}A\ {\isasymin}\ F\ i{\isachardoublequoteclose}\ \isacommand{using}\isamarkupfalse%
\ sets{\isacharunderscore}{\kern0pt}restr{\isacharunderscore}{\kern0pt}to{\isacharunderscore}{\kern0pt}subalg\ subalgebras\ asm\ \isacommand{by}\isamarkupfalse%
\ blast\ \isanewline
\ \ \ \ \isacommand{have}\isamarkupfalse%
\ {\isachardoublequoteopen}set{\isacharunderscore}{\kern0pt}lebesgue{\isacharunderscore}{\kern0pt}integral\ {\isacharparenleft}{\kern0pt}restr{\isacharunderscore}{\kern0pt}to{\isacharunderscore}{\kern0pt}subalg\ M\ {\isacharparenleft}{\kern0pt}F\ i{\isacharparenright}{\kern0pt}{\isacharparenright}{\kern0pt}\ A\ {\isacharparenleft}{\kern0pt}X\ i{\isacharparenright}{\kern0pt}\ {\isacharequal}{\kern0pt}\ set{\isacharunderscore}{\kern0pt}lebesgue{\isacharunderscore}{\kern0pt}integral\ M\ A\ {\isacharparenleft}{\kern0pt}X\ i{\isacharparenright}{\kern0pt}{\isachardoublequoteclose}\ \isacommand{using}\isamarkupfalse%
\ {\isacharasterisk}{\kern0pt}\ subalg\ asm\ \isacommand{by}\isamarkupfalse%
\ {\isacharparenleft}{\kern0pt}auto\ simp{\isacharcolon}{\kern0pt}\ set{\isacharunderscore}{\kern0pt}lebesgue{\isacharunderscore}{\kern0pt}integral{\isacharunderscore}{\kern0pt}def\ intro{\isacharcolon}{\kern0pt}\ integral{\isacharunderscore}{\kern0pt}subalgebra{\isadigit{2}}\ borel{\isacharunderscore}{\kern0pt}measurable{\isacharunderscore}{\kern0pt}scaleR\ adapted\ borel{\isacharunderscore}{\kern0pt}measurable{\isacharunderscore}{\kern0pt}indicator{\isacharparenright}{\kern0pt}\ \isanewline
\ \ \ \ \isacommand{also}\isamarkupfalse%
\ \isacommand{have}\isamarkupfalse%
\ {\isachardoublequoteopen}{\isachardot}{\kern0pt}{\isachardot}{\kern0pt}{\isachardot}{\kern0pt}\ {\isacharequal}{\kern0pt}\ set{\isacharunderscore}{\kern0pt}lebesgue{\isacharunderscore}{\kern0pt}integral\ M\ A\ {\isacharparenleft}{\kern0pt}cond{\isacharunderscore}{\kern0pt}exp\ M\ {\isacharparenleft}{\kern0pt}F\ i{\isacharparenright}{\kern0pt}\ {\isacharparenleft}{\kern0pt}X\ j{\isacharparenright}{\kern0pt}{\isacharparenright}{\kern0pt}{\isachardoublequoteclose}\ \isacommand{using}\isamarkupfalse%
\ {\isacharasterisk}{\kern0pt}\ assms{\isacharparenleft}{\kern0pt}{\isadigit{2}}{\isacharparenright}{\kern0pt}{\isacharbrackleft}{\kern0pt}OF\ asm{\isacharbrackright}{\kern0pt}\ cond{\isacharunderscore}{\kern0pt}exp{\isacharunderscore}{\kern0pt}set{\isacharunderscore}{\kern0pt}integral{\isacharbrackleft}{\kern0pt}OF\ integrable{\isacharbrackright}{\kern0pt}\ asm\ \isacommand{by}\isamarkupfalse%
\ auto\isanewline
\ \ \ \ \isacommand{finally}\isamarkupfalse%
\ \isacommand{have}\isamarkupfalse%
\ {\isachardoublequoteopen}set{\isacharunderscore}{\kern0pt}lebesgue{\isacharunderscore}{\kern0pt}integral\ {\isacharparenleft}{\kern0pt}restr{\isacharunderscore}{\kern0pt}to{\isacharunderscore}{\kern0pt}subalg\ M\ {\isacharparenleft}{\kern0pt}F\ i{\isacharparenright}{\kern0pt}{\isacharparenright}{\kern0pt}\ A\ {\isacharparenleft}{\kern0pt}X\ i{\isacharparenright}{\kern0pt}\ {\isacharequal}{\kern0pt}\ set{\isacharunderscore}{\kern0pt}lebesgue{\isacharunderscore}{\kern0pt}integral\ {\isacharparenleft}{\kern0pt}restr{\isacharunderscore}{\kern0pt}to{\isacharunderscore}{\kern0pt}subalg\ M\ {\isacharparenleft}{\kern0pt}F\ i{\isacharparenright}{\kern0pt}{\isacharparenright}{\kern0pt}\ A\ {\isacharparenleft}{\kern0pt}cond{\isacharunderscore}{\kern0pt}exp\ M\ {\isacharparenleft}{\kern0pt}F\ i{\isacharparenright}{\kern0pt}\ {\isacharparenleft}{\kern0pt}X\ j{\isacharparenright}{\kern0pt}{\isacharparenright}{\kern0pt}{\isachardoublequoteclose}\ \isacommand{using}\isamarkupfalse%
\ {\isacharasterisk}{\kern0pt}\ subalg\ \isacommand{by}\isamarkupfalse%
\ {\isacharparenleft}{\kern0pt}auto\ simp{\isacharcolon}{\kern0pt}\ set{\isacharunderscore}{\kern0pt}lebesgue{\isacharunderscore}{\kern0pt}integral{\isacharunderscore}{\kern0pt}def\ intro{\isacharbang}{\kern0pt}{\isacharcolon}{\kern0pt}\ integral{\isacharunderscore}{\kern0pt}subalgebra{\isadigit{2}}{\isacharbrackleft}{\kern0pt}symmetric{\isacharbrackright}{\kern0pt}\ borel{\isacharunderscore}{\kern0pt}measurable{\isacharunderscore}{\kern0pt}scaleR\ borel{\isacharunderscore}{\kern0pt}measurable{\isacharunderscore}{\kern0pt}cond{\isacharunderscore}{\kern0pt}exp\ borel{\isacharunderscore}{\kern0pt}measurable{\isacharunderscore}{\kern0pt}indicator{\isacharparenright}{\kern0pt}\isanewline
\ \ \isacommand{{\isacharbraceright}{\kern0pt}}\isamarkupfalse%
\isanewline
\ \ \isacommand{hence}\isamarkupfalse%
\ {\isachardoublequoteopen}AE\ {\isasymxi}\ in\ restr{\isacharunderscore}{\kern0pt}to{\isacharunderscore}{\kern0pt}subalg\ M\ {\isacharparenleft}{\kern0pt}F\ i{\isacharparenright}{\kern0pt}{\isachardot}{\kern0pt}\ X\ i\ {\isasymxi}\ {\isacharequal}{\kern0pt}\ cond{\isacharunderscore}{\kern0pt}exp\ M\ {\isacharparenleft}{\kern0pt}F\ i{\isacharparenright}{\kern0pt}\ {\isacharparenleft}{\kern0pt}X\ j{\isacharparenright}{\kern0pt}\ {\isasymxi}{\isachardoublequoteclose}\ \isacommand{using}\isamarkupfalse%
\ asm\ \isacommand{by}\isamarkupfalse%
\ {\isacharparenleft}{\kern0pt}intro\ r{\isachardot}{\kern0pt}density{\isacharunderscore}{\kern0pt}unique{\isacharunderscore}{\kern0pt}banach{\isacharcomma}{\kern0pt}\ auto\ intro{\isacharcolon}{\kern0pt}\ integrable{\isacharunderscore}{\kern0pt}in{\isacharunderscore}{\kern0pt}subalg\ subalg\ borel{\isacharunderscore}{\kern0pt}measurable{\isacharunderscore}{\kern0pt}cond{\isacharunderscore}{\kern0pt}exp\ integrable{\isacharparenright}{\kern0pt}\isanewline
\ \ \isacommand{thus}\isamarkupfalse%
\ {\isachardoublequoteopen}AE\ {\isasymxi}\ in\ M{\isachardot}{\kern0pt}\ X\ i\ {\isasymxi}\ {\isacharequal}{\kern0pt}\ cond{\isacharunderscore}{\kern0pt}exp\ M\ {\isacharparenleft}{\kern0pt}F\ i{\isacharparenright}{\kern0pt}\ {\isacharparenleft}{\kern0pt}X\ j{\isacharparenright}{\kern0pt}\ {\isasymxi}{\isachardoublequoteclose}\ \isacommand{using}\isamarkupfalse%
\ AE{\isacharunderscore}{\kern0pt}restr{\isacharunderscore}{\kern0pt}to{\isacharunderscore}{\kern0pt}subalg{\isacharbrackleft}{\kern0pt}OF\ subalg{\isacharbrackright}{\kern0pt}\ \isacommand{by}\isamarkupfalse%
\ blast\isanewline
\isacommand{qed}\isamarkupfalse%
\ {\isacharparenleft}{\kern0pt}simp\ add{\isacharcolon}{\kern0pt}\ integrable{\isacharparenright}{\kern0pt}%
\endisatagproof
{\isafoldproof}%
%
\isadelimproof
%
\endisadelimproof
%
\isadelimdocument
%
\endisadelimdocument
%
\isatagdocument
%
\isamarkupsubsection{Submartingale Lemmas%
}
\isamarkuptrue%
%
\endisatagdocument
{\isafolddocument}%
%
\isadelimdocument
%
\endisadelimdocument
\isacommand{context}\isamarkupfalse%
\ submartingale\isanewline
\isakeyword{begin}\isanewline
\isanewline
\isacommand{lemma}\isamarkupfalse%
\ cond{\isacharunderscore}{\kern0pt}exp{\isacharunderscore}{\kern0pt}diff{\isacharunderscore}{\kern0pt}nonneg{\isacharcolon}{\kern0pt}\isanewline
\ \ \isakeyword{assumes}\ {\isachardoublequoteopen}t\isactrlsub {\isadigit{0}}\ {\isasymle}\ i{\isachardoublequoteclose}\ {\isachardoublequoteopen}i\ {\isasymle}\ j{\isachardoublequoteclose}\isanewline
\ \ \isakeyword{shows}\ {\isachardoublequoteopen}AE\ x\ in\ M{\isachardot}{\kern0pt}\ cond{\isacharunderscore}{\kern0pt}exp\ M\ {\isacharparenleft}{\kern0pt}F\ i{\isacharparenright}{\kern0pt}\ {\isacharparenleft}{\kern0pt}{\isasymlambda}{\isasymxi}{\isachardot}{\kern0pt}\ X\ j\ {\isasymxi}\ {\isacharminus}{\kern0pt}\ X\ i\ {\isasymxi}{\isacharparenright}{\kern0pt}\ x\ {\isasymge}\ {\isadigit{0}}{\isachardoublequoteclose}\isanewline
%
\isadelimproof
\ \ %
\endisadelimproof
%
\isatagproof
\isacommand{using}\isamarkupfalse%
\ submartingale{\isacharunderscore}{\kern0pt}property{\isacharbrackleft}{\kern0pt}OF\ assms{\isacharbrackright}{\kern0pt}\ assms\ sigma{\isacharunderscore}{\kern0pt}finite{\isacharunderscore}{\kern0pt}subalgebra{\isachardot}{\kern0pt}cond{\isacharunderscore}{\kern0pt}exp{\isacharunderscore}{\kern0pt}diff{\isacharbrackleft}{\kern0pt}OF\ {\isacharunderscore}{\kern0pt}\ integrable{\isacharparenleft}{\kern0pt}{\isadigit{1}}{\isacharcomma}{\kern0pt}{\isadigit{1}}{\isacharparenright}{\kern0pt}{\isacharcomma}{\kern0pt}\ of\ {\isacharunderscore}{\kern0pt}\ j\ i{\isacharbrackright}{\kern0pt}\ sigma{\isacharunderscore}{\kern0pt}finite{\isacharunderscore}{\kern0pt}subalgebra{\isachardot}{\kern0pt}cond{\isacharunderscore}{\kern0pt}exp{\isacharunderscore}{\kern0pt}F{\isacharunderscore}{\kern0pt}meas{\isacharbrackleft}{\kern0pt}OF\ {\isacharunderscore}{\kern0pt}\ integrable\ adapted{\isacharcomma}{\kern0pt}\ of\ i{\isacharbrackright}{\kern0pt}\ \isacommand{by}\isamarkupfalse%
\ fastforce%
\endisatagproof
{\isafoldproof}%
%
\isadelimproof
\isanewline
%
\endisadelimproof
\isanewline
\isacommand{lemma}\isamarkupfalse%
\ add{\isacharbrackleft}{\kern0pt}intro{\isacharbrackright}{\kern0pt}{\isacharcolon}{\kern0pt}\isanewline
\ \ \isakeyword{assumes}\ {\isachardoublequoteopen}submartingale\ M\ F\ t\isactrlsub {\isadigit{0}}\ Y{\isachardoublequoteclose}\isanewline
\ \ \isakeyword{shows}\ {\isachardoublequoteopen}submartingale\ M\ F\ t\isactrlsub {\isadigit{0}}\ {\isacharparenleft}{\kern0pt}{\isasymlambda}i\ {\isasymxi}{\isachardot}{\kern0pt}\ X\ i\ {\isasymxi}\ {\isacharplus}{\kern0pt}\ Y\ i\ {\isasymxi}{\isacharparenright}{\kern0pt}{\isachardoublequoteclose}\isanewline
%
\isadelimproof
%
\endisadelimproof
%
\isatagproof
\isacommand{proof}\isamarkupfalse%
\ {\isacharminus}{\kern0pt}\isanewline
\ \ \isacommand{interpret}\isamarkupfalse%
\ Y{\isacharcolon}{\kern0pt}\ submartingale\ M\ F\ t\isactrlsub {\isadigit{0}}\ Y\ \isacommand{by}\isamarkupfalse%
\ {\isacharparenleft}{\kern0pt}rule\ assms{\isacharparenright}{\kern0pt}\isanewline
\ \ \isacommand{{\isacharbraceleft}{\kern0pt}}\isamarkupfalse%
\isanewline
\ \ \ \ \isacommand{fix}\isamarkupfalse%
\ i\ j\ {\isacharcolon}{\kern0pt}{\isacharcolon}{\kern0pt}\ {\isacharprime}{\kern0pt}b\ \isacommand{assume}\isamarkupfalse%
\ asm{\isacharcolon}{\kern0pt}\ {\isachardoublequoteopen}t\isactrlsub {\isadigit{0}}\ {\isasymle}\ i{\isachardoublequoteclose}\ {\isachardoublequoteopen}i\ {\isasymle}\ j{\isachardoublequoteclose}\isanewline
\ \ \ \ \isacommand{hence}\isamarkupfalse%
\ {\isachardoublequoteopen}AE\ {\isasymxi}\ in\ M{\isachardot}{\kern0pt}\ X\ i\ {\isasymxi}\ {\isacharplus}{\kern0pt}\ Y\ i\ {\isasymxi}\ {\isasymle}\ cond{\isacharunderscore}{\kern0pt}exp\ M\ {\isacharparenleft}{\kern0pt}F\ i{\isacharparenright}{\kern0pt}\ {\isacharparenleft}{\kern0pt}{\isasymlambda}x{\isachardot}{\kern0pt}\ X\ j\ x\ {\isacharplus}{\kern0pt}\ Y\ j\ x{\isacharparenright}{\kern0pt}\ {\isasymxi}{\isachardoublequoteclose}\ \isanewline
\ \ \ \ \ \ \isacommand{using}\isamarkupfalse%
\ sigma{\isacharunderscore}{\kern0pt}finite{\isacharunderscore}{\kern0pt}subalgebra{\isachardot}{\kern0pt}cond{\isacharunderscore}{\kern0pt}exp{\isacharunderscore}{\kern0pt}add{\isacharbrackleft}{\kern0pt}OF\ {\isacharunderscore}{\kern0pt}\ integrable\ submartingale{\isachardot}{\kern0pt}integrable{\isacharbrackleft}{\kern0pt}OF\ assms{\isacharbrackright}{\kern0pt}{\isacharcomma}{\kern0pt}\ of\ {\isachardoublequoteopen}F\ i{\isachardoublequoteclose}\ j\ j{\isacharbrackright}{\kern0pt}\ \isanewline
\ \ \ \ \ \ \ \ \ \ \ \ submartingale{\isacharunderscore}{\kern0pt}property{\isacharbrackleft}{\kern0pt}OF\ asm{\isacharbrackright}{\kern0pt}\ submartingale{\isachardot}{\kern0pt}submartingale{\isacharunderscore}{\kern0pt}property{\isacharbrackleft}{\kern0pt}OF\ assms\ asm{\isacharbrackright}{\kern0pt}\ add{\isacharunderscore}{\kern0pt}mono{\isacharbrackleft}{\kern0pt}of\ {\isachardoublequoteopen}X\ i\ {\isacharunderscore}{\kern0pt}{\isachardoublequoteclose}\ {\isacharunderscore}{\kern0pt}\ {\isachardoublequoteopen}Y\ i\ {\isacharunderscore}{\kern0pt}{\isachardoublequoteclose}{\isacharbrackright}{\kern0pt}\ \isacommand{by}\isamarkupfalse%
\ force\isanewline
\ \ \isacommand{{\isacharbraceright}{\kern0pt}}\isamarkupfalse%
\isanewline
\ \ \isacommand{thus}\isamarkupfalse%
\ {\isacharquery}{\kern0pt}thesis\ \isacommand{using}\isamarkupfalse%
\ assms\ \isacommand{by}\isamarkupfalse%
\ {\isacharparenleft}{\kern0pt}unfold{\isacharunderscore}{\kern0pt}locales{\isacharparenright}{\kern0pt}\ {\isacharparenleft}{\kern0pt}auto\ simp\ add{\isacharcolon}{\kern0pt}\ borel{\isacharunderscore}{\kern0pt}measurable{\isacharunderscore}{\kern0pt}add\ random{\isacharunderscore}{\kern0pt}variable\ adapted\ integrable\ Y{\isachardot}{\kern0pt}random{\isacharunderscore}{\kern0pt}variable\ Y{\isachardot}{\kern0pt}adapted\ submartingale{\isachardot}{\kern0pt}integrable{\isacharparenright}{\kern0pt}\ \ \isanewline
\isacommand{qed}\isamarkupfalse%
%
\endisatagproof
{\isafoldproof}%
%
\isadelimproof
\isanewline
%
\endisadelimproof
\isanewline
\isacommand{lemma}\isamarkupfalse%
\ diff{\isacharbrackleft}{\kern0pt}intro{\isacharbrackright}{\kern0pt}{\isacharcolon}{\kern0pt}\isanewline
\ \ \isakeyword{assumes}\ {\isachardoublequoteopen}supermartingale\ M\ F\ t\isactrlsub {\isadigit{0}}\ Y{\isachardoublequoteclose}\isanewline
\ \ \isakeyword{shows}\ {\isachardoublequoteopen}submartingale\ M\ F\ t\isactrlsub {\isadigit{0}}\ {\isacharparenleft}{\kern0pt}{\isasymlambda}i\ {\isasymxi}{\isachardot}{\kern0pt}\ X\ i\ {\isasymxi}\ {\isacharminus}{\kern0pt}\ Y\ i\ {\isasymxi}{\isacharparenright}{\kern0pt}{\isachardoublequoteclose}\isanewline
%
\isadelimproof
%
\endisadelimproof
%
\isatagproof
\isacommand{proof}\isamarkupfalse%
\ {\isacharminus}{\kern0pt}\isanewline
\ \ \isacommand{interpret}\isamarkupfalse%
\ Y{\isacharcolon}{\kern0pt}\ supermartingale\ M\ F\ t\isactrlsub {\isadigit{0}}\ Y\ \isacommand{by}\isamarkupfalse%
\ {\isacharparenleft}{\kern0pt}rule\ assms{\isacharparenright}{\kern0pt}\isanewline
\ \ \isacommand{{\isacharbraceleft}{\kern0pt}}\isamarkupfalse%
\isanewline
\ \ \ \ \isacommand{fix}\isamarkupfalse%
\ i\ j\ {\isacharcolon}{\kern0pt}{\isacharcolon}{\kern0pt}\ {\isacharprime}{\kern0pt}b\ \isacommand{assume}\isamarkupfalse%
\ asm{\isacharcolon}{\kern0pt}\ {\isachardoublequoteopen}t\isactrlsub {\isadigit{0}}\ {\isasymle}\ i{\isachardoublequoteclose}\ {\isachardoublequoteopen}i\ {\isasymle}\ j{\isachardoublequoteclose}\isanewline
\ \ \ \ \isacommand{hence}\isamarkupfalse%
\ {\isachardoublequoteopen}AE\ {\isasymxi}\ in\ M{\isachardot}{\kern0pt}\ X\ i\ {\isasymxi}\ {\isacharminus}{\kern0pt}\ Y\ i\ {\isasymxi}\ {\isasymle}\ cond{\isacharunderscore}{\kern0pt}exp\ M\ {\isacharparenleft}{\kern0pt}F\ i{\isacharparenright}{\kern0pt}\ {\isacharparenleft}{\kern0pt}{\isasymlambda}x{\isachardot}{\kern0pt}\ X\ j\ x\ {\isacharminus}{\kern0pt}\ Y\ j\ x{\isacharparenright}{\kern0pt}\ {\isasymxi}{\isachardoublequoteclose}\ \isanewline
\ \ \ \ \ \ \isacommand{using}\isamarkupfalse%
\ sigma{\isacharunderscore}{\kern0pt}finite{\isacharunderscore}{\kern0pt}subalgebra{\isachardot}{\kern0pt}cond{\isacharunderscore}{\kern0pt}exp{\isacharunderscore}{\kern0pt}diff{\isacharbrackleft}{\kern0pt}OF\ {\isacharunderscore}{\kern0pt}\ integrable\ supermartingale{\isachardot}{\kern0pt}integrable{\isacharbrackleft}{\kern0pt}OF\ assms{\isacharbrackright}{\kern0pt}{\isacharcomma}{\kern0pt}\ of\ {\isachardoublequoteopen}F\ i{\isachardoublequoteclose}\ j\ j{\isacharbrackright}{\kern0pt}\ \isanewline
\ \ \ \ \ \ \ \ \ \ \ \ submartingale{\isacharunderscore}{\kern0pt}property{\isacharbrackleft}{\kern0pt}OF\ asm{\isacharbrackright}{\kern0pt}\ supermartingale{\isachardot}{\kern0pt}supermartingale{\isacharunderscore}{\kern0pt}property{\isacharbrackleft}{\kern0pt}OF\ assms\ asm{\isacharbrackright}{\kern0pt}\ diff{\isacharunderscore}{\kern0pt}mono{\isacharbrackleft}{\kern0pt}of\ {\isachardoublequoteopen}X\ i\ {\isacharunderscore}{\kern0pt}{\isachardoublequoteclose}\ {\isacharunderscore}{\kern0pt}\ {\isacharunderscore}{\kern0pt}\ {\isachardoublequoteopen}Y\ i\ {\isacharunderscore}{\kern0pt}{\isachardoublequoteclose}{\isacharbrackright}{\kern0pt}\ \isacommand{by}\isamarkupfalse%
\ force\isanewline
\ \ \isacommand{{\isacharbraceright}{\kern0pt}}\isamarkupfalse%
\isanewline
\ \ \isacommand{thus}\isamarkupfalse%
\ {\isacharquery}{\kern0pt}thesis\ \isacommand{using}\isamarkupfalse%
\ assms\ \isacommand{by}\isamarkupfalse%
\ {\isacharparenleft}{\kern0pt}unfold{\isacharunderscore}{\kern0pt}locales{\isacharparenright}{\kern0pt}\ {\isacharparenleft}{\kern0pt}auto\ simp\ add{\isacharcolon}{\kern0pt}\ borel{\isacharunderscore}{\kern0pt}measurable{\isacharunderscore}{\kern0pt}diff\ random{\isacharunderscore}{\kern0pt}variable\ adapted\ integrable\ Y{\isachardot}{\kern0pt}random{\isacharunderscore}{\kern0pt}variable\ Y{\isachardot}{\kern0pt}adapted\ supermartingale{\isachardot}{\kern0pt}integrable{\isacharparenright}{\kern0pt}\ \ \isanewline
\isacommand{qed}\isamarkupfalse%
%
\endisatagproof
{\isafoldproof}%
%
\isadelimproof
\isanewline
%
\endisadelimproof
\isanewline
\isacommand{lemma}\isamarkupfalse%
\ scaleR{\isacharunderscore}{\kern0pt}nonneg{\isacharcolon}{\kern0pt}\ \isanewline
\ \ \isakeyword{assumes}\ {\isachardoublequoteopen}c\ {\isasymge}\ {\isadigit{0}}{\isachardoublequoteclose}\isanewline
\ \ \isakeyword{shows}\ {\isachardoublequoteopen}submartingale\ M\ F\ t\isactrlsub {\isadigit{0}}\ {\isacharparenleft}{\kern0pt}{\isasymlambda}i\ {\isasymxi}{\isachardot}{\kern0pt}\ c\ {\isacharasterisk}{\kern0pt}\isactrlsub R\ X\ i\ {\isasymxi}{\isacharparenright}{\kern0pt}{\isachardoublequoteclose}\isanewline
%
\isadelimproof
%
\endisadelimproof
%
\isatagproof
\isacommand{proof}\isamarkupfalse%
\isanewline
\ \ \isacommand{{\isacharbraceleft}{\kern0pt}}\isamarkupfalse%
\isanewline
\ \ \ \ \isacommand{fix}\isamarkupfalse%
\ i\ j\ {\isacharcolon}{\kern0pt}{\isacharcolon}{\kern0pt}\ {\isacharprime}{\kern0pt}b\ \isacommand{assume}\isamarkupfalse%
\ asm{\isacharcolon}{\kern0pt}\ {\isachardoublequoteopen}t\isactrlsub {\isadigit{0}}\ {\isasymle}\ i{\isachardoublequoteclose}\ {\isachardoublequoteopen}i\ {\isasymle}\ j{\isachardoublequoteclose}\isanewline
\ \ \ \ \isacommand{thus}\isamarkupfalse%
\ {\isachardoublequoteopen}AE\ {\isasymxi}\ in\ M{\isachardot}{\kern0pt}\ c\ {\isacharasterisk}{\kern0pt}\isactrlsub R\ X\ i\ {\isasymxi}\ {\isasymle}\ cond{\isacharunderscore}{\kern0pt}exp\ M\ {\isacharparenleft}{\kern0pt}F\ i{\isacharparenright}{\kern0pt}\ {\isacharparenleft}{\kern0pt}{\isasymlambda}{\isasymxi}{\isachardot}{\kern0pt}\ c\ {\isacharasterisk}{\kern0pt}\isactrlsub R\ X\ j\ {\isasymxi}{\isacharparenright}{\kern0pt}\ {\isasymxi}{\isachardoublequoteclose}\ \ \isanewline
\ \ \ \ \ \ \isacommand{using}\isamarkupfalse%
\ sigma{\isacharunderscore}{\kern0pt}finite{\isacharunderscore}{\kern0pt}subalgebra{\isachardot}{\kern0pt}cond{\isacharunderscore}{\kern0pt}exp{\isacharunderscore}{\kern0pt}scaleR{\isacharunderscore}{\kern0pt}right{\isacharbrackleft}{\kern0pt}OF\ {\isacharunderscore}{\kern0pt}\ integrable{\isacharcomma}{\kern0pt}\ of\ {\isachardoublequoteopen}F\ i{\isachardoublequoteclose}\ j\ c{\isacharbrackright}{\kern0pt}\ submartingale{\isacharunderscore}{\kern0pt}property{\isacharbrackleft}{\kern0pt}OF\ asm{\isacharbrackright}{\kern0pt}\ \isacommand{by}\isamarkupfalse%
\ {\isacharparenleft}{\kern0pt}fastforce\ intro{\isacharbang}{\kern0pt}{\isacharcolon}{\kern0pt}\ scaleR{\isacharunderscore}{\kern0pt}left{\isacharunderscore}{\kern0pt}mono{\isacharbrackleft}{\kern0pt}OF\ {\isacharunderscore}{\kern0pt}\ assms{\isacharbrackright}{\kern0pt}{\isacharparenright}{\kern0pt}\isanewline
\ \ \isacommand{{\isacharbraceright}{\kern0pt}}\isamarkupfalse%
\isanewline
\isacommand{qed}\isamarkupfalse%
\ {\isacharparenleft}{\kern0pt}auto\ simp\ add{\isacharcolon}{\kern0pt}\ borel{\isacharunderscore}{\kern0pt}measurable{\isacharunderscore}{\kern0pt}integrable\ borel{\isacharunderscore}{\kern0pt}measurable{\isacharunderscore}{\kern0pt}scaleR\ integrable\ random{\isacharunderscore}{\kern0pt}variable\ adapted\ borel{\isacharunderscore}{\kern0pt}measurable{\isacharunderscore}{\kern0pt}const{\isacharunderscore}{\kern0pt}scaleR{\isacharparenright}{\kern0pt}%
\endisatagproof
{\isafoldproof}%
%
\isadelimproof
\isanewline
%
\endisadelimproof
\isanewline
\isacommand{lemma}\isamarkupfalse%
\ scaleR{\isacharunderscore}{\kern0pt}le{\isacharunderscore}{\kern0pt}zero{\isacharcolon}{\kern0pt}\ \isanewline
\ \ \isakeyword{assumes}\ {\isachardoublequoteopen}c\ {\isasymle}\ {\isadigit{0}}{\isachardoublequoteclose}\isanewline
\ \ \isakeyword{shows}\ {\isachardoublequoteopen}supermartingale\ M\ F\ t\isactrlsub {\isadigit{0}}\ {\isacharparenleft}{\kern0pt}{\isasymlambda}i\ {\isasymxi}{\isachardot}{\kern0pt}\ c\ {\isacharasterisk}{\kern0pt}\isactrlsub R\ X\ i\ {\isasymxi}{\isacharparenright}{\kern0pt}{\isachardoublequoteclose}\isanewline
%
\isadelimproof
%
\endisadelimproof
%
\isatagproof
\isacommand{proof}\isamarkupfalse%
\isanewline
\ \ \isacommand{{\isacharbraceleft}{\kern0pt}}\isamarkupfalse%
\isanewline
\ \ \ \ \isacommand{fix}\isamarkupfalse%
\ i\ j\ {\isacharcolon}{\kern0pt}{\isacharcolon}{\kern0pt}\ {\isacharprime}{\kern0pt}b\ \isacommand{assume}\isamarkupfalse%
\ asm{\isacharcolon}{\kern0pt}\ {\isachardoublequoteopen}t\isactrlsub {\isadigit{0}}\ {\isasymle}\ i{\isachardoublequoteclose}\ {\isachardoublequoteopen}i\ {\isasymle}\ j{\isachardoublequoteclose}\isanewline
\ \ \ \ \isacommand{thus}\isamarkupfalse%
\ {\isachardoublequoteopen}AE\ {\isasymxi}\ in\ M{\isachardot}{\kern0pt}\ c\ {\isacharasterisk}{\kern0pt}\isactrlsub R\ X\ i\ {\isasymxi}\ {\isasymge}\ cond{\isacharunderscore}{\kern0pt}exp\ M\ {\isacharparenleft}{\kern0pt}F\ i{\isacharparenright}{\kern0pt}\ {\isacharparenleft}{\kern0pt}{\isasymlambda}{\isasymxi}{\isachardot}{\kern0pt}\ c\ {\isacharasterisk}{\kern0pt}\isactrlsub R\ X\ j\ {\isasymxi}{\isacharparenright}{\kern0pt}\ {\isasymxi}{\isachardoublequoteclose}\ \isanewline
\ \ \ \ \ \ \isacommand{using}\isamarkupfalse%
\ sigma{\isacharunderscore}{\kern0pt}finite{\isacharunderscore}{\kern0pt}subalgebra{\isachardot}{\kern0pt}cond{\isacharunderscore}{\kern0pt}exp{\isacharunderscore}{\kern0pt}scaleR{\isacharunderscore}{\kern0pt}right{\isacharbrackleft}{\kern0pt}OF\ {\isacharunderscore}{\kern0pt}\ integrable{\isacharcomma}{\kern0pt}\ of\ {\isachardoublequoteopen}F\ i{\isachardoublequoteclose}\ j\ c{\isacharbrackright}{\kern0pt}\ submartingale{\isacharunderscore}{\kern0pt}property{\isacharbrackleft}{\kern0pt}OF\ asm{\isacharbrackright}{\kern0pt}\ \isanewline
\ \ \ \ \ \ \ \ \ \ \ \ \isacommand{by}\isamarkupfalse%
\ {\isacharparenleft}{\kern0pt}fastforce\ intro{\isacharbang}{\kern0pt}{\isacharcolon}{\kern0pt}\ scaleR{\isacharunderscore}{\kern0pt}left{\isacharunderscore}{\kern0pt}mono{\isacharunderscore}{\kern0pt}neg{\isacharbrackleft}{\kern0pt}OF\ {\isacharunderscore}{\kern0pt}\ assms{\isacharbrackright}{\kern0pt}{\isacharparenright}{\kern0pt}\isanewline
\ \ \isacommand{{\isacharbraceright}{\kern0pt}}\isamarkupfalse%
\isanewline
\isacommand{qed}\isamarkupfalse%
\ {\isacharparenleft}{\kern0pt}auto\ simp\ add{\isacharcolon}{\kern0pt}\ borel{\isacharunderscore}{\kern0pt}measurable{\isacharunderscore}{\kern0pt}integrable\ borel{\isacharunderscore}{\kern0pt}measurable{\isacharunderscore}{\kern0pt}scaleR\ integrable\ random{\isacharunderscore}{\kern0pt}variable\ adapted\ borel{\isacharunderscore}{\kern0pt}measurable{\isacharunderscore}{\kern0pt}const{\isacharunderscore}{\kern0pt}scaleR{\isacharparenright}{\kern0pt}%
\endisatagproof
{\isafoldproof}%
%
\isadelimproof
\isanewline
%
\endisadelimproof
\isanewline
\isacommand{lemma}\isamarkupfalse%
\ uminus{\isacharbrackleft}{\kern0pt}intro{\isacharbrackright}{\kern0pt}{\isacharcolon}{\kern0pt}\isanewline
\ \ \isakeyword{shows}\ {\isachardoublequoteopen}supermartingale\ M\ F\ t\isactrlsub {\isadigit{0}}\ {\isacharparenleft}{\kern0pt}{\isacharminus}{\kern0pt}\ X{\isacharparenright}{\kern0pt}{\isachardoublequoteclose}\isanewline
%
\isadelimproof
\ \ %
\endisadelimproof
%
\isatagproof
\isacommand{unfolding}\isamarkupfalse%
\ fun{\isacharunderscore}{\kern0pt}Compl{\isacharunderscore}{\kern0pt}def\ \isacommand{using}\isamarkupfalse%
\ scaleR{\isacharunderscore}{\kern0pt}le{\isacharunderscore}{\kern0pt}zero{\isacharbrackleft}{\kern0pt}of\ {\isachardoublequoteopen}{\isacharminus}{\kern0pt}{\isadigit{1}}{\isachardoublequoteclose}{\isacharbrackright}{\kern0pt}\ \isacommand{by}\isamarkupfalse%
\ simp%
\endisatagproof
{\isafoldproof}%
%
\isadelimproof
\isanewline
%
\endisadelimproof
\isanewline
\isacommand{end}\isamarkupfalse%
\isanewline
\isanewline
\isacommand{context}\isamarkupfalse%
\ submartingale{\isacharunderscore}{\kern0pt}linorder\isanewline
\isakeyword{begin}\isanewline
\isanewline
\isacommand{lemma}\isamarkupfalse%
\ set{\isacharunderscore}{\kern0pt}integral{\isacharunderscore}{\kern0pt}le{\isacharcolon}{\kern0pt}\isanewline
\ \ \isakeyword{assumes}\ {\isachardoublequoteopen}A\ {\isasymin}\ F\ i{\isachardoublequoteclose}\ {\isachardoublequoteopen}t\isactrlsub {\isadigit{0}}\ {\isasymle}\ i{\isachardoublequoteclose}\ {\isachardoublequoteopen}i\ {\isasymle}\ j{\isachardoublequoteclose}\isanewline
\ \ \isakeyword{shows}\ {\isachardoublequoteopen}set{\isacharunderscore}{\kern0pt}lebesgue{\isacharunderscore}{\kern0pt}integral\ M\ A\ {\isacharparenleft}{\kern0pt}X\ i{\isacharparenright}{\kern0pt}\ {\isasymle}\ set{\isacharunderscore}{\kern0pt}lebesgue{\isacharunderscore}{\kern0pt}integral\ M\ A\ {\isacharparenleft}{\kern0pt}X\ j{\isacharparenright}{\kern0pt}{\isachardoublequoteclose}\ \ \isanewline
%
\isadelimproof
\ \ %
\endisadelimproof
%
\isatagproof
\isacommand{using}\isamarkupfalse%
\ submartingale{\isacharunderscore}{\kern0pt}property{\isacharbrackleft}{\kern0pt}OF\ assms{\isacharparenleft}{\kern0pt}{\isadigit{2}}{\isacharparenright}{\kern0pt}{\isacharcomma}{\kern0pt}\ of\ j{\isacharbrackright}{\kern0pt}\ assms\ subalgebras\isanewline
\ \ \isacommand{by}\isamarkupfalse%
\ {\isacharparenleft}{\kern0pt}subst\ sigma{\isacharunderscore}{\kern0pt}finite{\isacharunderscore}{\kern0pt}subalgebra{\isachardot}{\kern0pt}cond{\isacharunderscore}{\kern0pt}exp{\isacharunderscore}{\kern0pt}set{\isacharunderscore}{\kern0pt}integral{\isacharbrackleft}{\kern0pt}OF\ {\isacharunderscore}{\kern0pt}\ integrable\ assms{\isacharparenleft}{\kern0pt}{\isadigit{1}}{\isacharparenright}{\kern0pt}{\isacharcomma}{\kern0pt}\ of\ j{\isacharbrackright}{\kern0pt}{\isacharparenright}{\kern0pt}\isanewline
\ \ \ \ \ {\isacharparenleft}{\kern0pt}auto\ intro{\isacharbang}{\kern0pt}{\isacharcolon}{\kern0pt}\ scaleR{\isacharunderscore}{\kern0pt}left{\isacharunderscore}{\kern0pt}mono\ integral{\isacharunderscore}{\kern0pt}mono{\isacharunderscore}{\kern0pt}AE{\isacharunderscore}{\kern0pt}banach\ integrable{\isacharunderscore}{\kern0pt}mult{\isacharunderscore}{\kern0pt}indicator\ integrable\ simp\ add{\isacharcolon}{\kern0pt}\ subalgebra{\isacharunderscore}{\kern0pt}def\ set{\isacharunderscore}{\kern0pt}lebesgue{\isacharunderscore}{\kern0pt}integral{\isacharunderscore}{\kern0pt}def{\isacharparenright}{\kern0pt}%
\endisatagproof
{\isafoldproof}%
%
\isadelimproof
\isanewline
%
\endisadelimproof
\isanewline
\isacommand{lemma}\isamarkupfalse%
\ max{\isacharcolon}{\kern0pt}\isanewline
\ \ \isakeyword{assumes}\ {\isachardoublequoteopen}submartingale{\isacharunderscore}{\kern0pt}linorder\ M\ F\ t\isactrlsub {\isadigit{0}}\ Y{\isachardoublequoteclose}\isanewline
\ \ \isakeyword{shows}\ {\isachardoublequoteopen}submartingale{\isacharunderscore}{\kern0pt}linorder\ M\ F\ t\isactrlsub {\isadigit{0}}\ {\isacharparenleft}{\kern0pt}{\isasymlambda}i\ {\isasymxi}{\isachardot}{\kern0pt}\ max\ {\isacharparenleft}{\kern0pt}X\ i\ {\isasymxi}{\isacharparenright}{\kern0pt}\ {\isacharparenleft}{\kern0pt}Y\ i\ {\isasymxi}{\isacharparenright}{\kern0pt}{\isacharparenright}{\kern0pt}{\isachardoublequoteclose}\isanewline
%
\isadelimproof
%
\endisadelimproof
%
\isatagproof
\isacommand{proof}\isamarkupfalse%
\ {\isacharparenleft}{\kern0pt}unfold{\isacharunderscore}{\kern0pt}locales{\isacharparenright}{\kern0pt}\isanewline
\ \ \isacommand{interpret}\isamarkupfalse%
\ Y{\isacharcolon}{\kern0pt}\ submartingale{\isacharunderscore}{\kern0pt}linorder\ M\ F\ t\isactrlsub {\isadigit{0}}\ Y\ \isacommand{by}\isamarkupfalse%
\ {\isacharparenleft}{\kern0pt}rule\ assms{\isacharparenright}{\kern0pt}\isanewline
\ \ \isacommand{{\isacharbraceleft}{\kern0pt}}\isamarkupfalse%
\isanewline
\ \ \ \ \isacommand{fix}\isamarkupfalse%
\ i\ j\ {\isacharcolon}{\kern0pt}{\isacharcolon}{\kern0pt}\ {\isacharprime}{\kern0pt}b\ \isacommand{assume}\isamarkupfalse%
\ asm{\isacharcolon}{\kern0pt}\ {\isachardoublequoteopen}t\isactrlsub {\isadigit{0}}\ {\isasymle}\ i{\isachardoublequoteclose}\ {\isachardoublequoteopen}i\ {\isasymle}\ j{\isachardoublequoteclose}\isanewline
\ \ \ \ \isacommand{have}\isamarkupfalse%
\ {\isachardoublequoteopen}AE\ {\isasymxi}\ in\ M{\isachardot}{\kern0pt}\ max\ {\isacharparenleft}{\kern0pt}X\ i\ {\isasymxi}{\isacharparenright}{\kern0pt}\ {\isacharparenleft}{\kern0pt}Y\ i\ {\isasymxi}{\isacharparenright}{\kern0pt}\ {\isasymle}\ max\ {\isacharparenleft}{\kern0pt}cond{\isacharunderscore}{\kern0pt}exp\ M\ {\isacharparenleft}{\kern0pt}F\ i{\isacharparenright}{\kern0pt}\ {\isacharparenleft}{\kern0pt}X\ j{\isacharparenright}{\kern0pt}\ {\isasymxi}{\isacharparenright}{\kern0pt}\ {\isacharparenleft}{\kern0pt}cond{\isacharunderscore}{\kern0pt}exp\ M\ {\isacharparenleft}{\kern0pt}F\ i{\isacharparenright}{\kern0pt}\ {\isacharparenleft}{\kern0pt}Y\ j{\isacharparenright}{\kern0pt}\ {\isasymxi}{\isacharparenright}{\kern0pt}{\isachardoublequoteclose}\ \isacommand{using}\isamarkupfalse%
\ submartingale{\isacharunderscore}{\kern0pt}property\ Y{\isachardot}{\kern0pt}submartingale{\isacharunderscore}{\kern0pt}property\ asm\ \isacommand{unfolding}\isamarkupfalse%
\ max{\isacharunderscore}{\kern0pt}def\ \isacommand{by}\isamarkupfalse%
\ fastforce\isanewline
\ \ \ \ \isacommand{thus}\isamarkupfalse%
\ {\isachardoublequoteopen}AE\ {\isasymxi}\ in\ M{\isachardot}{\kern0pt}\ max\ {\isacharparenleft}{\kern0pt}X\ i\ {\isasymxi}{\isacharparenright}{\kern0pt}\ {\isacharparenleft}{\kern0pt}Y\ i\ {\isasymxi}{\isacharparenright}{\kern0pt}\ {\isasymle}\ cond{\isacharunderscore}{\kern0pt}exp\ M\ {\isacharparenleft}{\kern0pt}F\ i{\isacharparenright}{\kern0pt}\ {\isacharparenleft}{\kern0pt}{\isasymlambda}{\isasymxi}{\isachardot}{\kern0pt}\ max\ {\isacharparenleft}{\kern0pt}X\ j\ {\isasymxi}{\isacharparenright}{\kern0pt}\ {\isacharparenleft}{\kern0pt}Y\ j\ {\isasymxi}{\isacharparenright}{\kern0pt}{\isacharparenright}{\kern0pt}\ {\isasymxi}{\isachardoublequoteclose}\ \isacommand{using}\isamarkupfalse%
\ sigma{\isacharunderscore}{\kern0pt}finite{\isacharunderscore}{\kern0pt}subalgebra{\isachardot}{\kern0pt}cond{\isacharunderscore}{\kern0pt}exp{\isacharunderscore}{\kern0pt}max{\isacharbrackleft}{\kern0pt}OF\ {\isacharunderscore}{\kern0pt}\ integrable\ Y{\isachardot}{\kern0pt}integrable{\isacharcomma}{\kern0pt}\ of\ {\isachardoublequoteopen}F\ i{\isachardoublequoteclose}\ j\ j{\isacharbrackright}{\kern0pt}\ asm\ \isacommand{by}\isamarkupfalse%
\ {\isacharparenleft}{\kern0pt}fast\ intro{\isacharcolon}{\kern0pt}\ order{\isachardot}{\kern0pt}trans{\isacharparenright}{\kern0pt}\isanewline
\ \ \isacommand{{\isacharbraceright}{\kern0pt}}\isamarkupfalse%
\isanewline
\ \ \isacommand{show}\isamarkupfalse%
\ {\isachardoublequoteopen}{\isasymAnd}i{\isachardot}{\kern0pt}\ t\isactrlsub {\isadigit{0}}\ {\isasymle}\ i\ {\isasymLongrightarrow}\ {\isacharparenleft}{\kern0pt}{\isasymlambda}{\isasymxi}{\isachardot}{\kern0pt}\ max\ {\isacharparenleft}{\kern0pt}X\ i\ {\isasymxi}{\isacharparenright}{\kern0pt}\ {\isacharparenleft}{\kern0pt}Y\ i\ {\isasymxi}{\isacharparenright}{\kern0pt}{\isacharparenright}{\kern0pt}\ {\isasymin}\ borel{\isacharunderscore}{\kern0pt}measurable\ {\isacharparenleft}{\kern0pt}F\ i{\isacharparenright}{\kern0pt}{\isachardoublequoteclose}\ {\isachardoublequoteopen}{\isasymAnd}i{\isachardot}{\kern0pt}\ t\isactrlsub {\isadigit{0}}\ {\isasymle}\ i\ {\isasymLongrightarrow}\ integrable\ M\ {\isacharparenleft}{\kern0pt}{\isasymlambda}{\isasymxi}{\isachardot}{\kern0pt}\ max\ {\isacharparenleft}{\kern0pt}X\ i\ {\isasymxi}{\isacharparenright}{\kern0pt}\ {\isacharparenleft}{\kern0pt}Y\ i\ {\isasymxi}{\isacharparenright}{\kern0pt}{\isacharparenright}{\kern0pt}{\isachardoublequoteclose}\ \isacommand{by}\isamarkupfalse%
\ {\isacharparenleft}{\kern0pt}force\ intro{\isacharcolon}{\kern0pt}\ Y{\isachardot}{\kern0pt}integrable\ integrable\ assms{\isacharparenright}{\kern0pt}{\isacharplus}{\kern0pt}\isanewline
\isacommand{qed}\isamarkupfalse%
%
\endisatagproof
{\isafoldproof}%
%
\isadelimproof
\isanewline
%
\endisadelimproof
\isanewline
\isacommand{lemma}\isamarkupfalse%
\ max{\isacharunderscore}{\kern0pt}{\isadigit{0}}{\isacharcolon}{\kern0pt}\isanewline
\ \ \isakeyword{shows}\ {\isachardoublequoteopen}submartingale{\isacharunderscore}{\kern0pt}linorder\ M\ F\ t\isactrlsub {\isadigit{0}}\ {\isacharparenleft}{\kern0pt}{\isasymlambda}i\ {\isasymxi}{\isachardot}{\kern0pt}\ max\ {\isadigit{0}}\ {\isacharparenleft}{\kern0pt}X\ i\ {\isasymxi}{\isacharparenright}{\kern0pt}{\isacharparenright}{\kern0pt}{\isachardoublequoteclose}\isanewline
%
\isadelimproof
%
\endisadelimproof
%
\isatagproof
\isacommand{proof}\isamarkupfalse%
\ {\isacharminus}{\kern0pt}\isanewline
\ \ \isacommand{interpret}\isamarkupfalse%
\ zero{\isacharcolon}{\kern0pt}\ martingale{\isacharunderscore}{\kern0pt}linorder\ M\ F\ t\isactrlsub {\isadigit{0}}\ {\isachardoublequoteopen}{\isasymlambda}{\isacharunderscore}{\kern0pt}\ {\isacharunderscore}{\kern0pt}{\isachardot}{\kern0pt}\ {\isadigit{0}}{\isachardoublequoteclose}\ \isacommand{by}\isamarkupfalse%
\ {\isacharparenleft}{\kern0pt}force\ intro{\isacharcolon}{\kern0pt}\ martingale{\isacharunderscore}{\kern0pt}linorder{\isachardot}{\kern0pt}intro\ martingale{\isacharunderscore}{\kern0pt}order{\isachardot}{\kern0pt}intro{\isacharparenright}{\kern0pt}\isanewline
\ \ \isacommand{show}\isamarkupfalse%
\ {\isacharquery}{\kern0pt}thesis\ \isacommand{by}\isamarkupfalse%
\ {\isacharparenleft}{\kern0pt}intro\ zero{\isachardot}{\kern0pt}max\ submartingale{\isacharunderscore}{\kern0pt}linorder{\isachardot}{\kern0pt}intro\ submartingale{\isacharunderscore}{\kern0pt}axioms{\isacharparenright}{\kern0pt}\isanewline
\isacommand{qed}\isamarkupfalse%
%
\endisatagproof
{\isafoldproof}%
%
\isadelimproof
\isanewline
%
\endisadelimproof
\isanewline
\isacommand{end}\isamarkupfalse%
\isanewline
\isanewline
\isacommand{lemma}\isamarkupfalse%
\ {\isacharparenleft}{\kern0pt}\isakeyword{in}\ sigma{\isacharunderscore}{\kern0pt}finite{\isacharunderscore}{\kern0pt}adapted{\isacharunderscore}{\kern0pt}process{\isacharunderscore}{\kern0pt}order{\isacharparenright}{\kern0pt}\ submartingale{\isacharunderscore}{\kern0pt}of{\isacharunderscore}{\kern0pt}cond{\isacharunderscore}{\kern0pt}exp{\isacharunderscore}{\kern0pt}diff{\isacharunderscore}{\kern0pt}nonneg{\isacharcolon}{\kern0pt}\isanewline
\ \ \isakeyword{assumes}\ integrable{\isacharcolon}{\kern0pt}\ {\isachardoublequoteopen}{\isasymAnd}i{\isachardot}{\kern0pt}\ t\isactrlsub {\isadigit{0}}\ {\isasymle}\ i\ {\isasymLongrightarrow}\ \ integrable\ M\ {\isacharparenleft}{\kern0pt}X\ i{\isacharparenright}{\kern0pt}{\isachardoublequoteclose}\ \isanewline
\ \ \ \ \ \ \isakeyword{and}\ diff{\isacharunderscore}{\kern0pt}nonneg{\isacharcolon}{\kern0pt}\ {\isachardoublequoteopen}{\isasymAnd}i\ j{\isachardot}{\kern0pt}\ t\isactrlsub {\isadigit{0}}\ {\isasymle}\ i\ {\isasymLongrightarrow}\ i\ {\isasymle}\ j\ {\isasymLongrightarrow}\ AE\ x\ in\ M{\isachardot}{\kern0pt}\ cond{\isacharunderscore}{\kern0pt}exp\ M\ {\isacharparenleft}{\kern0pt}F\ i{\isacharparenright}{\kern0pt}\ {\isacharparenleft}{\kern0pt}{\isasymlambda}{\isasymxi}{\isachardot}{\kern0pt}\ X\ j\ {\isasymxi}\ {\isacharminus}{\kern0pt}\ X\ i\ {\isasymxi}{\isacharparenright}{\kern0pt}\ x\ {\isasymge}\ {\isadigit{0}}{\isachardoublequoteclose}\isanewline
\ \ \ \ \isakeyword{shows}\ {\isachardoublequoteopen}submartingale\ M\ F\ t\isactrlsub {\isadigit{0}}\ X{\isachardoublequoteclose}\isanewline
%
\isadelimproof
%
\endisadelimproof
%
\isatagproof
\isacommand{proof}\isamarkupfalse%
\ {\isacharparenleft}{\kern0pt}unfold{\isacharunderscore}{\kern0pt}locales{\isacharparenright}{\kern0pt}\isanewline
\ \ \isacommand{{\isacharbraceleft}{\kern0pt}}\isamarkupfalse%
\isanewline
\ \ \ \ \isacommand{fix}\isamarkupfalse%
\ i\ j\ {\isacharcolon}{\kern0pt}{\isacharcolon}{\kern0pt}\ {\isacharprime}{\kern0pt}b\ \isacommand{assume}\isamarkupfalse%
\ asm{\isacharcolon}{\kern0pt}\ {\isachardoublequoteopen}t\isactrlsub {\isadigit{0}}\ {\isasymle}\ i{\isachardoublequoteclose}\ {\isachardoublequoteopen}i\ {\isasymle}\ j{\isachardoublequoteclose}\isanewline
\ \ \ \ \isacommand{thus}\isamarkupfalse%
\ {\isachardoublequoteopen}AE\ {\isasymxi}\ in\ M{\isachardot}{\kern0pt}\ X\ i\ {\isasymxi}\ {\isasymle}\ cond{\isacharunderscore}{\kern0pt}exp\ M\ {\isacharparenleft}{\kern0pt}F\ i{\isacharparenright}{\kern0pt}\ {\isacharparenleft}{\kern0pt}X\ j{\isacharparenright}{\kern0pt}\ {\isasymxi}{\isachardoublequoteclose}\ \isanewline
\ \ \ \ \ \ \isacommand{using}\isamarkupfalse%
\ diff{\isacharunderscore}{\kern0pt}nonneg{\isacharbrackleft}{\kern0pt}OF\ asm{\isacharbrackright}{\kern0pt}\ sigma{\isacharunderscore}{\kern0pt}finite{\isacharunderscore}{\kern0pt}subalgebra{\isachardot}{\kern0pt}cond{\isacharunderscore}{\kern0pt}exp{\isacharunderscore}{\kern0pt}diff{\isacharbrackleft}{\kern0pt}OF\ {\isacharunderscore}{\kern0pt}\ integrable{\isacharparenleft}{\kern0pt}{\isadigit{1}}{\isacharcomma}{\kern0pt}{\isadigit{1}}{\isacharparenright}{\kern0pt}{\isacharcomma}{\kern0pt}\ of\ {\isachardoublequoteopen}F\ i{\isachardoublequoteclose}\ j\ i{\isacharbrackright}{\kern0pt}\isanewline
\ \ \ \ \ \ \ \ \ \ \ \ sigma{\isacharunderscore}{\kern0pt}finite{\isacharunderscore}{\kern0pt}subalgebra{\isachardot}{\kern0pt}cond{\isacharunderscore}{\kern0pt}exp{\isacharunderscore}{\kern0pt}F{\isacharunderscore}{\kern0pt}meas{\isacharbrackleft}{\kern0pt}OF\ {\isacharunderscore}{\kern0pt}\ integrable\ adapted{\isacharcomma}{\kern0pt}\ of\ i{\isacharbrackright}{\kern0pt}\ \isacommand{by}\isamarkupfalse%
\ fastforce\isanewline
\ \ \isacommand{{\isacharbraceright}{\kern0pt}}\isamarkupfalse%
\isanewline
\isacommand{qed}\isamarkupfalse%
\ {\isacharparenleft}{\kern0pt}intro\ integrable{\isacharparenright}{\kern0pt}%
\endisatagproof
{\isafoldproof}%
%
\isadelimproof
\isanewline
%
\endisadelimproof
\isanewline
\isacommand{lemma}\isamarkupfalse%
\ {\isacharparenleft}{\kern0pt}\isakeyword{in}\ sigma{\isacharunderscore}{\kern0pt}finite{\isacharunderscore}{\kern0pt}adapted{\isacharunderscore}{\kern0pt}process{\isacharunderscore}{\kern0pt}linorder{\isacharparenright}{\kern0pt}\ submartingale{\isacharunderscore}{\kern0pt}of{\isacharunderscore}{\kern0pt}set{\isacharunderscore}{\kern0pt}integral{\isacharunderscore}{\kern0pt}le{\isacharcolon}{\kern0pt}\isanewline
\ \ \isakeyword{assumes}\ integrable{\isacharcolon}{\kern0pt}\ {\isachardoublequoteopen}{\isasymAnd}i{\isachardot}{\kern0pt}\ t\isactrlsub {\isadigit{0}}\ {\isasymle}\ i\ {\isasymLongrightarrow}\ integrable\ M\ {\isacharparenleft}{\kern0pt}X\ i{\isacharparenright}{\kern0pt}{\isachardoublequoteclose}\isanewline
\ \ \ \ \ \ \isakeyword{and}\ {\isachardoublequoteopen}{\isasymAnd}A\ i\ j{\isachardot}{\kern0pt}\ t\isactrlsub {\isadigit{0}}\ {\isasymle}\ i\ {\isasymLongrightarrow}\ i\ {\isasymle}\ j\ {\isasymLongrightarrow}\ A\ {\isasymin}\ F\ i\ {\isasymLongrightarrow}\ set{\isacharunderscore}{\kern0pt}lebesgue{\isacharunderscore}{\kern0pt}integral\ M\ A\ {\isacharparenleft}{\kern0pt}X\ i{\isacharparenright}{\kern0pt}\ {\isasymle}\ set{\isacharunderscore}{\kern0pt}lebesgue{\isacharunderscore}{\kern0pt}integral\ M\ A\ {\isacharparenleft}{\kern0pt}X\ j{\isacharparenright}{\kern0pt}{\isachardoublequoteclose}\isanewline
\ \ \ \ \isakeyword{shows}\ {\isachardoublequoteopen}submartingale\ M\ F\ t\isactrlsub {\isadigit{0}}\ X{\isachardoublequoteclose}\isanewline
%
\isadelimproof
%
\endisadelimproof
%
\isatagproof
\isacommand{proof}\isamarkupfalse%
\ {\isacharparenleft}{\kern0pt}unfold{\isacharunderscore}{\kern0pt}locales{\isacharparenright}{\kern0pt}\isanewline
\ \ \isacommand{{\isacharbraceleft}{\kern0pt}}\isamarkupfalse%
\isanewline
\ \ \ \ \isacommand{fix}\isamarkupfalse%
\ i\ j\ {\isacharcolon}{\kern0pt}{\isacharcolon}{\kern0pt}\ {\isacharprime}{\kern0pt}b\ \isacommand{assume}\isamarkupfalse%
\ asm{\isacharcolon}{\kern0pt}\ {\isachardoublequoteopen}t\isactrlsub {\isadigit{0}}\ {\isasymle}\ i{\isachardoublequoteclose}\ {\isachardoublequoteopen}i\ {\isasymle}\ j{\isachardoublequoteclose}\isanewline
\ \ \ \ \isacommand{interpret}\isamarkupfalse%
\ r{\isacharcolon}{\kern0pt}\ sigma{\isacharunderscore}{\kern0pt}finite{\isacharunderscore}{\kern0pt}measure\ {\isachardoublequoteopen}restr{\isacharunderscore}{\kern0pt}to{\isacharunderscore}{\kern0pt}subalg\ M\ {\isacharparenleft}{\kern0pt}F\ i{\isacharparenright}{\kern0pt}{\isachardoublequoteclose}\ \isacommand{using}\isamarkupfalse%
\ asm\ sigma{\isacharunderscore}{\kern0pt}finite{\isacharunderscore}{\kern0pt}subalgebra{\isachardot}{\kern0pt}sigma{\isacharunderscore}{\kern0pt}fin{\isacharunderscore}{\kern0pt}subalg\ \isacommand{by}\isamarkupfalse%
\ blast\isanewline
\ \ \ \ \isacommand{{\isacharbraceleft}{\kern0pt}}\isamarkupfalse%
\isanewline
\ \ \ \ \ \ \isacommand{fix}\isamarkupfalse%
\ A\ \isacommand{assume}\isamarkupfalse%
\ {\isachardoublequoteopen}A\ {\isasymin}\ restr{\isacharunderscore}{\kern0pt}to{\isacharunderscore}{\kern0pt}subalg\ M\ {\isacharparenleft}{\kern0pt}F\ i{\isacharparenright}{\kern0pt}{\isachardoublequoteclose}\isanewline
\ \ \ \ \ \ \isacommand{hence}\isamarkupfalse%
\ {\isacharasterisk}{\kern0pt}{\isacharcolon}{\kern0pt}\ {\isachardoublequoteopen}A\ {\isasymin}\ F\ i{\isachardoublequoteclose}\ \isacommand{using}\isamarkupfalse%
\ asm\ sets{\isacharunderscore}{\kern0pt}restr{\isacharunderscore}{\kern0pt}to{\isacharunderscore}{\kern0pt}subalg\ subalgebras\ \isacommand{by}\isamarkupfalse%
\ blast\isanewline
\ \ \ \ \ \ \isacommand{have}\isamarkupfalse%
\ {\isachardoublequoteopen}set{\isacharunderscore}{\kern0pt}lebesgue{\isacharunderscore}{\kern0pt}integral\ {\isacharparenleft}{\kern0pt}restr{\isacharunderscore}{\kern0pt}to{\isacharunderscore}{\kern0pt}subalg\ M\ {\isacharparenleft}{\kern0pt}F\ i{\isacharparenright}{\kern0pt}{\isacharparenright}{\kern0pt}\ A\ {\isacharparenleft}{\kern0pt}X\ i{\isacharparenright}{\kern0pt}\ {\isacharequal}{\kern0pt}\ set{\isacharunderscore}{\kern0pt}lebesgue{\isacharunderscore}{\kern0pt}integral\ M\ A\ {\isacharparenleft}{\kern0pt}X\ i{\isacharparenright}{\kern0pt}{\isachardoublequoteclose}\ \isacommand{using}\isamarkupfalse%
\ {\isacharasterisk}{\kern0pt}\ asm\ subalgebras\ \isacommand{by}\isamarkupfalse%
\ {\isacharparenleft}{\kern0pt}auto\ simp{\isacharcolon}{\kern0pt}\ set{\isacharunderscore}{\kern0pt}lebesgue{\isacharunderscore}{\kern0pt}integral{\isacharunderscore}{\kern0pt}def\ intro{\isacharcolon}{\kern0pt}\ integral{\isacharunderscore}{\kern0pt}subalgebra{\isadigit{2}}\ borel{\isacharunderscore}{\kern0pt}measurable{\isacharunderscore}{\kern0pt}scaleR\ adapted\ borel{\isacharunderscore}{\kern0pt}measurable{\isacharunderscore}{\kern0pt}indicator{\isacharparenright}{\kern0pt}\ \isanewline
\ \ \ \ \ \ \isacommand{also}\isamarkupfalse%
\ \isacommand{have}\isamarkupfalse%
\ {\isachardoublequoteopen}{\isachardot}{\kern0pt}{\isachardot}{\kern0pt}{\isachardot}{\kern0pt}\ {\isasymle}\ set{\isacharunderscore}{\kern0pt}lebesgue{\isacharunderscore}{\kern0pt}integral\ M\ A\ {\isacharparenleft}{\kern0pt}cond{\isacharunderscore}{\kern0pt}exp\ M\ {\isacharparenleft}{\kern0pt}F\ i{\isacharparenright}{\kern0pt}\ {\isacharparenleft}{\kern0pt}X\ j{\isacharparenright}{\kern0pt}{\isacharparenright}{\kern0pt}{\isachardoublequoteclose}\ \isacommand{using}\isamarkupfalse%
\ {\isacharasterisk}{\kern0pt}\ assms{\isacharparenleft}{\kern0pt}{\isadigit{2}}{\isacharparenright}{\kern0pt}{\isacharbrackleft}{\kern0pt}OF\ asm{\isacharbrackright}{\kern0pt}\ asm\ sigma{\isacharunderscore}{\kern0pt}finite{\isacharunderscore}{\kern0pt}subalgebra{\isachardot}{\kern0pt}cond{\isacharunderscore}{\kern0pt}exp{\isacharunderscore}{\kern0pt}set{\isacharunderscore}{\kern0pt}integral{\isacharbrackleft}{\kern0pt}OF\ {\isacharunderscore}{\kern0pt}\ integrable{\isacharbrackright}{\kern0pt}\ \isacommand{by}\isamarkupfalse%
\ fastforce\isanewline
\ \ \ \ \ \ \isacommand{also}\isamarkupfalse%
\ \isacommand{have}\isamarkupfalse%
\ {\isachardoublequoteopen}{\isachardot}{\kern0pt}{\isachardot}{\kern0pt}{\isachardot}{\kern0pt}\ {\isacharequal}{\kern0pt}\ set{\isacharunderscore}{\kern0pt}lebesgue{\isacharunderscore}{\kern0pt}integral\ {\isacharparenleft}{\kern0pt}restr{\isacharunderscore}{\kern0pt}to{\isacharunderscore}{\kern0pt}subalg\ M\ {\isacharparenleft}{\kern0pt}F\ i{\isacharparenright}{\kern0pt}{\isacharparenright}{\kern0pt}\ A\ {\isacharparenleft}{\kern0pt}cond{\isacharunderscore}{\kern0pt}exp\ M\ {\isacharparenleft}{\kern0pt}F\ i{\isacharparenright}{\kern0pt}\ {\isacharparenleft}{\kern0pt}X\ j{\isacharparenright}{\kern0pt}{\isacharparenright}{\kern0pt}{\isachardoublequoteclose}\ \isacommand{using}\isamarkupfalse%
\ {\isacharasterisk}{\kern0pt}\ asm\ subalgebras\ \isacommand{by}\isamarkupfalse%
\ {\isacharparenleft}{\kern0pt}auto\ simp{\isacharcolon}{\kern0pt}\ set{\isacharunderscore}{\kern0pt}lebesgue{\isacharunderscore}{\kern0pt}integral{\isacharunderscore}{\kern0pt}def\ intro{\isacharbang}{\kern0pt}{\isacharcolon}{\kern0pt}\ integral{\isacharunderscore}{\kern0pt}subalgebra{\isadigit{2}}{\isacharbrackleft}{\kern0pt}symmetric{\isacharbrackright}{\kern0pt}\ borel{\isacharunderscore}{\kern0pt}measurable{\isacharunderscore}{\kern0pt}scaleR\ borel{\isacharunderscore}{\kern0pt}measurable{\isacharunderscore}{\kern0pt}cond{\isacharunderscore}{\kern0pt}exp\ borel{\isacharunderscore}{\kern0pt}measurable{\isacharunderscore}{\kern0pt}indicator{\isacharparenright}{\kern0pt}\isanewline
\ \ \ \ \ \ \isacommand{finally}\isamarkupfalse%
\ \isacommand{have}\isamarkupfalse%
\ {\isachardoublequoteopen}{\isadigit{0}}\ {\isasymle}\ set{\isacharunderscore}{\kern0pt}lebesgue{\isacharunderscore}{\kern0pt}integral\ {\isacharparenleft}{\kern0pt}restr{\isacharunderscore}{\kern0pt}to{\isacharunderscore}{\kern0pt}subalg\ M\ {\isacharparenleft}{\kern0pt}F\ i{\isacharparenright}{\kern0pt}{\isacharparenright}{\kern0pt}\ A\ {\isacharparenleft}{\kern0pt}{\isasymlambda}{\isasymxi}{\isachardot}{\kern0pt}\ cond{\isacharunderscore}{\kern0pt}exp\ M\ {\isacharparenleft}{\kern0pt}F\ i{\isacharparenright}{\kern0pt}\ {\isacharparenleft}{\kern0pt}X\ j{\isacharparenright}{\kern0pt}\ {\isasymxi}\ {\isacharminus}{\kern0pt}\ X\ i\ {\isasymxi}{\isacharparenright}{\kern0pt}{\isachardoublequoteclose}\ \isacommand{using}\isamarkupfalse%
\ {\isacharasterisk}{\kern0pt}\ asm\ subalgebras\ \isacommand{by}\isamarkupfalse%
\ {\isacharparenleft}{\kern0pt}subst\ set{\isacharunderscore}{\kern0pt}integral{\isacharunderscore}{\kern0pt}diff{\isacharcomma}{\kern0pt}\ auto\ simp\ add{\isacharcolon}{\kern0pt}\ set{\isacharunderscore}{\kern0pt}integrable{\isacharunderscore}{\kern0pt}def\ sets{\isacharunderscore}{\kern0pt}restr{\isacharunderscore}{\kern0pt}to{\isacharunderscore}{\kern0pt}subalg\ intro{\isacharbang}{\kern0pt}{\isacharcolon}{\kern0pt}\ integrable\ adapted\ integrable{\isacharunderscore}{\kern0pt}in{\isacharunderscore}{\kern0pt}subalg\ borel{\isacharunderscore}{\kern0pt}measurable{\isacharunderscore}{\kern0pt}scaleR\ borel{\isacharunderscore}{\kern0pt}measurable{\isacharunderscore}{\kern0pt}indicator\ borel{\isacharunderscore}{\kern0pt}measurable{\isacharunderscore}{\kern0pt}cond{\isacharunderscore}{\kern0pt}exp\ integrable{\isacharunderscore}{\kern0pt}mult{\isacharunderscore}{\kern0pt}indicator{\isacharparenright}{\kern0pt}\isanewline
\ \ \ \ \isacommand{{\isacharbraceright}{\kern0pt}}\isamarkupfalse%
\isanewline
\ \ \ \ \isacommand{hence}\isamarkupfalse%
\ {\isachardoublequoteopen}AE\ {\isasymxi}\ in\ restr{\isacharunderscore}{\kern0pt}to{\isacharunderscore}{\kern0pt}subalg\ M\ {\isacharparenleft}{\kern0pt}F\ i{\isacharparenright}{\kern0pt}{\isachardot}{\kern0pt}\ {\isadigit{0}}\ {\isasymle}\ cond{\isacharunderscore}{\kern0pt}exp\ M\ {\isacharparenleft}{\kern0pt}F\ i{\isacharparenright}{\kern0pt}\ {\isacharparenleft}{\kern0pt}X\ j{\isacharparenright}{\kern0pt}\ {\isasymxi}\ {\isacharminus}{\kern0pt}\ X\ i\ {\isasymxi}{\isachardoublequoteclose}\ \isanewline
\ \ \ \ \ \ \isacommand{by}\isamarkupfalse%
\ {\isacharparenleft}{\kern0pt}intro\ r{\isachardot}{\kern0pt}density{\isacharunderscore}{\kern0pt}nonneg\ integrable{\isacharunderscore}{\kern0pt}in{\isacharunderscore}{\kern0pt}subalg\ asm\ subalgebras\ borel{\isacharunderscore}{\kern0pt}measurable{\isacharunderscore}{\kern0pt}diff\ borel{\isacharunderscore}{\kern0pt}measurable{\isacharunderscore}{\kern0pt}cond{\isacharunderscore}{\kern0pt}exp\ adapted\ Bochner{\isacharunderscore}{\kern0pt}Integration{\isachardot}{\kern0pt}integrable{\isacharunderscore}{\kern0pt}diff\ integrable{\isacharunderscore}{\kern0pt}cond{\isacharunderscore}{\kern0pt}exp\ integrable{\isacharparenright}{\kern0pt}\isanewline
\ \ \ \ \isacommand{thus}\isamarkupfalse%
\ {\isachardoublequoteopen}AE\ {\isasymxi}\ in\ M{\isachardot}{\kern0pt}\ X\ i\ {\isasymxi}\ {\isasymle}\ cond{\isacharunderscore}{\kern0pt}exp\ M\ {\isacharparenleft}{\kern0pt}F\ i{\isacharparenright}{\kern0pt}\ {\isacharparenleft}{\kern0pt}X\ j{\isacharparenright}{\kern0pt}\ {\isasymxi}{\isachardoublequoteclose}\ \isacommand{using}\isamarkupfalse%
\ AE{\isacharunderscore}{\kern0pt}restr{\isacharunderscore}{\kern0pt}to{\isacharunderscore}{\kern0pt}subalg{\isacharbrackleft}{\kern0pt}OF\ subalgebras{\isacharbrackright}{\kern0pt}\ asm\ \isacommand{by}\isamarkupfalse%
\ simp\isanewline
\ \ \isacommand{{\isacharbraceright}{\kern0pt}}\isamarkupfalse%
\isanewline
\isacommand{qed}\isamarkupfalse%
\ {\isacharparenleft}{\kern0pt}intro\ integrable{\isacharparenright}{\kern0pt}%
\endisatagproof
{\isafoldproof}%
%
\isadelimproof
%
\endisadelimproof
%
\isadelimdocument
%
\endisadelimdocument
%
\isatagdocument
%
\isamarkupsubsection{Supermartingale Lemmas%
}
\isamarkuptrue%
%
\endisatagdocument
{\isafolddocument}%
%
\isadelimdocument
%
\endisadelimdocument
%
\begin{isamarkuptext}%
The following lemmas are exact duals of the ones for submartingales.%
\end{isamarkuptext}\isamarkuptrue%
\isacommand{context}\isamarkupfalse%
\ supermartingale\isanewline
\isakeyword{begin}\isanewline
\isanewline
\isacommand{lemma}\isamarkupfalse%
\ cond{\isacharunderscore}{\kern0pt}exp{\isacharunderscore}{\kern0pt}diff{\isacharunderscore}{\kern0pt}nonneg{\isacharcolon}{\kern0pt}\isanewline
\ \ \isakeyword{assumes}\ {\isachardoublequoteopen}t\isactrlsub {\isadigit{0}}\ {\isasymle}\ i{\isachardoublequoteclose}\ {\isachardoublequoteopen}i\ {\isasymle}\ j{\isachardoublequoteclose}\isanewline
\ \ \isakeyword{shows}\ {\isachardoublequoteopen}AE\ x\ in\ M{\isachardot}{\kern0pt}\ cond{\isacharunderscore}{\kern0pt}exp\ M\ {\isacharparenleft}{\kern0pt}F\ i{\isacharparenright}{\kern0pt}\ {\isacharparenleft}{\kern0pt}{\isasymlambda}{\isasymxi}{\isachardot}{\kern0pt}\ X\ i\ {\isasymxi}\ {\isacharminus}{\kern0pt}\ X\ j\ {\isasymxi}{\isacharparenright}{\kern0pt}\ x\ {\isasymge}\ {\isadigit{0}}{\isachardoublequoteclose}\isanewline
%
\isadelimproof
\ \ %
\endisadelimproof
%
\isatagproof
\isacommand{using}\isamarkupfalse%
\ assms\ supermartingale{\isacharunderscore}{\kern0pt}property{\isacharbrackleft}{\kern0pt}OF\ assms{\isacharbrackright}{\kern0pt}\ sigma{\isacharunderscore}{\kern0pt}finite{\isacharunderscore}{\kern0pt}subalgebra{\isachardot}{\kern0pt}cond{\isacharunderscore}{\kern0pt}exp{\isacharunderscore}{\kern0pt}diff{\isacharbrackleft}{\kern0pt}OF\ {\isacharunderscore}{\kern0pt}\ integrable{\isacharparenleft}{\kern0pt}{\isadigit{1}}{\isacharcomma}{\kern0pt}{\isadigit{1}}{\isacharparenright}{\kern0pt}{\isacharcomma}{\kern0pt}\ of\ {\isachardoublequoteopen}F\ i{\isachardoublequoteclose}\ i\ j{\isacharbrackright}{\kern0pt}\ \isanewline
\ \ \ \ \ \ \ \ sigma{\isacharunderscore}{\kern0pt}finite{\isacharunderscore}{\kern0pt}subalgebra{\isachardot}{\kern0pt}cond{\isacharunderscore}{\kern0pt}exp{\isacharunderscore}{\kern0pt}F{\isacharunderscore}{\kern0pt}meas{\isacharbrackleft}{\kern0pt}OF\ {\isacharunderscore}{\kern0pt}\ integrable\ adapted{\isacharcomma}{\kern0pt}\ of\ i{\isacharbrackright}{\kern0pt}\ \isacommand{by}\isamarkupfalse%
\ fastforce%
\endisatagproof
{\isafoldproof}%
%
\isadelimproof
\isanewline
%
\endisadelimproof
\isanewline
\isacommand{lemma}\isamarkupfalse%
\ add{\isacharbrackleft}{\kern0pt}intro{\isacharbrackright}{\kern0pt}{\isacharcolon}{\kern0pt}\isanewline
\ \ \isakeyword{assumes}\ {\isachardoublequoteopen}supermartingale\ M\ F\ t\isactrlsub {\isadigit{0}}\ Y{\isachardoublequoteclose}\isanewline
\ \ \isakeyword{shows}\ {\isachardoublequoteopen}supermartingale\ M\ F\ t\isactrlsub {\isadigit{0}}\ {\isacharparenleft}{\kern0pt}{\isasymlambda}i\ {\isasymxi}{\isachardot}{\kern0pt}\ X\ i\ {\isasymxi}\ {\isacharplus}{\kern0pt}\ Y\ i\ {\isasymxi}{\isacharparenright}{\kern0pt}{\isachardoublequoteclose}\isanewline
%
\isadelimproof
%
\endisadelimproof
%
\isatagproof
\isacommand{proof}\isamarkupfalse%
\ {\isacharminus}{\kern0pt}\isanewline
\ \ \isacommand{interpret}\isamarkupfalse%
\ Y{\isacharcolon}{\kern0pt}\ supermartingale\ M\ F\ t\isactrlsub {\isadigit{0}}\ Y\ \isacommand{by}\isamarkupfalse%
\ {\isacharparenleft}{\kern0pt}rule\ assms{\isacharparenright}{\kern0pt}\isanewline
\ \ \isacommand{{\isacharbraceleft}{\kern0pt}}\isamarkupfalse%
\isanewline
\ \ \ \ \isacommand{fix}\isamarkupfalse%
\ i\ j\ {\isacharcolon}{\kern0pt}{\isacharcolon}{\kern0pt}\ {\isacharprime}{\kern0pt}b\ \isacommand{assume}\isamarkupfalse%
\ asm{\isacharcolon}{\kern0pt}\ {\isachardoublequoteopen}t\isactrlsub {\isadigit{0}}\ {\isasymle}\ i{\isachardoublequoteclose}\ {\isachardoublequoteopen}i\ {\isasymle}\ j{\isachardoublequoteclose}\isanewline
\ \ \ \ \isacommand{hence}\isamarkupfalse%
\ {\isachardoublequoteopen}AE\ {\isasymxi}\ in\ M{\isachardot}{\kern0pt}\ X\ i\ {\isasymxi}\ {\isacharplus}{\kern0pt}\ Y\ i\ {\isasymxi}\ {\isasymge}\ cond{\isacharunderscore}{\kern0pt}exp\ M\ {\isacharparenleft}{\kern0pt}F\ i{\isacharparenright}{\kern0pt}\ {\isacharparenleft}{\kern0pt}{\isasymlambda}x{\isachardot}{\kern0pt}\ X\ j\ x\ {\isacharplus}{\kern0pt}\ Y\ j\ x{\isacharparenright}{\kern0pt}\ {\isasymxi}{\isachardoublequoteclose}\ \isanewline
\ \ \ \ \ \ \isacommand{using}\isamarkupfalse%
\ sigma{\isacharunderscore}{\kern0pt}finite{\isacharunderscore}{\kern0pt}subalgebra{\isachardot}{\kern0pt}cond{\isacharunderscore}{\kern0pt}exp{\isacharunderscore}{\kern0pt}add{\isacharbrackleft}{\kern0pt}OF\ {\isacharunderscore}{\kern0pt}\ integrable\ supermartingale{\isachardot}{\kern0pt}integrable{\isacharbrackleft}{\kern0pt}OF\ assms{\isacharbrackright}{\kern0pt}{\isacharcomma}{\kern0pt}\ of\ {\isachardoublequoteopen}F\ i{\isachardoublequoteclose}\ j\ j{\isacharbrackright}{\kern0pt}\ \isanewline
\ \ \ \ \ \ \ \ \ \ \ \ supermartingale{\isacharunderscore}{\kern0pt}property{\isacharbrackleft}{\kern0pt}OF\ asm{\isacharbrackright}{\kern0pt}\ supermartingale{\isachardot}{\kern0pt}supermartingale{\isacharunderscore}{\kern0pt}property{\isacharbrackleft}{\kern0pt}OF\ assms\ asm{\isacharbrackright}{\kern0pt}\ add{\isacharunderscore}{\kern0pt}mono{\isacharbrackleft}{\kern0pt}of\ {\isacharunderscore}{\kern0pt}\ {\isachardoublequoteopen}X\ i\ {\isacharunderscore}{\kern0pt}{\isachardoublequoteclose}\ {\isacharunderscore}{\kern0pt}\ {\isachardoublequoteopen}Y\ i\ {\isacharunderscore}{\kern0pt}{\isachardoublequoteclose}{\isacharbrackright}{\kern0pt}\ \isacommand{by}\isamarkupfalse%
\ force\isanewline
\ \ \isacommand{{\isacharbraceright}{\kern0pt}}\isamarkupfalse%
\isanewline
\ \ \isacommand{thus}\isamarkupfalse%
\ {\isacharquery}{\kern0pt}thesis\ \isacommand{using}\isamarkupfalse%
\ assms\ \isacommand{by}\isamarkupfalse%
\ {\isacharparenleft}{\kern0pt}unfold{\isacharunderscore}{\kern0pt}locales{\isacharparenright}{\kern0pt}\ {\isacharparenleft}{\kern0pt}auto\ simp\ add{\isacharcolon}{\kern0pt}\ borel{\isacharunderscore}{\kern0pt}measurable{\isacharunderscore}{\kern0pt}add\ random{\isacharunderscore}{\kern0pt}variable\ adapted\ integrable\ Y{\isachardot}{\kern0pt}random{\isacharunderscore}{\kern0pt}variable\ Y{\isachardot}{\kern0pt}adapted\ supermartingale{\isachardot}{\kern0pt}integrable{\isacharparenright}{\kern0pt}\ \ \isanewline
\isacommand{qed}\isamarkupfalse%
%
\endisatagproof
{\isafoldproof}%
%
\isadelimproof
\isanewline
%
\endisadelimproof
\isanewline
\isacommand{lemma}\isamarkupfalse%
\ diff{\isacharbrackleft}{\kern0pt}intro{\isacharbrackright}{\kern0pt}{\isacharcolon}{\kern0pt}\isanewline
\ \ \isakeyword{assumes}\ {\isachardoublequoteopen}submartingale\ M\ F\ t\isactrlsub {\isadigit{0}}\ Y{\isachardoublequoteclose}\isanewline
\ \ \isakeyword{shows}\ {\isachardoublequoteopen}supermartingale\ M\ F\ t\isactrlsub {\isadigit{0}}\ {\isacharparenleft}{\kern0pt}{\isasymlambda}i\ {\isasymxi}{\isachardot}{\kern0pt}\ X\ i\ {\isasymxi}\ {\isacharminus}{\kern0pt}\ Y\ i\ {\isasymxi}{\isacharparenright}{\kern0pt}{\isachardoublequoteclose}\isanewline
%
\isadelimproof
%
\endisadelimproof
%
\isatagproof
\isacommand{proof}\isamarkupfalse%
\ {\isacharminus}{\kern0pt}\isanewline
\ \ \isacommand{interpret}\isamarkupfalse%
\ Y{\isacharcolon}{\kern0pt}\ submartingale\ M\ F\ t\isactrlsub {\isadigit{0}}\ Y\ \isacommand{by}\isamarkupfalse%
\ {\isacharparenleft}{\kern0pt}rule\ assms{\isacharparenright}{\kern0pt}\isanewline
\ \ \isacommand{{\isacharbraceleft}{\kern0pt}}\isamarkupfalse%
\isanewline
\ \ \ \ \isacommand{fix}\isamarkupfalse%
\ i\ j\ {\isacharcolon}{\kern0pt}{\isacharcolon}{\kern0pt}\ {\isacharprime}{\kern0pt}b\ \isacommand{assume}\isamarkupfalse%
\ asm{\isacharcolon}{\kern0pt}\ {\isachardoublequoteopen}t\isactrlsub {\isadigit{0}}\ {\isasymle}\ i{\isachardoublequoteclose}\ {\isachardoublequoteopen}i\ {\isasymle}\ j{\isachardoublequoteclose}\isanewline
\ \ \ \ \isacommand{hence}\isamarkupfalse%
\ {\isachardoublequoteopen}AE\ {\isasymxi}\ in\ M{\isachardot}{\kern0pt}\ X\ i\ {\isasymxi}\ {\isacharminus}{\kern0pt}\ Y\ i\ {\isasymxi}\ {\isasymge}\ cond{\isacharunderscore}{\kern0pt}exp\ M\ {\isacharparenleft}{\kern0pt}F\ i{\isacharparenright}{\kern0pt}\ {\isacharparenleft}{\kern0pt}{\isasymlambda}x{\isachardot}{\kern0pt}\ X\ j\ x\ {\isacharminus}{\kern0pt}\ Y\ j\ x{\isacharparenright}{\kern0pt}\ {\isasymxi}{\isachardoublequoteclose}\ \isanewline
\ \ \ \ \ \ \isacommand{using}\isamarkupfalse%
\ sigma{\isacharunderscore}{\kern0pt}finite{\isacharunderscore}{\kern0pt}subalgebra{\isachardot}{\kern0pt}cond{\isacharunderscore}{\kern0pt}exp{\isacharunderscore}{\kern0pt}diff{\isacharbrackleft}{\kern0pt}OF\ {\isacharunderscore}{\kern0pt}\ integrable\ submartingale{\isachardot}{\kern0pt}integrable{\isacharbrackleft}{\kern0pt}OF\ assms{\isacharbrackright}{\kern0pt}{\isacharcomma}{\kern0pt}\ of\ {\isachardoublequoteopen}F\ i{\isachardoublequoteclose}\ j\ j{\isacharcomma}{\kern0pt}\ unfolded\ fun{\isacharunderscore}{\kern0pt}diff{\isacharunderscore}{\kern0pt}def{\isacharbrackright}{\kern0pt}\ \isanewline
\ \ \ \ \ \ \ \ \ \ \ \ supermartingale{\isacharunderscore}{\kern0pt}property{\isacharbrackleft}{\kern0pt}OF\ asm{\isacharbrackright}{\kern0pt}\ submartingale{\isachardot}{\kern0pt}submartingale{\isacharunderscore}{\kern0pt}property{\isacharbrackleft}{\kern0pt}OF\ assms\ asm{\isacharbrackright}{\kern0pt}\ diff{\isacharunderscore}{\kern0pt}mono{\isacharbrackleft}{\kern0pt}of\ {\isacharunderscore}{\kern0pt}\ {\isachardoublequoteopen}X\ i\ {\isacharunderscore}{\kern0pt}{\isachardoublequoteclose}\ {\isachardoublequoteopen}Y\ i\ {\isacharunderscore}{\kern0pt}{\isachardoublequoteclose}{\isacharbrackright}{\kern0pt}\ \isacommand{by}\isamarkupfalse%
\ force\isanewline
\ \ \isacommand{{\isacharbraceright}{\kern0pt}}\isamarkupfalse%
\isanewline
\ \ \isacommand{thus}\isamarkupfalse%
\ {\isacharquery}{\kern0pt}thesis\ \isacommand{using}\isamarkupfalse%
\ assms\ \isacommand{by}\isamarkupfalse%
\ {\isacharparenleft}{\kern0pt}unfold{\isacharunderscore}{\kern0pt}locales{\isacharparenright}{\kern0pt}\ {\isacharparenleft}{\kern0pt}auto\ simp\ add{\isacharcolon}{\kern0pt}\ borel{\isacharunderscore}{\kern0pt}measurable{\isacharunderscore}{\kern0pt}diff\ random{\isacharunderscore}{\kern0pt}variable\ adapted\ integrable\ Y{\isachardot}{\kern0pt}random{\isacharunderscore}{\kern0pt}variable\ Y{\isachardot}{\kern0pt}adapted\ submartingale{\isachardot}{\kern0pt}integrable{\isacharparenright}{\kern0pt}\ \ \isanewline
\isacommand{qed}\isamarkupfalse%
%
\endisatagproof
{\isafoldproof}%
%
\isadelimproof
\isanewline
%
\endisadelimproof
\isanewline
\isacommand{lemma}\isamarkupfalse%
\ scaleR{\isacharunderscore}{\kern0pt}nonneg{\isacharcolon}{\kern0pt}\ \isanewline
\ \ \isakeyword{assumes}\ {\isachardoublequoteopen}c\ {\isasymge}\ {\isadigit{0}}{\isachardoublequoteclose}\isanewline
\ \ \isakeyword{shows}\ {\isachardoublequoteopen}supermartingale\ M\ F\ t\isactrlsub {\isadigit{0}}\ {\isacharparenleft}{\kern0pt}{\isasymlambda}i\ {\isasymxi}{\isachardot}{\kern0pt}\ c\ {\isacharasterisk}{\kern0pt}\isactrlsub R\ X\ i\ {\isasymxi}{\isacharparenright}{\kern0pt}{\isachardoublequoteclose}\isanewline
%
\isadelimproof
%
\endisadelimproof
%
\isatagproof
\isacommand{proof}\isamarkupfalse%
\isanewline
\ \ \isacommand{{\isacharbraceleft}{\kern0pt}}\isamarkupfalse%
\isanewline
\ \ \ \ \isacommand{fix}\isamarkupfalse%
\ i\ j\ {\isacharcolon}{\kern0pt}{\isacharcolon}{\kern0pt}\ {\isacharprime}{\kern0pt}b\ \isacommand{assume}\isamarkupfalse%
\ asm{\isacharcolon}{\kern0pt}\ {\isachardoublequoteopen}t\isactrlsub {\isadigit{0}}\ {\isasymle}\ i{\isachardoublequoteclose}\ {\isachardoublequoteopen}i\ {\isasymle}\ j{\isachardoublequoteclose}\isanewline
\ \ \ \ \isacommand{thus}\isamarkupfalse%
\ {\isachardoublequoteopen}AE\ {\isasymxi}\ in\ M{\isachardot}{\kern0pt}\ c\ {\isacharasterisk}{\kern0pt}\isactrlsub R\ X\ i\ {\isasymxi}\ {\isasymge}\ cond{\isacharunderscore}{\kern0pt}exp\ M\ {\isacharparenleft}{\kern0pt}F\ i{\isacharparenright}{\kern0pt}\ {\isacharparenleft}{\kern0pt}{\isasymlambda}{\isasymxi}{\isachardot}{\kern0pt}\ c\ {\isacharasterisk}{\kern0pt}\isactrlsub R\ X\ j\ {\isasymxi}{\isacharparenright}{\kern0pt}\ {\isasymxi}{\isachardoublequoteclose}\isanewline
\ \ \ \ \ \ \isacommand{using}\isamarkupfalse%
\ sigma{\isacharunderscore}{\kern0pt}finite{\isacharunderscore}{\kern0pt}subalgebra{\isachardot}{\kern0pt}cond{\isacharunderscore}{\kern0pt}exp{\isacharunderscore}{\kern0pt}scaleR{\isacharunderscore}{\kern0pt}right{\isacharbrackleft}{\kern0pt}OF\ {\isacharunderscore}{\kern0pt}\ integrable{\isacharcomma}{\kern0pt}\ of\ {\isachardoublequoteopen}F\ i{\isachardoublequoteclose}\ j\ c{\isacharbrackright}{\kern0pt}\ supermartingale{\isacharunderscore}{\kern0pt}property{\isacharbrackleft}{\kern0pt}OF\ asm{\isacharbrackright}{\kern0pt}\ \isacommand{by}\isamarkupfalse%
\ {\isacharparenleft}{\kern0pt}fastforce\ intro{\isacharbang}{\kern0pt}{\isacharcolon}{\kern0pt}\ scaleR{\isacharunderscore}{\kern0pt}left{\isacharunderscore}{\kern0pt}mono{\isacharbrackleft}{\kern0pt}OF\ {\isacharunderscore}{\kern0pt}\ assms{\isacharbrackright}{\kern0pt}{\isacharparenright}{\kern0pt}\isanewline
\ \ \isacommand{{\isacharbraceright}{\kern0pt}}\isamarkupfalse%
\isanewline
\isacommand{qed}\isamarkupfalse%
\ {\isacharparenleft}{\kern0pt}auto\ simp\ add{\isacharcolon}{\kern0pt}\ borel{\isacharunderscore}{\kern0pt}measurable{\isacharunderscore}{\kern0pt}integrable\ borel{\isacharunderscore}{\kern0pt}measurable{\isacharunderscore}{\kern0pt}scaleR\ integrable\ random{\isacharunderscore}{\kern0pt}variable\ adapted\ borel{\isacharunderscore}{\kern0pt}measurable{\isacharunderscore}{\kern0pt}const{\isacharunderscore}{\kern0pt}scaleR{\isacharparenright}{\kern0pt}%
\endisatagproof
{\isafoldproof}%
%
\isadelimproof
\isanewline
%
\endisadelimproof
\isanewline
\isacommand{lemma}\isamarkupfalse%
\ scaleR{\isacharunderscore}{\kern0pt}le{\isacharunderscore}{\kern0pt}zero{\isacharcolon}{\kern0pt}\ \isanewline
\ \ \isakeyword{assumes}\ {\isachardoublequoteopen}c\ {\isasymle}\ {\isadigit{0}}{\isachardoublequoteclose}\isanewline
\ \ \isakeyword{shows}\ {\isachardoublequoteopen}submartingale\ M\ F\ t\isactrlsub {\isadigit{0}}\ {\isacharparenleft}{\kern0pt}{\isasymlambda}i\ {\isasymxi}{\isachardot}{\kern0pt}\ c\ {\isacharasterisk}{\kern0pt}\isactrlsub R\ X\ i\ {\isasymxi}{\isacharparenright}{\kern0pt}{\isachardoublequoteclose}\isanewline
%
\isadelimproof
%
\endisadelimproof
%
\isatagproof
\isacommand{proof}\isamarkupfalse%
\isanewline
\ \ \isacommand{{\isacharbraceleft}{\kern0pt}}\isamarkupfalse%
\isanewline
\ \ \ \ \isacommand{fix}\isamarkupfalse%
\ i\ j\ {\isacharcolon}{\kern0pt}{\isacharcolon}{\kern0pt}\ {\isacharprime}{\kern0pt}b\ \isacommand{assume}\isamarkupfalse%
\ asm{\isacharcolon}{\kern0pt}\ {\isachardoublequoteopen}t\isactrlsub {\isadigit{0}}\ {\isasymle}\ i{\isachardoublequoteclose}\ {\isachardoublequoteopen}i\ {\isasymle}\ j{\isachardoublequoteclose}\isanewline
\ \ \ \ \isacommand{thus}\isamarkupfalse%
\ {\isachardoublequoteopen}AE\ {\isasymxi}\ in\ M{\isachardot}{\kern0pt}\ c\ {\isacharasterisk}{\kern0pt}\isactrlsub R\ X\ i\ {\isasymxi}\ {\isasymle}\ cond{\isacharunderscore}{\kern0pt}exp\ M\ {\isacharparenleft}{\kern0pt}F\ i{\isacharparenright}{\kern0pt}\ {\isacharparenleft}{\kern0pt}{\isasymlambda}{\isasymxi}{\isachardot}{\kern0pt}\ c\ {\isacharasterisk}{\kern0pt}\isactrlsub R\ X\ j\ {\isasymxi}{\isacharparenright}{\kern0pt}\ {\isasymxi}{\isachardoublequoteclose}\ \isanewline
\ \ \ \ \ \ \isacommand{using}\isamarkupfalse%
\ sigma{\isacharunderscore}{\kern0pt}finite{\isacharunderscore}{\kern0pt}subalgebra{\isachardot}{\kern0pt}cond{\isacharunderscore}{\kern0pt}exp{\isacharunderscore}{\kern0pt}scaleR{\isacharunderscore}{\kern0pt}right{\isacharbrackleft}{\kern0pt}OF\ {\isacharunderscore}{\kern0pt}\ integrable{\isacharcomma}{\kern0pt}\ of\ {\isachardoublequoteopen}F\ i{\isachardoublequoteclose}\ j\ c{\isacharbrackright}{\kern0pt}\ supermartingale{\isacharunderscore}{\kern0pt}property{\isacharbrackleft}{\kern0pt}OF\ asm{\isacharbrackright}{\kern0pt}\ \isacommand{by}\isamarkupfalse%
\ {\isacharparenleft}{\kern0pt}fastforce\ intro{\isacharbang}{\kern0pt}{\isacharcolon}{\kern0pt}\ scaleR{\isacharunderscore}{\kern0pt}left{\isacharunderscore}{\kern0pt}mono{\isacharunderscore}{\kern0pt}neg{\isacharbrackleft}{\kern0pt}OF\ {\isacharunderscore}{\kern0pt}\ assms{\isacharbrackright}{\kern0pt}{\isacharparenright}{\kern0pt}\isanewline
\ \ \isacommand{{\isacharbraceright}{\kern0pt}}\isamarkupfalse%
\isanewline
\isacommand{qed}\isamarkupfalse%
\ {\isacharparenleft}{\kern0pt}auto\ simp\ add{\isacharcolon}{\kern0pt}\ borel{\isacharunderscore}{\kern0pt}measurable{\isacharunderscore}{\kern0pt}integrable\ borel{\isacharunderscore}{\kern0pt}measurable{\isacharunderscore}{\kern0pt}scaleR\ integrable\ random{\isacharunderscore}{\kern0pt}variable\ adapted\ borel{\isacharunderscore}{\kern0pt}measurable{\isacharunderscore}{\kern0pt}const{\isacharunderscore}{\kern0pt}scaleR{\isacharparenright}{\kern0pt}%
\endisatagproof
{\isafoldproof}%
%
\isadelimproof
\isanewline
%
\endisadelimproof
\isanewline
\isacommand{lemma}\isamarkupfalse%
\ uminus{\isacharbrackleft}{\kern0pt}intro{\isacharbrackright}{\kern0pt}{\isacharcolon}{\kern0pt}\isanewline
\ \ \isakeyword{shows}\ {\isachardoublequoteopen}submartingale\ M\ F\ t\isactrlsub {\isadigit{0}}\ {\isacharparenleft}{\kern0pt}{\isacharminus}{\kern0pt}\ X{\isacharparenright}{\kern0pt}{\isachardoublequoteclose}\isanewline
%
\isadelimproof
\ \ %
\endisadelimproof
%
\isatagproof
\isacommand{unfolding}\isamarkupfalse%
\ fun{\isacharunderscore}{\kern0pt}Compl{\isacharunderscore}{\kern0pt}def\ \isacommand{using}\isamarkupfalse%
\ scaleR{\isacharunderscore}{\kern0pt}le{\isacharunderscore}{\kern0pt}zero{\isacharbrackleft}{\kern0pt}of\ {\isachardoublequoteopen}{\isacharminus}{\kern0pt}{\isadigit{1}}{\isachardoublequoteclose}{\isacharbrackright}{\kern0pt}\ \isacommand{by}\isamarkupfalse%
\ simp%
\endisatagproof
{\isafoldproof}%
%
\isadelimproof
\isanewline
%
\endisadelimproof
\isanewline
\isacommand{end}\isamarkupfalse%
\isanewline
\isanewline
\isacommand{context}\isamarkupfalse%
\ supermartingale{\isacharunderscore}{\kern0pt}linorder\isanewline
\isakeyword{begin}\isanewline
\isanewline
\isacommand{lemma}\isamarkupfalse%
\ set{\isacharunderscore}{\kern0pt}integral{\isacharunderscore}{\kern0pt}ge{\isacharcolon}{\kern0pt}\isanewline
\ \ \isakeyword{assumes}\ {\isachardoublequoteopen}A\ {\isasymin}\ F\ i{\isachardoublequoteclose}\ {\isachardoublequoteopen}t\isactrlsub {\isadigit{0}}\ {\isasymle}\ i{\isachardoublequoteclose}\ {\isachardoublequoteopen}i\ {\isasymle}\ j{\isachardoublequoteclose}\isanewline
\ \ \isakeyword{shows}\ {\isachardoublequoteopen}set{\isacharunderscore}{\kern0pt}lebesgue{\isacharunderscore}{\kern0pt}integral\ M\ A\ {\isacharparenleft}{\kern0pt}X\ i{\isacharparenright}{\kern0pt}\ {\isasymge}\ set{\isacharunderscore}{\kern0pt}lebesgue{\isacharunderscore}{\kern0pt}integral\ M\ A\ {\isacharparenleft}{\kern0pt}X\ j{\isacharparenright}{\kern0pt}{\isachardoublequoteclose}\isanewline
%
\isadelimproof
\ \ %
\endisadelimproof
%
\isatagproof
\isacommand{using}\isamarkupfalse%
\ supermartingale{\isacharunderscore}{\kern0pt}property{\isacharbrackleft}{\kern0pt}OF\ assms{\isacharparenleft}{\kern0pt}{\isadigit{2}}{\isacharparenright}{\kern0pt}{\isacharcomma}{\kern0pt}\ of\ j{\isacharbrackright}{\kern0pt}\ assms\ subalgebras\isanewline
\ \ \isacommand{by}\isamarkupfalse%
\ {\isacharparenleft}{\kern0pt}subst\ sigma{\isacharunderscore}{\kern0pt}finite{\isacharunderscore}{\kern0pt}subalgebra{\isachardot}{\kern0pt}cond{\isacharunderscore}{\kern0pt}exp{\isacharunderscore}{\kern0pt}set{\isacharunderscore}{\kern0pt}integral{\isacharbrackleft}{\kern0pt}OF\ {\isacharunderscore}{\kern0pt}\ integrable\ assms{\isacharparenleft}{\kern0pt}{\isadigit{1}}{\isacharparenright}{\kern0pt}{\isacharcomma}{\kern0pt}\ of\ j{\isacharbrackright}{\kern0pt}{\isacharparenright}{\kern0pt}\isanewline
\ \ \ \ \ {\isacharparenleft}{\kern0pt}auto\ intro{\isacharbang}{\kern0pt}{\isacharcolon}{\kern0pt}\ scaleR{\isacharunderscore}{\kern0pt}left{\isacharunderscore}{\kern0pt}mono\ integral{\isacharunderscore}{\kern0pt}mono{\isacharunderscore}{\kern0pt}AE{\isacharunderscore}{\kern0pt}banach\ integrable{\isacharunderscore}{\kern0pt}mult{\isacharunderscore}{\kern0pt}indicator\ integrable\ simp\ add{\isacharcolon}{\kern0pt}\ subalgebra{\isacharunderscore}{\kern0pt}def\ set{\isacharunderscore}{\kern0pt}lebesgue{\isacharunderscore}{\kern0pt}integral{\isacharunderscore}{\kern0pt}def{\isacharparenright}{\kern0pt}%
\endisatagproof
{\isafoldproof}%
%
\isadelimproof
\isanewline
%
\endisadelimproof
\isanewline
\isacommand{lemma}\isamarkupfalse%
\ min{\isacharcolon}{\kern0pt}\isanewline
\ \ \isakeyword{assumes}\ {\isachardoublequoteopen}supermartingale{\isacharunderscore}{\kern0pt}linorder\ M\ F\ t\isactrlsub {\isadigit{0}}\ Y{\isachardoublequoteclose}\isanewline
\ \ \isakeyword{shows}\ {\isachardoublequoteopen}supermartingale{\isacharunderscore}{\kern0pt}linorder\ M\ F\ t\isactrlsub {\isadigit{0}}\ {\isacharparenleft}{\kern0pt}{\isasymlambda}i\ {\isasymxi}{\isachardot}{\kern0pt}\ min\ {\isacharparenleft}{\kern0pt}X\ i\ {\isasymxi}{\isacharparenright}{\kern0pt}\ {\isacharparenleft}{\kern0pt}Y\ i\ {\isasymxi}{\isacharparenright}{\kern0pt}{\isacharparenright}{\kern0pt}{\isachardoublequoteclose}\isanewline
%
\isadelimproof
%
\endisadelimproof
%
\isatagproof
\isacommand{proof}\isamarkupfalse%
\ {\isacharparenleft}{\kern0pt}unfold{\isacharunderscore}{\kern0pt}locales{\isacharparenright}{\kern0pt}\isanewline
\ \ \isacommand{interpret}\isamarkupfalse%
\ Y{\isacharcolon}{\kern0pt}\ supermartingale{\isacharunderscore}{\kern0pt}linorder\ M\ F\ t\isactrlsub {\isadigit{0}}\ Y\ \isacommand{by}\isamarkupfalse%
\ {\isacharparenleft}{\kern0pt}rule\ assms{\isacharparenright}{\kern0pt}\isanewline
\ \ \isacommand{{\isacharbraceleft}{\kern0pt}}\isamarkupfalse%
\isanewline
\ \ \ \ \isacommand{fix}\isamarkupfalse%
\ i\ j\ {\isacharcolon}{\kern0pt}{\isacharcolon}{\kern0pt}\ {\isacharprime}{\kern0pt}b\ \isacommand{assume}\isamarkupfalse%
\ asm{\isacharcolon}{\kern0pt}\ {\isachardoublequoteopen}t\isactrlsub {\isadigit{0}}\ {\isasymle}\ i{\isachardoublequoteclose}\ {\isachardoublequoteopen}i\ {\isasymle}\ j{\isachardoublequoteclose}\isanewline
\ \ \ \ \isacommand{have}\isamarkupfalse%
\ {\isachardoublequoteopen}AE\ {\isasymxi}\ in\ M{\isachardot}{\kern0pt}\ min\ {\isacharparenleft}{\kern0pt}X\ i\ {\isasymxi}{\isacharparenright}{\kern0pt}\ {\isacharparenleft}{\kern0pt}Y\ i\ {\isasymxi}{\isacharparenright}{\kern0pt}\ {\isasymge}\ min\ {\isacharparenleft}{\kern0pt}cond{\isacharunderscore}{\kern0pt}exp\ M\ {\isacharparenleft}{\kern0pt}F\ i{\isacharparenright}{\kern0pt}\ {\isacharparenleft}{\kern0pt}X\ j{\isacharparenright}{\kern0pt}\ {\isasymxi}{\isacharparenright}{\kern0pt}\ {\isacharparenleft}{\kern0pt}cond{\isacharunderscore}{\kern0pt}exp\ M\ {\isacharparenleft}{\kern0pt}F\ i{\isacharparenright}{\kern0pt}\ {\isacharparenleft}{\kern0pt}Y\ j{\isacharparenright}{\kern0pt}\ {\isasymxi}{\isacharparenright}{\kern0pt}{\isachardoublequoteclose}\ \isacommand{using}\isamarkupfalse%
\ supermartingale{\isacharunderscore}{\kern0pt}property\ Y{\isachardot}{\kern0pt}supermartingale{\isacharunderscore}{\kern0pt}property\ asm\ \isacommand{unfolding}\isamarkupfalse%
\ min{\isacharunderscore}{\kern0pt}def\ \isacommand{by}\isamarkupfalse%
\ fastforce\isanewline
\ \ \ \ \isacommand{thus}\isamarkupfalse%
\ {\isachardoublequoteopen}AE\ {\isasymxi}\ in\ M{\isachardot}{\kern0pt}\ min\ {\isacharparenleft}{\kern0pt}X\ i\ {\isasymxi}{\isacharparenright}{\kern0pt}\ {\isacharparenleft}{\kern0pt}Y\ i\ {\isasymxi}{\isacharparenright}{\kern0pt}\ {\isasymge}\ cond{\isacharunderscore}{\kern0pt}exp\ M\ {\isacharparenleft}{\kern0pt}F\ i{\isacharparenright}{\kern0pt}\ {\isacharparenleft}{\kern0pt}{\isasymlambda}{\isasymxi}{\isachardot}{\kern0pt}\ min\ {\isacharparenleft}{\kern0pt}X\ j\ {\isasymxi}{\isacharparenright}{\kern0pt}\ {\isacharparenleft}{\kern0pt}Y\ j\ {\isasymxi}{\isacharparenright}{\kern0pt}{\isacharparenright}{\kern0pt}\ {\isasymxi}{\isachardoublequoteclose}\ \isacommand{using}\isamarkupfalse%
\ sigma{\isacharunderscore}{\kern0pt}finite{\isacharunderscore}{\kern0pt}subalgebra{\isachardot}{\kern0pt}cond{\isacharunderscore}{\kern0pt}exp{\isacharunderscore}{\kern0pt}min{\isacharbrackleft}{\kern0pt}OF\ {\isacharunderscore}{\kern0pt}\ integrable\ Y{\isachardot}{\kern0pt}integrable{\isacharcomma}{\kern0pt}\ of\ {\isachardoublequoteopen}F\ i{\isachardoublequoteclose}\ j\ j{\isacharbrackright}{\kern0pt}\ asm\ \isacommand{by}\isamarkupfalse%
\ {\isacharparenleft}{\kern0pt}fast\ intro{\isacharcolon}{\kern0pt}\ order{\isachardot}{\kern0pt}trans{\isacharparenright}{\kern0pt}\isanewline
\ \ \isacommand{{\isacharbraceright}{\kern0pt}}\isamarkupfalse%
\isanewline
\ \ \isacommand{show}\isamarkupfalse%
\ {\isachardoublequoteopen}{\isasymAnd}i{\isachardot}{\kern0pt}\ t\isactrlsub {\isadigit{0}}\ {\isasymle}\ i\ {\isasymLongrightarrow}\ {\isacharparenleft}{\kern0pt}{\isasymlambda}{\isasymxi}{\isachardot}{\kern0pt}\ min\ {\isacharparenleft}{\kern0pt}X\ i\ {\isasymxi}{\isacharparenright}{\kern0pt}\ {\isacharparenleft}{\kern0pt}Y\ i\ {\isasymxi}{\isacharparenright}{\kern0pt}{\isacharparenright}{\kern0pt}\ {\isasymin}\ borel{\isacharunderscore}{\kern0pt}measurable\ {\isacharparenleft}{\kern0pt}F\ i{\isacharparenright}{\kern0pt}{\isachardoublequoteclose}\ {\isachardoublequoteopen}{\isasymAnd}i{\isachardot}{\kern0pt}\ t\isactrlsub {\isadigit{0}}\ {\isasymle}\ i\ {\isasymLongrightarrow}\ integrable\ M\ {\isacharparenleft}{\kern0pt}{\isasymlambda}{\isasymxi}{\isachardot}{\kern0pt}\ min\ {\isacharparenleft}{\kern0pt}X\ i\ {\isasymxi}{\isacharparenright}{\kern0pt}\ {\isacharparenleft}{\kern0pt}Y\ i\ {\isasymxi}{\isacharparenright}{\kern0pt}{\isacharparenright}{\kern0pt}{\isachardoublequoteclose}\ \isacommand{by}\isamarkupfalse%
\ {\isacharparenleft}{\kern0pt}force\ intro{\isacharcolon}{\kern0pt}\ Y{\isachardot}{\kern0pt}integrable\ integrable\ assms{\isacharparenright}{\kern0pt}{\isacharplus}{\kern0pt}\isanewline
\isacommand{qed}\isamarkupfalse%
%
\endisatagproof
{\isafoldproof}%
%
\isadelimproof
\isanewline
%
\endisadelimproof
\isanewline
\isacommand{lemma}\isamarkupfalse%
\ min{\isacharunderscore}{\kern0pt}{\isadigit{0}}{\isacharcolon}{\kern0pt}\isanewline
\ \ \isakeyword{shows}\ {\isachardoublequoteopen}supermartingale{\isacharunderscore}{\kern0pt}linorder\ M\ F\ t\isactrlsub {\isadigit{0}}\ {\isacharparenleft}{\kern0pt}{\isasymlambda}i\ {\isasymxi}{\isachardot}{\kern0pt}\ min\ {\isadigit{0}}\ {\isacharparenleft}{\kern0pt}X\ i\ {\isasymxi}{\isacharparenright}{\kern0pt}{\isacharparenright}{\kern0pt}{\isachardoublequoteclose}\isanewline
%
\isadelimproof
%
\endisadelimproof
%
\isatagproof
\isacommand{proof}\isamarkupfalse%
\ {\isacharminus}{\kern0pt}\isanewline
\ \ \isacommand{interpret}\isamarkupfalse%
\ zero{\isacharcolon}{\kern0pt}\ martingale{\isacharunderscore}{\kern0pt}linorder\ M\ F\ t\isactrlsub {\isadigit{0}}\ {\isachardoublequoteopen}{\isasymlambda}{\isacharunderscore}{\kern0pt}\ {\isacharunderscore}{\kern0pt}{\isachardot}{\kern0pt}\ {\isadigit{0}}{\isachardoublequoteclose}\ \isacommand{by}\isamarkupfalse%
\ {\isacharparenleft}{\kern0pt}force\ intro{\isacharcolon}{\kern0pt}\ martingale{\isacharunderscore}{\kern0pt}linorder{\isachardot}{\kern0pt}intro{\isacharparenright}{\kern0pt}\isanewline
\ \ \isacommand{show}\isamarkupfalse%
\ {\isacharquery}{\kern0pt}thesis\ \isacommand{by}\isamarkupfalse%
\ {\isacharparenleft}{\kern0pt}intro\ zero{\isachardot}{\kern0pt}min\ supermartingale{\isacharunderscore}{\kern0pt}linorder{\isachardot}{\kern0pt}intro\ supermartingale{\isacharunderscore}{\kern0pt}axioms{\isacharparenright}{\kern0pt}\isanewline
\isacommand{qed}\isamarkupfalse%
%
\endisatagproof
{\isafoldproof}%
%
\isadelimproof
\isanewline
%
\endisadelimproof
\isanewline
\isacommand{end}\isamarkupfalse%
\isanewline
\isanewline
\isacommand{lemma}\isamarkupfalse%
\ {\isacharparenleft}{\kern0pt}\isakeyword{in}\ sigma{\isacharunderscore}{\kern0pt}finite{\isacharunderscore}{\kern0pt}adapted{\isacharunderscore}{\kern0pt}process{\isacharunderscore}{\kern0pt}order{\isacharparenright}{\kern0pt}\ supermartingale{\isacharunderscore}{\kern0pt}of{\isacharunderscore}{\kern0pt}cond{\isacharunderscore}{\kern0pt}exp{\isacharunderscore}{\kern0pt}diff{\isacharunderscore}{\kern0pt}le{\isacharunderscore}{\kern0pt}zero{\isacharcolon}{\kern0pt}\isanewline
\ \ \isakeyword{assumes}\ integrable{\isacharcolon}{\kern0pt}\ {\isachardoublequoteopen}{\isasymAnd}i{\isachardot}{\kern0pt}\ t\isactrlsub {\isadigit{0}}\ {\isasymle}\ i\ {\isasymLongrightarrow}\ integrable\ M\ {\isacharparenleft}{\kern0pt}X\ i{\isacharparenright}{\kern0pt}{\isachardoublequoteclose}\ \isanewline
\ \ \ \ \ \ \isakeyword{and}\ diff{\isacharunderscore}{\kern0pt}le{\isacharunderscore}{\kern0pt}zero{\isacharcolon}{\kern0pt}\ {\isachardoublequoteopen}{\isasymAnd}i\ j{\isachardot}{\kern0pt}\ t\isactrlsub {\isadigit{0}}\ {\isasymle}\ i\ {\isasymLongrightarrow}\ i\ {\isasymle}\ j\ {\isasymLongrightarrow}\ AE\ x\ in\ M{\isachardot}{\kern0pt}\ cond{\isacharunderscore}{\kern0pt}exp\ M\ {\isacharparenleft}{\kern0pt}F\ i{\isacharparenright}{\kern0pt}\ {\isacharparenleft}{\kern0pt}{\isasymlambda}{\isasymxi}{\isachardot}{\kern0pt}\ X\ j\ {\isasymxi}\ {\isacharminus}{\kern0pt}\ X\ i\ {\isasymxi}{\isacharparenright}{\kern0pt}\ x\ {\isasymle}\ {\isadigit{0}}{\isachardoublequoteclose}\isanewline
\ \ \ \ \isakeyword{shows}\ {\isachardoublequoteopen}supermartingale\ M\ F\ t\isactrlsub {\isadigit{0}}\ X{\isachardoublequoteclose}\isanewline
%
\isadelimproof
%
\endisadelimproof
%
\isatagproof
\isacommand{proof}\isamarkupfalse%
\ \isanewline
\ \ \isacommand{{\isacharbraceleft}{\kern0pt}}\isamarkupfalse%
\isanewline
\ \ \ \ \isacommand{fix}\isamarkupfalse%
\ i\ j\ {\isacharcolon}{\kern0pt}{\isacharcolon}{\kern0pt}\ {\isacharprime}{\kern0pt}b\ \isacommand{assume}\isamarkupfalse%
\ asm{\isacharcolon}{\kern0pt}\ {\isachardoublequoteopen}t\isactrlsub {\isadigit{0}}\ {\isasymle}\ i{\isachardoublequoteclose}\ {\isachardoublequoteopen}i\ {\isasymle}\ j{\isachardoublequoteclose}\isanewline
\ \ \ \ \isacommand{thus}\isamarkupfalse%
\ {\isachardoublequoteopen}AE\ {\isasymxi}\ in\ M{\isachardot}{\kern0pt}\ X\ i\ {\isasymxi}\ {\isasymge}\ cond{\isacharunderscore}{\kern0pt}exp\ M\ {\isacharparenleft}{\kern0pt}F\ i{\isacharparenright}{\kern0pt}\ {\isacharparenleft}{\kern0pt}X\ j{\isacharparenright}{\kern0pt}\ {\isasymxi}{\isachardoublequoteclose}\ \isanewline
\ \ \ \ \ \ \isacommand{using}\isamarkupfalse%
\ diff{\isacharunderscore}{\kern0pt}le{\isacharunderscore}{\kern0pt}zero{\isacharbrackleft}{\kern0pt}OF\ asm{\isacharbrackright}{\kern0pt}\ sigma{\isacharunderscore}{\kern0pt}finite{\isacharunderscore}{\kern0pt}subalgebra{\isachardot}{\kern0pt}cond{\isacharunderscore}{\kern0pt}exp{\isacharunderscore}{\kern0pt}diff{\isacharbrackleft}{\kern0pt}OF\ {\isacharunderscore}{\kern0pt}\ integrable{\isacharparenleft}{\kern0pt}{\isadigit{1}}{\isacharcomma}{\kern0pt}{\isadigit{1}}{\isacharparenright}{\kern0pt}{\isacharcomma}{\kern0pt}\ of\ {\isachardoublequoteopen}F\ i{\isachardoublequoteclose}\ j\ i{\isacharbrackright}{\kern0pt}\ \isanewline
\ \ \ \ \ \ \ \ \ \ \ \ sigma{\isacharunderscore}{\kern0pt}finite{\isacharunderscore}{\kern0pt}subalgebra{\isachardot}{\kern0pt}cond{\isacharunderscore}{\kern0pt}exp{\isacharunderscore}{\kern0pt}F{\isacharunderscore}{\kern0pt}meas{\isacharbrackleft}{\kern0pt}OF\ {\isacharunderscore}{\kern0pt}\ integrable\ adapted{\isacharcomma}{\kern0pt}\ of\ i{\isacharbrackright}{\kern0pt}\ \isacommand{by}\isamarkupfalse%
\ fastforce\isanewline
\ \ \isacommand{{\isacharbraceright}{\kern0pt}}\isamarkupfalse%
\isanewline
\isacommand{qed}\isamarkupfalse%
\ {\isacharparenleft}{\kern0pt}intro\ integrable{\isacharparenright}{\kern0pt}%
\endisatagproof
{\isafoldproof}%
%
\isadelimproof
\isanewline
%
\endisadelimproof
\isanewline
\isacommand{lemma}\isamarkupfalse%
\ {\isacharparenleft}{\kern0pt}\isakeyword{in}\ sigma{\isacharunderscore}{\kern0pt}finite{\isacharunderscore}{\kern0pt}adapted{\isacharunderscore}{\kern0pt}process{\isacharunderscore}{\kern0pt}linorder{\isacharparenright}{\kern0pt}\ supermartingale{\isacharunderscore}{\kern0pt}of{\isacharunderscore}{\kern0pt}set{\isacharunderscore}{\kern0pt}integral{\isacharunderscore}{\kern0pt}ge{\isacharcolon}{\kern0pt}\isanewline
\ \ \isakeyword{assumes}\ integrable{\isacharcolon}{\kern0pt}\ {\isachardoublequoteopen}{\isasymAnd}i{\isachardot}{\kern0pt}\ t\isactrlsub {\isadigit{0}}\ {\isasymle}\ i\ {\isasymLongrightarrow}\ integrable\ M\ {\isacharparenleft}{\kern0pt}X\ i{\isacharparenright}{\kern0pt}{\isachardoublequoteclose}\ \isanewline
\ \ \ \ \ \ \isakeyword{and}\ {\isachardoublequoteopen}{\isasymAnd}A\ i\ j{\isachardot}{\kern0pt}\ t\isactrlsub {\isadigit{0}}\ {\isasymle}\ i\ {\isasymLongrightarrow}\ i\ {\isasymle}\ j\ {\isasymLongrightarrow}\ A\ {\isasymin}\ F\ i\ {\isasymLongrightarrow}\ set{\isacharunderscore}{\kern0pt}lebesgue{\isacharunderscore}{\kern0pt}integral\ M\ A\ {\isacharparenleft}{\kern0pt}X\ j{\isacharparenright}{\kern0pt}\ {\isasymle}\ set{\isacharunderscore}{\kern0pt}lebesgue{\isacharunderscore}{\kern0pt}integral\ M\ A\ {\isacharparenleft}{\kern0pt}X\ i{\isacharparenright}{\kern0pt}{\isachardoublequoteclose}\ \isanewline
\ \ \ \ \isakeyword{shows}\ {\isachardoublequoteopen}supermartingale\ M\ F\ t\isactrlsub {\isadigit{0}}\ X{\isachardoublequoteclose}\isanewline
%
\isadelimproof
%
\endisadelimproof
%
\isatagproof
\isacommand{proof}\isamarkupfalse%
\ {\isacharminus}{\kern0pt}\isanewline
\ \ \isacommand{interpret}\isamarkupfalse%
\ {\isacharunderscore}{\kern0pt}{\isacharcolon}{\kern0pt}\ adapted{\isacharunderscore}{\kern0pt}process\ M\ F\ t\isactrlsub {\isadigit{0}}\ {\isachardoublequoteopen}{\isacharminus}{\kern0pt}X{\isachardoublequoteclose}\ \isacommand{by}\isamarkupfalse%
\ {\isacharparenleft}{\kern0pt}rule\ uminus{\isacharunderscore}{\kern0pt}adapted{\isacharparenright}{\kern0pt}\isanewline
\ \ \isacommand{interpret}\isamarkupfalse%
\ uminus{\isacharunderscore}{\kern0pt}X{\isacharcolon}{\kern0pt}\ sigma{\isacharunderscore}{\kern0pt}finite{\isacharunderscore}{\kern0pt}adapted{\isacharunderscore}{\kern0pt}process{\isacharunderscore}{\kern0pt}linorder\ M\ F\ t\isactrlsub {\isadigit{0}}\ {\isachardoublequoteopen}{\isacharminus}{\kern0pt}X{\isachardoublequoteclose}\ \isacommand{{\isachardot}{\kern0pt}{\isachardot}{\kern0pt}}\isamarkupfalse%
\isanewline
\ \ \isacommand{note}\isamarkupfalse%
\ {\isacharasterisk}{\kern0pt}\ {\isacharequal}{\kern0pt}\ set{\isacharunderscore}{\kern0pt}integral{\isacharunderscore}{\kern0pt}uminus{\isacharbrackleft}{\kern0pt}unfolded\ set{\isacharunderscore}{\kern0pt}integrable{\isacharunderscore}{\kern0pt}def{\isacharcomma}{\kern0pt}\ OF\ integrable{\isacharunderscore}{\kern0pt}mult{\isacharunderscore}{\kern0pt}indicator{\isacharbrackleft}{\kern0pt}OF\ {\isacharunderscore}{\kern0pt}\ integrable{\isacharbrackright}{\kern0pt}{\isacharbrackright}{\kern0pt}\isanewline
\ \ \isacommand{have}\isamarkupfalse%
\ {\isachardoublequoteopen}supermartingale\ M\ F\ t\isactrlsub {\isadigit{0}}\ {\isacharparenleft}{\kern0pt}{\isacharminus}{\kern0pt}{\isacharparenleft}{\kern0pt}{\isacharminus}{\kern0pt}\ X{\isacharparenright}{\kern0pt}{\isacharparenright}{\kern0pt}{\isachardoublequoteclose}\isanewline
\ \ \ \ \isacommand{using}\isamarkupfalse%
\ ord{\isacharunderscore}{\kern0pt}eq{\isacharunderscore}{\kern0pt}le{\isacharunderscore}{\kern0pt}trans{\isacharbrackleft}{\kern0pt}OF\ {\isacharasterisk}{\kern0pt}\ ord{\isacharunderscore}{\kern0pt}le{\isacharunderscore}{\kern0pt}eq{\isacharunderscore}{\kern0pt}trans{\isacharbrackleft}{\kern0pt}OF\ le{\isacharunderscore}{\kern0pt}imp{\isacharunderscore}{\kern0pt}neg{\isacharunderscore}{\kern0pt}le{\isacharbrackleft}{\kern0pt}OF\ assms{\isacharparenleft}{\kern0pt}{\isadigit{2}}{\isacharparenright}{\kern0pt}{\isacharbrackright}{\kern0pt}\ {\isacharasterisk}{\kern0pt}{\isacharbrackleft}{\kern0pt}symmetric{\isacharbrackright}{\kern0pt}{\isacharbrackright}{\kern0pt}{\isacharbrackright}{\kern0pt}\ subalgebras\isanewline
\ \ \ \ \isacommand{by}\isamarkupfalse%
\ {\isacharparenleft}{\kern0pt}intro\ submartingale{\isachardot}{\kern0pt}uminus\ uminus{\isacharunderscore}{\kern0pt}X{\isachardot}{\kern0pt}submartingale{\isacharunderscore}{\kern0pt}of{\isacharunderscore}{\kern0pt}set{\isacharunderscore}{\kern0pt}integral{\isacharunderscore}{\kern0pt}le{\isacharparenright}{\kern0pt}\ \isanewline
\ \ \ \ \ \ \ {\isacharparenleft}{\kern0pt}clarsimp\ simp\ add{\isacharcolon}{\kern0pt}\ fun{\isacharunderscore}{\kern0pt}Compl{\isacharunderscore}{\kern0pt}def\ subalgebra{\isacharunderscore}{\kern0pt}def\ integrable\ {\isacharbar}{\kern0pt}\ fastforce{\isacharparenright}{\kern0pt}{\isacharplus}{\kern0pt}\isanewline
\ \ \isacommand{thus}\isamarkupfalse%
\ {\isacharquery}{\kern0pt}thesis\ \isacommand{unfolding}\isamarkupfalse%
\ fun{\isacharunderscore}{\kern0pt}Compl{\isacharunderscore}{\kern0pt}def\ \isacommand{by}\isamarkupfalse%
\ simp\isanewline
\isacommand{qed}\isamarkupfalse%
%
\endisatagproof
{\isafoldproof}%
%
\isadelimproof
%
\endisadelimproof
%
\isadelimdocument
%
\endisadelimdocument
%
\isatagdocument
%
\isamarkupsubsection{Discrete Time Martingales%
}
\isamarkuptrue%
%
\endisatagdocument
{\isafolddocument}%
%
\isadelimdocument
%
\endisadelimdocument
\isacommand{locale}\isamarkupfalse%
\ nat{\isacharunderscore}{\kern0pt}martingale\ {\isacharequal}{\kern0pt}\ martingale\ M\ F\ {\isachardoublequoteopen}{\isadigit{0}}\ {\isacharcolon}{\kern0pt}{\isacharcolon}{\kern0pt}\ nat{\isachardoublequoteclose}\ X\ \isakeyword{for}\ M\ F\ X\isanewline
\isacommand{locale}\isamarkupfalse%
\ nat{\isacharunderscore}{\kern0pt}submartingale\ {\isacharequal}{\kern0pt}\ submartingale\ M\ F\ {\isachardoublequoteopen}{\isadigit{0}}\ {\isacharcolon}{\kern0pt}{\isacharcolon}{\kern0pt}\ nat{\isachardoublequoteclose}\ X\ \isakeyword{for}\ M\ F\ X\isanewline
\isacommand{locale}\isamarkupfalse%
\ nat{\isacharunderscore}{\kern0pt}supermartingale\ {\isacharequal}{\kern0pt}\ supermartingale\ M\ F\ {\isachardoublequoteopen}{\isadigit{0}}\ {\isacharcolon}{\kern0pt}{\isacharcolon}{\kern0pt}\ nat{\isachardoublequoteclose}\ X\ \isakeyword{for}\ M\ F\ X\isanewline
\isanewline
\isacommand{locale}\isamarkupfalse%
\ nat{\isacharunderscore}{\kern0pt}submartingale{\isacharunderscore}{\kern0pt}linorder\ {\isacharequal}{\kern0pt}\ submartingale{\isacharunderscore}{\kern0pt}linorder\ M\ F\ {\isachardoublequoteopen}{\isadigit{0}}\ {\isacharcolon}{\kern0pt}{\isacharcolon}{\kern0pt}\ nat{\isachardoublequoteclose}\ X\ \isakeyword{for}\ M\ F\ X\isanewline
\isacommand{locale}\isamarkupfalse%
\ nat{\isacharunderscore}{\kern0pt}supermartingale{\isacharunderscore}{\kern0pt}linorder\ {\isacharequal}{\kern0pt}\ supermartingale{\isacharunderscore}{\kern0pt}linorder\ M\ F\ {\isachardoublequoteopen}{\isadigit{0}}\ {\isacharcolon}{\kern0pt}{\isacharcolon}{\kern0pt}\ nat{\isachardoublequoteclose}\ X\ \isakeyword{for}\ M\ F\ X\isanewline
\isanewline
\isacommand{sublocale}\isamarkupfalse%
\ nat{\isacharunderscore}{\kern0pt}submartingale{\isacharunderscore}{\kern0pt}linorder\ {\isasymsubseteq}\ nat{\isacharunderscore}{\kern0pt}submartingale%
\isadelimproof
\ %
\endisadelimproof
%
\isatagproof
\isacommand{{\isachardot}{\kern0pt}{\isachardot}{\kern0pt}}\isamarkupfalse%
%
\endisatagproof
{\isafoldproof}%
%
\isadelimproof
%
\endisadelimproof
\isanewline
\isacommand{sublocale}\isamarkupfalse%
\ nat{\isacharunderscore}{\kern0pt}supermartingale{\isacharunderscore}{\kern0pt}linorder\ {\isasymsubseteq}\ nat{\isacharunderscore}{\kern0pt}supermartingale%
\isadelimproof
\ %
\endisadelimproof
%
\isatagproof
\isacommand{{\isachardot}{\kern0pt}{\isachardot}{\kern0pt}}\isamarkupfalse%
%
\endisatagproof
{\isafoldproof}%
%
\isadelimproof
%
\endisadelimproof
\isanewline
\isanewline
\isacommand{lemma}\isamarkupfalse%
\ {\isacharparenleft}{\kern0pt}\isakeyword{in}\ nat{\isacharunderscore}{\kern0pt}martingale{\isacharparenright}{\kern0pt}\ predictable{\isacharunderscore}{\kern0pt}const{\isacharcolon}{\kern0pt}\isanewline
\ \ \isakeyword{assumes}\ {\isachardoublequoteopen}nat{\isacharunderscore}{\kern0pt}predictable{\isacharunderscore}{\kern0pt}process\ M\ F\ X{\isachardoublequoteclose}\isanewline
\ \ \isakeyword{shows}\ {\isachardoublequoteopen}AE\ {\isasymxi}\ in\ M{\isachardot}{\kern0pt}\ X\ i\ {\isasymxi}\ {\isacharequal}{\kern0pt}\ X\ j\ {\isasymxi}{\isachardoublequoteclose}\isanewline
%
\isadelimproof
%
\endisadelimproof
%
\isatagproof
\isacommand{proof}\isamarkupfalse%
\ {\isacharminus}{\kern0pt}\isanewline
\ \ \isacommand{have}\isamarkupfalse%
\ {\isacharasterisk}{\kern0pt}{\isacharcolon}{\kern0pt}\ {\isachardoublequoteopen}AE\ {\isasymxi}\ in\ M{\isachardot}{\kern0pt}\ X\ i\ {\isasymxi}\ {\isacharequal}{\kern0pt}\ X\ {\isadigit{0}}\ {\isasymxi}{\isachardoublequoteclose}\ \isakeyword{for}\ i\isanewline
\ \ \isacommand{proof}\isamarkupfalse%
\ {\isacharparenleft}{\kern0pt}induction\ i{\isacharparenright}{\kern0pt}\isanewline
\ \ \ \ \isacommand{case}\isamarkupfalse%
\ {\isadigit{0}}\isanewline
\ \ \ \ \isacommand{then}\isamarkupfalse%
\ \isacommand{show}\isamarkupfalse%
\ {\isacharquery}{\kern0pt}case\ \isacommand{by}\isamarkupfalse%
\ {\isacharparenleft}{\kern0pt}simp\ add{\isacharcolon}{\kern0pt}\ bot{\isacharunderscore}{\kern0pt}nat{\isacharunderscore}{\kern0pt}def{\isacharparenright}{\kern0pt}\isanewline
\ \ \isacommand{next}\isamarkupfalse%
\isanewline
\ \ \ \ \isacommand{case}\isamarkupfalse%
\ {\isacharparenleft}{\kern0pt}Suc\ i{\isacharparenright}{\kern0pt}\isanewline
\ \ \ \ \isacommand{interpret}\isamarkupfalse%
\ S{\isacharcolon}{\kern0pt}\ nat{\isacharunderscore}{\kern0pt}adapted{\isacharunderscore}{\kern0pt}process\ M\ F\ {\isachardoublequoteopen}{\isasymlambda}i{\isachardot}{\kern0pt}\ X\ {\isacharparenleft}{\kern0pt}Suc\ i{\isacharparenright}{\kern0pt}{\isachardoublequoteclose}\ \isacommand{by}\isamarkupfalse%
\ {\isacharparenleft}{\kern0pt}intro\ nat{\isacharunderscore}{\kern0pt}predictable{\isacharunderscore}{\kern0pt}process{\isachardot}{\kern0pt}adapted{\isacharunderscore}{\kern0pt}Suc\ assms{\isacharparenright}{\kern0pt}\isanewline
\ \ \ \ \isacommand{show}\isamarkupfalse%
\ {\isacharquery}{\kern0pt}case\ \isacommand{using}\isamarkupfalse%
\ Suc\ S{\isachardot}{\kern0pt}adapted{\isacharbrackleft}{\kern0pt}of\ i{\isacharbrackright}{\kern0pt}\ martingale{\isacharunderscore}{\kern0pt}property{\isacharbrackleft}{\kern0pt}OF\ {\isacharunderscore}{\kern0pt}\ le{\isacharunderscore}{\kern0pt}SucI{\isacharcomma}{\kern0pt}\ of\ i{\isacharbrackright}{\kern0pt}\ sigma{\isacharunderscore}{\kern0pt}finite{\isacharunderscore}{\kern0pt}subalgebra{\isachardot}{\kern0pt}cond{\isacharunderscore}{\kern0pt}exp{\isacharunderscore}{\kern0pt}F{\isacharunderscore}{\kern0pt}meas{\isacharbrackleft}{\kern0pt}OF\ {\isacharunderscore}{\kern0pt}\ integrable{\isacharcomma}{\kern0pt}\ of\ {\isachardoublequoteopen}F\ i{\isachardoublequoteclose}\ {\isachardoublequoteopen}Suc\ i{\isachardoublequoteclose}{\isacharbrackright}{\kern0pt}\ \isacommand{by}\isamarkupfalse%
\ fastforce\isanewline
\ \ \isacommand{qed}\isamarkupfalse%
\isanewline
\ \ \isacommand{show}\isamarkupfalse%
\ {\isacharquery}{\kern0pt}thesis\ \isacommand{using}\isamarkupfalse%
\ {\isacharasterisk}{\kern0pt}{\isacharbrackleft}{\kern0pt}of\ i{\isacharbrackright}{\kern0pt}\ {\isacharasterisk}{\kern0pt}{\isacharbrackleft}{\kern0pt}of\ j{\isacharbrackright}{\kern0pt}\ \isacommand{by}\isamarkupfalse%
\ force\isanewline
\isacommand{qed}\isamarkupfalse%
%
\endisatagproof
{\isafoldproof}%
%
\isadelimproof
\isanewline
%
\endisadelimproof
\isanewline
\isacommand{lemma}\isamarkupfalse%
\ {\isacharparenleft}{\kern0pt}\isakeyword{in}\ nat{\isacharunderscore}{\kern0pt}sigma{\isacharunderscore}{\kern0pt}finite{\isacharunderscore}{\kern0pt}adapted{\isacharunderscore}{\kern0pt}process{\isacharparenright}{\kern0pt}\ martingale{\isacharunderscore}{\kern0pt}of{\isacharunderscore}{\kern0pt}set{\isacharunderscore}{\kern0pt}integral{\isacharunderscore}{\kern0pt}eq{\isacharunderscore}{\kern0pt}Suc{\isacharcolon}{\kern0pt}\isanewline
\ \ \isakeyword{assumes}\ integrable{\isacharcolon}{\kern0pt}\ {\isachardoublequoteopen}{\isasymAnd}i{\isachardot}{\kern0pt}\ integrable\ M\ {\isacharparenleft}{\kern0pt}X\ i{\isacharparenright}{\kern0pt}{\isachardoublequoteclose}\isanewline
\ \ \ \ \ \ \isakeyword{and}\ {\isachardoublequoteopen}{\isasymAnd}A\ i{\isachardot}{\kern0pt}\ A\ {\isasymin}\ F\ i\ {\isasymLongrightarrow}\ set{\isacharunderscore}{\kern0pt}lebesgue{\isacharunderscore}{\kern0pt}integral\ M\ A\ {\isacharparenleft}{\kern0pt}X\ i{\isacharparenright}{\kern0pt}\ {\isacharequal}{\kern0pt}\ set{\isacharunderscore}{\kern0pt}lebesgue{\isacharunderscore}{\kern0pt}integral\ M\ A\ {\isacharparenleft}{\kern0pt}X\ {\isacharparenleft}{\kern0pt}Suc\ i{\isacharparenright}{\kern0pt}{\isacharparenright}{\kern0pt}{\isachardoublequoteclose}\ \isanewline
\ \ \ \ \isakeyword{shows}\ {\isachardoublequoteopen}nat{\isacharunderscore}{\kern0pt}martingale\ M\ F\ X{\isachardoublequoteclose}\isanewline
%
\isadelimproof
%
\endisadelimproof
%
\isatagproof
\isacommand{proof}\isamarkupfalse%
\ {\isacharparenleft}{\kern0pt}intro\ nat{\isacharunderscore}{\kern0pt}martingale{\isachardot}{\kern0pt}intro\ martingale{\isacharunderscore}{\kern0pt}of{\isacharunderscore}{\kern0pt}set{\isacharunderscore}{\kern0pt}integral{\isacharunderscore}{\kern0pt}eq{\isacharparenright}{\kern0pt}\isanewline
\ \ \isacommand{fix}\isamarkupfalse%
\ i\ j\ A\ \isacommand{assume}\isamarkupfalse%
\ asm{\isacharcolon}{\kern0pt}\ {\isachardoublequoteopen}i\ {\isasymle}\ j{\isachardoublequoteclose}\ {\isachardoublequoteopen}A\ {\isasymin}\ sets\ {\isacharparenleft}{\kern0pt}F\ i{\isacharparenright}{\kern0pt}{\isachardoublequoteclose}\isanewline
\ \ \isacommand{show}\isamarkupfalse%
\ {\isachardoublequoteopen}set{\isacharunderscore}{\kern0pt}lebesgue{\isacharunderscore}{\kern0pt}integral\ M\ A\ {\isacharparenleft}{\kern0pt}X\ i{\isacharparenright}{\kern0pt}\ {\isacharequal}{\kern0pt}\ set{\isacharunderscore}{\kern0pt}lebesgue{\isacharunderscore}{\kern0pt}integral\ M\ A\ {\isacharparenleft}{\kern0pt}X\ j{\isacharparenright}{\kern0pt}{\isachardoublequoteclose}\ \isacommand{using}\isamarkupfalse%
\ asm\isanewline
\ \ \isacommand{proof}\isamarkupfalse%
\ {\isacharparenleft}{\kern0pt}induction\ {\isachardoublequoteopen}j\ {\isacharminus}{\kern0pt}\ i{\isachardoublequoteclose}\ arbitrary{\isacharcolon}{\kern0pt}\ i\ j{\isacharparenright}{\kern0pt}\isanewline
\ \ \ \ \isacommand{case}\isamarkupfalse%
\ {\isadigit{0}}\isanewline
\ \ \ \ \isacommand{then}\isamarkupfalse%
\ \isacommand{show}\isamarkupfalse%
\ {\isacharquery}{\kern0pt}case\ \isacommand{using}\isamarkupfalse%
\ asm\ \isacommand{by}\isamarkupfalse%
\ simp\isanewline
\ \ \isacommand{next}\isamarkupfalse%
\isanewline
\ \ \ \ \isacommand{case}\isamarkupfalse%
\ {\isacharparenleft}{\kern0pt}Suc\ n{\isacharparenright}{\kern0pt}\isanewline
\ \ \ \ \isacommand{hence}\isamarkupfalse%
\ {\isacharasterisk}{\kern0pt}{\isacharcolon}{\kern0pt}\ {\isachardoublequoteopen}n\ {\isacharequal}{\kern0pt}\ j\ {\isacharminus}{\kern0pt}\ Suc\ i{\isachardoublequoteclose}\ \isacommand{by}\isamarkupfalse%
\ linarith\isanewline
\ \ \ \ \isacommand{have}\isamarkupfalse%
\ {\isachardoublequoteopen}Suc\ i\ {\isasymle}\ j{\isachardoublequoteclose}\ \isacommand{using}\isamarkupfalse%
\ Suc{\isacharparenleft}{\kern0pt}{\isadigit{2}}{\isacharcomma}{\kern0pt}{\isadigit{3}}{\isacharparenright}{\kern0pt}\ \isacommand{by}\isamarkupfalse%
\ linarith\isanewline
\ \ \ \ \isacommand{thus}\isamarkupfalse%
\ {\isacharquery}{\kern0pt}case\ \isacommand{using}\isamarkupfalse%
\ sets{\isacharunderscore}{\kern0pt}F{\isacharunderscore}{\kern0pt}mono{\isacharbrackleft}{\kern0pt}OF\ {\isacharunderscore}{\kern0pt}\ le{\isacharunderscore}{\kern0pt}SucI{\isacharbrackright}{\kern0pt}\ Suc{\isacharparenleft}{\kern0pt}{\isadigit{4}}{\isacharparenright}{\kern0pt}\ Suc{\isacharparenleft}{\kern0pt}{\isadigit{1}}{\isacharparenright}{\kern0pt}{\isacharbrackleft}{\kern0pt}OF\ {\isacharasterisk}{\kern0pt}{\isacharbrackright}{\kern0pt}\ \isacommand{by}\isamarkupfalse%
\ {\isacharparenleft}{\kern0pt}auto\ intro{\isacharcolon}{\kern0pt}\ assms{\isacharparenleft}{\kern0pt}{\isadigit{2}}{\isacharparenright}{\kern0pt}{\isacharbrackleft}{\kern0pt}THEN\ trans{\isacharbrackright}{\kern0pt}{\isacharparenright}{\kern0pt}\isanewline
\ \ \isacommand{qed}\isamarkupfalse%
\isanewline
\isacommand{qed}\isamarkupfalse%
\ {\isacharparenleft}{\kern0pt}simp\ add{\isacharcolon}{\kern0pt}\ integrable{\isacharparenright}{\kern0pt}%
\endisatagproof
{\isafoldproof}%
%
\isadelimproof
\isanewline
%
\endisadelimproof
\isanewline
\isacommand{lemma}\isamarkupfalse%
\ {\isacharparenleft}{\kern0pt}\isakeyword{in}\ nat{\isacharunderscore}{\kern0pt}sigma{\isacharunderscore}{\kern0pt}finite{\isacharunderscore}{\kern0pt}adapted{\isacharunderscore}{\kern0pt}process{\isacharparenright}{\kern0pt}\ martingale{\isacharunderscore}{\kern0pt}nat{\isacharcolon}{\kern0pt}\isanewline
\ \ \isakeyword{assumes}\ integrable{\isacharcolon}{\kern0pt}\ {\isachardoublequoteopen}{\isasymAnd}i{\isachardot}{\kern0pt}\ integrable\ M\ {\isacharparenleft}{\kern0pt}X\ i{\isacharparenright}{\kern0pt}{\isachardoublequoteclose}\ \isanewline
\ \ \ \ \ \ \isakeyword{and}\ {\isachardoublequoteopen}{\isasymAnd}i{\isachardot}{\kern0pt}\ AE\ {\isasymxi}\ in\ M{\isachardot}{\kern0pt}\ X\ i\ {\isasymxi}\ {\isacharequal}{\kern0pt}\ cond{\isacharunderscore}{\kern0pt}exp\ M\ {\isacharparenleft}{\kern0pt}F\ i{\isacharparenright}{\kern0pt}\ {\isacharparenleft}{\kern0pt}X\ {\isacharparenleft}{\kern0pt}Suc\ i{\isacharparenright}{\kern0pt}{\isacharparenright}{\kern0pt}\ {\isasymxi}{\isachardoublequoteclose}\ \isanewline
\ \ \ \ \isakeyword{shows}\ {\isachardoublequoteopen}nat{\isacharunderscore}{\kern0pt}martingale\ M\ F\ X{\isachardoublequoteclose}\isanewline
%
\isadelimproof
%
\endisadelimproof
%
\isatagproof
\isacommand{proof}\isamarkupfalse%
\ {\isacharparenleft}{\kern0pt}unfold{\isacharunderscore}{\kern0pt}locales{\isacharparenright}{\kern0pt}\isanewline
\ \ \isacommand{fix}\isamarkupfalse%
\ i\ j\ {\isacharcolon}{\kern0pt}{\isacharcolon}{\kern0pt}\ nat\ \isacommand{assume}\isamarkupfalse%
\ asm{\isacharcolon}{\kern0pt}\ {\isachardoublequoteopen}i\ {\isasymle}\ j{\isachardoublequoteclose}\isanewline
\ \ \isacommand{show}\isamarkupfalse%
\ {\isachardoublequoteopen}AE\ {\isasymxi}\ in\ M{\isachardot}{\kern0pt}\ X\ i\ {\isasymxi}\ {\isacharequal}{\kern0pt}\ cond{\isacharunderscore}{\kern0pt}exp\ M\ {\isacharparenleft}{\kern0pt}F\ i{\isacharparenright}{\kern0pt}\ {\isacharparenleft}{\kern0pt}X\ j{\isacharparenright}{\kern0pt}\ {\isasymxi}{\isachardoublequoteclose}\ \isacommand{using}\isamarkupfalse%
\ asm\isanewline
\ \ \isacommand{proof}\isamarkupfalse%
\ {\isacharparenleft}{\kern0pt}induction\ {\isachardoublequoteopen}j\ {\isacharminus}{\kern0pt}\ i{\isachardoublequoteclose}\ arbitrary{\isacharcolon}{\kern0pt}\ i\ j{\isacharparenright}{\kern0pt}\isanewline
\ \ \ \ \isacommand{case}\isamarkupfalse%
\ {\isadigit{0}}\isanewline
\ \ \ \ \isacommand{hence}\isamarkupfalse%
\ {\isachardoublequoteopen}j\ {\isacharequal}{\kern0pt}\ i{\isachardoublequoteclose}\ \isacommand{by}\isamarkupfalse%
\ simp\isanewline
\ \ \ \ \isacommand{thus}\isamarkupfalse%
\ {\isacharquery}{\kern0pt}case\ \isacommand{using}\isamarkupfalse%
\ sigma{\isacharunderscore}{\kern0pt}finite{\isacharunderscore}{\kern0pt}subalgebra{\isachardot}{\kern0pt}cond{\isacharunderscore}{\kern0pt}exp{\isacharunderscore}{\kern0pt}F{\isacharunderscore}{\kern0pt}meas{\isacharbrackleft}{\kern0pt}OF\ {\isacharunderscore}{\kern0pt}\ integrable\ adapted{\isacharcomma}{\kern0pt}\ THEN\ AE{\isacharunderscore}{\kern0pt}symmetric{\isacharbrackright}{\kern0pt}\ \isacommand{by}\isamarkupfalse%
\ blast\isanewline
\ \ \isacommand{next}\isamarkupfalse%
\isanewline
\ \ \ \ \isacommand{case}\isamarkupfalse%
\ {\isacharparenleft}{\kern0pt}Suc\ n{\isacharparenright}{\kern0pt}\isanewline
\ \ \ \ \isacommand{have}\isamarkupfalse%
\ j{\isacharcolon}{\kern0pt}\ {\isachardoublequoteopen}j\ {\isacharequal}{\kern0pt}\ Suc\ {\isacharparenleft}{\kern0pt}n\ {\isacharplus}{\kern0pt}\ i{\isacharparenright}{\kern0pt}{\isachardoublequoteclose}\ \isacommand{using}\isamarkupfalse%
\ Suc\ \isacommand{by}\isamarkupfalse%
\ linarith\isanewline
\ \ \ \ \isacommand{have}\isamarkupfalse%
\ n{\isacharcolon}{\kern0pt}\ {\isachardoublequoteopen}n\ {\isacharequal}{\kern0pt}\ n\ {\isacharplus}{\kern0pt}\ i\ {\isacharminus}{\kern0pt}\ i{\isachardoublequoteclose}\ \isacommand{using}\isamarkupfalse%
\ Suc\ \isacommand{by}\isamarkupfalse%
\ linarith\isanewline
\ \ \ \ \isacommand{have}\isamarkupfalse%
\ {\isacharasterisk}{\kern0pt}{\isacharcolon}{\kern0pt}\ {\isachardoublequoteopen}AE\ {\isasymxi}\ in\ M{\isachardot}{\kern0pt}\ cond{\isacharunderscore}{\kern0pt}exp\ M\ {\isacharparenleft}{\kern0pt}F\ {\isacharparenleft}{\kern0pt}n\ {\isacharplus}{\kern0pt}\ i{\isacharparenright}{\kern0pt}{\isacharparenright}{\kern0pt}\ {\isacharparenleft}{\kern0pt}X\ j{\isacharparenright}{\kern0pt}\ {\isasymxi}\ {\isacharequal}{\kern0pt}\ X\ {\isacharparenleft}{\kern0pt}n\ {\isacharplus}{\kern0pt}\ i{\isacharparenright}{\kern0pt}\ {\isasymxi}{\isachardoublequoteclose}\ \isacommand{unfolding}\isamarkupfalse%
\ j\ \isacommand{using}\isamarkupfalse%
\ assms{\isacharparenleft}{\kern0pt}{\isadigit{2}}{\isacharparenright}{\kern0pt}{\isacharbrackleft}{\kern0pt}THEN\ AE{\isacharunderscore}{\kern0pt}symmetric{\isacharbrackright}{\kern0pt}\ \isacommand{by}\isamarkupfalse%
\ blast\isanewline
\ \ \ \ \isacommand{have}\isamarkupfalse%
\ {\isachardoublequoteopen}AE\ {\isasymxi}\ in\ M{\isachardot}{\kern0pt}\ cond{\isacharunderscore}{\kern0pt}exp\ M\ {\isacharparenleft}{\kern0pt}F\ i{\isacharparenright}{\kern0pt}\ {\isacharparenleft}{\kern0pt}X\ j{\isacharparenright}{\kern0pt}\ {\isasymxi}\ {\isacharequal}{\kern0pt}\ cond{\isacharunderscore}{\kern0pt}exp\ M\ {\isacharparenleft}{\kern0pt}F\ i{\isacharparenright}{\kern0pt}\ {\isacharparenleft}{\kern0pt}cond{\isacharunderscore}{\kern0pt}exp\ M\ {\isacharparenleft}{\kern0pt}F\ {\isacharparenleft}{\kern0pt}n\ {\isacharplus}{\kern0pt}\ i{\isacharparenright}{\kern0pt}{\isacharparenright}{\kern0pt}\ {\isacharparenleft}{\kern0pt}X\ j{\isacharparenright}{\kern0pt}{\isacharparenright}{\kern0pt}\ {\isasymxi}{\isachardoublequoteclose}\ \isacommand{by}\isamarkupfalse%
\ {\isacharparenleft}{\kern0pt}intro\ cond{\isacharunderscore}{\kern0pt}exp{\isacharunderscore}{\kern0pt}nested{\isacharunderscore}{\kern0pt}subalg\ integrable\ subalg{\isacharcomma}{\kern0pt}\ simp\ add{\isacharcolon}{\kern0pt}\ subalgebra{\isacharunderscore}{\kern0pt}def\ sets{\isacharunderscore}{\kern0pt}F{\isacharunderscore}{\kern0pt}mono{\isacharparenright}{\kern0pt}\isanewline
\ \ \ \ \isacommand{hence}\isamarkupfalse%
\ {\isachardoublequoteopen}AE\ {\isasymxi}\ in\ M{\isachardot}{\kern0pt}\ cond{\isacharunderscore}{\kern0pt}exp\ M\ {\isacharparenleft}{\kern0pt}F\ i{\isacharparenright}{\kern0pt}\ {\isacharparenleft}{\kern0pt}X\ j{\isacharparenright}{\kern0pt}\ {\isasymxi}\ {\isacharequal}{\kern0pt}\ cond{\isacharunderscore}{\kern0pt}exp\ M\ {\isacharparenleft}{\kern0pt}F\ i{\isacharparenright}{\kern0pt}\ {\isacharparenleft}{\kern0pt}X\ {\isacharparenleft}{\kern0pt}n\ {\isacharplus}{\kern0pt}\ i{\isacharparenright}{\kern0pt}{\isacharparenright}{\kern0pt}\ {\isasymxi}{\isachardoublequoteclose}\ \isacommand{using}\isamarkupfalse%
\ cond{\isacharunderscore}{\kern0pt}exp{\isacharunderscore}{\kern0pt}cong{\isacharunderscore}{\kern0pt}AE{\isacharbrackleft}{\kern0pt}OF\ integrable{\isacharunderscore}{\kern0pt}cond{\isacharunderscore}{\kern0pt}exp\ integrable\ {\isacharasterisk}{\kern0pt}{\isacharbrackright}{\kern0pt}\ \isacommand{by}\isamarkupfalse%
\ force\isanewline
\ \ \ \ \isacommand{thus}\isamarkupfalse%
\ {\isacharquery}{\kern0pt}case\ \isacommand{using}\isamarkupfalse%
\ Suc{\isacharparenleft}{\kern0pt}{\isadigit{1}}{\isacharparenright}{\kern0pt}{\isacharbrackleft}{\kern0pt}OF\ n{\isacharbrackright}{\kern0pt}\ \isacommand{by}\isamarkupfalse%
\ fastforce\isanewline
\ \ \isacommand{qed}\isamarkupfalse%
\isanewline
\isacommand{qed}\isamarkupfalse%
\ {\isacharparenleft}{\kern0pt}simp\ add{\isacharcolon}{\kern0pt}\ integrable{\isacharparenright}{\kern0pt}%
\endisatagproof
{\isafoldproof}%
%
\isadelimproof
\isanewline
%
\endisadelimproof
\isanewline
\isacommand{lemma}\isamarkupfalse%
\ {\isacharparenleft}{\kern0pt}\isakeyword{in}\ nat{\isacharunderscore}{\kern0pt}sigma{\isacharunderscore}{\kern0pt}finite{\isacharunderscore}{\kern0pt}adapted{\isacharunderscore}{\kern0pt}process{\isacharparenright}{\kern0pt}\ martingale{\isacharunderscore}{\kern0pt}of{\isacharunderscore}{\kern0pt}cond{\isacharunderscore}{\kern0pt}exp{\isacharunderscore}{\kern0pt}diff{\isacharunderscore}{\kern0pt}Suc{\isacharunderscore}{\kern0pt}eq{\isacharunderscore}{\kern0pt}zero{\isacharcolon}{\kern0pt}\isanewline
\ \ \isakeyword{assumes}\ integrable{\isacharcolon}{\kern0pt}\ {\isachardoublequoteopen}{\isasymAnd}i{\isachardot}{\kern0pt}\ integrable\ M\ {\isacharparenleft}{\kern0pt}X\ i{\isacharparenright}{\kern0pt}{\isachardoublequoteclose}\ \isanewline
\ \ \ \ \ \ \isakeyword{and}\ {\isachardoublequoteopen}{\isasymAnd}i{\isachardot}{\kern0pt}\ AE\ {\isasymxi}\ in\ M{\isachardot}{\kern0pt}\ cond{\isacharunderscore}{\kern0pt}exp\ M\ {\isacharparenleft}{\kern0pt}F\ i{\isacharparenright}{\kern0pt}\ {\isacharparenleft}{\kern0pt}{\isasymlambda}{\isasymxi}{\isachardot}{\kern0pt}\ X\ {\isacharparenleft}{\kern0pt}Suc\ i{\isacharparenright}{\kern0pt}\ {\isasymxi}\ {\isacharminus}{\kern0pt}\ X\ i\ {\isasymxi}{\isacharparenright}{\kern0pt}\ {\isasymxi}\ {\isacharequal}{\kern0pt}\ {\isadigit{0}}{\isachardoublequoteclose}\ \isanewline
\ \ \ \ \isakeyword{shows}\ {\isachardoublequoteopen}nat{\isacharunderscore}{\kern0pt}martingale\ M\ F\ X{\isachardoublequoteclose}\isanewline
%
\isadelimproof
%
\endisadelimproof
%
\isatagproof
\isacommand{proof}\isamarkupfalse%
\ {\isacharparenleft}{\kern0pt}intro\ martingale{\isacharunderscore}{\kern0pt}nat\ integrable{\isacharparenright}{\kern0pt}\ \isanewline
\ \ \isacommand{fix}\isamarkupfalse%
\ i\ \isanewline
\ \ \isacommand{show}\isamarkupfalse%
\ {\isachardoublequoteopen}AE\ {\isasymxi}\ in\ M{\isachardot}{\kern0pt}\ X\ i\ {\isasymxi}\ {\isacharequal}{\kern0pt}\ cond{\isacharunderscore}{\kern0pt}exp\ M\ {\isacharparenleft}{\kern0pt}F\ i{\isacharparenright}{\kern0pt}\ {\isacharparenleft}{\kern0pt}X\ {\isacharparenleft}{\kern0pt}Suc\ i{\isacharparenright}{\kern0pt}{\isacharparenright}{\kern0pt}\ {\isasymxi}{\isachardoublequoteclose}\ \isacommand{using}\isamarkupfalse%
\ cond{\isacharunderscore}{\kern0pt}exp{\isacharunderscore}{\kern0pt}diff{\isacharbrackleft}{\kern0pt}OF\ integrable{\isacharparenleft}{\kern0pt}{\isadigit{1}}{\isacharcomma}{\kern0pt}{\isadigit{1}}{\isacharparenright}{\kern0pt}{\isacharcomma}{\kern0pt}\ of\ i\ {\isachardoublequoteopen}Suc\ i{\isachardoublequoteclose}\ i{\isacharbrackright}{\kern0pt}\ cond{\isacharunderscore}{\kern0pt}exp{\isacharunderscore}{\kern0pt}F{\isacharunderscore}{\kern0pt}meas{\isacharbrackleft}{\kern0pt}OF\ integrable\ adapted{\isacharcomma}{\kern0pt}\ of\ i{\isacharbrackright}{\kern0pt}\ assms{\isacharparenleft}{\kern0pt}{\isadigit{2}}{\isacharparenright}{\kern0pt}{\isacharbrackleft}{\kern0pt}of\ i{\isacharbrackright}{\kern0pt}\ \isacommand{by}\isamarkupfalse%
\ fastforce\isanewline
\isacommand{qed}\isamarkupfalse%
%
\endisatagproof
{\isafoldproof}%
%
\isadelimproof
%
\endisadelimproof
%
\isadelimdocument
%
\endisadelimdocument
%
\isatagdocument
%
\isamarkupsubsection{Discrete Time Submartingales%
}
\isamarkuptrue%
%
\endisatagdocument
{\isafolddocument}%
%
\isadelimdocument
%
\endisadelimdocument
\isacommand{lemma}\isamarkupfalse%
\ {\isacharparenleft}{\kern0pt}\isakeyword{in}\ nat{\isacharunderscore}{\kern0pt}submartingale{\isacharparenright}{\kern0pt}\ predictable{\isacharunderscore}{\kern0pt}mono{\isacharcolon}{\kern0pt}\isanewline
\ \ \isakeyword{assumes}\ {\isachardoublequoteopen}nat{\isacharunderscore}{\kern0pt}predictable{\isacharunderscore}{\kern0pt}process\ M\ F\ X{\isachardoublequoteclose}\ {\isachardoublequoteopen}i\ {\isasymle}\ j{\isachardoublequoteclose}\isanewline
\ \ \isakeyword{shows}\ {\isachardoublequoteopen}AE\ {\isasymxi}\ in\ M{\isachardot}{\kern0pt}\ X\ i\ {\isasymxi}\ {\isasymle}\ X\ j\ {\isasymxi}{\isachardoublequoteclose}\isanewline
%
\isadelimproof
\ \ %
\endisadelimproof
%
\isatagproof
\isacommand{using}\isamarkupfalse%
\ assms{\isacharparenleft}{\kern0pt}{\isadigit{2}}{\isacharparenright}{\kern0pt}\isanewline
\isacommand{proof}\isamarkupfalse%
\ {\isacharparenleft}{\kern0pt}induction\ {\isachardoublequoteopen}j\ {\isacharminus}{\kern0pt}\ i{\isachardoublequoteclose}\ arbitrary{\isacharcolon}{\kern0pt}\ i\ j{\isacharparenright}{\kern0pt}\isanewline
\ \ \isacommand{case}\isamarkupfalse%
\ {\isadigit{0}}\isanewline
\ \ \isacommand{then}\isamarkupfalse%
\ \isacommand{show}\isamarkupfalse%
\ {\isacharquery}{\kern0pt}case\ \isacommand{by}\isamarkupfalse%
\ simp\ \isanewline
\isacommand{next}\isamarkupfalse%
\isanewline
\ \ \isacommand{case}\isamarkupfalse%
\ {\isacharparenleft}{\kern0pt}Suc\ n{\isacharparenright}{\kern0pt}\isanewline
\ \ \isacommand{hence}\isamarkupfalse%
\ {\isacharasterisk}{\kern0pt}{\isacharcolon}{\kern0pt}\ {\isachardoublequoteopen}n\ {\isacharequal}{\kern0pt}\ j\ {\isacharminus}{\kern0pt}\ Suc\ i{\isachardoublequoteclose}\ \isacommand{by}\isamarkupfalse%
\ linarith\isanewline
\ \ \isacommand{interpret}\isamarkupfalse%
\ S{\isacharcolon}{\kern0pt}\ nat{\isacharunderscore}{\kern0pt}adapted{\isacharunderscore}{\kern0pt}process\ M\ F\ {\isachardoublequoteopen}{\isasymlambda}i{\isachardot}{\kern0pt}\ X\ {\isacharparenleft}{\kern0pt}Suc\ i{\isacharparenright}{\kern0pt}{\isachardoublequoteclose}\ \isacommand{by}\isamarkupfalse%
\ {\isacharparenleft}{\kern0pt}intro\ nat{\isacharunderscore}{\kern0pt}predictable{\isacharunderscore}{\kern0pt}process{\isachardot}{\kern0pt}adapted{\isacharunderscore}{\kern0pt}Suc\ assms{\isacharparenright}{\kern0pt}\isanewline
\ \ \isacommand{have}\isamarkupfalse%
\ {\isachardoublequoteopen}Suc\ i\ {\isasymle}\ j{\isachardoublequoteclose}\ \isacommand{using}\isamarkupfalse%
\ Suc{\isacharparenleft}{\kern0pt}{\isadigit{2}}{\isacharcomma}{\kern0pt}{\isadigit{3}}{\isacharparenright}{\kern0pt}\ \isacommand{by}\isamarkupfalse%
\ linarith\isanewline
\ \ \isacommand{thus}\isamarkupfalse%
\ {\isacharquery}{\kern0pt}case\ \isacommand{using}\isamarkupfalse%
\ Suc{\isacharparenleft}{\kern0pt}{\isadigit{1}}{\isacharparenright}{\kern0pt}{\isacharbrackleft}{\kern0pt}OF\ {\isacharasterisk}{\kern0pt}{\isacharbrackright}{\kern0pt}\ S{\isachardot}{\kern0pt}adapted{\isacharbrackleft}{\kern0pt}of\ i{\isacharbrackright}{\kern0pt}\ submartingale{\isacharunderscore}{\kern0pt}property{\isacharbrackleft}{\kern0pt}OF\ {\isacharunderscore}{\kern0pt}\ le{\isacharunderscore}{\kern0pt}SucI{\isacharcomma}{\kern0pt}\ of\ i{\isacharbrackright}{\kern0pt}\ sigma{\isacharunderscore}{\kern0pt}finite{\isacharunderscore}{\kern0pt}subalgebra{\isachardot}{\kern0pt}cond{\isacharunderscore}{\kern0pt}exp{\isacharunderscore}{\kern0pt}F{\isacharunderscore}{\kern0pt}meas{\isacharbrackleft}{\kern0pt}OF\ {\isacharunderscore}{\kern0pt}\ integrable{\isacharcomma}{\kern0pt}\ of\ {\isachardoublequoteopen}F\ i{\isachardoublequoteclose}\ {\isachardoublequoteopen}Suc\ i{\isachardoublequoteclose}{\isacharbrackright}{\kern0pt}\ \isacommand{by}\isamarkupfalse%
\ fastforce\isanewline
\isacommand{qed}\isamarkupfalse%
%
\endisatagproof
{\isafoldproof}%
%
\isadelimproof
\isanewline
%
\endisadelimproof
\isanewline
\isacommand{lemma}\isamarkupfalse%
\ {\isacharparenleft}{\kern0pt}\isakeyword{in}\ nat{\isacharunderscore}{\kern0pt}sigma{\isacharunderscore}{\kern0pt}finite{\isacharunderscore}{\kern0pt}adapted{\isacharunderscore}{\kern0pt}process{\isacharunderscore}{\kern0pt}linorder{\isacharparenright}{\kern0pt}\ submartingale{\isacharunderscore}{\kern0pt}of{\isacharunderscore}{\kern0pt}set{\isacharunderscore}{\kern0pt}integral{\isacharunderscore}{\kern0pt}le{\isacharunderscore}{\kern0pt}Suc{\isacharcolon}{\kern0pt}\isanewline
\ \ \isakeyword{assumes}\ integrable{\isacharcolon}{\kern0pt}\ {\isachardoublequoteopen}{\isasymAnd}i{\isachardot}{\kern0pt}\ integrable\ M\ {\isacharparenleft}{\kern0pt}X\ i{\isacharparenright}{\kern0pt}{\isachardoublequoteclose}\ \isanewline
\ \ \ \ \ \ \isakeyword{and}\ {\isachardoublequoteopen}{\isasymAnd}A\ i{\isachardot}{\kern0pt}\ A\ {\isasymin}\ F\ i\ {\isasymLongrightarrow}\ set{\isacharunderscore}{\kern0pt}lebesgue{\isacharunderscore}{\kern0pt}integral\ M\ A\ {\isacharparenleft}{\kern0pt}X\ i{\isacharparenright}{\kern0pt}\ {\isasymle}\ set{\isacharunderscore}{\kern0pt}lebesgue{\isacharunderscore}{\kern0pt}integral\ M\ A\ {\isacharparenleft}{\kern0pt}X\ {\isacharparenleft}{\kern0pt}Suc\ i{\isacharparenright}{\kern0pt}{\isacharparenright}{\kern0pt}{\isachardoublequoteclose}\ \isanewline
\ \ \ \ \isakeyword{shows}\ {\isachardoublequoteopen}nat{\isacharunderscore}{\kern0pt}submartingale\ M\ F\ X{\isachardoublequoteclose}\isanewline
%
\isadelimproof
%
\endisadelimproof
%
\isatagproof
\isacommand{proof}\isamarkupfalse%
\ {\isacharparenleft}{\kern0pt}intro\ nat{\isacharunderscore}{\kern0pt}submartingale{\isachardot}{\kern0pt}intro\ submartingale{\isacharunderscore}{\kern0pt}of{\isacharunderscore}{\kern0pt}set{\isacharunderscore}{\kern0pt}integral{\isacharunderscore}{\kern0pt}le{\isacharparenright}{\kern0pt}\isanewline
\ \ \isacommand{fix}\isamarkupfalse%
\ i\ j\ A\ \isacommand{assume}\isamarkupfalse%
\ asm{\isacharcolon}{\kern0pt}\ {\isachardoublequoteopen}i\ {\isasymle}\ j{\isachardoublequoteclose}\ {\isachardoublequoteopen}A\ {\isasymin}\ sets\ {\isacharparenleft}{\kern0pt}F\ i{\isacharparenright}{\kern0pt}{\isachardoublequoteclose}\isanewline
\ \ \isacommand{show}\isamarkupfalse%
\ {\isachardoublequoteopen}set{\isacharunderscore}{\kern0pt}lebesgue{\isacharunderscore}{\kern0pt}integral\ M\ A\ {\isacharparenleft}{\kern0pt}X\ i{\isacharparenright}{\kern0pt}\ {\isasymle}\ set{\isacharunderscore}{\kern0pt}lebesgue{\isacharunderscore}{\kern0pt}integral\ M\ A\ {\isacharparenleft}{\kern0pt}X\ j{\isacharparenright}{\kern0pt}{\isachardoublequoteclose}\ \isacommand{using}\isamarkupfalse%
\ asm\isanewline
\ \ \isacommand{proof}\isamarkupfalse%
\ {\isacharparenleft}{\kern0pt}induction\ {\isachardoublequoteopen}j\ {\isacharminus}{\kern0pt}\ i{\isachardoublequoteclose}\ arbitrary{\isacharcolon}{\kern0pt}\ i\ j{\isacharparenright}{\kern0pt}\isanewline
\ \ \ \ \isacommand{case}\isamarkupfalse%
\ {\isadigit{0}}\isanewline
\ \ \ \ \isacommand{then}\isamarkupfalse%
\ \isacommand{show}\isamarkupfalse%
\ {\isacharquery}{\kern0pt}case\ \isacommand{using}\isamarkupfalse%
\ asm\ \isacommand{by}\isamarkupfalse%
\ simp\isanewline
\ \ \isacommand{next}\isamarkupfalse%
\isanewline
\ \ \ \ \isacommand{case}\isamarkupfalse%
\ {\isacharparenleft}{\kern0pt}Suc\ n{\isacharparenright}{\kern0pt}\isanewline
\ \ \ \ \isacommand{hence}\isamarkupfalse%
\ {\isacharasterisk}{\kern0pt}{\isacharcolon}{\kern0pt}\ {\isachardoublequoteopen}n\ {\isacharequal}{\kern0pt}\ j\ {\isacharminus}{\kern0pt}\ Suc\ i{\isachardoublequoteclose}\ \isacommand{by}\isamarkupfalse%
\ linarith\isanewline
\ \ \ \ \isacommand{have}\isamarkupfalse%
\ {\isachardoublequoteopen}Suc\ i\ {\isasymle}\ j{\isachardoublequoteclose}\ \isacommand{using}\isamarkupfalse%
\ Suc{\isacharparenleft}{\kern0pt}{\isadigit{2}}{\isacharcomma}{\kern0pt}{\isadigit{3}}{\isacharparenright}{\kern0pt}\ \isacommand{by}\isamarkupfalse%
\ linarith\isanewline
\ \ \ \ \isacommand{thus}\isamarkupfalse%
\ {\isacharquery}{\kern0pt}case\ \isacommand{using}\isamarkupfalse%
\ sets{\isacharunderscore}{\kern0pt}F{\isacharunderscore}{\kern0pt}mono{\isacharbrackleft}{\kern0pt}OF\ {\isacharunderscore}{\kern0pt}\ le{\isacharunderscore}{\kern0pt}SucI{\isacharbrackright}{\kern0pt}\ Suc{\isacharparenleft}{\kern0pt}{\isadigit{4}}{\isacharparenright}{\kern0pt}\ Suc{\isacharparenleft}{\kern0pt}{\isadigit{1}}{\isacharparenright}{\kern0pt}{\isacharbrackleft}{\kern0pt}OF\ {\isacharasterisk}{\kern0pt}{\isacharbrackright}{\kern0pt}\ \isacommand{by}\isamarkupfalse%
\ {\isacharparenleft}{\kern0pt}auto\ intro{\isacharcolon}{\kern0pt}\ assms{\isacharparenleft}{\kern0pt}{\isadigit{2}}{\isacharparenright}{\kern0pt}{\isacharbrackleft}{\kern0pt}THEN\ order{\isacharunderscore}{\kern0pt}trans{\isacharbrackright}{\kern0pt}{\isacharparenright}{\kern0pt}\isanewline
\ \ \isacommand{qed}\isamarkupfalse%
\isanewline
\isacommand{qed}\isamarkupfalse%
\ {\isacharparenleft}{\kern0pt}simp\ add{\isacharcolon}{\kern0pt}\ integrable{\isacharparenright}{\kern0pt}%
\endisatagproof
{\isafoldproof}%
%
\isadelimproof
\isanewline
%
\endisadelimproof
\isanewline
\isacommand{lemma}\isamarkupfalse%
\ {\isacharparenleft}{\kern0pt}\isakeyword{in}\ nat{\isacharunderscore}{\kern0pt}sigma{\isacharunderscore}{\kern0pt}finite{\isacharunderscore}{\kern0pt}adapted{\isacharunderscore}{\kern0pt}process{\isacharunderscore}{\kern0pt}linorder{\isacharparenright}{\kern0pt}\ submartingale{\isacharunderscore}{\kern0pt}nat{\isacharcolon}{\kern0pt}\isanewline
\ \ \isakeyword{assumes}\ integrable{\isacharcolon}{\kern0pt}\ {\isachardoublequoteopen}{\isasymAnd}i{\isachardot}{\kern0pt}\ integrable\ M\ {\isacharparenleft}{\kern0pt}X\ i{\isacharparenright}{\kern0pt}{\isachardoublequoteclose}\ \isanewline
\ \ \ \ \ \ \isakeyword{and}\ {\isachardoublequoteopen}{\isasymAnd}i{\isachardot}{\kern0pt}\ AE\ {\isasymxi}\ in\ M{\isachardot}{\kern0pt}\ X\ i\ {\isasymxi}\ {\isasymle}\ cond{\isacharunderscore}{\kern0pt}exp\ M\ {\isacharparenleft}{\kern0pt}F\ i{\isacharparenright}{\kern0pt}\ {\isacharparenleft}{\kern0pt}X\ {\isacharparenleft}{\kern0pt}Suc\ i{\isacharparenright}{\kern0pt}{\isacharparenright}{\kern0pt}\ {\isasymxi}{\isachardoublequoteclose}\ \isanewline
\ \ \ \ \isakeyword{shows}\ {\isachardoublequoteopen}nat{\isacharunderscore}{\kern0pt}submartingale\ M\ F\ X{\isachardoublequoteclose}\isanewline
%
\isadelimproof
\ \ %
\endisadelimproof
%
\isatagproof
\isacommand{using}\isamarkupfalse%
\ subalg\ integrable\ assms{\isacharparenleft}{\kern0pt}{\isadigit{2}}{\isacharparenright}{\kern0pt}\isanewline
\ \ \isacommand{by}\isamarkupfalse%
\ {\isacharparenleft}{\kern0pt}intro\ submartingale{\isacharunderscore}{\kern0pt}of{\isacharunderscore}{\kern0pt}set{\isacharunderscore}{\kern0pt}integral{\isacharunderscore}{\kern0pt}le{\isacharunderscore}{\kern0pt}Suc\ ord{\isacharunderscore}{\kern0pt}le{\isacharunderscore}{\kern0pt}eq{\isacharunderscore}{\kern0pt}trans{\isacharbrackleft}{\kern0pt}OF\ set{\isacharunderscore}{\kern0pt}integral{\isacharunderscore}{\kern0pt}mono{\isacharunderscore}{\kern0pt}AE{\isacharunderscore}{\kern0pt}banach\ cond{\isacharunderscore}{\kern0pt}exp{\isacharunderscore}{\kern0pt}set{\isacharunderscore}{\kern0pt}integral{\isacharbrackleft}{\kern0pt}symmetric{\isacharbrackright}{\kern0pt}{\isacharbrackright}{\kern0pt}{\isacharcomma}{\kern0pt}\ simp{\isacharparenright}{\kern0pt}\isanewline
\ \ \ \ \ {\isacharparenleft}{\kern0pt}meson\ in{\isacharunderscore}{\kern0pt}mono\ integrable{\isacharunderscore}{\kern0pt}mult{\isacharunderscore}{\kern0pt}indicator\ set{\isacharunderscore}{\kern0pt}integrable{\isacharunderscore}{\kern0pt}def\ subalgebra{\isacharunderscore}{\kern0pt}def{\isacharcomma}{\kern0pt}\ meson\ integrable{\isacharunderscore}{\kern0pt}cond{\isacharunderscore}{\kern0pt}exp\ in{\isacharunderscore}{\kern0pt}mono\ integrable{\isacharunderscore}{\kern0pt}mult{\isacharunderscore}{\kern0pt}indicator\ set{\isacharunderscore}{\kern0pt}integrable{\isacharunderscore}{\kern0pt}def\ subalgebra{\isacharunderscore}{\kern0pt}def{\isacharcomma}{\kern0pt}\ fast{\isacharplus}{\kern0pt}{\isacharparenright}{\kern0pt}%
\endisatagproof
{\isafoldproof}%
%
\isadelimproof
\isanewline
%
\endisadelimproof
\isanewline
\isacommand{lemma}\isamarkupfalse%
\ {\isacharparenleft}{\kern0pt}\isakeyword{in}\ nat{\isacharunderscore}{\kern0pt}sigma{\isacharunderscore}{\kern0pt}finite{\isacharunderscore}{\kern0pt}adapted{\isacharunderscore}{\kern0pt}process{\isacharunderscore}{\kern0pt}linorder{\isacharparenright}{\kern0pt}\ submartingale{\isacharunderscore}{\kern0pt}of{\isacharunderscore}{\kern0pt}cond{\isacharunderscore}{\kern0pt}exp{\isacharunderscore}{\kern0pt}diff{\isacharunderscore}{\kern0pt}Suc{\isacharunderscore}{\kern0pt}nonneg{\isacharcolon}{\kern0pt}\isanewline
\ \ \isakeyword{assumes}\ integrable{\isacharcolon}{\kern0pt}\ {\isachardoublequoteopen}{\isasymAnd}i{\isachardot}{\kern0pt}\ integrable\ M\ {\isacharparenleft}{\kern0pt}X\ i{\isacharparenright}{\kern0pt}{\isachardoublequoteclose}\ \isanewline
\ \ \ \ \ \ \isakeyword{and}\ {\isachardoublequoteopen}{\isasymAnd}i{\isachardot}{\kern0pt}\ AE\ {\isasymxi}\ in\ M{\isachardot}{\kern0pt}\ cond{\isacharunderscore}{\kern0pt}exp\ M\ {\isacharparenleft}{\kern0pt}F\ i{\isacharparenright}{\kern0pt}\ {\isacharparenleft}{\kern0pt}{\isasymlambda}{\isasymxi}{\isachardot}{\kern0pt}\ X\ {\isacharparenleft}{\kern0pt}Suc\ i{\isacharparenright}{\kern0pt}\ {\isasymxi}\ {\isacharminus}{\kern0pt}\ X\ i\ {\isasymxi}{\isacharparenright}{\kern0pt}\ {\isasymxi}\ {\isasymge}\ {\isadigit{0}}{\isachardoublequoteclose}\ \isanewline
\ \ \ \ \isakeyword{shows}\ {\isachardoublequoteopen}nat{\isacharunderscore}{\kern0pt}submartingale\ M\ F\ X{\isachardoublequoteclose}\isanewline
%
\isadelimproof
%
\endisadelimproof
%
\isatagproof
\isacommand{proof}\isamarkupfalse%
\ {\isacharparenleft}{\kern0pt}intro\ submartingale{\isacharunderscore}{\kern0pt}nat\ integrable{\isacharparenright}{\kern0pt}\ \isanewline
\ \ \isacommand{fix}\isamarkupfalse%
\ i\ \isanewline
\ \ \isacommand{show}\isamarkupfalse%
\ {\isachardoublequoteopen}AE\ {\isasymxi}\ in\ M{\isachardot}{\kern0pt}\ X\ i\ {\isasymxi}\ {\isasymle}\ cond{\isacharunderscore}{\kern0pt}exp\ M\ {\isacharparenleft}{\kern0pt}F\ i{\isacharparenright}{\kern0pt}\ {\isacharparenleft}{\kern0pt}X\ {\isacharparenleft}{\kern0pt}Suc\ i{\isacharparenright}{\kern0pt}{\isacharparenright}{\kern0pt}\ {\isasymxi}{\isachardoublequoteclose}\ \isacommand{using}\isamarkupfalse%
\ cond{\isacharunderscore}{\kern0pt}exp{\isacharunderscore}{\kern0pt}diff{\isacharbrackleft}{\kern0pt}OF\ integrable{\isacharparenleft}{\kern0pt}{\isadigit{1}}{\isacharcomma}{\kern0pt}{\isadigit{1}}{\isacharparenright}{\kern0pt}{\isacharcomma}{\kern0pt}\ of\ i\ {\isachardoublequoteopen}Suc\ i{\isachardoublequoteclose}\ i{\isacharbrackright}{\kern0pt}\ cond{\isacharunderscore}{\kern0pt}exp{\isacharunderscore}{\kern0pt}F{\isacharunderscore}{\kern0pt}meas{\isacharbrackleft}{\kern0pt}OF\ integrable\ adapted{\isacharcomma}{\kern0pt}\ of\ i{\isacharbrackright}{\kern0pt}\ assms{\isacharparenleft}{\kern0pt}{\isadigit{2}}{\isacharparenright}{\kern0pt}{\isacharbrackleft}{\kern0pt}of\ i{\isacharbrackright}{\kern0pt}\ \isacommand{by}\isamarkupfalse%
\ fastforce\isanewline
\isacommand{qed}\isamarkupfalse%
%
\endisatagproof
{\isafoldproof}%
%
\isadelimproof
\isanewline
%
\endisadelimproof
\isanewline
\isacommand{lemma}\isamarkupfalse%
\ {\isacharparenleft}{\kern0pt}\isakeyword{in}\ nat{\isacharunderscore}{\kern0pt}submartingale{\isacharunderscore}{\kern0pt}linorder{\isacharparenright}{\kern0pt}\ partial{\isacharunderscore}{\kern0pt}sum{\isacharunderscore}{\kern0pt}scaleR{\isacharcolon}{\kern0pt}\isanewline
\ \ \isakeyword{assumes}\ {\isachardoublequoteopen}nat{\isacharunderscore}{\kern0pt}adapted{\isacharunderscore}{\kern0pt}process\ M\ F\ C{\isachardoublequoteclose}\ {\isachardoublequoteopen}{\isasymAnd}i{\isachardot}{\kern0pt}\ AE\ {\isasymxi}\ in\ M{\isachardot}{\kern0pt}\ {\isadigit{0}}\ {\isasymle}\ C\ i\ {\isasymxi}{\isachardoublequoteclose}\ {\isachardoublequoteopen}{\isasymAnd}i{\isachardot}{\kern0pt}\ AE\ {\isasymxi}\ in\ M{\isachardot}{\kern0pt}\ C\ i\ {\isasymxi}\ {\isasymle}\ R{\isachardoublequoteclose}\isanewline
\ \ \isakeyword{shows}\ {\isachardoublequoteopen}nat{\isacharunderscore}{\kern0pt}submartingale\ M\ F\ {\isacharparenleft}{\kern0pt}{\isasymlambda}n\ {\isasymxi}{\isachardot}{\kern0pt}\ {\isasymSum}i{\isacharless}{\kern0pt}n{\isachardot}{\kern0pt}\ C\ i\ {\isasymxi}\ {\isacharasterisk}{\kern0pt}\isactrlsub R\ {\isacharparenleft}{\kern0pt}X\ {\isacharparenleft}{\kern0pt}Suc\ i{\isacharparenright}{\kern0pt}\ {\isasymxi}\ {\isacharminus}{\kern0pt}\ X\ i\ {\isasymxi}{\isacharparenright}{\kern0pt}{\isacharparenright}{\kern0pt}{\isachardoublequoteclose}\isanewline
%
\isadelimproof
%
\endisadelimproof
%
\isatagproof
\isacommand{proof}\isamarkupfalse%
{\isacharminus}{\kern0pt}\isanewline
\ \ \isacommand{interpret}\isamarkupfalse%
\ C{\isacharcolon}{\kern0pt}\ nat{\isacharunderscore}{\kern0pt}adapted{\isacharunderscore}{\kern0pt}process\ M\ F\ C\ \isacommand{by}\isamarkupfalse%
\ {\isacharparenleft}{\kern0pt}rule\ assms{\isacharparenright}{\kern0pt}\isanewline
\ \ \isacommand{interpret}\isamarkupfalse%
\ C{\isacharprime}{\kern0pt}{\isacharcolon}{\kern0pt}\ nat{\isacharunderscore}{\kern0pt}adapted{\isacharunderscore}{\kern0pt}process\ M\ F\ {\isachardoublequoteopen}{\isasymlambda}i\ {\isasymxi}{\isachardot}{\kern0pt}\ C\ {\isacharparenleft}{\kern0pt}i\ {\isacharminus}{\kern0pt}\ {\isadigit{1}}{\isacharparenright}{\kern0pt}\ {\isasymxi}\ {\isacharasterisk}{\kern0pt}\isactrlsub R\ {\isacharparenleft}{\kern0pt}X\ i\ {\isasymxi}\ {\isacharminus}{\kern0pt}\ X\ {\isacharparenleft}{\kern0pt}i\ {\isacharminus}{\kern0pt}\ {\isadigit{1}}{\isacharparenright}{\kern0pt}\ {\isasymxi}{\isacharparenright}{\kern0pt}{\isachardoublequoteclose}\ \isacommand{by}\isamarkupfalse%
\ {\isacharparenleft}{\kern0pt}intro\ nat{\isacharunderscore}{\kern0pt}adapted{\isacharunderscore}{\kern0pt}process{\isachardot}{\kern0pt}intro\ adapted{\isacharunderscore}{\kern0pt}process{\isachardot}{\kern0pt}scaleR{\isacharunderscore}{\kern0pt}right{\isacharunderscore}{\kern0pt}adapted\ adapted{\isacharunderscore}{\kern0pt}process{\isachardot}{\kern0pt}diff{\isacharunderscore}{\kern0pt}adapted{\isacharcomma}{\kern0pt}\ unfold{\isacharunderscore}{\kern0pt}locales{\isacharparenright}{\kern0pt}\ {\isacharparenleft}{\kern0pt}auto\ intro{\isacharcolon}{\kern0pt}\ adaptedD\ C{\isachardot}{\kern0pt}adaptedD{\isacharparenright}{\kern0pt}{\isacharplus}{\kern0pt}\isanewline
\ \ \isacommand{interpret}\isamarkupfalse%
\ C{\isacharprime}{\kern0pt}{\isacharprime}{\kern0pt}{\isacharcolon}{\kern0pt}\ nat{\isacharunderscore}{\kern0pt}adapted{\isacharunderscore}{\kern0pt}process\ M\ F\ {\isachardoublequoteopen}{\isasymlambda}n\ {\isasymxi}{\isachardot}{\kern0pt}\ {\isasymSum}i{\isacharless}{\kern0pt}n{\isachardot}{\kern0pt}\ C\ i\ {\isasymxi}\ {\isacharasterisk}{\kern0pt}\isactrlsub R\ {\isacharparenleft}{\kern0pt}X\ {\isacharparenleft}{\kern0pt}Suc\ i{\isacharparenright}{\kern0pt}\ {\isasymxi}\ {\isacharminus}{\kern0pt}\ X\ i\ {\isasymxi}{\isacharparenright}{\kern0pt}{\isachardoublequoteclose}\ \isacommand{by}\isamarkupfalse%
\ {\isacharparenleft}{\kern0pt}rule\ C{\isacharprime}{\kern0pt}{\isachardot}{\kern0pt}partial{\isacharunderscore}{\kern0pt}sum{\isacharunderscore}{\kern0pt}Suc{\isacharunderscore}{\kern0pt}adapted{\isacharbrackleft}{\kern0pt}unfolded\ diff{\isacharunderscore}{\kern0pt}Suc{\isacharunderscore}{\kern0pt}{\isadigit{1}}{\isacharbrackright}{\kern0pt}{\isacharparenright}{\kern0pt}\isanewline
\ \ \isacommand{interpret}\isamarkupfalse%
\ S{\isacharcolon}{\kern0pt}\ nat{\isacharunderscore}{\kern0pt}sigma{\isacharunderscore}{\kern0pt}finite{\isacharunderscore}{\kern0pt}adapted{\isacharunderscore}{\kern0pt}process{\isacharunderscore}{\kern0pt}linorder\ M\ F\ {\isachardoublequoteopen}{\isacharparenleft}{\kern0pt}{\isasymlambda}n\ {\isasymxi}{\isachardot}{\kern0pt}\ {\isasymSum}i{\isacharless}{\kern0pt}n{\isachardot}{\kern0pt}\ C\ i\ {\isasymxi}\ {\isacharasterisk}{\kern0pt}\isactrlsub R\ {\isacharparenleft}{\kern0pt}X\ {\isacharparenleft}{\kern0pt}Suc\ i{\isacharparenright}{\kern0pt}\ {\isasymxi}\ {\isacharminus}{\kern0pt}\ X\ i\ {\isasymxi}{\isacharparenright}{\kern0pt}{\isacharparenright}{\kern0pt}{\isachardoublequoteclose}\ \isacommand{{\isachardot}{\kern0pt}{\isachardot}{\kern0pt}}\isamarkupfalse%
\isanewline
\ \ \isacommand{have}\isamarkupfalse%
\ {\isachardoublequoteopen}integrable\ M\ {\isacharparenleft}{\kern0pt}{\isasymlambda}x{\isachardot}{\kern0pt}\ C\ i\ x\ {\isacharasterisk}{\kern0pt}\isactrlsub R\ {\isacharparenleft}{\kern0pt}X\ {\isacharparenleft}{\kern0pt}Suc\ i{\isacharparenright}{\kern0pt}\ x\ {\isacharminus}{\kern0pt}\ X\ i\ x{\isacharparenright}{\kern0pt}{\isacharparenright}{\kern0pt}{\isachardoublequoteclose}\ \isakeyword{for}\ i\ \isacommand{using}\isamarkupfalse%
\ assms{\isacharparenleft}{\kern0pt}{\isadigit{2}}{\isacharcomma}{\kern0pt}{\isadigit{3}}{\isacharparenright}{\kern0pt}{\isacharbrackleft}{\kern0pt}of\ i{\isacharbrackright}{\kern0pt}\ \isacommand{by}\isamarkupfalse%
\ {\isacharparenleft}{\kern0pt}intro\ Bochner{\isacharunderscore}{\kern0pt}Integration{\isachardot}{\kern0pt}integrable{\isacharunderscore}{\kern0pt}bound{\isacharbrackleft}{\kern0pt}OF\ integrable{\isacharunderscore}{\kern0pt}scaleR{\isacharunderscore}{\kern0pt}right{\isacharcomma}{\kern0pt}\ OF\ Bochner{\isacharunderscore}{\kern0pt}Integration{\isachardot}{\kern0pt}integrable{\isacharunderscore}{\kern0pt}diff{\isacharcomma}{\kern0pt}\ OF\ integrable{\isacharparenleft}{\kern0pt}{\isadigit{1}}{\isacharcomma}{\kern0pt}{\isadigit{1}}{\isacharparenright}{\kern0pt}{\isacharcomma}{\kern0pt}\ of\ R\ {\isachardoublequoteopen}Suc\ i{\isachardoublequoteclose}\ i{\isacharbrackright}{\kern0pt}{\isacharparenright}{\kern0pt}\ {\isacharparenleft}{\kern0pt}auto\ simp\ add{\isacharcolon}{\kern0pt}\ mult{\isacharunderscore}{\kern0pt}mono{\isacharparenright}{\kern0pt}\isanewline
\ \ \isacommand{moreover}\isamarkupfalse%
\ \isacommand{have}\isamarkupfalse%
\ {\isachardoublequoteopen}AE\ {\isasymxi}\ in\ M{\isachardot}{\kern0pt}\ {\isadigit{0}}\ {\isasymle}\ cond{\isacharunderscore}{\kern0pt}exp\ M\ {\isacharparenleft}{\kern0pt}F\ i{\isacharparenright}{\kern0pt}\ {\isacharparenleft}{\kern0pt}{\isasymlambda}{\isasymxi}{\isachardot}{\kern0pt}\ {\isacharparenleft}{\kern0pt}{\isasymSum}i{\isacharless}{\kern0pt}Suc\ i{\isachardot}{\kern0pt}\ C\ i\ {\isasymxi}\ {\isacharasterisk}{\kern0pt}\isactrlsub R\ {\isacharparenleft}{\kern0pt}X\ {\isacharparenleft}{\kern0pt}Suc\ i{\isacharparenright}{\kern0pt}\ {\isasymxi}\ {\isacharminus}{\kern0pt}\ X\ i\ {\isasymxi}{\isacharparenright}{\kern0pt}{\isacharparenright}{\kern0pt}\ {\isacharminus}{\kern0pt}\ {\isacharparenleft}{\kern0pt}{\isasymSum}i{\isacharless}{\kern0pt}i{\isachardot}{\kern0pt}\ C\ i\ {\isasymxi}\ {\isacharasterisk}{\kern0pt}\isactrlsub R\ {\isacharparenleft}{\kern0pt}X\ {\isacharparenleft}{\kern0pt}Suc\ i{\isacharparenright}{\kern0pt}\ {\isasymxi}\ {\isacharminus}{\kern0pt}\ X\ i\ {\isasymxi}{\isacharparenright}{\kern0pt}{\isacharparenright}{\kern0pt}{\isacharparenright}{\kern0pt}\ {\isasymxi}{\isachardoublequoteclose}\ \isakeyword{for}\ i\ \isanewline
\ \ \ \ \isacommand{using}\isamarkupfalse%
\ sigma{\isacharunderscore}{\kern0pt}finite{\isacharunderscore}{\kern0pt}subalgebra{\isachardot}{\kern0pt}cond{\isacharunderscore}{\kern0pt}exp{\isacharunderscore}{\kern0pt}measurable{\isacharunderscore}{\kern0pt}scaleR{\isacharbrackleft}{\kern0pt}OF\ {\isacharunderscore}{\kern0pt}\ calculation\ {\isacharunderscore}{\kern0pt}\ C{\isachardot}{\kern0pt}adapted{\isacharcomma}{\kern0pt}\ of\ i{\isacharbrackright}{\kern0pt}\ \isanewline
\ \ \ \ \ \ \ \ \ \ cond{\isacharunderscore}{\kern0pt}exp{\isacharunderscore}{\kern0pt}diff{\isacharunderscore}{\kern0pt}nonneg{\isacharbrackleft}{\kern0pt}OF\ {\isacharunderscore}{\kern0pt}\ le{\isacharunderscore}{\kern0pt}SucI{\isacharcomma}{\kern0pt}\ OF\ {\isacharunderscore}{\kern0pt}\ order{\isachardot}{\kern0pt}refl{\isacharcomma}{\kern0pt}\ of\ i{\isacharbrackright}{\kern0pt}\ assms{\isacharparenleft}{\kern0pt}{\isadigit{2}}{\isacharcomma}{\kern0pt}{\isadigit{3}}{\isacharparenright}{\kern0pt}{\isacharbrackleft}{\kern0pt}of\ i{\isacharbrackright}{\kern0pt}\ \isacommand{by}\isamarkupfalse%
\ {\isacharparenleft}{\kern0pt}fastforce\ simp\ add{\isacharcolon}{\kern0pt}\ scaleR{\isacharunderscore}{\kern0pt}nonneg{\isacharunderscore}{\kern0pt}nonneg\ integrable{\isacharparenright}{\kern0pt}\isanewline
\ \ \isacommand{ultimately}\isamarkupfalse%
\ \isacommand{show}\isamarkupfalse%
\ {\isacharquery}{\kern0pt}thesis\ \isacommand{by}\isamarkupfalse%
\ {\isacharparenleft}{\kern0pt}intro\ S{\isachardot}{\kern0pt}submartingale{\isacharunderscore}{\kern0pt}of{\isacharunderscore}{\kern0pt}cond{\isacharunderscore}{\kern0pt}exp{\isacharunderscore}{\kern0pt}diff{\isacharunderscore}{\kern0pt}Suc{\isacharunderscore}{\kern0pt}nonneg\ Bochner{\isacharunderscore}{\kern0pt}Integration{\isachardot}{\kern0pt}integrable{\isacharunderscore}{\kern0pt}sum{\isacharcomma}{\kern0pt}\ blast{\isacharplus}{\kern0pt}{\isacharparenright}{\kern0pt}\isanewline
\isacommand{qed}\isamarkupfalse%
%
\endisatagproof
{\isafoldproof}%
%
\isadelimproof
\isanewline
%
\endisadelimproof
\isanewline
\isacommand{lemma}\isamarkupfalse%
\ {\isacharparenleft}{\kern0pt}\isakeyword{in}\ nat{\isacharunderscore}{\kern0pt}submartingale{\isacharunderscore}{\kern0pt}linorder{\isacharparenright}{\kern0pt}\ partial{\isacharunderscore}{\kern0pt}sum{\isacharunderscore}{\kern0pt}scaleR{\isacharprime}{\kern0pt}{\isacharcolon}{\kern0pt}\isanewline
\ \ \isakeyword{assumes}\ {\isachardoublequoteopen}nat{\isacharunderscore}{\kern0pt}predictable{\isacharunderscore}{\kern0pt}process\ M\ F\ C{\isachardoublequoteclose}\ {\isachardoublequoteopen}{\isasymAnd}i{\isachardot}{\kern0pt}\ AE\ {\isasymxi}\ in\ M{\isachardot}{\kern0pt}\ {\isadigit{0}}\ {\isasymle}\ C\ i\ {\isasymxi}{\isachardoublequoteclose}\ {\isachardoublequoteopen}{\isasymAnd}i{\isachardot}{\kern0pt}\ AE\ {\isasymxi}\ in\ M{\isachardot}{\kern0pt}\ C\ i\ {\isasymxi}\ {\isasymle}\ R{\isachardoublequoteclose}\isanewline
\ \ \isakeyword{shows}\ {\isachardoublequoteopen}nat{\isacharunderscore}{\kern0pt}submartingale\ M\ F\ {\isacharparenleft}{\kern0pt}{\isasymlambda}n\ {\isasymxi}{\isachardot}{\kern0pt}\ {\isasymSum}i{\isacharless}{\kern0pt}n{\isachardot}{\kern0pt}\ C\ {\isacharparenleft}{\kern0pt}Suc\ i{\isacharparenright}{\kern0pt}\ {\isasymxi}\ {\isacharasterisk}{\kern0pt}\isactrlsub R\ {\isacharparenleft}{\kern0pt}X\ {\isacharparenleft}{\kern0pt}Suc\ i{\isacharparenright}{\kern0pt}\ {\isasymxi}\ {\isacharminus}{\kern0pt}\ X\ i\ {\isasymxi}{\isacharparenright}{\kern0pt}{\isacharparenright}{\kern0pt}{\isachardoublequoteclose}\isanewline
%
\isadelimproof
%
\endisadelimproof
%
\isatagproof
\isacommand{proof}\isamarkupfalse%
\ {\isacharminus}{\kern0pt}\isanewline
\ \ \isacommand{interpret}\isamarkupfalse%
\ C{\isacharcolon}{\kern0pt}\ nat{\isacharunderscore}{\kern0pt}predictable{\isacharunderscore}{\kern0pt}process\ M\ F\ C\ \isacommand{by}\isamarkupfalse%
\ {\isacharparenleft}{\kern0pt}rule\ assms{\isacharparenright}{\kern0pt}\isanewline
\ \ \isacommand{interpret}\isamarkupfalse%
\ Suc{\isacharunderscore}{\kern0pt}C{\isacharcolon}{\kern0pt}\ nat{\isacharunderscore}{\kern0pt}adapted{\isacharunderscore}{\kern0pt}process\ M\ F\ {\isachardoublequoteopen}{\isasymlambda}i{\isachardot}{\kern0pt}\ C\ {\isacharparenleft}{\kern0pt}Suc\ i{\isacharparenright}{\kern0pt}{\isachardoublequoteclose}\ \isacommand{using}\isamarkupfalse%
\ C{\isachardot}{\kern0pt}adapted{\isacharunderscore}{\kern0pt}Suc\ \isacommand{{\isachardot}{\kern0pt}}\isamarkupfalse%
\isanewline
\ \ \isacommand{show}\isamarkupfalse%
\ {\isacharquery}{\kern0pt}thesis\ \isacommand{by}\isamarkupfalse%
\ {\isacharparenleft}{\kern0pt}intro\ partial{\isacharunderscore}{\kern0pt}sum{\isacharunderscore}{\kern0pt}scaleR{\isacharbrackleft}{\kern0pt}of\ {\isacharunderscore}{\kern0pt}\ R{\isacharbrackright}{\kern0pt}\ assms{\isacharparenright}{\kern0pt}\ {\isacharparenleft}{\kern0pt}intro{\isacharunderscore}{\kern0pt}locales{\isacharparenright}{\kern0pt}\isanewline
\isacommand{qed}\isamarkupfalse%
%
\endisatagproof
{\isafoldproof}%
%
\isadelimproof
%
\endisadelimproof
%
\isadelimdocument
%
\endisadelimdocument
%
\isatagdocument
%
\isamarkupsubsection{Discrete Time Supermartingales%
}
\isamarkuptrue%
%
\endisatagdocument
{\isafolddocument}%
%
\isadelimdocument
%
\endisadelimdocument
\isacommand{lemma}\isamarkupfalse%
\ {\isacharparenleft}{\kern0pt}\isakeyword{in}\ nat{\isacharunderscore}{\kern0pt}supermartingale{\isacharparenright}{\kern0pt}\ predictable{\isacharunderscore}{\kern0pt}mono{\isacharcolon}{\kern0pt}\isanewline
\ \ \isakeyword{assumes}\ {\isachardoublequoteopen}nat{\isacharunderscore}{\kern0pt}predictable{\isacharunderscore}{\kern0pt}process\ M\ F\ X{\isachardoublequoteclose}\ {\isachardoublequoteopen}i\ {\isasymle}\ j{\isachardoublequoteclose}\isanewline
\ \ \isakeyword{shows}\ {\isachardoublequoteopen}AE\ {\isasymxi}\ in\ M{\isachardot}{\kern0pt}\ X\ i\ {\isasymxi}\ {\isasymge}\ X\ j\ {\isasymxi}{\isachardoublequoteclose}\isanewline
%
\isadelimproof
\ \ %
\endisadelimproof
%
\isatagproof
\isacommand{using}\isamarkupfalse%
\ assms{\isacharparenleft}{\kern0pt}{\isadigit{2}}{\isacharparenright}{\kern0pt}\isanewline
\isacommand{proof}\isamarkupfalse%
\ {\isacharparenleft}{\kern0pt}induction\ {\isachardoublequoteopen}j\ {\isacharminus}{\kern0pt}\ i{\isachardoublequoteclose}\ arbitrary{\isacharcolon}{\kern0pt}\ i\ j{\isacharparenright}{\kern0pt}\isanewline
\ \ \isacommand{case}\isamarkupfalse%
\ {\isadigit{0}}\isanewline
\ \ \isacommand{then}\isamarkupfalse%
\ \isacommand{show}\isamarkupfalse%
\ {\isacharquery}{\kern0pt}case\ \isacommand{by}\isamarkupfalse%
\ simp\ \isanewline
\isacommand{next}\isamarkupfalse%
\isanewline
\ \ \isacommand{case}\isamarkupfalse%
\ {\isacharparenleft}{\kern0pt}Suc\ n{\isacharparenright}{\kern0pt}\isanewline
\ \ \isacommand{hence}\isamarkupfalse%
\ {\isacharasterisk}{\kern0pt}{\isacharcolon}{\kern0pt}\ {\isachardoublequoteopen}n\ {\isacharequal}{\kern0pt}\ j\ {\isacharminus}{\kern0pt}\ Suc\ i{\isachardoublequoteclose}\ \isacommand{by}\isamarkupfalse%
\ linarith\isanewline
\ \ \isacommand{interpret}\isamarkupfalse%
\ S{\isacharcolon}{\kern0pt}\ nat{\isacharunderscore}{\kern0pt}adapted{\isacharunderscore}{\kern0pt}process\ M\ F\ {\isachardoublequoteopen}{\isasymlambda}i{\isachardot}{\kern0pt}\ X\ {\isacharparenleft}{\kern0pt}Suc\ i{\isacharparenright}{\kern0pt}{\isachardoublequoteclose}\ \isacommand{by}\isamarkupfalse%
\ {\isacharparenleft}{\kern0pt}intro\ nat{\isacharunderscore}{\kern0pt}predictable{\isacharunderscore}{\kern0pt}process{\isachardot}{\kern0pt}adapted{\isacharunderscore}{\kern0pt}Suc\ assms{\isacharparenright}{\kern0pt}\isanewline
\ \ \isacommand{have}\isamarkupfalse%
\ {\isachardoublequoteopen}Suc\ i\ {\isasymle}\ j{\isachardoublequoteclose}\ \isacommand{using}\isamarkupfalse%
\ Suc{\isacharparenleft}{\kern0pt}{\isadigit{2}}{\isacharcomma}{\kern0pt}{\isadigit{3}}{\isacharparenright}{\kern0pt}\ \isacommand{by}\isamarkupfalse%
\ linarith\isanewline
\ \ \isacommand{thus}\isamarkupfalse%
\ {\isacharquery}{\kern0pt}case\ \isacommand{using}\isamarkupfalse%
\ Suc{\isacharparenleft}{\kern0pt}{\isadigit{1}}{\isacharparenright}{\kern0pt}{\isacharbrackleft}{\kern0pt}OF\ {\isacharasterisk}{\kern0pt}{\isacharbrackright}{\kern0pt}\ S{\isachardot}{\kern0pt}adapted{\isacharbrackleft}{\kern0pt}of\ i{\isacharbrackright}{\kern0pt}\ supermartingale{\isacharunderscore}{\kern0pt}property{\isacharbrackleft}{\kern0pt}OF\ {\isacharunderscore}{\kern0pt}\ le{\isacharunderscore}{\kern0pt}SucI{\isacharcomma}{\kern0pt}\ of\ i{\isacharbrackright}{\kern0pt}\ sigma{\isacharunderscore}{\kern0pt}finite{\isacharunderscore}{\kern0pt}subalgebra{\isachardot}{\kern0pt}cond{\isacharunderscore}{\kern0pt}exp{\isacharunderscore}{\kern0pt}F{\isacharunderscore}{\kern0pt}meas{\isacharbrackleft}{\kern0pt}OF\ {\isacharunderscore}{\kern0pt}\ integrable{\isacharcomma}{\kern0pt}\ of\ {\isachardoublequoteopen}F\ i{\isachardoublequoteclose}\ {\isachardoublequoteopen}Suc\ i{\isachardoublequoteclose}{\isacharbrackright}{\kern0pt}\ \isacommand{by}\isamarkupfalse%
\ fastforce\isanewline
\isacommand{qed}\isamarkupfalse%
%
\endisatagproof
{\isafoldproof}%
%
\isadelimproof
\isanewline
%
\endisadelimproof
\ \ \isanewline
\isacommand{lemma}\isamarkupfalse%
\ {\isacharparenleft}{\kern0pt}\isakeyword{in}\ nat{\isacharunderscore}{\kern0pt}sigma{\isacharunderscore}{\kern0pt}finite{\isacharunderscore}{\kern0pt}adapted{\isacharunderscore}{\kern0pt}process{\isacharunderscore}{\kern0pt}linorder{\isacharparenright}{\kern0pt}\ supermartingale{\isacharunderscore}{\kern0pt}of{\isacharunderscore}{\kern0pt}set{\isacharunderscore}{\kern0pt}integral{\isacharunderscore}{\kern0pt}ge{\isacharunderscore}{\kern0pt}Suc{\isacharcolon}{\kern0pt}\isanewline
\ \ \isakeyword{assumes}\ integrable{\isacharcolon}{\kern0pt}\ {\isachardoublequoteopen}{\isasymAnd}i{\isachardot}{\kern0pt}\ integrable\ M\ {\isacharparenleft}{\kern0pt}X\ i{\isacharparenright}{\kern0pt}{\isachardoublequoteclose}\ \isanewline
\ \ \ \ \ \ \isakeyword{and}\ {\isachardoublequoteopen}{\isasymAnd}A\ i{\isachardot}{\kern0pt}\ A\ {\isasymin}\ F\ i\ {\isasymLongrightarrow}\ set{\isacharunderscore}{\kern0pt}lebesgue{\isacharunderscore}{\kern0pt}integral\ M\ A\ {\isacharparenleft}{\kern0pt}X\ i{\isacharparenright}{\kern0pt}\ {\isasymge}\ set{\isacharunderscore}{\kern0pt}lebesgue{\isacharunderscore}{\kern0pt}integral\ M\ A\ {\isacharparenleft}{\kern0pt}X\ {\isacharparenleft}{\kern0pt}Suc\ i{\isacharparenright}{\kern0pt}{\isacharparenright}{\kern0pt}{\isachardoublequoteclose}\ \isanewline
\ \ \ \ \isakeyword{shows}\ {\isachardoublequoteopen}nat{\isacharunderscore}{\kern0pt}supermartingale\ M\ F\ X{\isachardoublequoteclose}\isanewline
%
\isadelimproof
%
\endisadelimproof
%
\isatagproof
\isacommand{proof}\isamarkupfalse%
\ {\isacharminus}{\kern0pt}\isanewline
\ \ \isacommand{interpret}\isamarkupfalse%
\ {\isacharunderscore}{\kern0pt}{\isacharcolon}{\kern0pt}\ adapted{\isacharunderscore}{\kern0pt}process\ M\ F\ {\isadigit{0}}\ {\isachardoublequoteopen}{\isacharminus}{\kern0pt}X{\isachardoublequoteclose}\ \isacommand{by}\isamarkupfalse%
\ {\isacharparenleft}{\kern0pt}rule\ uminus{\isacharunderscore}{\kern0pt}adapted{\isacharparenright}{\kern0pt}\isanewline
\ \ \isacommand{interpret}\isamarkupfalse%
\ uminus{\isacharunderscore}{\kern0pt}X{\isacharcolon}{\kern0pt}\ nat{\isacharunderscore}{\kern0pt}sigma{\isacharunderscore}{\kern0pt}finite{\isacharunderscore}{\kern0pt}adapted{\isacharunderscore}{\kern0pt}process{\isacharunderscore}{\kern0pt}linorder\ M\ F\ {\isachardoublequoteopen}{\isacharminus}{\kern0pt}X{\isachardoublequoteclose}\ \isacommand{{\isachardot}{\kern0pt}{\isachardot}{\kern0pt}}\isamarkupfalse%
\isanewline
\ \ \isacommand{note}\isamarkupfalse%
\ {\isacharasterisk}{\kern0pt}\ {\isacharequal}{\kern0pt}\ set{\isacharunderscore}{\kern0pt}integral{\isacharunderscore}{\kern0pt}uminus{\isacharbrackleft}{\kern0pt}unfolded\ set{\isacharunderscore}{\kern0pt}integrable{\isacharunderscore}{\kern0pt}def{\isacharcomma}{\kern0pt}\ OF\ integrable{\isacharunderscore}{\kern0pt}mult{\isacharunderscore}{\kern0pt}indicator{\isacharbrackleft}{\kern0pt}OF\ {\isacharunderscore}{\kern0pt}\ integrable{\isacharbrackright}{\kern0pt}{\isacharbrackright}{\kern0pt}\isanewline
\ \ \isacommand{have}\isamarkupfalse%
\ {\isachardoublequoteopen}nat{\isacharunderscore}{\kern0pt}supermartingale\ M\ F\ {\isacharparenleft}{\kern0pt}{\isacharminus}{\kern0pt}{\isacharparenleft}{\kern0pt}{\isacharminus}{\kern0pt}\ X{\isacharparenright}{\kern0pt}{\isacharparenright}{\kern0pt}{\isachardoublequoteclose}\ \isanewline
\ \ \ \ \isacommand{using}\isamarkupfalse%
\ ord{\isacharunderscore}{\kern0pt}eq{\isacharunderscore}{\kern0pt}le{\isacharunderscore}{\kern0pt}trans{\isacharbrackleft}{\kern0pt}OF\ {\isacharasterisk}{\kern0pt}\ ord{\isacharunderscore}{\kern0pt}le{\isacharunderscore}{\kern0pt}eq{\isacharunderscore}{\kern0pt}trans{\isacharbrackleft}{\kern0pt}OF\ le{\isacharunderscore}{\kern0pt}imp{\isacharunderscore}{\kern0pt}neg{\isacharunderscore}{\kern0pt}le{\isacharbrackleft}{\kern0pt}OF\ assms{\isacharparenleft}{\kern0pt}{\isadigit{2}}{\isacharparenright}{\kern0pt}{\isacharbrackright}{\kern0pt}\ {\isacharasterisk}{\kern0pt}{\isacharbrackleft}{\kern0pt}symmetric{\isacharbrackright}{\kern0pt}{\isacharbrackright}{\kern0pt}{\isacharbrackright}{\kern0pt}\ subalgebras\isanewline
\ \ \ \ \isacommand{by}\isamarkupfalse%
\ {\isacharparenleft}{\kern0pt}intro\ nat{\isacharunderscore}{\kern0pt}supermartingale{\isachardot}{\kern0pt}intro\ submartingale{\isachardot}{\kern0pt}uminus\ nat{\isacharunderscore}{\kern0pt}submartingale{\isachardot}{\kern0pt}axioms\ uminus{\isacharunderscore}{\kern0pt}X{\isachardot}{\kern0pt}submartingale{\isacharunderscore}{\kern0pt}of{\isacharunderscore}{\kern0pt}set{\isacharunderscore}{\kern0pt}integral{\isacharunderscore}{\kern0pt}le{\isacharunderscore}{\kern0pt}Suc{\isacharparenright}{\kern0pt}\ \isanewline
\ \ \ \ \ \ \ {\isacharparenleft}{\kern0pt}clarsimp\ simp\ add{\isacharcolon}{\kern0pt}\ fun{\isacharunderscore}{\kern0pt}Compl{\isacharunderscore}{\kern0pt}def\ subalgebra{\isacharunderscore}{\kern0pt}def\ integrable\ {\isacharbar}{\kern0pt}\ fastforce{\isacharparenright}{\kern0pt}{\isacharplus}{\kern0pt}\isanewline
\ \ \isacommand{thus}\isamarkupfalse%
\ {\isacharquery}{\kern0pt}thesis\ \isacommand{unfolding}\isamarkupfalse%
\ fun{\isacharunderscore}{\kern0pt}Compl{\isacharunderscore}{\kern0pt}def\ \isacommand{by}\isamarkupfalse%
\ simp\isanewline
\isacommand{qed}\isamarkupfalse%
%
\endisatagproof
{\isafoldproof}%
%
\isadelimproof
\isanewline
%
\endisadelimproof
\isanewline
\isacommand{lemma}\isamarkupfalse%
\ {\isacharparenleft}{\kern0pt}\isakeyword{in}\ nat{\isacharunderscore}{\kern0pt}sigma{\isacharunderscore}{\kern0pt}finite{\isacharunderscore}{\kern0pt}adapted{\isacharunderscore}{\kern0pt}process{\isacharunderscore}{\kern0pt}linorder{\isacharparenright}{\kern0pt}\ supermartingale{\isacharunderscore}{\kern0pt}nat{\isacharcolon}{\kern0pt}\isanewline
\ \ \isakeyword{assumes}\ integrable{\isacharcolon}{\kern0pt}\ {\isachardoublequoteopen}{\isasymAnd}i{\isachardot}{\kern0pt}\ integrable\ M\ {\isacharparenleft}{\kern0pt}X\ i{\isacharparenright}{\kern0pt}{\isachardoublequoteclose}\ \isanewline
\ \ \ \ \ \ \isakeyword{and}\ {\isachardoublequoteopen}{\isasymAnd}i{\isachardot}{\kern0pt}\ AE\ {\isasymxi}\ in\ M{\isachardot}{\kern0pt}\ X\ i\ {\isasymxi}\ {\isasymge}\ cond{\isacharunderscore}{\kern0pt}exp\ M\ {\isacharparenleft}{\kern0pt}F\ i{\isacharparenright}{\kern0pt}\ {\isacharparenleft}{\kern0pt}X\ {\isacharparenleft}{\kern0pt}Suc\ i{\isacharparenright}{\kern0pt}{\isacharparenright}{\kern0pt}\ {\isasymxi}{\isachardoublequoteclose}\ \isanewline
\ \ \ \ \isakeyword{shows}\ {\isachardoublequoteopen}nat{\isacharunderscore}{\kern0pt}supermartingale\ M\ F\ X{\isachardoublequoteclose}\isanewline
%
\isadelimproof
%
\endisadelimproof
%
\isatagproof
\isacommand{proof}\isamarkupfalse%
\ {\isacharminus}{\kern0pt}\isanewline
\ \ \isacommand{interpret}\isamarkupfalse%
\ {\isacharunderscore}{\kern0pt}{\isacharcolon}{\kern0pt}\ adapted{\isacharunderscore}{\kern0pt}process\ M\ F\ {\isadigit{0}}\ {\isachardoublequoteopen}{\isacharminus}{\kern0pt}X{\isachardoublequoteclose}\ \isacommand{by}\isamarkupfalse%
\ {\isacharparenleft}{\kern0pt}rule\ uminus{\isacharunderscore}{\kern0pt}adapted{\isacharparenright}{\kern0pt}\isanewline
\ \ \isacommand{interpret}\isamarkupfalse%
\ uminus{\isacharunderscore}{\kern0pt}X{\isacharcolon}{\kern0pt}\ nat{\isacharunderscore}{\kern0pt}sigma{\isacharunderscore}{\kern0pt}finite{\isacharunderscore}{\kern0pt}adapted{\isacharunderscore}{\kern0pt}process{\isacharunderscore}{\kern0pt}linorder\ M\ F\ {\isachardoublequoteopen}{\isacharminus}{\kern0pt}X{\isachardoublequoteclose}\ \isacommand{{\isachardot}{\kern0pt}{\isachardot}{\kern0pt}}\isamarkupfalse%
\isanewline
\ \ \isacommand{have}\isamarkupfalse%
\ {\isachardoublequoteopen}AE\ {\isasymxi}\ in\ M{\isachardot}{\kern0pt}\ {\isacharminus}{\kern0pt}\ X\ i\ {\isasymxi}\ {\isasymle}\ cond{\isacharunderscore}{\kern0pt}exp\ M\ {\isacharparenleft}{\kern0pt}F\ i{\isacharparenright}{\kern0pt}\ {\isacharparenleft}{\kern0pt}{\isasymlambda}x{\isachardot}{\kern0pt}\ {\isacharminus}{\kern0pt}\ X\ {\isacharparenleft}{\kern0pt}Suc\ i{\isacharparenright}{\kern0pt}\ x{\isacharparenright}{\kern0pt}\ {\isasymxi}{\isachardoublequoteclose}\ \isakeyword{for}\ i\ \isacommand{using}\isamarkupfalse%
\ assms{\isacharparenleft}{\kern0pt}{\isadigit{2}}{\isacharparenright}{\kern0pt}\ cond{\isacharunderscore}{\kern0pt}exp{\isacharunderscore}{\kern0pt}uminus{\isacharbrackleft}{\kern0pt}OF\ integrable{\isacharcomma}{\kern0pt}\ of\ i\ {\isachardoublequoteopen}Suc\ i{\isachardoublequoteclose}{\isacharbrackright}{\kern0pt}\ \isacommand{by}\isamarkupfalse%
\ force\isanewline
\ \ \isacommand{hence}\isamarkupfalse%
\ {\isachardoublequoteopen}nat{\isacharunderscore}{\kern0pt}supermartingale\ M\ F\ {\isacharparenleft}{\kern0pt}{\isacharminus}{\kern0pt}{\isacharparenleft}{\kern0pt}{\isacharminus}{\kern0pt}\ X{\isacharparenright}{\kern0pt}{\isacharparenright}{\kern0pt}{\isachardoublequoteclose}\ \isacommand{by}\isamarkupfalse%
\ {\isacharparenleft}{\kern0pt}intro\ nat{\isacharunderscore}{\kern0pt}supermartingale{\isachardot}{\kern0pt}intro\ submartingale{\isachardot}{\kern0pt}uminus\ nat{\isacharunderscore}{\kern0pt}submartingale{\isachardot}{\kern0pt}axioms\ uminus{\isacharunderscore}{\kern0pt}X{\isachardot}{\kern0pt}submartingale{\isacharunderscore}{\kern0pt}nat{\isacharparenright}{\kern0pt}\ {\isacharparenleft}{\kern0pt}auto\ simp\ add{\isacharcolon}{\kern0pt}\ fun{\isacharunderscore}{\kern0pt}Compl{\isacharunderscore}{\kern0pt}def\ integrable{\isacharparenright}{\kern0pt}\isanewline
\ \ \isacommand{thus}\isamarkupfalse%
\ {\isacharquery}{\kern0pt}thesis\ \isacommand{unfolding}\isamarkupfalse%
\ fun{\isacharunderscore}{\kern0pt}Compl{\isacharunderscore}{\kern0pt}def\ \isacommand{by}\isamarkupfalse%
\ simp\isanewline
\isacommand{qed}\isamarkupfalse%
%
\endisatagproof
{\isafoldproof}%
%
\isadelimproof
\isanewline
%
\endisadelimproof
\isanewline
\isacommand{lemma}\isamarkupfalse%
\ {\isacharparenleft}{\kern0pt}\isakeyword{in}\ nat{\isacharunderscore}{\kern0pt}sigma{\isacharunderscore}{\kern0pt}finite{\isacharunderscore}{\kern0pt}adapted{\isacharunderscore}{\kern0pt}process{\isacharunderscore}{\kern0pt}linorder{\isacharparenright}{\kern0pt}\ supermartingale{\isacharunderscore}{\kern0pt}of{\isacharunderscore}{\kern0pt}cond{\isacharunderscore}{\kern0pt}exp{\isacharunderscore}{\kern0pt}diff{\isacharunderscore}{\kern0pt}Suc{\isacharunderscore}{\kern0pt}le{\isacharunderscore}{\kern0pt}zero{\isacharcolon}{\kern0pt}\isanewline
\ \ \isakeyword{assumes}\ integrable{\isacharcolon}{\kern0pt}\ {\isachardoublequoteopen}{\isasymAnd}i{\isachardot}{\kern0pt}\ integrable\ M\ {\isacharparenleft}{\kern0pt}X\ i{\isacharparenright}{\kern0pt}{\isachardoublequoteclose}\ \isanewline
\ \ \ \ \ \ \isakeyword{and}\ {\isachardoublequoteopen}{\isasymAnd}i{\isachardot}{\kern0pt}\ AE\ {\isasymxi}\ in\ M{\isachardot}{\kern0pt}\ cond{\isacharunderscore}{\kern0pt}exp\ M\ {\isacharparenleft}{\kern0pt}F\ i{\isacharparenright}{\kern0pt}\ {\isacharparenleft}{\kern0pt}{\isasymlambda}{\isasymxi}{\isachardot}{\kern0pt}\ X\ {\isacharparenleft}{\kern0pt}Suc\ i{\isacharparenright}{\kern0pt}\ {\isasymxi}\ {\isacharminus}{\kern0pt}\ X\ i\ {\isasymxi}{\isacharparenright}{\kern0pt}\ {\isasymxi}\ {\isasymle}\ {\isadigit{0}}{\isachardoublequoteclose}\ \isanewline
\ \ \ \ \isakeyword{shows}\ {\isachardoublequoteopen}nat{\isacharunderscore}{\kern0pt}supermartingale\ M\ F\ X{\isachardoublequoteclose}\isanewline
%
\isadelimproof
%
\endisadelimproof
%
\isatagproof
\isacommand{proof}\isamarkupfalse%
\ {\isacharparenleft}{\kern0pt}intro\ supermartingale{\isacharunderscore}{\kern0pt}nat\ integrable{\isacharparenright}{\kern0pt}\ \isanewline
\ \ \isacommand{fix}\isamarkupfalse%
\ i\isanewline
\ \ \isacommand{show}\isamarkupfalse%
\ {\isachardoublequoteopen}AE\ {\isasymxi}\ in\ M{\isachardot}{\kern0pt}\ X\ i\ {\isasymxi}\ {\isasymge}\ cond{\isacharunderscore}{\kern0pt}exp\ M\ {\isacharparenleft}{\kern0pt}F\ i{\isacharparenright}{\kern0pt}\ {\isacharparenleft}{\kern0pt}X\ {\isacharparenleft}{\kern0pt}Suc\ i{\isacharparenright}{\kern0pt}{\isacharparenright}{\kern0pt}\ {\isasymxi}{\isachardoublequoteclose}\ \isacommand{using}\isamarkupfalse%
\ cond{\isacharunderscore}{\kern0pt}exp{\isacharunderscore}{\kern0pt}diff{\isacharbrackleft}{\kern0pt}OF\ integrable{\isacharparenleft}{\kern0pt}{\isadigit{1}}{\isacharcomma}{\kern0pt}{\isadigit{1}}{\isacharparenright}{\kern0pt}{\isacharcomma}{\kern0pt}\ of\ i\ {\isachardoublequoteopen}Suc\ i{\isachardoublequoteclose}\ i{\isacharbrackright}{\kern0pt}\ cond{\isacharunderscore}{\kern0pt}exp{\isacharunderscore}{\kern0pt}F{\isacharunderscore}{\kern0pt}meas{\isacharbrackleft}{\kern0pt}OF\ integrable\ adapted{\isacharcomma}{\kern0pt}\ of\ i{\isacharbrackright}{\kern0pt}\ assms{\isacharparenleft}{\kern0pt}{\isadigit{2}}{\isacharparenright}{\kern0pt}{\isacharbrackleft}{\kern0pt}of\ i{\isacharbrackright}{\kern0pt}\ \isacommand{by}\isamarkupfalse%
\ fastforce\isanewline
\isacommand{qed}\isamarkupfalse%
%
\endisatagproof
{\isafoldproof}%
%
\isadelimproof
\isanewline
%
\endisadelimproof
%
\isadelimtheory
\isanewline
%
\endisadelimtheory
%
\isatagtheory
\isacommand{end}\isamarkupfalse%
%
\endisatagtheory
{\isafoldtheory}%
%
\isadelimtheory
%
\endisadelimtheory
%
\end{isabellebody}%
\endinput
%:%file=Martingale.tex%:%
%:%6=2%:%
%:%7=3%:%
%:%12=4%:%
%:%13=4%:%
%:%14=5%:%
%:%15=6%:%
%:%29=8%:%
%:%41=10%:%
%:%50=12%:%
%:%60=14%:%
%:%61=14%:%
%:%62=15%:%
%:%63=16%:%
%:%64=16%:%
%:%65=17%:%
%:%66=17%:%
%:%67=18%:%
%:%68=19%:%
%:%69=19%:%
%:%71=19%:%
%:%75=19%:%
%:%83=19%:%
%:%84=20%:%
%:%85=20%:%
%:%87=20%:%
%:%91=20%:%
%:%99=20%:%
%:%100=21%:%
%:%101=22%:%
%:%102=22%:%
%:%103=23%:%
%:%104=24%:%
%:%105=24%:%
%:%107=24%:%
%:%111=24%:%
%:%119=24%:%
%:%120=25%:%
%:%121=26%:%
%:%122=26%:%
%:%123=27%:%
%:%124=27%:%
%:%125=28%:%
%:%126=29%:%
%:%127=29%:%
%:%129=29%:%
%:%133=29%:%
%:%141=29%:%
%:%142=30%:%
%:%143=30%:%
%:%145=30%:%
%:%149=30%:%
%:%157=30%:%
%:%158=31%:%
%:%159=32%:%
%:%160=33%:%
%:%161=34%:%
%:%162=34%:%
%:%163=35%:%
%:%164=36%:%
%:%165=36%:%
%:%166=37%:%
%:%167=37%:%
%:%168=38%:%
%:%169=39%:%
%:%170=39%:%
%:%172=39%:%
%:%176=39%:%
%:%184=39%:%
%:%185=40%:%
%:%186=40%:%
%:%188=40%:%
%:%192=40%:%
%:%200=40%:%
%:%201=41%:%
%:%202=42%:%
%:%203=42%:%
%:%204=43%:%
%:%205=44%:%
%:%206=44%:%
%:%207=45%:%
%:%208=45%:%
%:%209=46%:%
%:%210=47%:%
%:%211=47%:%
%:%213=47%:%
%:%217=47%:%
%:%225=47%:%
%:%226=48%:%
%:%227=48%:%
%:%229=48%:%
%:%233=48%:%
%:%241=48%:%
%:%242=49%:%
%:%243=50%:%
%:%244=51%:%
%:%245=52%:%
%:%246=52%:%
%:%247=53%:%
%:%248=54%:%
%:%249=54%:%
%:%250=55%:%
%:%251=55%:%
%:%252=56%:%
%:%253=57%:%
%:%254=57%:%
%:%256=57%:%
%:%260=57%:%
%:%268=57%:%
%:%269=58%:%
%:%270=58%:%
%:%272=58%:%
%:%276=58%:%
%:%284=58%:%
%:%285=59%:%
%:%286=60%:%
%:%287=60%:%
%:%288=61%:%
%:%289=62%:%
%:%290=62%:%
%:%291=63%:%
%:%292=63%:%
%:%293=64%:%
%:%294=65%:%
%:%295=65%:%
%:%297=65%:%
%:%301=65%:%
%:%309=65%:%
%:%310=66%:%
%:%311=66%:%
%:%313=66%:%
%:%317=66%:%
%:%332=68%:%
%:%342=70%:%
%:%343=70%:%
%:%344=71%:%
%:%345=72%:%
%:%346=73%:%
%:%347=74%:%
%:%348=74%:%
%:%349=75%:%
%:%350=75%:%
%:%351=76%:%
%:%352=76%:%
%:%354=76%:%
%:%358=76%:%
%:%366=76%:%
%:%367=77%:%
%:%368=78%:%
%:%369=78%:%
%:%370=79%:%
%:%371=80%:%
%:%374=81%:%
%:%378=81%:%
%:%379=81%:%
%:%380=82%:%
%:%381=82%:%
%:%386=82%:%
%:%389=83%:%
%:%390=84%:%
%:%391=84%:%
%:%392=85%:%
%:%393=86%:%
%:%396=87%:%
%:%400=87%:%
%:%401=87%:%
%:%402=88%:%
%:%403=88%:%
%:%408=88%:%
%:%411=89%:%
%:%412=90%:%
%:%413=90%:%
%:%415=90%:%
%:%419=90%:%
%:%420=90%:%
%:%427=90%:%
%:%428=91%:%
%:%429=92%:%
%:%430=92%:%
%:%432=92%:%
%:%436=92%:%
%:%437=92%:%
%:%451=94%:%
%:%461=96%:%
%:%462=96%:%
%:%463=97%:%
%:%464=98%:%
%:%465=99%:%
%:%466=100%:%
%:%467=100%:%
%:%468=101%:%
%:%469=102%:%
%:%470=102%:%
%:%472=102%:%
%:%476=102%:%
%:%477=102%:%
%:%478=102%:%
%:%485=102%:%
%:%486=103%:%
%:%487=103%:%
%:%489=103%:%
%:%493=103%:%
%:%508=105%:%
%:%518=107%:%
%:%519=107%:%
%:%520=108%:%
%:%521=109%:%
%:%522=110%:%
%:%523=111%:%
%:%524=111%:%
%:%525=112%:%
%:%526=113%:%
%:%527=113%:%
%:%529=113%:%
%:%533=113%:%
%:%534=113%:%
%:%535=113%:%
%:%542=113%:%
%:%543=114%:%
%:%544=114%:%
%:%546=114%:%
%:%550=114%:%
%:%560=116%:%
%:%562=118%:%
%:%563=118%:%
%:%564=119%:%
%:%571=120%:%
%:%572=120%:%
%:%573=121%:%
%:%574=121%:%
%:%575=122%:%
%:%576=122%:%
%:%577=122%:%
%:%578=123%:%
%:%579=123%:%
%:%580=123%:%
%:%581=123%:%
%:%582=124%:%
%:%583=124%:%
%:%584=125%:%
%:%585=125%:%
%:%586=126%:%
%:%587=126%:%
%:%588=126%:%
%:%589=127%:%
%:%590=127%:%
%:%591=127%:%
%:%592=128%:%
%:%593=128%:%
%:%594=128%:%
%:%595=128%:%
%:%596=129%:%
%:%611=131%:%
%:%621=133%:%
%:%622=133%:%
%:%623=134%:%
%:%624=135%:%
%:%625=136%:%
%:%626=136%:%
%:%627=137%:%
%:%628=138%:%
%:%631=139%:%
%:%635=139%:%
%:%636=139%:%
%:%637=140%:%
%:%638=141%:%
%:%639=141%:%
%:%644=141%:%
%:%647=142%:%
%:%648=143%:%
%:%649=143%:%
%:%650=144%:%
%:%651=145%:%
%:%658=146%:%
%:%659=146%:%
%:%660=147%:%
%:%661=147%:%
%:%662=147%:%
%:%663=147%:%
%:%664=148%:%
%:%665=148%:%
%:%666=148%:%
%:%667=148%:%
%:%668=149%:%
%:%669=149%:%
%:%670=149%:%
%:%671=149%:%
%:%672=149%:%
%:%673=150%:%
%:%674=150%:%
%:%675=150%:%
%:%676=150%:%
%:%677=151%:%
%:%683=151%:%
%:%686=152%:%
%:%687=153%:%
%:%688=153%:%
%:%689=154%:%
%:%696=155%:%
%:%697=155%:%
%:%698=156%:%
%:%699=156%:%
%:%700=157%:%
%:%701=157%:%
%:%702=157%:%
%:%703=158%:%
%:%704=158%:%
%:%705=158%:%
%:%706=158%:%
%:%707=159%:%
%:%708=159%:%
%:%709=159%:%
%:%710=159%:%
%:%711=160%:%
%:%712=160%:%
%:%713=161%:%
%:%714=161%:%
%:%715=161%:%
%:%716=162%:%
%:%722=162%:%
%:%725=163%:%
%:%726=164%:%
%:%727=164%:%
%:%728=165%:%
%:%731=166%:%
%:%735=166%:%
%:%736=166%:%
%:%737=166%:%
%:%742=166%:%
%:%745=167%:%
%:%746=168%:%
%:%747=168%:%
%:%748=169%:%
%:%749=170%:%
%:%756=171%:%
%:%757=171%:%
%:%758=172%:%
%:%759=172%:%
%:%760=172%:%
%:%761=173%:%
%:%762=173%:%
%:%763=174%:%
%:%764=174%:%
%:%765=174%:%
%:%766=175%:%
%:%767=175%:%
%:%768=176%:%
%:%769=176%:%
%:%770=177%:%
%:%771=177%:%
%:%772=178%:%
%:%773=178%:%
%:%774=179%:%
%:%775=179%:%
%:%776=179%:%
%:%777=180%:%
%:%778=180%:%
%:%779=181%:%
%:%785=181%:%
%:%788=182%:%
%:%789=183%:%
%:%790=183%:%
%:%791=184%:%
%:%792=185%:%
%:%799=186%:%
%:%800=186%:%
%:%801=187%:%
%:%802=187%:%
%:%803=187%:%
%:%804=188%:%
%:%805=188%:%
%:%806=189%:%
%:%807=189%:%
%:%808=189%:%
%:%809=190%:%
%:%810=190%:%
%:%811=191%:%
%:%812=191%:%
%:%813=192%:%
%:%814=192%:%
%:%815=193%:%
%:%816=193%:%
%:%817=194%:%
%:%818=194%:%
%:%819=194%:%
%:%820=194%:%
%:%821=195%:%
%:%827=195%:%
%:%830=196%:%
%:%831=197%:%
%:%832=197%:%
%:%833=198%:%
%:%834=199%:%
%:%835=199%:%
%:%836=200%:%
%:%837=201%:%
%:%838=202%:%
%:%845=203%:%
%:%846=203%:%
%:%847=204%:%
%:%848=204%:%
%:%849=205%:%
%:%850=205%:%
%:%851=205%:%
%:%852=206%:%
%:%853=206%:%
%:%854=207%:%
%:%855=207%:%
%:%856=208%:%
%:%857=208%:%
%:%858=209%:%
%:%859=209%:%
%:%860=210%:%
%:%861=210%:%
%:%866=210%:%
%:%869=211%:%
%:%870=212%:%
%:%871=212%:%
%:%872=213%:%
%:%873=214%:%
%:%874=215%:%
%:%881=216%:%
%:%882=216%:%
%:%883=217%:%
%:%884=217%:%
%:%885=217%:%
%:%886=218%:%
%:%887=218%:%
%:%888=218%:%
%:%889=218%:%
%:%890=219%:%
%:%891=219%:%
%:%892=219%:%
%:%893=220%:%
%:%894=220%:%
%:%895=221%:%
%:%896=221%:%
%:%897=221%:%
%:%898=222%:%
%:%899=222%:%
%:%900=222%:%
%:%901=222%:%
%:%902=223%:%
%:%903=223%:%
%:%904=223%:%
%:%905=223%:%
%:%906=224%:%
%:%907=224%:%
%:%908=224%:%
%:%909=224%:%
%:%910=224%:%
%:%911=225%:%
%:%912=225%:%
%:%913=225%:%
%:%914=225%:%
%:%915=225%:%
%:%916=226%:%
%:%917=226%:%
%:%918=227%:%
%:%919=227%:%
%:%920=227%:%
%:%921=227%:%
%:%922=228%:%
%:%923=228%:%
%:%924=228%:%
%:%925=228%:%
%:%926=229%:%
%:%927=229%:%
%:%941=231%:%
%:%951=233%:%
%:%952=233%:%
%:%953=234%:%
%:%954=235%:%
%:%955=236%:%
%:%956=236%:%
%:%957=237%:%
%:%958=238%:%
%:%961=239%:%
%:%965=239%:%
%:%966=239%:%
%:%967=239%:%
%:%972=239%:%
%:%975=240%:%
%:%976=241%:%
%:%977=241%:%
%:%978=242%:%
%:%979=243%:%
%:%986=244%:%
%:%987=244%:%
%:%988=245%:%
%:%989=245%:%
%:%990=245%:%
%:%991=246%:%
%:%992=246%:%
%:%993=247%:%
%:%994=247%:%
%:%995=247%:%
%:%996=248%:%
%:%997=248%:%
%:%998=249%:%
%:%999=249%:%
%:%1000=250%:%
%:%1001=250%:%
%:%1002=251%:%
%:%1003=251%:%
%:%1004=252%:%
%:%1005=252%:%
%:%1006=252%:%
%:%1007=252%:%
%:%1008=253%:%
%:%1014=253%:%
%:%1017=254%:%
%:%1018=255%:%
%:%1019=255%:%
%:%1020=256%:%
%:%1021=257%:%
%:%1028=258%:%
%:%1029=258%:%
%:%1030=259%:%
%:%1031=259%:%
%:%1032=259%:%
%:%1033=260%:%
%:%1034=260%:%
%:%1035=261%:%
%:%1036=261%:%
%:%1037=261%:%
%:%1038=262%:%
%:%1039=262%:%
%:%1040=263%:%
%:%1041=263%:%
%:%1042=264%:%
%:%1043=264%:%
%:%1044=265%:%
%:%1045=265%:%
%:%1046=266%:%
%:%1047=266%:%
%:%1048=266%:%
%:%1049=266%:%
%:%1050=267%:%
%:%1056=267%:%
%:%1059=268%:%
%:%1060=269%:%
%:%1061=269%:%
%:%1062=270%:%
%:%1063=271%:%
%:%1070=272%:%
%:%1071=272%:%
%:%1072=273%:%
%:%1073=273%:%
%:%1074=274%:%
%:%1075=274%:%
%:%1076=274%:%
%:%1077=275%:%
%:%1078=275%:%
%:%1079=276%:%
%:%1080=276%:%
%:%1081=276%:%
%:%1082=277%:%
%:%1083=277%:%
%:%1084=278%:%
%:%1085=278%:%
%:%1090=278%:%
%:%1093=279%:%
%:%1094=280%:%
%:%1095=280%:%
%:%1096=281%:%
%:%1097=282%:%
%:%1104=283%:%
%:%1105=283%:%
%:%1106=284%:%
%:%1107=284%:%
%:%1108=285%:%
%:%1109=285%:%
%:%1110=285%:%
%:%1111=286%:%
%:%1112=286%:%
%:%1113=287%:%
%:%1114=287%:%
%:%1115=288%:%
%:%1116=288%:%
%:%1117=289%:%
%:%1118=289%:%
%:%1119=290%:%
%:%1120=290%:%
%:%1125=290%:%
%:%1128=291%:%
%:%1129=292%:%
%:%1130=292%:%
%:%1131=293%:%
%:%1134=294%:%
%:%1138=294%:%
%:%1139=294%:%
%:%1140=294%:%
%:%1141=294%:%
%:%1146=294%:%
%:%1149=295%:%
%:%1150=296%:%
%:%1151=296%:%
%:%1152=297%:%
%:%1153=298%:%
%:%1154=298%:%
%:%1155=299%:%
%:%1156=300%:%
%:%1157=301%:%
%:%1158=301%:%
%:%1159=302%:%
%:%1160=303%:%
%:%1163=304%:%
%:%1167=304%:%
%:%1168=304%:%
%:%1169=305%:%
%:%1170=305%:%
%:%1171=306%:%
%:%1176=306%:%
%:%1179=307%:%
%:%1180=308%:%
%:%1181=308%:%
%:%1182=309%:%
%:%1183=310%:%
%:%1190=311%:%
%:%1191=311%:%
%:%1192=312%:%
%:%1193=312%:%
%:%1194=312%:%
%:%1195=313%:%
%:%1196=313%:%
%:%1197=314%:%
%:%1198=314%:%
%:%1199=314%:%
%:%1200=315%:%
%:%1201=315%:%
%:%1202=315%:%
%:%1203=315%:%
%:%1204=315%:%
%:%1205=316%:%
%:%1206=316%:%
%:%1207=316%:%
%:%1208=316%:%
%:%1209=317%:%
%:%1210=317%:%
%:%1211=318%:%
%:%1212=318%:%
%:%1213=318%:%
%:%1214=319%:%
%:%1220=319%:%
%:%1223=320%:%
%:%1224=321%:%
%:%1225=321%:%
%:%1226=322%:%
%:%1233=323%:%
%:%1234=323%:%
%:%1235=324%:%
%:%1236=324%:%
%:%1237=324%:%
%:%1238=325%:%
%:%1239=325%:%
%:%1240=325%:%
%:%1241=326%:%
%:%1247=326%:%
%:%1250=327%:%
%:%1251=328%:%
%:%1252=328%:%
%:%1253=329%:%
%:%1254=330%:%
%:%1255=330%:%
%:%1256=331%:%
%:%1257=332%:%
%:%1258=333%:%
%:%1265=334%:%
%:%1266=334%:%
%:%1267=335%:%
%:%1268=335%:%
%:%1269=336%:%
%:%1270=336%:%
%:%1271=336%:%
%:%1272=337%:%
%:%1273=337%:%
%:%1274=338%:%
%:%1275=338%:%
%:%1276=339%:%
%:%1277=339%:%
%:%1278=340%:%
%:%1279=340%:%
%:%1280=341%:%
%:%1281=341%:%
%:%1286=341%:%
%:%1289=342%:%
%:%1290=343%:%
%:%1291=343%:%
%:%1292=344%:%
%:%1293=345%:%
%:%1294=346%:%
%:%1301=347%:%
%:%1302=347%:%
%:%1303=348%:%
%:%1304=348%:%
%:%1305=349%:%
%:%1306=349%:%
%:%1307=349%:%
%:%1308=350%:%
%:%1309=350%:%
%:%1310=350%:%
%:%1311=350%:%
%:%1312=351%:%
%:%1313=351%:%
%:%1314=352%:%
%:%1315=352%:%
%:%1316=352%:%
%:%1317=353%:%
%:%1318=353%:%
%:%1319=353%:%
%:%1320=353%:%
%:%1321=354%:%
%:%1322=354%:%
%:%1323=354%:%
%:%1324=354%:%
%:%1325=355%:%
%:%1326=355%:%
%:%1327=355%:%
%:%1328=355%:%
%:%1329=355%:%
%:%1330=356%:%
%:%1331=356%:%
%:%1332=356%:%
%:%1333=356%:%
%:%1334=356%:%
%:%1335=357%:%
%:%1336=357%:%
%:%1337=357%:%
%:%1338=357%:%
%:%1339=357%:%
%:%1340=358%:%
%:%1341=358%:%
%:%1342=359%:%
%:%1343=359%:%
%:%1344=360%:%
%:%1345=360%:%
%:%1346=361%:%
%:%1347=361%:%
%:%1348=361%:%
%:%1349=361%:%
%:%1350=362%:%
%:%1351=362%:%
%:%1352=363%:%
%:%1353=363%:%
%:%1367=365%:%
%:%1379=367%:%
%:%1381=369%:%
%:%1382=369%:%
%:%1383=370%:%
%:%1384=371%:%
%:%1385=372%:%
%:%1386=372%:%
%:%1387=373%:%
%:%1388=374%:%
%:%1391=375%:%
%:%1395=375%:%
%:%1396=375%:%
%:%1397=376%:%
%:%1398=376%:%
%:%1403=376%:%
%:%1406=377%:%
%:%1407=378%:%
%:%1408=378%:%
%:%1409=379%:%
%:%1410=380%:%
%:%1417=381%:%
%:%1418=381%:%
%:%1419=382%:%
%:%1420=382%:%
%:%1421=382%:%
%:%1422=383%:%
%:%1423=383%:%
%:%1424=384%:%
%:%1425=384%:%
%:%1426=384%:%
%:%1427=385%:%
%:%1428=385%:%
%:%1429=386%:%
%:%1430=386%:%
%:%1431=387%:%
%:%1432=387%:%
%:%1433=388%:%
%:%1434=388%:%
%:%1435=389%:%
%:%1436=389%:%
%:%1437=389%:%
%:%1438=389%:%
%:%1439=390%:%
%:%1445=390%:%
%:%1448=391%:%
%:%1449=392%:%
%:%1450=392%:%
%:%1451=393%:%
%:%1452=394%:%
%:%1459=395%:%
%:%1460=395%:%
%:%1461=396%:%
%:%1462=396%:%
%:%1463=396%:%
%:%1464=397%:%
%:%1465=397%:%
%:%1466=398%:%
%:%1467=398%:%
%:%1468=398%:%
%:%1469=399%:%
%:%1470=399%:%
%:%1471=400%:%
%:%1472=400%:%
%:%1473=401%:%
%:%1474=401%:%
%:%1475=402%:%
%:%1476=402%:%
%:%1477=403%:%
%:%1478=403%:%
%:%1479=403%:%
%:%1480=403%:%
%:%1481=404%:%
%:%1487=404%:%
%:%1490=405%:%
%:%1491=406%:%
%:%1492=406%:%
%:%1493=407%:%
%:%1494=408%:%
%:%1501=409%:%
%:%1502=409%:%
%:%1503=410%:%
%:%1504=410%:%
%:%1505=411%:%
%:%1506=411%:%
%:%1507=411%:%
%:%1508=412%:%
%:%1509=412%:%
%:%1510=413%:%
%:%1511=413%:%
%:%1512=413%:%
%:%1513=414%:%
%:%1514=414%:%
%:%1515=415%:%
%:%1516=415%:%
%:%1521=415%:%
%:%1524=416%:%
%:%1525=417%:%
%:%1526=417%:%
%:%1527=418%:%
%:%1528=419%:%
%:%1535=420%:%
%:%1536=420%:%
%:%1537=421%:%
%:%1538=421%:%
%:%1539=422%:%
%:%1540=422%:%
%:%1541=422%:%
%:%1542=423%:%
%:%1543=423%:%
%:%1544=424%:%
%:%1545=424%:%
%:%1546=424%:%
%:%1547=425%:%
%:%1548=425%:%
%:%1549=426%:%
%:%1550=426%:%
%:%1555=426%:%
%:%1558=427%:%
%:%1559=428%:%
%:%1560=428%:%
%:%1561=429%:%
%:%1564=430%:%
%:%1568=430%:%
%:%1569=430%:%
%:%1570=430%:%
%:%1571=430%:%
%:%1576=430%:%
%:%1579=431%:%
%:%1580=432%:%
%:%1581=432%:%
%:%1582=433%:%
%:%1583=434%:%
%:%1584=434%:%
%:%1585=435%:%
%:%1586=436%:%
%:%1587=437%:%
%:%1588=437%:%
%:%1589=438%:%
%:%1590=439%:%
%:%1593=440%:%
%:%1597=440%:%
%:%1598=440%:%
%:%1599=441%:%
%:%1600=441%:%
%:%1601=442%:%
%:%1606=442%:%
%:%1609=443%:%
%:%1610=444%:%
%:%1611=444%:%
%:%1612=445%:%
%:%1613=446%:%
%:%1620=447%:%
%:%1621=447%:%
%:%1622=448%:%
%:%1623=448%:%
%:%1624=448%:%
%:%1625=449%:%
%:%1626=449%:%
%:%1627=450%:%
%:%1628=450%:%
%:%1629=450%:%
%:%1630=451%:%
%:%1631=451%:%
%:%1632=451%:%
%:%1633=451%:%
%:%1634=451%:%
%:%1635=452%:%
%:%1636=452%:%
%:%1637=452%:%
%:%1638=452%:%
%:%1639=453%:%
%:%1640=453%:%
%:%1641=454%:%
%:%1642=454%:%
%:%1643=454%:%
%:%1644=455%:%
%:%1650=455%:%
%:%1653=456%:%
%:%1654=457%:%
%:%1655=457%:%
%:%1656=458%:%
%:%1663=459%:%
%:%1664=459%:%
%:%1665=460%:%
%:%1666=460%:%
%:%1667=460%:%
%:%1668=461%:%
%:%1669=461%:%
%:%1670=461%:%
%:%1671=462%:%
%:%1677=462%:%
%:%1680=463%:%
%:%1681=464%:%
%:%1682=464%:%
%:%1683=465%:%
%:%1684=466%:%
%:%1685=466%:%
%:%1686=467%:%
%:%1687=468%:%
%:%1688=469%:%
%:%1695=470%:%
%:%1696=470%:%
%:%1697=471%:%
%:%1698=471%:%
%:%1699=472%:%
%:%1700=472%:%
%:%1701=472%:%
%:%1702=473%:%
%:%1703=473%:%
%:%1704=474%:%
%:%1705=474%:%
%:%1706=475%:%
%:%1707=475%:%
%:%1708=476%:%
%:%1709=476%:%
%:%1710=477%:%
%:%1711=477%:%
%:%1716=477%:%
%:%1719=478%:%
%:%1720=479%:%
%:%1721=479%:%
%:%1722=480%:%
%:%1723=481%:%
%:%1724=482%:%
%:%1731=483%:%
%:%1732=483%:%
%:%1733=484%:%
%:%1734=484%:%
%:%1735=484%:%
%:%1736=485%:%
%:%1737=485%:%
%:%1738=485%:%
%:%1739=486%:%
%:%1740=486%:%
%:%1741=487%:%
%:%1742=487%:%
%:%1743=488%:%
%:%1744=488%:%
%:%1745=489%:%
%:%1746=489%:%
%:%1747=490%:%
%:%1748=491%:%
%:%1749=491%:%
%:%1750=491%:%
%:%1751=491%:%
%:%1752=492%:%
%:%1767=494%:%
%:%1777=496%:%
%:%1778=496%:%
%:%1779=497%:%
%:%1780=497%:%
%:%1781=498%:%
%:%1782=498%:%
%:%1783=499%:%
%:%1784=500%:%
%:%1785=500%:%
%:%1786=501%:%
%:%1787=501%:%
%:%1788=502%:%
%:%1789=503%:%
%:%1790=503%:%
%:%1792=503%:%
%:%1796=503%:%
%:%1804=503%:%
%:%1805=504%:%
%:%1806=504%:%
%:%1808=504%:%
%:%1812=504%:%
%:%1820=504%:%
%:%1821=505%:%
%:%1822=506%:%
%:%1823=506%:%
%:%1824=507%:%
%:%1825=508%:%
%:%1832=509%:%
%:%1833=509%:%
%:%1834=510%:%
%:%1835=510%:%
%:%1836=511%:%
%:%1837=511%:%
%:%1838=512%:%
%:%1839=512%:%
%:%1840=513%:%
%:%1841=513%:%
%:%1842=513%:%
%:%1843=513%:%
%:%1844=514%:%
%:%1845=514%:%
%:%1846=515%:%
%:%1847=515%:%
%:%1848=516%:%
%:%1849=516%:%
%:%1850=516%:%
%:%1851=517%:%
%:%1852=517%:%
%:%1853=517%:%
%:%1854=517%:%
%:%1855=518%:%
%:%1856=518%:%
%:%1857=519%:%
%:%1858=519%:%
%:%1859=519%:%
%:%1860=519%:%
%:%1861=520%:%
%:%1867=520%:%
%:%1870=521%:%
%:%1871=522%:%
%:%1872=522%:%
%:%1873=523%:%
%:%1874=524%:%
%:%1875=525%:%
%:%1882=526%:%
%:%1883=526%:%
%:%1884=527%:%
%:%1885=527%:%
%:%1886=527%:%
%:%1887=528%:%
%:%1888=528%:%
%:%1889=528%:%
%:%1890=529%:%
%:%1891=529%:%
%:%1892=530%:%
%:%1893=530%:%
%:%1894=531%:%
%:%1895=531%:%
%:%1896=531%:%
%:%1897=531%:%
%:%1898=531%:%
%:%1899=532%:%
%:%1900=532%:%
%:%1901=533%:%
%:%1902=533%:%
%:%1903=534%:%
%:%1904=534%:%
%:%1905=534%:%
%:%1906=535%:%
%:%1907=535%:%
%:%1908=535%:%
%:%1909=535%:%
%:%1910=536%:%
%:%1911=536%:%
%:%1912=536%:%
%:%1913=536%:%
%:%1914=537%:%
%:%1915=537%:%
%:%1916=538%:%
%:%1917=538%:%
%:%1922=538%:%
%:%1925=539%:%
%:%1926=540%:%
%:%1927=540%:%
%:%1928=541%:%
%:%1929=542%:%
%:%1930=543%:%
%:%1937=544%:%
%:%1938=544%:%
%:%1939=545%:%
%:%1940=545%:%
%:%1941=545%:%
%:%1942=546%:%
%:%1943=546%:%
%:%1944=546%:%
%:%1945=547%:%
%:%1946=547%:%
%:%1947=548%:%
%:%1948=548%:%
%:%1949=549%:%
%:%1950=549%:%
%:%1951=549%:%
%:%1952=550%:%
%:%1953=550%:%
%:%1954=550%:%
%:%1955=550%:%
%:%1956=551%:%
%:%1957=551%:%
%:%1958=552%:%
%:%1959=552%:%
%:%1960=553%:%
%:%1961=553%:%
%:%1962=553%:%
%:%1963=553%:%
%:%1964=554%:%
%:%1965=554%:%
%:%1966=554%:%
%:%1967=554%:%
%:%1968=555%:%
%:%1969=555%:%
%:%1970=555%:%
%:%1971=555%:%
%:%1972=555%:%
%:%1973=556%:%
%:%1974=556%:%
%:%1975=556%:%
%:%1976=557%:%
%:%1977=557%:%
%:%1978=557%:%
%:%1979=557%:%
%:%1980=558%:%
%:%1981=558%:%
%:%1982=558%:%
%:%1983=558%:%
%:%1984=559%:%
%:%1985=559%:%
%:%1986=560%:%
%:%1987=560%:%
%:%1992=560%:%
%:%1995=561%:%
%:%1996=562%:%
%:%1997=562%:%
%:%1998=563%:%
%:%1999=564%:%
%:%2000=565%:%
%:%2007=566%:%
%:%2008=566%:%
%:%2009=567%:%
%:%2010=567%:%
%:%2011=568%:%
%:%2012=568%:%
%:%2013=568%:%
%:%2014=568%:%
%:%2015=569%:%
%:%2030=571%:%
%:%2040=573%:%
%:%2041=573%:%
%:%2042=574%:%
%:%2043=575%:%
%:%2046=576%:%
%:%2050=576%:%
%:%2051=576%:%
%:%2052=577%:%
%:%2053=577%:%
%:%2054=578%:%
%:%2055=578%:%
%:%2056=579%:%
%:%2057=579%:%
%:%2058=579%:%
%:%2059=579%:%
%:%2060=580%:%
%:%2061=580%:%
%:%2062=581%:%
%:%2063=581%:%
%:%2064=582%:%
%:%2065=582%:%
%:%2066=582%:%
%:%2067=583%:%
%:%2068=583%:%
%:%2069=583%:%
%:%2070=584%:%
%:%2071=584%:%
%:%2072=584%:%
%:%2073=584%:%
%:%2074=585%:%
%:%2075=585%:%
%:%2076=585%:%
%:%2077=585%:%
%:%2078=586%:%
%:%2084=586%:%
%:%2087=587%:%
%:%2088=588%:%
%:%2089=588%:%
%:%2090=589%:%
%:%2091=590%:%
%:%2092=591%:%
%:%2099=592%:%
%:%2100=592%:%
%:%2101=593%:%
%:%2102=593%:%
%:%2103=593%:%
%:%2104=594%:%
%:%2105=594%:%
%:%2106=594%:%
%:%2107=595%:%
%:%2108=595%:%
%:%2109=596%:%
%:%2110=596%:%
%:%2111=597%:%
%:%2112=597%:%
%:%2113=597%:%
%:%2114=597%:%
%:%2115=597%:%
%:%2116=598%:%
%:%2117=598%:%
%:%2118=599%:%
%:%2119=599%:%
%:%2120=600%:%
%:%2121=600%:%
%:%2122=600%:%
%:%2123=601%:%
%:%2124=601%:%
%:%2125=601%:%
%:%2126=601%:%
%:%2127=602%:%
%:%2128=602%:%
%:%2129=602%:%
%:%2130=602%:%
%:%2131=603%:%
%:%2132=603%:%
%:%2133=604%:%
%:%2134=604%:%
%:%2139=604%:%
%:%2142=605%:%
%:%2143=606%:%
%:%2144=606%:%
%:%2145=607%:%
%:%2146=608%:%
%:%2147=609%:%
%:%2150=610%:%
%:%2154=610%:%
%:%2155=610%:%
%:%2156=611%:%
%:%2157=611%:%
%:%2158=612%:%
%:%2163=612%:%
%:%2166=613%:%
%:%2167=614%:%
%:%2168=614%:%
%:%2169=615%:%
%:%2170=616%:%
%:%2171=617%:%
%:%2178=618%:%
%:%2179=618%:%
%:%2180=619%:%
%:%2181=619%:%
%:%2182=620%:%
%:%2183=620%:%
%:%2184=620%:%
%:%2185=620%:%
%:%2186=621%:%
%:%2192=621%:%
%:%2195=622%:%
%:%2196=623%:%
%:%2197=623%:%
%:%2198=624%:%
%:%2199=625%:%
%:%2206=626%:%
%:%2207=626%:%
%:%2208=627%:%
%:%2209=627%:%
%:%2210=627%:%
%:%2211=628%:%
%:%2212=628%:%
%:%2213=628%:%
%:%2214=629%:%
%:%2215=629%:%
%:%2216=629%:%
%:%2217=630%:%
%:%2218=630%:%
%:%2219=630%:%
%:%2220=631%:%
%:%2221=631%:%
%:%2222=631%:%
%:%2223=631%:%
%:%2224=632%:%
%:%2225=632%:%
%:%2226=632%:%
%:%2227=633%:%
%:%2228=633%:%
%:%2229=634%:%
%:%2230=634%:%
%:%2231=635%:%
%:%2232=635%:%
%:%2233=635%:%
%:%2234=635%:%
%:%2235=636%:%
%:%2241=636%:%
%:%2244=637%:%
%:%2245=638%:%
%:%2246=638%:%
%:%2247=639%:%
%:%2248=640%:%
%:%2255=641%:%
%:%2256=641%:%
%:%2257=642%:%
%:%2258=642%:%
%:%2259=642%:%
%:%2260=643%:%
%:%2261=643%:%
%:%2262=643%:%
%:%2263=643%:%
%:%2264=644%:%
%:%2265=644%:%
%:%2266=644%:%
%:%2267=645%:%
%:%2282=647%:%
%:%2292=649%:%
%:%2293=649%:%
%:%2294=650%:%
%:%2295=651%:%
%:%2298=652%:%
%:%2302=652%:%
%:%2303=652%:%
%:%2304=653%:%
%:%2305=653%:%
%:%2306=654%:%
%:%2307=654%:%
%:%2308=655%:%
%:%2309=655%:%
%:%2310=655%:%
%:%2311=655%:%
%:%2312=656%:%
%:%2313=656%:%
%:%2314=657%:%
%:%2315=657%:%
%:%2316=658%:%
%:%2317=658%:%
%:%2318=658%:%
%:%2319=659%:%
%:%2320=659%:%
%:%2321=659%:%
%:%2322=660%:%
%:%2323=660%:%
%:%2324=660%:%
%:%2325=660%:%
%:%2326=661%:%
%:%2327=661%:%
%:%2328=661%:%
%:%2329=661%:%
%:%2330=662%:%
%:%2336=662%:%
%:%2339=663%:%
%:%2340=664%:%
%:%2341=664%:%
%:%2342=665%:%
%:%2343=666%:%
%:%2344=667%:%
%:%2351=668%:%
%:%2352=668%:%
%:%2353=669%:%
%:%2354=669%:%
%:%2355=669%:%
%:%2356=670%:%
%:%2357=670%:%
%:%2358=670%:%
%:%2359=671%:%
%:%2360=671%:%
%:%2361=672%:%
%:%2362=672%:%
%:%2363=673%:%
%:%2364=673%:%
%:%2365=674%:%
%:%2366=674%:%
%:%2367=675%:%
%:%2368=676%:%
%:%2369=676%:%
%:%2370=676%:%
%:%2371=676%:%
%:%2372=677%:%
%:%2378=677%:%
%:%2381=678%:%
%:%2382=679%:%
%:%2383=679%:%
%:%2384=680%:%
%:%2385=681%:%
%:%2386=682%:%
%:%2393=683%:%
%:%2394=683%:%
%:%2395=684%:%
%:%2396=684%:%
%:%2397=684%:%
%:%2398=685%:%
%:%2399=685%:%
%:%2400=685%:%
%:%2401=686%:%
%:%2402=686%:%
%:%2403=686%:%
%:%2404=686%:%
%:%2405=687%:%
%:%2406=687%:%
%:%2407=687%:%
%:%2408=688%:%
%:%2409=688%:%
%:%2410=688%:%
%:%2411=688%:%
%:%2412=689%:%
%:%2418=689%:%
%:%2421=690%:%
%:%2422=691%:%
%:%2423=691%:%
%:%2424=692%:%
%:%2425=693%:%
%:%2426=694%:%
%:%2433=695%:%
%:%2434=695%:%
%:%2435=696%:%
%:%2436=696%:%
%:%2437=697%:%
%:%2438=697%:%
%:%2439=697%:%
%:%2440=697%:%
%:%2441=698%:%
%:%2447=698%:%
%:%2452=699%:%
%:%2457=700%:%



% optional bibliography
\bibliographystyle{abbrv}
\bibliography{root}

\end{document}

%%% Local Variables:
%%% mode: latex
%%% TeX-master: t
%%% End:
